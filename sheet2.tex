
\documentclass[a4paper]{article}
\usepackage{geometry}
\geometry{margin=1in}
\usepackage[english,bulgarian]{babel}
\usepackage{amssymb}
\usepackage{amsmath}
\usepackage{mathrsfs}
\usepackage{latexsym}
\usepackage{amsthm}
\usepackage{enumerate}
\usepackage[colorlinks=true, linkcolor=blue,pdfstartview=FitV,
citecolor=green, urlcolor=blue]{hyperref}


\setlength{\parskip}{2.3ex}            % vertical space between paragraphs
\setlength{\parindent}{0in}            % amount of indentation of paragraph
%this package allows for hyperlinks within the pdf document


\newtheorem{thm}{Теорема}
\newtheorem{dfn}{Дефиниция}
\newtheorem{problem}{Задача}
\newtheorem{example}{Пример}
\newcommand{\A}{\mathfrak{A}}
\newcommand{\B}{\mathfrak{B}}
\renewcommand{\C}{\mathfrak{C}}
\newcommand{\D}{\mathfrak{D}}
\newcommand{\N}{\mathbb{N}}
\newcommand{\R}{\mathbb{R}}
\newcommand{\Ls}{\mathscr{L}}
\newcommand{\Fs}{\mathscr{F}}
\newcommand{\Rs}{\mathscr{R}}
\newcommand{\Ps}{\mathscr{P}}
\newcommand{\As}{\mathscr{A}}
\newcommand{\Bs}{\mathscr{B}}
\newcommand{\Is}{\mathscr{I}}
\newcommand{\Ss}{\mathscr{S}}
\newcommand{\xn}{x_{1},\dots,x_{n}}

\newcommand{\xs}{overline{x}}

\newcommand{\ys}{overline{y}}

\newcommand{\zs}{overline{z}}

\begin{document}

\author{Stefan Vatev}

\section{Операции върху множества}

Дефинираме следните операции върху множества:
\begin{enumerate}[(i)]
  \item
    Сечение, $A\cap B = \{x\ \mid\ x\in A\ \&\ x\in B\}$;
  \item
    Обединение, $A\cup B = \{x\ \mid x\in A\ \vee\ x\in B\}$
  \item
    $\bigcup^{n}_{i=1} A_i = \{x \mid \exists i (1\leq i\leq n\ \&\ x\in A_i \}$;
  \item
    $\bigcap^{n}_{i=1} A_i = \{x \mid \forall i (1\leq i\leq n \rightarrow x\in A_i)\}$;
  \item
    Разлика, $A\setminus B = \{x\ \mid\ x\in A\ \&\ x\not\in B\}$;
  \item
    Симетрична разлика, $A\triangle B = (A\backslash B)\cup (B\backslash A)$;
  \item
    $\bigcup A = \{x\mid (\exists y\in A)[x\in y]\}$;
  \item
    $\bigcap A = \{x\mid (\forall y\in A)[x\in y]\}$;
  \item
    Степенно множество, $\Ps A = \{x\mid x\subseteq A\}$.
\end{enumerate}

Тук имаме проблем с значението на $\bigcap\emptyset$.
На пръв поглед изглежда, че $\bigcap\emptyset$ е множеството от всички множества $V$, 
но ние знаем, че такова множество не съществува.
Това в известен смисъл е аналог на делението на нула.
Ние ще приемем, че $\bigcap\emptyset = \emptyset$.


\begin{example}
  Нека $A = \{x\in\N\mid x > 1\}$ и $B = \{x\in\N\mid x>3\}$. Тогава :
    \begin{enumerate}[]
    \item
      $A\cap B = \{x\in\N\mid x > 3\}$,
    \item
      $A\cup B = \{x\in\N\mid x > 1\}$,
    \item
      $A\setminus B = \{x\in\N\mid 1<x\leq 3\}$,
    \item
      $B\setminus A = \emptyset$,
    \item
      $A\triangle B = \{x\in\N\mid 1<x\leq 3\}$
    \end{enumerate}
\end{example}


\begin{problem}
  Нека $A = \{x\in\R\mid |x|\leq 1\}$ и $B = \{x\in\R\mid |x-1|\leq \frac{1}{2}\}$.
  Намерете $A\cup B$, $A\cap B$, $A\setminus B$, $B\setminus A$, $A\triangle B$.
\end{problem}



\begin{example}
  \[\bigcap\{\{1,2,3,4\},\{2,4\},\{1,3,4\}\} = \{4\}\]
  \[\bigcup\{\{3\},\{2,4\},\{1,4\}\} = \{1,2,3,4\}\]
  \[\bigcap\{\{a\},\{a,b\}\} = \{a\}\cap\{a,b\} = \{a\}\]
  \[\bigcup\bigcap\{\{a\},\{a,b\}\}  = \bigcup\{a\} = a\]
\end{example}


\begin{problem}
  Нека $B = \{\{1,2\},\{2,3\}, \{1,3\}, \{\emptyset\}\}$.
  Намерете $\bigcup{B}$, $\bigcap{B}$, $\bigcap\bigcup{B}$ и $\bigcup\bigcap{B}$.
\end{problem}


\begin{example}
  Ето няколко примера, които показват действието на някои от операциите
  \begin{enumerate}[1)]
  \item
    \begin{enumerate}[]
    \item
      $\Ps\emptyset = \{\emptyset\}$
    \item
      $\Ps\{\emptyset\} = \{\emptyset,\{\emptyset\}\}$
    \item
      $\Ps\{\emptyset,\{\emptyset\}\} = \{\emptyset,\{\emptyset\},\{\{\emptyset\}\}, \{\emptyset,\{\emptyset\}\}\}$
    \end{enumerate}
  \item
    \begin{enumerate}[]
    \item
      $\bigcup\{\emptyset\} = \emptyset$
    \item
      $\bigcup\{\emptyset,\{\emptyset\}\} = \{\emptyset\}$
    \item      
      $\bigcup\{\emptyset,\{\emptyset\},\{\{\emptyset\}\}, \{\emptyset,\{\emptyset\}\}\} = \{\emptyset,\{\emptyset\}\}$
    \end{enumerate}
  \item
    $\bigcap\{\emptyset,\{\emptyset\}\} = \emptyset$
\end{enumerate}
\end{example}



\begin{problem}
  \begin{enumerate}
  \item
    Намерете двуелементно множество такова, че всеки елемент на множеството да е също и негово подмножество.
  \item
    Намерете триелементно множество такова, че всеки елемент на множеството да е също и негово подмножество.
  \item
    Намерете четириелементно множество такова, че всеки елемент на множеството да е също и негово подмножество.
\end{enumerate}
\end{problem}


\begin{problem}
  Докажете:
  \begin{enumerate}
  \item
    $\bigcup\Ps A = A$;
  \item
    $A\subseteq\Ps\bigcup A$; кога имаме равенство?
  \item
    $\Ps A \cap \Ps B = \Ps(A\cap B)$;
  \item
    $\Ps A \cup \Ps B \subseteq\Ps(A\cup B)$; кога имаме равенство?
  \item
    съществуват множества $a$ и $B$, за които $a\in B$, но $\Ps{a}\not\subseteq\Ps{B}$;
  \item
    ако $a\in B$, то $\Ps{a}\in\Ps\Ps{B}$;
  \item
    $\{\emptyset,\{\emptyset\}\} \in \Ps\Ps{A}$, за всяко множество $A$.
  \end{enumerate}
\end{problem}



\begin{problem}
  Проверете:
\begin{enumerate}[a)]
  \item
    $A\cup(B\cap C) = (A\cup B)\cap(A\cup C)$
  \item
    $X\subseteq A\ \&\ X\subseteq B \rightarrow X\subseteq A\cap B$
  \item
    $A\subseteq X\ \&\ B\subseteq X \rightarrow A\cup B\subseteq X$
  \item
    $(\bigcup^{n}_{i=1} A_i) \cap B = \bigcup^{n}_{i=1} (A_i \cap B)$
  \item
    $(\bigcap^{n}_{i=1} A_i) \cup B = \bigcap^{n}_{i=1} (A_i \cup B)$
  \item
    $A\subseteq B \iff A\setminus B = \emptyset \iff A\cup B = B \iff A\cap B = A$
  \item
    $A\backslash B = A \iff A\cap B = \emptyset$
  \item
    $A\backslash B = A\backslash (A\cap B)$
  \item
    $(A\cup B)\setminus C = (A\setminus C) \cup (B\setminus C)$
  \item
    $X\backslash (A\cup B) = (X\backslash A)\cap(X\backslash B)$
  \item
    $X\backslash(\bigcup^{n}_{i=1} A_i) = \bigcap^{n}_{i=1} (X\backslash A_i)$
  \item
    $X\backslash (A\cap B) = (X\backslash A)\cup(X\backslash B)$
  \item
    $X\backslash(\bigcap^{n}_{i=1} A_i) = \bigcup^{n}_{i=1} (X\backslash A_i)$
  \item
    $A\cup\bigcap B = \{A\cup X\mid X\in B\}$, за $B\neq\emptyset$
  \item
    $A\cap\bigcup B = \{A\cap X\mid X\in B\}$
  \item
    $(A\backslash B)\backslash C = (A\backslash C)\backslash(B \backslash C)$
  \item
    $A\backslash (B\backslash C) = (A\backslash B) \cup (A\cap C)$
  \item
    $A\triangle B = B\triangle A$
  \item
    $A\triangle(B\triangle C) = (A\triangle B)\triangle C$
  \item
    $A\backslash B = A\triangle(A\cap B)$
  \item
    $A\cap(B\triangle C) = (A\cup B)\triangle(A\cup C)$
  \item
    $A\cup B = (A\triangle B)\cup(A\cap B)$
  \item
    $A\triangle B = \emptyset \iff A = B$
  \item
    $A\triangle B = C \iff B\triangle C = A \iff C\triangle A = B$
  \end{enumerate}
\end{problem}

\begin{problem}
  Да се решат системите с променлива $X$:
  \begin{enumerate}[(a)]
  \item
    \begin{tabular}{l c l}
      $\big|A\setminus X$ & $= $ & $ B$\\
      $\big|X\setminus A $ & $=$ & $ C$
    \end{tabular}, където са дадени множествата $A,B,C$ и $B\subseteq A$, $A\cap C = \emptyset$;
  \item
    \begin{tabular}{l c l}
      $\big|A\cap X$ & $= $ & $ B$\\
      $\big|A\cup X $ & $=$ & $ C$
    \end{tabular}, където са дадени множествата $A,B,C$ и $B\subseteq A\subseteq C$;
  \item
    \begin{tabular}{l c l}
      $\big|A\setminus X$ & $= $ & $ B$\\
      $\big|A\cup X $ & $=$ & $ C$
    \end{tabular}, където са дадени множествата $A,B,C$ и $B\subseteq A\subseteq C$.
  \end{enumerate}
\end{problem}



\begin{problem}
  Нека множеството $A$ е дефинирано по следния начин:
  \begin{enumerate}
  \item
    $0\in A$
  \item
    Ако $x\in A$, то $2x+1 \in A$.
\end{enumerate}
Намерете $A$.
\end{problem}
\begin{proof}
  $A = \{2^n - 1\ \mid n\in\N\}$.
\end{proof}

\begin{thm}
  Нека множеството $A$ е дефинирано по следния начин:
  \begin{enumerate}[(1)]
  \item
    $1\in A$
  \item
    Ако $m,n\in A$, то $2m+3n \in A$.
  \item
    Всички елементи на $A$ са добавени или по правило (1) или правило (2).
\end{enumerate}
Намерете $A$.
\end{thm}
\begin{proof}
  Нека $B = \{n \mid n\equiv 1 (mod 12)\ \vee n\equiv 5 (mod 12) \}$.
  Искаме да докажем, че $A = B$.
  Първо ще докажем, че $A\subseteq B$.
  За целта проверяваме, че $1\in B$ и ако $m,n \in B$, то $2m+3n \in B$.
  
  За другата посока, т.е. $B\subseteq A$, трябва да докажем, че ако
  за всяко $k\leq n$ е вярно, че $12k+1 \in B$ и $12k + 5 \in B$,
  то е вярно, че $12(n+1)+1 \in B $ и $12(n+1) + 5 \in B$.
\end{proof}



\end{document}


%%% Local Variables: 
%%% mode: latex
%%% TeX-master: t
%%% End: 

\section{Предикатно смятане}

\subsection*{Квантори}

Свойствата или отношенията на елементите в произволно множество се наричат {\bf предикати}.

Нека имаме един едноместен предикат $P(\cdot)$ дефиниран в множеството $A$.
\begin{enumerate}[(I)]
\item 
  {\bf Квантор за общност} $\forall x$.
  Записът $(\forall x \in A) P(x)$ означава, че за всеки елемент $a$ в $A$, 
  твърдението $P(a)$ има стойност истина.
  Например, $(\forall x\in\Nat)[(x+1)^2 = x^2+2x+1]$.
\item
  {\bf Квантор за съществуване} $\exists x$.
  Записът $(\exists x \in A) P(x)$ означава, че съществува елемент $a$ в $A$, 
  за който твърдението $P(a)$ има стойност истина.
  Например, $(\exists x \in\mathbf{C})[x^2 = -1]$.
\end{enumerate}

\begin{example}
  \begin{itemize}
  \item
    За всяко естествено число, съществува по-голямо от него:
    \[(\forall x\in\Nat)(\exists z\in\Nat)[x < z].\]
  \item
    Съществува естествено число, от което няма по-малко:
    \[(\exists x\in\Nat)(\forall y\in\Nat)[x < y \vee x = y].\]
    Нека да означим с $Zero(x)$ предиката, който казва, че $x$ е най-малкото число, т.е.
    \[(\forall y)[x < y \vee x =y].\]
  \item
    Нека $S(x,y)$ да бъде предиката, който казва, че $y = x+1$ в естествените числа:
    \[S(x,y) \equiv (x < y\ \wedge\ (\forall z\in\Nat)[x < z\ \rightarrow (z = y\ \vee\ y < z)].\]
  \item
    $O(x)$ - $x$ е числото $1$:
    \[(\exists y)[Z(y)\ \wedge\ S(y,x)].\]
  \item
    $Div(x,y)$ - $x$ се дели на $y$:
    \[(\exists z)[x = y.z].\]
  \item
    $Prime(x)$ - $x$ е просто число:
    \[x \geq 2\ \wedge\ (\forall y\in\Nat)[\neg (O(y)\ \wedge Z(y))\ \rightarrow\ \neg Div(x,y)].\]
  \end{itemize}
\end{example}


\subsection*{Закони на предикатното смятане}

\begin{enumerate}[(I)]
  \item
    $\neg\forall x P(x) \iff \exists x \neg P(x)$
  \item
    $\neg\exists x P(x) \iff \forall x \neg P(x)$
  \item
    $\forall x P(x) \iff \neg\exists x \neg P(x)$
  \item
    $\exists x P(x) \iff \neg\forall x \neg P(x)$
  \item
    $\forall x \forall y P(x) \iff \forall y\forall x P(x)$
  \item
    $\exists x\exists y P(x,y) \iff \exists y \exists x P(x)$  
  \item
    $\exists x\forall y P(x,y) \rightarrow \forall y \exists x P(x,y)$
\end{enumerate}


\begin{problem}
  Да означим с $K(x,y)$ твърдението ``$x$ познава $y$''.
  Изразете като формула следните твърдения.
  \begin{enumerate}[1)]
  \item
    \marginpar{$\forall x \exists y K(x,y)$}
    Всеки познава някого.
  \item
    \marginpar{$\exists x \forall y K(x,y)$}
    Някой познава всеки.
  \item
    \marginpar{$\exists x\forall y K(y,x)$}
    Някой е познаван от всички.
  \item
    \marginpar{$\forall x \exists y(K(x,y)\wedge \neg K(y,x)) $}
    Всеки знае някой, който не го познава.
  \item
    \marginpar{$\exists x \forall y(K(y,x)\ \rightarrow K(x,y))$}
    Има такъв, който знае всеки, който го познава.
  \end{enumerate}
\end{problem}

\begin{problem}
  Нека $G(x)$ означава, че човекът $x$ е добър.
  \begin{enumerate}[1)]
  \item
    Изразете с формула твърдението, че всички хора са добри {\em без}
    да използвате квантора $\forall$, а само квантора $\exists$ и логическите връзки.
  \item
    Изразете с формула твърдението, че {\em поне един} човек е добър {\em без}
    да използвате квантора $\exists$, а само квантора $\forall$ и логическите връзки.
  \end{enumerate}
\end{problem}

\begin{problem}
  \marginpar{(От \cite{smullyan})}
  На един остров живеели два вида обитатели - благородници и негодници.
  Благородниците винаги казвали истината, а негодниците винаги лъжели.
  Един пътешественик попаднал на този остров и искал да разбере повече за
  неговите обитатели. 
  Всеки обитател на острова му казал:
  \begin{enumerate}[a)]
  \item
    \marginpar{$\forall x(K(x) \leftrightarrow (\forall xK(x)\vee\forall x\neg K(x)))\rightarrow\ ?$}
    ``Всички тук сме от един и същ вид''.
    Какви са жителите на острова?
  \item
    \marginpar{$\forall x(K(x) \leftrightarrow (\exists xK(x)\wedge\exists x\neg K(x)))\rightarrow\ ?$}
    ``Някои от  нас са благородници и някои от нас са негодници''.
    Какви са жителите на острова?
  \end{enumerate}
\end{problem}
\begin{proof}
  Нека с $K(x)$ да означаваме, че жителят $x$ е благородник (от англ. Knight),
  и съответно с $\neg K(x)$ ще означаваме, че жителят $x$ е негодник.
  \begin{enumerate}[a)]
  \item 
    Tвърдението, което казва, че всички обитетели са от един и същ вид може да се преведе 
    на езика на предикатното смятане като:
    \[\forall x K(x) \vee \forall x \neg K(x).\]
    Тогава, понеже ако обитателят $x$ е благородник, той винаги казва истината, 
    следната формула е вярна:
    \[(\forall x)[K(x) \rightarrow (\forall x K(x) \vee \forall x \neg K(x))].\]
    Съответно ако $x$ е негодник, то той винаги казва лъжа, 
    \[(\forall x)[\neg K(x) \rightarrow \neg (\forall x K(x) \vee \forall x \neg K(x))].\]
    Така получаваме формулата
    \[(\forall x)[K(x) \iff (\forall x K(x) \vee \forall x \neg K(x))].\]
    \begin{itemize}
    \item 
      Ако има благородник на острова, то всички са благородници.
    \item
      Ако има негодник на острова, то има и благородник.
      Но щом има благородник, всички са благородници. Противоречие.
    \end{itemize}
  \item
    \begin{itemize}
    \item
      Ако има негодник на острова, то всички са негодници.
    \item
      Ако има благородник на острова, то има и негодник.
      Щом има негодник, то  всички са негодници. Противоречие.
    \end{itemize}
  \end{enumerate}
\end{proof}

\begin{problem}
  След това пътешественикът попаднал на друг остров, на който
  той силно се интересувал от това дали обитателите пушат.
  Жителите на този остров му отговорили така.
  \begin{enumerate}[a)]
  \item
    \marginpar{$\forall x(K(x) \leftrightarrow \forall y(K(y)\rightarrow S(y))) \rightarrow\ ?$}
    ``Всички благородници пушат.''
    Какви са жителите на острова и пушат ли ?
  \item
    \marginpar{$\forall x(K(x) \leftrightarrow \exists y(\neg K(y)\wedge S(y))) \rightarrow\ ?$}
    ``Някои негодници пушат.''
    Какви са жителите на острова и пушат ли ?
  \end{enumerate}
\end{problem}
\begin{solution}
  \begin{enumerate}[a)]
  \item 
    \begin{itemize}
    \item 
      Има благородник на острова. Тогава всички благородници пушат.
      Ако има също така и негодник на острова, то тогава има благородник, който не пуше.
      Това е  противоречие. Следователно, ако има поне един благородник, то всички обитатели са благородници.
    \item
      Има негодник на острова. Тогава има благородник на острова, който не пуше.
      Но тогава пък всички благородници пушат, което е противоречие.
    \end{itemize}
    Заключаваме, че всички обитателя на острова са благородни пушачи.
  \item
    Да допуснем, че има благородник на острова. Следователно има и негодник, който пуше.
    Но тогава $(\forall y)[K(y) \vee \neg S(y)]$, което означава, че
    всички негодници са непушачи. Достигнахме до противоречие.
    Следователно всички обитатели на острова са негодници, които са непушачи.
  \end{enumerate}
\end{solution}

\begin{problem}
  Пътешественикът отишъл и на трети остров, на който всички обитатели били от един и същ вид.
  \marginpar{Добавяме $\forall xK(x) \vee \forall x\neg K(x)$}
  \begin{enumerate}[a)]
  \item
    \marginpar{$\forall x(K(x) \leftrightarrow (S(x) \rightarrow \forall yS(y)))$}
    ``Ако аз пуша, то всички пушат.''
  \item
    \marginpar{$\forall x(K(x) \leftrightarrow (\exists yS(y) \rightarrow S(x)))$}
    ``Ако някой обитател на острова пуше, то и аз пуша.''
  \item
    \marginpar{$\forall x(K(x) \leftrightarrow (\exists yS(y) \wedge \neg S(x)))$}
    ``Някои пушат, но аз не.''
  \item
    \marginpar{$\forall x(K(x) \leftrightarrow \exists yS(y))\ \wedge$ $\forall x(K(x) \iff \neg S(x))$}
    ``Някои пушат.'' и след това добавили - ``Аз не пуша.''.
  \end{enumerate}
  Какво можем да кажем за обитателите на този остров?
\end{problem}
\begin{solution}
  \begin{enumerate}[a)]
  \item
    Ако всички негодници, то $(\forall x)[S(x)\ \wedge\ \exists y\neg S(y)]$,
    което е невъзможно.
    Следователно, всички са благородници. Тогава или всички пушат или никой не пуше.
  \item
    Ако всички са негодници, то $(\forall x)[\exists y S(y)\ \wedge\ \neg S(x)]$,
    което е невъзможно. Следователно, всички са благородници.
    Тогава или всички пушат или никой не пуше.
  \item
    Ако всички са благородници, то $(\forall x)[\exists y S(y)\ \wedge \neg S(x)]$,
    което е невъзможно. Следователно, всички  са негодници.
    Тогава $(\forall x)[\forall y\neg S(y)\ \vee S(x)]$, което означава, че
    или всички пушат или никой не пуше.
  \item
    Тази ситуация е невъзможна!
  \end{enumerate}
\end{solution}


%%% Local Variables: 
%%% mode: latex
%%% TeX-master: "discrete-math"
%%% End: 

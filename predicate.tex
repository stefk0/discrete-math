\section*{Предикатно смятане}

\subsection*{Закони на предикатното смятане}

\begin{enumerate}[(I)]
  \item
    $\neg\forall x P(x) \iff \exists x \neg P(x)$
  \item
    $\neg\exists x P(x) \iff \forall x \neg P(x)$
  \item
    $\forall x P(x) \iff \neg\exists x \neg P(x)$
  \item
    $\exists x P(x) \iff \neg\forall x \neg P(x)$
  \item
    $\forall x \forall y P(x) \iff \forall y\forall x P(x)$
  \item
    $\exists x\exists y P(x,y) \iff \exists y \exists x P(x)$  
  \item
    $\exists x\forall y P(x,y) \rightarrow \forall y \exists x P(x,y)$
\end{enumerate}


\begin{problem}
  Като използвате символа $<$, логическите връзки и квантори, 
  изразете като формула следните твърдения.
  \begin{enumerate}[1)]
  \item
    $x$ е по-малко от $y$.
  \item
    За всяко число има по-голямо от него.
  \item
    За всяко число има по-малко от него.
  \item
    Всяко число е по-голямо от някое число.
  \end{enumerate}
\end{problem}

\begin{problem}
  Нека $G(x)$ означава, че човекът $x$ е добър.
  \begin{enumerate}[1)]
  \item
    Изразете с формула твърдението, че всички хора са добри {\em без}
    да използвате квантора $\forall$, а само квантора $\exists$ и логическите връзки.
  \item
    Изразете с формула твърдението, че {\em поне един} човек е добър {\em без}
    да използвате квантора $\exists$, а само квантора $\forall$ и логическите връзки.
  \end{enumerate}
\end{problem}

\begin{problem}
  На един остров живеели два вида обитатели - благородници и негодници.
  Благородниците винаги казвали истината, а негодниците винаги лъжели.
  Един пътешественик попаднал на този остров и искал да разбере повече за
  неговите обитатели.
  Всеки обитател на острова му казал:
  \begin{enumerate}[a)]
  \item
    \marginpar{$\forall x(K(x) \leftrightarrow (\forall xK(x)\vee\forall x\neg K(x)))\rightarrow\ ?$}
    ``Всички тук сме от един и същ вид''.
    Какви жителите на острова?
  \item
    \marginpar{$\forall x(K(x) \leftrightarrow (\exists xK(x)\wedge\exists x\neg K(x)))\rightarrow\ ?$}
    ``Някои от  нас са благородници и някои от нас са негодници''.
    Какви жителите на острова?
  \end{enumerate}
\end{problem}

\begin{problem}
  След това пътешественикът попаднал на друг остров, на който
  той силно се интересувал от това дали обитателите пушат.
  \begin{enumerate}[a)]
  \item
    \marginpar{$\forall x(K(x) \leftrightarrow \forall y(K(y)\rightarrow S(y))) \rightarrow\ ?$}
    ``Всички благородници пушат.''
    Какви са жителите на острова и пушат ли ?
  \item
    \marginpar{$\forall x(K(x) \leftrightarrow \exists y(\neg K(y)\wedge S(y))) \rightarrow\ ?$}
    ``Някои негодници пушат.''
    Какви са жителите на острова и пушат ли ?
  \end{enumerate}
\end{problem}

\begin{problem}
  Пътешественикът отишъл и на трети остров, на който всички обитатели били от един и същ вид.
  \marginpar{Добавяме $\forall xK(x) \vee \forall x\neg K(x)$}
  \begin{enumerate}[a)]
  \item
    \marginpar{$\forall x(K(x) \leftrightarrow (S(x) \rightarrow \forall yS(y)))$}
    ``Ако аз пуша, то всички пушат.''
  \item
    \marginpar{$\forall x(K(x) \leftrightarrow (\exists yS(y) \rightarrow S(x)))$}
    ``Ако някой обитател на острова пуше, то и аз пуша.''
  \item
    \marginpar{$\forall x(K(x) \leftrightarrow (\exists yS(y) \wedge \neg S(x)))$}
    ``Някои пушат, но аз не.''
  \item
    \marginpar{$\forall x(K(x) \leftrightarrow \exists yS(y))\ \wedge$ $\forall x(K(x) \iff \neg S(x))$}
    На първия ден всеки му казал, ``Някои пушат.'', 
    а на втория ден, ``Аз не пуша.''.
  \end{enumerate}
  Какво можем да кажем за обитателите на този остров?
\end{problem}



%%% Local Variables: 
%%% mode: latex
%%% TeX-master: "discrete-math"
%%% End: 

\documentclass[a4paper,10pt]{report}
% \documentclass{report}

% \usepackage[papersize={3.6in,4.8in},hmargin=0.1in,vmargin={0.1in,0.1in}]{geometry}  % page geometry

% \setlength{\oddsidemargin}{4cm}
% \setlength{\evensidemargin}{4cm}
%\usepackage{ucs}


% \usepackage[bindingoffset=2cm]{geometry}
\usepackage[english,bulgarian]{babel}
\usepackage[utf8]{inputenc}
\usepackage[colorlinks=true, linkcolor=blue,pdfstartview=FitV,
citecolor=green, urlcolor=blue]{hyperref}
\usepackage{pifont}
\usepackage{amssymb}
\usepackage{amsmath}
\usepackage{mathrsfs}
\usepackage{latexsym}
\usepackage{amsthm}
\usepackage{paralist}
\usepackage{enumerate}
\usepackage{makeidx}
\usepackage{layout}
\usepackage{framed}
\usepackage{bussproofs}

\usepackage{algorithm}
\floatname{algorithm}{Алгоритъм}
%\usepackage{algorithmic}
\usepackage[noend]{algpseudocode}
%\usepackage{algpseudocode}

%%%%%%%%%%%%%%% TIKZ Package %%%%%%%%%%%%%%%%%%%%%%%
\usepackage{tikz}
\usepackage{pgf}
\usetikzlibrary{arrows,automata}
%%%%%%%%%%%%%%%%%%%%%%%%%%%%%%%%%%%%%%%%%%%%%%%%%%%%
\usepackage{caption}
\usepackage{subcaption}

\theoremstyle{definition}
\newtheorem{thm}{Теорема}
\newtheorem{crl}{Следствие}
\newtheorem{cor}{Следствие}
\newtheorem{lemma}{Лема}
\newtheorem{prop}{Твърдение}
\newtheorem{dfn}{Определение}
\newtheorem{problem}{Задача}
\newtheorem{example}{Пример}
\newtheorem{question}{Въпрос}
\newtheorem*{remark}{Забележка}
\renewenvironment{proof}{\noindent{\bf Доказателство.}\hspace*{1em}}{\qed\par}
\newenvironment{solution}{\noindent{\bf Решение.}\hspace*{1em}}{\qed\par}
\newenvironment{hint}{\noindent{\bf Упътване.}\hspace*{1em}}{\qed\par}

\newcommand{\A}{\mathcal{A}}
\newcommand{\B}{\mathcal{B}}
\renewcommand{\C}{\mathcal{C}}
\newcommand{\M}{\mathcal{M}}
\renewcommand{\L}{\mathcal{L}}
\newcommand{\D}{\mathcal{D}}
\newcommand{\R}{\mathbb{R}}
\newcommand{\Z}{\mathbb{Z}}
\newcommand{\N}{\mathcal{N}}
\newcommand{\Q}{\mathbb{Q}}
\newcommand{\Ls}{\mathscr{L}}
\newcommand{\Fs}{\mathscr{F}}
\newcommand{\Rs}{\mathscr{R}}
\newcommand{\Ps}{\mathscr{P}}
\newcommand{\As}{\mathscr{A}}
\newcommand{\Bs}{\mathscr{B}}
\newcommand{\Es}{\mathscr{E}}
\newcommand{\Is}{\mathscr{I}}
\newcommand{\Ss}{\mathscr{S}}
\newcommand{\xn}{x_{1},\dots,x_{n}}

\newcommand{\Nat}{\mathbb{N}}
\newcommand{\Int}{\mathbb{Z}}
\newcommand{\Real}{\mathbb{R}}

\newcommand{\xs}{overline{x}}

\newcommand{\ys}{overline{y}}

\newcommand{\zs}{overline{z}}
\newcommand{\ov}[1]{\overline{#1}}
\newcommand{\abs}[1]{\lvert{#1}\rvert}
\newcommand{\pair}[1]{\langle{#1}\rangle}

\newcommand{\FA}{\langle{Q,\Sigma,s,\delta,F}\rangle}
\newcommand{\FAn}[1]{\langle{Q_#1,\Sigma,s_#1,\delta_#1,F_#1}\rangle}
\newcommand{\NFA}{\langle{Q,\Sigma,s,\Delta,F}\rangle}
\newcommand{\NFAn}[1]{\langle{Q_#1,\Sigma,s_#1,\Delta_#1,F_#1}\rangle}
\newcommand{\PDA}{\langle{Q,\Sigma,\Gamma,\#,s,\Delta,F}\rangle}
\newcommand{\PDAn}[1]{\langle{Q_#1,\Sigma,\Gamma,\#,s_#1,\Delta_#1,F_#1}\rangle}
\newcommand{\CFG}{\langle{V,\Sigma,R,S}\rangle}
\newcommand{\TM}{\langle{Q,\Sigma,\Gamma,\delta,s,\bot,F}\rangle}

\renewcommand{\iff}{\ \leftrightarrow\ }

\newcommand{\Th}[1]{{\em Теорема~\ref{th:#1}}}
\newcommand{\Lem}[1]{{\em Лема~\ref{lem:#1}}}
\newcommand{\Prob}[1]{{\em Задача~\ref{pr:#1}}}
\newcommand{\Prop}[1]{{\em Твърдение~\ref{pr:#1}}}

\setlength{\marginparsep}{0.5cm}
\setlength{\oddsidemargin}{0.3cm}
\setlength{\hoffset}{0cm}
\setlength{\marginparwidth}{110pt}
\let\oldmarginpar\marginpar

\renewcommand\marginpar[1]{\-\oldmarginpar[\raggedleft\scriptsize #1]%
{\raggedright\scriptsize #1}}

% \setlist[itemize]{leftmargin=*}
% \renewcommand\marginpar[1]{}


%\renewcommand\marginpar[1]{\oldmarginpar{\scriptsize #1}}

\fontfamily{garamond}
\title{Записки по Дискретна математика}
\author{Стефан Вътев\thanks{ел. поща: \href{stefanv@fmi.uni-sofia.bg}{stefanv@fmi.uni-sofia.bg}}}
%, Факултет по математика и информатика, Софийски университет ,,Св. Климент Охридски''}}

\usepackage{breqn}         % automatic equation breaking

\makeindex
\begin{document}
\maketitle
\layout

\tableofcontents
\chapter{Основни понятия по логика}

\section{Съждително смятане}
\label{sect:propositional}
\marginpar{На англ. Propositional calculus}

Съждителното смятане наподобява аритметичното смятане.
Вместо аритметичните операции $+,-,\cdot,/$, имаме съждителни операции като $\neg, \wedge, \vee$.
Например, $(p\vee q)\rightarrow \neg  r$ е логически израз.
Освен това, докато аритметичните променливи приемат стойности произволни числа, то
съждителните променливи приемат само стойности {\bf истина (1)} или {\bf лъжа (0)}.

{\bf Съждителен израз} наричаме съвкупността от съждителни променливи $p,q,r,\dots$, свързани със знаците за логически операции
$\neg, \vee, \wedge, \rightarrow, \leftrightarrow$ и скоби, определящи реда на операциите.

\subsection*{Съждителни операции}

\begin{itemize}
\item
  Отрицание $\neg$
\item 
  Дизюнкция $\vee$
\item
  Конюнкция $\wedge$
\item
  Импликация $\rightarrow$
\item
  Еквивалентност $\iff$
\end{itemize}

Ще използваме таблица за истинност за да определим стойностите на основните съждителни операции
при всички възможни набори на стойностите на променливите.

\[
\begin{array}{|c|c|c|c|c|c|c|c|c|}
  \hline
  p & q & \neg p & p \vee q & p \wedge q & p \rightarrow q & \neg p \vee q & p \iff q & \ov{p}q\ \vee\ p\ov{q} \\
  \hline
  0 & 0 & 1 & 0 & 0 & 1 & 1 & 1 & 1\\
  \hline
  0 & 1 & 1 & 1 & 0 & 1 & 1 & 0 & 0\\
  \hline
  1 & 0 & 0 & 1 & 0 & 0 & 0 & 0 & 0\\
  \hline
  1 & 1 & 0 & 1 & 1 & 1 & 1 & 1 & 1\\
  \hline
\end{array}
\]


{\bf Съждително верен} (валиден) е този логически израз, който има верностна стойност {\bf 1} при всички възможни набори на
стойностите на съждителните променливи в израза, т.е. стълбът на израза в таблицата за истинност трябва да съдържа само 
стойности {\bf 1}. 

За два съждителни израза $\varphi$ и $\psi$ са {\bf еквивалентни}, означаваме $\varphi \equiv \psi$, ако са съставени от 
едни и същи съждителни променливи и двата израза имат едни и същи верностни стойности при всички комбинации от верностни 
стойности на променливите. С други думи, колоните на двата израза в таблиците им за истинност трябва да съвпадат.
Например, лесно се вижда, че 
\[p\to q \equiv \neg p \vee q.\]

\subsection*{Съждителни закони}

\begin{enumerate}[I)]
  \item
    {\bf Комутативен закон}
    \[p\vee q \equiv q\vee p\] 
    \[p \wedge q \equiv q \wedge p\]
  \item
    {\bf Асоциативен закон}
    \[(p\vee q)\vee r \equiv p\vee(q\vee r)\]
    \[(p\ \wedge\ q)\ \wedge\ r \equiv p\ \wedge\ (q\ \wedge\ r)\]
  \item
    {\bf Дистрибутивен закон}
    \[p\ \wedge\ (q \vee r) \equiv (p\ \wedge q)\vee (p\ \wedge\ r)\]
    \[p\vee (q\ \wedge\ r) \equiv (p\vee q)\ \wedge\ (p\vee r)\]
  \item
    {\bf Закони на де Морган}
    \[\neg(p \wedge q) \equiv (\neg p \vee \neg q)\]
    \[\neg(p\vee q) \equiv (\neg p \wedge \neg q)\]
  \item
    {\bf Закон за контрапозицията}
    \[p\rightarrow q \equiv \neg q \rightarrow \neg p\]
  \item
    {\bf Обобщен закон за контрапозицията}
    \[(p \wedge q)\rightarrow r \equiv (p \wedge \neg r) \rightarrow \neg q\]
  \item
    {\bf Закон за изключеното трето}
    \[p\vee \neg p \equiv {\mathbf 1}\]
  \item
    {\bf Закон за силогизма (транзитивност)}
    \[[(p\rightarrow q)\ \wedge\ (q\rightarrow r)] \rightarrow (p\rightarrow r) \equiv {\mathbf 1}\]
\end{enumerate}

Лесно се проверява с таблиците за истинност, че законите са валидни.

\begin{example}
  Нека например да проверим едно от правилата на де Морган и закона
  за контрапозицията.
  \[
  \begin{array}{|c|c|c|c|c|c|c|c|c|}
    \hline
    p & q & p\wedge q & \neg(p\wedge q) & \neg p & \neg q & \neg p \vee \neg q & p \rightarrow q & \neg q \rightarrow \neg p\\
    \hline
    0 & 0 & 0 & 1 & 1 & 1 & 1 & 1 & 1 \\
    \hline
    0 & 1 & 0 & 1 & 1 & 0 & 1 & 1 & 1 \\
    \hline
    1 & 0 & 0 & 1 & 0 & 1 & 1 & 0 & 0 \\
    \hline
    1 & 1 & 1 & 0 & 0 & 0 & 0 & 1 & 1 \\
    \hline
  \end{array}
  \]
  При всички стойности на променливите $p$ и $q$, стълбовете съответстващи на $\neg(p \wedge q)$ и $\neg p \vee \neg q$
  съвпадат. Следователно, законът на де Морган е валиден.
  По същия начин се съобразява, че законът за контрапозицията е валиден.
\end{example}


\begin{example}
  Можем да докажем валидността на законите и по друг начин, а именно чрез допускане на противното.
  Така ще докажем, че законът за силогизма е валиден.
  
  Да допуснем, че съществува стойност на променливите $p$,$q$,$r$, за които
  \[\underbrace{[(p\rightarrow q)\ \wedge\ (q\rightarrow r)]}_{\mathbf{1}} \rightarrow \underbrace{(p\rightarrow r)}_{\mathbf{0}} \equiv {\mathbf 0}\]
  Това означава, че
  \[p \equiv \mathbf{1}, r \equiv \mathbf{0}.\]
  Тогава
  \[\underbrace{[(\mathbf{1}\rightarrow q)\ \wedge\ (q\rightarrow \mathbf{0})]}_{\mathbf{1}} \rightarrow \underbrace{(\mathbf{1}\rightarrow \mathbf{0})}_{\mathbf{0}} \equiv {\mathbf 0}\]
  \begin{itemize}
  \item 
    Ако $q \equiv \mathbf{0}$, то $(\mathbf{1}\rightarrow \mathbf{0})\ \wedge\ (\mathbf{0}\rightarrow \mathbf{0}) \equiv \mathbf{0} \wedge \mathbf{1} \equiv \mathbf{0}$,
    следователно този случай е невъзможен.
  \item
    Ако $q \equiv \mathbf{1}$, то $(\mathbf{1}\rightarrow \mathbf{1})\ \wedge\ (\mathbf{1}\rightarrow \mathbf{0}) \equiv \mathbf{1} \wedge \mathbf{0} \equiv \mathbf{0}$,
    следователно този случай също е невъзможен.
  \end{itemize}
  И в двата случая за $q$ достигнахме до противоречие.
  Следователно нашето допускане не е вярно, което означава, че
  при всяка стойност на променливите $p$, $q$, $r$,
  \[[(p\rightarrow q)\ \wedge\ (q\rightarrow r)] \rightarrow (p\rightarrow r) \equiv {\mathbf 1}.\]
\end{example}


% \begin{tabular}{|c|c|c|c|c|c|c|c|c|}
%   \hline
%   $p$ & $q$ & $r$ & $p \rightarrow q$& $q \rightarrow r$ & $p \rightarrow r$ & $(p \rightarrow q) \wedge (q \rightarrow r)$ & $[(p \rightarrow q) \wedge (q \rightarrow r)] \rightarrow (q \rightarrow r)$\\
%   \hline
%   $0$ & $0$ & $0$ & & & & & \\
%   $0$ & $0$ & $0$ & & & & & \\
%   $0$ & $0$ & $0$ & & & & & \\
%   $0$ & $0$ & $0$ & & & & & \\
%   $0$ & $0$ & $0$ & & & & & \\
%   $0$ & $0$ & $0$ & & & & & \\
%   $0$ & $0$ & $0$ & & & & & \\
%   $0$ & $0$ & $0$ & & & & & \\
%   \hline
% \end{tabular}

\begin{problem}
  Проверете дали следните съждителни формули са валидни.
  \begin{enumerate}[a)]
  \item
    $(p\wedge q)\rightarrow p$;
  \item
    $p\rightarrow(p\vee q)$;
  \item
    $(p\rightarrow q) \iff (\neg q \rightarrow \neg p)$;
  \item
    $p\rightarrow q \equiv \neg p \vee q$
  \item
    $(p\ \wedge\ q) \rightarrow r \equiv p \rightarrow (q\rightarrow r)$
  \item
    $p\ \wedge\ q \equiv \neg(\neg p \vee \neg q)$
  \item
    $p \leftrightarrow q \equiv (p\rightarrow q)\ \wedge\ (q\rightarrow p)$
  \item
    $\neg(p\wedge q) \equiv (\neg p \vee \neg q)$;
  \item
    $\neg(p\vee q) \equiv (\neg p \wedge \neg q)$;
  \item
    $\neg(p\rightarrow q) \equiv (p\wedge \neg q)$.
  \end{enumerate}
\end{problem}

\begin{remark}
  Обърнете внимание, че $(p\rightarrow q)\rightarrow r$ {\bf не} е еквивалентно на $p\rightarrow (q\rightarrow r)$.
  Например вземете $p \equiv q \equiv r \equiv \mathbf{0}$.
\end{remark}


% \begin{problem}
%   В един затвор имало трима арестанти - {\bf А}, {\bf Б} и {\bf В}, който дори бил и сляп.
%   Шерифът решил да пусне един от тях на свобода, затова завързал очите на {\bf А} и {\bf Б} и
%   им сложил по една шапка на главите. Шапките били общо пет, като три от тях били бели и две - черни.
%   След това шерифът им свалил превръзките на очите и им казал, че ще пусне този, който позне какъв
%   цвят е шапката му.
%   \begin{enumerate}[]
%   \item
%     {\bf А} казал, че не знае какъв цвят е шапката му и шерифът го върнал в ареста.
%   \item
%     {\bf Б} казал, че не знае какъв цвят е шапката му и шерифът го върнал в ареста.
%   \item
%     Слепият затворник {\bf В} отговорил правилно какъв цвят е шапката му и шерифът го пуснал.
%   \item
%     Какъв е цвета на шапката на {\bf В}?
%   \end{enumerate}
% \end{problem}
% \begin{proof}
%   Нека да означим с $а$ твърдението ``цветът на шапката на {\bf А} e черен'',
%   а съответно с $\ov{a}$ твърдението ``цветът на шапката на {\bf А} e бял'',
%   По аналогичен начин означаваме съждителните променливи $b$ и $c$ за арестантите {\bf Б} и {\bf В}.
%   Превеждаме твърденията в съждителни формули:
%   \begin{enumerate}[A)]
%   \item
%     Щом {\bf А} не знае какъв цвят е шапката му, то със сигурност шапките на главите на другите арестанти 
%     не са и двете черни. Получаваме
%     \[\ov{b}c\ \vee\ b\ov{c}\ \vee\ \ov{b}\ov{c}\]
%   \item
%     Щом и {\bf Б} не знае какъв цвят е шапката му, то шапката на {\bf В} не е черна, защото това ще означава, че
    
%   \end{enumerate}

  
%   Получаваме $(\ov{b}c\ \vee\ b\ov{c}\ \vee\ \ov{b}\ov{c})\ \wedge\ (a\ov{c}\ \vee\ \ov{a}\ov{c})$
% \end{proof}
\begin{remark}
  За удобство, понякога ще пишем $\ov{p}$ вместо $\neg p$ и $pq$ вместо $p \wedge q$.
\end{remark}


\begin{problem}
  \marginpar{(От \cite{smullyan})}
  Да предположим, че сме на остров, който се обитава негодници и благородници.
  Негодниците винаги лъжат, а благородниците винаги казват истината.
  Срещаме трима обитатели на този остров, наречени {\bf А}, {\bf Б} и {\bf В}.
  \begin{enumerate}[a)]
  \item
    \begin{enumerate}[]
    \item
      \marginpar{$a \iff \ov{a}\ov{b}\ov{c}$}
      {\bf А} казва ``Всички сме негодници''.
    \item
      \marginpar{$b \iff (\ov{a}\ov{b}c\vee \ov{a}b\ov{c}\vee a\ov{b}\ov{c})$}
      {\bf Б} казва ``Точно един от нас е благородник''.
    \item
      Какви са {\bf А},{\bf Б} и {\bf В}?
    \end{enumerate}
  \item
    \begin{enumerate}[]
    \item
      \marginpar{$a \iff \ov{a}\ov{b}\ov{c}$}
      {\bf А} казва ``Всички сме негодници''.
    \item
      \marginpar{$b \iff (\ov{a}bc\vee ab\ov{c}\vee a\ov{b}c)$}
      {\bf Б} казва ``Точно един от нас е негодник''.
    \item
      Може ли да определим какъв е {\bf Б}?
    \item
      Може ли да определим какъв е {\bf В}?
    \end{enumerate}
  \item
    \begin{enumerate}[]
    \item
      \marginpar{$a \iff \ov{b}$}
      {\bf А} казва ``{\bf Б} е негодник''.
    \item
      \marginpar{$b \iff (ac \vee \ov{a}\ov{c})$}
      {\bf Б} казва ``{\bf А} и {\bf В} са от един и същ тип, т.е. или и двамата са благородници, или и двамата са негодници''.
    \item
      Какъв е {\bf В}?
    \end{enumerate}
  \end{enumerate}
\end{problem}
\begin{proof}
  \begin{enumerate}[a)]
  \item
    Нека съждителната променлива $a$ да има стойност {\bf 1}, ако {\bf A} е благородник и нека има стойност {\bf 0}, 
    ако {\bf A} е негодник.
    Тогава
    \begin{enumerate}[A)]
    \item
      Ако {\bf А} е благородник, то {\bf А,Б,В} са негодници се превежда на езика на съждителното смятане като
      \[a \rightarrow \ov{a}\ov{b}\ov{c}.\]
      Ако {\bf А} е негодник, то той лъже, следователно не е вярно, че всички са негодници. Това се превежда на езика на съждителното смятане като
      \[\ov{a} \rightarrow \ov{\ov{a}\ov{b}\ov{c}},\] 
      което е еквивалентно на \[\ov{a}\ov{b}\ov{c}\rightarrow a.\]
      Следователно, в двата случая за {\bf A} получаваме
      \[(a \rightarrow \ov{a}\ov{b}\ov{c}) \wedge (\ov{a}\ov{b}\ov{c}\rightarrow a) \equiv \mathbf{1},\]
      или
      \[a \iff \ov{a}\ov{b}\ov{c} \equiv \mathbf{1}.\]
      Сега получаваме следните еквивалентни преобразования:
      \begin{align*}
        a \iff \ov{a}\ov{b}\ov{c} & \equiv\ a\ov{a}\ov{b}\ov{c} \vee \ov{a}(\ov{\ov{a}\ov{b}\ov{c}})\\
        & \equiv\ \ov{a}(a \vee b \vee c)\\
        & \equiv\ \ov{a}b \vee \ov{a}c\\
        & \equiv\ \mathbf{1}.
      \end{align*}
    \item
      Правим аналогични разсъждения и за {\bf Б}.
      \begin{align*}
        b \iff (\ov{a}\ov{b}c\vee \ov{a}b\ov{c}\vee a\ov{b}\ov{c}) & \equiv\ b(\ov{a}\ov{b}c\vee \ov{a}b\ov{c}\vee a\ov{b}\ov{c})\vee \ov{b}(\ov{\ov{a}\ov{b}c\vee \ov{a}b\ov{c}\vee a\ov{b}\ov{c}})\\
        & \equiv\ \ov{a}b\ov{c}\ \vee\ \ov{b}(a\vee b \vee \ov{c})(a\vee\ov{b}\vee c)(\ov{a} \vee b \vee c)\\
        & \equiv\ \ov{a}b\ov{c}\ \vee\ \ov{b}(a \vee a\ov{b} \vee ac \vee ab \vee bc \vee a\ov{c} \vee \ov{b}\ov{c})(\ov{a} \vee b \vee c)\\
        & \equiv\ \ov{a}b\ov{c}\ \vee\ \ov{b}(a \vee bc \vee \ov{b}\ov{c})(\ov{a} \vee b \vee c)\\
        & \equiv\ \ov{a}b\ov{c}\ \vee\ \ov{a}\ov{b}\ov{c} \vee  a\ov{b}c.\\
        & \equiv\ \mathbf{1}. 
      \end{align*}
    \item
      Сега взимаме конюнкцията на А) и Б).
      \[(\ov{a}b \vee \ov{a}c)\wedge (\ov{a}\ov{b}\ov{c} \vee  a\ov{b}c \vee \ov{a}b\ov{c}) \equiv\ \ov{a}b\ov{c} \equiv {\mathbf 1}.\]
      Заключаваме, че {\bf А} и {\bf В} са негодници, а {\bf Б} е благородник.
    \end{enumerate}
  \end{enumerate}
\end{proof}

%%% Local Variables: 
%%% mode: latex
%%% TeX-master: "discrete-math"
%%% End: 

\section*{Предикатно смятане}

\subsection*{Закони на предикатното смятане}

\begin{enumerate}[(I)]
  \item
    $\neg\forall x P(x) \iff \exists x \neg P(x)$
  \item
    $\neg\exists x P(x) \iff \forall x \neg P(x)$
  \item
    $\forall x P(x) \iff \neg\exists x \neg P(x)$
  \item
    $\exists x P(x) \iff \neg\forall x \neg P(x)$
  \item
    $\forall x \forall y P(x) \iff \forall y\forall x P(x)$
  \item
    $\exists x\exists y P(x,y) \iff \exists y \exists x P(x)$  
  \item
    $\exists x\forall y P(x,y) \rightarrow \forall y \exists x P(x,y)$
\end{enumerate}


\begin{problem}
  Като използвате символа $<$, логическите връзки и квантори, 
  изразете като формула следните твърдения.
  \begin{enumerate}[1)]
  \item
    $x$ е по-малко от $y$.
  \item
    За всяко число има по-голямо от него.
  \item
    За всяко число има по-малко от него.
  \item
    Всяко число е по-голямо от някое число.
  \end{enumerate}
\end{problem}

\begin{problem}
  Нека $G(x)$ означава, че човекът $x$ е добър.
  \begin{enumerate}[1)]
  \item
    Изразете с формула твърдението, че всички хора са добри {\em без}
    да използвате квантора $\forall$, а само квантора $\exists$ и логическите връзки.
  \item
    Изразете с формула твърдението, че {\em поне един} човек е добър {\em без}
    да използвате квантора $\exists$, а само квантора $\forall$ и логическите връзки.
  \end{enumerate}
\end{problem}

\begin{problem}
  На един остров живеели два вида обитатели - благородници и негодници.
  Благородниците винаги казвали истината, а негодниците винаги лъжели.
  Един пътешественик попаднал на този остров и искал да разбере повече за
  неговите обитатели.
  Всеки обитател на острова му казал:
  \begin{enumerate}[a)]
  \item
    \marginpar{$\forall x(K(x) \leftrightarrow (\forall xK(x)\vee\forall x\neg K(x)))\rightarrow\ ?$}
    ``Всички тук сме от един и същ вид''.
    Какви жителите на острова?
  \item
    \marginpar{$\forall x(K(x) \leftrightarrow (\exists xK(x)\wedge\exists x\neg K(x)))\rightarrow\ ?$}
    ``Някои от  нас са благородници и някои от нас са негодници''.
    Какви жителите на острова?
  \end{enumerate}
\end{problem}

\begin{problem}
  След това пътешественикът попаднал на друг остров, на който
  той силно се интересувал от това дали обитателите пушат.
  \begin{enumerate}[a)]
  \item
    \marginpar{$\forall x(K(x) \leftrightarrow \forall y(K(y)\rightarrow S(y))) \rightarrow\ ?$}
    ``Всички благородници пушат.''
    Какви са жителите на острова и пушат ли ?
  \item
    \marginpar{$\forall x(K(x) \leftrightarrow \exists y(\neg K(y)\wedge S(y))) \rightarrow\ ?$}
    ``Някои негодници пушат.''
    Какви са жителите на острова и пушат ли ?
  \end{enumerate}
\end{problem}

\begin{problem}
  Пътешественикът отишъл и на трети остров, на който всички обитатели били от един и същ вид.
  \marginpar{Добавяме $\forall xK(x) \vee \forall x\neg K(x)$}
  \begin{enumerate}[a)]
  \item
    \marginpar{$\forall x(K(x) \leftrightarrow (S(x) \rightarrow \forall yS(y)))$}
    ``Ако аз пуша, то всички пушат.''
  \item
    \marginpar{$\forall x(K(x) \leftrightarrow (\exists yS(y) \rightarrow S(x)))$}
    ``Ако някой обитател на острова пуше, то и аз пуша.''
  \item
    \marginpar{$\forall x(K(x) \leftrightarrow (\exists yS(y) \wedge \neg S(x)))$}
    ``Някои пушат, но аз не.''
  \item
    \marginpar{$\forall x(K(x) \leftrightarrow \exists yS(y))\ \wedge$ $\forall x(K(x) \iff \neg S(x))$}
    На първия ден всеки му казал, ``Някои пушат.'', 
    а на втория ден, ``Аз не пуша.''.
  \end{enumerate}
  Какво можем да кажем за обитателите на този остров?
\end{problem}



%%% Local Variables: 
%%% mode: latex
%%% TeX-master: "discrete-math"
%%% End: 

\chapter{Теория на множествата}


\section{Декартово произведение}
  Въвеждаме операция наредена двойка $\langle{x,y}\rangle$, която искаме да има следните свойства:
  \begin{enumerate}
  \item
    $\langle{x,y}\rangle = \langle{x',y'}\rangle \iff x = x' \ \&\ y = y'$;
  \item
    класът $A\times B = \{\langle{x,y}\rangle\ \mid\ x\in A\ \&\ x\in B\}$ е множество.
\end{enumerate}


\begin{dfn}[Куратовски]
  Наредена двойка\index{наредена двойка} $\langle{x,y}\rangle = \{\{x\},\{x,y\}\}$
\end{dfn}

Първото свойство се проверява лесно.
За второто свойство, достатъчно е да покажем, че за произволни множества $A,B$ можем да 
изберем множество $C$, за което е изпълнено, че
\[x\in A\ \&\ x\in B \rightarrow \{x,\{x,y\}\}\in C.\]
Ако успеем да намерим такова множество $C$, то тогава от аксиомата за отделянето следва, че $A\times B$
е множество, защото $A\times B = \{ z\in C\ \mid\ (\exists x\in A)(\exists y\in B)[z = \langle{x,y}\rangle]\}$ е множество.

Лесно може да се провери, че $C = \Ps(\Ps(A\cup B))$ върши работа.

Възможно е да се дадат и други дефиниции на наредена двойка.
\begin{problem}
  Проверете кои от следните операции отговарят на условията за наредена двойка.
  \begin{enumerate}
  \item
    $\langle{x,y}\rangle_{1} = \{x,y\}$;
  \item
    $\langle{x,y}\rangle_{2} = \{x,\{y\}\}$;
  \item
    $\langle{x,y}\rangle_{3} = \{\{\emptyset,\{x\}\},\{\{y\}\}\}$;
  \item
    $\langle{x,y}\rangle_{4} = \{\{0,x\},\{1,y\}\}$, 
    където $0,1$ са различни обекти.
\end{enumerate}
\end{problem}



\begin{problem}
  Проверете:
  \begin{enumerate}
  \item
    $A\times B = \emptyset \iff A = \emptyset \vee B = \emptyset$
  \item
    $A\times(B\cup C) = (A\times B)\cup(A\times C)$
  \item
    $A\times(B\cap C) = (A\times B)\cap(A\times C)$ 
  \item
    $A\times(B\backslash C) = (A\times B)\backslash(A\times C)$
  \item
    $(A\cap B)\times (C\cap D) = (A\times C)\cap(B\times D)$
  \item
    $(A\cup B)\times (C\cup D) = (A\times C)\cup(B\times D)$
  \item
    $(A\backslash C)\times(B\backslash D)\subsetneq (A\times B)\backslash(C\times D)$
  \end{enumerate}
\end{problem}

\section{Операции върху множества}

Дефинираме следните операции върху множества:
\begin{enumerate}[(i)]
  \item
    Сечение, $A\cap B = \{x\ \mid\ x\in A\ \&\ x\in B\}$;
  \item
    Обединение, $A\cup B = \{x\ \mid x\in A\ \vee\ x\in B\}$
  \item
    $\bigcup^{n}_{i=1} A_i = \{x \mid \exists i (1\leq i\leq n\ \&\ x\in A_i \}$;
  \item
    $\bigcap^{n}_{i=1} A_i = \{x \mid \forall i (1\leq i\leq n \rightarrow x\in A_i)\}$;
  \item
    Разлика, $A\setminus B = \{x\ \mid\ x\in A\ \&\ x\not\in B\}$;
  \item
    Симетрична разлика, $A\triangle B = (A\backslash B)\cup (B\backslash A)$;
  \item
    $\bigcup A = \{x\mid (\exists y\in A)[x\in y]\}$;
  \item
    $\bigcap A = \{x\mid (\forall y\in A)[x\in y]\}$;
  \item
    Степенно множество, $\Ps A = \{x\mid x\subseteq A\}$.
\end{enumerate}

Тук имаме проблем с значението на $\bigcap\emptyset$.
На пръв поглед изглежда, че $\bigcap\emptyset$ е множеството от всички множества $V$, 
но ние знаем, че такова множество не съществува.
Това в известен смисъл е аналог на делението на нула.
Ние ще приемем, че $\bigcap\emptyset = \emptyset$.


\begin{example}
  Нека $A = \{x\in\N\mid x > 1\}$ и $B = \{x\in\N\mid x>3\}$. Тогава :
    \begin{enumerate}[]
    \item
      $A\cap B = \{x\in\N\mid x > 3\}$,
    \item
      $A\cup B = \{x\in\N\mid x > 1\}$,
    \item
      $A\setminus B = \{x\in\N\mid 1<x\leq 3\}$,
    \item
      $B\setminus A = \emptyset$,
    \item
      $A\triangle B = \{x\in\N\mid 1<x\leq 3\}$
    \end{enumerate}
\end{example}


\begin{problem}
  Нека $A = \{x\in\R\mid |x|\leq 1\}$ и $B = \{x\in\R\mid |x-1|\leq \frac{1}{2}\}$.
  Намерете $A\cup B$, $A\cap B$, $A\setminus B$, $B\setminus A$, $A\triangle B$.
\end{problem}



\begin{example}
  \[\bigcap\{\{1,2,3,4\},\{2,4\},\{1,3,4\}\} = \{4\}\]
  \[\bigcup\{\{3\},\{2,4\},\{1,4\}\} = \{1,2,3,4\}\]
  \[\bigcap\{\{a\},\{a,b\}\} = \{a\}\cap\{a,b\} = \{a\}\]
  \[\bigcup\bigcap\{\{a\},\{a,b\}\}  = \bigcup\{a\} = a\]
\end{example}


\begin{problem}
  Нека $B = \{\{1,2\},\{2,3\}, \{1,3\}, \{\emptyset\}\}$.
  Намерете $\bigcup{B}$, $\bigcap{B}$, $\bigcap\bigcup{B}$ и $\bigcup\bigcap{B}$.
\end{problem}


\begin{example}
  Ето няколко примера, които показват действието на някои от операциите
  \begin{enumerate}[1)]
  \item
    \begin{enumerate}[]
    \item
      $\Ps\emptyset = \{\emptyset\}$
    \item
      $\Ps\{\emptyset\} = \{\emptyset,\{\emptyset\}\}$
    \item
      $\Ps\{\emptyset,\{\emptyset\}\} = \{\emptyset,\{\emptyset\},\{\{\emptyset\}\}, \{\emptyset,\{\emptyset\}\}\}$
    \end{enumerate}
  \item
    \begin{enumerate}[]
    \item
      $\bigcup\{\emptyset\} = \emptyset$
    \item
      $\bigcup\{\emptyset,\{\emptyset\}\} = \{\emptyset\}$
    \item      
      $\bigcup\{\emptyset,\{\emptyset\},\{\{\emptyset\}\}, \{\emptyset,\{\emptyset\}\}\} = \{\emptyset,\{\emptyset\}\}$
    \end{enumerate}
  \item
    $\bigcap\{\emptyset,\{\emptyset\}\} = \emptyset$
\end{enumerate}
\end{example}



\begin{problem}
  \begin{enumerate}
  \item
    Намерете двуелементно множество такова, че всеки елемент на множеството да е също и негово подмножество.
  \item
    Намерете триелементно множество такова, че всеки елемент на множеството да е също и негово подмножество.
  \item
    Намерете четириелементно множество такова, че всеки елемент на множеството да е също и негово подмножество.
\end{enumerate}
\end{problem}


\begin{problem}
  Докажете:
  \begin{enumerate}
  \item
    $\bigcup\Ps A = A$;
  \item
    $A\subseteq\Ps\bigcup A$; кога имаме равенство?
  \item
    $\Ps A \cap \Ps B = \Ps(A\cap B)$;
  \item
    $\Ps A \cup \Ps B \subseteq\Ps(A\cup B)$; кога имаме равенство?
  \item
    съществуват множества $a$ и $B$, за които $a\in B$, но $\Ps{a}\not\subseteq\Ps{B}$;
  \item
    ако $a\in B$, то $\Ps{a}\in\Ps\Ps{B}$;
  \item
    $\{\emptyset,\{\emptyset\}\} \in \Ps\Ps{A}$, за всяко множество $A$.
  \end{enumerate}
\end{problem}



\begin{problem}
  Проверете:
\begin{enumerate}[a)]
  \item
    $A\cup(B\cap C) = (A\cup B)\cap(A\cup C)$
  \item
    $X\subseteq A\ \&\ X\subseteq B \rightarrow X\subseteq A\cap B$
  \item
    $A\subseteq X\ \&\ B\subseteq X \rightarrow A\cup B\subseteq X$
  \item
    $(\bigcup^{n}_{i=1} A_i) \cap B = \bigcup^{n}_{i=1} (A_i \cap B)$
  \item
    $(\bigcap^{n}_{i=1} A_i) \cup B = \bigcap^{n}_{i=1} (A_i \cup B)$
  \item
    $A\subseteq B \iff A\setminus B = \emptyset \iff A\cup B = B \iff A\cap B = A$
  \item
    $A\backslash B = A \iff A\cap B = \emptyset$
  \item
    $A\backslash B = A\backslash (A\cap B)$
  \item
    $(A\cup B)\setminus C = (A\setminus C) \cup (B\setminus C)$
  \item
    $X\backslash (A\cup B) = (X\backslash A)\cap(X\backslash B)$
  \item
    $X\backslash(\bigcup^{n}_{i=1} A_i) = \bigcap^{n}_{i=1} (X\backslash A_i)$
  \item
    $X\backslash (A\cap B) = (X\backslash A)\cup(X\backslash B)$
  \item
    $X\backslash(\bigcap^{n}_{i=1} A_i) = \bigcup^{n}_{i=1} (X\backslash A_i)$
  \item
    $A\cup\bigcap B = \{A\cup X\mid X\in B\}$, за $B\neq\emptyset$
  \item
    $A\cap\bigcup B = \{A\cap X\mid X\in B\}$
  \item
    $(A\backslash B)\backslash C = (A\backslash C)\backslash(B \backslash C)$
  \item
    $A\backslash (B\backslash C) = (A\backslash B) \cup (A\cap C)$
  \item
    $A\triangle B = B\triangle A$
  \item
    $A\triangle(B\triangle C) = (A\triangle B)\triangle C$
  \item
    $A\backslash B = A\triangle(A\cap B)$
  \item
    $A\cap(B\triangle C) = (A\cup B)\triangle(A\cup C)$
  \item
    $A\cup B = (A\triangle B)\cup(A\cap B)$
  \item
    $A\triangle B = \emptyset \iff A = B$
  \item
    $A\triangle B = C \iff B\triangle C = A \iff C\triangle A = B$
  \end{enumerate}
\end{problem}

\begin{problem}
  Да се решат системите с променлива $X$:
  \begin{enumerate}[(a)]
  \item
    \begin{tabular}{l c l}
      $\big|A\setminus X$ & $= $ & $ B$\\
      $\big|X\setminus A $ & $=$ & $ C$
    \end{tabular}, където са дадени множествата $A,B,C$ и $B\subseteq A$, $A\cap C = \emptyset$;
  \item
    \begin{tabular}{l c l}
      $\big|A\cap X$ & $= $ & $ B$\\
      $\big|A\cup X $ & $=$ & $ C$
    \end{tabular}, където са дадени множествата $A,B,C$ и $B\subseteq A\subseteq C$;
  \item
    \begin{tabular}{l c l}
      $\big|A\setminus X$ & $= $ & $ B$\\
      $\big|A\cup X $ & $=$ & $ C$
    \end{tabular}, където са дадени множествата $A,B,C$ и $B\subseteq A\subseteq C$.
  \end{enumerate}
\end{problem}



\begin{problem}
  Нека множеството $A$ е дефинирано по следния начин:
  \begin{enumerate}
  \item
    $0\in A$
  \item
    Ако $x\in A$, то $2x+1 \in A$.
\end{enumerate}
Намерете $A$.
\end{problem}
\begin{proof}
  $A = \{2^n - 1\ \mid n\in\N\}$.
\end{proof}

\begin{thm}
  Нека множеството $A$ е дефинирано по следния начин:
  \begin{enumerate}[(1)]
  \item
    $1\in A$
  \item
    Ако $m,n\in A$, то $2m+3n \in A$.
  \item
    Всички елементи на $A$ са добавени или по правило (1) или правило (2).
\end{enumerate}
Намерете $A$.
\end{thm}
\begin{proof}
  Нека $B = \{n \mid n\equiv 1 (mod 12)\ \vee n\equiv 5 (mod 12) \}$.
  Искаме да докажем, че $A = B$.
  Първо ще докажем, че $A\subseteq B$.
  За целта проверяваме, че $1\in B$ и ако $m,n \in B$, то $2m+3n \in B$.
  
  За другата посока, т.е. $B\subseteq A$, трябва да докажем, че ако
  за всяко $k\leq n$ е вярно, че $12k+1 \in B$ и $12k + 5 \in B$,
  то е вярно, че $12(n+1)+1 \in B $ и $12(n+1) + 5 \in B$.
\end{proof}



\section{Релации}


\begin{dfn}
  Една релация $R \subseteq A^2$ е:
  \begin{enumerate}[1)]
  \item
    антирефлексивна, ако
    $(\forall x\in A)[(x,x)\not\in R]$
  \item
    рефликсивна, ако
    $(\forall x\in A)((x,x)\in R)$;
  \item
    транзитивна, ако
    $(\forall x,y,z\in A)((x,y)\in R\ \&\ (y,z)\in R \rightarrow (x,z)\in R)$;
  \item
    симетрична, ако
    $(\forall x,y\in A)((x,y)\in R \rightarrow (y,x)\in R)$;
  \item
    антисиметрична, ако
    $(\forall x,y\in A)((x,y)\in R\ \&\ (y,x)\in R \rightarrow x = y)$;
  \item
    асиметрична, ако
    $(\forall x,y)[(x,y)\in R \rightarrow (y,x)\not\in R]$.
\end{enumerate}
\end{dfn}

\begin{enumerate}[(i)]
\item
  Композиция на релации
  $S\circ T = \{\langle{x,y}\rangle \mid (\exists z)[\langle{x,z}\rangle\in T\ \&\ \langle{z,y}\rangle \in S]\}$;
\item
  $R^{-1} = \{\langle{x,y}\rangle \mid \langle{y,x}\rangle \in R\}$;
\item
  $\overline{R} = \{\langle{x,y}\rangle \in A\times B \mid\langle{x,y}\rangle\not\in R\}$;
\end{enumerate}

  
\begin{problem}
  Докажете, че:
  \begin{enumerate}[a)]
  \item
    $R$ е симетрична тогава и само тогава, когато $R^{-1}\subseteq R$;
  \item
    $R$ е транзитивна тогава и само тогава, когато $R\circ R\subseteq R$;
  \item
    $R$ е транзитивна и симетрична тогава и само тогава, когато $R = R^{-1}\circ R$.
\end{enumerate}
\end{problem}
\begin{proof}
  \begin{enumerate}[a)]
  \item
    Задачата се разделя на две подзадачи.
    \begin{enumerate}[(i)]
    \item
      Нека $R$ да бъде симетрична. Ще докажем, че $R^{-1}\subseteq R$, т.е.
      \[(\forall x\forall y)[(x,y)\in R^{-1} \rightarrow (y,x)\in R].\]
      Нека $(x,y)\in R^{-1}$. Тогава имаме, че $(y,x)\in R$ и следователно $(x,y)\in R$,
      защото $R$ е симетрична.
    \item
      Нека $R^{-1}\subseteq R$. Щe докажем, че $R$ е симетрична, т.е.
      \[(\forall x\forall y)[(x,y)\in R \rightarrow (y,x)\in R].\]
      Нека $(x,y)\in R$, следователно $(y,x)\in R^{-1}$.
      Тогава от $R^{-1}\subseteq R$ следва, че $(y,x)\in R$.
    \end{enumerate}
  \end{enumerate}
\end{proof}


\begin{problem}
  Нека $R$ да бъде релация на еквивалентност върху $B$ и $f:A\to B$.
  Дефинираме множеството \[Q = \{((x,y)\in A\times A\mid (f(x),f(y))\in R\}.\]
  Докажете, че $Q$ е релация на еквивалентност.
\end{problem}

\begin{problem}
  Нека $\{(a,b)\}\subseteq R$, за някои $a\neq b$.
  Докажете, че ако $R$ е симетрична, то $R$ не е антисиметрична.
\end{problem}
\begin{proof}
  Нека $R$ е симетрична.
\end{proof}



\begin{problem}
  Проверете за $R$ дали е рефлексивна, транзитивна, симетрична, антисиметрична или асиметрична релация.
  \begin{enumerate}[a)]
  \item
    $R\subsetneq \mathbb{N}^2, aRb \iff a | b$ 
  \item
    $R\subseteq \R^2 , aRb \iff a.b > 0$ 
  \item
    $R\subseteq \R^2, aRb \iff a+b = 0$
  \item
    $R\subseteq \R^2, aRb \iff a+b = 5$ 
  \item
    $R\subseteq \R^2, aRb \iff a+b\mbox{ е четно }$ 
  \item
    $R\subseteq (\R^2)^2, \langle{a,b}\rangle R \langle{c,d}\rangle \iff a+d = b+c$ 
  \item
    $R\subseteq (\R^2)^2, \langle{a,b}\rangle R \langle{c,d}\rangle \iff a.d = b.c$ 
  \item
    $R_{m}\subseteq \Z^2, m\in \Z, m>0, aR_{m}b \iff m\mid (a - b)$ 
  \item
    $R\subseteq \R^2, xRy \iff (x-y)\mbox{ е рационално число}$ 
  \item
    $aRb \iff a,b\in\N\ \&\ (a = b \vee a+1 = b)$ 
  \item
    $aRb \iff a,b\in\N\ \&\ (\exists k\in\N)(a+k = b)$
  \item
    Нека $\leq_1$ е ч.н. върху $A$, $\leq_2$ е ч.н. върху $B$.
    $\langle{a,b}\rangle R\langle{c,d}\rangle \iff a\leq_{1}c\ \&\ b\leq_{2}d$ 
  \item
    Нека $\leq_1$ е ч.н. върху $A$, $\leq_2$ е ч.н. върху $B$.
    $\langle{a,b}\rangle R\langle{c,d}\rangle \iff a\leq_{1}c\ \vee\ b\leq_{2}d$
  \item
    $f:X\rightarrow Y$, $R\subseteq (2^{X})^{2}, ARB \iff f(A) = f(B)$ 
  \end{enumerate}
\end{problem}


\begin{dfn}
  \begin{enumerate}
  \item
    Азбука е крайно множество $X = \{a_1,\dots,a_n\}, \varepsilon\not\in X$.
    Елементите на $X$ наричаме букви.
  \item
    Думи над азбуката $X$ са:
    \begin{enumerate}
    \item
      $\varepsilon$ е дума над $X$, наричаме я празната дума.
    \item
      Нека $\alpha$ е дума над $X$. 
      Тогава за всяко $i\leq n$ имаме, че $\alpha a_i$ е дума над $X$;
    \item
      няма други думи над $X$.
    \end{enumerate}
  \item
    Нека $\alpha=a_{i_1}\dots a_{i_m}$ и $\beta=b_{j_1}\dots b_{j_n}$ са думи над $X$.
    $\alpha$ е начало на $\beta$, ако $m\leq n$ и $(\forall k\leq m)(a_{i_k} = b_{j_k})$.
    $\alpha$ е край на $\beta$, ако $m\leq n$ и $(\forall k\leq m)(a_{i_{(m-k)}} = b_{j_{(n-k)}})$.
  \end{enumerate}
  Означаваме с $X^n$ множеството от всички думи с дължина $n$ над азбуката $X$, $X^0 = \{\varepsilon\}$.
  С $X^{*}$ означаваме множеството от всички думи над азбуката $X$, т.е. $X^{*} = \bigcup_{0\leq n} X^{n}$.
\end{dfn}

Дефинираме дължината $|\alpha|$ на думите $\alpha \in X^*$ с индукция по построението на $\alpha$.
\begin{enumerate}[(i)]
  \item
    Ако $\alpha = \varepsilon$, то $|\alpha| = 0$;
  \item
    Ако $\alpha = \beta a$, за някоя дума $\beta\in X^*$ и някоя буква $a\in X$. Тогава \[|\alpha| = |\beta| + 1.\]
\end{enumerate}

Дефинираме операцията {\em конкатенация}\index{конкатенация} $\cdot$ на две думи $\alpha$ и $\beta$ от $X^*$ с индукция по дължината $\beta$:
\begin{enumerate}[(i)]
  \item
    $|\beta| = 0$, т.е. $\beta = \varepsilon$.
    Тогава \[\alpha\cdot\beta = \alpha.\]
  \item
    $|\beta| = n+1$, т.е. $\beta = \gamma b$, за някоя дума $\gamma$, $|\gamma| = n$, и някоя буква $b\in X$.
    Тогава \[\alpha\cdot\beta = (\alpha\cdot\gamma)\cdot b.\]
\end{enumerate}

Сега можем да даден алтернативни дефиниции на понятията начало и край на дума.
Казваме, че думата $\alpha$ е начало на думата $\beta$, ако съществува дума $\gamma$ такава, че
$\beta = \alpha\cdot\gamma$.
Аналогично дефинираме $\alpha$ да бъде край на думата $\beta$.


\begin{problem}
  Определете релациите:
  \begin{enumerate}[a)]
  \item
    $\alpha R \beta \iff \alpha \mbox{ е начало на }\beta$ 
  \item
    $\alpha R \beta \iff \alpha \mbox{ е край на }\beta$
  \item
    $\alpha R \beta \iff \alpha \mbox{ е начало на }\beta \vee (\exists\alpha_1\in X^{*})(\exists a,b\in X)(\alpha_1 a \mbox{ е начало на }\alpha\ \&\ \alpha_1 b \mbox{ е начало на } \beta)$
  \item
    $R\subseteq (\{0,1\}^{n})^{2}, \langle{a_1,\dots,a_n}\rangle R \langle{b_1,\dots,b_n}\rangle \iff a_1\leq b_1\ \&\dots\ \&\ a_n\leq b_n$
    Забележете, че това не е лексикографската наредба.
  \item
    $R\subseteq (\{0,1\}^{n})^{2}, \langle{a_1,\dots,a_n}\rangle R \langle{b_1,\dots,b_n}\rangle \iff (\exists i : 1\leq i\leq n)((\forall j < i)(a_j = b_j)\ \&\ a_i \leq b_i)$
\end{enumerate}
\end{problem}
\begin{proof}
  \begin{enumerate}[1)]
  \item[в)]
    \begin{enumerate}[(i)]
    \item
      Ще проверим дали $R$ е рефлексивна, т.е. дали $(\forall\gamma\in X^{*})[\gamma R\gamma]$.
      Очевидно $R$ е рефлексивна, защото всяка дума е начало на самата себе си ($\alpha = \alpha\cdot\varepsilon$).
    \item
      Ще проверим дали $R$ е транзитивна, т.е. дали $(\forall\alpha,\beta,\gamma\in X^*)[\alpha R\beta\wedge\beta R\gamma\rightarrow\alpha R\gamma]$.
      Тук трябва да разгледаме четири случая:
      \begin{enumerate}[a)]
      \item
        $\beta = \alpha\cdot\delta$ и $\gamma = \beta\cdot\rho$, за някои $\delta$ и $\rho$.
        Тогава $\gamma = \alpha\cdot(\delta\cdot\rho)$, следователно $\alpha$ е начало на $\gamma$.
      \item
        $\beta = \alpha\cdot\delta$ и $\beta = \rho\cdot a\cdot \beta'$ и $\gamma = \rho\cdot b\cdot\gamma'$.
        Ако $\alpha$ е начало на $\rho$, то $\alpha R \gamma$.
        Ако $\rho\cdot a$ е начало на $\alpha$, тогава $\alpha = \rho\cdot c\cdot\alpha'$ и $\alpha R \gamma$.
      \item
        $\alpha = \rho\cdot a\cdot \alpha'$ и $\beta = \rho\cdot b\cdot\beta'$ и $\gamma = \beta\cdot\delta$ се разглежда аналогично.
      \item
        $\alpha = \rho\cdot a\cdot \alpha'$ и $\beta = \rho\cdot b\cdot\beta'$ и $\beta = \delta\cdot c\cdot \beta'$ и $\gamma = \delta\cdot d\cdot\gamma'$.
        Ако $\rho\cdot b$ е начало на $\delta$, то $\delta = \rho\cdot b\cdot\nu$ и $\gamma = \rho\cdot b\cdot \gamma''$. Тогава $\alpha R \gamma$.
        Ако $\delta\cdot c$ е начало на $\rho$, то $\rho = \delta\cdot c\cdot\nu$ и $\alpha = \delta\cdot c\cdot\alpha''$. Получаваме, че 
        $\alpha R \gamma$.
      \end{enumerate}
    \item
      Ще проверим дали $R$ е симетрична, т.е. дали $(\forall\alpha,\beta\in X^*)[\alpha R\beta \rightarrow \beta R\alpha]$.
      Ако $\alpha = \varepsilon$, то $\alpha R\beta$, но нямаме $\beta R\alpha$.
      Следователно, релацията не е симетрична.
    \item
      Лесно се вижда също, че $R$ не е антисиметрична.
    \end{enumerate}
    
  \end{enumerate}
  
\end{proof}


\begin{problem}
  За всяко естествено число $n$, дефинираме релацията $R_n \subseteq (X^*)^2$ като
  \[\alpha R_n \beta \iff [|\alpha| > n\ \wedge |\beta| > n\wedge (\forall i < n)[a_i = b_i]].\]
  Докажете, че $R_n$ е релация на еквивалентност и намерете броя на класовете на еквивалентност.
\end{problem}

\begin{problem}
  За всяко естествено число $n$, дефинираме релацията $R_n \subseteq (X^*)^2$ като
  \[\alpha R_n \beta \iff \alpha = \beta \vee [|\alpha| > n\ \wedge |\beta| > n\wedge (\forall i < n)[a_i = b_i]].\]
  Докажете, че $R_n$ е релация на еквивалентност и намерете броя на класовете на еквивалентност.
\end{problem}
\begin{proof}
  Броят на класовете на еквивалентност е $\frac{|X|^n - 1}{|X| - 1}$.
\end{proof}


\begin{problem}
  За всяко естествено число $n$, дефинираме релацията $R_n \subseteq (X^*)^2$ като
  \[\alpha R_n \beta \iff |\alpha| = |\beta| > n\wedge (\forall i > n)[a_i = b_i].\]
  Докажете, че $R_n$ е релация на еквивалентност.
\end{problem}

\begin{problem}
  Нека $R$ е релация върху паметта на един компютър и е дефинирана като
  \[xRy \iff x,y\mbox{ са указатели в паметта и }*x = *y.\]
  Докажете, че $R$ е релация на еквивалентност.
\end{problem}



\begin{dfn}
  Нека $R$ е релация.
  Множеството $[x]_R$ се дефинира като
  \[[x]_R = \{t\mid xRt\}.\]
  Ако $R$ е релация на еквивалентност и $x\in Field(R)$, то $[x]_R$ клас на еквивалентност за $x$ (по модул $R$).
\end{dfn}

\begin{example}
  Нека $\sim$ е бинарна релация върху $\N$, дефинирана като
  \[x\sim y \iff x\equiv y\ (mod\ 4).\]
  $\sim$ е релация на еквивалентност и има четири класове на еквивалентност
  \[[0]_\sim, [1]_\sim, [2]_\sim, [3]_\sim.\]
\end{example}

\begin{problem}
  Да дефинираме релацията $R$ върху реалните числа, като:
  \[xRy \iff (x-y)\in\Z.\]
  Намерете $[1]_R$ и $[\frac{1}{2}]_R$.
\end{problem}


\begin{lemma}
  Нека $R$ е релация на еквивалентност върху $A$ и $x,y\in A$. Тогава \[[x]_R = [y]_R \iff xRy.\]
\end{lemma}


\begin{problem}
  Покажете, че за всяка релация $R$ и $x$, $[x]_R = R[\{x\}]$.
\end{problem}

%% Rosen Textbook
\begin{problem}
  Кои от следните релации върху множеството от функциите от $\Z$ в $\Z$.
  \begin{enumerate}[a)]
  \item
    $\{(f,g)\mid f(1) = g(1)\}$.
  \item
    $\{(f,g)\mid f(0) = g(0)\wedge f(1) = g(1)\}$.
  \item
    $\{(f,g)\mid (\forall x\in\Z)[f(x)-g(x) = 1]\}$.
  \item
    $\{(f,g)\mid (\exists c\in\Z)(\forall x\in\Z)[f(x)-g(x) = c]\}$.
  \item
    $\{(f,g)\mid f(0) = g(1)\wedge f(1) = g(0)\}$.
  \end{enumerate}
\end{problem}


\begin{problem}
  Нека $A$ е непразно множество и $f$ е функция с $Domain(f) = A$.
  Дефинираме $R$ върху $A$ като:
  \[\{(x,y)\mid x,y\in A\wedge f(x) = f(y)\}.\]
  Докажете, че
  \begin{enumerate}[(i)]
  \item
    $R$ е релация на еквивалентност.
  \item
    Определете класовете на еквивалентност на $R$.
\end{enumerate}

\begin{problem}
  Нека $R_1$ и $R_2$ са симетрични релации.
  Проверете дали $\overline{R_1}$, $R_1\cap R_2$ и $R_1 \cup R_2$ са симетрични.
\end{problem}

\begin{problem}
  Докажете, че подмножество на всяка антисиметрична релация е също антисиметрична.
\end{problem}

\end{problem}

\chapter{Релации}
\index{релация}

За да дадем определение на понятието релация, трябва първо 
да въведем понятието декартово произведение на множества,
което пък от своя страна се основава на понятието наредена двойка.

\subsection*{Наредена двойка}

За два елемента $a$ и $b$ въвеждаме опрецията {\bf наредена двойка} $\pair{a,b}$.
Наредената двойка $\pair{a,b}$ има следното характеристичното свойство:
\[a_1 = a_2\ \wedge\ b_1 = b_2\ \iff\ \pair{a_1,b_1} = \pair{a_2,b_2}.\]
Понятието наредена двойка може да се дефинира по много начини, стига да изпълнява харектеристичното свойство.
Ето примери как това може да стане:
\begin{enumerate}[1)]
\item
  \marginpar{Norbert Wiener (1914)}
  Първото теоретико-множествено определение на понятието наредена двойка е
  дадено от Норберт Винер:
  \[\pair{a,b} = \{\{\{a\},\emptyset\},\{\{b\}\}\}.\]
\item
  \marginpar{Kazimierz Kuratowski (1921)}
  Определението на Куратовски се приема за ,,стандартно`` в наши дни:
  \[\pair{a,b} = \{\{a\},\{a,b\}\}.\]
\end{enumerate}

\begin{problem}
  Докажете, че горните дефиниции наистина изпълняват харектеристичното свойство за наредени двойки.
\end{problem}

\begin{remark}
  %\marginpar{Пример за рекурсивна дефиниция}
  Сега можем да въведем понятието наредена $n$-орка $\pair{a_1,\dots,a_n}$ за всяко естествено число $n \geq 1$:
  \begin{align*}
    & \pair{a_1} = a_1,\\
    & \pair{a_1,a_2,\dots,a_n} = \pair{a_1,\pair{a_2,\dots,a_n}}
  \end{align*}
\end{remark}
 
\section{Декартово произведение}
\marginpar{На англ. cartesian product}
\index{декартово произведение}
\marginpar{Считаме, че $(A\times B)\times C = A\times (B\times C) = A\times B \times C$}
За две множества $A$ и $B$, определяме тяхното декартово произведение като
\[A\times B = \{\pair{a,b}\mid a\in A\ \&\ b\in B\}.\]
За краен брой множества $A_1,A_2,\dots,A_n$, определяме
\[A_1\times A_2\times\cdots\times A_n = \{\pair{a_1,a_2,\dots,a_n}\mid a_1 \in A_1\ \&\ a_2\in A_2\ \&\ \dots\ \&\ a_n \in A_n\}.\]

Подмножествата $R$ от вида $R \subseteq A\times A\times\cdots\times A$ се наричат релации.

\begin{problem}
  Преверете, че:
  \begin{enumerate}[1)]
  \item 
    $A\times(B\cup C) = (A\times B) \cup (A\times C)$.
  \item
    $(A\cup B)\times C = (A\times C)\cup (B\times C)$.
  \item 
    $A\times(B\cap C) = (A\times B) \cap (A\times C)$.
  \item
    $(A \cap B)\times C = (A \times C)\cap(B\times C)$.
  \item 
    $A\times(B\setminus C) = (A\times B) \setminus (A\times C)$.
  \item
    $(A\setminus B)\times C = (A\times C)\setminus (B\times C)$.
  \end{enumerate}
\end{problem}

\section{Основни видове бинарни релации}
\marginpar{Бинарни релации}
Релациите от вида $R\ \subseteq\ A\times A$ са важен клас, който ще срещаме често.
Да разгледаме няколко основни вида релации от този клас:
\begin{enumerate}[I)]
\item
  {\bf рефликсивна}, ако
  \[(\forall x\in A)[\pair{x,x}\in R].\]
  Например, релацията $\leq\ \subseteq\ \Nat\times\Nat$ е рефлексивна, защото
  \[(\forall x\in \Nat)[x \leq x].\]
\item
  {\bf антирефлексивна}, ако
  \[(\forall x\in A)[\pair{x,x}\not\in R].\]
  Например, релацията $<\ \subseteq\ \Nat\times\Nat$ е антирефлексивна, защото
  \[(\forall x\in \Nat)[x \not< x].\]
\item
  {\bf транзитивна}, ако
  \[(\forall x,y,z\in A)[\pair{x,y}\in R\ \&\ \pair{y,z}\in R \rightarrow \pair{x,z}\in R].\]
  Например, релацията $\leq\ \subseteq\ \Nat\times\Nat$ е транзитивна, защото
  \[(\forall x,y,z\in A)[x \leq y\ \&\ y \leq z\ \rightarrow\ x\leq z].\]
\item
  {\bf симетрична}, ако
  \[(\forall x,y\in A)[\pair{x,y}\in R \rightarrow \pair{y,x}\in R].\]
  Например, релацията $=\ \subseteq\ \Nat\times\Nat$ е рефлексивна, защото
  \[(\forall x,y\in \Nat)[x = y\ \rightarrow\ y = x].\]
\item
  {\bf антисиметрична}, ако
  \[(\forall x,y\in A)[\pair{x,y}\in R\ \&\ \pair{y,x}\in R \rightarrow x = y].\]
  Например, релацията $\leq\ \subseteq\ \Nat\times\Nat$ е антисиметрична, защото
  \[(\forall x,y,z\in A)[x \leq y\ \&\ y \leq x\ \rightarrow\ x = y].\]
\item
  {\bf асиметрична}, ако
  \[(\forall x,y)[\pair{x,y}\in R \rightarrow \pair{y,x}\not\in R].\]
  Например, релацията $\leq\ \subseteq\ \Nat\times\Nat$ е асиметрична, защото
  \[(\forall x,y\in \Nat)[x < y\ \rightarrow\ y \not< x].\]
\end{enumerate}

\begin{remark}
  Добре е да запомните как се наричат тези основни видове релации,
  защото ще ги използваме често.
\end{remark}

\begin{example}
  Да обобщим примерите от по-горе.
  \begin{enumerate}[a)]
  \item
    Релацията $\leq\ \subseteq\ \Nat\times\Nat$ е рефлексивна, транзитивна и антисиметрична.
  \item
    Релацията $<\ \subseteq\ \Nat\times\Nat$ е антирефлексивна, транзитивна и асиметрична.
  \item
    Релацията $=\ \subseteq\ \Nat\times\Nat$ е рефлексивна, транзитивна и симетрична.
  \end{enumerate}
\end{example}

\begin{problem}
  Проверете кои от горе-изброените свойства притежава релацията $R$:
  \begin{enumerate}[a)]
  \item
    \marginpar{Озн. $\Nat^2 = \Nat\times\Nat$}
    $R\subseteq \Nat^2$ и е определна като 
    \marginpar{$a\vert b \iff (\exists k\in\Nat)(b = k\cdot a)$}
    \[(a,b) \in R \iff a | b.\]
  \item
    $R \subseteq \Z\times \Z$ е определена като
    \[\pair{x,y}\in R \iff \mbox{НОД}(x,y) = 1\]% \iff \neg(\exists z > 1)[\ z\vert x\ \wedge\ z\vert y\ ]\]
  \item
    \marginpar{Озн. $\R$ - реалните числа}
    $R\subseteq \Int^2$ и е определена като
    \[(a,b) \in R \iff a\cdot b > 0.\]
  \item
    $R\subseteq \Int^2$ и е определена като 
    \[(a,b) \in R \iff a+b = 0.\]
  \item
    $R\subseteq \Int^2$ и е определена като
    \[(a,b) \in R \iff a+b = 5.\]
  \item
    $R\subseteq \Int^2$ и е определена като 
    \[(a,b) \in R \iff a+b\mbox{ е четно}.\]
  \item
    $R\subseteq (\Int^2)^2$ и е определена като
    \[(\langle{a,b}\rangle, \langle{c,d}\rangle) \in R \iff a+d = b+c.\]
  \item
    $R\subseteq (\Int^2)^2$ и е определена като
    \[(\langle{a,b}\rangle,\langle{c,d}\rangle) \in R \iff a\cdot d = b\cdot c.\]
  \item
    \marginpar{Озн. $\Z$ - целите числа}
    $R_{m}\subseteq \Z^2, m\in \Z, m>0$ и е определена като
    \[(a,b) \in R_m \iff m\mid (a - b).\]
  \item
    $R\subseteq \R^2$ и е определена като 
    \[(x,y) \in R \iff (x-y)\mbox{ е рационално число}.\]
  \item
    \marginpar{Озн. $\Q$ - рационалните числа}
    $R \subseteq \Q^2$ и е определена като
    \[(p,r) \in R\ \iff\ p-r \mbox{ е цяло число}.\]
  \item
    $R \subseteq \Nat^2$ и е определена като
    \[(a,b) \in R \iff a = b \vee a+1 = b.\]
  \item
    $R \subseteq \Nat^2$ и е определена като
    \[\pair{a,b}\in R \iff (\exists k\in\Nat)[a+k = b].\]
  \item
    Нека $\leq_1\ \subseteq\ A^2$ и $\leq_2\ \subseteq\ B^2$ са частични наредби.
    $R \subseteq A^2\times B^2$ е определена като
    \[(\pair{a,b}, \pair{c,d}) \in R \iff a\leq_{1}c\ \wedge\ b\leq_{2}d .\]
  \item
    Нека $\leq_1\ \subseteq\ A^2$ и $\leq_2\ \subseteq\ B^2$ са частични наредби.
    $R \subseteq A^2\times B^2$ е определена като
    \[(\pair{a,b}, \pair{c,d}) \in R \iff a\leq_{1}c\ \vee\ b\leq_{2}d .\]
  \item
    $f:X\rightarrow Y$ е функция, $R\ \subseteq\ \Ps(X)\times\Ps(X)$ и 
    \[(A,B)\in R \iff f(A) = f(B).\]
  \end{enumerate}
\end{problem}

\begin{problem}
  Нека $R$ и $S$ са релации на еквивалентност върху множеството $A$.
  Какви свойства притежават следните релации:
  \begin{enumerate}[a)]
  \item
    $R \cap S$;
  \item
    $R \cup S$;
  \item
    $R \setminus S$;
  % \item
  %   $R \circ S$.
  \end{enumerate}
\end{problem}

%% Rosen Textbook
\begin{problem}
  Кои от следните бинарни релации върху множеството на функциите от $\Z$ в $\Z$
  са релации на еквивалентност? Опишете техните класове на еквивалентност.
  \begin{enumerate}[a)]
  \item
    $R = \{(f,g)\mid f(1) = g(1)\}$.
  \item
    $R = \{(f,g)\mid f(0) = g(0)\wedge f(1) = g(1)\}$.
  \item
    $R =\{(f,g)\mid (\forall x\in\Z)[f(x)-g(x) = 1]\}$.
  \item
    $R = \{(f,g)\mid (\exists c\in\Z)(\forall x\in\Z)[f(x)-g(x) = c]\}$.
  \item
    $R = \{(f,g)\mid f(0) = g(1)\wedge f(1) = g(0)\}$.
  \end{enumerate}
\end{problem}

\section{Релации над думи}

\begin{itemize}
\item
  Азбука е крайно множество $\Sigma = \{a_1,\dots,a_n\}$ като елементите $a_i$ на $\Sigma$ наричаме {\bf букви}.
\item
  Нека да фиксираме един елемент $\varepsilon \not\in \Sigma$.
  Сега ще определим {\bf думите} над азбуката $\Sigma$. Това са:
  \begin{itemize}
  \item
    \marginpar{наричаме $\varepsilon$ празната дума}
    $\varepsilon$ е дума над $\Sigma$;
  \item
    \marginpar{т.е. думите са крайни последователности от букви}
    Нека $\alpha$ е дума над $\Sigma$ и $a \in \Sigma$ е буква.
    Тогава $\alpha a$ е дума над $\Sigma$;
  \item
    няма други думи над $\Sigma$.
  \end{itemize}
\item
  Означаваме с $\Sigma^n$ множеството от всички думи с дължина $n$ над азбуката $\Sigma$, $\Sigma^0 = \{\varepsilon\}$,
  защото празната дума е единствената дума с дължина $0$.
\item
  Със $\Sigma^\star$ означаваме множеството от всички думи над азбуката $\Sigma$, т.е.
  \marginpar{$0 \in \Nat$}
  \[\Sigma^\star = \bigcup_{n\in\Nat} \Sigma^{n}.\]
\end{itemize}

Сега ще определим функцията {\bf дължина}\index{дума!дължина} на дума.
\marginpar{Функцията $\abs{\cdot}:\Sigma^\star\to\Nat$ е винаги сюрективна. Кога е биективна?}
Дължината $|\alpha|$ на думата $\alpha \in \Sigma^\star$ се определя с индукция по построението на $\alpha$.
\begin{enumerate}[(i)]
  \item
    Нека $\alpha = \varepsilon$. Тогава $|\alpha| = 0$.
  \item
    Нека $\alpha = \beta a$, за някоя дума $\beta\in \Sigma^\star$ и някоя буква $a\in X$.
    Тогава \[|\alpha| = |\beta| + 1.\]
\end{enumerate}

Определяме функцията {\bf конкатенация}\index{дума!конкатенация} $\cdot$, т.е.
слепване на две думи.
За всеки две думи $\alpha$ и $\beta$ от $\Sigma^\star$ определяме тяхната конкатенация с индукция по дължината $\beta$:
\marginpar{$\cdot:\Sigma^\star\times \Sigma^\star \to \Sigma^\star$ е винаги сюрективна. Може ли да бъде биективна?}
\begin{enumerate}[(i)]
  \item
    $|\beta| = 0$, т.е. $\beta = \varepsilon$.
    Тогава $\alpha\cdot\beta = \alpha$.
  \item
    $|\beta| = n+1$, т.е. $\beta = \gamma b$, за някоя дума $\gamma$, $|\gamma| = n$, и някоя буква $b\in\Sigma$.
    Тогава $\alpha\cdot\beta = (\alpha\cdot\gamma)\cdot b$.
\end{enumerate}

Казваме, че думата $\alpha$ е {\bf начало} на думата $\beta$, ако съществува дума $\gamma \in \Sigma^\star$ такава, че
$\beta = \alpha\cdot\gamma$. Обикновено означаваме $\alpha \preceq \beta$.
Аналогично дефинираме $\alpha$ да бъде {\bf край} на думата $\beta$, ако съществува $\gamma \in \Sigma^\star$ такава, че
$\beta = \gamma \cdot \alpha$.

\begin{problem}
  Нека $\Sigma = \{0,1\}$.
  Какви свойства имат следните бинарни релации над $\Sigma^\star$ ?
  Ако $R$ е релация на  еквивалентност, то опишете нейните класове на еквивалентност 
  и намерете техния брой.
  \begin{enumerate}[a)]
  \item
    $(\alpha,\beta) \in R \iff \alpha \preceq \beta$;
  \item
    $(\alpha,\beta) \in R  \iff (\exists\gamma\in \Sigma^\star)[\exists a,b\in \Sigma)(\gamma a \preceq \alpha\ \&\ \gamma b \preceq \beta\ \&\ a \neq b]$;
  \item
    $(\alpha,\beta) \in R  \iff (\exists\gamma\in \Sigma^\star)[\exists a,b\in \Sigma)(\gamma a \preceq \alpha\ \&\ \gamma b \preceq \beta\ \&\ a < b]$;
  \item
    $(\alpha,\beta) \in R  \iff \alpha \preceq \beta \vee (\exists\gamma\in \Sigma^\star)[\exists a,b\in \Sigma)(\gamma a \preceq \alpha\ \&\ \gamma b \preceq \beta\ \&\ a < b]$;
  \item
    $(\alpha,\beta) \in R \iff |\alpha| = |\beta|\ \& (\forall i \leq |\alpha|)[a_i \leq b_i]$;
  \item
    $(\alpha,\beta) \in R \iff (\forall i \leq \min\{|\alpha|,|\beta|\})[a_i \leq b_i]$;
  \item
    $(\alpha,\beta) \in R \iff (\exists \gamma_1,\gamma_2 \in \Sigma^\star)[\beta = \gamma_1 \alpha \gamma_2]$;
  \item
    за фиксирано число $n$,
    \[(\alpha,\beta) \in R \iff (\exists\gamma\in\Sigma^n)[\gamma\preceq\alpha\ \&\ \gamma\preceq\beta].\]
  \item
    за фиксирано число $n$,
    \[(\alpha,\beta) \in R \iff \alpha = \beta \vee (\alpha \neq \beta\ \&\ (\exists\gamma\in\Sigma^n)[\gamma\preceq\alpha\ \&\ \gamma\preceq\beta]).\]
  \item
    за фиксирано число $n$,
    \[(\alpha, \beta)\in R \iff |\alpha| = |\beta| > n\ \&\ (\forall i > n)[a_i = b_i].\]
  \end{enumerate}
\end{problem}
\newpage
\section{Операции върху релации}
\begin{enumerate}[I)]
\item
  {\bf Композиция} на две релации $S \subseteq B\times C$ и $T \subseteq A\times B$ е релацията $S\circ T \subseteq A\times C$,
  определена като:
  \[S\circ T = \{\langle{a,c}\rangle \mid (\exists b \in B)[\pair{a,b}\in T\ \&\ \pair{b,c} \in S]\}.\]
\item
  {\bf Обръщане} на релацията $R \subseteq A\times B$ е релацията $R^{-1}\subseteq B\times A$, 
  определена като:
  \[R^{-1} = \{\pair{x,y} \mid \pair{y,x} \in R\}.\]
% \item
  
%   $\overline{R} = \{\langle{x,y}\rangle \in A\times B \mid\langle{x,y}\rangle\not\in R\}$;
\end{enumerate}

\begin{problem}
  Дайте пример за релации $R$ и $S$, за които
  \[R\circ S \neq S\circ R.\]
\end{problem}

  
\begin{problem}
  Докажете, че:
  \begin{enumerate}[a)]
  \item
    $R$ е симетрична тогава и само тогава, когато $R^{-1}\subseteq R$;
  \item
    $R$ е транзитивна тогава и само тогава, когато $R\circ R\subseteq R$;
  \item
    $R$ е транзитивна и симетрична тогава и само тогава, когато $R = R^{-1}\circ R$.
\end{enumerate}
\end{problem}
\begin{proof}
  \begin{enumerate}[a)]
  \item
    Задачата се разделя на две подзадачи.
    \begin{enumerate}[(i)]
    \item
      Нека $R$ да бъде симетрична. Ще докажем, че $R^{-1}\subseteq R$, т.е.
      \[(\forall x\forall y)[(x,y)\in R^{-1} \rightarrow (x,y)\in R].\]
      Нека $(x,y)\in R^{-1}$. Тогава по определение имаме, че $(y,x)\in R$ и следователно $(x,y)\in R$,
      защото $R$ е симетрична.
    \item
      Нека $R^{-1}\subseteq R$. Щe докажем, че $R$ е симетрична, т.е.
      \[(\forall x\forall y)[(x,y)\in R \rightarrow (y,x)\in R].\]
      Нека $(x,y)\in R$, следователно по определение $(y,x)\in R^{-1}$.
      Тогава от $R^{-1}\subseteq R$ следва, че $(y,x)\in R$.
    \end{enumerate}
  \item
  \item
    Нека $R^{-1}\circ R = R$. Ще докажем, че $R$
    е симетрична и транзитивна.
    \begin{enumerate}[(i)]
    \item
      Ще докажем, че $R$ e симетрична.
      За целта е достатъчно да вземем произволна двойка $\pair{x,y} \in R$
      и да покажем, че $\pair{y,x} \in R$.
      Нека 
      % \begin{prooftree}
      %   \AxiomC{$\pair{x,y} \in R$}
      %   \RightLabel{\scriptsize($R^{-1}\circ R = R$)}
      %   \UnaryInfC{$\pair{x,y} \in R^{-1}\circ R$}
      %   \UnaryInfC{$(\exists z)[\pair{x,z} \in R\ \wedge\ \pair{z,y} \in R^{-1}]$}
      %   \UnaryInfC{$(\exists z)[\pair{z,x} \in R^{-1}\ \wedge\ \pair{y,z} \in R]$}
      %   \UnaryInfC{$(\exists z)[\pair{y,z} \in R\ \wedge\ \pair{z,x} \in R^{-1}]$}
      %   \UnaryInfC{$\pair{y,x} \in R^{-1}\circ R$}
      %   \RightLabel{\scriptsize($R^{-1}\circ R = R$)}
      %   \UnaryInfC{$\pair{y,x} \in R$}
      % \end{prooftree}
      \begin{align*}
        \pair{x,y} \in R & \iff \pair{x,y} \in R^{-1}\circ R & (\text{имаме, че }R^{-1}\circ R = R)\\
        & \iff (\exists z)[\pair{x,z} \in R\ \wedge\ \pair{z,y} \in R^{-1}]\\
        & \iff (\exists z)[\pair{z,x} \in R^{-1}\ \wedge\ \pair{y,z} \in R]\\
        & \iff (\exists z)[\pair{y,z} \in R\ \wedge\ \pair{z,x} \in R^{-1}]\\
        & \iff \pair{y,x} \in R^{-1}\circ R\\
        & \iff \pair{y,x} \in R.
      \end{align*}
      
      Следователно,
      \[(\forall x,y \in A)[\pair{x,y} \in R\ \rightarrow\ \pair{y,x} \in R].\]
    \item
      Ще докажем, че $R$ e транзитивна.
      За целта е достатъчно да вземем произволни двойки $\pair{x,y} \in R$
      и $\pair{y,z} \in R$, то $\pair{x,z} \in R$.
      
      \begin{prooftree}
      \AxiomC{$\pair{x,y} \in R$}
      \AxiomC{$\pair{y,z} \in R$}
      \RightLabel{\scriptsize($R$ е симетрична)}
      \UnaryInfC{$\pair{z,y} \in R$}
      \UnaryInfC{$\pair{y,z} \in R^{-1}$}
      \BinaryInfC{$\pair{x,z} \in R^{-1}\circ R$}
      \RightLabel{\scriptsize($R^{-1}\circ R = R$)}
      \UnaryInfC{$\pair{x,z} \in R$}
      \end{prooftree}
      Следователно,
      \[(\forall x,y,z \in A)[(\pair{x,y} \in R\ \wedge\ \pair{y,z} \in R) \rightarrow\ \pair{x,z} \in R].\]
    \end{enumerate}
    Нека сега $R$ е транзитивна и симетрична.
    Ще докажем, че $R^{-1}\circ R = R$.
    
    \begin{enumerate}[(i)]
    \item 
      Първо да отбележим, че
      \begin{prooftree}
        \AxiomC{$\pair{x,y} \in R$}
        \AxiomC{$\pair{x,y} \in R$}
        \RightLabel{\scriptsize($R$ е симетрична) }
        \UnaryInfC{$\pair{y,x} \in R$}
        \RightLabel{\scriptsize($R$ е транзитивна) }
        \BinaryInfC{$\pair{x,x} \in R\ \wedge \pair{y,y}\in R$}
      \end{prooftree}
      Следователно,
      \[(\forall x \in Dom(R))[\pair{x,x} \in R].\]
    \item
      Ще докажем, че $R^{-1}\circ R \subseteq R$.
      
      \begin{prooftree}
        \AxiomC{$\pair{x,z}\in R$}
        \AxiomC{$\pair{x,y} \in R^{-1}\circ R$}
        \RightLabel{\scriptsize(Съществува $z$)}
        \UnaryInfC{$\pair{z,y} \in R^{-1}$}
        \UnaryInfC{$\pair{y,z} \in R$}
        \RightLabel{\scriptsize($R$ е симетрична)}
        \UnaryInfC{$\pair{z,y} \in R$}
        \RightLabel{\scriptsize($R$ е транзитивна)}
        \BinaryInfC{$\pair{x,y} \in R$}
      \end{prooftree}
    \item
      Ще докажем, че $R \subseteq R^{-1}\circ R$.
      \begin{prooftree}
        \AxiomC{$\pair{x,y}\in R$}
        \AxiomC{$\pair{x,y}\in R$}
        \RightLabel{\scriptsize(От (i))}
        \UnaryInfC{$\pair{y,y} \in R$}
        \UnaryInfC{$\pair{y,y} \in R^{-1}$}
        \RightLabel{\scriptsize(по деф.)}
        \BinaryInfC{$\pair{x,y} \in R^{-1}\circ R$}
      \end{prooftree}
    \end{enumerate}
  \end{enumerate}
  
\end{proof}

\begin{problem}
  Нека $\{\pair{a,b}\}\subseteq R$, за някои $a\neq b$.
  Докажете, че ако $R$ е симетрична, то $R$ не е антисиметрична.
\end{problem}
% \begin{proof}
%   Нека $R$ е симетрична.
% \end{proof}

\begin{problem}
  Нека $R$ да бъде релация на еквивалентност върху $B$ и $f:A\to B$.
  Дефинираме множеството 
  \[Q = \{((x,y)\in A\times A\mid (f(x),f(y))\in R\}.\]
  Докажете, че $Q$ е релация на еквивалентност.
\end{problem}

\begin{problem}
  Нека $R$ е релация върху $A$.
  Да определим релациите:
  \begin{itemize}
  \item 
    $S = \{\pair{x,y}\mid \pair{x,y} \in R\ \wedge\ \pair{y,x} \in R\}$;
  \item
    $T = \{\pair{x,y} \mid \pair{x,y}\in R\ \wedge\ \pair{y,x}\not\in R\}$.
  \end{itemize}
  Докажете, че:
  \begin{enumerate}[a)]
  \item 
    $S$ е симетрична и $T$ е антисиметрична.
  \item
    $\pair{x,y} \in R\ \iff\ (\pair{x,y}\in S \vee \pair{x,y} \in T)$;
  \item
    ако $R$ е транзитивна, то $S$ и $T$ са също транзитивни, но обратната посока не е вярна.
  \end{enumerate}
\end{problem}

Нека $R \subseteq A\times A$ е бинарна релация.
Да определим множеството $[x]_R$ като
\[[x]_R = \{y\in A\mid \pair{x,y} \in R\}.\]
Ако $R$ е релация на еквивалентност и $x\in Field(R)$, то наричаме множеството $[x]_R$ {\bf клас на еквивалентност} за $x$ относно релацията $R$.

\begin{example}
  \marginpar{$x \equiv y \mod 4 \iff (\exists k \in \Z)(x = y + 4\cdot k)$}
  Нека $\sim_4\ \subseteq\ \Nat\times \Nat$ е бинарна релация, дефинирана като
  \[x\sim_4 y \iff x\equiv y \mod 4.\]
  $\sim_4$ е релация на еквивалентност и има четири класа на еквивалентност
  \[[0]_{\sim_4}, [1]_{\sim_4}, [2]_{\sim_4}, [3]_{\sim_4}.\]
\end{example}

\begin{problem}
  Да дефинираме релацията $R \subseteq \R\times \R$ като:
  \[\pair{x,y}\in R \iff (x-y)\in\Z.\]
  Намерете множествата $[1]_R$ и $[\frac{1}{2}]_R$.
\end{problem}

\section{Наредби}
Обикновено се изучават релации притежаващи различни комбинации от горните свойства. 
Сега ще изброим няколко основни вида релации (понякога се наричат наредби).
Релацията $R \subseteq A\times A$ се нарича:
\begin{itemize}
\item
  \marginpar{На англ. partial order}
  {\bf частична наредба}, ако тя е рефлексивна, транзитивна и антисиметрична.
  Например, $\leq$ е частична наредба.
  Също така, релацията $\subseteq$ между множества е частична наредба.
\item 
  \marginpar{На англ. equivalence relation}
  {\bf релация на еквивалентност}, ако тя е рефлексивна, транзитивна и симетрична.
  Например, $=$ е релация на еквивалентност.
\item
  \marginpar{На англ. linear order}
  {\bf линейна наредба}\index{наредба!линейна}, ако $R$ е частична наредба, 
  и за всеки два елемента $x,y$ точно едно от $\pair{x,y} \in R$, $x = y$, $\pair{y,x}\in R$ е изпълнено.
\item
  \marginpar{На англ. well-founded order}
  {\bf фундирана наредба}\index{наредба!фундирана}, 
  ако всяко непразно подмножество $X\subseteq A$ притежава поне един {\em минимален} елемент, т.е.
  \[(\forall X\subseteq A)[X\neq\emptyset \rightarrow (\exists m\in X)\neg(\exists y\in X)[\pair{y,m} \in R]].\]
\item
  \marginpar{На англ. well-ordered relation}
  {\bf добра наредба}\index{наредба!добра}, ако всяко непразно подмножество $X\subseteq A$ има {\em най-малък} елемент , т.е.
  \[(\forall X\subseteq A)[X\neq\emptyset \rightarrow (\exists m\in X)(\forall y\in X)[\pair{m,y}\in R \vee m = y]].\]
\item
  \marginpar{lexicographical order}
  {\bf лексикографска наредба} върху частичната наредба $(X,<)$ ще наричаме 
  наредбата $(X\times X,\prec)$, където
  \[\pair{x,y}\prec\pair{x^\prime,y^\prime}\ \iff\ x<x^\prime\ \vee\ (x = x^\prime\ \wedge\ y < y^\prime).\]
\end{itemize}

\begin{remark}
  Обърнете внимание на разликата между понятията {\em минимален} и {\em най-малък} елемент относно релацията $R$.
  \begin{itemize}
  \item $x_0$ е {\bf минимален} елемент за множеството $X \subseteq A$ относно $R$,
    ако не съществуват елементи $y \in X$, за които $\pair{y,x}\in R$, т.е.
    \[(\forall y \in X)[\pair{y,x} \not\in R].\]
  \item $x_0$ е {\bf най-малък} елемент за множеството $X \subseteq A$ относно $R$,
    ако за всеки друг елемент $y \in X$ е изпълнено $\pair{x,y}\in R$, т.е.
    \[(\forall y \in X)[x\neq y \rightarrow \pair{x,y} \in R].\]
  \end{itemize}
\end{remark}

\begin{problem}
  Да се докаже, че $(\Nat^2,\prec)$ е добре наредено множество.
\end{problem}

\begin{problem}
  Докажете, че следните две условия за частично наредено множество $(X,<)$ са еквивалентни:
  \begin{enumerate}[a)]
  \item
    всяко непразно подмножество на $X$ има минимален елемент;
  \item
    не съществува строго намаляваща редица $x_1>x_2>x_3>\dots$ то елементи на $X$.
  \end{enumerate}
\end{problem}

\begin{problem}
  Да се докаже, че частично нареденото множество $(X,<)$ е добре наредено тогава и само тогава, когато 
  $(X,<)$ е фундирано и $<$ е линейна наредба върху $X$.
\end{problem}

\begin{problem}% от СЕП
  Кои от изброените множества са фундирани? Кои са добре наредени?
  \begin{enumerate}[a)]
  \item
    $(\Nat,<)$;
  \item
    $(\Z,<)$;
  \item
    $(X^*, <)$, където за $\alpha,\beta\in X^*$, $\alpha < \beta \iff \alpha\mbox{ е поддума на }\beta$;
  \item
    $(2^\Nat,\subsetneq)$;
  \item
    $(Fin(\Nat),\subsetneq)$, където $Fin(\Nat)$ е съвкупността от всички крайни подмножества на $\Nat$;
  \item
    $(\Nat^+,|)$, където $m|n \iff m\neq n\ \&\ m\mbox{ дели }n$.
  \end{enumerate}
\end{problem}


\begin{problem}
  Нека $A$ е непразно множество и $f$ е функция с $Dom(f) = A$.
  Дефинираме релацията $R\subseteq A\times A$ като:
  \[R = \{(x,y)\mid x,y\in A\wedge f(x) = f(y)\}.\]
  Докажете, че
  \begin{enumerate}[a)]
  \item
    $R$ е релация на еквивалентност.
  \item
    Определете класовете на еквивалентност на $R$.
\end{enumerate}
\end{problem}

\begin{problem}
  Докажете, че подмножество на всяка антисиметрична релация е също антисиметрична.
\end{problem}





%%% Local Variables: 
%%% mode: latex
%%% TeX-master: "discrete-math"
%%% End: 

\section*{Функции}
\index{функция}

Релацията $R \subseteq A\times B$ се нарича {\bf функция}\index{функция} от $A$ в $B$, ако
\begin{enumerate}[i)]
  \item
    $Dom(R) = A$, т.е.
    \[(\forall a\in A)(\exists b\in B)[(a,b)\in R].\]
  \item
    За всеки елемент $a\in A$ съотвества {\em точно един} елемент $b \in B$, т.е.
    \[(\forall x\in A)(\forall y_1,y_2 \in B)(\langle{x,y_1}\rangle\in R\ \wedge\ \langle{x,y_2}\rangle\in R \rightarrow y_1 = y_2).\]
\end{enumerate}
Обикновено означаваме функциите като $f:A\to B$ и
вместо $(a,b)\in f$ пишем $f(a) = b$.
Казваме, че функцията $f$ e
\begin{itemize}
\item
  \marginpar{също казваме, че $f$ е обратима}
  {\bf инекция}\index{функция!инекция}, ако 
  \[(\forall x_1,x_2\in A)[x_1\neq x_2 \rightarrow f(x_1)\neq f(x_2)],\]
  или еквивалентно,
  \[(\forall x_1,x_2\in A)[f(x_1) = f(x_2) \rightarrow x_1 = x_2].\]
\item
  \marginpar{също казваме, че $f$ е върху $B$}
  {\bf сюрекция}\index{функция!сюрекция}, ако 
  \[(\forall y\in B)(\exists x\in A)[f(x) = y].\]
\item
  {\bf биекция}\index{функция!биекция}, ако е инекция и сюрекция.
\end{itemize}

\begin{problem}
  Дайте примери за функция $f:\mathbb{N}\rightarrow\Z$, която е:
  \begin{enumerate}
  \item
    нито инективна, нито сюрективна;
  \item
    инективна, но не е сюрективна;
  \item
    сюрективна, но не е инективна;
  \item
    сюрективна и инективна.
  \end{enumerate}
\end{problem}

\begin{problem}
  Докажете:
  \begin{enumerate}[a)]
  \item
    Ако $f,g$ са функции, то $f\cap g$ е функция;
  \item
    Нека $f,g$ са функции и $(\forall x)[x\in Dom(f)\cap Dom(g)\rightarrow f(x) = g(x)]$.
    Докажете, че $f\cup g$ е функция.
  \end{enumerate}
\end{problem}

\begin{prb}
  За всяка от следните  функции $f$ определете дали $f$ е
  инекция, сюрекция или биекция.
  \begin{enumerate}[a)]
  \item
    $f: \mathbb{R}\rightarrow \mathbb{R}$, $f(x) = 2x+3$.
  \item
    $f: \mathbb{R}\rightarrow \mathbb{R}$, $f(x) = x^2 - 4x +2$.
  \item 
    $f: \mathbb{R}\rightarrow \mathbb{R}$, $f(x) = x^3+7$.    
  \item
    $f: \mathbb{N}\rightarrow \mathbb{N}$, 
    \begin{align*}
      f(x) = 
      \begin{cases}
        x+1, & \mbox{ ако }x\mbox{ е четно}\\
        x-1, & \mbox{ ако }x\mbox{ е нечетно}\\
      \end{cases}
    \end{align*}
  \item
    \marginpar{$rem(x,3)$ - остатък при деление на $3$}
    $f: \mathbb{N}\rightarrow \mathbb{N}$, $f(x) = rem(x,3)$.
  \item 
    \marginpar{НОД - най-голям общ делител}
    $f: \mathbb{N} \times \mathbb{N}\rightarrow \mathbb{N}$,
    $f(x, y) = \mbox{ НОД}(x,y)$.
  \item 
    $f: \mathbb{N} \times \mathbb{N}\rightarrow \mathbb{N}$,
    $f(x, y) = 3x+2y$.
  \item 
    $f: \Nat \times \Nat\rightarrow \Nat$,
    $f(x, y) = 2^x(2y+1)-1$.
  \item 
    $f: \Nat \times \Nat\rightarrow \Nat$,
    $f(x, y) = 2x(2y+1)$.
  \item
    $f: \Nat \times \Nat\rightarrow \Nat$,
    $f(x, y) = 2x(2^y+1)$.
  \item
    $f: \Nat \times \Nat\rightarrow \Nat$,
    $f(x, y) = 2^x3^y$.
  \item
    $f: \Nat \times \Nat\rightarrow \Nat$,
    $f(x, y) = 2^x6^y$.
  \item 
    $f: \mathbb{R} \times \mathbb{R}\rightarrow \mathbb{R}$,
    $f(x, y) = x^2+y^2$.
  \item
    
  \end{enumerate}
\end{prb}


% \item 
%   $f: \mathbb{Q}\rightarrow \mathbb{Q}$, $f(x) =
%   \cstwo{0}{$x=0$}{\frac{1}{x}}{$x\neq 0$}$.
% \item
%   $f: \mathbb{R} \rightarrow \mathbb{R}$, $f(x) = |x|+1$.
% \item 
%   $f: (-\frac{\pi}{2}, \frac{\pi}{2})\rightarrow \mathbb{R}$,
%   $f(x) = tg x$.
% \item 
%   $f: \mathbb{N} \times \mathbb{N}\rightarrow \mathbb{N}$,
%   $f(x, y) = 3^x.5^y$.
% \item
%   $f: \mathbb{R}^+ \rightarrow \mathbb{N}$, $f(x) = \lfloor x
%   \rfloor$. (най-голямото естествено  число, по-малко или равно на
%   $x$. )
% \item
%   $f: \mathbb{N} \rightarrow \mathbb{N}$, $f(x) = (x+1)$-вото
%   просто число.
% \end{enumerate}


\subsection*{Операции върху функции}

Нека е дадена функцията $f:A\to B$.
Ще разгледаме няколко основни операции върху функции.
\begin{enumerate}[I)]
\item
  {\bf Образ}
  
  Нека $X\subseteq A$. {\em Образ на множеството} $X$ под действието на функцията $f$, наричаме
  множеството: \[f(X) = \{b\in B \mid f(a) = b\ \wedge\ a \in X\}.\]
\item
  {\bf Първообраз}

  Нека $Y\subseteq B$. {\em Първообраз на множеството} $Y$ под действието на функцията $f$, наричаме
  множеството: \[f^{-1}(Y) = \{a\in A \mid f(a) = b\ \wedge\ b \in Y\}.\]
\item
  {\bf Обратна функция}
  За всяка биективна функция $f:A\to B$, определяме нейната обратна функция $g:B \to  A$ като:
  \[(\forall a \in A)(\forall b \in Ran(f))[g(b) = a\ \iff\ f(a) = b].\]
  Обикновено означаваме $g$ като $f^{-1}$.
% \item 
%   {\bf Рестрикция}

%   Нека $X\subseteq A$. {\em Рестрикция} на $f$ до множеството $X$, наричаме
%   множеството: \[f\upharpoonright X = \{\langle{x,y}\rangle\mid f(x) = y\ \wedge\ x\in X\} =  f\cap X\times B.\]
% \item
%   {\bf Затваряне}
  
%   Нека в този случай $f:A\to A$ и нека $X\subseteq A$.
%   За всяко $n \geq 0$ определяме $X_0 = X$ и $X_{n+1} = X_n \cup f(X_n)$.
%   {\em Затваряне} на множеството $X$ относно функцията $f$ е множеството
%   \[f[X] = \bigcup_{n\in\Nat} X_n. \]
  
\item
  {\bf Композиция}

  Нека са дадени функциите $f:A\to B$ и $g:C\to A$.
  {\em Композиция} на $f$ и $g$ е функцията $f\circ g: C \to B$ определена като
  \[f\circ g = \{\pair{c,b}\mid (\exists a\in A)[g(c) = a\ \wedge\ f(a) = b]\}.\]
  \marginpar{Най-напред прилагаме $g$ и след това $f$}
  Композицията на $f$ и $g$ може да се запише и така:
  \[(\forall c\in C)[(f\circ g)(c) = f(g(c))]\]
\end{enumerate}

\begin{prb}
  Нека $f: A\to B$, $g: B\to C$ са функции.
  Вярно ли е, че:
  \begin{enumerate}
  \item 
    Ако $f$  не е инекция, то $g\circ f$ не е инекция?
  \item
    Ако $g$  не е инекция, то $g\circ f$ не е инекция?
  \item 
    Ако $f$  не е сюрекция, то $g\circ f$ не е сюрекция?
  \item
    Ако $g$  не е сюрекция, то $g\circ f$ не е сюрекция?
  \end{enumerate}
\end{prb}

\begin{prb}
  Нека $f: A\to B$, $g: B\to C$ са функции.
  Вярно ли е, че:
  \begin{enumerate}[1)]
  \item
    $f,g$ са инективни, то $g\circ f$ е инективна?
  \item
    $f,g$ са сюрективни, то $g\circ f$ е сюрективна?
  \item
    $f,g$ са биективни, то $g\circ f$ е биективна?
  \item
    $g\circ f$ е сюрективна,  то $f,g$ са сюрективни ?
  \item
    $g\circ f$ е инективна, то $f,g$ са инективни ?
  \end{enumerate}
\end{prb}

\begin{prb}
  Нека $f: A\to B$, $g: B\to C$ са {\em биективни} функции.
  Докажете, че
  \[(g\circ f)^{-1} = f^{-1}\circ g^{-1}.\]
\end{prb}

\begin{prb}
  $f(x) = \pair{g(x),h(x)}$.
\end{prb}


% \begin{dfn}
%   Дефинираме следните операции върху релацията $R\subseteq A\times{B}$:
%   \begin{enumerate}
%   \item
%     Дефиниционна област
%     $Domain(R) = \{x\mid (\exists y)\langle{x,y}\rangle\in R \}$;
%   \item
%     Област от стойности
%     $Range(R) = \{y\mid (\exists x)[\langle{x,y}\rangle\in R]\}$;
%   \item
%     Поле
%     $Field(R) = Domain(R) \cup Range(R)$;
%   \item
%     Рестрикция
%     $R\upharpoonright{C} = \{\langle{x,y}\rangle\mid \langle{x,y}\rangle\in R\ \&\ x\in C\}$;
%   \item
%     Образ
%     $R[C] = \{ y \mid (\exists x)[ x\in C\ \&\ \langle{x,y}\rangle\in R]\} = Range(R\upharpoonright{C})$.


% \end{enumerate}
% \end{dfn}

% \begin{example}
%   Нека да разгледаме релацията \[F = \{\langle{\emptyset, a}\rangle,\langle{\{\emptyset\}, b}\rangle\}.\]
%   Лесно се вижда, че $F$ е функция.
%   Имаме, че \[F^{-1} = \{\langle{a,\emptyset}\rangle,\langle{b, \{\emptyset\}}\rangle\}\] е функция тогава и само тогава, когато  $a\neq b$.
%   Обърнете внимание, че
%   \[F\upharpoonright{\emptyset} = \emptyset \mbox{, но } F\upharpoonright\{\emptyset\} = \{\langle{\emptyset,a}\rangle\}.\]
%   Освен това, $F(\{\emptyset\}) = \{a\}$ и $F(\{\emptyset\}) = b$.
% \end{example}

% Нека $f$ е функция и $A$ е множество.
% \begin{enumerate}[(i)]
% \item
%   $f(A) = \{y \mid (\exists x\in A)(f(x) = y)\}$
% \item
%   $f^{-1}(A) = \{x \mid f(x)\in A\}$
% \end{enumerate}


\begin{problem}
  Нека а дадена произволна функция $f:A \to B$.
  Проверете:
  \begin{enumerate}[a)]
  \item
    $(\forall X,Y \subseteq A)[f(X)\cup f(Y) = f(X\cup Y)]$;
  \item
    $f(\bigcup_{i\in I}X_i) = \bigcup_{i\in I}(X_i)$
  \item
    при какви условия за $f$,
    $(\forall X,Y \subseteq A)[f(X\cap Y) = f(X)\cap f(Y)]$.
  % \item
  %   $f(\bigcap_{i\in I}A_i) \subseteq \bigcap_{i\in I}f(A_i)$
  \item
    при какви условия за $f$,
    $(\forall X,Y \subseteq A)[f(X)\backslash f(Y) = f(X\backslash Y)]$.
  \item
    \marginpar{Опр. $(\forall x\in X)\ id_X(x) = x$}
    при какви условия за $f$, $f\circ f^{-1} = id_{B}$.
  \item
     при какви условия за $f$, $f^{-1}\circ f = id_{A}$.
   % \item
     % $Dom(f\circ g) = g^{-1}(Dom(f))$, където $g$ е функция.
  \item
    $(\forall X,Y \subseteq B)[f^{-1}(X\cup Y) = f^{-1}(X)\cup f^{-1}(Y)$.
  \item
    $(\forall X,Y \subseteq B)[f^{-1}(X\cap Y) = f^{-1}(X)\cap f^{-1}(Y)$.
  \item
    $(\forall X,Y \subseteq B)[f^{-1}(X\backslash Y) = f^{-1}(X)\backslash f^{-1}(Y)]$.
  \item
    при какви условия за $f$,
    $(\forall X\subseteq A)[X =  f^{-1}(f(X))]$.
  \item
    при какви условия за $f$,
    $(\forall Y \subseteq B)[Y = f(f^{-1}(Y))]$.
  \item
    $(\forall X\subseteq A)(\forall Y\subseteq B)[f(X)\cap Y = f(X\cap f^{-1}(Y))]$.
  \item
    $(\forall X \subseteq A)(\forall Y \subseteq B)[f(X)\cap Y = \emptyset \iff X\cap f^{-1}(Y) = \emptyset]$.
  \item
    $(\forall X \subseteq A)(\forall Y \subseteq B)[f(X)\subseteq Y \iff X\subseteq f^{-1}(Y)]$.
  \end{enumerate}
\end{problem}
\newpage
\begin{problem}%Л.М. 18 / 23
  Нека $f,g$ са функции. При какви условия :
  \begin{enumerate}
  \item
    $f^{-1}$ е функция?
  \item
    $f\circ g$ е инективна функция?
  \end{enumerate}
\end{problem}

% \begin{problem}
%   Дайте примери за функция $f:\mathbb{N}\rightarrow\Z$, която е:
%   \begin{enumerate}
%   \item
%     нито инективна, нито сюрективна;
%   \item
%     инективна, но не е сюрективна;
%   \item
%     сюрективна, но не е инективна;
%   \item
%     сюрективна и инективна.
% \end{enumerate}
% \end{problem}


\begin{problem}
  Нека е дадена релацията $R\subseteq A\times B$.
  Докажете, че $R$ е биективна функция тогава и само тогава, когато $R\circ R^{-1} = id_A$ и $R^{-1}\circ R = id_B$.
\end{problem}

\begin{problem}
  Нека $f$ е инективна функция от $A$ в $B$ и $g:\Ps(A) \rightarrow \Ps(B)$, дефинирана като 
  \[(\forall X \subseteq A)[g(X) = f(X)].\]
  Докажете, че $g$ е инективна.
\end{problem}

\begin{problem}
  Нека $f:A\rightarrow B$ и $g:B\rightarrow\Ps(A)$, дефинирана като 
  \[(\forall b \in B)[g(b) = \{x\in A\mid f(x) = b\}].\]
  Докажете, че ако $f$ е сюрективна, то $g$ е инективна.
  Вярна ли е обратната посока?
\end{problem}

\cite{hein}

%%% Local Variables: 
%%% mode: latex
%%% TeX-master: "discrete-math"
%%% End: 

\chapter{Мощност на множества}

\section{Основни понятия}

\begin{itemize}
\item 
  Казваме, че едно множество $A$ е {\bf изброимо безкрайно}\index{множество!изброимо безкрайно}, ако съществува 
  биекция от $A$ върху $\Nat$.
\item
  Едно множество е {\bf изброимо}, ако е или крайно или безкрайно изброимо.
\item
  Казваме, че едно множество $A$ е {\bf неизброимо}\index{множество!неизброимо}, ако $A$ е безкрайно и {\bf не} съществува 
  биекция от $A$ върху $\Nat$.
\item
  Казваме, че мощността на едно множество $A$ е не по-голяма от мощността на множеството $B$, 
  което записваме като $\abs{A} \leq \abs{B}$, ако съществува {\em инекция} $f:A \to B$.
  Възможно е да използваме и означението $A \preceq B$.
\item
  Когато множеството $A$ е крайно, например $A = \{a_1,\dots,a_n\}$, 
  ще записваме $\abs{A} = n$.
\item
  Две множества $A$ и $B$ са равномощни, $|A| = |B|$, ако съществува биекция от $A$ върху $B$.
  Алтернативен запис е $A \sim B$.
\item
  Записваме $\abs{A} < \abs{B}$, ако $\abs{A} \leq \abs{B}$ и $\abs{A} \neq \abs{B}$.
  Алтернативен запис е $A \precneqq B$, т.е. $A \preceq B$ и $A \not\sim B$.
\end{itemize}


\section{Сравняване на мощности}


\begin{framed}
\begin{thm}[Кантор-Шрьодер-Бернщайн]
  \index{Кантор-Шрьодер-Бернщайн}
  \label{th:ksb}
  За всеки две множества $A$ и $B$,
  \[A \preceq B\ \&\ B \preceq A \implies A \sim B.\]
\end{thm}
\end{framed}
\begin{proof}
  \marginpar{Според уикипедия това доказателство е на Гюла Кьониг(1906), синът на Денеш Кьониг}
  Без ограничение на общността, нека $A\cap B = \emptyset$.
  Нека също така да фиксираме инективни функции $f:A\rightarrow B$ и $g:B\rightarrow A$.
  Ще построим биективна функция $h:A\rightarrow B$.
  
  Понеже $g$ е инективна, то $g^{-1}$ също е (частична) функция. За $a\in A$, имаме следното:
  \[
  g^{-1}(\{a\}) = 
  \begin{cases}
    \emptyset, & a \not\in Range(g)\\
    \{b\}, & g(a) = b 
  \end{cases}
  \]
  Ако $g^{-1}(\{a\}) = \{b\}$, то наричаме $b$ {\em наследник} на $a$.
  Аналогично, понеже $f$ е инективна, то $f^{-1}$ също е (частична) функция и за $b\in B$:
  \[
  f^{-1}(\{b\}) = 
  \begin{cases}
    \emptyset, & a \not\in Range(f)\\
    \{a\}, & f(b) = a 
  \end{cases}
  \]
  Ако $f^{-1}(\{b\}) = \{a\}$, то казваме, че $a$ е {\em наследник} на $b$.
  Продължавайки същата схема, можем да се опитаме да намерим наследника на $a$ и т.н.
  За елемента $a$ имаме три възможни изхода от тази процедура:
  \begin{enumerate}[i)]
  \item
    $a$ има като последен наследник някой елемент от $A$;
  \item
    $a$ има като последен наследник някой елемент от $B$;
  \item
    $a$ има безкрайно много наследника.
\end{enumerate}
Например, следната верига
\[a \stackrel{g^{-1}}{\longrightarrow} b \stackrel{f^{-1}}{\longrightarrow}a_1 \stackrel{g^{-1}}{\longrightarrow} b_1 \stackrel{f^{-1}}{\longrightarrow}a_2\stackrel{g^{-1}}{\longrightarrow}\emptyset\]
показва, че $a \in A_1$, защото последния наследник на $a$ е елемента $a_2 \in A$.
Да означим множествата:
\begin{align*}
  A_1 = & \{a\in A \mid \mbox{ веригата с начало $a$ завършва с елемент от } A\}\\
  A_2 = & \{a\in A \mid \mbox{ веригата с начало $a$ завършва с елемент от } B\}\\
  A_3 = & \{a\in A \mid \mbox{ веригата с начало $a$ е безкрайна} \}.
\end{align*}
Лесно се съобразява, че $A_1\cup A_2\cup A_3 = A$ и 
множествата $A_1$, $A_2$ и $A_3$ нямат общи елементи.
Аналогично дефинираме:
\begin{align*}
  B_1 = & \{b\in B \mid \mbox{ веригата с начало $b$ завършва с елемент от} A\}\\
  B_2 = & \{b\in B \mid \mbox{ веригата с начало $b$ завършва с елемент от } B\}\\
  B_3 = & \{b\in B \mid \mbox{ веригата с начало $b$ е безкрайна} \}.
\end{align*}
Отново $B_1\cup B_2\cup B_3 = B$ и множествата $B_1$, $B_2$ и $B_3$ нямат общи елементи.

Да разгледаме функциите $f_i = f\upharpoonright{A_i}$ и $g_i = g\upharpoonright{B_i}$, $i = 1,2,3$. 
Лесно се съобразява, че 
\begin{align*}
  f_i:& A_i\to B_i,\\
  g_i:& B_i\to A_i.
\end{align*}
Например, нека $a \in A_1$ и $b = f(a)$. Да съобразим, че наистина $b \in B_1$.
Имаме, че $f^{-1}(\{b\}) = \{a\}$, т.е. $a$ е {\em наследник} на $b$.
Получаваме веригата:
\[b \stackrel{f^{-1}}{\longrightarrow}a \stackrel{g^{-1}}{\longrightarrow} b_1 \stackrel{f^{-1}}{\longrightarrow} a_1 \stackrel{g^{-1}}{\longrightarrow}\dots \stackrel{g^{-1}}{\longrightarrow}a' \in A\]
Това означава, че наистина $b \in B_1$.

Ясно е, че всички тези функции $f_i$, $g_i$ са инективни, $i = 1,2,3$.
За да построим биективна функция $h:A\to B$ е достатъчно да докажем, че 
поне една функция във всяка от двойките $(f_i,g_i)$, $i = 1,2,3$ е биективна.
Тогава ще получим $h$ като ,,слепим'' три такива биекции.
Кои от тях са биективни? Достатъчно е да проверим кои от тях са сюрективни.
Ще разгледаме всички шест функции.
\begin{enumerate}[i)]
\item 
  Да разгледаме $b \in B_1$. Това означава, че веригата започваща с $b$
  завършва в $A$. Следователно, съществува 
  $a \in A_1$, за който $a = f^{-1}(b)$.
  Заключаваме, че $f_1$ е сюрективна и следователно биективна.
\item
  Да разгледаме $b \in B_2$. Това означава, че веригата започваща с $b$
  завършва в $B$.
  Обаче може $b$ изобщо да няма наследници, т.е.
  възможно е $f^{-1}(\{b\}) = \emptyset$.
  Това означава, че може $Range(f_2) \subsetneqq B_2$ и
  нямаме гаранция, че $f_2$ е сюрективна.
\item
  Да разгледаме $b \in B_3$. Това означава, че веригата за $b$
  е безкрайно дълга.
  Следователно, съществува $a \in A_3$, за което $f^{-1}(b) = a$.
  Заключаваме, че $f_3$ е сюрективна и следователно биективна.
\item
  Да разгледаме $a \in A_1$. Това означава, че веригата за $a$
  завършва в $A$. 
  Обаче пак както в {\em ii)} може $a$ изобщо да няма наследници, т.е.
  $g^{-1}(\{a\}) = \emptyset$.
  Това означава, че може $Range(g_1) \subsetneqq A_1$ и 
  може $g_1$ да не е сюрективна.
\item
  Да разгледаме $a \in A_2$. Това означава, че веригата за $a$ завършва в  $B$.
  Следователно, съществува $b \in B_2$, за който $b = g^{-1}(a)$.
  Заключаваме, че $g_2$ е сюрективна и следователно биективна.
\item
  Да разгледаме $a \in A_3$. Това означава, че веригата с начало $a$ е безкрайно дълга.
  Следователно, съществува $b \in B_3$, за което $g^{-1}(a) = b$.
  Заключаваме, че $g_3$ е сюрективна и следователно.
\end{enumerate}

Накрая получаваме, че функциите $f_1,f_3$ и $g_2,g_3$ са биективни.

Да определим биекция $h:A\rightarrow B$ по следния начин:
\[
h(a) =
\begin{cases}
  f_1(a),     & \quad \text{ако $a\in A_1$}\\
  g^{-1}_2(a), & \quad \text{ако $a\in A_2$}\\
  f_3(a),     & \quad \text{ако $a\in A_3$}\\
\end{cases}
\]
Така доказахме, че множествата $A$ и $B$ са {\bf равномощни}, т.е. $A \sim B$.
\end{proof}

\begin{cor}
  Ако $A \subseteq B \subseteq C$ и $A \sim C$, то $B \sim C$.
\end{cor}
% \begin{proof}
%   Понеже $B \subseteq C$, то $\abs{B} \leq \abs{C}$.
%   Понеже $\abs{C} = \abs{A}$, то съществува инекция $g:C \to A$ и $g:C\to B$ също е инекция.
%   Получаваме, че $\abs{C} \leq \abs{B}$.
%   Тогава от \Th{ksb} следва, че $\abs{B} = \abs{C}$.
% \end{proof}

\begin{example}
  Не е лесно да се докаже, че $(0,1)_\Real \sim (0,1]_\Real$ като се посочи биекция.
  Обаче с \Th{ksb} това не е толкова трудно.
  \begin{enumerate}[1)]
  \item 
    Очевидно е, че има инекция $(0,1)_\Real$ в $(0,1]_\Real$;
  \item
    Можем да дефинираме инекция $f:(0,1]_\Real \to (0,1)_\Real$
    като $f(x) = \frac{x}{2}$.    
  \end{enumerate}
  Като имаме 1) и 2), от \Th{ksb} следва, че двете множества са равномощни.
  Макар и да не сме посочили такава биекция, то от теоремата знаем, че тя съществува.  
\end{example}

% \begin{problem}
%   Докажете, че ако $g:A\rightarrow B$ е сюрекция, то $|A|\geq |B|$;
% \end{problem}
% \begin{proof}
%   Понеже $g$ е сюрективна, то $g^{-1}(\{b\}) \neq \emptyset$, за всяко $b \in B$.
%   Да отбележим също, че $b \neq b'$, то $g^{-1}(\{b\}) \cap g^{-1}(\{b'\}) = \emptyset$.
%   Всяка функция $f:B\to A$, която изпълнява свойството, че $f(b) \in g^{-1}(\{b\})$,
%   е инективна и следователно $|B| \leq |A|$.
% \end{proof}

\begin{framed}
\begin{remark}
  Напълно възможно е за две множества $A$ и $B$ да имаме, че  $B \subsetneqq A$, но $A \sim B$.
  Например, нека $A = \Nat$ и $B = \{2n\mid n\in\Nat\}$.
\end{remark}
\end{framed}

\begin{thm}
  \marginpar{Това означава, че можем да образуваме все по-неизброими множества. Например, $\Ps(\Real)$ има по-голяма мощност от $\Real$}
  Нека $A$ е множество и $\Ps(A)$ е множеството от всички подмножества на $A$.
  Докажете, че $A \precneqq \Ps(A)$.
\end{thm}
\begin{proof}
  Функцията $h:A \to \Ps(A)$ определена като $h(a) = \{a\}$ е инекция.
  Следователно, $A \preceq \Ps(A)$.

  Да допуснем, че $A \sim \Ps(A)$, т.е. 
  съществува биекция $f:A\rightarrow \Ps(A)$.
  Да разгледаме множеството \[B=\{a\in A\ \mid a\notin f(a)\}\in\Ps(A).\]
  Щом $f$ е биекция, съществува {\em единствено} $a_0\in A: f(a_0) = B$.
  Но тогава имаме следното:
  \begin{itemize}
  \item
    ако $a_0\in B$, то $a_0 \not\in f(a_0)$ и тогава $a_0\not\in B$;
  \item
    ако $a_0\not\in B$, то $a_0 \in f(a_0)$ и тогава $a_0\in B$.
  \end{itemize}
  И в двата случая достигаме до противоречие.
  Следователно {\bf не съществува биекция} от $A$ върху $\Ps(A)$.
  Накрая заключаваме, че $A \precneqq \Ps(A)$.
\end{proof}

\section{Изброими множества}

\begin{prop}
  Множеството $\Nat\times\Nat$ е изброимо безкрайно.
\end{prop}
\begin{hint}
  Целта е да намерим биекция от $\Nat\times \Nat$ върху $\Nat$.
  Съществуват много такива функции.
  \begin{enumerate}[a)]
  \item 
    \marginpar{Нарича се Канторово кодиране. Има удобно графично представяне}
    Разгледайте функцията 
    \[\pi(x,y) = \frac{1}{2}((x+y)^2+3x+y).\]
  \item
    Разгледайте функцията
    \[\pi(x,y) = 2^x(2y+1)-1.\]
  \end{enumerate}
\end{hint}

\begin{prop}
  \label{pr:pi-k}
  За всяко $k$, множеството $\Nat^k$ е изброимо безкрайно.
\end{prop}
\begin{hint}
  Индукция по $k \geq 2$.
  \begin{itemize}
  \item 
    За $k = 2$, от предишното твърдение имаме биекцията $\pi:\Nat^2 \to \Nat$.
    Да положим $\pi_2 = \pi$.
  \item
    Нека $k = m+1$.
    Тогава \[\pi_{m+1}(n_1,\dots,n_{m+1}) = \pi_m(f(n_1,\dots,n_m),n_{m+1}),\]
    където сме използвали биекцията $\pi_m:\Nat^m \to \Nat$, която имаме от И.П.
  \end{itemize}
\end{hint}

\begin{prop}
  Ако $A$ е изброимо безкрайно множество, то $A^k$ също е изброимо безкрайно множество,
  където $k \geq 2$ е естествено число.
\end{prop}

\begin{prop}
  Да разгледаме редица от изброимо безкрайни множества $A_0,A_1,\dots$ със свойството, че $i \neq j \implies A_i \cap A_j = \emptyset$.
  Тогава множеството 
  \[B = \bigcup^\infty_{i=0}A_i\] е изброимо безкрайно.
\end{prop}

\begin{framed}
  \begin{thm}[Кантор 1874]
    Множеството на рационалните числа $\Q$ е изброимо безкрайно.
  \end{thm}
\end{framed}
\begin{hint}
  Разгледайте за $n = 1,2,3\dots$ множествата 
  \[Q_n = \left\{\frac{m}{n} \mid m \in \Int\ \&\ \text{НОД}(m,n)=1\right\}.\]
  Всяко от тези множества е изброимо безкрайно.
  Тогава 
  \[\Q = \bigcup_{n\geq 1}Q_n\]
  е изброимо безкрайно множество.
\end{hint}

\begin{problem}
  Докажете, че следните множества са изброимо безкрайни:
  \begin{enumerate}[a)]
  \item 
    $A \cup B$, където поне едното от $A$ и $B$ е изброимо безкрайно;
  \item
    $\bigcup_{i\in\Nat} A_i$, където всяко от множествата $A_i$ да е изброимо безкрайно, за $i = 0,1,2,\dots$;
  \item
    $A \times B$, където поне едно от множествата $A$ и $B$ е изброимо безкрайно;
  \item
    \marginpar{Озн. $A^\star = \bigcup_{n\in\Nat}A^n$}
    $A^\star$, където $A$ е крайна азбука;
  \item
    $A^\star$, където $A$ е изброимо безкрайна азбука;
  \item
    $B$ - множеството от тези думи над азбуката $\{0,1\}$, които не започват с $0$, с изключение на 
    думата $0$, т.е. $B = \{0, 1, 10, 11, 100, 101, 110, 111, \dots\}$;
  \item
    $\Ps_{fin}(\Nat)$ - множеството от всички крайни подмножества от естествени числа;
  \item
    $\Ps_{fin}(A^*)$ - множеството от всички крайни подмножества на $A^*$, за произволна азбука 
    крайна или изброимо безкрайна азбука $A$;
  \item
    съвкупността от всички полиноми на една променлива с цели коефициенти;
  \item
    съвкупността от всички реални алгебрични числа (т.е. корени на полиноми с цели коефициенти).
  \item
    $[0,1]_{\mathbb{Q}} = \{q \in \mathbb{Q} \mid 0 \leq q \leq 1\}$;
  \item
    $[a,b]_{\mathbb{Q}} = \{q \in \mathbb{Q} \mid a \leq q \leq b\}$, за произволни рационални числа $a < b$;
  \end{enumerate}
\end{problem}
\begin{proof}
  \begin{enumerate}[a)]
  % \item
  %   Щом $A \subseteq A\cup B$, то $\abs{A} \leq \abs{A\cup B}$ и следователно
  %   $\abs{\Nat} \leq \abs{A\cup B}$.
  %   Нека $f:A\to \Nat$ и $g:B\to\Nat$ са инекции.
  %   Функцията $h:A\cup B\to \Nat$, определена като:
  %   \begin{align*}
  %     h(x) = 
  %     \begin{cases}
  %       2f(x), & x \in A\setminus B\\
  %       2g(x) + 1, & \mbox{ иначе}
  %     \end{cases}
  %   \end{align*}
  %   също е инекция.
  %   Тогава $\abs{A\cup B} \leq \abs{\Nat}$ и следователно, 
  %   $A\cup B$ е изброимо множество. 
  %   Съобразете, че $A\cup B$ е също така и безкрайно.
  \item[г)]
    Нека $A = \{a_1,\dots,a_k\}$.
    Лесно се съобразява, че $\abs{A^n} = k^n$.
    За някое $n$, да разгледаме множеството от думи 
    \[A^n = \{\alpha^n_{1},\alpha^n_{2},\dots, \alpha^n_{{k^n}}\}.\]
    \marginpar{\todo Докажете, че $f$ е биекция!}
    Можем да дефинираме инективната функция $f_n : A^n \to \Nat$ като
    \[f_n(\alpha^n_{i}) = \sum_{i<n} k^i + i.\]
    Понеже $A^n \cap A^{n+1} = \emptyset$, 
    то $f = \bigcup_n f_n : A^\star \to \Nat$ е функция.
  \item[д)]
    Нека $\kappa:\Nat \to A$ е биекция.
    Да изброим всички букви като $a_i = \kappa(i)$ за $i = 0,1,2,\dots$.
    \marginpar{\todo Докажете, че $f$ е биекция!}
    Дефинираме биекция $f:A^\star \to \Nat$ по следния начин:
    \[f(a_0,\dots,a_n) = \pi(n, \pi_{n+1}(a_0,\dots,a_n)),\]
    където използваме функциите дефинирани в Твърдение \ref{pr:pi-k}.
  \item[ж)]
    Нека на крайното множество от естествени числа
    \[D = \{n_0 < n_1 < \cdots < n_k\}\]
    да съпоставим числото $v = 2^{n_0} + 2^{n_1} + \cdots + 2^{n_k}$, което ще наричаме код на $D$.
    \marginpar{Ако $D = \{1,3,4\}$, то $v = (11010)_2 = 26$}
    С $D_v$ ще означаваме крайното множество с код $v$.
    Разгледайте $f:\Nat \to \Ps_{fin}(\Nat)$ дефинирана като
    \marginpar{\todo Докажете, че $f$ е биекция!}
    $f(v) = D_v$.
  % \item[к)]
  %   Всеки помином може да се представи като дума над азбуката $\Nat$.
  % \item[м)]
  %   Да разгледаме инективната функцията $f:\Nat \to [0,1]_{\mathbb{Q}}$ определена като:
  %   \[f(n) = \frac{1}{2^n}.\]
  %   Използвайте теоремата на Кантор-Шрьодер-Бернщайн.
  % \item[н)]
  %   Да фиксираме $a < a_1 < b_1 < b$.
  %   Дефинираме $f:\Nat \to [a,b]_{\mathbb{Q}}$ по следния начин:
  %   \begin{align*}
  %     & f(0) = \frac{a_1+b_1}{2}\\
  %     & f(n+1) = \frac{a_1+f(n)}{2}.
  %   \end{align*}
  \end{enumerate}
\end{proof}

\begin{problem}
  Нека $A$ е изброимо безкрайно множество.
  Докажете, че всяко $I \subseteq A$ е изброимо безкрайно или крайно.
\end{problem}
\begin{hint}
  Достатъчно е да разгледаме случая $A = \Nat$.
  Да разгледаме безкрайното подмножество $I \subseteq \Nat$.
  За да докажем, че то е {\em изброимо}, ще построим биекция $f:\Nat \to I$.
  Нека
  \begin{align*}
    f(0)   & = \min\{i \mid i \in I\}\\
    f(n+1) &= \min\{i \mid i \in I \setminus\{f(0),\dots,f(n)\}\}.
  \end{align*}
  Докажете, че $f$ е биекция от $\Nat$ върху $I$.
\end{hint}

\section{Неизброими множества}

\begin{problem}
  Докажете, че $\Nat^\Nat = \{f \mid f:\Nat \to \Nat\}$ е неизброимо.
\end{problem}
\begin{proof}
  Ще приложим метода на диагонализацията. 
  Да допуснем, че $\Nat^\Nat$ е изброимо.
  Тогава можем да подредим в редица всички функции \[f_0,f_1,f_2,\dots.\]
  Дефинираме функция $\kappa$, като $\kappa(i) = f_i(i)+1$.
  Да допуснем, че $\kappa = f_n$, за някое $n$.
  Но $\kappa(n) = f_n(n)+1 \neq f_n(n)$, следователно стигаме до противоречие.
  Заключаваме, че $\Nat^\Nat$ е неизброимо.
\end{proof}

% \begin{thm}[Кантор]
%   Интервалът от реални числа $[0,1]$ е неизброим.
% \end{thm}
% \begin{proof}
%   Да допуснем, че
%   \[[0,1] = \{r_1,r_2,\dots,r_n,\dots\},\]
%   т.е. можем да изброим всички реални числа в интервала $[0,1]$.
%   Нека $I_0 = [0, 1]$, $a_0 = 0$, $b_0 = 1$.
%   Да разгледаме интервалите $[0,1/3]$, $[1/3,2/3]$ и $[2/3,1]$ 
%   и да означим като $I_1 = [a_1,b_1]$ един от тях, за който $r_1 \not\in I_1$.
%   Ясно е, че $b_1-a_1 = 1/3$ и $[a_1,b_1] \subset [a_0,b_0]$.
%   Продължаваме процедурата като разгледаме интервала $[a_1,b_1]$ разделен пак на три равни части.
%   Избираме една от тези части $I_2 = [a_2,b_2]$, за която $r_2 \not\in [a_2,b_2]$.
%   Ясно е, че $b_2-a_2 = 1/3^2$ и $[a_2,b_2] \subset [a_1,b_1]$.
%   По този начин продължаваме процедурата като на стъпка $n+1$ 
%   избираме интервал 
%   \[I_{n+1} = [a_{n+1},b_{n+1}],\] за който $r_{n+1} \not\in [a_{n+1},b_{n+1}]$.
%   Тогава $b_{n+1}-a_{n+1} = 1/3^{n+1}$ и $I_{n+1} \subset I_n$.

%   Накрая получаваме безкрайна редица $\{I_n\}_{n\in\Nat}$, като имаме свойствата:
%   \[(\forall n)[I_{n+1}\subset I_n],\]
%   \[0\leq a_n \leq a_{n+1} < b_{n+1} \leq b_n \leq 1.\]
%   Редиците $\{a_n\}$ и $\{b_n\}$ са монотонни и ограничени, следователно са сходящи 
%   (т.е. съществуват $\lim_{n\to\infty} a_n$ и $\lim_{n\to\infty} b_n$).
%   Освен това, от \[(\forall n)[b_n-a_n \leq 1/3^n]\] следва, че 
%   \[\lim_{n\to\infty}(b_n-a_n) = 0\] и тогава съществува реално число 
%   \[r = \lim_n a_n = \lim_n b_n.\]
%   За това число $r$,
%   \[(\forall n)[r \neq r_n],\]
%   защото $r \in I_n$, но $r_n \not\in I_n$.
%   Достигаме до противоречие.
%   Следователно заключаваме, че не можем да подредим всички реални числа в интервала $[0,1]$
%   в една редица.
% \end{proof}

Представяме и друго доказателство на горната теорема.
\begin{problem}
  Докажете, че отвореният интервал от реални числа $(0,1)_\Real$ е неизброимо множество.
\end{problem}
\begin{proof}
  Да допуснем, че интервалът $(0,1)_\Real$ е изброим. Това означава, че можем да подредим всички елементи на $(0,1)_\Real$ в редица.
  Да представим всяко реално число в интервала $(0,1)_\Real$ в неговата десетична форма.
  Някои реални числа могат да имат по две десетични форми.
  Например, 
  \[0.2 = 0.1999999\dots.\]
  Нека винаги избираме тази, която започва с по-малко число, например избираме $0.1999\dots$ вместо $0.2$.
  Да подредим всички реални числа в интервала $(0,1)_\Real$ в редица:
  \begin{align*}
    r_0 & = 0.d_{00}d_{01}d_{02}\dots\\
    r_1 & = 0.d_{10}d_{11}d_{12}\dots\\
    \vdots\\
    r_n & = 0.d_{n01}d_{n1}d_{n2}\dots\\
    \vdots\\
  \end{align*}

  Да изберем две различни числа между 1 и 9. Например, 5 и 7.
  Нека 
  \begin{align*}
    k_i = 
    \begin{cases}
      5, & \mbox{ ако } d_{ii} = 7,\\
      7, & \mbox{ ако } d_{ii} \neq 7.
    \end{cases}
  \end{align*}
  Дефинираме $\kappa$ като реалното число, което има десетично представяне $0.k_0k_1k_2\dots$.
  Ясно е, че $\kappa \in (0,1)_\Real$. Ще покажем, че $\kappa \neq r_n$ за всяко $n$.
  Да отбележим, че понеже в $\kappa$ не участват редици от $0$-ли или $9$-ки, то със сигурност 
  реалното число $\kappa$ има единствено десетично представяне.
  Да допуснем, че $\kappa = r_n$, за някое $n$.
  Но $k_{nn} \neq d_{nn}$, следователно стигаме до противоречие.
  Заключаваме, че $(0,1)_\Real$ е {\bf неизброимо}.
\end{proof}

\begin{remark}
  От последната задача директно следва, че множеството $\Real$ е {\bf неизброимо} безкрайно.
\end{remark}


В следващата задача ще видим, че е удобно да можем да представяме 
всяко реално число в двоична бройна система.
Например,
\begin{align*}
  5.375 & = (1.2^2 + 0.2^1 + 1.2^0).(0.2^{-1} + 1.2^{-2} + 1.2^{-3} + 0.2^{-4} + 0.2^{-5} + \dots)\\
  & = (101.011)_2\\
  1 & = 0.(1.2^{-1} + 1.2^{-2} + 1.2^{-3} + 1.2^{-4} + 1.2^{-5} + \dots)\\
  & = (0.11111\dots)_2.
\end{align*}
Естествено, много реални числа ще имат безкраен запис в двоична бройна система.
Да разгледаме $\pi = 3.14159\dots$ и да видим как можем да намираме все по-добри негови 
апроксимации в двоична бройна система.
Да умножим $\pi$ по $2^3$. Получаваме число между $25$ и $26$. 
$25 = (11001)_2$ и следователно двоичният запис на $\pi$ започва с $(11.001)_2$, т.е.
преместваме двоичната точка $3$ места наляво. Да проверим дали това наистина е така.
Сега ако умножим $\pi$ по $2^6$, получаваме число между $201$ и $202$.
$201 = (11001001)_2$. Наистина, двоичният запис на $\pi$ започва с $(11.001001)_2$,
т.е. преместване двоичната точка $6$ места наляво.
% \end{framed}

% \begin{problem}
%   Множеството $\Ps(\Nat)$ е равномощно с това на затворения интервал от реални числа $[0,1]$.
% \end{problem}
% \begin{proof}
%   Ще използваме Теорема \ref{th:ksb}. За целта, първо ще докажем $|\Ps(\Nat)|\leq |[0,1]|$ и след това $|[0,1]|\leq |\Ps(\Nat)|$.
%   \begin{enumerate}[1)]
%   \item 
%     \marginpar{На практика доказваме, че $|\Ps(\Nat)| \leq |(0,1)|$}
%     Ще докажем $|\Ps(\Nat)|\leq |[0,1]|$.
%     Дефинираме $h:\Ps(\Nat)\to [0,1]$ като на всяко подмножество от естествени числа съпоставяме реално число в десетичен запис.
%     \[h(S) = 0.d_0d_1d_2\dots,\mbox{ където } d_i = 1,\mbox{ ако }i\in S\mbox{, иначе } d_i = 0.\]
%     Например,
%     \begin{enumerate}[]
%     \item
%       $h(\emptyset) = 0.0000\dots$
%     \item
%       $h(\{0\}) = 0.100000\dots$
%     \item
%       $h(\{1,2\}) = 0.011000000\dots$
%     \item
%       $h(\Nat) = 0.11111111111\dots$
%     \end{enumerate}
%     Лесно се вижда, че $h$ е инекция, следователно $|\Ps(\Nat)|\leq|[0,1]|$.
%   \item
%     Ще докажем $\abs{[0,1]} \leq \abs{\Ps(\Nat)}$.
%     Сега ще построим инекция $g:[0,1]\to\Ps(\Nat)$, като
%     за всеки елемент $b\in[0,1]$ избираме едно негово {\em двоично представяне} (може да има повече от едно)
%     $b = (0.b_0b_1b_2\dots)_2$ и дефинираме \[g(b) = \{i\in\Nat\mid b_i = 1\}.\]
%   \end{enumerate}
  
%   Ето няколко примера:
%   \begin{enumerate}[]
%   \item
%     $g(0) = g((0.00000\dots)_2) = \emptyset$.
%   \item
%     $g(1/4) = g((0.01000\dots)_2) = \{1\}$.
%   \item
%     $g(1/2) = g((0.10000\dots)_2) = \{0\}$.
%   \item
%     $g(3/4) = g((0.1100000\dots)_2) = \{0,1\}$.
%   \item
%     $g(3/8) = g((0.01100000\dots)_2) = \{1,2\}$.
%   \item
%     $g(1) = g((0.1111111\dots)_2) = \Nat$.
%   \end{enumerate}
  
%   Едно число може да има две представяния, но ние сме сигурни, че различни числа имат различни преставяния.
% \end{proof}

\begin{problem}
  Докажете, че следните множества са равномощни и следователно са {\bf неизброими}.
  \begin{enumerate}[a)]
  \item
    Множеството на реалните числа $\Real$;
  \item
    $(0,1)_\Real = \{x \in \Real \mid 0 < x < 1\}$;
  \item
    $[0,1]_\Real = \{x \in \Real \mid 0 \leq x \leq 1\}$;
  \item
    $(a,b)_\Real = \{x \in \Real \mid a < x < b\}$, където $a<b$ са произволни реални числа.
  \end{enumerate}
\end{problem}
\begin{hint}
  \begin{description}
  \item[а) $\leftrightarrow$ б)]
    Един начин е да използваме \Th{ksb}. Това означава, че е достатъчно да дефинираме инекция $f:\Real \to (0,1)_\Real$.

    % За определеност, ако $r$ има повече от едно представяния, нека сме избрали това, което е по-голямо относно лексикографската наредба.
    % Например, ако $r = 31.999999\cdots = 32.00000\cdots$, то избираме $32.0000000\cdots$.
    % На всяко реално число \[r=r_1r_2\cdots r_n.p_1p_2\cdots p_k \cdots,\]
    % където $r_1 \neq 0$, съпоставяме реалното число в интервала $(0,1)_\Real$:
    % \[r' = 0.\underbrace{7\cdots 7}_{n+1} r_1r_2\cdots r_np_1p_2\cdots p_k \cdots.\]
    % Докажете, че функцията $f(r) = r'$ е инективна.
    Да разгледаме следните функции:
    \begin{itemize}
    \item 
      $f_1: (1,\infty)_\Real \to (0,\frac{1}{4})_\Real$ е инекция дефинирана като
      \[f_1(x) = \frac{1}{4x}.\]
    \item
      $f_2: [0,1]_\Real \to [\frac{1}{4},\frac{1}{2}]_\Real$ е инекция дефинирана като
      \[f_2(x) = \frac{1}{4} + \frac{x}{4}.\]
    \item
      $f_3: (-1,0)_\Real \to (\frac{1}{2},\frac{3}{4})_\Real$ е инекция дефинирана като
      \[f_3(x) = \frac{1}{2} - \frac{x}{4}.\]
    \item
      $f_4: (-\infty,-1]_\Real \to (\frac{3}{4},1]_\Real$ е инекция дефинирана като
      \[f_4(x) = \frac{3}{4} - \frac{1}{4x}.\]
    \end{itemize}

    За друга инекция, нека да разглеждаме реалните числа в двоичен запис.
    \marginpar{Например, $(7,25)_{10} = (111,01)_2$}
    Ако едно реално число има две представяния в двоичен запис, то взимаме по-голямото от двете в лексикографската наредба.
    Сега на реалното число $r$ съпоставяме една крайна дума $r_1\cdots r_n$ и (потенциално безкрайна) дума $p_1\cdots p_k\cdots$, такива че
    \[(r)_{10} = (r_1\cdots r_n,p_1p_2\cdots p_n \cdots)_2,\]
    и трябва да помним дали числото е положително или отрицателно.
    Сега да разгледаме следната (потенциално безкрайна) дума над азбуката $\{0,1,2\}$:
    \[\hat{r} = \underbrace{0\cdots 0}_{i}\underbrace{2\cdots 2}_{n+1}r_1r_2\cdots r_n p_1p_2\cdots, \]
    където $i = 0$ ако $r$ е положително, $i = 1$ ако $r$ е отрицателно, и $i = 2$ ако $r = 0$.
    \marginpar{\todo Проверете дали това е инекция!}
    Функцията $f(r) = (0.\hat{r})_{10}$ е инективна функция от $\Real$ в $(0,1)_\Real$.
    
    Трето решение ще бъде директно, без позоваване на \Th{ksb}.
    Знаем, че $\tan: (-\pi/2,\pi/2) \to \Real$ е биекция.
    Освен това, $f: (0,1) \to (-\pi/2,\pi/2)$, дефинирана като
    \[f(x) = \pi/2 - \pi x\] също е биекция.
    Тогава функцията $\tan\circ f : (0,1) \to \Real$ е биекция.    
    
    % Като четвърто решение, да разгледаме $f:(0,1)_\Real \to \Real$, където
    % \[f(x) = \frac{1}{x} + \frac{1}{1-x}.\]
    % Докажете, че $f$ е биективна.
  \item[a) $\leftrightarrow$ в)]
    Използвайте \Th{ksb} с едно от първите две решения на а) $\leftrightarrow$ б).
  \item[б) $\leftrightarrow$ г)]
    \marginpar{\todo Проверете, че $f$ е биекция!}
    Разгледайте функцията $f: (0,1)_\Real \to (a,b)_\Real$, където
    \[f(x) = a + (b-a)x.\]
  \end{description}
\end{hint}


\begin{problem}
  Докажете, че следните множества са равномощни и следователно са {\bf неизброими}.
  \begin{enumerate}[a)]
  \item
    $\Ps(\Nat) = \{A \mid A \subseteq \Nat\}$;
  \item
    $(0,1)_\Real = \{r \in \Real \mid 0 < r < 1\}$;
  \item
    $\Nat^\Nat = \{f\ \mid\ f:\Nat\to\Nat\text{ тотална}\}$.
  \item
    $2^\Nat = \{f\ \mid\ f:\Nat\to\{0,1\}\text{ тотална}\}$
  \end{enumerate}
\end{problem}
\begin{hint}
  \begin{description}
  \item[а) $\rightarrow$ б)]
    \marginpar{За а) $\rightarrow$ б) и б) $\rightarrow$ а) строим две инекции. След това използваме \Th{ksb} за да получим биекция между множествата от  а) и б)}
    Дефинираме $h:\Ps(\Nat)\to (0,1)$ като на всяко подмножество от естествени числа съпоставяме реално число в десетичен запис.
    \[h(S) = 0.d_0d_1d_2\dots,\mbox{ където } d_i = 1,\mbox{ ако }i\in S\mbox{, иначе } d_i = 0.\]
    Например,
    \begin{itemize}
    \item
      $h(\emptyset) = 0.0000\dots$
    \item
      $h(\{0\}) = 0.100000\dots$
    \item
      $h(\{1,2\}) = 0.011000000\dots$
    \item
      $h(\Nat) = 0.11111111111\dots$
    \end{itemize}
    Лесно се вижда, че $h$ е инекция, следователно $|\Ps(\Nat)|\leq|(0,1)|$.

  \item[б) $\rightarrow$ а)]
    Ще построим инекция $g:(0,1)_\Real\to\Ps(\Nat)$, като
    за всеки елемент $b\in (0,1)_\Real$ избираме едно негово {\em двоично представяне} (може да има повече от едно)
    $b = (0.b_0b_1b_2\dots)_2$ и дефинираме \[g(b) = \{i\in\Nat\mid b_i = 1\}.\]
    За определеност, ако едно реално число има повече от едно представяния, избираме това, което е най-голямо относно лексикографската наредба.
    Например, 
    \[1/2 = (0,100000\dots)_2 = (0,011111\dots)_2.\]

    Ето няколко примера:
    \begin{itemize}
    \item
      $g(0) = g((0.00000\dots)_2) = \emptyset$.
    \item
      $g(1/4) = g((0.01000\dots)_2) = \{1\}$.
    \item
      $g(1/2) = g((0.10000\dots)_2) = \{0\}$.
    \item
      $g(3/4) = g((0.1100000\dots)_2) = \{0,1\}$.
    \item
      $g(3/8) = g((0.01100000\dots)_2) = \{1,2\}$.
    \item
      $g(1) = g((0.1111111\dots)_2) = \Nat$.
    \end{itemize}
    
    Едно число може да има две представяния, но ние сме сигурни, че различни числа имат различни представяния.
  \item[а) $\rightarrow$ г)]
    Да разгледаме едно множество $A \subseteq \Nat$.
    Съпоставяме на $A$ функцията $f_A:\Nat \to \{0,1\}$  по следния начин:
    \marginpar{$f_A$ се нарича характеристична функция за $A$}
    \[
    f_A(n) = 
    \begin{cases}
      1, & n \in A\\
      0, & n\not\in A.
    \end{cases}
    \]
    Проверете какви свойства има функцията $h:\Ps(A) \to 2^\Nat$ дефинирана като $h(A) = f_A$.
  \item[г) $\rightarrow$ а)]
    Да разгледаме функцията $f:\Nat\to\{0,1\}$.
    На нея съпоставяме множеството $A_f = \{n \in \Nat \mid f(n) = 1\}$.
    Проверете какви свойства има функцията $h:2^\Nat \to \Ps(A)$ дефинирана като $h(f) = A_f$.
  \item[б) $\rightarrow$ в)]
    Да разгледаме една функция $f:\Nat\to\Nat$.
    На нея съпоставяме реалното число $r_f \in (0,1)_\Real$, където
    \[r_f = 0,\underbrace{00\dots 0}_{f(0)+1}1\underbrace{00\dots 0}_{f(1)+1}1\dots\underbrace{00\dots 0}_{f(2)+1}1\dots\]
    Проверете какви свойства има функцията $h:\Nat^\Nat \to (0,1)_\Real$ дефинирана като $h(f) = r_f$.
  \item[в) $\rightarrow$ б)]
    Да разгледаме едно реално число $r\in (0,1)_\Real$, където
    \[r = 0, r_0r_1r_2r_3\dots\]
    На това число съпоставяме функцията $f_r:\Nat\to\Nat$ като $f_r(n) = r_n$.
    Проверете какви свойства има функцията $h:(0,1)_\Real \to \Nat^\Nat$ дефинирана като $h(r) = f_r$.
  % \item[б) $\rightarrow$ е)]
  %   Всяко реално число $r$ може да се представи като безкрайна редица от рационални числа $\{q_n\}$,
  %   за която $\lim_n q_n = r$.
  %   Понеже има биекция между рационалните и естествените числа, то на всяко реално $r$ число може да се съпостави
  %   функция $f_r:\Nat\to \mathbb{Q}$, за която $\lim_n f_r(n) = r$.
  %   Следователно, имаме сюрекция от $\Nat^\Nat$ върху $\Real$, т.е. $\abs{\Real} \leq \abs{\Nat^\Nat}$.
  \end{description}
\end{hint}

\begin{problem}
  \marginpar{Озн. $B^A = \{f \mid f:A\to B\}$}
  Нека $A \sim B$ и $C \sim D$. Докажете, че $B^A \sim D^C$.
\end{problem}

% \section*{Библиография}
% \begin{enumerate}[]
% \item 
%   Много задачи могат да се намерят в \cite[§1.4]{lavrov-maksimova}.
% \item
%   Добро изложение може да се намери в \cite[Глава 7]{prove-it}.
% \end{enumerate}

%%% Local Variables: 
%%% mode: latex
%%% TeX-master: "discrete-math"
%%% End: 

\chapter{Доказателства на твърдения}


Ще разгледаме два основни метода за доказателства на твърдения.

\section{Допускане на противното}

Да приемем, че искаме да докажем, че свойството $P(x)$
е вярно за всяко естествено число.
Един начин да направим това е следния:
\begin{itemize}
\item 
  Допускаме, че съществува елемент $n$, за който $\neg P(n)$.
\item
  Използвайки, че $\neg P(n)$ правим извод, от който следва факт, за който знаем, че винаги е лъжа.
  Това означава, че доказваме следното твърдение
  \[\exists x \neg P(x) \rightarrow \mathbf{0}.\]
\item
  Тогава можем да заключим, че $\forall x P(x)$, защото имаме следния извод:
  \begin{prooftree}
    \AxiomC{$\exists x \neg P(x) \rightarrow \mathbf{0}$}
    \UnaryInfC{$\mathbf{1} \rightarrow \neg \exists x \neg P(x)$}
    \UnaryInfC{$\neg \exists x \neg P(x)$}
    \UnaryInfC{$\forall x P(x)$}
  \end{prooftree}
\end{itemize}

Ще илюстрираме този метод на доказателство като решим няколко прости задачи.

\begin{problem}
  \label{prob:even-number-square}
  За всяко $a \in \Int$, ако $a^2$ е четно, то $a$ е четно.
\end{problem}
\begin{proof}
  Твърдението може да се запише като
  \[(\forall a\in\Z)[a^2\mbox{ е четно}\ \rightarrow\ a\mbox{ е четно}].\]
  \marginpar{$\neg (\forall x)(A(x) \rightarrow B(x))$ е еквивалентно на $ \equiv (\exists x)(A(x) \wedge \neg B(x))$}
  Да допуснем противното, т.е.
  \[(\exists a\in\Z)[a^2\mbox{ е четно}\ \wedge\ a\mbox{ не е четно}].\]
  Да вземем едно такова нечетно $a$, за което $a^2$ е четно.
  Това означава, че $a = 2k+1$, за някое $k \in \Z$,
  и \[a^2 = (2k+1)^2 = 4k^2 + 4k + 1,\]
  което очевидно е нечетно число.
  Но ние допуснахме, че $a^2$ е четно.
  Така достигаме до противоречие, следователно нашето допускане е грешно 
  и 
  \[(\forall a\in\Z)[a^2\mbox{ е четно}\ \rightarrow\ a\mbox{ е четно}].\]
\end{proof}


\begin{thm}[Основна теорема на аритметиката]
  \label{th:main-arithmetic}
  Всяко естествено число $n \geq 2$ може да се запише {\bf по единствен начин} като произведение на прости числа.
\end{thm}
\begin{proof}
  Вече знаем от Задача \ref{prob:number-prod-prime}, че всяко число може да се представи като произведение на прости числа.
  Да допуснем, че има такива, които имат няколко различни такива представяния.
  От всички тези числа, нека изберем {\em най-малкото с това свойство}.
  Да го означим с $n$. Имаме, че  
  \[n = p_1p_2\dots p_k = q_1q_2\dots q_m.\]
  Можем да приемем, че простите числа $p_1,\dots,p_k$ и $q_1,\dots,q_m$ са подредени във възходящ ред.
  Без ограничение на общността, нека $p_1 < q_1$.
  Знаем със сигурност, че $p_1 \neq q_1$, защото ако $p_1 = q_1$, то $n' = n/p_1$ 
  е число по-малко от $n$, което има две различни представяния като произведение на прости числа, което е противоречие с избора ни на $n$ - 
  най-малкото такова число.
  
  И така, нека $p_1 < q_1$. Следователно, $p_1 < q_i$, за $i = 1, \dots, m$.
  Тогава съществуват числа $a_i$, $r_i$, за които:
  \[q_i = a_ip_1 + r_i.\]
  Ясно е, че $0 < r_i < p_1 < q_i$, $i = 1,\dots, m$ и 
  \[n' = r_1r_2\dots r_m < q_1q_2\dots q_m = n.\]
  Числото $n'$ може да се представи като произведение на прости числа, като разложим $r_1,\dots,r_m$.
  Знаем, че в това произведение {\em не участва} $p_1$, защото $r_i < p_1$.
  Освен това,
  \[n = q_1q_2\dots q_m = (a_1p_1+r_1)(a_2p_1 + r_2)\dots(a_mp_1+r_m) = A + \underbrace{r_1r_2\dots r_m}_{n'}.\]
  Понеже $p_1 | A$ и $p_1 | n$, то $p_1 | n'$.
  Това означава, че можем да получим друго представяне на $n'$ като произведение на прости числа, в което {\em участва} $p_1$.
  Това е противоречие с минималността на $n$.
\end{proof}

\begin{thm}[Безу]
  Нека $a, b \in \Int$, като поне едно от двете не е $0$.
  Тогава съществуват $x,y \in \Int$, такива че 
  \[xa + yb = \text{НОД}(a,b).\]
\end{thm}
\begin{proof}
  За дадените числа $a, b \in \Int$, да разгледаме множеството
  \[S = \{x \mid x > 0\ \&\ x = ma + nb, \text{ за някои }m,n \in \Int \}.\]
  Лесно се съобразява, че $S \neq \emptyset$.
  
  Да вземем {\em най-малкия елемент} $s \in S$.
  Тогава  $s = ua+vb$.
  Да разгледаме произволен елемент $x \in S$, $x = ma + nb$ и да {\bf допуснем}, че $s$ {\bf не дели} $x$.
  Тогава $x = qs + r$ и $0 < r < s$. Имаме равенствата:
  \begin{align*}
    r & = qs - x\\
      & = qua + qvb - ma - nb\\
      & = a(qu - m) + b(qv - n)
  \end{align*}
  Понеже $r > 0$, то $r \in S$.
  Но от $s > r$ достигаме до {\bf противоречие} с минималността на $s$.
  Следователно, $s$ {\bf дели }$x$.

  Да означим $d = \text{НОД}(a,b)$.
  Вече знаем, че за всяко $x \in S$, $s|x$.
  Понеже $\abs{a}, \abs{b} \in S$, то имаме, че 
  $s$ дели $\abs{a}$, $\abs{b}$, и следователно $1 \leq s \leq d$.
  Но по определение, $d \vert a$ и $d \vert b$.
  Тогава, $d \vert ua$ и $d \vert vb$ и $d \vert (ua+vb) = s$.
  Получаваме, че $d \leq s$.
  Следователно, $\text{НОД}(a,b) = s$.
\end{proof}

\begin{problem}
  Докажете, че
  \[\mbox{НОК}(a,b) = \frac{a.b}{\mbox{НОД}(a,b)}.\]
\end{problem}
\begin{proof}
  Нека $D = \mbox{НОД}(a,b)$.
  Това означава, че
  \[a = Da_1,\ b = Db_1,\ \mbox{НОД}(a_1,b_1) = 1.\]
  Ще докажем, че за всяко $g$, за което $a | g$ и $b | g$,
  то съществува $q_1$, такова че 
  \[g = \frac{ab}{D}q_1.\]
  Да разгледаме $g$, такова че $a | g$ и $b | g$.
  \begin{align*}
    D|a \text{ и } a|g\ \Rightarrow\ & g = Da_1q,\\
    D|b \text{ и } b|g\ \Rightarrow\ & \frac{g}{b} = \frac{Da_1q}{Db_1} = \frac{a_1q}{b_1}.
  \end{align*}
  Понеже $(a_1,b_1) = 1$, то $q = b_1q_1$.
  Тогава $\frac{g}{b} = a_1q_1$ и 
  \[g = \frac{ab}{D}q_1.\]
  Тогава за $q_1 = 1$, 
  \[g = \frac{ab}{D}.\]
\end{proof}

\begin{problem}
  За всеко $a,b \in \Z$ и за всяко просто число $p$,
  ако $p\vert ab$, то $p\vert a$ или $p\vert b$ (може и двете).
\end{problem}
\begin{proof}
  Нека $p \vert ab$. Тогава $ab = kp$.
  Знаем, че $ab = p^{n_1}_1\dots p^{n_m}_m = kp$.
  Тогава $p = p_i$, за някое $i = 1,\dots,m$.
  Следва, че $p$ участва в разлагането на прости множители или на $a$ или на $b$.
\end{proof}


\begin{problem}
  Докажете, че следните числа {\bf не} са рационални:
  \begin{enumerate}[a)]
  \item
    $\sqrt{2},\sqrt{3},\sqrt{6}$;
  \item
    $\sqrt{p}$, където $p$ е просто число;
  \item
    $\sqrt{n}$, където $n$ не е точен квадрат;
  \item
    $\sqrt{pq}$ и $\sqrt{\frac{p}{q}}$, където $p$ и $q$ са различни прости числа;
  \item
    $log_23$.
  \end{enumerate}
\end{problem}
\begin{proof}
  \begin{enumerate}[a)]
  \item
    Да допуснем, че $\sqrt{2}$ е рационално число. Тогава  съществуват $a,b \in \Z$, такива че:
    \[\sqrt{2} = \frac{a}{b}.\]
    Без ограничение, можем да приемем, че $a$ и $b$ са естествени числа,
    които нямат общи делители, т.е. не можем да съкратим дробта $\frac{a}{b}$.
    Получаваме, че \[2b^2 = a^2.\]
    Тогава $a^2$ е четно число и от Задача \ref{prob:even-number-square}, $a$ е четно число.
    Нека $a = 2k$. Получаваме, че
    \[2b^2 = 4k^2,\]
    от което следва, че
    \[b^2 = 2k^2.\]
    Това означава, че $b$ също е четно число, $b = 2n$, за някое $n \in \Z$.
    Следователно, $a$ и $b$ са четни числа и имат общ делител $2$,
    което е противоречие с нашето допускане, че $a$ и $b$ нямат общи делители.
    Така достигаме до противоречие.
    Накрая заключаваме, че $\sqrt{2}$ не е рационално число.
  \item
    Да допуснем, че $\sqrt{p} = \frac{m}{n}$ и $m$ и $n$ са взаимно прости.
    Тогава $n^2p = m^2$. От това следва, че $p | m^2$ и следователно $p | m$.
    Нека $m = kp$. Тогава $n^2p = k^2p^2$ и $n^2 = k^2p$.
    Сега имаме, че $p | n$, но така стигаме до противоречие с факта, че $m$ и $n$
    са взаимно прости.
  \end{enumerate}
\end{proof}

\section{Индукция върху естествените числа}

\marginpar{Да напомним, че естествените числа са $\Nat = \{0,1,2,\dots\}$}
Доказателството с индукция по $\Nat$ представлява следната схема:
\begin{prooftree}
  \AxiomC{$P(0)$}
  \AxiomC{$(\forall x\in\Nat)[P(x)\rightarrow P(x+1)]$}
  \BinaryInfC{$(\forall x\in\Nat) P(x)$}
\end{prooftree}

Това означава, че ако искаме да докажем, че свойството $P(x)$ е вярно за всяко естествено число $x$,
то трябва да докажем първо, че е изпълнено $P(0)$ и след това, за произволно естествено число $x$, ако $P(x)$ вярно, то също така е вярно $P(x+1)$.

\begin{problem}
  \label{prob:number-prod-prime}  
  Всяко естествено число $n \geq 2$ може да се запише като произведение на прости числа.
\end{problem}
\begin{proof}
  Доказателството протича с индукция по $n \geq 2$.
  \begin{enumerate}[a)]
  \item 
    За $n = 2$  е ясно.
  \item
    Ако $n+1$ е просто число, то всичко е ясно.
    Ако $n+1$ е съставно, то \[n + 1 = n_1\cdot n_2.\]
    Тогава $n_1 = p^{n_1}_1\cdots p^{n_k}_k$ и $n_2 = q^{m_1}_1\cdots q^{m_r}_r$,
    където $p_1,\dots,p_k$ и $q_1,\dots,q_r$ са прости числа.
    Тогава е ясно, че $n+1$ също е произведение на прости числа.
  \end{enumerate}
\end{proof}

\begin{problem}
  Докажете, че за всяко $n$, 
  \[\sum^n_{i=0} 2^i = 2^{n+1} - 1.\]
\end{problem}
\begin{proof}
  Доказателството протича с индукция по $n$.
  \begin{itemize}
  \item 
    За $n = 0$, $\sum^0_{i=0}2^i = 1 = 2^{1} - 1$.
  \item
    Нека твърдението е вярно за $n$.
    Ще докажем, че твърдението е вярно за $n+1$.
    \begin{align*}
      \sum^{n+1}_{i=0} 2^i & = \sum^{n}_{i=0}2^i + 2^{n+1}\\
      & = 2^{n+1} - 1 + 2^{n+1} & (\text{от И.П.})\\
      & = 2.2^{n+1} - 1 \\
      & = 2^{(n+1)+1} - 1.
    \end{align*}
  \end{itemize}
\end{proof}

\begin{problem}
  Докажете, че:
  \begin{enumerate}[a)]
  \item
    $3^n$ е нечетно;
  \item
    $n < 2^n$;
  \item
    $2^n < n!$ за $n \geq 4$;
  \item
    \marginpar{$a\vert b\ \iff\ (\exists c\in\Nat)(b = c\cdot a)$}
    $3 \vert (n^3 - n)$;
  \item
    $6 \vert (n^3 + 11n)$;
  \item
    $9 \vert (2^{2n} + 15n - 1)$;
  \item
    $57 \vert (7^{n+2} + 8^{2n+1})$;
  \item
    % \marginpar{$\abs{A}$ - брой елементи на $A$}
    % \marginpar{$\abs{\Ps(A)}$ - брой на подмножествата на $A$}
    за всяко крайно множество $A$,
    ако $\abs{A} = n$, то $\abs{\Ps(A)} = 2^n$;
    \marginpar{$\Ps(A) = \{B\mid B\subseteq A\}$}
  \item
    $C\setminus \bigcup^n_{i=0}A_i = \bigcap^n_{i=0}(C\setminus A_i)$;
  \item
    $C\setminus \bigcap^n_{i=0}A_i = \bigcup^n_{i=0}(C\setminus A_i)$;
  \item
    ако $(\forall i \leq n)[A_i \subseteq B_i]$, то
    $\bigcap^n_{i=0}A_i \subseteq \bigcap^n_{i=0}B_i$;
  \item
    ако $(\forall i \leq n)[A_i \subseteq B_i]$, то
    $\bigcup^n_{i=0}A_i \subseteq \bigcup^n_{i=0}B_i$;
  \item
    $(\bigcap^n_{i=0} A_i)\cup B = \bigcap^n_{i=0} (A_i\cup B)$;
  \item
    $(\bigcup^n_{i=0} A_i)\cap B = \bigcup^n_{i=0} (A_i\cap B)$;
  \item
    $\bigcap^n_{i=0} (A_i\setminus B) = (\bigcap^n_{i=0} A_i) \setminus B$;
  \item
    $\neg (p_1\vee p_2\vee\dots\vee p_n) \iff (\neg p_1 \wedge \neg p_2 \wedge\dots\wedge \neg p_n)$;
  \item
    $\neg (p_1\wedge p_2\wedge\dots\wedge p_n) \iff (\neg p_1 \vee \neg p_2 \vee \dots \vee \neg p_n)$;

  \item
    $\sum^n_{i=0} ar^i = \frac{ar^{n+1}-a}{r-1}$ за $r \neq 1$;
  \item
    $\sum^n_{i=1}i = \frac{n(n+1)}{2}$;
  \item
    $\sum^n_{i=1}i^2 = \frac{n(n+1)(2n+1)}{6}$;
  \item
    $\sum^n_{i=1}i^3 = \frac{n^2(n+1)^2}{4}$;
  \item
    $\sum^n_{i=1}i^4 = \frac{n(n+1)(2n+1)(3n^2+3n-1)}{30}$;
  \item
    $\sum^n_{i=0}(2i+1)^2 = \frac{(n+1)(2n+1)(2n+3)}{3}$;
  \item
    $\sum^n_{i=1}\frac{1}{i(i+1)} = \frac{n}{n+1}$;
  \item
    $\sum^n_{i=0}(-\frac{1}{2})^i = \frac{2^{n+1}+(-1)^n}{3\cdot 2^n}$;
  \end{enumerate}
\end{problem}

%  \begin{problem}
%   \begin{enumerate}
%   \item
%     за $n > 1$ е изпълнено
%     \[\frac{1}{n+1} + \frac{1}{n+2} + \cdots + \cdots \frac{1}{2n} > \frac{13}{24};\]
%   \item
%     за произволни реални числа $a_1,\dots,a_k \geq 0$,
%     \[\sqrt[k]{a_1\cdots a_k}\leq \frac{a_1+\cdots +a_k}{k};\]
%   \item
%     за произволни реални числа $a_1,\dots,a_n \geq 0$,
%     \[(1+a_1)\cdots (1+a_n) \geq (1+ \sqrt[n]{a_1\cdots a_n}).\]
%   \item
%     $\sum^n_{i=1}\frac{1}{\sqrt{i}} > 2(\sqrt{n+1} - 1)$;
%   \end{enumerate}
% \end{problem}
% \begin{proof}
%   \begin{enumerate}[]
%   \item
%     Първо, доказва се с индукция, че твърдението е вярно за $n = 2k$.
%     Очевидно е вярно за $n = 2$.
%     Да допуснем, че $\sqrt[k]{\prod^{k}_{i=1} a_i}\leq \frac{\sum^{k}_{i=1} a_i}{k}$.
%     Ще докажем, че $\sqrt[2k]{\prod^{2k}_{i=1} a_i}\leq \frac{\sum^{2k}_{i=1} a_i}{2k}$.
%     \[
%     \begin{array}{lll}
%       \sqrt[2k]{\prod^{2k}_{i=1} a_i} & = & \sqrt[2]{\sqrt[k]{\prod^{k}_{i=1} a_i}\sqrt[k]{\prod^{2k}_{i=k+1} a_i}}\\
%       & \leq & \frac{\sqrt[k]{\prod^{k}_{i=1} a_i} + \sqrt[k]{\prod^{2k}_{i=k+1} a_i}}{2} \\
%       & \leq & \frac{\frac{\sum^{k}_{i=1} a_i}{k} + \frac{\sum^{2k}_{i=k+1} a_i}{k}}{2}\\
%       & = & \frac{\sum^{2k}_{i=1} a_i}{2k}\\
%     \end{array}
%     \]
    
%     Сега, ще докажем, че ако твърдението е вярно за $n = k$, то е вярно и за $n = k-1$.
%     Нека \[\sqrt[k]{\prod^{k}_{i=1}a_i} \leq \frac{\sum^{k}_{i=1} a_i}{k}.\]
%     Това е вярно за произволни $a_1,\dots,a_k$.
%     Нека да изберем $a_k$, така че
%     \[\frac{\sum^{k}_{i=1} a_i}{k} = \frac{\sum^{k-1}_{i=1} a_i}{k-1},\] т.е.
%     \[a_k = \frac{\sum^{k-1}_{i=1} a_i}{k-1}.\]
%     Получаваме, че:
%     \[
%     \begin{array}{rll}
%       \sqrt[k]{\prod^{k}_{i=1}a_i} & \leq & \frac{\sum^{k}_{i=1} a_i}{k}\\
%       \frac{(\prod^{k-1}_{i=1}a_i)(\sum^{k-1}_{i=1}a_i)}{k-1} & \leq & (\frac{\sum^{k-1}_{i=1}a_i}{k-1})^{k} \\
%       \prod^{k-1}_{i=1}a_i & \leq & (\frac{\sum^{k-1}_{i=1}a_i}{k-1})^{k-1}\\
%       \sqrt[k-1]{\prod^{k-1}_{i=1}a_i} & \leq & \frac{\sum^{k-1}_{i=1}a_i}{k-1}\\
%     \end{array}
%     \]
%   \end{enumerate}
% \end{proof}

% \begin{problem}
%   \marginpar{Нютонов бином}
%   Нека положим 
%   \[\binom{n}{m} = \frac{n!}{(n-i)!i!}\]
%   Проверете:
%   \begin{enumerate}[a)]
%   \item 
%     $2^n = \sum^n_{i=0}\binom{n}{i}$;
%   \item
%     $(x+y)^n = \sum^n_{i=0} \binom{n}{i}x^{n-i}y^{i}$;
%   \item
%     $\binom{n+1}{m+1} = \binom{n}{m} + \binom{n}{m+1}$;
%   \end{enumerate}
% \end{problem}

% \begin{problem}
%   \marginpar{Наричат се хармонични числа}
%   Нека да положим
%   $H_n = \sum^n_{i=1}\frac{1}{i}$.
%   Проверете:
%   \begin{enumerate}[a)]
%   \item
%     $\sum^n_{i=1}H_i = (n+1)H_n - n$.
%   \item
%     $H_{2^k} \geq 1+ k/2$, за всяко $k \geq 0$.
%   \end{enumerate}
% \end{problem}

\begin{problem}
  Докажете, че за всяко естествено число $k$,
  \[(2k+1)^4 \equiv 1\ (\bmod\ 4).\]
\end{problem}

\begin{problem}
  Докажете, че за произволни числа $x,y$ и естествено число $n$ е изпълнено равенството:
  \[(x+y)^n = \sum^{n}_{i=0}\binom{n}{i}x^iy^{n-i}.\]
\end{problem}
\begin{problem}[Теорема на Ферма]
  Нека $p$ е просто число. Тогава докажете, че за числото $a$:
  \begin{enumerate}[i)]
  \item
    $a^p \equiv a\ (\bmod\ p)$;
  \item 
    $a^{p-1} \equiv 1\ (\bmod\ p)$, ако $a$ не се дели на $p$.
  \end{enumerate}
\end{problem}
\begin{proof}
  Да разгледаме следното равенство:
  \[(x+1)^p = x^p + \binom{p}{1}x^{p-1} + \binom{p}{2}x^{p-2} + \dots + 1 = \sum^{p}_{i=0}\binom{p}{i}x^i\]
  За $i = 1,\dots,p-1$, всяко от числата $\binom{p}{i}$ се дели на $p$, то 
  \[(x+1)^p \equiv x^p + 1\ (\bmod\ p).\]
  Нека сега да разгледаме следната редица, която от се получава от горното равенство за $x = a-1,a-2,\dots,2$:
  \begin{align*}
    a^p & \equiv (a-1)^p+1\ (\bmod\ p)\\
    (a-1)^p & \equiv (a-2)^p+1\ (\bmod\ p)\\
    \dots & \dots\dots\\
    2^p & \equiv 1\ (\bmod\ p).
  \end{align*}
  Като съберем тези сравнения, получаваме:
  \[a^p \equiv a\ (\bmod\ p).\]
  Ако $a$ не се дели на $p$, то
  \[a^{p-1} \equiv 1\ (\bmod\ p).\]
\end{proof}

\begin{problem}
  \marginpar{Числа на Фибоначи}
  Да определим следната редица:
  \[F_0 = 0,F_1 = 1,\dots,F_{n+2} = F_{n} + F_{n+1}.\]
  Проверете:
  \begin{enumerate}[a)]
  \item
    $\sum^n_{i=0} F^2_i = F_{n}F_{n+1}$;
  \item
    $\sum^n_{i=1} F_{2i-1} = F_{2n}$;
  \item
    $\sum^{2n}_{i=1}F_{i-1}F_{i} = F^2_{2n}$;
  \item
    единствено членовете от вида $F_{3n}$ са четни;
  \item
    за $n > 0$, $F_{n+1}F_{n-1} - F^2_n = (-1)^n$;
  \item
    $F_{m+n} = F_{m-1}\cdot F_{n} + F_m \cdot F_{n-1}$;
  \item
    ако $m\vert n$, то $F_m \vert F_n$.
  \item
    \marginpar{Използвайте, че $\phi^2 = \phi + 1$}
    ако $n\geq 3$, то $F_n > \phi^{n-2}$,
    където $\phi = \frac{1+\sqrt{5}}{2}$.
  \end{enumerate}
\end{problem}

\newpage 
\subsection*{Пълна индукция върху $\Nat$}

Доказателство с пълна индукция по $\Nat$ за свойството $P$ представлява следната схема:
\begin{prooftree}
  \AxiomC{$(\forall x\in\Nat)[(\forall y\in \Nat)[y < x\ \rightarrow P(y)]\rightarrow P(x)]$}
  \UnaryInfC{$(\forall x\in\Nat) P(x)$}
\end{prooftree}
Нека да проверим принципа за пълна индукция.
Да допуснем, че принципът не е верен, т.е. за някое свойство $P$ е изпълнено, че
\[(\forall x\in\Nat)[(\forall y\in \Nat)[y < x\ \rightarrow P(y)]\rightarrow P(x)]\ \wedge\ (\exists x\in\Nat) \neg P(x).\]
Да вземем най-малкия елемент $n_0$, за който $\neg P(n_0)$. От нашето допускане знаем, че такова $n_0$ съществува.
Тогава \[(\forall y\in \Nat)[y < n_0\ \rightarrow P(y)]\]
и следователно:
\begin{prooftree}
  \AxiomC{$(\forall y\in \Nat)[y < n_0\ \rightarrow P(y)]$}
  \AxiomC{$(\forall x\in\Nat)[(\forall y\in \Nat)[y < x\ \rightarrow P(y)]\rightarrow P(x)]$}
  \UnaryInfC{$(\forall y\in \Nat)[y < n_0\ \rightarrow P(y)]\rightarrow P(n_0)$}
  \BinaryInfC{$P(n_0)$}
\end{prooftree}
Така достигаме до противоречие, защото получаваме, че $P(n_0)\wedge \neg P(n_0)$.

\begin{prop}
  Двата принципа са еквивалентни.
\end{prop}
\begin{proof}
  $1)\to 2)$.
  Нека имаме $1)$ и нека $(\forall n\in\Nat)[(\forall m \in\Nat)[m < n \rightarrow P(m)]\rightarrow P(n)]$.
  Ще докажем, че $(\forall n \in \Nat)[P(n)]$.
  Да рагледаме свойството \[Q(n) = P(0)\ \wedge\ \dots\ \wedge\ P(n).\]
  Тогава имаме, че $Q(0)$ и $(\forall n\in\Nat)[Q(n) \rightarrow P(n+1)]$.
  Но от $Q(n) \rightarrow P(n+1)$ следва, че $Q(n) \rightarrow Q(n) \wedge P(n+1)$, т.е.
  $(\forall n\in\Nat)[Q(n)\rightarrow Q(n+1)]$.
  Така полуваме по математическата индукция, че $(\forall n\in\Nat)[Q(n)]$.
  Но тогава е очевидно, че имаме $(\forall n\in\Nat)[P(n)]$,
  защото $Q(n) \rightarrow P(n)$.

  $2) \to 1)$.
  Нека имаме $2)$ и $P(0)\ \wedge\ (\forall n\in\Nat)[P(n)\rightarrow P(n+1)]$.
  Ще докажем, че $(\forall n \in \Nat)[P(n)]$.
  Понеже имаме 2), достатъчно е да докажем
  \[(\forall n\in\Nat)[(\forall m \in\Nat)[m < n \rightarrow P(m)]\rightarrow P(n)].\]
  \begin{itemize}
  \item 
    Ако $n = 0$, то е ясно, че
    $(\forall m \in\Nat)[m < 0 \rightarrow P(m)]\rightarrow P(0)$.
  \item
    Ако $n > 0$, то от $P(n)\rightarrow P(n+1)$ получваме, че 
    \[P(0)\ \wedge\ P(1)\ \wedge\ \dots\ P(n) \to P(n+1),\] т.е.
    $(\forall m \in\Nat)[m < n \rightarrow P(m)]\rightarrow P(n)$.
  \end{itemize}
  Обединявайки двата случая получаваме, че 
  \[(\forall n\in\Nat)[(\forall m \in\Nat)[m < n \rightarrow P(m)]\rightarrow P(n)].\]
  И тогава от пълната математичска индукция следва, че $(\forall n \in \Nat)[P(n)]$
\end{proof}

% \begin{prop}
%   Двете форми на индукция са еквивалентни.
% \end{prop}
% \begin{proof}
%   \begin{enumerate}
%   \item 
%     Нека имаме схемата за ``обикновена'' индукция.
%     Ще докажем, че тогава имаме схемата за пълна индукция.
%     Нека \[(\forall x\in\Nat)[(\forall y\in \Nat)[y < x\ \rightarrow P(y)]\rightarrow P(x)].\]
%     Ще докажем с ``обикновена'' индукция, че $(\forall x\in\Nat)P(x)$.
    
%     Да положим $Q(x) \equiv (\forall y\in\Nat)[y<x \rightarrow P(y)]$.
%     Очевидно е, че $Q(0)$ е изпълнено.
%     Освен това, 
%     \begin{prooftree}
%       \AxiomC{$(\forall x\in\Nat)[(\forall y\in \Nat)[y < x\ \rightarrow P(y)]\rightarrow P(x)]$}
%       \UnaryInfC{$(\forall x\in\Nat)[Q(x)\rightarrow P(x)]$}
%       \UnaryInfC{$(\forall x\in\Nat)[Q(x)\rightarrow Q(x)\wedge P(x)]$}
%       \AxiomC{$(\forall x\in\Nat)[Q(x)\wedge P(x)\ \rightarrow\ Q(x+1)]$}
%       \BinaryInfC{$(\forall x\in\Nat)[Q(x)\rightarrow Q(x+1)]$}
%     \end{prooftree}
%     Получаваме, че 
%     \begin{prooftree}
%       \AxiomC{$Q(0)$}
%       \AxiomC{$(\forall x\in\Nat)[Q(x)\rightarrow Q(x+1)]$}
%       \BinaryInfC{$(\forall x\in\Nat)Q(x)$}
%       \AxiomC{$(\forall x\in\Nat)Q(x)\ \rightarrow\ (\forall x\in\Nat)P(x)$}
%       \BinaryInfC{$(\forall x\in\Nat)P(x)$}
%     \end{prooftree}
%   \item
%     Нека сега имаме схемата за пълна индукция.
%     Ще докажем, че имаме и схемата за ``обикновена'' индукция.
%     За тази цел, нека имаме, че $P(0)$ и $(\forall x\in\Nat)[P(x) \rightarrow P(x+1)]$.
%     Достатъчно е да докажем, че
%     \[(\forall x\in\Nat)[(\forall y\in \Nat)[y < x \rightarrow P(y)] \rightarrow P(x)],\]
%     защото тогава ще приложим пълна индукция и ще получим, че \[(\forall x\in\Nat)P(x).\]
    
%     Да допуснем противното, т.е.
%     \[(\exists x\in\Nat)[(\forall y\in \Nat)[y < x \rightarrow P(y)] \wedge \neg P(x)].\]
%     Да разгледаме едно такова $x_0$, за което
%     \[(\forall y\in \Nat)[y < x_0 \rightarrow P(y)] \wedge \neg P(x_0).\]
%     \begin{itemize}
%     \item 
%       Ако $x_0 = 0$, то очевидно е изпълнено, че $(\forall y\in\Nat)[y < 0 \rightarrow P(y)]$.
%       Тогава $\neg P(0)$, което е противоречие.
%     \item
%       Ако $x_0 > 0$, тогава от
%       $(\forall y\in\Nat)[y < x_0 \rightarrow P(y)]\ \rightarrow\ P(x_0-1)$
%       следва, че $P(x_0-1)$.
%       Тогава 
%       \begin{prooftree}
%         \AxiomC{$(\forall x)[P(x)\rightarrow P(x+1)]$}
%         \AxiomC{$P(x_0-1)$}
%         \BinaryInfC{$P(x_0)$}
%       \end{prooftree}
%       Отново достигаме до противоречие.
%     \end{itemize}
%     Следователно нашето допускане е невярно и
%     \[(\forall x\in\Nat)[(\forall y\in \Nat)[y < x \rightarrow P(y)] \rightarrow P(x)].\]
%   \end{enumerate}
% \end{proof}

\begin{problem}
  Докажете, че за всяко $x,y\in\Nat$
  \[f(x,y) = x^y,\]
  където
  \begin{align*}
    f(x,y) = 
    \begin{cases}
      1, & x\neq 0\ \wedge\ y = 0\\
      f(x,y-1) * x, & x\neq 0\ \wedge y\mbox{ е нечетно}\\
      f(x,y/2) * f(x,y/2), & x\neq 0\ \wedge y\mbox{ е четно}
    \end{cases}
  \end{align*}
\end{problem}


% \begin{problem}
%   Нека фунцкията $f:\Nat\to\Nat$ е определена като
%   \begin{align*}
%     f(x) = x-10, & x > 100\\
%     f(x) = f(f(x+11)), & x < 100.
%   \end{align*}
%   Докажете, че $(\forall x \leq 100)[f(x) = 91]$.
% \end{problem}

\subsection*{Индукция върху $\Nat\times\Nat$}

\begin{dfn}
  Определяме лексикографската наредба $\prec$ върху $\Nat\times\Nat$ като
  \[\pair{x,y} \prec \pair{x^\prime,y^\prime}\ \iff\ x < x^\prime \vee (x = x^\prime\ \wedge\ y < y^\prime).\]
  Наричаме двойката $\pair{x_0,y_0}$ {\em минимална} за множеството $A \subseteq \Nat\times\Nat$, ако
  \[\pair{x_0,y_0}\in A\ \wedge\ (\forall \pair{x,y}\in A)[\pair{x,y}\not\prec\pair{x_0,y_0}].\]
\end{dfn}

\begin{prop}
  Всяко непразно подмножество $A\subseteq \Nat\times \Nat$ притежава поне един {\em минимален} елемент.
\end{prop}

\begin{prop}
  Не съществуват безкрайни строго намаляващи редици относно $\succ$ в $\Nat\times\Nat$, т.е.
  не съществува 
  \[\pair{x_0,y_0} \succ \pair{x_1,y_1}\succ \pair{x_2,y_2} \succ \dots \succ \pair{x_n,y_n}\succ\dots\]
\end{prop}

\begin{dfn}
  Доказателството с индукция върху $\Nat\times\Nat$ представлява следната схема:
  \begin{prooftree}
    \AxiomC{$(\forall\pair{x,y})[(\forall \pair{x^\prime,y^\prime})[\pair{x^\prime,y^\prime} \prec \pair{x,y}\ \rightarrow\ P(x^\prime,y^\prime)]\ \rightarrow P(x,y)]$}
    \UnaryInfC{$\forall \pair{x,y} P(x,y)$}
  \end{prooftree}  
\end{dfn}

Да проверим схемата.
Да допуснем, че тя не е вярна, т.е. за някое свойство $P$ е изпълнено, че
\[(\forall\pair{x,y})[(\forall \pair{x^\prime,y^\prime})[\pair{x^\prime,y^\prime} \prec \pair{x,y}\ \rightarrow\ P(x^\prime,y^\prime)]\ \rightarrow P(x,y)],\]
но \[\exists \pair{x,y} \neg P(x,y),\]
т.е. съществува $\pair{x,y} \in \Nat\times\Nat$, за което $\neg P(x,y)$.
Да разгледаме
\[A = \{\pair{x,y}\in \Nat\times\Nat\mid \neg P(x,y)\}.\]
Щом $A$ е непразно, то $A$ има минимален елемент $\pair{x_0,y_0}$.
Тогава
\[(\forall \pair{x^\prime,y^\prime})[\pair{x^\prime,y^\prime} \prec\pair{x_0,y_0}\ \rightarrow P(x^\prime,y^\prime)].\]
Но ние имаме, че 
\[(\forall \pair{x^\prime,y^\prime})[\pair{x^\prime,y^\prime} \prec \pair{x_0,y_0}\ \rightarrow\ P(x^\prime,y^\prime)]\ \rightarrow P(x_0,y_0).\]
Това означава, че $P(x_0,y_0)$, което е противоречие.

\begin{remark}
  За да докажем едно свойство $P$ с индукция по лексикографската наредба върху $\Nat\times\Nat$,
  първо доказваме $P$ за минималната двойка $\pair{0,0}$.
  След това доказваме, че ако $P$ е вярно за всички двойки $\pair{x^\prime,y^\prime}\prec \pair{x,y}$,
  то $P$ е вярно и за $\pair{x,y}$.
\end{remark}

\begin{problem}
  Докажете,  че $f(x,y) = \abs{x-y}$, където
  \begin{align*}
    f(x,y) = 
    \begin{cases}
      y, & x = 0\\
      x, & y = 0\\
      f(x-1,y-1), & \mbox{ иначе}
    \end{cases}
  \end{align*}
\end{problem}
\begin{proof}
  Индукция по $(\Nat^2,\prec)$, където $\prec$ е лексикографската наредба.
  Имаме един минимален елемент $(0,0)$.
  \[f(0,0) = 0 = \abs{0 - 0}.\]
  Да допуснем, че за всяко $(u,v) \prec (x,y)$, 
  \[f(u,v) = \abs{u - v}.\]
  Тогава ако $x > 0, y = 0$, то
  \[f(x,0) = x = \abs{x - 0}.\]
  Ако $x> 0, y > 0$, то
  \[f(x,y) = f(x-1,y-1) = \abs{x-1-y+1} =\abs{x-y}.\]
\end{proof}

\begin{problem}
  Докажете, че $f(x,y) = \mbox{НОД}(x,y)$, където
  за $x,y \in \Nat$,
  \begin{align*}
    f(x,y) = 
    \begin{cases}
      f(x-y,y), & x > y\\
      f(y,x), & x < y\\
      x, & x = y.
    \end{cases}
  \end{align*}
\end{problem}


\newpage
\subsection*{Фундирани множества}
\marginpar{Англ. well-founded sets}

Понякога се налага да правим индукция по по-сложни множества от това на естествените
числа.

\begin{dfn}
  Нека е дадена двойката $(A,R)$, където $A$ е множество, а $R\subseteq A^2$.
  Казваме, че $A$ е фундирано множество относно $R$, ако 
  \begin{itemize}
  \item 
    \marginpar{$R$ задава строга частична наредба върху $A$}
    $R$ е анти-рефлексивна, транзитивна, асиметрична.
  \item
    ако всяко непразно подмножество $X\subseteq A$ притежава поне един {\em минимален} елемент, т.е.
    \[(\forall X\subseteq A)[X\neq\emptyset \rightarrow (\exists m\in X)\neg(\exists y\in X)[\pair{y,m} \in R]].\]
  \end{itemize}
\end{dfn}

Обърнете внимание, че минималният елемент може да не е уникален.
\begin{example}
  Нека да определим $\prec$ върху $\N$ като
  \[\pair{x,y}\prec\pair{x^\prime,y^\prime}\ \iff\ x < y\ \wedge\ x^\prime < y^\prime.\]
  Тогава например $X = \{\pair{1,2},\pair{2,1},\pair{2,2},\pair{2,3}\}$
  има два минимални елемента - $\pair{1,2}$ и $\pair{2,1}$.
\end{example}


\begin{prop}
  Нека $\prec$ е строга частична наредба върху $A$.
  Следните твърдения са  еквивалентни:
  \begin{enumerate}[a)]
  \item
    ако всяко непразно подмножество $X\subseteq A$ притежава поне един {\em минимален} елемент, т.е.
    \[(\forall X\subseteq A)[X\neq\emptyset \rightarrow (\exists x\in X)\neg(\exists y\in X)[y \prec x]];\]
  \item
    не съществуват безкрайни редици от вида
    \[x_0 \succ x_1 \succ x_2 \succ \cdots \succ x_n \succ \cdots\]
  \end{enumerate}
\end{prop}
\begin{proof}
  \begin{enumerate}
  \item[а)$\ \to\ $б)]
    Да допуснем, че съществува безкрайно-намаляваща редица
    \[x_0 \succ x_1 \succ x_2 \succ \cdots \succ x_n \succ \cdots\]
    Нека $X = \{x_i\mid i \in \Nat\}$.
    Тогава лесно се вижда, че в $X$ няма минимален елемент, което е противоречие.
  \item[б)$\ \to\ $а)]
    Да допуснем, че съществува непразно множество $X \subseteq A$, което не притежава минимален елемент, т.е.
    \[(\forall x\in X)(\exists y\in X)[y \prec x].\]
    Ще построим безкрайно-намаляваща редица относно $\prec$.
    Да вземем произволен $x_0 \in X$. 
    Знаем, че съществува $y \in X, x_0 \prec y$.
    Нека да изберем едно такова $y\in X$ и да означим $x_1 = y$.
    По този начин можем да построим
    \[x_0 \prec x_1 \prec x_2 \prec \cdots \]
  \end{enumerate}
\end{proof}

\begin{prop}
  Нека $(A_1,\prec_1)$ и $(A_2,\prec_2)$ са фундирани.
  Тогава \[(A_1\times A_2, \prec)\]
  е фундирано множество, където
  \[\pair{a_1,a_2}\prec \pair{a^\prime_1,a^\prime_2}\ \iff\ a_1\prec_1 a^\prime_1\ \vee\ (a_1 = a^\prime_1\ \wedge\ a_2\prec_2 a^\prime_2)\]
\end{prop}
\begin{proof}
  Да допуснем, че съществува
  безкрайно намаляваща редица относно $\prec$:
  \[(x_0,y_0)\succ(x_1,y_1) \succ \cdots \succ (x_n,y_n)\succ\cdots\]
  Да разгледаме редицата само от първите компоненти :
  \[x_0 \succeq x_1 \succeq \cdots \succeq x_n \succeq \cdots\]
  Това означава, че съществува число $n_1$, такова че 
  \[(\forall k \geq n_1)[x_{n_1} = x_k].\]
  В противен случай ще получим безкрайно намаляваща редица, което ще бъде
  противоречие с фундираността на $A_1$.
  Аналогично, съществува $n_2$, такова че
  \[(\forall k \geq n_2)[y_{n_2} = y_k].\]
  В противен случай ще получим безкрайно намаляваща редица, което ще бъде
  противоречие с фундираността на $A_2$.
  Нека \[n = \max(n_1,n_2).\]
  Тогава 
  \[(\forall k\geq n)[(x_n,y_n) = (x_k,y_k)].\]
  Така достигаме до противоречие с 
  \[(\forall k \geq n)[(x_n,y_n) \succ (x_k,y_k)].\]
  Следователно $\prec$ задава фундирана наредба върху $A_1\times A_2$.
\end{proof}



\subsection*{Индукция по фундирани наредби}

Доказателството с индукция по фундираното множество $(A,\prec)$ представлява следната схема:
\begin{prooftree}
  \AxiomC{$(\forall x \in A)[(\forall y\in A)[y \prec x\ \rightarrow\ P(y)]\ \rightarrow P(x)]$}
  \UnaryInfC{$(\forall x \in A) P(x)$}
\end{prooftree}
Ако допуснем, че съществува $x \in A$, за което $\neg P(x)$, то да разгледаме
\[X = \{x\in A\mid \neg P(x)\}.\]
Щом това множество е непразно, то $X$ има поне един минимален елемент $x_0$.
Тогава
\[(\forall y\in A)[y \prec x_0\ \rightarrow P(y)].\]
Но това означава, че $P(x_0)$, което е противоречие.

\begin{problem}
  Докажете,  че $f(x,y) = \abs{x-y}$, където
  \begin{align*}
    f(x,y) = 
    \begin{cases}
      y, & x = 0\\
      x, & y = 0\\
      f(x-1,y-1), & \mbox{ иначе}
    \end{cases}
  \end{align*}
\end{problem}
\begin{proof}
  Индукция по $(\Nat^2,\prec)$, където $\prec$ е лексикографската наредба.
  Имаме един минимален елемент $(0,0)$.
  \[f(0,0) = 0 = \abs{0 - 0}.\]
  Да допуснем, че за всяко $(u,v) \prec (x,y)$, 
  \[f(u,v) = \abs{u - v}.\]
  Тогава ако $x > 0, y = 0$, то
  \[f(x,0) = x = \abs{x - 0}.\]
  Ако $x> 0, y > 0$, то
  \[f(x,y) = f(x-1,y-1) = \abs{x-1-y+1} =\abs{x-y}.\]
\end{proof}

\begin{problem}
  Докажете, че $f(x,y) = \mbox{НОД}(x,y)$, където
  за $x,y \in \Nat$,
  \begin{align*}
    f(x,y) = 
    \begin{cases}
      f(x-y,y), & x > y\\
      f(y,x), & x < y\\
      x, & x = y.
    \end{cases}
  \end{align*}
\end{problem}

\begin{problem}
  Докажете, че $f(x,y) = \binom{x}{y}$, където
  за $x \geq y, x,y\in\Nat$,
  \begin{align*}
    f(x,y) = 
    \begin{cases}
      1, & x = 0\ \vee\ y = 0\ \vee\ x = y\\
      f(x-1,y) + f(x-1,y-1), & \mbox{ иначе}
    \end{cases}
  \end{align*}
\end{problem}

% \begin{problem}
%   \begin{align*}
%     f(x,y) = 
%     \begin{cases}
%       y+1, & x = 0\\
%       f(x-1,1), & x > 0\ \wedge\ y = 0\\
%       f(x-1,f(x,y-1), & x > 0\ \wedge\ y > 0.
%     \end{cases}
%   \end{align*}
%   Докажете, че $f$ е тотална функция.

% \end{problem}


% \begin{problem}
%   Докажете, че за всеки $m,n\in\Nat$
%   съществуват $p,q\in\Z$, такива че
%   \[p\cdot m +  q\cdot n = \mbox{НОД}(m,n).\]
% \end{problem}

$\mbox{НОК}(x,y) = z$ точно тогава, когато
$z$ е най-малкото число, за което е изпълнено свойството $x | z\ \&\ y | z$.



%%% Local Variables: 
%%% mode: latex
%%% TeX-master: "discrete-math"
%%% End: 

\chapter{Комбинаторика}

\section{Основни понятия}

\begin{description}
\item[(0+R+)]
  {\bf Конфигурации с подредба и с повторение.}
  Също така се наричат пермутации с повторение.
  Това е броят $P_r(n,k)$ на всички думи с дължина $k$ над $n$-елементна азбука.
  \[P_r(n,k) = n^k\]
  С тази формула можем да намираме всички $k$-буквени думи над азбука с $n$ букви.
  Например, всички 4-буквени думи над азбуката $\{a,b,c\}$ са $3^4$ на брой.
  Иначе казано, това са всички начини да изберем по една буква от всяка урна:
  \[
  \left(\begin{array}{c}
      a\\
      b\\
      c\\
      \end{array}
    \right)
  \left(\begin{array}{c}
      a\\
      b\\
      c\\
      \end{array}
    \right)
  \left(\begin{array}{c}
      a\\
      b\\
      c\\
      \end{array}
    \right)
  \left(\begin{array}{c}
      a\\
      b\\
      c\\
      \end{array}
    \right)
  \]
  и ги подреждаме в редица.
\item[(0+R--)]
  \marginpar{Тук $k \leq n$.}
  {\bf Конфигурации с подредба, но без повторение.}
  Съща така се наричат пермутации.
  Това е броят $P(n,k)$ на думите с дължина $k$ над азбука с $n$ букви, като нямаме повторения на буквите.
  \[P(n,k) = n(n-1)\cdots(n-k+1) = \frac{n!}{(n-k)!}.\]
  Например, всички 3-буквени думи {\em без повторения} над азбуката $\Sigma = \{a,b,c,d\}$
  са $4!3!2!$ на брой.
  Как можем да генерираме всички такива 3-буквени думи?
  Започваме с 3 пълни урни:
  \[
  \left(\begin{array}{c}
      a\\
      b\\
      c\\
      d
      \end{array}
    \right)
  \left(\begin{array}{c}
      a\\
      b\\
      c\\
      d
      \end{array}
    \right)
  \left(\begin{array}{c}
      a\\
      b\\
      c\\
      d
      \end{array}
    \right)
  \]
  От първата урна избираме произволен елемент измежду 4-те букви. Например $b$.
  Това ще бъди първият символ на нашата дума. Понеже той не може да се повтаря,
  ние премахваме $b$ от другите урни. Оставаме с втора и трета урна:
  \[
  \left(\begin{array}{c}
      a\\
      c\\
      d
      \end{array}
    \right)
  \left(\begin{array}{c}
      a\\
      c\\
      d
      \end{array}
    \right).
    \]
    От втората урна избираме произволен елемент измежду 3-те останали букви.
    Нека да изберем от втората урна $a$.
    Това означава, че нашата дума ще започва с $ba$.
    Отново, понеже не искаме $a$ да се повтаря, премахваме $a$ от третата урна. Оставаме само с третата урна:
      \[
      \left(\begin{array}{c}
          c\\
          d
        \end{array}
      \right).
    \]
    За третата буква от нашата дума избираме измежду $c$ и $d$.
    Нека да изберем $d$.
    Така генерирахме думата $bad$.
  \item[(0--R--)]
  {\bf Конфигурации без подредба и без повторение.}
  \marginpar{Тук също $k \leq n$.}
  \marginpar{Също така се наричат комбинации}
  Това е броят $C(n,k)$ на $k$-елементните подмножества (т.е. елементите {\em не са подредени}) на едно $n$-елементно множество.
  Имаме следната връзка с пермутации без повторение:
  \[P(n,k) = C(n,k)\cdot P(k,k),\] 
  т.е. за да получим всички думи с дължина $k$ {\em без повторения на буквите},
  можем първо да изберем едно множество от $k$ букви и след това да ги подредим тези $k$ на брой букви в една редица.
  Следователно,
  \marginpar{$\binom{n}{k}$ - чете се $n$ над $k$}
  \[C(n,k) =  \frac{P(n,k)}{P(k,k)} = \frac{n!}{(n-k)!k!} = \binom{n}{k}.\]
  Например, всички $3$ елементни подмножества на $\{1,2,3,4\}$ са
  \[\{1,2,3\},\{1,2,4\},\{1,3,4\},\{2,3,4\}.\]

  Като друг пример, броят на всички комбинации от правилно попълнени фишове в тото 6 от 49 са $\binom{49}{6}$.
  Всеки правилно попълнен фиш еднозначно се определя като множество от 6 елемента измежду числата $\{1,2,\dots,49\}$,
  защото не е важен реда на попълване на числата.
\item[(0-- R+)]
  {\bf Комбинации без подредба и с повторение.}
  Мултимножество е съвкупност от обекти, в които позволяваме повторение на елементи.
  Например, $\{3,1,1,2\}$ е мултимножество и $\{3,1,1,2\} = \{2,1,3,1\}$,
  но $\{3,1,2\} \neq \{3,1,1,2\}$.
  Броят на $n$-елементните мулти-подмножества на едно $k$-елементно множество е:
  \[C(n+k-1,k-1) = \binom{n+k-1}{k-1}.\]
  Нека да видим как можем да достигнем до тази формула като намерим всички 4-елементни мулти-подмножества
  на $\{a,b,c\}$. Ще видим, че на всяко такова мулти-множество можем да съпоставим редица от 6 кутии,
  като в две от тези кутии са отбелязани с $\star$, а в другите кутии са буквите от азбуката, избрани по следния начин - 
  в кутиите до първата $\star$ поставяме $a$; в кутиите между двете $\star$ поставяме $b$; и в кутиите след втората $\star$
  поставяме $c$.
  Например, на следната редица от кутии:
  
  \begin{tabular}{|l|l|l|l|l|l|}
    \hline
    a & a & $\star$ & $\star$ & c & c \\
    \hline
  \end{tabular}  
  съответства мулти-множеството $\{a,a,c,c\}$, а на редицата от кутии:

  \begin{tabular}{|l|l|l|l|l|l|}
    \hline
    a & $\star$ & b & $\star$ & c & c \\
    \hline
  \end{tabular}
  съответства мулти-множеството $\{a,b,c,c\}$
  
  Всяка такава подредба се определя еднозначно от позициите на двете $\star$.
  Следователно, всички мулти-множества са $\binom{4+3-1}{3-1} = \binom{6}{2}$.
\end{description}



\begin{problem}
  Отговорете на следните въпроси:
  \begin{enumerate}[a)]
  \item
    \marginpar{Отг. $2^8$}
    Колко битови низове с дължина един байт има ?
  \item
    Колко са всички подмножества на множеството $A$ с $8$ елемента ?
  \item 
    \marginpar{Отг. $2^5$}
    Колко битови низове с дължина един байт започват с 1 завършват с 00 ?
  \item
    \marginpar{Отг. $62! - 52!$}
    Всеки потребител на една компютърна система има парола, която е дълга между 6 и 8 символа.
    Всеки символ е малка или голяма буква, или цифра.
    Всяка парола трябва да съдържа поне една цифра.
    Колко такива пароли има?
  \item
    \marginpar{Отг. $4!$}
    По колко начина можем да подредим елементите $\{a,b,c,d\}$ ?
  \item 
    \marginpar{Отг. $5!$}
    Колко думи може да се образуват от буквите в $ABCDEFG$, които съдържат $ABC$.
  \item
    \marginpar{Отг. $\binom{11}{1}\binom{10}{4}\binom{6}{4}\binom{2}{2}$}
    Колко различни думи могат да се образуват като разместим буквите на думата $MISSISSIPPI$?
  \item
    Колко различни думи могат да се образуват като разместим буквите на думата $TENNESSEE$?
  \item
    Колко различни думи могат да се образуват като разместим буквите на думата $SUCCESS$?
  \item
    Колко различни думи могат да се образуват като разместим буквите на думата АБРАКАДАБРА?
  \item
    Колко различни думи могат да се образуват като разместим буквите на думата ПЕРПЕРИКОН?
  \item
    \marginpar{Отг. $10\cdot 9\cdot 8$}
    В състезание участват 10 отбора. 
    По колко начина могат да се разпределят златните, сребърните и бронзовите медали?
  \item
    \marginpar{Не искаме числата да започват с нула. Отг. $5! - 4!$}
    Колко различни петцифрени числа могат да се образуват чрез разместване на цифрите от 0,1,2,3,4?
  \item
    \marginpar{Отг. $\binom{8}{1}\binom{7}{3}\binom{4}{4}$}
    По колко различни начина могат да се настанят осем студенти в три стаи съответно с едно, три и четири легла?
  \item
    %\marginpar{Отг. $\binom{n}{1}\binom{n-1}{1}\binom{n-2}{1}\binom{n-3}{1}$}
    По колко различни начина четирима младежи могат да поканят на танц четири от $n$ девойки?
  \item
    %\marginpar{Отг. $\binom{6}{2}\binom{4}{2}\binom{2}{2}$}
    Шест различни предмета се боядисват по следния начин: два зелен, два червен, два син цвят.
    По колко различни начина могат да се боядисат предметите?  
  \item
    По колко различни начина могат да се разпределят 10 специалисти в 4 цеха така, че в тях да попаднат съответно по 1,2,3 и 4 души?
  \item
    \marginpar{Отг. $(n+1)! - n! - n!$}
    Иванчо и $n$ негови приятели отиват на кино.
    По колко различни начина могат всички да седнат заедно на един ред, така че Иванчо е винаги
    между двама негови приятели.
  \item
    \marginpar{Отг. $\binom{m}{k}\binom{N-M}{n-k}$}
    В партида от $N$ изделия, $M$ са бракувани.
    По колко различни начина могат да се вземат от партидата $n$ изделия, така че точно $k$ от тях да бъдат бракувани ($M\leq N, k\leq n\leq N$)?
  \item
    \marginpar{Отг. $\binom{4}{2}\binom{48}{4}$}
    От колода с 52 карти се изваждат 6 произволни карти без връщане.
    По колко различни начина могат да се извадят картите, така че две от тях да са дами?
  \item
    \marginpar{Отг. $\binom{4}{2}\binom{4}{2}\binom{44}{2}$}
    От колода с 52 карти се изваждат 6 произволни карти без връщане.
    По колко различни начина могат да се извадят картите, така че две от тях да са тройки и две осмици?
  \item
    \marginpar{Отг. $\binom{48}{24}\binom{4}{2}$}
    По колко различни начина може да се раздели колода от 52 карти на две пачки от по 26 карти така, че във всяка от тях да има по две дами?
  \item
    \marginpar{Отг. $\binom{8}{2}\binom{6}{2}\binom{4}{2}\binom{2}{2}$}
    По колко начина може да се разпределят 8 подаръка между 4 лица, така че всеки да получи по два подаръка?
  \item
    %\marginpar{Отг. $\binom{40}{1}\binom{39}{1}\binom{38}{5}$}
    Провежда се събрание с $40$ присъстващи.
    По колко начина може да се избере председател, секретар и 5 членна комисия?
  \end{enumerate}
\end{problem}


\begin{problem}
  От колода с $52$ карти се избират $11$. По колко различни начина могат да се изберат извадки, в които се срещат:
  \begin{enumerate}[a)]
  \item
    \marginpar{Отг. $\binom{48}{10}\binom{4}{1}$}
    точно $1$ ас;
  \item
    \marginpar{Отг. $\binom{52}{11} - \binom{48}{11} - \binom{48}{10}\binom{4}{1}$}
    поне $2$ валета;
  \item
    \marginpar{Отг. $\binom{39}{7}\binom{13}{4}$}
    точно $4$ пики;
  \item
    \marginpar{Отг. $\binom{52}{11} + \binom{39}{10}\binom{13}{1} + \binom{39}{9}\binom{13}{2}$}
    най-много $2$ кари;
  \item
    \marginpar{Отг. $\binom{3}{2}\binom{12}{2}\binom{36}{7} + \binom{3}{1}\binom{12}{1}\binom{36}{8}$}
    точно $2$ аса и $2$ точно трефи;
  \item
    точно $2$ аса и не повече от $2$ трефи;
  \end{enumerate}
\end{problem}

\begin{problem}
  \begin{enumerate}[a)]
  \item
%    \marginpar{Отг. $\binom{6}{2} = 15$}
    Колко е максималният брой прави, които могат да се прекарат през 6 точки?
  \item
 %   \marginpar{Отг. $\binom{10}{2} - 2$}
    Колко е максималният брой прави, които могат да се прекарат през 10 точки, три от които лежат на една права? 
  \item
  %  \marginpar{Отг. $\binom{7}{2}$}
    В колко точки се пресичат 7 прави от една равнина, никои три от които не минават през една точка и никои две не са успоредни?
  \item
   % \marginpar{$\binom{10}{2} - (\binom{4}{2} - 1) - \binom{3}{2} = 37$}
    В колко точки се пресичат 10 прави от една равнина, като три от тези прави са успоредни и четири други минават през една и съща точка.
\end{enumerate}
\end{problem}


\begin{problem}
  Да разгледаме азбуката $\Sigma = \{a_1,\dots,a_n\}$.
  Да се намерят всички $k$-буквени думи над азбуката $\Sigma$, за $k \leq n$, където:
  \begin{enumerate}[a)]
  \item
%    \marginpar{Отг. $\frac{n!}{(n-k)!}$}
    нито една буква не се повтаря;
  \item
 %   \marginpar{Отг. $n^{\lceil{\frac{k}{2}}\rceil}$}
    са симетрични;
  \item
  %  \marginpar{Отг. $n(n-1)^{k-1}$}
    нямат две последователни еднакви букви;
  \item
%    \marginpar{Отг. $n^k - n(n-1)^{k-1}$}
    имат две последователни еднакви букви;
  \item
 %   \marginpar{Отг. $\binom{k}{2}n\cdot\frac{(n-1)!}{(n-1-(k-2))!}$}
    съществува само една буква, която се среща точно два пъти;
  \item
%    \marginpar{Отг. $n^k - \frac{n!}{(n-k)!}$}
    съществува буква, която се повтаря;
  \item
%    \marginpar{Отг. $\binom{k}{3}n(n-1)^{k-3}$}
    съществува буква, която се среща точно три пъти;
  \item
 %   \marginpar{Отг. $2\binom{k}{2}(n-2)^{k-2}$}
    буквите $a, b\in \Sigma$ се срещат точно по веднъж;
  \end{enumerate}
\end{problem}



% \begin{problem}
%   Докажете, че:
%   \begin{enumerate}[a)]
%   % \item
%   %   $(x+y)^n = \sum^{n}_{i=0}\binom{n}{i}x^iy^{n-i}$;
%   \item
%     $2^n = \sum^n_{k=0}\binom{n}{k}$;
%   \item
%     $3^n = \sum^n_{k=0}2^n\binom{n}{k}$;
%   \item
%     $\binom{n}{k} = \binom{n}{n-k}$;
%   \item
%     $\binom{n+1}{k} = \binom{n}{k} + \binom{n}{k-1}$
%   \item 
%     $\binom{n+m}{r} = \sum^r_{k=0}\binom{n}{r-k}\binom{m}{k}$;
%   \item
%     $\binom{2n}{n} = \sum^n_{k=0}\binom{n}{k}^2$;
%   \item
%     $\binom{n+1}{r+1} = \sum^n_{j=r}\binom{j}{r}$;
%   \item
%     $\binom{2n}{2} = 2\binom{n}{2} + n^2$;
%   \item
%     $\binom{n+r+1}{r} = \sum^r_{k=0}\binom{n+k}{k}$;
%   \item
%     $n2^{n-1} = \sum^{n}_{k=1} k\binom{n}{k}$;
%   \item
%     $n\binom{2n-1}{n-1} = \sum^{n}_{k=1}k\binom{n}{k}^2$;
%   \end{enumerate}
% \end{problem}



\section{Принцип на включването и изключването}

\begin{prop}
  За две крайни множества $A$ и $B$,
  \begin{enumerate}[a)]
  \item 
    ако $A \cap B = \emptyset$, то $\abs{A\cup B} = \abs{A} + \abs{B}$.
  \item
    ако $A\subseteq B$, то $\abs{B\setminus A} = \abs{B} - \abs{A}$.
  \item
    $\abs{A\cup B} = \abs{A} + \abs{B} - \abs{A\cap B}$.
  \end{enumerate}
\end{prop}
\begin{proof}
  \begin{enumerate}[a)]
  % \item
  %   Индукция по броя на елементите на $B$.
  %   \begin{itemize}
  %   \item
  %     $\abs{B} = 0$, то $B = \emptyset$ и тогава за произволно множество $А$,
  %     \[\abs{A\cup B} = \abs{A} + \abs{B}.\]
  %   \item
  %     $\abs{B} = 1$, то $B = \{b\}$ и тогава за произволно крайно множество $A$, за което $b \not\in A$,
  %     е очевидно, че \[\abs{A\cup\{b\}} = \abs{A} + 1.\]
  %   \item
  %     $\abs{B} = n+1$, то $B = B^\prime \cup \{b\}$, $\abs{B^\prime} = n$ и 
  %     нека $A$ е произволно крайно множество, за което $A \cap B = \emptyset$.
  %     \begin{align*}
  %       \abs{A \cup B} = \abs{(A \cup B^\prime) \cup \{b\}} = \abs{A\cup B^\prime} + 1 = \abs{A} + \abs{B^\prime} + 1 = \abs{A} + \abs{B}.
  %     \end{align*}
  %   \end{itemize}
  % \item
  %   Ако $A \subseteq B$, то $B\setminus A\ \cup\ A = B$ и $B\setminus A\ \cap\ A = \emptyset$. Тогава от a):
  %   \begin{align*}
  %     \abs{B} = \abs{B\setminus A\ \cup\ A} = \abs{B\setminus A} + \abs{A}.
  %   \end{align*}
  %   Следователно,
  %   \[\abs{B\setminus A} = \abs{B} - \abs{A}.\]
  \item[в)]
    Имаме, че:
    \begin{align*}
      A\cup B & = A \setminus B\ \cup\ (A\cap B)\ \cup\ B\setminus A\\
      & = A\setminus (A\cap B)\ \cup\ (A\cap B)\ \cup\ B\setminus (A\cap B)
    \end{align*}
    Трите множества в дясната страна на равенството са непресичащи се.
    Тогава, използвайки а) и б), 
    \begin{align*}
      \abs{A \cup B} & = \abs{ A\setminus (A\cap B)} + \abs{A\cap B} + \abs{B\setminus (A\cap B)}\\
      & = \abs{A} - \abs{A\cap B} + \abs{A \cap B} + \abs{B} - \abs{A\cap B}\\
      & = \abs{A} + \abs{B} + \abs{A \cap B}
    \end{align*}
  \end{enumerate}
\end{proof}

\begin{prop}
  Докажете, че за всеки три крайни множества $A$, $B$ и $C$,
  \[\abs{A\cup B \cup C} = \abs{A}+\abs{B}+\abs{C} - \abs{A\cap B} - \abs{B\cap C} - \abs{A\cap C}+ \abs{A\cap B \cap C}.\]
\end{prop}
\begin{proof}
  \begin{align*}
    \abs{(A \cup B) \cup C} & = \abs{A \cup B} + \abs{C} - \abs{(A\cup B)\cap C}\\
    & = (\abs{A} + \abs{B} - \abs{A\cap B}) + \abs{C} - \abs{(A\cap C)\cup(B\cap C)}\\
    & = \abs{A} + \abs{B} + \abs{C} - \abs{A\cap B} - (\abs{A\cap C} + \abs{B\cap C} - \abs{(A\cap C) \cap (B\cap C)})\\
    & = \abs{A} + \abs{B} + \abs{C} - \abs{A\cap B} - \abs{A\cap C} - \abs{B\cap C} + \abs{A\cap B \cap C}
  \end{align*}
\end{proof}


\begin{framed}
\begin{thm}
  Нека $A_1\dots A_n$ са $n$ на брой крайни множества и $n\geq 2$. Тогава:
  \begin{align*}
    |A_1\cup A_2\cup \dots \cup A_n| = & \sum^n_{i=1} |A_i| - \sum_{i < j} |A_{i}\cap A_{j}| + \\
    & \sum_{i < j < k} |A_{i}\cap A_{j}\cap A_{k}|- \dots + (-1)^{n-1}|A_1 \cap A_2\dots \cap A_n|.    
  \end{align*}
\end{thm}
\end{framed}

\begin{problem}
  Колко решения в естествените числа имат уравненията:
  \begin{enumerate}[a)]
  \item
    $x_1+x_2+x_3 = 15$;
  \item
    $x_1 + x_2 + x_3 = 15$, като $x_2 < 3$;
  \item
    % \marginpar{Отг. $\binom{13}{2} - \binom{12}{1} - \binom{11}{1} - \binom{10}{1}$}
    $x_1 + x_2 + x_3 = 15$, като $x_2 \geq 3$;
  \item
    $x_1+x_2+x_3 = 15$, като $x_1 \geq 2$, $x_2 \geq 3$;
  \item
    $x_1+x_2+x_3 = 15$, като $x_1 \geq 2, x_2 \geq 3, x_3 \leq 8$;
  \item
    $x_1+x_2+x_3+x_4 = 25$, като $x_1 < 2$;
  \item
    $x_1+x_2+x_3+x_4 = 25$, като $x_1 < 2$ и $x_3 = 2$;
  \item
    $x_1+x_2+x_3+x_4 = 25$, като $x_1 < 2$ и $x_3 < 2$;
  \end{enumerate}
\end{problem}
\begin{solution}
  \begin{enumerate}[a)]
  \item
    Търсим броят на елементите на 
    \[A = \{(x_1,x_2,x_3) \in \Nat^3\mid x_1 + x_2 + x_3 = 15\}.\]
    Това са всички $15$ елементни мултимножества на  $\{x_1,x_2,x_3\}$.
    Например, мултимножеството $\{x_1,x_1,x_3,x_2,x_1,x_3\}$ отговаря на решение на уравнението $x_1 + x_2 + x_3 = 6$,
    където $x_1 = 3$, $x_2 = 1$, $x_3 = 2$.
    Следователно,
    \[\abs{A} = \binom{15 + 3 - 1}{3-1}.\]
  \item
    Търсим броя на елементите на 
    \begin{align*}
      A_2 =\ &\{(x_1,x_2,x_3) \in \Nat^3\mid x_1 + x_2 + x_3 = 15\ \&\ x_2 < 3\}\\
      =\ & \{(x_1,x_3) \in \Nat^3\mid x_1 + 0 + x_3 = 15\}\ \cup \\ 
         & \{(x_1,x_3) \in \Nat^3\mid x_1 + 1 + x_3 = 15\}\ \cup \\ 
         & \{(x_1,x_3) \in \Nat^3\mid x_1 + 2 + x_3 = 15\}.
    \end{align*}
    Лесно се съобразява, че
    \[\abs{A_2} = \binom{16}{1} + \binom{15}{1} + \binom{14}{1}.\]
  \item
    Отговорът е
    \[\abs{A} - \abs{A_2} = \binom{17}{2} - \binom{16}{1} - \binom{15}{1} - \binom{14}{1}.\]
  \item
    Да разгледаме множествата:
    \begin{align*}
      A & = \{(x_1,x_2,x_3) \in \Nat^3\mid x_1 + x_2 + x_3 = 15\},\\
      A_1 & = \{(x_1,x_2,x_3) \in \Nat^3\mid x_1 + x_2 + x_3 = 15\ \&\ x_1 < 2\},\\
      A_2 & = \{(x_1,x_2,x_3) \in \Nat^3\mid x_1 + x_2 + x_3 = 15\ \&\ x_2 < 3\}.
    \end{align*}
    Ние търсим колко елемента има множеството 
    \[B = \{(x_1,x_2,x_3) \in \Nat^3\mid x_1 + x_2 + x_3 = 15\ \&\ x_1 \geq 2\ \&\ x_2 \geq 3\}.\]
    Понеже \[B = (A\setminus A_1) \cap (A\setminus A_2) = A \setminus (A_1 \cup A_2),\]
    трябва да намерим $\abs{A}$ и $\abs{A_1 \cup A_2}$. Тогава отговорът на задачата е:
    \[\abs{B} = \abs{A} - \abs{A_1 \cup A_2}.\]

    Лесно се вижда, че:
    \begin{align*}
      & \abs{A}  = \binom{17}{2} = 136\\
      & \abs{A_1} = 16 + 15 = 31\\
      & \abs{A_2} = 16 + 15 + 14 = 45\\
      & \abs{A_1\cap A_2} = 6.
    \end{align*}
    Освен това, от принципа за включването и изключването, 
    \[\abs{A_1 \cup A_2} = \abs{A_1} + \abs{A_2} - \abs{A_1\cap A_2} = 31 + 45 - 6 = 70.\]

    Следователно, отговорът е 
    \[\abs{B} = \abs{A} - \abs{A_1 \cup A_2} = 136 - 70 = 66.\]
  \item[д)]
    Използваме същата идея и означения както в горната задача.
    Да разгледаме множествата:
    \begin{align*}
      A & = \{(x_1,x_2,x_3) \in \Nat^3\mid x_1 + x_2 + x_3 = 15\},\\
      A_1 & = \{(x_1,x_2,x_3) \in \Nat^3\mid x_1 + x_2 + x_3 = 15\ \&\ x_1 < 2\},\\
      A_2 & = \{(x_1,x_2,x_3) \in \Nat^3\mid x_1 + x_2 + x_3 = 15\ \&\ x_2 < 3\},\\
      A_3 & = \{(x_1,x_2,x_3) \in \Nat^3\mid x_1 + x_2 + x_3 = 15\ \&\ x_3 > 8\}.
    \end{align*}
    Тук търсим колко елемента има множеството
    \[B = \{(x_1,x_2,x_3) \in \Nat^3\mid x_1 + x_2 + x_3 = 15\ \&\ x_1 \geq 2\ \&\ x_2 \geq 3\ \&\ x_3 \leq 8\}.\]
    Понеже
    \[B = (A \setminus A_1) \cap (A\setminus A_2) \cap (A\setminus A_3) = A \setminus (A_1 \cup A_2 \cup A_3),\]
    трябва да намерим $\abs{A}$ и $\abs{A_1 \cup A_2 \cup A_3}$.
    Тогава отговорът на задачата е 
    \[\abs{B} = \abs{A} - \abs{A_1 \cup A_2 \cup A_3}.\]
    Лесно се вижда, че:
    \begin{align*}
      & \abs{A}  = 136\\
      & \abs{A_1} = 16 + 15 = 31\\
      & \abs{A_2} = 16 + 15 + 14 = 45\\
      & \abs{A_3} = 7 + 6 + 5 + 4 + 3 + 2 + 1 = 28\\
      & \abs{A_1\cap A_2} = 2.3 = 6\\
      & \abs{A_1\cap A_3}  = 7 + 6 = 13\\
      & \abs{A_2\cap A_3}  = 7 + 6 + 5 = 18\\
      & \abs{A_1\cap A_2\cap A_3} = \abs{A_1 \cap A_2} = 6.
    \end{align*}
    Сега от принципа за включването и изключването, 
    \begin{align*}
      \abs{A_1 \cup A_2 \cup A_3} & = \abs{A_1} + \abs{A_2} + \abs{A_3} - \abs{A_1 \cap A_2} - \abs{A_1 \cap A_3} - \abs{A_2 \cap A_3} + \abs{A_1 \cap A_2 \cap A_3}\\
      & = 31 + 45 + 28 - 6 - 13 - 18 + 6 = 73
    \end{align*}
    Следователно, отговорът е:
    \[\abs{B} = \abs{A} - \abs{A_1\cup A_2 \cup A_3} = 136 - 73 = 63.\]
  \end{enumerate}
\end{solution}

\begin{problem} % Гаврилов стр. 265, зад. 7
  Нека $U$ е множество от $n$ елемента, $n\geq 3$. За всяко множество $X\subseteq U$, с $\overline{X}$ означаваме $U\setminus X$.
  Също така, за множества $X$ и $Y$, понякога ще пишем $XY$ вместо $X \cap Y$.
  Намерете броя на:
  \begin{enumerate}[a)]
  \item
    \marginpar{Отг. $4^n$}
    двойките $(X,Y)$ за $X,Y\subseteq U$;
  \item
    \marginpar{Отг. $2\binom{n}{1} 2^{n-1}$}
    двойките $(X,Y)$ за $X,Y\subseteq U$, за които $\vert{X}\vert = 1$;
  \item
    \marginpar{Отг. $2^2\binom{n}{2}2^{n-2}$}
    двойките $(X,Y)$ за $X,Y\subseteq U$, за които $\vert{X}\vert = 2$;
  \item
    %\marginpar{Отг. $4^n - 3^{n}$}
    двойките $(X,Y)$ за $X,Y\subseteq U$, за които $\vert{X}\vert \geq 1$;
  \item
    двойките $(X,Y)$ за $X,Y\subseteq U$, за които $\vert{X}\vert = k$ и $k < n$;
  \item
    %\marginpar{Отг. $3^{n} + \binom{n}{1} 3^{n-1}$}
    двойките $(X,Y)$ за $X,Y\subseteq U$, за които $\vert{X}\vert \leq 1$;
  \item
    \marginpar{Отг. $\binom{n}{1}\binom{n}{1} = n^2$}
    двойките $(X,Y)$ за $X,Y\subseteq U$, за които $\vert{X}\vert = 1$ и $\vert{Y}\vert = 1$;
  \item
    \marginpar{Нямаме буква $XY$. Отг. $3^n$}
    двойките $(X,Y)$ за $X,Y\subseteq U$, за които $X \cap Y = \emptyset$;
  \item
    \marginpar{Една буква $XY$. Отг. $\binom{n}{1}3^{n-1}$}
    двойките $(X,Y)$ за $X,Y\subseteq U$, за които $\abs{X \cap Y} = 1$;
  \item
    двойките $(X,Y)$ за $X,Y\subseteq U$, за които $\abs{X \cap Y} = k$ и $k < n$;
  \item
    %\marginpar{Отг. $2\binom{n}{1}2^{n-1} = n2^n$}
    двойките $(X,Y)$ за $X,Y\subseteq U$, за които $\abs{(X\setminus Y) \cup (Y\setminus X)} = 1$;
  \item
    двойките $(X,Y)$ за $X,Y\subseteq U$, за които $X\cap Y = \emptyset$ и
    $|X|\geq 1$, $|Y|\geq 1$;
  \item
    двойките $(X,Y)$ за $X,Y\subseteq U$, за които $X\cap Y = \emptyset$ и 
    $|X|\geq 2, |Y|\geq 2$;
  \item
    двойките $(X,Y)$ за $X,Y\subseteq U$, за които $\vert{X\setminus Y}\vert = 1$;
  \item
    двойките $(X,Y)$ за $X,Y\subseteq U$, за които $\vert{X\setminus Y}\vert = k$ и $k < n$;
  \item
    двойките $(X,Y)$ за $X,Y\subseteq U$, за които $|(X\setminus Y)\cup(Y\setminus X)| = 1$ и 
    $|X|\geq 2, |Y|\geq 2$;
  \item
    %\marginpar{Използвайте принципа за вкл. и изкл.}
    двойките $(X,Y)$ за $X,Y\subseteq U$, за които $X\cap Y = \emptyset$, $|X|\geq 2$ и $|Y|\geq 3$;
  \item
    % \marginpar{Използвайте принципа за вкл. и изкл.}
    двойките $(X,Y)$ за $X,Y\subseteq U$, за които $|(X\setminus Y)\cup(Y\setminus X)| = 1$, $X\cap Y = \emptyset$, $|X|\geq 2$ и $|Y|\geq 3$;
  \item
    \marginpar{Отг. $8^n$}
    тройките $(X,Y,Z)$ за $X,Y,Z\subseteq U$;
  \item
    \marginpar{Отг. $6^n$}
    тройките $(X,Y,Z)$ за $X,Y,Z\subseteq U$, за които $X \cap Y = \emptyset$;
  \item
    \marginpar{$U = X\overline{Y}Z\cup X\overline{YZ} \cup \overline{X}Y\overline{Z}$. Отг. $3^n$}
    тройките $(X,Y,Z)$ за $X,Y,Z\subseteq U$, за които $X\cup Y\overline{Z} = \overline{X}\cup\overline{Y}$;
  \item
    тройките $(X,Y,Z)$ за $X,Y,Z\subseteq U$, за които $Y\cup X = Z\cup\overline{Y}$;
  \item
    тройките $(X,Y,Z)$ за $X,Y,Z\subseteq U$, за които $X\cup Y\overline{Z} = \overline{X}\cup\overline{Y}$ и
    $|Z| = 0$.
  \item
    тройките $(X,Y,Z)$ за $X,Y,Z\subseteq U$, за които $X\cup Y\overline{Z} = \overline{X}\cup\overline{Y}$ и
    $|X|\geq 1, |Y|\geq 1, |Z| = 1$.
  \item
    тройките $(X,Y,Z)$ за $X,Y,Z\subseteq U$, за които $X\cup Y\overline{Z} = \overline{X}\cup\overline{Y}$ и
    $|X|\geq 1, |Y|\geq 1, |Z|\leq 1$.
  \item
    тройките $(X,Y,Z)$ за $X,Y,Z\subseteq U$, за които $X \cup YZ = \ov{X} \cup \ov{Z}$;
  \item
    тройките $(X,Y,Z)$ за $X,Y,Z\subseteq U$, за които $X\ov{Y} \cup YZ = U$;
  \end{enumerate}
\end{problem}
\begin{solution}
  \begin{enumerate}[a)]
  \item 
    Разглеждаме азбука $\Sigma = \{XY, X\ov{Y}, \ov{X}Y, \ov{X}\ov{Y}\}$.
    Да разгледаме $U = \{u_1,\dots,u_n\}$.
    На всеки елемент на  $\{(X,Y) \mid X,Y \subseteq U\}$ можем еднозначно да съпоставим 
    дума  $\alpha = a_1\cdots a_n$ над азбуката $\Sigma$
    по следния начин:
    \begin{itemize}
    \item 
      ако $u_i \in X \cap Y$, то $a_i = XY$;
    \item 
      ако $u_i \in X \cap  \ov{Y}$, то $a_i = X\ov{Y}$;
    \item 
      ако $u_i \in \ov{X} \cap Y$, то $a_i = \ov{X}Y$;
    \item 
      ако $u_i \in \ov{X} \cap \ov{Y}$, то $a_i = \ov{X}\ov{Y}$.
    \end{itemize}
    Да разгледаме няколко примера:
    \begin{itemize}
    \item
      на двойката $(\{u_1\},\{u_2\})$ съпоставяме думата $\alpha = a_1\cdots a_n$,
      където $a_1 = X\ov{Y}$, $a_2 = \ov{X}Y$, $a_i = \ov{X}\ov{Y}$ за $i \geq 3$.
    \item 
      на двойката $(U,\{u_2\})$ съпоставяме думата $\alpha = a_1\cdots a_n$,
      където $a_2 = XY$ и $a_i = X\ov{Y}$ за $i \neq 2$.
    \item
      на двойката $(\{u_1\},\{u_1,u_2\})$ съпоставяме думата $\alpha = a_1\cdots a_n$,
      където $a_1 = XY$ и $a_2 = \ov{X}Y$, $a_i = \ov{X}\ov{Y}$ за $i \geq 3$.
    \end{itemize}
    Понеже всички думи с дължина $n$ над азбука с $4$ букви са $4^n$, 
    то всички двойки $(X,Y)$ са също $4^n$.
  \item
    Трябва да намерим всичи думи с дължина $n$ над азбуката $\{XY,X\ov{Y},\ov{X}Y,\ov{X}\ov{Y}\}$,
    като буквите $XY$ и $X\ov{Y}$ се срещат общо веднъж.
    Това означава, че от $n$ позиции трябва да изберем една, в която да поставим $XY$ или $X\ov{Y}$,
    а в останалите $n-1$ позиции поставяме буквите $\ov{X}Y$ или $\ov{X}\ov{Y}$.
    Така получаваме като резултат \[2\binom{n}{1}2^{n-1} = n2^n.\]
  \item
    Тук разглеждаме тези думи с дължина $n$ над азбуката $\Sigma$, като
    има {\em точно едно} срещане на $XY$ или $X\ov{Y}$.
    Всички тези думи са $2\binom{n}{1}2^{n-1}$
  \item[ф)]
    Понеже $X \cap Y = \emptyset$, то в думите не се срещат буквите $XYZ$ и $XY\ov{Z}$.
    Така остават $6$ възможни букви и оттук веднага следва, че всички такива думи са $6^n$.    
  \item[х)]
    Да разгледаме какво означава $X\cup Y\ov{Z} = \ov{X} \cup \ov{Y}$.
    \begin{itemize}
    \item 
      Ако $x \in X$, то $x \in X \cup Y\ov{Z}$ и следователно $x \in \ov{X} \cup \ov{Y}$.
      Тогава е ясно, че $x \in \ov{Y}$. Това означава, че имаме буквите $X\ov{Y}Z$ и $X\ov{Y}\ov{Z}$.
    \item
      Ако $x \in \ov{X}$, то $X \in \ov{X} \cup \ov{Y}$ и следователно $x \in X \cup Y\ov{Z}$.
      Сега получаваме, че $x \in Y\ov{Z}$. Това означава, че имаме буквата $\ov{X}Y\ov{Z}$.
    \end{itemize}
    Видяхме, че с горното ограничение трябва да разгледаме само думите с дължина $n$ съставени от три букви.
    Те са общо $3^n$ на брой.
  \end{enumerate}
\end{solution}

% \begin{problem}
%   Дайте комбинаторно доказателство на 
%   \begin{enumerate}[1)]
%   \item
%     $n! = \sum^{n}_{k=0}(-1)^k\binom{n}{k}(n-k)^n$
%   \item
%     $\binom{x+a}{n} = \sum^{k}_{i=0}\binom{x}{i}\binom{a}{n-i}$
%   \end{enumerate}
% \end{problem}

% \begin{problem}
%   \begin{enumerate}[a)]
%   \item
%     \marginpar{Отг. $\binom{n+4-1}{4-1}$}
%     По колко начина могат да се изберат $n$ монети от купчина монети с номинал 5, 10, 20 и 50 стотинки?
%   % \item
%   %   %\marginpar{Отг. }
%   %   По колко начина могат да се изтеглят $13$ от $52$ карти, ако ги различаваме само по цвета?
%   \item
%     %\marginpar{Отг. $\binom{n+m-1}{m-1}$}
% Намерете броя на възможните начини за разпределение на $n$ {\bf неразличими} топки в $m$ различни кутии.
% \item
%     %\marginpar{Отг. $\binom{(n-m)+m-1}{m-1}$}
%     Намерете броя на възможните начини за разпределение на $n$ {\bf неразличими} топки в $m$ различни кутии, 
%     ако няма празна кутия.
%   \item
%     %\marginpar{Отг. $\binom{n+m-1}{m-1} - \binom{(n-m)+m-1}{m-1}$}
%     Намерете броя на възможните начини за разпределение на $n$ {\bf неразличими} топки в $m$ различни кутии,
%     ако съществува поне една празна кутия.
%   \item
%     %\marginpar{$\binom{n+m-1}{m-1}/m!$ ???}
%     Да се намери броя на възможните начини за разпределения на $n$ {\bf неразличими} топки в $m$ {\bf неразличими} кутии.
%   % \item
%   %   %\marginpar{}
%   %   Да се намери броя на възможните начини за разпределения на $n$ {\bf различими} топки в $m$ различни кутии.
%   \end{enumerate}
% \end{problem}

% \begin{problem}
%   Множеството от всички двоични вектори от $\{0,1\}^{n}$, които във фиксирани $n-k$ позиции имат равни значения,
%   ги наричаме $k$-равнини, за $k\leq n$.
%   \begin{enumerate}[a)]
%   \item
%     \marginpar{Отг. $2^{k}$}
%     Колко различни вектора има в една $k$-равнина?
%   \item
%     \marginpar{Отг. $2^{n-k}\binom{n}{n-k}$}
%     Колко различни $k$-равнини има в $\{0,1\}^{n}$?
%   \item
%     \marginpar{Отг. $\binom{n}{n-k}$}
%     Колко различни $k$-равнини съдържат даден фиксиран $n$-мерен вектор?
%   % \item
%   %   %\marginpar{Отг. $$}
%   %   Колко различни $k$-равнини съдържат дадена $l$-равнина, $0\leq l < k$.
%   \end{enumerate}
% \end{problem}


\begin{problem}
  Отговорете на следните въпроси!
  \begin{enumerate}[a)]
  \item
%    \marginpar{$2.10!10!$}
    По колко начина могат $n$ момчета и $n$ момичета да седнат на ред с $2n$ стола, като няма двама от един пол седящи един до друг?
  \item
 %   \marginpar{$2.10!10! - 2.7.3!3!$}
    По колко начина могат $n$ момчета и $n$ момичета да седнат на ред с $2n$ стола, като няма двама от един пол седящи един до друг и Иванчо и Марийка не седят един до друг? 
  \item
  %  \marginpar{Отг. $2.(n-1).(n-2)!$}
    По колко различни начина могат да се подредят на рафт $n$ книги, така че две от тях, определени предварително, да са една до друга?
  \item
   % \marginpar{Отг. $\frac{(n-1)!}{2}$}
    Колко различни гердана могат да се направят от $n$ различни перли, като се използват всичките?
  \item
%    \marginpar{Отг. $2.(n-2)(n-3)!$}
    На хоро в кръг са хванали общо $n$ души, между които и Иванчо и Марийка.
    Колко са възможните подредби, при които Иванчо и Марийка са един до друг?
 \item
    На хоро в кръг са хванали общо $n$ души, между които и Иванчо и Марийка.
    Колко са възможните подредби, при които Иванчо и Марийка не са един до друг?
  \item
 %   \marginpar{Отг. $2^n(n-1)!$}
    Имаме $n$ съпружески двойки, които седят на $2n$ места около една кръгла маса. 
    По колко начина могат да седнат всички двойки, ако ротациите се броят за едно и също подреждане, и
    всеки мъж седи до половинката си.
  \item
    \marginpar{Това е както при герданите}
    Две сядания на една кръгла маса не са различни, ако всеки от седящите има едни и същи съседи.
    По колко различни начина могат да седнат около една кръгла маса:
    \begin{enumerate}
    \item
   %   \marginpar{Отг. $\frac{(n-1)!}{2}$}
      $n (\geq 2)$ човека;
    \item
    %  \marginpar{Отг. $\frac{2n!n!}{2.2n} = \frac{n!(n-1)!}{2}$}
      $n$ мъже и $n$ жени, като двама души от един и същ пол не седят един до друг.
    \end{enumerate}
  \end{enumerate}
\end{problem}


\section{Комбинаторни задачи за функции}

\begin{problem}
  \marginpar{Отг. $\binom{11}{5}$}
  Нека $(a_1,a_2,\dots,a_{12})$ е пермутация на числата от 1 до 12, за които е изпълнено условието:
  \[a_1 > a_2 > a_3 > a_4 > a_5 > a_6 < a_7 < a_8 < a_9 < a_{10} < a_{11} < a_{12}.\]
  Намерете броя на тези пермутации.  
\end{problem}

\begin{problem}
  Да фиксираме естествените числа $m$ и $n$.
  Една функция \[f:\{1,\dots,n\}\to\{1,\dots,m\}\] е монотонно ненамаляваща, ако
  \[(\forall i\forall j)[1\leq i<j\leq n \rightarrow f(i)\leq f(j)].\]
  \begin{enumerate}[a)]
  \item
    \marginpar{Отг. $\binom{n+m-1}{m-1}$}
    Колко такива функции съществуват?
  \item
    \marginpar{Във всяка кутийка има топка. Отг. $\binom{(n-m)+m-1}{m-1}$}
    Колко от тези функции са сюрективни при $n\geq m$?
  \item
    \marginpar{Отг. $\binom{m}{n}$}
    Колко от тези функции са инективни при $n\leq m$?
  \end{enumerate}
\end{problem}

\begin{problem}
  Да разгледаме функциите от вида $f:A\to B$,
  където $\abs{A} = k$, $\abs{B} = n$.
  \begin{enumerate}[a)]
  \item 
    \marginpar{Отг. $n^k$}
    Колко са всички тези функции?
  \item
    \marginpar{Само ако $k\leq n$, $n!/(n-k)!$}
    Колко от тези функции са инективни?
  \item
    \marginpar{$\binom{n+k-1}{k-1}$}
    Колко от тези функции са монотонно ненамаляващи?
  \item
    \marginpar{Само ако $n = k$, $n!$}
    Колко от тези функции са биективни?
  \item
    \marginpar{Това е по-трудно.}
    Колко от тези функции са сюрективни?
  \end{enumerate}
\end{problem}


\begin{problem}
  Да разгледаме функциите от вида $f:A\to B$, където $\abs{A} = k$, $\abs{B} = n$.
  Колко от тези функции са сюрективни?
\end{problem}
\begin{solution}
  Нека $B = \{b_1,b_2,\dots,b_n\}$.
  Да означим с $F$ всички функции от вида $f:A\to B$.
  \marginpar{$Range(f) = \{f(a) \mid a\in A\}$}
  \[F_i = \{f:A\to B\mid b_i \not\in Range(f)\}.\]
  Да означим с $S$ сюрективните функции $f:A\to B$.
  Понеже сюрективните функции са тези, за които $Range(f) = B$, то
  \[S = F\setminus(F_1\cup F_2 \cup \dots \cup F_n).\]
  Лесно се съобразява, че имаме следните равенства:
  \begin{align*}
    & \abs{F} = n^k\\
    & \abs{F_i} = (n-1)^k\\
    & \abs{F_i \cap F_j} = (n-2)^k\\
    & \sum^n_{i=1}\abs{F_i} = n.(n-1)^k\\
    & \sum_{i < j}\abs{F_i \cap F_j} = \binom{n}{2}(n-2)^k\\
    & \sum_{i < j < l}\abs{F_i \cap F_j \cap F_l} = \binom{n}{3}(n-3)^k\\
    & \dots
  \end{align*}
  Тогава, прилагайки принципа на включването и изключването, получаваме:
  \begin{align*}
    \abs{S} & = \abs{F} - \sum_i \abs{F_i} + \sum_{i < j}\abs{F_i\cap F_j} - \sum_{i < j < l}\abs{F_i \cap F_j \cap F_l} + \dots \\
    & = n^k - \binom{n}{1}(n-1)^k + \binom{n}{2}(n-2)^k - \binom{n}{3}(n-3)^k + \dots\\
    & = n^k + (-1)^1\binom{n}{1}(n-1)^k + (-1)^2\binom{n}{2}(n-2)^k + (-1)^3\binom{n}{3}(n-3)^k + \dots\\
    & = \sum^n_{i=0}(-1)^i\binom{n}{i}(n-i)^k
  \end{align*}
\end{solution}

Сега ще видим едно приложение на горната задача.

\begin{problem}
  \marginpar{Колко са сюрективните функции $f:A\to B$, като $\abs{A} = 7$, $\abs{B}=3$?}
  Нека да имаме 7 топки с номер на всяка от тях и нека имаме 3 различни кутии, отново номерирани.
  По колко начина можем да поставим топките в кутиите, така че във всяка кутия да има поне по една топка ?
\end{problem}

\newpage

\begin{remark}
  Да разгледаме множествата $A=\{1,2,\dots,n\}$ и $\Sigma = \{a_1,a_2,\dots,a_k\}$.
  Тогава имаме следните преводи между езика на функциите и езика на думите и азбуките.
  \newline
  \begin{tabular}{|l|l|}
    \hline
    функциите от вида $f:A\to \Sigma$ & думите с дължина $n$ над азбуката $\Sigma$ \\
    \hline
    \hline
    {\bf всички} такива функции & {\bf всички} такива думи\\
    \hline
    {\bf инективните} функции, $n \leq k$ & думите  {\bf без повторения на букви} \\
    \hline
    {\bf сюрективните} функции, $n \geq k$ & думите, в които {\bf всяка буква се среща} \\
    \hline
    {\bf биективните} функции, $n = k$ & думите, в които всяка буква се \\
    & {\bf среща точно веднъж} \\
    \hline
  \end{tabular}

  % Да разгледаме две произволни множества $A = \{1,2,\dots,n\}$ и $B=\{1,2,\dots,k\}$.
  % Да напмним, че една функция $f:A \to B$ е {\bf монотонно ненамаляваща}, ако
  % \[(\forall i\forall j)[1\leq i<j\leq n \rightarrow f(i)\leq f(j)].\]
  % Тогава имаме следните еквивалентни преводи между езика на функциите и езика на кутиите и топките:
  % \newline
  % \begin{tabular}{|l|l|}
  %   \hline
  %   мон. ненамаляващите $f:A\to B$ & \\
  %   \hline
  %   \hline
  %   {\bf всички} такива функции & {\bf всички} такива думи\\
  %   \hline
  %   {\bf инективните} функции при $n \leq k$ & думите, в които {\bf няма повторения на букви} \\
  %   \hline
  %   {\bf сюрективните} функции при $n \geq k$ & във всяка кутия има поне една топка\\
  %   \hline
  %   {\bf биективните} функции при $n = k$ & {\bf всяка буква се среща точно веднъж} \\
  %   \hline
  % \end{tabular}

\end{remark}

% \subsection{Пълно разбъркване на множество}
% \marginpar{На англ. derangement}
% Нека $A = \{a_1,\dots a_n\}$ е произволно множество от $n$ елемента.
% Една пермутация $f$ на елементите на $A$ наричаме {\bf пълно разбъркване}, 
% ако $f$ няма неподвижни точки, т.е. $(\forall a\in A)[f(a)\neq a]$.
% % Можем да означим едно пълно разбъркване $f$ на $A$ 
% % като редицата $(a_{i_1},a_{i_2},\dots,a_{i_n})$,
% % където $f(j) = a_{i_j}$.
% \begin{example}
%   Нека $A = \{1,2,3\}$. Да изредим всички пълни разбърквания на $A$:
%   \[ (2,3,1), \quad  (3,1,2)\]
%   \end{align*}
%   Нека сега $A = \{1,2,3,4\}$. Пълните разбърквания на $A$ са:
%   \begin{align*}
%     & (2,1,4,3), \quad (2,4,1,3), \quad (2,3,4,1)\\
%     & (3,1,4,2), \quad (3,4,1,2), \quad (3,4,2,1)\\
%     & (4,1,2,3), \quad (4,3,2,1), \quad (4,3,1,2)\\
%   \end{align*}
% \end{example}

% \begin{problem}
%   Напишете програма, която намира всички пълни разбърквания на зададено като вход множество $A$.
% \end{problem}


% \begin{problem}
%   Означаваме с $D(n)$ всички пълни разбърквания на едно множество с $n$ елемента.
%   \begin{enumerate}[a)]
%   \item
%     Докажете, че $D(n) - nD(n-1) = (-1)^{n}$ за $n \geq 2$.
%   \item
%     Намерете $D(n)$.
% \end{enumerate}
% \end{problem}
% \begin{proof}
%   \begin{enumerate}[a)]
%   \item
%     Доказателството е с индукция по $n$.
%     Лесно се  съобразява, че имаме това свойство за $n=2$, 
%     защото $D(2) = 1$, а $D(1) = 0$.
    
%     Да приемем, че твърдението е вярно за $n > 2$, като примем, че то е вярно за $n-1$.
%     Нека $A = \{a_1,a_2,\dots,a_n\}$ и да фиксираме последния елемент $a_n$ на $A$.
%     Да означим $B = A\setminus\{a_n\}$.
%     %$B$ имат $D(n-1)$ разбърквания.
%     Да разгледаме едно пълно разбъркване на $B$, $\pi(i) = a_i$, $i = 1,\dots,n-1$.
% %    \[\pi = [a_{i_1}, a_{i_2},\dots, a_{i_{n-1}}].\]
%     Ако заменим $\pi(j)$ с $a_n$ и поставим $\pi(j)$ на $n$-та позиция, то получаваме разбъркване $\rho$ на $A$.
%     Това можем да направим за всяко $ 1 \leq j \leq n-1$ и всяко разбъркване на $B$ и получаваме всеки път ново разбъркване на $A$.
%     Това са общо $(n-1)D(n-1)$ разбърквания.

%     За съжаление, това не са всички пермутации, които ни трябват.
%     Нека сега $\pi_i$ е пермутация на $B$, която има точно една неподвижна точка и $\pi_i(i) = a_i$.
%     Като сменим $a_i$ с $a_n$ и поставим $a_i$ на $n$-та позиция, то получаваме ново разбъркване на $A$.
%     Лесно се съобразява, че разбъркванията получени по този начин са $(n-2)D(n-2)$ 
%     и с това се изчерпват начините за генериране на разбърквания на $A$ от пермутациите на $B$.   

%     Следователно, \[D(n) = (n-1)D(n-1) + (n-2)D(n-2).\]
%     От индукционното предположение знаем, че \[D(n-1) = (n-1)D(n-2) + (-1)^{n-1}.\]
%     Като заместим, получаваме \[D(n) = (n-1)D(n-1) + D(n-1) - (-1)^{n-1},\]
%     което ни дава крайния резултат \[D(n) = nD(n-1) + (-1)^{n}.\]
        
%   \item
%     Може да се реши и с индукция, използвайки предишната подточка.
%     Ще дадем директно рещение като приложим принципа за включване и изключване.
%     Да означим с $P$ броят на всички пермутации на $A$.
%     Да означим с $P_i$ броят на пермутациите, които запазват $i$-тия елемент от $A$.
%     Следователно,
%     \begin{align*}
%       D(n) =\ & |P\setminus{(P_1\cup P_2 \cup \dots \cup P_n)}|\\
%       =\ & |P| - |P_1\cup P_2 \cup \dots \cup P_n|\\
%       =\ & |P| - \sum^n_{i = 1}|P_i| + \sum_{i < j }|P_i\cap P_j| - \dots \\
%       =\ & n! - \sum^n_{i = 1}(n-1)! + \sum_{i<j} (n-2)! - \dots\\
%       =\ & \binom{n}{0}n! + (-1)\binom{n}{1}(n-1)! + (-1)^2\binom{n}{2}(n-2)! - \dots\\
%       =\ & \sum^n_{k=0}(-1)^k\binom{n}{k}(n-k)!
%     \end{align*}
%   \end{enumerate}
% \end{proof}

% \begin{problem}
%   Дадени са $n$ кутии и $m$ неразличими топки.
%   По колко начина могат да се разпределят всички топки в кутиите, така че нито в една кутия да няма повече от $r$ топки?
% \end{problem}
% \begin{proof}
%   Първо, нека отбележим, че $r.n \geq m$. В противен случай, няма да можем да разпределим всички топки в кутиите.
%   Имаме $S = \binom{n+m-1}{n-1}$ общо начини за разпределяне на топките, без ограничение за брой топки в една кутия.
 
%   Да разгледаме едно от тези разпределения на топките.
%   Нека при него във всяка кутия има $r_i$ топки и $\sum^{n}_{i=1} r_i = m$.
%   Нека $p = \lfloor{\frac{m}{r}}\rfloor$.
%   Ясно е, че не може да има повече от $p$ кутии с повече от $r$ топки.
 
%   \begin{enumerate}
%     \item
%       Ако в $i$-тата кутия има повече от $r$ топки, т.е. $r_i = (r+1)+r'_i$, то 
%       броят на различните начини за разпределения на $m-(r+1)$ топки в $n$ кутии е 
%       \[S_1 = \binom{n}{1}\binom{m-(r+1)+(n-1)}{n-1}.\]
%       $S_1$ е броят на разпределенията с поне една кутия с повече от $r$ топки.
%     \item
%       Ако в $i$-тата и $j$-тата кутии има повече от $r$ топки, т.е.
%       $r_i = (r+1)+r'_i, r_j = (r+1)+r'_j$.
%       Тогава броят на различните разпределения на $m-2(r+1)$ топки в $n$ кутии е
%       \[S_2 = \binom{n}{2}\binom{m-2(r+1)+(n-1)}{n-1}.\]
%       $S_2$ е броят на разпределенията с поне две кутии с повече от $r$ кутии.
%     \item
%       Продължаваме да дефинираме $S_i$ до $i=p$.
%   \end{enumerate}

%   Накрая от принципа за включването и изключването получаваме, че крайният резултат е $S + \sum^{p}_{i=1}(-1)^{i}S_i$.
% \end{proof}


\section{Принцип на Дирихле}
\begin{problem}%[\cite{rosen}, стр. 351]
  В един месец от 30 дни се провежда баскетболен турни, в който се играе поне един мач на ден, но всички мачове са не повече от 45.
  Покажете, че има период от последователни дни от месеца, в който се провеждат точно 14 мача.
\end{problem}
\begin{proof}
  Нека $a_j$ означава сумата на всички проведени мачове в първите $j$ дни на месеца.
  Търсим такива $i<j$, че $a_j - a_i = 14$.
  От условието следва, че редицата $a_1,a_2,\dots, a_{30}$ е строго монотонно растяща и
  \[(\forall j)[0\leq j\leq 30 \rightarrow a_j \leq 45].\]
  Ясно е също, че редицата $a_1+14,a_2+14,\dots,a_{30}+14$ е строго монотонно растяща.
  Образуваме редица от 60 елемента $a_1,\dots,a_{30},a_1+14,\dots,a_{30}+14$, като 
  всеки елемент на редицата приема стойност от 1 до 59.
  Тогава от принципа на Дирихле следва, че съществуват два елемента на редицата, които са равни.
  Първите 30 са различни са различни помежду си, вторите 30 елемента също са различни помежду си.
  Следователно, $a_i = a_j + 14$ за някои $i,j$.
  Тогава в дните от $j$ до $i$ са проведени точно 14 мача.
\end{proof}

\begin{problem}%[\cite{rosen}, стр. 351]
  Нека имаме редица $a_0,\dots,a_n$от $n+1$ произволни числа, ненадвишаващи $2n$.
  Покажете, че трябва да съществува $i$ такова, че $a_i\vert a_j$ за някое $j\neq i$.
\end{problem}
\begin{proof}
  Да представим всеки от елементите на редицата $a_j = 2^{k_j}q_j$, където $q_j$ е нечетно.
  Да разгледаме редицата от нечетни числа $q_0,\dots,q_n$, като имаме и условието $q_i \leq 2n$.
  Имаме само $n$ нечетни числа в интервала $[0,2n]$, следователно $q_i = q_j = q$, за някои $i,j$.
  Тогава $a_i = 2^{k_i}q$ и $a_j = 2^{k_j}q$ и е ясно, че или $a_i\vert a_j$ или $a_j\vert a_i$.  
\end{proof}

% \begin{lemma}
%   Нека имаме редица от $n^2 + 1$ различни реални числа $a_1,a_2,\dots a_{n^2+1}$.
%   Тогава съществува подредица с дължина $n+1$, която е или монотонно растяща или монотонно намаляваща.
% \end{lemma}
% \begin{proof}
%   Искаме да намерим индекси $1\leq i_1<i_2<\dots i_{n+1}\leq n^2+1$, за които или $a_{i_1}<a_{i_2}<\dots< a_{i_{n+1}}$ или
%   $a_{i_1}>a_{i_2}>\dots> a_{i_{n+1}}$.

%   Нека за всяко $1\leq i \leq n^2+1$ да означим с $\eta_i$ дължината на най-дългата монотонно растяща редица, която започва от $a_i$.
%   Ако имаме за някое $i$, $\eta_i \geq n+1$, то сме готови.
%   Иначе, от принципа на Дирихле, съществува $1\leq j \leq n$, редица $i_1<i_2<\dots<i_m$ за $m \geq \lceil{\frac{n^2+1}{n}}\rceil = n+1$ и
%   $\eta_{i_1} = \eta_{i_2} = \dots = \eta_{i_m} = j$.
%   Ако $a_{i_k} < a_{i{k+1}}$, то тогава $\eta_{i_k}\geq j+1$, което е противоречие.
%   Следователно, $a_{i_k} > a_{i{k+1}}$ и така получаваме монотонно намаляваща редица $a_{i_1} > a_{i_2} > \dots > a_{i_m}$, която има дължина поне $n+1$.
% \end{proof}

\begin{problem}
  Нека имаме редица от $n$ произволни, не непременно различни, естествени числа $a_1,\dots,a_n$.
  Тогава има подредица от последователни елементи $a_i,a_{i+1},\dots,a_{j}$, за които
  $n | (\sum^{j}_{k=i}a_k)$.
\end{problem}
\begin{proof}
  Да разгледаме редицата от $n+1$ елемента:
  \[\sum^0_{i=1}a_i,\sum^1_{i=1}a_i,\dots,\sum^n_{i=1}a_i.\]
  Тъй като има $n$ различни остатъка при деление на $n$, то от принципа на Дирихле следва, че 
  има поне два елемента $\sum^l_{i=1}a_i, \sum^k_{i=1}a_i$, за $l<k$, които дават един и същ остатък при деление на $n$.
  Получаваме, че \[n|(\sum^l_{i=1}a_i - \sum^k_{i=1}a_i)\ \Rightarrow\ n|(\sum^k_{i=l+1}a_i).\]
\end{proof}


%%% Local Variables: 
%%% mode: latex
%%% TeX-master: "discrete-math"
%%% End: 

\chapter{Булеви функции}
\index{булева функция}
\section{Съждителни формули}
\index{съждителни формули}

Да припомним таблицата за истинност на някои от основните булеви функции на два аргумента.
\marginpar{$x\oplus y$ - симетрична разлика}

\vspace{8pt}
\begin{tabular}{|c|c|c|c|c|c|c|c|c|c|}
  \hline
  $x$ & $y$ & $\overline{x}$ & $x \vee y$ & $xy$ & $x \rightarrow y$ & $\overline{x}\vee y$ & $x \iff y$ & $x \oplus y$ & $x\overline{y} \vee \overline{x}y$\\
  \hline
  \hline
  0 & 0 & 1 & 0 & 0 & 1 & 1 & 1 & 0 & 0 \\
  \hline
  0 & 1 & 1 & 1 & 0 & 1 & 1 & 0 & 1 & 1 \\
  \hline
  1 & 0 & 0 & 1 & 0 & 0 & 0 & 0 & 1 & 1 \\
  \hline
  1 & 1 & 0 & 1 & 1 & 1 & 1 & 1 & 0 & 0 \\
  \hline
\end{tabular}
\vspace{8pt}

Ще казваме, че две съждителни формули $\varphi$ и $\psi$ са тъждествено еквивалентни, ако имат имат еднакви стълбове в 
съответните таблици за истинност. В такъв случай, ще пишем $\varphi \equiv \psi$.

\subsection{Основни свойства}

\Stefan{Това да се сложи още в началото при съждителното смятане. Операцията $\oplus$ направо да се махне оттук}

\marginpar{За да улесним записа, често вместо $x\wedge y$ пишем $x\cdot y$ или $xy$. Също така, вместо $\neg x$ пишем $\overline{x}$}
\begin{enumerate}[1)]
\item
  Комутативни свойства
  \[xy \equiv yx,\quad x\vee y \equiv y\vee x,\quad x\oplus y \equiv y\oplus x\]
\item
  Асоциативни свойства
  \[(xy)z \equiv x(yz),\quad (x\vee y)\vee z \equiv x\vee (y\vee z),\quad (x\oplus y)\oplus z \equiv x\oplus (y\oplus z)\]
\item
  Лесно се проверява с таблиците за истинност, че:
  \[x\oplus y \equiv x\ov{y}\vee \ov{x}y \equiv (x\vee y)(\ov{x}\vee\ov{y})\]
\item
  Свойства на отрицанието
  \[x\ov{x} \equiv 0, \quad x\vee\ov{x} \equiv x\vee 1,\quad x\oplus\ov{x} \equiv 1\]
\item
  Закон за двойното отрицание
  \[\ov{\ov{x}} \equiv x\]
\item
  Свойства на константите
  \[x\cdot 0 \equiv 0, \quad x\cdot 1 \equiv x,\quad x\vee 0 \equiv x,\quad x\vee 1 \equiv 1,\quad x\oplus 0 \equiv x, \quad x\oplus 1 \equiv \ov{x}\]
\item
  Дистрибутивни свойства
  \begin{enumerate}[]
  \item
    $x(y\vee z) \equiv xy \vee xz$,
  \item
    $xy \vee z \equiv (x\vee z)(y\vee z)$,
  \item
    $(x\oplus y)z \equiv xz \oplus yz$.
  \end{enumerate}
\item
  Идемпотентентни свойства
  \[xx \equiv x, \quad x\vee x \equiv x\]
\item
  Свойства на отрицанието
  \[x\ov{x} \equiv 0, \quad x\vee\ov{x} \equiv 1, \quad x\oplus\ov{x} \equiv 1\]
\item
  Закони на Де Морган
  \[\ov{xy} \equiv \ov{x}\vee\ov{y}, \quad \ov{x\vee y} \equiv \ov{x}\cdot\ov{y}\]
\end{enumerate}

\begin{problem}
  \marginpar{\cite[стр. 30]{gavrilov}}
  Проверете еквивалентни ли са съждителните формули $\varphi$ и $\psi$ като използвате еквивалентни преобразования.
  \begin{enumerate}[a)]
  \item
    $\varphi = (x\oplus yz)\rightarrow (\overline{x}\rightarrow (y\rightarrow z))$,
    $\psi = x\rightarrow ((y\rightarrow z)\rightarrow x)$;
  \item
    $\varphi = (\overline{x}\vee \overline{y}.z)\rightarrow ((x\rightarrow y)\rightarrow (y\vee z)\rightarrow\overline{x})$,
    $\psi = (x\rightarrow y)\rightarrow(\overline{y}\rightarrow\overline{x})$;
  \item
    $\varphi = (x.\overline{y}\vee \overline{x}.z)\oplus ((y\rightarrow z)\rightarrow \overline{x}.y)$,
    $\psi = (x.(\overline{y}.\overline{z})\oplus y)\oplus z$;
  \item
    $\varphi = x\rightarrow ((\ov{x}.\ov{y}\rightarrow(\ov{x}.\ov{z}\rightarrow y))\rightarrow y).z$,
    $\psi = \ov{x.(y\rightarrow\ov{z})}$.
  \item
    $\varphi = \ov{((x\vee y) \rightarrow y.z)\vee (y\rightarrow x.z)} \vee (x\rightarrow (\ov{y}\rightarrow z))$,
    $\psi = (x\rightarrow y)\vee z$.
  \end{enumerate}
\end{problem}
\begin{solution}
  \begin{enumerate}[a)]
  \item
    Ще направим еквивалентни преобразования върху двете формули докато получим един и същ резултат.
    \begin{align*}
      \psi =\ &  x\rightarrow ((y\rightarrow z)\rightarrow x)\\
      \equiv\ & \overline{x}\vee (\overline{y\rightarrow z}\vee x)\  \equiv\ 1\\
      \varphi =\ & (x\oplus yz)\rightarrow (\overline{x}\rightarrow (y\rightarrow z))\\
      \equiv\ & \overline{(x\vee yz)(\overline{x}\vee\overline{yz})} \vee x\vee \overline{y}\vee z\\
      \equiv\ & \overline{(x\vee yz)}\vee\overline{(\overline{x}\vee\overline{yz})} \vee x\vee \overline{y}\vee z\\
      \equiv\ & \overline{x}.\overline{yz} \vee xyz \vee x\vee \overline{y}\vee z \equiv\ \overline{x}(\overline{y}\vee\overline{z}) \vee x\vee \overline{y}\vee z\\
      \equiv\ & \overline{x}.\overline{y} \vee \overline{x}.\overline{z} \vee x\vee \overline{y}\vee z \equiv\ \overline{x}.\overline{z} \vee x\vee \overline{y}\vee z\\
      \equiv\ & \overline{(x\vee z)} \vee (x\vee z)\vee \overline{y} \equiv 1 \vee \ov{y} \equiv 1.
    \end{align*}
  \item[в)]
    Правим отново същото.
    \begin{align*}
      \psi =\ & (x.(\overline{y}.\overline{z})\oplus y)\oplus z\\
      \equiv\ & x(y\oplus 1)(z\oplus 1) \oplus y \oplus z\\
      \equiv\ & xyz \oplus xy \oplus xz \oplus x \oplus y \oplus z\\
      \varphi =\ & (x\overline{y}\vee \overline{x}z)\oplus ((y\rightarrow z)\rightarrow \overline{x}y)\\
      \equiv\ & (x\ov{y}\vee \ov{x}z) \oplus (\ov{\ov{y}\vee z} \vee \ov{x}y)\\
      \equiv\ & x\ov{y}\oplus \ov{x}z \oplus (y\ov{z} \oplus \ov{x}y) \\
      \equiv\ & x\ov{y}\oplus \ov{x}z\oplus \ov{x}y\ov{z} \oplus y\ov{z} \oplus \ov{x}y \\
      \equiv\ & xy \oplus x \oplus xz \oplus z \oplus (x\oplus 1)y(z\oplus 1) \oplus yz\oplus y \oplus xy \oplus y \\
      \equiv\ & x \oplus xz \oplus z \oplus (x\oplus 1)y(z\oplus 1) \oplus yz\oplus  \\
      \equiv\ & x \oplus xz \oplus z \oplus xyz \oplus yz \oplus xy \oplus y \oplus yz\oplus\\
      \equiv\ & xyz \oplus xy \oplus xz \oplus x \oplus y \oplus z.
     \end{align*}
\end{enumerate}
\end{solution}

\section{Дизюнктивна нормална форма}

\begin{itemize}
\item
  \index{конюнкт}
  {\bf Конюнкт} на променливите $x_1,x_2,\dots,x_n$ представлява съждителна формула от вида 
  \[x^{\sigma_1}_1x^{\sigma_2}_2 \cdots x^{\sigma_n}_n,\]
  където $x^{\sigma_i}_i = x_i$, ако $\sigma_i = 1$ и $x^{\sigma_i}_i = \overline{x}_i$, ако $\sigma_i = 0$.
\item
  \index{дизюнктивна нормална форма}
  Една съждителна формула $\Phi(x_1,\dots,x_n)$ е в {\bf дизюнктивна нормална форма (ДНФ)}, ако
  тя представлява дизюнкция от конюнкти на някои от променливите на $\Phi$.
  Например, формулата 
  \[\Phi(x,y,z) = \ov{x}y \vee z\ov{y}\]
  е в дизюнктивна формална форма.
\item
  \index{съвършена дизюнктивна нормална форма}
  Една съждителна формула $\Phi(x_1,\dots,x_n)$ е в {\bf съвършена дизюнктивна нормална форма (СДНФ)}, ако
  тя е е в ДНФ и всеки конюнкт участват всичките променливи $x_1,\dots,x_n$.
  За една булева функция $f(x_1,\dots,x_n)$, можем да намерим формула $\Phi(x_1,\dots,x_n)$ в СДНФ еквивалентна на нея по следния начин:
  \[\Phi(x_1,\dots,x_n) = \bigvee_{\stackrel{(\sigma_1\dots \sigma_n) \in \{0,1\}^n}{f(\sigma_1, \dots \sigma_n) = 1}}x_1^{\sigma_1}x_2^{\sigma_2}\dots x_n^{\sigma_n}.\]
\end{itemize}

\begin{problem}
  \marginpar{\cite[стр. 50]{gavrilov}}
  С помощта на еквивалентни преобразувания постройте ДНФ на булевите функции
  \begin{enumerate}[a)]
  \item
    \marginpar{$xy\overline{z} \vee \overline{x}z \vee \overline{y}z$}
    $f(x,y,z) = (\ov{x}\vee\ov{y}\vee\ov{z})\cdot(xy\vee z)$;
  \item
    \marginpar{$\overline{x}y\overline{z} \vee xyz \vee x\overline{y}z$}
    $f(x,y,z) = (\overline{x}y\oplus z)\cdot(xz\rightarrow y)$;
  \item
    $f(x,y,z) = (x\vee y\overline{z})\cdot(x\ov{y}\vee\ov{z})\cdot(\ov{xy}\vee z)$;
  \item
    $f(x,y,z,t) = (x\vee y\ov{z}.\ov{t})((\ov{x}\vee t)\oplus yz)\vee \ov{y}\cdot(z\vee \ov{x\ov{t}})$;
  \item
    $f(x,y,z,t) = (x\rightarrow y).(y\rightarrow \ov{z}).(z\rightarrow x\ov{t})$;
  \end{enumerate}
\end{problem}

\begin{problem}
  \marginpar{\cite[стр. 50]{gavrilov}}
  По дадена ДНФ на булевата функция $f$ постройте нейната СДНФ.
  \begin{enumerate}[1)]
  \item
    \marginpar{}
    $f(x,y,z) = xy\vee\ov{z}$;
  \item
    \marginpar{}
    $f(x,y,z) = \ov{x}.\ov{y} \vee y\ov{z} \vee z\ov{z}$;
  \item
    $f(x,y,z) = x\vee yz \vee \ov{x}.\ov{z}$;
  \item
    $f(x,y,z) = x\vee \ov{y}\vee \ov{x}z$;
  \item
    $f(x,y,z,t) = xy\ov{z} \vee xz\ov{t}$;
  \item
    $f(x,y,z,t) = xy \vee \ov{y}t \vee z\ov{t}$.
  \end{enumerate}
\end{problem}

\begin{problem}
  Представете в СДНФ следните булеви функции:
  \begin{enumerate}[1)]
  \item
    $f(x,y,z) = (x\vee y)\rightarrow z$;
  \item
    $f(x,y,z) = (01010001)$;
  \item
    $f(x,y,z) = (11001010)$;
  \item
    $f(x,y,z,t) = (x\rightarrow yzt)(z\rightarrow x\ov{y})$;
  \item
    $f(x,y,z,t) = (x\oplus y)(z\rightarrow \ov{y}t)$;
  \end{enumerate}
\end{problem}


\section{Класовете $T_0$ и $T_1$}
\index{булева функция!запазваща константите}
\index{$T_0,T_1$}

\begin{itemize}
\item 
  Нека $c\in\{0,1\}$. 
  Казваме, че булевата функция $f(\xn)$ запазва константата $c$, ако $f(c,c,\dots,c) = c$.
\item
  Означаваме с $T_0$ функциите, които запазват константата $0$ и с $T_1$ тези, които запазват константата $1$.
\item
  С $T^n_0$ и $T^n_1$ означаваме тези функции, които са на $n$ променливи и принадлежат на $T_0$ или $T_1$ съответно.
\end{itemize}

\begin{problem}
  \marginpar{\cite[стр. 73]{gravrilov}}
  Принадлежи ли функцията $f$ на множеството $T_1 \setminus T_0$ ?
  \begin{enumerate}[a)]
  \item
    \marginpar{Да}
    $f(x,y,z) = (x\rightarrow y)(y\rightarrow z)(z\rightarrow x)$;
  \item
    \marginpar{Да}
    $f(x,y,z) = x\rightarrow(y\rightarrow (z\rightarrow x))$;
  \item
    \marginpar{Да}
    $f(x,y,z) = xyz \vee \ov{x}y \vee \ov{y}$;
  \end{enumerate}
\end{problem}

\begin{problem}
  При какви $n$ функцията $f(x_1,\dots, x_n)$ принадлежи на $T_0\setminus T_1$?
  \begin{enumerate}[1)]
  \item
    $f(\xn) = x_1\oplus x_2 \oplus\dots\oplus x_n$;
  \item
    $f(\xn) = (\bigoplus^{n-1}_{i=1} x_ix_{i+1})\oplus x_nx_1$;
  \end{enumerate}
\end{problem}

\begin{prop}
  Класовете $T_0$ и $T_1$ са затворени, т.е. $[T_0] = T_0$ и $[T_1] = T_1$.
\end{prop}


\section{Самодвойнствени булеви функции}
\index{булева функция!самодвойнствена}
\begin{itemize}
\item 
  Нека е дадена булевата функция $f(\xn)$. Дефинираме булевата функция $f^\star(\xn)$ като
  \[f^\star(\xn) = \overline{f}(\overline{x}_1,\dots,\overline{x}_n).\]
\item
  Ще наричаме $f^\star$ {\bf двойнствена} функция на $f$.
\item
  Ако $f = f^\star$, то ще наричаме $f$ {\bf самодвойнствена} функция.
\item
  Ще означаваме с $S$ множеството от всички самодвойнствени булеви функции, а с $S^n$ тези на $n$ променливи.
\end{itemize}

\begin{prop}
  Класът на самодвойнствените функции е затворен, т.е. $[S] = S$.
  Освен това, $S \subsetneqq \Fs_2$.
\end{prop}

\begin{problem}
  \marginpar{\cite[стр. 31]{gavrilov}}
  Проверете дали функцията $g$ е двойнствена на $f$.
  \begin{enumerate}[1)]
  \item
    \marginpar{Да}
    $f(x,y) = x\rightarrow y$, $g(x,y) = \overline{x}.y$;
  \item
    \marginpar{Не}
    $f(x,y) = (\overline{x}\rightarrow\overline{y})\rightarrow(y\rightarrow x)$, \\
    $g(x,y) = (x\rightarrow y).(\overline{y}\rightarrow\overline{x})$;
  \item
    \marginpar{Да}
    $f(x,y,z) = xy \rightarrow z$,\\
    $g(x,y,z) = \overline{x}.\overline{y}.z$;
  \item
    $f(x,y,z,t) = (x\vee y\vee z).t\vee x.y.z$, \\
    $g(x,y,z,t) = (x\vee y\vee z).t\vee x.y.z$;
  \item
    $f(x,y,z,t) = xy\vee yz\vee zt\vee tx$, \\
    $g(x,y,z,t) = xz\vee yt$;
  \item
    $f(x,y,z,t) = (x\rightarrow y).(z\rightarrow t)$, \\
    $g(x,y,z,t) = (x\rightarrow\overline{z}).(x\rightarrow t).(\overline{y}\rightarrow\overline{z}).(\overline{y}\rightarrow t)$.
  \end{enumerate}
\end{problem}

\begin{problem}
  Проверете самодвойнствена ли е $f$.
  \begin{enumerate}[a)]
  \item
    \marginpar{Не}
    $f(x,y) = x\vee y$;
  \item
    \marginpar{Не}
    $f(x,y) = x\rightarrow y$;
  \item
    \marginpar{Не}
    $f(x,y) = x\oplus y$;
  \item
    \marginpar{Да}
    $f_4(x,y,z) = xy\vee yz\vee zx$;
  \item
    \marginpar{Да}
    $f_5(x,y,z) = x\oplus y\oplus z\oplus 1$;
  \item
    \marginpar{Да}
    $f_6(x,y,z) = xyz\oplus xy\ov{z}\oplus yz\oplus xz$.
  \item
    \marginpar{Не}
    $f_7(x,y,z) = xyz\oplus xy\oplus yz\oplus xz$;
  \item
    \marginpar{Не}
    $f(x,y,z) = (x\rightarrow y)\oplus (y\rightarrow z)\oplus (y\rightarrow x)$;
  \item
    \marginpar{Не}
    $f(x,y,z) = (x\rightarrow y)\oplus (y\rightarrow z)\oplus (z\rightarrow x)\oplus z$;
  \end{enumerate}
\end{problem}
\begin{proof}
  \begin{table}[H]
    \begin{subtable}{0.5\textwidth}
      \begin{tabular}[b]{|c||c|c|c|}
        \hline
        $xz$ & а) & б) & в)\\
        \hline
        $00$ & $0$ & $1$ & $0$ \\
        \hline
        $01$ & $1$ & $1$ & $1$\\
        \hline
        \hline
        $10$ & $1$ & $0$ & $1$\\
        \hline
        $11$ & $1$ & $1$ & $0$\\
        \hline
      \end{tabular}
    \end{subtable}
    \begin{subtable}{0.5\textwidth}
      \begin{tabular}[b]{|c||c|c|c|c|c|}
        \hline
        $xyz$ & г) & д) & е) & ж) & з)\\
        \hline
        $000$ & $0$ & $1$ & $0$ & $0$ & $1$\\
        \hline
        $001$ & $0$ & $0$ & $0$ & $0$ & $1$\\
        \hline
        $010$ & $0$ & $0$ & $0$ & $0$ & $1$\\
        \hline
        $011$ & $1$ & $1$ & $1$ & $1$ & $0$\\
        \hline
        \hline
        $100$ & $0$ & $0$ & $0$ & $0$ & $0$\\
        \hline
        $101$ & $1$ & $1$ & $1$ & $1$ & $0$\\
        \hline
        $110$ & $1$ & $1$ & $1$ & $1$ & $0$\\
        \hline
        $111$ & $1$ & $0$ & $1$ & $0$ & $1$\\
        \hline
      \end{tabular}
    \end{subtable}
  \end{table}
\end{proof}


\begin{problem}
  Проверете дали функцията $f$ е самодвойнствена, ако е зададена векторно:
  \begin{enumerate}[1)]
  \item
    \marginpar{Да}
    $\alpha_f = (01001101)$;
  \item
    \marginpar{Не}
    $\alpha_f = (01100110)$;
  \item
    $\alpha_f = (1100 1001 0110 1100)$;
  \item
    $\alpha_f = (1110 0111 0001 1000)$;
  \item
    $\alpha_f = (1100 0011 0011 1100)$;
  \item
    $\alpha_f = (1001 0110 1001 0110)$;
  \item
    $\alpha_f = (1100 0011 1010 0101)$;
  \end{enumerate}
\end{problem}

\begin{problem}
  Заменете $-$ в $\chi_f$ с $0$ или $1$ за да получите характеристичен вектор на самодвойнствена функция.\\
  \begin{inparaenum}[a)]
  \item
    $\chi_f = (1-0-)$;
  \item
    $\chi_f = (01-0-0--)$;
  \item
    $\chi_f = (--01--11)$;
  \end{inparaenum}
\end{problem}

\section{Полином на Жегалкин}
\index{полином на Жегалкин}
\begin{itemize}
\item 
  Полином на Жегалкин на 2 променливи е формула от вида:
  \[a_0\oplus a_1x_1\oplus a_2x_2  \oplus a_{12}x_1x_2  ,\]
  където $a_0,a_1,a_2,a_{12}$ приемат стойности 0 или 1.
\item
  Полином на Жегалкин на 3 променливи е формула от вида:
  \[a_0\oplus a_1x_1\oplus a_2x_2 \oplus a_3x_3 \oplus a_{12}x_1x_2 \oplus a_{13}x_1x_3 \oplus a_{23} x_2x_3 \oplus a_{123}x_1x_2x_3,\]  
  където $a_0,a_1\dots,a_{123}$ приемат стойности 0 или 1.
\item
  Полином на Жегалкин на $n$ променливи е формула от вида:
  \[a_0 \oplus \bigoplus_{1\leq i\leq n}a_i x_i\oplus \bigoplus_{1\leq i<j \leq n}a_{ij} x_ix_j\oplus \bigoplus_{1\leq i<j<k \leq n}a_{ijk} x_ix_jx_k \dots  \oplus a_{12\dots n} x_1x_2\dots x_n,\]
\end{itemize}

\begin{thm}
  Всяка булева функция има единствен полином на Жегалкин.
\end{thm}
\begin{hint}
  Всеки полином на Жегалкин представя различна булева функция.
  Всички полиноми на Жегалкин на $n$ променливи са $2^{2^n}$.
  Всички булеви функции на $n$ променливи са $2^{2^n}$.
\end{hint}


\begin{problem}
  По метода на неопределените коефициенти, намерете полинома на Жегалкин на функцията 
  \begin{enumerate}[a)]
  \item
    $f(x,y) = x\vee y$;
  \item
    $f(x,y,z) = x\vee y \vee z$;
  \item
    $f(x,y,z) = x\rightarrow (y \rightarrow z)$;
  \item
    $f(x,y,z) = x(y\vee\overline{z})$.
  \end{enumerate}
\end{problem}
\begin{proof}
  \begin{enumerate}[a)]
  \item
    Понеже общият вид на булевата функция е $f(x,y) = a_0\oplus a_1 x \oplus a_2 y \oplus a_3 xy $,
    трябва да намерим коефициентите $a_0,a_1,a_2,a_3$.
    \begin{align*}
      & a_0\oplus a_1 0 \oplus a_2 0 \oplus a_3 0 = 0 \vee 0 = 0\\
      & a_0\oplus a_1 1 \oplus a_2 0 \oplus a_3 0 = 1 \vee 0 = 1\\
      & a_0\oplus a_1 0 \oplus a_2 1 \oplus a_3 0 = 0 \vee 1 = 1\\
      & a_0\oplus a_1 1 \oplus a_2 1 \oplus a_3 1 = 1 \vee 1 = 1.
    \end{align*}
    Следователно, $x\vee y \equiv x\oplus y\oplus xy$.
  \end{enumerate}
\end{proof}

\begin{problem}
  Използвайки еквивалентности от вида $\overline{A} = A\oplus 1$ и $A\vee B = AB\oplus A\oplus B$, 
  намерете полинома на Жегалкин на функцията:
  \begin{enumerate}[a)]
  \item
    \marginpar{$1 \oplus x \oplus xy$}
    $f(x,y) = x\rightarrow y$;
  \item
    \marginpar{$1 \oplus xy \oplus xyz$}
    $f(x,y,z) = (x\rightarrow (y\rightarrow z))$;
  \item
    \marginpar{$x\oplus z\oplus xy\oplus xz \oplus xyz$}
    $f(x,y,z) = ((x\rightarrow y)\rightarrow z)$;
  \item
    \marginpar{$x\oplus z\oplus xy\oplus xz \oplus xyz$}
    $f(x,y,z) = (x\rightarrow (y\rightarrow z)).((x\rightarrow y)\rightarrow z)$;
  \item
    $f(x,y,z,t) = (x\rightarrow y)\rightarrow (z\rightarrow xt)$;
  \item
    $f(x,y,z,t) = x\vee (y\rightarrow ((z\rightarrow y)\rightarrow t)$;
  \item
    $f(x,y,z,t) = (x\vee y\vee z)t \vee xyz$.
  \end{enumerate}
\end{problem}

\section{Линейни функции}
\index{булева функция!линейна}

\begin{itemize}
\item
  Знаем, че всяка булева функция може да се представи {\em по единствен начин}
  с полином на Жегалкин.
\item
  Всяка булева функция $f(\xn)$ с полином на Жегалкин от вида 
  \[a_0\oplus a_1x_1 \oplus a_2x_2 \dots\oplus a_nx_n\] наричаме {\bf линейна}.
\item
  Ще означаваме с $L$ множеството от всички линейни булеви функции, а с $L^n$ тези на $n$ променливи.
\end{itemize}

\begin{problem}
  \marginpar{Отг. $2^n$}
  Колко са всички линейни булеви функции на $n$ променливи?
\end{problem}


\begin{prop}
  Класът на линейните функции е затворен, т.е. $[L] = L$.
  Освен това, $L \subsetneqq \Fs_2$.
\end{prop}


\begin{problem}
  Линейна ли е функцията $f$ с характеристичен вектор $\chi_f = (1001011010010110)$?
\end{problem}

\begin{problem}
  Заменете $-$ в $\chi_f = (-110---0)$ с $0$ или $1$, така че да получите $f$ линейна.
\end{problem}


\begin{problem}
  Проверете дали $f$ е линейна функция.
  \begin{enumerate}
  \item
    \marginpar{Не}
    $f = x\rightarrow y$;
  \item
    \marginpar{Да}
    $f = \ov{x\rightarrow y}\oplus \ov{x}y$;
  \item
    \marginpar{Не}
    $f = xy\vee \ov{x}.\ov{y}\vee z$;
  \item
    \marginpar{Не}
    $f = xy\ov{z}\vee x\ov{y}$;
  \item
    \marginpar{Да}
    $f = (x\vee yz)\oplus xyz$;
  \item
    $f = (x\vee yz)\oplus \ov{x}yz$;
  \item
    $\chi_f = (1100 0011)$;
  \item
    $\chi_f = (1001 0110 0110 1001)$;
  \end{enumerate}
\end{problem}

\begin{problem}
  Заменете $-$ в $\chi_f$ с $0$ или $1$, така че да получите $f$ линейна.
  \begin{enumerate}[a)]
  \item
    $\chi_f = (10-1)$;
  \item
    $\chi_f = (100-0---)$;
  \item
    $\chi_f = (-001--1-)$;
  \item
    $\chi_f = (11-0---1)$;
  \item
    $\chi_f = (-0-1--00)$;
  \item
    $\chi_f = (--10----0--1-110)$;
  \end{enumerate}
\end{problem}
\begin{proof}
  а) $(1001)$; б) $f = 1\oplus x \oplus y\oplus z$; в) $f = 1\oplus x\oplus y\oplus z$ ;
  г) $f = 1\oplus x\oplus y$; д) $f = x\oplus y$;
\end{proof}


\section{Монотонни функции}
\index{булева функция!монотонна}
\begin{itemize}
\item 
  Нека $\alpha = (a_1,a_1,\dots,a_n)$ и $\beta = (b_1,b_2,\dots,b_n)$ са два булеви вектора с равна дължина.
  \marginpar{Това не е лексикографската наредба!}
  Дефинираме релацията $\preceq$ между тях по следния начин.
  \[\alpha \preceq \beta \iff \abs{\alpha} = \abs{\beta}\wedge (\forall i \leq \abs{\alpha})[a_i \leq b_i].\]
  Ето няколко примера:
  \begin{itemize}
  \item 
    $(0,1,0) \preceq (0,1,1)$;
  \item
    $(0,1,0) \not\preceq (1,0,1)$;
  \item
    $(1,0,1) \not\preceq (0,1,0)$.
  \end{itemize}
\item
  Булевата фунция $f(\xn)$ наричаме {\bf монотонна}, ако 
  \[(\forall \alpha,\beta\in J^n_2 )[\alpha\preceq\beta \rightarrow f(\alpha) \leq f(\beta)].\]  
\item
  Ще означаваме с $M$ множеството от всички монотонни булеви функции, а с $M^n$ тези на $n$ променливи.
\end{itemize}

\begin{prop}
  Класът на монотонните функции е затворен, т.е. $[M] = M$.
  Освен това, $M \subsetneqq \Fs_2$.
\end{prop}

\begin{problem}
  Проверете монотонни ли са функциите:
  \begin{enumerate}[a)]
  \item
    \marginpar{Да}
    $f(x,y) = x\rightarrow (y\rightarrow x)$;
  \item
    \marginpar{Не}
    $f(x,y) = x\rightarrow (x\rightarrow y)$;
  \item
    \marginpar{Да}
    $f(x,y) = (x\oplus y)xy$;
  \item
    \marginpar{Да}
    $f(x,y,z) = xy\oplus yz \oplus zx$;
  \item
    \marginpar{Не}
    $f(x,y,z) = xy\oplus yz \oplus zx \oplus x$;
  \end{enumerate}
\end{problem}

\begin{problem}
  За немонотонните функции $f$, намерете съседни $\alpha$, $\beta$, такива че
  $\alpha \prec \beta$ и $f(\alpha) > f(\beta)$.
  \begin{enumerate}[a)]
  \item
    \marginpar{Отг. $\alpha = (010)$, $\beta = (110)$}
    $f = xyz \vee \ov{x}y$;
  \item
    \marginpar{Отг. $\alpha = (010)$, $\beta = (110)$}
    $f = x\oplus y\oplus z$;
  \item
    % \marginpar{Отг. $\alpha = (110)$, $\beta = (111)$}
    $f = xy\oplus z$;
  \item
     % \marginpar{Отг. $\alpha = (010)$, $\beta = (011)$}
    $f = x\vee y\ov{z}$;
  \item
    % \marginpar{Отг. $\alpha = (0111)$, $\beta = (1111)$}
    $f = xz\oplus yt$;
  \item
    % \marginpar{Отг. $\alpha = (1110)$, $\beta = (1111)$}
    $f(x,y,z,t) = (xyt\rightarrow yz)\oplus t$;
  \end{enumerate}
\end{problem}
% \begin{proof}
%   \begin{enumerate}[a)]
%   \item
%     $\alpha = (010)$, $\beta = (110)$;
%   \item
%     $\alpha = (010)$, $\beta = (110)$;
%   \item
%     $\alpha = (110)$, $\beta = (111)$;
%   \item
%     $\alpha = (010)$, $\beta = (011)$;
%   \item
%     $\alpha = (0111)$, $\beta = (1111)$;
%   \item
%     $\alpha = (1110)$, $\beta = (1111)$;
%   \end{enumerate}
% \end{proof}


\section{Пълнота и затворени класове}
\index{базис}
\begin{itemize}
\item 
  Нека $F\subseteq \Fs_2$ е множество от булеви функции.
  С индукция дефинираме следната редица за всяко $n \in \Nat$:
  \begin{align*}
    F_0 = & F\cup \{I^m_k \mid m,k\in\Nat, 1\leq k\leq m\}\\
    F_{n+1} = & F_n\ \cup\\
    & \{h \mid (\exists f, g_1\dots g_m \in F_n)[h(x_1\dots x_k) =  f(g_1(x_1\dots x_k), \dots, g_m(x_1\dots x_k)]\},
  \end{align*}
  Затварянето на $F$ по отношение на суперпозиция наричаме множеството:
  \[[F] = \bigcup_{n\in \mathbb{N}}F_n.\]
\end{itemize}

% Така множеството $[F]$ се задава с индукция от базово множество
% $F\cup \{I^n_k \mid 1\leq k\leq n \}$ по правилото суперпозиция. От
% Тема 1, знаем, че това е най-малкото множество $X$, което съдържа
% базовото множество и е затворено относно суперпозиция.


\begin{dfn}
  \index{пълно множество}
  Нека $F\subseteq \Fs_2$ е множество от булеви функции. 
  $F$ е {\bf пълно} множество, ако $[F] = \Fs_2$.
  Това означава, че всяка булева функция може да се представи като суперпозиция на функции от множеството $F$.
  $F$ се нарича базис, ако не съществува $G \subsetneqq F$, за което $[G] = \Fs_2$.
\end{dfn}

\begin{thm}[Бул]
  \index{Бул}
  Множеството $\{x\vee y,\ov{x},x\wedge y\}$ е пълно.
\end{thm}
\begin{hint}
  Ще докажем, че за всяка булева функция $f \in \Fs_2$ е изпълнено, че
  $f \in [\{x\vee y, \ov{x}, x \wedge y\}]$.
  Ще разгледаме два случая.
  \begin{itemize}
  \item 
    Нека $f = \bf{0}$. Тогава $f(x_1,\dots,x_n) \equiv x_1\wedge\ov{x}_1$.
  \item
    Нека $f \neq \bf{0}$. Тогава лесно се съобразява, че
    \marginpar{Да напомним, че $x^1 = x$, $x^0 = \ov{x}$}
    \[f(x_1,\dots,x_n) \equiv \bigvee_{\stackrel{a_1,\dots,a_n:}{f(a_1,\dots,a_n) = 1}} x^{a_1}_1\dots x^{a_n}_n.\]
  \end{itemize}
\end{hint}

\begin{example}
  Нека да фиксираме булевите функции $c(x,y) = x\wedge y$, $d(x,y) = x\vee y$, $n(x) = \ov{x}$.
  Булевата функция \[f(x,y,z) = \ov{\ov{x}y \vee z}\] се изразява чрез суперпозиция на функциите $c(x,y)$, $d(x,y)$ и $n(x)$
  по следния начин:
  \[f(x,y,z) \equiv n(d(c(n(x),y),z)).\]
\end{example}


\begin{framed}
  \begin{thm}[Критерий за пълнота на Пост-Яблонский]
    \index{Пост-Яблонский}
    Нека $P\subseteq \Fs_2$ е непразно множество от булеви функции. Множеството $P$ е {\em пълно} тогава и само тогава, когато то не е подмножество на 
    нито едно от множествата $T_0,T_1,S,M,L$.
  \end{thm}
\end{framed}

% \begin{proof}
%   Първо, нека $P$ е пълно множество, т.е. $[P] = \Fs_2$.
%   Ако допуснем, че съществува $K \in \{T_0,T_1,S,M,L\}$, за което $P \subseteq K$, то
%   \[\Fs_2 = [P] \subseteq [K] = K,\]
%   от което следва, че $\Fs_2 = K$, което е противоречие.
  
%   Нека сега имаме, че за всяко $K \in \{T_0,T_1,S,M,L\}$, $P \not\subseteq K$.
%   Ще докажем, че $\{\ov{x}, x\vee y\} \subseteq [P]$. Тогава, от теоремата на Бул $\Fs_2 = [\{\ov{x}, x\vee y\}] \subseteq [P]$,
%   ще следва, че $[P] = \Fs_2$.
%   Да фиксираме функциите:
%   \[f_0 \in P\setminus T_0,\ f_1 \in P \setminus T_1,\ f_S \in P\setminus S,\ f_M \in P\setminus M,\ f_L \in P\setminus L.\]
  
%   Можем да приемем, че функциите $f_0$ и $f_1$ са едноаргументни,
%   защото ако например $f_0$ е $n$-аргументна, то ще вземем 
%   $g_0(x) = f_0(x,\dots,x) = f_0(I^1_1(x),\dots,I^1_1(x)) \in [P]$.
%   Имаме, че $f_0(0) = 1$ и $f_1(1) = 0$. 
%   Ще разгледаме четири случая за другите стойности на аргументите:
%   \begin{enumerate}[1)]
%   \item 
%     ако $f_0(1) = 0$ и $f_1(0) = 0$, тогава $f_0 \equiv \ov{x}$, $f_1 \equiv {\bf 0}$ и $f_0(f_1(x)) = 1$;
%   \item
%     ако $f_0(1) = 0$ и $f_1(0) = 1$, тогава $f_0 \equiv f_1\equiv \ov{x}$;
%   \item
%     ако $f_0(1) = 1$ и $f_1(0) = 0$, тогава $f_0 \equiv {\bf 1}$ и $f_1 \equiv {\bf 0}$;
%   \item
%     ако $f_0(1) = 1$ и $f_1(0) = 1$, тогава $f_0 \equiv {\bf 1}$, $f_1 \equiv \ov{x}$ и $f_1(f_0(x)) = 0$.
%   \end{enumerate}
%   В случаите $1)$, $3)$, $4)$ получихме, че ${\bf 0}, {\bf 1} \in [P]$.
%   Ще докажем, че и в случая $2)$ също имаме, че ${\bf 0}, {\bf 1} \in [P]$.
%   За целта ще трябва да разгледаме и функцията $f_S \not\in S$.
%   За нея знаем, че съществуват стойности на аргументите ѝ $a_1,\dots,a_n$, такива че:
%   \[f_S(a_1,\dots,a_n) = f_S(\ov{a}_1,\dots,\ov{a}_n).\]
%   Да разгледаме едноместната функция $g_S$, дефинирана като:
%   \[g_S(x) = f_S(x^{a_1},\dots,x^{a_n}) \in [P].\]
%   \begin{align*}
%     g_S(0) & = f_S(0^{a_1},\dots,0^{a_n}) & (0^0 = 1, 0^1 = 0)\\
%     & = f_S(\ov{a}_1,\dots,\ov{a}_n) = f_S(a_1,\dots,a_n) & (\text{защото } f_S\not\in S)\\
%     & = f_S(1^{a_1},\dots,1^{a_n}) & (1^0 = 0, 1^1 = 1)\\
%     & = g_S(1).
%   \end{align*}
%   Следователно $g_S$ е константа функция. 
%   Понеже сме в случая $2)$, то $\ov{x} \in [P]$ и следователно $\ov{g}_S$ е другата константна функция.
%   Заключаваме, че във всичките четири случая получаваме, че ${\bf 0}, {\bf 1} \in [P]$.
  
%   Сега ще докажем, че имаме $\ov{x} \in [P]$.
%   Да разгледаме $f_M \not\in M$. Това означава, че съществуват вектори $\alpha  \prec \beta$, които се различават
%   само в една стойност и $f_M(\alpha) = 1$, $f_M(\beta) = 0$.
%   Имаме, че  $\alpha = (a_1,\dots,a_{k-1},0,a_{k+1},\dots,a_n)$ и $\beta = (a_1,\dots,a_{k-1},1,a_{k+1},\dots,a_n)$.
%   Да разгледаме едноместната булева функция $g_M$, дефинираната като:
%   \[g_M(x) = f_M(a_1,\dots,a_{k-1},x,a_{k+1},\dots,a_n).\]
%   Понеже вече доказахме, че ${\bf 0},{\bf 1} \in [P]$, то $g_M \in [P]$.
%   Лесно се вижда, че $g_M(x) = \ov{x}$. 

%   Остава да докажем, че $xy \in [P]$.
%   Да разгледаме $f_L \not\in L$. Това означава, че $f_L$ има следния вид:
%   \[f_L(x_1,\dots,x_n) = a_0 \oplus a_1x_1\oplus a_2x_2 \dots\oplus a_n x_n\oplus \dots \oplus a_lx_{i_1}x_{i_2}\dots x_{i_k}\oplus\dots \]
%   Нека $x_{i_1}x_{i_2}\dots x_{i_k}$ е първият нелинеен член в записа на $f_L$.
%   Да разгледаме двуместната булева функция $g_L$, дефинирана като:
%   \[g_L(x_{i_1},x_{i_2}) = a_0 \oplus a_{i_1}x_{i_1} \oplus a_{i_2}x_{i_2} \oplus x_{i_1}x_{i_2},\]
%   т.е. тя е получена от $f_L$ като $x_{j} = 1$ за $j \in \{i_3,\dots,i_k\}$ и $x_j = 0$ за $j \not\in\{i_1,i_2,\dots,i_k\}$.
  
%   Можем да приемем, че $a_0 = 0$, защото ако $a_0 = 1$, то ще разгледаме функцията $g^\prime_L(x_{i_1},x_{i_2}) = \ov{g}_L(x_{i_1},x_{i_2}) \in [P]$.
%   За нея, $g^\prime(0,0) = a_0 = 0$.
%   И така, нека $g_L$ има вида
%   \[g_L(x,y) = ax\oplus by \oplus xy.\]
%   Отново трябва да разгледаме четири случая за стойностите на $a$ и $b$.
%   \begin{itemize}
%   \item 
%     ако $a = 0$, $b = 0$, тогава $g_L(x,y) = xy$;
%   \item 
%     ако $a = 0$, $b = 1$, тогава $g_L(x,y) = y \oplus xy$ и 
%     \[g_L(\ov{x},y) = y\oplus (x\oplus 1)y = y \oplus y \oplus xy = xy;\]
%   \item 
%     ако $a = 1$, $b = 0$, тогава $g_L(x,y) = x\oplus xy$ и
%     \[g_L(x, \ov{y}) = x\oplus x(y\oplus 1) = xy;\]
%   \item 
%     ако $a = 1$, $b = 1$, тогава $g_L(x,y) = x \oplus y \oplus xy$ и
%     \[\ov{g}_L(\ov{x},\ov{y}) = xy.\]
%   \end{itemize}
%   Във всички случаи получихме, че $xy \in [P]$.
% \end{proof}

\begin{example}
  Да проверим дали следното множество от булеви функции е пълно
  $A = \{xy, x\vee y, x\oplus y\oplus z\oplus 1\}$.
  Пресмятаме за всяка от функциите на кои от петте класа принадлежат:
  \newline
  \newline
  \begin{tabular}[b]{|c|c|c|c|c|c|}
    \hline
    & $T_0$ & $T_1$ & $L$ & $S$ & $M$\\
    \hline
    $xy$ & $+$ & $+$ & $-$ & $-$ & $+$\\
    \hline
    $x\vee y$ & $+$ & $+$ & $-$ & $-$ & $+$\\
    \hline
    $x\oplus y\oplus z\oplus 1$ & $-$ & $-$ & $+$ & $+$ & $-$ \\
    \hline
  \end{tabular}
  \newline
  
  От таблицата имаме, че:
  \begin{itemize}
  \item 
    $x \oplus y \oplus z \oplus 1 \not\in T_0$. Следователно, $A \not\subseteq T_0$.
  \item 
    $x \oplus y \oplus z \oplus 1 \not\in T_1$. Следователно, $A \not\subseteq T_1$.
  \item
    $xy \not\in L$. Следователно, $A \not\subseteq L$.
  \item
    $xy \not\in S$. Следователно, $A \not\subseteq S$.
  \item
    $x \oplus y \oplus z \oplus 1 \not\in M$. Следователно, $A \not\subseteq M$.
  \end{itemize}
  Според критерия на Пост-Яблонский, множеството $A$ е пълно.
\end{example}


\begin{problem} %Гаврилов, стр. 83, зад. 6.1
  Пълна ли е системата от функции?
  \begin{enumerate}[a)]
  \item
    $A = \{1, xy(x\oplus z)\}$;
  \item
    $A = \{x\rightarrow y, x\oplus y\}$;
  \item
    $A = \{0, \ov{x}, x(y\oplus z)\oplus yz\}$;
  \item
    $A = \{x\rightarrow y, \ov{x}\rightarrow \ov{y}x, x\oplus y\oplus z, 1\}$;
  \item
    $A = \{\ov{y}\rightarrow\ov{x}, \ov{x}\rightarrow \ov{y}x, x\oplus y\oplus z, 0\}$;
  \item
    $A = \{\ov{y}\rightarrow \ov{x}z, (y\vee \ov{x} \rightarrow x, x\oplus y\oplus z, 1\}$;
  \item
    $A = \{x\oplus z \oplus 1, x \to \ov{y}, x\oplus (y \vee z) \oplus 1\}$;
  \item
    $A = \{1,\ov{x}, x(y\leftrightarrow z)\oplus\ov{x}(y\oplus z), x\leftrightarrow y\}$;
  \item
    $A = \{\ov{x}, x(y\leftrightarrow z) \leftrightarrow yz, x \oplus y \oplus z\}$;
  \item
    $A = \{\ov{x}, x(y\leftrightarrow z) \leftrightarrow (y\vee z), x \oplus y \oplus z\}$;
  \item
    $A = \{\chi_{f_1} = (0110), \chi_{f_2} = (1100 0011), \chi_{f_3} = (1001 0110)\}$;
  \item
    $A = \{\chi_{f_1} = (11), \chi_{f_2} = (00), \chi_{f_3} = (0011 0101)\}$;
  \end{enumerate}
\end{problem}
\begin{solution}
  \begin{figure}[H]
    \begin{subfigure}[b]{0.5\textwidth}
      \begin{tabular}[b]{|c|c|c|c|c|c|}
        \hline
        & $T_0$ & $T_1$ & $L$ & $S$ & $M$\\
        \hline
        $1$ & $-$ & $+$ & $+$ & $-$ & $+$\\
        \hline
        $xy(x\oplus z)$ & $+$ & $-$ & $-$ & $-$ & $-$\\
        \hline
      \end{tabular}
      \caption{}      
    \end{subfigure}
    \begin{subfigure}[b]{0.5\textwidth}
      \begin{tabular}[b]{|c|c|c|c|c|c|}
        \hline
        & $T_0$ & $T_1$ & $L$ & $S$ & $M$\\
        \hline
        $x\rightarrow y$ & $-$ & $+$ & $-$ & $-$ & $+$\\
        \hline
        $x\oplus y$ & $+$ & $-$ & $+$ & $-$ & $-$\\
        \hline
      \end{tabular}
      \caption{}
    \end{subfigure}
  \end{figure}
\end{solution}

\begin{problem} % Гаврилов, стр. 83
  Проверете пълно ли е множеството от булеви функции:
  \begin{enumerate}[a)]
  \item
    \marginpar{Да}
    $A = (S\cap M)\cup(L\setminus M)$;
  \item
    $A = ((L\cap M)\setminus T_1)\cup (S\cap T_1)$;
  \item
    $A = (L\cap M)\cup (S\setminus T_0)$;
  \item
    $A = (L\cap T_1)\cup (S\cap M)$;
  \item
    $A = (M\setminus S)\cup(L\cap S)$;
  \item
    $A = (M\setminus T_0)\cup (L\setminus S)$;
  \item
    $A = (M\setminus T_0) \cup (S\setminus L)$.
  \end{enumerate}
\end{problem}
\begin{solution}
  Във всяка една от задачите трябва да проверим дали
  $A \not\subseteq T_0$, 
  $A \not\subseteq T_1$, 
  $A \not\subseteq S$, 
  $A \not\subseteq M$ и 
  $A \not\subseteq L$.
  \begin{enumerate}[a)]
  \item 
    Нека $A = (S \cap M) \cup (L\setminus M)$.
    \begin{itemize}
    \item 
      Да разгледаме функцията 
      $f(x) = x \oplus 1$.
      Лесно се съобразява, че $f \in L\setminus M$, откъдето  следва, че $f \in A$.
      Обаче ние имаме, че $f \not\in T_0, T_1, M$.
      Следователно, $A \not\subseteq T_0, T_1, M$.
    \item
      Да разгледаме $g(x,y) = x\oplus y \oplus 1$.
      За нея имаме, че $g \in A$, защото $g \in L\setminus M$ и освен това $g \not\in S$.
    \item
      Остана да проверим, че $A \not\subseteq L$.
      Да разгледаме 
      \[h(x,y,z) = xy\oplus yz \oplus xz.\]
      Тогава $h \in A$, защото $h \in S\cap M$.
      Ясно е, че $h \not\in L$.
    \end{itemize}
  \item
    Нека $A = ((L\cap M)\setminus T_1)\cup (S\cap T_1)$.
    \begin{itemize}
    \item 
      Ако една функция е монотонна, но не запазва $1$-цата, то тогава 
      със сигурност тази функция е константата $0$, т.е.
      \[(L \cap M)\setminus T_1 = \{0\}.\]
      % $0 \not\in S$ и оттук $A \not\subseteq S$.
    \item
      Ако $f \in S \cap T_1$, то това означава $f(1,\dots,1) = 1$ и $f(0,\dots,0) = 0$.
      Следователно, $S \cap T_1 \subseteq T_0$.
      Получаваме, че $A =  \{0\} \cup (S \cap T_1) \subseteq T_0$.
    \end{itemize}    
    Следователно $A$ не е пълен клас.
  \item
    Нека $A = (L\cap M)\cup (S\setminus T_0)$.
    \begin{itemize}
    \item
      $0 \in L \cap M$ и следователно $A \not\subseteq T_1$;
    \item
      $1 \in L \cap M$ и следователно $A \not\subseteq T_0$;
    \item
      И двете константи са в $L \cap M$, но както знаем, те  не са самодвойнствени.
      Следователно $A \not\subseteq S$.
    \item
      Да разгледаме
      \[h(x,y,z) = xy\oplus xz \oplus yz \oplus 1.\]
      Лесно се съобразява, че $h \in S \setminus T_0$, но $h \not\in L$.
      Следователно, $A \not\subseteq L$.
      Освен това, $h \not \in M$. Следователно, $A \not\subseteq M$.
    \end{itemize}
  \item
    Нека $A = (L \cap T_1) \cup (S \cap M)$.
    Ако $f \in S \cap M$, то $f$ не е константа и $f(1,\dots,1) = 1$.
    Следователно, $f \in T_1$.
    Получаваме, че $S \cap M \subseteq T_1$.
    Заключаваме, че $A \subseteq T_1$.
  \item
    Нека $A = (M\setminus S)\cup(L\cap S)$.
    \begin{itemize}
    \item 
      $x\oplus 1 \in L \cap S$, но $x \oplus 1 \not\in T_0, T_1$.
      Следователно, $A \not\subseteq T_0,T_1$.
      Освен това, $x\oplus 1 \not\in M$.
      И така, $A \not\subseteq M$.
    \item
      Нека $f(x,y) \equiv x \vee y \equiv xy \oplus x \oplus y \oplus 1$.
      Имаме, че $f \in M \setminus S$ и $f \not\in L$.
      Следователно, $A \not\subseteq L$ и $A \not\subseteq S$.
    \end{itemize}
  \item
    Нека $A = (M \setminus T_0) \cup (L \setminus S)$.
    Имаме, че $M \setminus T_0 = \{1\}$.
    Следователно, $A \subseteq L$.
  \item
    Нека $A = (M \setminus T_0) \cup (S \setminus L)$.
    \begin{itemize}
    \item 
      Имаме, че $1 \in M\setminus T_0$.
      Следователно, $A \not\subseteq T_0$ и $A \not\subseteq S$.
    \item
      Нека $h(x,y,z) \equiv xy \oplus xz \oplus yz \oplus 1$.
      Тогава $h \in S\setminus L$ и $h \not\in T_1$, $h \not \in M$.
      Заключаваме, че $A \not\subseteq T_1, M, L$.
    \end{itemize}
  \end{enumerate}
\end{solution}

\begin{problem} % Гаврилов, стр. 84
  Проверете дали системата от функции $A$ е базис?
  \begin{enumerate}[a)]
  \item
    \marginpar{Не}
    $A = \{x\rightarrow y, x\oplus y, x\vee y\}$;
  \item
    \marginpar{Да}
    $A = \{x\oplus y\oplus z, x\vee y, 0, 1\}$;
  \item
    \marginpar{Не, $A \subseteq T_1$}
    $A = \{x\oplus y\oplus yz, x \oplus y \oplus 1\}$;
  \item
    \marginpar{Не}
    $A = \{xy \vee z, xy \oplus z, xy \iff z\}$;
  \end{enumerate}
\end{problem}

\begin{problem}
  Намерете всички базиси на класа $A$, където:
  \begin{enumerate}[a)]
  \item 
    $A = \{1, \ov{x}, xy(x\oplus y), x \oplus y \oplus xy \oplus yz \oplus zx\}$;
  \item
    $A = \{0, x\oplus y, x \to y, xy \iff xz\}$;
  \item
    $A = \{0,1,x\oplus y \oplus z, xy \oplus zx \oplus yz, xy \oplus z, x \vee y\}$;
  \item
    $A = \{xy, x\vee y, xy\vee z, x\oplus y, x \to y\}$.
  \end{enumerate}
\end{problem}

\begin{problem}
  Намерете броя на булевите функции на $n$ променливи, които принадлежат на следните класове:
  \begin{enumerate}[a)]
  \item
    \marginpar{$2^{2^n-1}$}
    $T_0$, $T_1$;
  \item
    \marginpar{$2^{2^n-2}$}
    $T_0 \cap T_1$
  \item
%    \marginpar{$3.2^{2^n-2}$}
    $T_0 \cup T_1$
  \item
 %   \marginpar{$2^{2^n-2}$}
    $T_0 \setminus T_1$;
  \item
    \marginpar{$2^{2^{n-1}}$}
    $S$;
  \item
    $T_0 \cap S$, $T_1 \cap S$;
  \item
    $T_0 \cap T_1 \cap S$;
  \item
    $S \setminus T_0$, $S \setminus (T_0 \cap T_1)$, $S \setminus (T_0 \cup T_1)$;
  \item
    \marginpar{$2^{n+1}$}
    $L$;
  \item
    $T_0 \cap L$, $T_1 \cap L$;
  \item
    $T_0 \cap T_1 \cap L$;
  \item
    $M \setminus T_1$;
  \item
    $M \setminus T_0$;
  \end{enumerate}
\end{problem}


\section*{Литература}

\begin{itemize}
\item 
  \cite[Глави 1 и 2]{gavrilov} предлага много задачи за булеви функции.
  Част от задачите в тази глава са взети оттам.
\end{itemize}

%%% Local Variables: 
%%% mode: latex
%%% TeX-master: "discrete-math"
%%% End: 

% \chapter{Езици и автомати}

\begin{lemma}[за разрастването (регулярни езици)]
  % \index{лема за разрастването!регулярни езици}
  % \label{lem:pumping-reg}
  % \marginpar{На англ.\\ Pumping Lemma}
  Нека $\Ls$ да бъде безкраен регулярен език.
  Съществува число $n\geq 1$, зависещо само от $\Ls$, 
  за което за всяка дума $\alpha\in \Ls, \abs{\alpha}\geq n$ може да 
  бъде записана във вида $\alpha = xyz$ и 
  \begin{enumerate}
  \item
    $|y|\geq 1$;
  \item
    $|xy|\leq n$;
  \item
    % \marginpar{$i = 0\ \rightarrow\ xz \in \Ls$}
    $(\forall i\in\N)[xy^iz \in \Ls]$.
  \end{enumerate}
\end{lemma}

\begin{crl}
  Регулярният език $\Ls$, 
  разпознаван от КДА $M$ е непразен тoчно тогава, когато съдържа дума $\alpha, \abs{\alpha} \leq \abs{Q}$.
\end{crl}

\section{Регулярни езици}
\begin{problem}
  Нека $\Sigma = \{0,1\}$.  Проверете дали $L$ е регулярен, където
  \begin{enumerate}[1)]
  \item
    $L_1 = \{0^i1^i\ \mid\ i\geq 0\}$;
  \item
    $L_2 = \{0^i1^j\ \mid\ i > j\}$;
  \item
    $L_3 = \{0^{2n}\ \mid\ i\geq 1\}$;
  \item
    $L_4 = \{0^1m1^n0^{m+n}\ \mid\ m\geq 1\ \&\ n\geq 1\}$;
  \item
    $L_5 = \{0^n\ \mid\ n\mbox{ е просто }\}$;
  \item
    $L_6 = \{w\mid w\in\{0,1\}^\star\mbox{ има равен брой нули и единици}\}$;
  \item
    $L_7 = \{ww\mid w\in\{0,1\}^\star\}$;
  \item
    $L_8 = \{1^{n^2}\mid n\geq 0\}$;
  \item
    $L_{9} = \{0^n1^n2^n\mid n\geq 0\}$;
  \item
    $L_{10} = \{www\mid w\in \{0,1\}^\star\}$;
  \item
    $L_{11} = \{0^{2^n}\mid n\geq 0\}$;
  \item
    $L_{12} = \{0^m1^n\mid n\neq m\}$;
  \end{enumerate}
\end{problem}


%%% Local Variables: 
%%% mode: latex
%%% TeX-master: "discrete-math"
%%% End: 
?
% 
\chapter{Контекстно-свободни езици}

\section{Свойства}

Някои свойства на контекстно-свободните езици:
\begin{itemize}
\item 
  те са затворени относно операцията обединение, т.е.
  ако $L_1, L_2$ са контекстно-свободни, то езикът $L_1 \cup L_2$ е контекстно-свободен; 
\item
  те са затворени относно операцията конкатенация, т.е.
  ако $L_1, L_2$ са контекстно-свободни, то езикът $L_1 \cdot L_2$ е контекстно-свободен; 
\item
  те са затворени относно операцията звезда на Клини, т.е.
  ако $L$ е контекстно-свободен, то езикът $L^\star = \bigcup_{n\in\Nat} L^n$ е контекстно-свободен; 
\item
  те {\bf не} са затворени относно операцията сечение, т.е.
  съществуват контекстно-свободни езици $L_1, L_2$, за които езикът $L_1 \cap L_2$ {\bf не} е контекстно-свободен; 
\item
  те са затворени относно сечение с регулярен език, т.е.
  ако $L$ е контекстно-свободен език и $R$ е регулярен език, то езикът $L = L \cap R$ е контекстно-свободен; 
\item
  те {\bf не} са затворени относно операцията допълнение, т.е.
  съществува контекстно-свободен език $L$, за който езикът $\Sigma^\star\setminus L$ {\bf не} е контекстно-свободен; 
\item
  те са затворени относно хомоморфизми, т.е.
  ако $L \subseteq \Sigma^\star_1$ е контекстно-свободен език и $h:\Sigma_1\to\Sigma^\star_2$ е хомоморфизъм, 
  то езикът $h(L) = \{h(\alpha) \in \Sigma^\star_2 \mid \alpha \in L\}$
  е контекстно-свободен.
\item
  те са затворени относно обратни хомоморфизми, т.е.
  ако $L\subseteq \Sigma^\star_2$ е контекстно-свободен език и $h:\Sigma_1\to\Sigma^\star_2$ е хомоморфизъм, 
  то езикът $h^{-1}(L) = \{\alpha \in \Sigma^\star_1 \mid h(\alpha) \in L\}$
  е контекстно-свободен.
\end{itemize}

\section{Контекстно-свободни граматики}
% От Сипсер, същото е в слайдовете на Сашка
% Малко е тъпо, че в Пападимитриу дефиницията е различна. Там \Sigma \subseteq V
\begin{dfn}
  \marginpar{На англ. context-free grammar}
  Контекстно-свободна граматика e четворка \[G = (V,\Sigma,R,S),\]
  където
  \begin{itemize}
  \item
    $V$ е крайно множество от {\em променливи};
  \item
    $\Sigma$ е крайно множество от {\em терминали}, $\Sigma \cap V = \emptyset$;
  \item
    $R \subseteq V\times (V\cup\Sigma)^\star$, крайно множество от {\em правила};
  \item
    $S \in V$ е началната променлива. 
  \end{itemize}
  Обикновено ще пишем $A \rightarrow_G v$ вместо $(A,v) \in R$.
  Пишем $u \Rightarrow_G v$, ако съществуват думи $x,y\in (\Sigma\cup V)^\star$, $A\in V$,
  правило $A\rightarrow_G v^\prime$ и $u = xAy$, $v = xv^\prime y$.
  Езикът генериран от $G$, $L(G) = \{\alpha\in\Sigma^\star\mid S \Rightarrow^\star_G \alpha\}$.
\end{dfn}

\begin{problem}
  Да се докаже, че езикът $\{\alpha \in \{a,b\}^\star\mid n_a(\alpha) = n_b(\alpha)\}$ 
  е контекстно свободен.
\end{problem}
\begin{proof}
  $S \rightarrow aSbS\vert bSaS \vert\varepsilon$  и да се докаже, че
  ако $\alpha = a\alpha^\prime$ и $n_a(\alpha) = n_b(\alpha)$,
  то съществуват $\alpha_1, \alpha_2$, $\alpha = a\alpha_1b\alpha_2$ и
  $n_a(\alpha_1) = n_b(\alpha_1)$, $n_a(\alpha_2) = n_b(\alpha_2)$.
  Аналогично е, ако $\alpha = b\alpha^\prime$.

  Алтернативна граматика е $S\rightarrow aB\vert bA, A\rightarrow a\vert aS\vert bAA, B\rightarrow b\vert bS\vert aBB$.
  
  Да се обясни защо граматиката $S\rightarrow aB\vert Ba\vert \varepsilon, B\rightarrow bS\vert Sb$ не работи.
\end{proof}

\begin{problem}
  Докажете, че следните езици са контекстно-свободни:
  \begin{enumerate}[1)]
  \item
    $\{ww^R \mid w \in \{a,b\}^\star\}$;
  \item
    $\{w \in \{a,b\}^\star \mid w = w^R\}$;
  \item
    $\{a^nb^{2n} \mid n \in \Nat\}$;
  \item
    $\{a^nb^k \mid n > k\}$;
  \item
    $\{a^nb^k \mid n \geq 2k\}$;
  \item
    $\{a^mb^nc^k\mid m+n \geq k\}$;
  \item
    $\{a^nb^mc^{n+m}\mid n,m \in \Nat\}$;
  \end{enumerate}
\end{problem}
\begin{proof}
  За 2), $S \rightarrow aSa \vert bSb \vert a\vert b \vert \varepsilon$.
  Докажете, че ако $w = w^R$, то $\abs{w} = 2n$, $w = w_1w^R_1$ и ако $\abs{w} = 2n+1$, то $w = w_1aw^R_1$, $w = w_1bw^R_1$.
  За 6), $S \rightarrow aSc\vert aS \vert B, B\vert bSc\vert bS\vert\varepsilon$.
\end{proof}


\section{Езици, които не са контекстно-свободни}

\begin{lemma}[за нарастването (контекстно-свободни езици)]
  \index{лема за нарастването!контекстно-свободни езици}
  \label{lem:pumping-context} 
  За всеки КСЕ $L\neq\{\epsilon\}$ съществува $n>0$ такова,
  че ако $\alpha\in L, \abs{\alpha} \geq p$, то $\alpha=xyuvw$ и
  \begin{enumerate}
  \item
    $\abs{yv}\geq 1$,
  \item
    $\abs{yuv}\leq p$, и
  \item
    $(\forall i\geq 0)[xy^iuv^iw\in L]$.
\end{enumerate}
\end{lemma}

\begin{crl}
  Нека $G$ е контекстно-свободна граматика и $n$ е константата за разрастването на $G$.
  Тогава $\abs{L(G)} = \infty$ точно тогава, когато съществува $z \in L(G)$, за която $n \leq \abs{z} \leq 2n$.
\end{crl}

\begin{problem}
  Да се даде пример за език $L$, който {\bf не} е контекстно-свободен, но удовлетворява
  лемата за разрастването.
\end{problem}

\begin{example}
  Приложете лемата за нарастването за да докажете, че 
  следните езици не са контекстно-свободни:
  \begin{itemize}
  \item
    $L_1 = \{a^ib^jc^k\ \mid\ 0 \leq i \leq j \leq k\}$;
  \item
    $L_2 = \{ww\mid w\in \{a,b\}^\star\}$;
  \end{itemize}
\end{example}
\begin{proof}
  \begin{enumerate}[1)]
  \item
    Разгледайте $w = a^pb^pc^p$.
    \begin{enumerate}[a)]
    \item
      Знаем, че поне една от $y$ и $v$ не е празната дума.
      Имаме три случая за поддумите $y$ и $v$.
      \begin{enumerate}[i)]
      \item
        $a$ не се среща в $y$ и $v$.
        Тогава $xy^0vu^0w$ съдържа повече $a$ от $b$ или $c$.
      \item
        $b$ не се среща в $y$ и $v$.
        Ако $a$ се среща в $y$ или $v$, тогава $xy^2uv^2w$ съдържа повече $a$ от $b$
        Ако $c$ се среща в $y$ или $v$, тогава $xy^0uv^0w$ съдържа по-малко $c$ от $b$.
      \item
        $c$ не се среща в $y$ и $v$.
        Тогава $xy^2uv^2w$ съдържа повече $a$ или $b$ от $c$.
      \end{enumerate}      
    \item
      $y$ или $v$ е съставена от две букви.  Контекстно-свободните езици {\bf не} са затворени относно сечение и допълнение.
      Тук разглеждаме $xy^2uv^2w$ и съобразяваме, че редът на буквите е нарушен.
    \end{enumerate}
  \item
    \marginpar{Защо $\alpha = a^pba^pb$ не е добър кандидат?}
    Разгледайте $\alpha = a^pb^pa^pb^p$.
    \begin{enumerate}[a)]
    \item
      Ако $yuv$ е в първата част на думата, то 
      $xy^0uv^0w = a^ib^ja^pb^p \not\in L_3$.
      Аналогично ако $yuv$ е във втората част на думата.
    \item
      Ако $yuv$ е в двете части на думата, то 
      Но $xy^0uv^0w = a^pb^ia^jb^p \not\in L_3$.
    \end{enumerate}    
  \end{enumerate}
\end{proof}


\begin{problem}
  Проверете дали следните езици са контекстно-свободни:
  \begin{enumerate}[a)]
  \item
    $\{a^nb^{2n}c^{3n}\ \mid\ n\in\N\}$;
  \item
    $\{a^mb^n\mid\ m \neq n\}$;
  \item
    $\{www\mid w\in \{a,b\}^\star\}$;
  \item
    $\{ww^R\mid w\in \{a,b\}^\star\}$;
  \item
    $\{a^{n^2}b^n\ \mid n \in \Nat\}$;
  \item
    $\{a^p\ \mid\ p\mbox{ е просто }\}$;
  \item
    $\{a^nb^na^nb^n\mid n\geq 0\}$;
  \item
    $\{w \in \{a,b\}^\star \mid w = w^R\}$;
  \item
    % Дефиниция на подниз
    $\{w c x\mid w,x\in \{a,b\}^\star\ \&\ w\mbox{ е подниз на }x\}$;
  \item
    $\{x_1 c x_2 c \dots c x_k\mid k\geq 2\ \&\ x_i\in\{a,b\}^\star\ \&\ (\exists i,j)[i \neq j\ \&\ x_i = x_j]\}$;
  \item
    $\{a^ib^jc^k\mid i,j,k\geq 0\ \&\ (i = j \vee j = k)\}$;
  \item
    $\{\alpha \in \{a,b,c\}^\star\mid n_a(\alpha) = n_b(\alpha) = n_c(\alpha)\}$;
  \item
    $\{a,b\}^\star \setminus \{a^nb^n\mid n\in \Nat\}$;
  \end{enumerate}
\end{problem}
% \begin{proof}
%   \begin{enumerate}
  % \item
    % За думата $w = a^pb^pc^p = xyuvw$ разгледайте различните случаи за $y$ и $v$.
  % \item[2)]
  %   Разгледайте $w = a^pb^pc^p$.
  %   \begin{enumerate}[a)]
  %   \item
  %     Знаем, че поне една от $y$ и $v$ не е празната дума.
  %     Имаме три случая за поддумите $y$ и $v$.
  %     \begin{enumerate}[i)]
  %     \item
  %       $a$ не се среща в $y$ и $v$.
  %       Тогава $xy^0vu^0w$ съдържа повече $a$ от $b$ или $c$.
  %     \item
  %       $b$ не се среща в $y$ и $v$.
  %       Ако $a$ се среща в $y$ или $v$, тогава $xy^2uv^2w$ съдържа повече $a$ от $b$
  %       Ако $c$ се среща в $y$ или $v$, тогава $xy^0uv^0w$ съдържа по-малко $c$ от $b$.
  %     \item
  %       $c$ не се среща в $y$ и $v$.
  %       Тогава $xy^2uv^2w$ съдържа повече $a$ или $b$ от $c$.
  %     \end{enumerate}      
  %   \item
  %     $y$ или $v$ е съставена от две букви.  Контекстно-свободните езици {\bf не} са затворени относно сечение и допълнение.
  %     Тук разглеждаме $xy^2uv^2w$ и съобразяваме, че редът на буквите е нарушен.
  %   \end{enumerate}
  % \item[3)]
  %   \marginpar{Защо $\alpha = a^pba^pb$ не е добър кандидат?}
  %   Разгледайте $\alpha = a^pb^pa^pb^p$.
  %   \begin{enumerate}[a)]
  %   \item
  %     Ако $yuv$ е в първата част на думата, то 
  %     $xy^0uv^0w = a^ib^ja^pb^p \not\in L_3$.
  %     Аналогично ако $yuv$ е във втората част на думата.
  %   \item
  %     Ако $yuv$ е в двете части на думата, то 
  %     Но $xy^0uv^0w = a^pb^ia^jb^p \not\in L_3$.
  %   \end{enumerate}
%   \item[10)]
%     Контекстно-свободен е. Лесно може да се напише контекстно-свободна граматика за този език.
%   \item[12)]
%     Разгледайте езика $L = L_{12} \cap a^\star b^\star c^\star$.
%   \end{enumerate}
% \end{proof}

\begin{problem}
  Проверете кои от следните езици са контекстно-свободни:
  \begin{enumerate}[a)]
  \item
    $\{a^mb^nc^k\mid m = n \vee n = k \vee m = k\}$;
  \item
    $\{a^mb^nc^k\mid m \neq n \vee n \neq k \vee m \neq k\}$;
  \item
    $\{a^mb^nc^k\mid m = n \wedge n = k \wedge m = k\}$;
  \item
    $\{w \in \{a,b,c\}^\star\mid n_a(w) \neq n_b(w) \vee n_a(w) \neq n_c(w) \vee n_b(w) \neq n_c(w)\}$.
  \end{enumerate}
\end{problem}


\section{Алгоритми}

\subsection{Нормална Форма на Чомски}

\begin{problem}
  Нека е дадена граматиката  $G = \pair{\{S,A,B,C,D,E\}, \{a,b\},S, R}$.
  \begin{enumerate}[a)]
  \item
    Намерете всички нетерминали, от които в $G$ се извежда празната дума.
  \item
    Принадлежи ли празната дума на $L(G)$?
  \item
    Постройте граматика $G_1$ без $\varepsilon$-правила, за която $L(G_1)=L(G)\setminus\{\varepsilon\}$.
  \end{enumerate}
  Множеството от правила $R$ на граматиката $G$ е зададено като:
  \begin{enumerate}
  \item
    $R = \{S\rightarrow D,D\rightarrow AD|b,A\rightarrow ACB|BC|a, B\rightarrow ABCA|CEC,C\rightarrow \varepsilon|CA|a, E\rightarrow \varepsilon|aEb\}$;
  \item
    $R = \{S \rightarrow aD, D\rightarrow \varepsilon|ABBA|ADD,A\rightarrow DEB|a,B\rightarrow DDD|DC|b,C\rightarrow CCE|a, E\rightarrow \varepsilon|bEa\}$;
  \item
    $R = \{ S\rightarrow D,D\rightarrow AD|b,A\rightarrow AB|BC|a, B\rightarrow AB|CC, C\rightarrow \varepsilon|CA|a, E\rightarrow a|EB\}$;
  \item
    $R = \{ S \rightarrow AD|a, D\rightarrow \varepsilon|BB|AD,A\rightarrow DB|a,B\rightarrow DD|DC|b,C\rightarrow CE|a, E\rightarrow AB|b|EA\}$;
  \item
    $R =\{S\rightarrow AS|SB|SS,B\rightarrow CA|b, C\rightarrow AA|a|BA,A\rightarrow \varepsilon|BS\}$;
  \item
    $R = \{S\rightarrow AB|AC,B\rightarrow \varepsilon |BC|b,A\rightarrow BB|CC|a,C\rightarrow CS|a\}$;
  \item
    $R = \{S\rightarrow AS|SB|SS,B\rightarrow AC|b, C\rightarrow A|a|AB,A\rightarrow \varepsilon|BS\}$;
  \item
    $R = \{S\rightarrow BA|CA,B\rightarrow \varepsilon |BC|b,A\rightarrow BB|CC|a, C\rightarrow CS|a\}$;
  \item
    $R = \{S\rightarrow AS|b,A\rightarrow AC|BC|a, B\rightarrow BC|CC,C\rightarrow \varepsilon|CA|a\}$;
  \item
    $R = \{S\rightarrow \varepsilon|BA|AS,A\rightarrow SB|a,B\rightarrow SS|SC|b,
    C\rightarrow CC|a\}$; 
  \end{enumerate}
\end{problem}

\begin{problem}
  Нека е дадена граматиката  $G = \pair{\{S,A,B,C\}, \{a,b\}, S, R}$.
  Използвайте обща конструкция, за да премахнете "дългите" правила 
  (т.е. с дължина поне 2, които не са в н.ф. на Чомски) от $ G$ като при това получите к.св. граматика $G_1$ 
  с език $L(G)=L(G_1)$, където:
  \begin{enumerate}
  \item
    $R = \{S \rightarrow \varepsilon|ab|aAba, A\rightarrow aBCb, B\rightarrow bbb, C\rightarrow aC\vert aCaC\}\rangle$;
  \item
    $R = \{S \rightarrow \epsilon|ab|baAb, A\rightarrow BaBb,B\rightarrow b,C\rightarrow AbA\vert aCCa\}$;
  \item
    $R = \{A\rightarrow BSB|a,B\rightarrow ba|BC,C\rightarrow BaSA|a|b,S\rightarrow CC|b\}$;
  \item
    $R = \{A\rightarrow BAS,B\rightarrow CB,C\rightarrow ab|ABbS,S\rightarrow CC|b\}$;
  \end{enumerate}
\end{problem}


\begin{problem}
  Използвайте обща конструкция, за да премахнете преименуващите правила от граматиката $G$ като при това запазите езика,
  където $G = \pair{\{A,B,C,S\},\{a,b\}, S, R}$ и
  \begin{enumerate}
  \item
    $R = \{A\rightarrow B|S,B\rightarrow C|BC,C\rightarrow AB|a|b,S\rightarrow B|CC|b\}$;
  \item
    $R = \{A\rightarrow B,B\rightarrow S|C|BC,C\rightarrow a|AB,S\rightarrow C|CC|b\}$;
  \item
    $R = \{A\rightarrow B|CC|a,B\rightarrow S|AB,C\rightarrow SC|b,S\rightarrow A|CC|b\}$;
  \item
    $R = \{A\rightarrow BB|b,B\rightarrow S|SS|b,C\rightarrow B|a,S\rightarrow C|AB|a\}$;
  \item
    $R = \{S\rightarrow A|a,A\rightarrow B|C|b, B\rightarrow AB, C\rightarrow CC|a\}$;
  \item
    $R = \{S\rightarrow A|B, A\rightarrow a|C|AB, B\rightarrow b|C, C\rightarrow CS|a|b\}$;
  \end{enumerate}
\end{problem}

\begin{problem}
  Намерете контекстно-свободна граматика в нормална форма на Чомски за езиците от задача 6.
  
\end{problem}


\subsection{Проблемът за принадлежност}

\begin{problem}
  Нека е дадена граматиката $\Gamma=\pair{\{a,b\}, \{S,A,B,C\},S,R}$.
  Използвайте алгоритъма за динамично програмиране (CYK), за да проверите дали
  думата $\alpha$ принадлежи на $L(G)$, където правилата на граматиката $R$ и думата $\alpha$
  са зададени като:
  \begin{enumerate}
  \item
    $R =\{S\rightarrow a| AB|AC, C\rightarrow SB|AS,A\rightarrow a, B\rightarrow b\}$, $\alpha=aaabb$;
  \item
    $R = \{S\rightarrow BA| CA|a, C\rightarrow BS|SA,A\rightarrow a, B\rightarrow b\}$, $\alpha=bbaaa$;
  \item
    $R =\{S\rightarrow AB|BC, A\rightarrow BA|a,B\rightarrow CC|b, C\rightarrow AB|a\}$, $\alpha=baaba$;
  \item
    $R = \{S\rightarrow AB, A\rightarrow AC|a|b,B\rightarrow CB|a, C\rightarrow a\}$, $\alpha=babaa$;
  \item
    $R = \{S\rightarrow BA|SS|b, A\rightarrow SA|a,B\rightarrow BS|b\}$, $\alpha = bbbaa$;
  \item
    $R = \{S\rightarrow AB| BS|b, A\rightarrow SS|a,B\rightarrow BA|b\}$, $\alpha = babab$;
  \item
    $R = \{S\rightarrow BA| AS|a, A\rightarrow AB|a,B\rightarrow SS|b\}$, $\alpha = ababa$;
  \item
    $R = \{S\rightarrow AB|a, A\rightarrow BA|SS|a,B\rightarrow SS|b\}$, $\alpha = aabba$.
  \end{enumerate}
\end{problem}


\section{Недетерминирани стекови автомати}

\index{автомат!недетерминиран стеков}
\marginpar{На англ. Push-down automaton}
%Sipser p.102
\begin{dfn}[стр. 102 от \cite{sipser}]
  Недетерминиран краен стеков автомат е \[P = \PDA,\] където 
  \begin{itemize}
  \item
    $Q$ е крайно множество от състояния;
  \item  
    $\Sigma$ е крайна входна азбука;
  \item
    $\Gamma$ е крайна стекова азбука;
  \item
    $\# \in \Gamma$ е символ за дъно на стека;
  \item
    $s\in Q$ е начално състояние;
  \item
    $\Delta:Q\times\Sigma_\varepsilon\times\Gamma\rightarrow \Ps_{fin}(Q\times\Gamma^\star)$ 
    е {\em частична} функция на преходите;    
  \item
    $F\subseteq Q$ е множество от заключителни състояния.
  \end{itemize}
\end{dfn}

Нека $P$ е стеков автомат. Тогава
\begin{itemize}
\item
  $\Ls_F(P)$ е езика, който се разпознава от $P$ {\bfс финално състояние},
  \[\Ls_F(P) = \{w\mid (q_0,w,\#) \vdash^\star_P (q,\varepsilon,\alpha)\ \&\ q \in F\}.\]    
\item
  $\Ls_S(P)$ е езика, който се разпознава от $P$  {\bf с празен стек},
  \[\Ls_S(P) = \{w\mid (q_0,w,\#) \vdash^\star_P (q,\varepsilon,\varepsilon)\}.\]    
\end{itemize}

\begin{thm}
  Класът на езиците, които се разпознават от краен стеков автомат, съвпада с
  класа на контекстно-свободните езици.
\end{thm}

\begin{problem}
  Нека е дадена граматиката $G = \pair{\{S,A,B\},\{a,b\},S,R\}}$.
  Постройте стеков автомат $P = \PDA$, такъв че $L_S(P) = L(G)$, където правилата $R$ на граматиката $G$ са зададени като:
  \begin{enumerate} 
    % За едно тези двете да се даде пример как става 
  \item
    $R = \{S\rightarrow ASB\vert \varepsilon, A\rightarrow aAa\vert a, B\rightarrow bBb\vert b\}$;
  \item
    $R = \{S\rightarrow ASB\vert \varepsilon, A\rightarrow aA\vert a, B\rightarrow Bb\vert b\}$;
  \item
    $R =\{S\rightarrow SA|\varepsilon,A\rightarrow BSa|B, B\rightarrow b|BS|ab\}$;
  \item
    $R = \{S\rightarrow AS|\varepsilon,A\rightarrow SaBB|A, B\rightarrow b|BBbS|AA\}$;
  % \item
  %   $\Gamma=\langle\{S,A,B,C,D,E\},\{a,b\},S,$\\
  %   $\{S \rightarrow aD, D\rightarrow ab|ABBA|ADD,A\rightarrow DEB|a,B\rightarrow DDD|DC|b,C\rightarrow CCE|a, E\rightarrow ba|bEa\}\rangle$;
  % \item
  %   $\Gamma=\langle\{S,A,B,C,D,E\},\{a,b\},S,$\\
  %   $\{S\rightarrow D,D\rightarrow AD|b,A\rightarrow ACB|BC|a, B\rightarrow ABCA|CEC, C\rightarrow \varepsilon|CA|a, E\rightarrow ab|aEb\}\rangle$;
  % \item
  %   $\Gamma=\langle\{S,A,B,C,D,E\},\{a,b\},S,$\\
  %   $\{S \rightarrow aD, D\rightarrow \varepsilon|ABBA|ADD,A\rightarrow DEB|a,B\rightarrow DDD|DC|b,C\rightarrow CCE|a, E\rightarrow \varepsilon|bEa\}\rangle$;
  % \item
  %   $\Gamma=\langle\{S,A,B,C,D,E\},\{a,b\},S,$\\
  %   $\{S\rightarrow D,D\rightarrow AD|b,A\rightarrow ACB|BC|a, B\rightarrow ABCA|CEC, C\rightarrow \varepsilon|CA|a, E\rightarrow \varepsilon|aEb\}\rangle$;

  % \item
  %   $\Gamma=\langle\{S,A,B,C,D,E\},\{a,b\},S,$\\
  %   $\{S\rightarrow DD,D\rightarrow DDA|b,A\rightarrow CAB|a, B\rightarrow BCA|CCE, C\rightarrow \varepsilon|CA|a, E\rightarrow \varepsilon|EaE\}\rangle$;
  % \item
  %   $\Gamma=\langle\{S,A,B,C,D,E\},\{a,b\},S,$\\
  %   $\{S\rightarrow DD,D\rightarrow DA|b,A\rightarrow CAB|a, B\rightarrow BCA|CCE, C\rightarrow \varepsilon|CA|a, E\rightarrow \varepsilon|EaE\}\rangle$;
  \end{enumerate}
\end{problem}

\section{Въпроси}
Вярно ли е, че:
\begin{itemize}
\item
  \marginpar{Да} 
  ако $L$ е контекстно-свободен език, то езикът $L \cap \{a^{2n}b^{2k}\mid n,k\in\Nat\}$ е контекстно-свободен ?
\item
  \marginpar{Да}
  ако $L$ е безкраен контекстно-свободен език, то съществува безкрайна редица от регулярни езици $L_1,L_2,\dots$,
  за които $L = \bigcup_{i\in\Nat}L_i$ ?
\item
  \marginpar{Не}
  за всяка безкрайна редица от регулярни езици $L_1,L_2,\dots$, то 
  езикът $L = \bigcup_{i\in\Nat}L_i$ е контекстно-свободен ?
\item
  за всеки регулярен език $R$ и всеки контекстно-свободен език $L$, то $L \cap R$ е контекстно-свободен ?
\item
  за всеки регулярен език $R$ и всеки контекстно-свободен език $L$, то $L \cup R$ е контекстно-свободен ?
\item
  за всеки регулярен език $R$ и всеки контекстно-свободен език $L$, то $L \setminus R$ е контекстно-свободен ?
\item
  за всеки регулярен език $R$ и всеки контекстно-свободен език $L$, то $R \setminus L$ е контекстно-свободен ?
\item
  съществува регулярен език $R$ и контекстно-свободен език $L$, за които $L \cap R$ не е контекстно-свободен ?
\item
  съществува регулярен език $R$ и нерегулярен, но контекстно-свободен език $L$, за които $L \cap R$ е регулярен ?
\item
  за всеки два нерегулярни, но контекстно-свободни езика $L_1,L_2$, то $L_1\cup L_2$ е регулярен ?
\item
  съществуват два нерегулярни, но контекстно-свободни езика $L_1,L_2$, за които $L_1\setminus L_2$ е регулярен ?
\item
  съществуват два нерегулярни, но контекстно-свободни езика $L_1,L_2$, за които $L_1\cap L_2$ е регулярен ?
\item
  съществуват два нерегулярни, но контекстно-свободни езика $L_1,L_2$, за които $L_1\cup L_2$ е регулярен ?
\item
  съществува регулярен език $R$, който може да се представи като $R = L_1 \cup L_2$, където
  $L_1 \cap L_2 = \emptyset$, $L_1,L_2$ са нерегулярни, но контекстно-свободни ?
\item
  езикът $\{a,b\}^\star \setminus \{a^nb^n \mid n\in\Nat\}$ е регулярен ?
\item
  езикът $\{a,b\}^\star \setminus \{a^nb^{2k+1} \mid n,k\in\Nat\}$ е регулярен ?
\item
  езикът $\{a,b\}^\star \setminus \{a^nb^{k} \mid n > k\}$ е регулярен ?
\item
  езикът $\{a,b\}^\star \setminus \{a^nbba^{n} \mid n \in \Nat\}$ е регулярен ?
\item
  езикът $\{a,b\}^\star \setminus \{a^nb^n \mid n\in\Nat\}$ е контекстно-свободен ?
\item
  езикът $\{a,b,c\}^\star \setminus \{a^nb^mc^k \mid m < n\ \&\ m < k\}$ е контекстно-свободен ?
\end{itemize}

Нека е дадена контекстно-свободната граматика $G$ с правила \[S\rightarrow a\vert AB \vert AC, A \rightarrow a, B\rightarrow b, C\rightarrow SB.\]
Вярно ли е, че ако приложим CYK алгоритъма върху думата $\alpha$, където
\begin{itemize}
\item 
  $\alpha = aabb$, то $N[1,1] = \{S\}$.
\item 
  $\alpha = aabb$, то $N[3,3] = \{B\}$.
\item 
  $\alpha = aabb$, то $N[1,4] = \{\}$.
\item
  $\alpha = baab$, то $N[2,4] = \{\}$.
\item
  $\alpha = baab$, то $N[1,3] = \{\}$.
\end{itemize}



%%% Local Variables: 
%%% mode: latex
%%% TeX-master: "discrete-math"
%%% End: 

% \chapter{Машини на Тюринг}

\newcommand{\tape}[1]{\dots\bot\bot\bot{#1}\bot\bot\bot\dots}
$\M = \TM$

\begin{itemize}
\item 
  $Q$ - състояния;
\item
  $\Sigma$ - азбука за входа;
\item
  $\Gamma$ - азбука за лентата, $\Sigma \subseteq \Gamma$;
\item
  $\delta:Q\times\Gamma \to Q\times \Gamma \times \{L,R\}$ - (частична) функция на преходите;
\item
  $s$ - начално състояние, $s \in Q$;
\item
  $\bot$ - празен символ,  $\bot \in \Gamma \setminus \Sigma$;
\item
  $F$ - финални състояния, $F \subseteq Q$.
\end{itemize}

\begin{itemize}
\item 
  Първоначално, лентата съдържа входа, който е обграден от безкрайно много символи $\bot$ и в двете посоки.
\item
  МТ се намира в началното състояние $s$ и главата е върху най-левия символ от входа.
\item
  Работим с конфигурации $(\alpha, q, \beta) \in \Gamma^\star\times Q \times \Gamma^\star$. Това означава, че
  машината се намира в състояние $q$ и лентата има вида
  \[\tape{\alpha\beta}\]
\item
  Ако $\delta(q,Z) = (p,Y,R)$, то пишем
  $(\alpha, q, Z\beta) \vdash (\alpha Y, p, \beta)$.
  При $Z = \bot$, също така можем да запишем 
  $(\alpha, q, \varepsilon) \vdash (\alpha Y, p, \varepsilon)$

  Ако $\delta(q,Z) = (p,Y,L)$, то пишем
  $(\alpha X, q, Z\beta) \vdash (\alpha , p, XY\beta)$.
  При $X = \bot$, $(\varepsilon, q, Z\beta) \vdash (\varepsilon, p, BY\beta)$.
\item
  Езикът, който се {\bf разпознава чрез финални състояния} от машината $M$ е:
  \[L_F(M) = \{\alpha\in\Sigma^\star \mid (\varepsilon, s, \alpha) \vdash^\star (\beta, q, \gamma)\ \&\ q\in F\ \&\ \beta,\gamma\in\Gamma^\star\}.\]
\item
  Езикът, който се {\bf разпознава чрез спиране} от $M$ е:
  \[L_H(M) = \{\alpha \in \Sigma^\star \mid (\varepsilon, s, \alpha) \vdash^\star (\beta, q, X\gamma)\ \&\ \neg !\delta(q,X)\}\]
\end{itemize}

Може да се докаже, че се разпознават едни и същи езици.

\index{език!полуразрешим}
Езиците, които се разпознават от МТ се наричат {\bf полуразрешими езици}.

\subsection*{Канонична подредба на $\Sigma^\star$}

Нека $\Sigma = \{a_0,a_1,\dots,a_{k-1}\}$.
Подреждаме думите по ред на тяхната дължина.
Думите с еднаква дължина подреждаме по техния числов ред, т.е.
гледаме на буквите $a_i$ като числото $i$ в $k$-ична бройна система.
Тогава подреждаме думите с дължина $n$, са числата от $0$ до $k^n-1$,
записани в $k$-ична бройна система.
Да означим с $w_i$ $i$-тата дума в $\Sigma^\star$ при тази подредба.

\begin{example}
  Ако $\Sigma = \{0,1\}$, то наредбата започва така:
  \[\varepsilon, 0, 1, \underbrace{00, 01, 10, 11}_{\text{от $0$ до $3$}}, \underbrace{000, 001, 010, 011, 100, 101, 110, 111}_{\text{от $0$ до $7$}}, 0000, 0001, \dots\]  
\end{example}

\subsection*{Многолентови машини на Тюринг}

\subsection*{Недетерминистични машини на Тюринг}

\begin{thm}
  Ако $L$ се разпознава от НМТ $\M_1$, до $L$
  също се разпознава и от ДМТ $\M_2$.
\end{thm}

\subsection*{Полуразрешими и разрешими езици}

\begin{thm}
  Ако $L$ и $\Sigma^\star \setminus L$ са полуразрешими езици, то $L$ е разрешим език.
\end{thm}

\section{Универсална машина на Тюринг}
За простота, нека $\Sigma = \{0,1\}$ и $\Gamma = \{0,1,\bot\}$.
\begin{itemize}
\item 
  $X_1 = 0$, $X_2 = 1$, $X_3 = \bot$;
\item
  $D_1 = L$, $D_2 = R$
\end{itemize}

\subsection*{Кодиране на преход}
Да разгледаме прехода $\delta(q_i,X_j) = (q_k,X_l,D_m)$.
Кодираме този преход по следния начин:
\[0^i10^j10^k10^l10^m\]
Да Обърнем внимание, че в този двоичен код няма последователни единици и той 
започва и завършва с нула.
\subsection*{Кодиране на машина на Тюринг}
За да кодираме една машина на Тюринг $\M$ е достатъчно да кодираме функцията на преходите $\delta$.
Понеже $\delta$ е крайна функция, нека с числото $r$ да означим броя на всички възможни преходи.
По описания по-горе начин, нека $code_i$ е числото в двоичен запис, получено за $i$-тия преход на $\delta$.
Тогава кодът на $\M$ е следното число в двоичен запис:
\[\pair{\M} = 111\ code_1\ 11\ code_2\ 11\ \cdots\ 11\ code_r\ 111.\]
\begin{itemize}
\item
  Лесно се съобразява, че за две МТ $\M$ и $\M'$ с различни функции на преходите, имаме $\pair{\M} \neq \pair{\M'}$.
\item
  Ще казваме, че числото $r$ е {\bf код на } $\M$, ако числото $r$, записано в двоичен запис представлява думата $\pair{\M}$.
  Оттук нататък, когато пишем $\M_r$, ще имаме предвид машината на Тюринг с код $r$.
\item
  С $\pair{\M,w}$ ще означаваме кода на МТ $\M$ при вход $w$ е числото с двоичен запис описанието на $\M$ и след това прикрепена думата $w$.
  При едно число $r = \pair{M,w}$, лесно се намира кода на $\M$.
  Просто започваме да четем двоичния запис на $r$ докато не срещнем за втори път $111$.
  След това започва думата $w$.
\end{itemize}

\section{Пример за език, който не се разпознава от МТ}

Да разгледаме безкрайната таблица $\{a_{ij} \mid i,j \in \Nat\}$, където:
\begin{align*}
  a_{ij} = 
  \begin{cases}
    1, & \text{ ако } w_i \in L(\M_j), \\
    0, & \text{ ако } w_i \not\in L(\M_j).
  \end{cases}
\end{align*}
Идеята е да вземем диагонала на тази таблица.

\begin{framed}
  Езикът 
  $L_d = \{w_i \mid w_i \not\in L(\M_i)\}$ не се разпознава от MT.
\end{framed}
Да допуснем, че $L_d$ се разпознава от МТ, т.е. $L_d = L(\M_i)$, за някоя МТ с код $i$.
Тогава:
\begin{itemize}
\item 
  Ако
  $w_i \in L_d\ \rightarrow\ w_i \in L(\M_i)\ \rightarrow\ w_i \not\in L_d$;
\item
  Ако 
  $w_i \not\in L_d\ \rightarrow\ w_i \not\in L(\M_i)\ \rightarrow\ w_i \in L_d$.
\end{itemize}

\section{Универсалният език $L_u$}

Да разгледаме езика $L_u = \{\pair{\M,w} \mid w\in L(\M)\}$.

\section{Проблемът за съответствие на Пост (PCP)}

\subsection*{MPCP}

\section{Разрешими и полуразрешими езици}

\section{Проблеми за безконтекстни езици}

\begin{lemma}
  Нека е дадена $\M = \TM$.
  Тогава езикът 
  \[L = \{\alpha\sharp\beta^R \mid \alpha,\beta \in \Gamma^\star Q \Gamma^\star\ \&\  \alpha \vdash_\M \beta\}\]
  е безконтекстен.
\end{lemma}
\begin{proof}
  Ще покажем, че съществува стеков автомат $P$, за който $\L_S(P) = L$.
  Четем буквата $X$. Тогава:
  \begin{itemize}
  \item 
    ако $\delta_\M(q,X) =(p,Y,R)$, то слагаме $Yp$ на върха на стека;
  \item
    ако $\delta_\M(q,X) =(p,Y,L)$, то ако $Z$ е върха на стека, заменяме $Z$ с $pZY$;
  \end{itemize}
\end{proof}

\begin{lemma}
  Нека е дадена $\M = \TM$.
  Тогава езикът 
  \[L = \{\alpha\sharp\beta^R \mid \alpha,\beta \in \Gamma^\star Q \Gamma^\star\ \&\  \alpha \not\vdash_\M \beta\}\]
  е безконтекстен.
\end{lemma}


\begin{thm}
  Неразрешим е проблемът за проверка дали при дадени две произволни безконтекстни граматики $G_1$ и $G_2$,
  $\L(G_1) \cap \L(G_2) = \emptyset$.  
\end{thm}

\begin{thm}
  Неразрешим е проблемът за проверка дали при дадена произволна безконтекстна граматика $G$,
  $\L(G) = \Sigma^\star$.  
\end{thm}


\section{Въпроси}

Вярно ли е, че следният проблем е {\em разрешим}:
\begin{itemize}
\item
  за произволна безконтекстна граматика $G$, проверява дали $\L(G) = \emptyset$?
\item
  за произволна безконтекстна граматика $G$, проверява дали $\L(G) = \Sigma^\star$?
\item
  за произволни безконтекстни граматики $G_1$ и $G_2$, проверява дали $\L(G_1) \cap \L(G_2) = \emptyset$?
\item
  за произволни безконтекстни граматики $G_1$ и $G_2$, проверява дали $\L(G_1) \cap \L(G_2) = \Sigma^\star$?
\item
  за произволни безконтекстни граматики $G_1$ и $G_2$, проверява дали $\L(G_1) = \L(G_2)$?
\item
  за произволни безконтекстни граматики $G_1$ и $G_2$, проверява дали $\L(G_1) \subseteq \L(G_2)$?
\item
  за произволна безконтекстна граматика $G$ и произволен регулярен израз $r$,
  проверява дали $\L(G) = \L(r)$?
\item
  за произволна безконтекстна граматика $G$ и произволен регулярен израз $r$,
  проверява дали $\L(G) \subseteq \L(r)$?
\item
  за произволна безконтекстна граматика $G$ и произволен регулярен израз $r$,
  проверява дали $\L(r) \subseteq \L(G)$?
% \item
%   за произволни безконтекстни граматики $G_1$ и $G_2$, проверява дали $\L(G_1) \subseteq \L(G_2)$ 
%   е безконтекстен език ?
% \item
%   за произволна безконтекстна граматика $G$, проверява дали $\Sigma^\star \setminus \L(G)$
%   е безконтекстен език ?
% \item
%   за произволна безконтекстна граматика $G$, проверява дали $\L(G)$ е регулярен език?
\end{itemize}
%%% Local Variables: 
%%% mode: latex
%%% TeX-master: "discrete-math"
%%% End: 

%\chapter{Теория на Графите}

\section{Неориентирани графи}

\begin{dfn}
  Неориентиран граф\index{неориентиран!граф} $G$ е наредена тройка $(V,E,\psi_G)$, където
  $V$ е непразно множество, $V,E$ са непресичащи се множества, и $\psi_G$ асоциира с всеки елемент $e\in E$
  ненаредена двойка от елементи на $V$.
  Елементите на $V$ наричаме върхове, а елементите на $E$ ребра.
\end{dfn}

Нека да въведем някои означения.
Под прост {\bf граф}\index{прост!граф} $G$ ще разбираме неориентиран граф без примки и повтарящи се ребра.
С $\delta(G)$\index{$\delta(G)$} ще означаваме минималната степен в графа $G$, а с $\Delta(G)$\index{$\Delta(G)$} - максималната.
Означаваме $\nu(G) = |V|, \epsilon(G) = |E|$.
Броят на свързаните компоненти на $G$ означаваме с $\omega(G)$.

\begin{problem}
  Докажете следното неравенство:
  \[\delta \leq \frac{2\varepsilon}{\nu} \leq \Delta.\]
\end{problem}
\begin{proof}
  \[\nu\delta \leq\sum_{1\leq i \leq\nu} deg_G(v_i) \leq \nu\Delta.\]
\end{proof}


\section{Дървета}

\begin{dfn}
  Дърво е свързан граф без цикли.
\end{dfn}

\begin{thm}
  В дърво, между всеки два върха има единствен път.
\end{thm}

\begin{thm}
  Ако $G$ е дърво, то $\varepsilon(G) = \nu(G) - 1$.
\end{thm}
\begin{proof}
  С индукция по $\nu$. За $\nu = 1$ е ясно.
  Да допуснем за $G$ с брой ребра $<\nu$ и ще докажем за $G$.
  Нека $uv\in E$ и $G'$ се получава като изтрием това ребро.
  Получаваме два свързани ациклични графа $G_1, G_2$.
  Следователно те са дървета и от и.п. 
  \[\varepsilon(G_1) = \nu(G_1) - 1\ \&\ \varepsilon(G_2) = \nu(G_2) - 1.\]
  Получаваме, че 
  \[\varepsilon(G - uv) = \varepsilon(G_1) + \varepsilon(G_2) = \nu(G_1) + \nu(G_2) - 2.\]
  Накрая, $\varepsilon(G) = \varepsilon(G-uv) + 1 = \nu(G_1) + \nu(G_2) - 2 + 1= \nu(G) - 1$.
\end{proof}

\begin{crl}
  Всяко нетривиално дърво има поне два върха със степен 1.
\end{crl}
\begin{proof}
  Ясно е, че $(\forall v\in V)[d(v) \geq 1]$.
  Знаем, че \[\sum_{v\in V}d(v) = 2\varepsilon = 2\nu - 2,\] от където следва, че има поне два върха със степен 1.
\end{proof}

\subsection{Покриващи дървета}

Означаваме с $\tau(G)$ броят на покриващите дървета на $G$ (не неизоморфните, а всички).

Нека $u\neq v, e = (u,v)\in E$. С $G.e = (E',V')$ означаваме графа получен от $G$, като премахваме реброто $e$ и 
съединяваме краищата $u,v$. Ясно е, че $\nu(G.e) = \nu(G) - 1, \varepsilon(G.e) = \varepsilon(G) - 1, \omega(G.e) = \omega(G)$.

\begin{thm}
  Ако $e\in E$ не е примка, то
  $\tau(G) = \tau(G-e) + \tau(G.e)$.
\end{thm}
\begin{proof}
  $\tau(G) = n + m$, където $n$ е броят на покриващите дървета на $G$, в които не участва $e$
  и $m$ е броят на покриващите дървета, в които участва $e$.
  Ясно е, че $n = \tau(G-e)$.
  На всяко покриващо дърво $T$, в което участва $e$, съответства покриващо дърво $T.e$ на $G.e$.
  Това съответствие е взаимно-еднозначно, следователно $m = \tau(G.e)$.
\end{proof}

\begin{problem}
  Колко на брой са всички изоморфни и неизоморфни покриващи дървета на $K_3,K_4$ ?
\end{problem}


\begin{problem}
  Покажете, че ако $G$ е дърво и има връх със степен $\geq k$, то $G$ има поне $k$ върха със степен 1.
\end{problem}

\begin{problem}
  Нека $G$ е свързан граф.
  Докажете, че всеки два най-дълги пътя в свързан граф имат общ връх.
\end{problem}

\begin{problem}
  За $G$ прост граф, докажете, че $\varepsilon = \binom{\nu}{2}$ т.с.т.к. $G$ е пълен.
\end{problem}

\begin{problem}
  Докажете, че в граф с $\nu\geq 2$, има поне два върха с еднаква степен.
\end{problem}

\begin{problem}
  Докажете, че:
  \begin{enumerate}
  \item
    във всеки неориентиран граф броят на върховете с нечетна степен е четен;
  \item
    всеки регулярен граф с нечетна степен има четен брой върхове;
  \item
    всеки граф с $\varepsilon > \binom{\nu-1}{2}$ е свързан.
    Дайде пример за несвързан граф с $\varepsilon = \binom{\nu-1}{2}$.
  \item
    във граф всички върхове имат степен поне $d$.
    Докажете, че в графа има път с дължина $d$.
  \end{enumerate}
\end{problem}


\begin{problem} % зад. 1.22
  Да разгледаме графа $G$ (без примки и без кратни ребра) със $s$ компоненти на свързаност.
  Докажете, че $\nu - s \leq \varepsilon \leq \binom{\nu-s+1}{2}$.
\end{problem}

\begin{problem}
  Нега $G$ е граф с $n$ върха и в $G$ няма прост цикъл с дължина 3.
  Докажете, че $G$ има най-много $\lfloor{\frac{n^2}{4}}\rfloor$ ребра.
\end{problem}

\begin{problem}
  Нека $G$ е произволен граф без примки и кратни ребра, а $\overline{G}$ е неговото допълнение.
  Докажете, че поне един от графите $G$, $\overline{G}$ е свързан;
\end{problem}


\begin{problem}
  \begin{enumerate}
  \item
    Да се построят всички неизоморфни графи на 1,2,3 и 4 върха.
  \item
    Намерете броя на ребрата на граф без цикли с $n$ върха и $k$ компоненти.
  \end{enumerate}
\end{problem}

\section{Ориентирани графи}

%%% Local Variables: 
%%% mode: latex
%%% TeX-master: "discrete-math"
%%% End: 

\chapter {Алгоритми за графи}

\section{Обхождане на граф}

Нека е даден ориентирания граф $G = (V,E)$.
При обхождането на граф, всеки връх може да бъде в едно от три състояния, или както сме ги означили тук, в един от три цвята.
Ако един връх $v$ е
\begin{itemize}
\item 
  бял, то той още не е срещнат.
\item
  сив, то той е срещнат, но още не е напълно обработен.
\item
  черен, то той е напълно обработен.
\end{itemize}


\subsection{Обхождане в широчина}


\begin{algorithm}
  \caption{Инициализация}
  \label{alg:bfs-init}
  \begin{algorithmic}[1]
    \Procedure{BFS-INIT}{$G$,$r$}
    \ForAll{$v \in V \setminus \{r\}$}
    \State $\texttt{color}[v] := \texttt{WHITE}$
    \State $\texttt{dist}[v] := \infty$
    \State $\texttt{pred}[v] := \texttt{NIL}$
    \EndFor
    \State $\texttt{color}[r]:= \texttt{GRAY}$
    \State $\texttt{dist}[r]:= 0$
    \State $\texttt{pred}[r]:= \texttt{NIL}$
    \EndProcedure
  \end{algorithmic}
\end{algorithm}

За този алгоритъм най-удобно е да имаме масив $\texttt{Adj}$ с дължина $\abs{V}$,
като $\texttt{Adj}[u]$ дава списък с преките наследници на $u$, т.е.
\[\texttt{Adj}[u] = \{v \in V \mid (u,v) \in E\}.\]

\begin{algorithm}[H]
  \caption{Обхождане на граф в широчина}
  \label{alg:bfs}
  \begin{algorithmic}[1]
    \Procedure{BFS}{$G$,$r$}
    \State \Call{BFS-INIT}{$G$,$r$}
    \State $Q := \emptyset$\Comment{Опашката $Q$ съдържа точно сивите върхове}
    \State put($Q$,$r$)
    \While {$Q \neq \emptyset$}
    \State $u := get(Q)$\Comment{$u$ е премахнат от опашката}
    \ForAll{$v \in \texttt{Adj}[u]$}
    \If{$\texttt{WHITE} = \texttt{color}[v]$}
    \State $put(Q,v)$
    \State $\texttt{color}[v] := \texttt{GRAY}$
    \State $\texttt{pred}[v] := u$
    \State $\texttt{dist}[v] := \texttt{dist}[u] + 1$
    \EndIf
    \EndFor
    \State $\texttt{color}[u] := \texttt{BLACK}$
    \EndWhile
    \EndProcedure
  \end{algorithmic}
\end{algorithm}

\begin{itemize}
\item 
  Алгоритъмът работи както за ориентирани, така и за неориентирани графи.
\item
  {\bf Дължина на път} (без цикли) е броят на ребрата, които участват в пътя.
  Например, за пътя $p = (v_0,\dots,v_k)$ в графа $G$,
  неговата дължина е $k$, защото ребрата, които участват в $p$
  са $\{(v_0,v_1),(v_1,v_2),\dots,(v_{k-1},v_k)\}$ и са общо $k$ на брой.
  Обикновено ще означаваме дължината на пътя $p$ с $\abs{p}$ и пишем $v_0 \stackrel{p}{\leadsto} v_k$.
\item
  Нека да означим за всеки два върха $u,v \in V$,
  \begin{align*}
    \delta(u,v) =
    \begin{cases}
      \min\{\abs{p} \mid u\stackrel{p}{\leadsto}v\}, & \text{ ако има път между }u, v \\
      \infty, & \text{ ако няма път}
    \end{cases}
  \end{align*}
\item
  Имаме свойството, че за граф $G = (V,E)$ и един връх $s \in V$,
  ако $(u,v) \in E$, то
  \[\delta(s,v) \leq \delta(s,u) + 1.\]
\item
  За граф $G = (V,E)$, и фиксиран връх $r\in V$, означаваме
  \[G_{pred} = (V_{pred},E_{pred}),\]
  където
  \begin{align*}
    V_{pred} & = \{v \in V\mid \texttt{pred}[v] \neq \texttt{NIL}\} \cup \{s\},\\
    E_{pred} & = \{(u,v) \in E \mid \texttt{pred}[v] = u\}.
  \end{align*}
  След изпълнение на BFS($G$,$r$), $G_{pred}$ представлява дърво с корен $r$, 
  за всеки достижим в $G$ от $r$ връх $v$, $G_{pred}$ съдържа единствен прост път $r \stackrel{p}{\leadsto} v$, 
  като $p$ е най-къс измежду всички пътища свързващи $r$ с $v$ в $G$.
\end{itemize}

\begin{thm}
  Нека е даден граф $G = (V,E)$ и един връх $r \in V$.
  След изпълнение на BFS($G$,$r$) получаваме, че
  \[(\forall v \in V)[\texttt{dist}[v] = \delta(r,v)].\]
\end{thm}

\subsection{Обхождане в дълбочина}

\begin{algorithm}[H]
  \caption{Обхождане в дълбочина}
  \label{alg:dfs-visit}
  \begin{algorithmic}[1]
    \Procedure{DFS-VISIT}{$G$,$u$}
    % \State $t := t+1$
    % \State $in[u] := t$
    \State $\texttt{color}[u] := \texttt{GRAY}$\Comment{Върхът $u$ е посетен, но не е обработен}
    \ForAll{$v \in \texttt{Adj}[u]$}
    \If {$\texttt{WHITE} = \texttt{color}[v]$}
    \State $\texttt{pred}[v] := u$
    \State \Call{DFS-VISIT}{$v$}
    \EndIf
    \EndFor
    % \State $t := t+1$
    % \State $out[u] := t$
    \State $\texttt{color}[u] := \texttt{BLACK}$\Comment{Приключили сме с $u$}
    \EndProcedure
    \Statex
    \Procedure{DFS}{$G$}
    \ForAll{$v \in V$}\Comment{Инициализация}
    \State $\texttt{color}[v] := \texttt{WHITE}$
    \State $\texttt{pred}[v] := \texttt{NIL}$
    \EndFor    
    % \State $t := 0$
    \ForAll{$v \in V$}
    \If{$\texttt{WHITE} = \texttt{color}[v]$}
    \State\Call{DFS-VISIT}{$G$,$v$}
    \EndIf
    \EndFor
    \EndProcedure
  \end{algorithmic}
\end{algorithm}


\section{Минимално покриващо дърво на граф}


\begin{itemize}
\item
  Тук ще разглеждаме само {\bf неориентирани} графи $G = (V,E,w)$ с тегла по ребрата
  зададени с функцията $w:E\to\R$.
\item
  Един граф $G = (V,E)$ се нарича {\bf свързан}, ако има път между всеки два $v,v^\prime \in V$.
\item 
  Един неориентиран граф $G$ се нарича {\bf дърво}, ако $G$ е свързан и без цикли.
\item
  {\bf Покриващо дърво} за свъзан неориентиран граф $G = (V,E)$,
  е дърво $T = (V,E^\prime)$, $E^\prime \subseteq E$.
\item
  Тегло на едно подмножество от ребра $U \subseteq E$ е числото
  \[w(U) = \sum_{e \in U} w(e).\]
\item
  {\bf Минимално покриващо дърво} за свъзан неориентиран претеглен граф $G = (V,E,w)$
  е покриващо дърво $T$, за което 
  \[w(T) = \min\{w(T^\prime) \mid T^\prime\mbox{ е покриващо дърво за }G\}.\]
\end{itemize}

\subsection{Алгоритъм на Прим}

Нека е даден неориентиран претеглен {\bf свързан} граф $G = (V,E,w)$.

\begin{algorithm}[H]
  \caption{Намиране на покриващо дърво (Прим)}

  \begin{algorithmic}[1]
    \Procedure{PRIM}{$G,r$}
    \State $U := \{r\}$\Comment{Започваме от дърво с корен $r$ и без ребра}
    \State $S := \emptyset$
    \While{$(\exists (x,y)\in E)[x\in U\ \&\ y \in V\setminus U]$}
    \State Избираме $(u,v) \in E$, за което
    \State $w(u,v) = \min\{w(x,y) \mid x\in U\ \&\ y \in V\setminus U\ \&\ (x,y) \in E\}$
    \State $U := U\cup\{v\}$
    \State $S := S \cup\{(u,v)\}$
    \EndWhile
    \State \textbf{return} $(U,S)$\Comment{Връщаме като резултат полученото дърво}
    \EndProcedure
  \end{algorithmic}
\end{algorithm}

% \begin{enumerate}
% \item 
%   Започваме от дървото $T_0 = (\{r\},\emptyset)$.
% \item
%   Нека сме построили дървото $T_i = (V_i,E_i)$.
%   Избираме ребро $(v,v^\prime) \in E$ такова ,че
%   \[w(v,v^\prime) = \min\{w(x,y) \mid x\in V_i\ \&\ y \in V\setminus V_i\ \&\ (x,y) \in E\}.\]
%   Образуваме \[T_{i+1} = (V_i\cup\{v^\prime\}, E_i \cup \{(v,v^\prime)\}).\]
% \item
%   Алгоритъмът завършва когато $V_i = V$.
% \end{enumerate}

\subsection{Алгоритъм на Крускал}

Нека е даден неориентиран {\bf свързан} претеглен граф $G = (V,E,w)$.

\begin{algorithm}[h!]
  \caption{Намиране на покриващо дърво (Крускал)}
  
  \begin{algorithmic}[1]
    \Procedure{KRUSKAL}{$G$}
    \State $X = \emptyset$\Comment{$X$ ще бъде колекция от дървета}
    \For{$v \in V$}
    \State Добавяме дървото $T = (\{v\},\emptyset)$ към колекцията $X$
    \EndFor
    \State$E^\prime := $\Call{Sort}{$E$,$w$}\Comment{Сортираме $E$ във възходящ ред относно тегла им}
    \Statex
    \ForAll{$(u,v) \in E^\prime$}
    \State Нека $u$ е връх в дървото $T_u \in X$
    \State Нека $v$ е връх в дървото $T_v \in X$
    \If{$T_u \neq T_v$}
    \State $W := V_u\cup V_v$
    \State $R := E_u\cup E_v\cup\{(u,v)\}$
    \State $T := (W,R)$
    \State Премахваме $T_u$ и $T_v$ от колекцията $X$
    \State Добавяме дървото $T$ към $X$
    \EndIf
    \EndFor
    \State \Return единственото дърво останало в $X$
    \EndProcedure
  \end{algorithmic}
\end{algorithm}

\section{Минимални пътища от даден връх}

\begin{itemize}
\item
  С $u \stackrel{p}{\leadsto} v$ означаваме, че $p$ е път от $u$ до $v$.
\item
  Тук ще разглеждаме {\bf ориентирани} графи $G = (V,E)$, като имаме и 
  функция $w: E\to \R$, която задава {\bf тегла} на ребрата на графа.
\item 
  {\bf Цена на път} $p = (v_0,\dots,v_k)$ в графа означаваме 
  \[w(p) = \sum_{i<k} w(v_i,v_{i+1}).\]
\item
  За всеки два върха $u,v \in V$, означаваме
  \begin{align*}
    \delta(u,v) = 
    \begin{cases}
      \min\{w(p)\mid u \stackrel{p}{\leadsto} v\}, \mbox{ ако има път от }u\mbox{ до }v\\
      \infty, \mbox{ иначе }
    \end{cases}
  \end{align*}
\item
  {\bf Минимален път} $p$ от $u$ до $v$ е такъв път, за който $w(p) = \delta(u,v)$.
\item
  Имаме следното важно свойство.
  Нека $u \stackrel{p}{\leadsto} v$ и $p$ е {\bf минимален път}.
  Да означим $p = (v_0,\dots,v_k)$ и $p_{ij} = (v_i,\dots,v_j)$ за $0\leq i \leq j \leq k$.
  Тогава за всеки $0\leq i \leq j \leq k$, 
  $p_{ij}$ е {\bf минимален път} от $v_i$ до $v_j$.
\item
  {\bf Цикъл} е път $p = (v_0,\dots,v_k)$, където $v_0 = v_k$.
\item
  Също така казваме, че по пътя $p = (v_0,\dots,v_k)$ има цикъл, ако
  за някои $0 \leq i < j \leq k$ имаме, че $v_i = v_j$.
\item
  Ако има цикъл с отрицателно тегло по някой път от $u$ до $v$, то 
  тогава пишем, че $\delta(u,v) = -\infty$.
\item
  Нека $u \stackrel{p}{\leadsto} v$ и $p$ е с минимално тегло.
  Тогава няма цикъл с положително тегло по $p$.
\item
  Нека $u \stackrel{p}{\leadsto} v$ и $p$ е с минимално тегло.
  Можем без ограничение на общността да приемем, че няма цикли с нулево тегло
  по $p$.
\item
  Важно свойство е, че броят на върховете по всички минални пътища е $\leq \abs{V}$.
\end{itemize}

Нека да фиксираме един връх $s \in V$.
Целта ни е да намерим минимални пътища от $s$ до всички достижими от $s$ върхове,
както и тяхната цена. 
Да отбележим, че ако имаме отрицателен по някой път $s \leadsto v$, то задачата не е добре 
дефинирана, защото тогава $\delta(s,v) = -\infty$.

За тази цел въвеждаме два масива, $\texttt{dist}$ и $\texttt{pred}$, с дължина $\abs{V}$.
\begin{itemize}
\item 
  $\texttt{dist}[v]$ - дава цена на минимален път от $s$ до $v$.
  Ако $\texttt{dist}[v] = \infty$, то не е намерен път $s\leadsto v$.
\item
  $\texttt{pred}[v]$ - дава предшественика на $v$ по този минимален път, т.е.
  ако $\texttt{pred}[v] = u$, то $s \leadsto u \to v$.
  Ако $\texttt{pred}[v] = \texttt{NIL}$, то не е намерен път $s \leadsto v$.
\end{itemize}

\begin{algorithm}[h!]
  \caption{Инициализация}
  \label{alg:init}
  \begin{algorithmic}[1]
    \Procedure{INIT}{$s$}
    \ForAll{$v \in V$}
    \State $\texttt{dist}[v] := \infty$
    \State $\texttt{pred}[v] := \texttt{NIL}$
    \EndFor
    \State $\texttt{dist}[s] := 0$
    \EndProcedure
  \end{algorithmic}
\end{algorithm}

\begin{algorithm}[h!]
  \caption{Търсене на по-добър кандидат}
  \label{alg:update}
  \begin{algorithmic}[1]
    \Procedure{UPDATE}{$u$,$v$}
    \If{$\texttt{dist}[v] > \texttt{dist}[u] + w(u,v)$}
    \State $\texttt{dist}[v] := \texttt{dist}[u] + w(u,v)$
    \State $\texttt{pred}[v] := u$
    \EndIf
    \EndProcedure
  \end{algorithmic}
\end{algorithm}

\subsection{Основни свойства}
  
\begin{prop}[Неравенство на триъгълника]
  \label{prop:triangle}
  За всяко $(u,v) \in E$,
  \[\delta(s,v) \leq \delta(s,u) + w(u,v).\]
\end{prop}

\begin{prop}
  \label{prop:upper-bound}
  Нека сме изпълнили INIT(s).
  Тогава имаме свойството \[(\forall v\in V)[\texttt{dist}[v] \geq \delta(s,v)].\]
  То се запазва и след прозволен брой изпълнения на UPDATE върху ребра на графа.
  Освен това, ако веднъж $\texttt{dist}[v] = \delta(s,v)$, то стойността на $\texttt{dist}[v]$
  повече не се променя.
\end{prop}
\begin{proof}
  Индукция по броя $i$ на изпълнения на UPDATE.
  За $i = 0$ е очевидно.
  Ще докажем твърдението за $i > 0$ изпълнения на UPDATE.
  Нека $\texttt{dist}[v] > \texttt{dist}[u] + w(u,v)$ и изпълним UPDATE(u,v).
  Тогава като използваме индукционното предположение и неравенството на триъгълника,
  \begin{align*}
    \texttt{dist}[v] & = \texttt{dist}[u] + w(u,v)\\
    & \geq \delta(s,u) + w(u,v)\\
    & \geq  \delta(s,v).
  \end{align*}

  Ясно е, че веднъж достигнали $\texttt{dist}[v] = \delta(s,v)$, $\texttt{dist}[v]$
  не може да се промени, защото тази стойност може само да намалява, а ние
  сме достигнали нейния минумум.
\end{proof}

\begin{prop}
  \label{prop:no-path}
  Нека сме изпълнили INIT($s$) и нека няма път от $s$ до $v$.
  Тогава имаме свойството
  \[\texttt{dist}[v] = \delta(s,v) = \infty.\]
  То се запазва и след прозволен брой изпълнения на UPDATE върху ребра на графа.
\end{prop}
\begin{proof}
  Щом няма път от $s$ до $v$, то $\delta(s,v) = \infty$.
  От Твърдение \ref{prop:upper-bound}, $\texttt{dist}[v] \geq \delta(s,v) = \infty$.
  Следователно, $\texttt{dist}[v] = \infty$.
\end{proof}

\begin{prop}
  \label{prop:converge}
  Нека $s\leadsto u \to v$ е път с минимално тегло.
  Нека сме изпълнили INIT(s) и няколко пъти $\texttt{UPDATE}$, като измежду тях и UPDATE($u$,$v$).
  Ако преди изпълнението на UPDATE($u$,$v$) 
  имаме, че $\texttt{dist}[u] = \delta(s,u)$, то след това изпълнение
  $\texttt{dist}[v] = \delta(s,v)$ и стойността на $\texttt{dist}[v]$ повече не се променя.
\end{prop}
\begin{proof}
  Първо да отбележим, че за $(u,v) \in E$, веднага след изпълнението на UPDATE(u,v) имаме, че
  \[\texttt{dist}[v] \leq \texttt{dist}[u] + w(u,v).\]
  Ако $\texttt{dist}[u] = \delta(s,u)$, то от Твърдение \ref{prop:upper-bound} това равенство се запазва.
  Получаваме, че:
  \begin{align*}
    \texttt{dist}[v] & \leq \texttt{dist}[u] + w(u,v)\\
    & = \delta(s,u) + w(u,v)\\
    & = \delta(s,v),
  \end{align*}
  защото $s\leadsto u \to v$ е път с минимална дължина.
  Тогава $\texttt{dist}[v] \leq \delta(s,v)$ и следователно 
  \[\texttt{dist}[v] = \delta(s,v),\]
  защото пак от Твърдение \ref{prop:upper-bound}, винаги е изпълнено, че $\texttt{dist}[v] \geq \delta(s,v)$,
\end{proof}

% \begin{prop}
%   \label{prop:path-update}
%   Да разгледаме пътя $p = (v_0,\dots,v_k)$, като $v_0 = s$.
%   Нека сме изпълнили INIT(s) и след това няколко пъти UPDATE, като сме включили 
%   UPDATE($v_{i}$,$v_{i+1}$), за всяко $0\leq i < k$, в този ред на изпълнение.
%   Тогава най-накрая получаваме, че $\texttt{dist}[v_k] = \delta(s,v_k)$.
% \end{prop}
% \begin{proof}
%   Индукция по $i$.
%   В началото, $\texttt{dist}[v_0] = \texttt{dist}[s] = 0 = \delta(s,s)$.
%   Ако $\texttt{dist}[v_{i-1}] = \delta(s,v_{i-1})$, то след изпълнение на UPDATE($v_{i-1}$,$v_{i}$),
%   получаваме от Твърдение \ref{prop:converge}, че $\texttt{dist}[v_i] = \delta(s,v_{i})$.
% \end{proof}

% \begin{prop}
%   \label{prop:tree-shortest-path}
%   Нека сега да приемем, че в нашия граф няма цикли с отрицателни тегла, достижими от $s$
%   и нека сме изпълнили INIT(s) и произволен брой пъти UPDATE.
%   Тогава:
%   \begin{enumerate}[1)]
%   \item 
%     $G_{pred}$ е дърво с корен $s$.
%   \item
%     ако $(\forall v\in V)[\texttt{dist}[v] = \delta(s,v)]$, то $G_{pred}$ е дърво на пътищата с минимални тегла с корен $s$.
%   \end{enumerate}
% \end{prop}
% \begin{proof}
%   \begin{enumerate}[1)]
%   \item 
%     Първо ще докажем, че $G_{pred}$ е насочен ацикличен граф и след
%     това, че няма пътища $p \neq p^\prime$ от вида $s\stackrel{p}{\leadsto} v$ и $s\stackrel{p^\prime}{\leadsto} v$.
%     \begin{itemize}
%     \item 
%       Да допуснем, че $G_{pred}$ е цикличен граф.
%       Нека $\gamma = (v_0,\dots,v_k)$ е цикъл, $v_0 = v_k$, който се е получил точно след изпълнение на 
%       UPDATE($v_{k-1}$,$v_k$).

%       Да разгледаме ситуацията точно преди това изпълнение на UPDATE($v_{k-1}$,$v_{k}$).
%       Имаме, че 
%       \[(\forall i < k-1)[\texttt{pred}[v_{i+1}] = v_{i}]\]
%       от което следва, че
%       \begin{equation}
%         \label{ineq}
%         (\forall i < k-1)[\texttt{dist}[v_{i+1}] \geq \texttt{dist}[v_i] + w(v_i,v_{i+1})].
%       \end{equation}
%       Щом още нямаме цикъл преди изпълнението на UPDATE($v_{k-1}$,$v_k$),
%       то стойността на $\texttt{pred}[v_k]$ се променя при извикването на UPDATE($v_{k-1}$,$v_k$).
%       Оттук следва, че
%       \[\texttt{dist}[v_{k}] > \texttt{dist}[v_{k-1}] + w(v_{k-1},v_k).\]
%       Комбинирайки с Неравенство (\ref{ineq}) получаваме, че
%       \begin{align*}
%         \sum^{k}_{i = 1} \texttt{dist}[v_{i}] & > \sum^{k-1}_{i=0} (\texttt{dist}[v_i] + w(v_{i},v_{i+1}))\\
%         & = \sum^{k-1}_{i=0} \texttt{dist}[v_i] + w(\gamma),
%       \end{align*}
%       но понеже $v_0 = v_k$, 
%       \[\sum^{k-1}_{i=0} \texttt{dist}[v_i] = \sum^{k}_{i=1} \texttt{dist}[v_i]\]
%       и тогава
%       \[0 > w(\gamma).\]
%       Получаваме, че цикълът $\gamma$ има отрицателно тегло, което е противоречие.
%     \item
%       Да допуснем, че в $G_{pred}$ има $p \neq p^\prime$  и  $s\stackrel{p}{\leadsto} v$ и $s\stackrel{p^\prime}{\leadsto} v$.
%       Това означава, че съществуват $x \neq y$,
%       $s \leadsto u \leadsto x \to z \leadsto v$ и $s \leadsto u \leadsto y \to z \leadsto v$.
%       По определение, $pred(z) = x \neq y = pred(z)$. Противоречие.
%     \end{itemize}
%   \item
%     \begin{itemize}
%     \item 
%       Лесно се съобразява, че $V_{pred}$ съдържа точно върховете достижими от $s$,
%       т.е. ако $v \in V_{pred}$, то съществува път $p = (s,\dots,v)$.
%       % защото $v$ е достижим от $s$ точно когато $\delta(s,v) = \texttt{dist}[v] < \infty$, 
%       % но тогава $\texttt{pred}[v] \neq NIL$.
%     \item
%       Вече доказахме в 1), че $G_{pred}$ е дърво с корен $s$.
%     \item
%       Остана да докажем, че ако имаме $s\stackrel{p}{\leadsto} v$ в $G_{pred}$, 
%       то $p$ е път с минимално тегло в $G$.
%       Нека $p = (v_0,\dots,v_k)$, $v_0 = s$, $v_k = v$.
%       По условие, 
%       \[(\forall i < k)[\texttt{dist}[v_i] = \delta(s,v_i)],\]
%       а от факта, че $(\forall i < k)[\texttt{pred}[v_i] = v_{i-1}]$ следва, че
%       \[(\forall i < k)[\texttt{dist}[v_i] \geq \texttt{dist}[v_{i-1}] + w(v_{i-1},v_{i})].\]
%       Като обединим горните две неравенства, получваме, че
%       \begin{equation}
%         \label{eq:weight}
%         (\forall i < k)[w(v_{i-1},v_{i}) \leq \delta(s,v_{i}) - \delta(s,v_{i-1})],
%       \end{equation}
%       Тогава
%       \begin{align*}
%         w(p) & = \sum^k_{i=1} w(v_{i-1},v_{i}) & (\text{по деф.})\\
%         & \leq \sum^k_{i=1} (\delta(s,v_i)- \delta(s,v_{i-1})) & (\ref{eq:weight})\\
%         & = \delta(s,v_k) - \delta(s,v_0)\\
%         & = \delta(s,v_k) - \delta(s,s) & (v_0 = s)\\
%         & = \delta(s,v_k) - 0\\
%         & = \delta(s,v_k).
%       \end{align*}
%       Следователно, 
%       \[w(p) \leq \delta(s,v_k).\]
%       Понеже $\delta(s,v_k)$ е минималното тегло на път от $s$ до $v_k$,
%       то $w(p) = \delta(s,v_k)$.
%       Следователно, $p$ е път с минимално тегло.
%     \end{itemize}
%   \end{enumerate}
% \end{proof}




\subsection{Алгоритъм на Дейкстра}
\index{Дейкстра!алгоритъм}

В този алгоритъм, разглеждаме ориентирани графи $G = (V,E,w)$ с {\em положителни} тегла (или цени) по ребрата, т.е. 
за всяко $(u,v) \in E$, $w(u,v) \geq 0$.

\begin{algorithm}[H]
  \caption{Пътища с мин. тегло от върха $s$ (Дейкстра)}
  \label{alg:dijkstra}
  
  \begin{algorithmic}[1]
    \Require{$w:E\to \R^+$}
    \Procedure{DIJKSTRA}{$s$}
    \State \Call{INIT}{$s$}
    \State $U := V$
    \While{$U \neq\emptyset$}
    \State Избираме $u_0\in U$, за който $\texttt{dist}[u_0] = \min\{\texttt{dist}[v] \mid v\in U\} $
    \State $U := V^\prime\setminus\{u_0\}$
    \ForAll{ $v\in Adj[u_0]$ }
    \State\Call{UPDATE}{$u_0$,$v$}
    \EndFor
    \EndWhile
    \EndProcedure
    % \Return $\delta$
  \end{algorithmic}
\end{algorithm}

\begin{thm}
  Нека $G$ е ориентиран граф с неотрицателни тегла по ребрата.
  След изпълнението на алгоритъма на Дейкстра с начален връх $s$,
  \[(\forall v \in V)[dist[v] = \delta(s,v)].\]
\end{thm}
\begin{proof}
  Ще докажем, че на всяка итерация на while-цикъла, 
  \[(\forall v\in V\setminus V^\prime)[dist[v] = \delta(s,v)].\]
  Първоначално $V\setminus V^\prime = \emptyset$.
  Ще докажем, че на всяка итерация на while-цикъла, за върха $u$, който сме премахнали от $V^\prime$,
  е изпълнено, че $dist[u] = \delta(s,u)$.
  За целта да допуснем противното и нека $u$ е първия връх, който е премахнат от $V^\prime$,
  за който $dist[u] \neq \delta(s,u)$.
  Лесно се съобразява, че $u \neq s$.
  Освен това, трябва $s \leadsto u$, защото иначе $dist[u] = \delta(s,u) = \infty$ според Твърдение \ref{prop:no-path}.
  Нека $s \stackrel{p}{\leadsto} u$ и $p$ е път с минимално тегло.
  Да разбием пътя $p$ по следния начин:
  \[s \stackrel{p_1}{\leadsto}x\to y\stackrel{p_2}{\leadsto}u,\]
  където $y$ е първия връх по пътя $p$, за който $y\not\in V^\prime$.
  Ясно е, че тогава $x \in V^\prime$ и тогава $dist[x] = \delta(s,x)$, 
  защото ние избрахме $u$ да бъде първия връх, за който $dist[u] \neq \delta(s,u)$.
  На итерацията на while-цикъла, на която добавяме $x$ към $V^\prime$, 
  ние изпълняваме UPDATE(x,y) и според Твърдение \ref{prop:converge}, $dist[y] = \delta(s,y)$.
  Но понеже $y$ е преди $u$ по път с минимално тегло и при положение, че няма ребра с отрицателни тегла,
  \[\delta(s,y) \leq \delta(s,u).\]
  Тогава
  \begin{align*}
    dist[y] & = \delta(s,y) \\
    & \leq \delta(s,u)\\
    & \leq dist[u], \mbox{според Твърдение \ref{prop:upper-bound}}.
  \end{align*}
  Но понеже $y,v \not\in V^\prime$ и сме избрали $u$ вместо $y$, то това означава, че
  \[dist[u] \leq dist[y].\]
  Следователно, 
  \[dist[y] = dist[u]\]
  и тогава 
  \[dist[u] = \delta(s,u),\]
  с което достигаме до противоречие.
\end{proof}
\begin{cor}
  $G_{pred}$ е дърво на минималните пътища с корен $s$.
\end{cor}


Ако във $V^\prime$ има останали върхове $v$, то те имат $\delta(v) = \infty$, т.е.
те са недостижими от $s$ и следователно пътят от $s$ до $v$ има дължина $\infty$.

Фигура \ref{fig:dijkstra-table} илюстрира как се променя функцията $\delta$ по време на изпълнението на алгоритъма.
Освен това, можем да намерим не само стойността на най-късите пътища, но
и списък с ребрата, които участват във всеки от тях.
% Фигура \ref{fig:dijkstra-graph} илюстрира това.
% Ребрата, оцветени в зелено, са тези, които участват в най-късите пътища.
% Жълти ребра са тези, които са кандидати да участват в най-късите пътища.
% Червени са тези ребра, които са били вече обходени и са отхвърлени като част от най-къс път.

\tikzstyle{weight} = [font=\small]
\tikzstyle{value} = [font=\small]
\tikzstyle{edge} = [draw,thick,-]
\tikzstyle{nodedecorate}=[shape=circle,draw,thick,font=\small]
\tikzstyle{arrowdecorate}=[->,>=stealth,thick]

% Rename: selected --> current
\tikzstyle{selected vertex}=[vertex, fill=yellow!50]
\tikzstyle{selected edge} = [draw,line width=5pt,-,yellow!50]

\tikzstyle{vertex}=[circle,minimum size=15pt,inner sep=0pt]
\tikzstyle{sure vertex} = [vertex, fill=green!30]

\tikzstyle{path edge} = [draw,line width=5pt,-,red!50]

\tikzstyle{sure edge} = [draw,line width=5pt,-,green!30]
% \tikzstyle{ignored edge} = [draw,line width=5pt,-,black!20]


\begin{figure}[!htbp]
  \begin{subfigure}[b]{0.5\textwidth}
    \begin{tikzpicture}[]
      
      \foreach \nodename/\x/\y/\direction/\navigate in { a/1/1/above/north,
        b/0/0/left/west, c/1/-1.5/below/south, d/3/1/above/north, e/3/-1.5/below/south, f/5/0.5/right/east, g/5/2.5/right/east}
      {
        \node (\nodename) at (\x,\y) [nodedecorate] {};
        \node [\direction] at (\nodename.\navigate) {$\nodename$};
      }
      %% edges or lines
      \path
      \foreach \startnode/\endnode/\direction/\weight in {b/a/above/7,
        b/c/below/2, c/a/left/4, a/d/below/4, c/e/below/5, d/c/left/8, e/d/right/3}
      {
        (\startnode) edge[arrowdecorate] node[\direction] {$\weight$} (\endnode)
      }
      
      \foreach \startnode/\endnode/\direction/\angle/\weight in {
        a/g/above/15/10, d/f/above/15/5, d/g/above/-15/2, f/d/below/15/1, g/f/right/15/6, e/f/below/-15/7}
      {
        (\startnode) edge[arrowdecorate,bend left=\angle] node[\direction] {$\weight$} (\endnode)
      };
    \end{tikzpicture}
    \caption{По-долу ще приложим алгоритъма на Дейкстра върху този граф}
    \label{subfig:dijkstra}
  \end{subfigure}
  \qquad
  \begin{subfigure}[b]{0.5\textwidth}
  \begin{tikzpicture}[]
    
    \foreach \nodename/\x/\y/\direction/\navigate in { a/1/1/above/north,
      s/0/0/left/west, b/1/-1.5/below/south}
      {
        \node (\nodename) at (\x,\y) [nodedecorate] {};
        \node [\direction] at (\nodename.\navigate) {$\nodename$};
      }
      %% edges or lines
      \path
      \foreach \startnode/\endnode/\direction/\weight in {s/a/above/3,
        s/b/below/2, a/b/right/-2}
      {
        (\startnode) edge[arrowdecorate] node[\direction] {$\weight$} (\endnode)
      };
    \end{tikzpicture}
    \caption{Пример, за който алгоритъмът на Дейкстра не дава верен резултат (Защо?)}
  \end{subfigure}
  \caption{}
  \end{figure}

  \begin{figure}[!htbp]
    \begin{subtable}[b]{0.5\textwidth}
      \begin{tabular}[b]{|c|c|c|c|c|c|c|c|c|}
        \hline
        $a$ & $b$ & $c$ & $d$ & $e$ & $f$ & $g$\\
        \hline
        $\infty$ & {\bf \framebox{0}} & $\infty$ & $\infty$ & $\infty$ & $\infty$ & $\infty$ \\
        \hline
        7 & $\colon$ & {\bf \framebox{2}} & $\infty$ & $\infty$ & $\infty$ & $\infty$ \\
        \hline
        {\bf \framebox{6}} & $\colon$ & $\colon$ & $\infty$ & 7 & $\infty$ & $\infty$ \\
        \hline
        $\colon$ & $\colon$ & $\colon$ & 10 & {\bf \framebox{7}} & $\infty$ & 16 \\
        \hline
        $\colon$ & $\colon$ & $\colon$ & {\bf \framebox{10}} & $\colon$ & 14 & {\bf 16} \\
        \hline
        $\colon$ & $\colon$ & $\colon$ & $\colon$ & $\colon$ & 14 & {\bf \framebox{12}} \\
        \hline
        $\colon$ & $\colon$ & $\colon$ & $\colon$ & $\colon$ & {\bf \framebox{14}} & $\colon$ \\
        \hline
        $\colon$ & $\colon$ & $\colon$ & $\colon$ & $\colon$ & $\colon$ & $\colon$ \\
        \hline
      \end{tabular}
      \caption{Масива $\texttt{dist}$ за начален връх $b$}
    \end{subtable}
    \qquad
    \begin{subtable}[b]{0.5\textwidth}
      \begin{tabular}[b]{|c|c|c|c|c|c|c|c|c|}
        \hline
        $a$ & $b$ & $c$ & $d$ & $e$ & $f$ & $g$\\
        \hline
        $\texttt{NIL}$ & \framebox{\texttt{NIL}} & $\texttt{NIL}$ & $\texttt{NIL}$ & $\texttt{NIL}$ & $\texttt{NIL}$ & $\texttt{NIL}$ \\
        \hline
        $b$ & $\colon$ & \framebox{$b$} & $\texttt{NIL}$ & $\texttt{NIL}$ & $\texttt{NIL}$ & $\texttt{NIL}$ \\
        \hline
        {\bf \framebox{c}} & $\colon$ & $\colon$ & $\texttt{NIL}$ & c & $\texttt{NIL}$ & $\texttt{NIL}$ \\
        \hline
        $\colon$ & $\colon$ & $\colon$ & a & {\bf \framebox{c}} & $\texttt{NIL}$ & a \\
        \hline
        $\colon$ & $\colon$ & $\colon$ & {\bf \framebox{a}} & $\colon$ & c & a \\
        \hline
        $\colon$ & $\colon$ & $\colon$ & $\colon$ & $\colon$ & c & {\bf \framebox{d}} \\
        \hline
        $\colon$ & $\colon$ & $\colon$ & $\colon$ & $\colon$ & {\bf \framebox{c}} & $\colon$ \\
        \hline
        $\colon$ & $\colon$ & $\colon$ & $\colon$ & $\colon$ & $\colon$ & $\colon$ \\
        \hline
      \end{tabular}
      \caption{Масива $\texttt{pred}$ за начален връх $b$}
    \end{subtable}
    \caption{Алгоритъм на Дейкстра с начален връх $b$ за графа от (\ref{subfig:dijkstra})}
  \label{fig:dijkstra-table}
\end{figure}

% \begin{figure}[!htbp]
%   \index{Дейкстра!алгоритъм}
%   

\subfigure[Започваме от съседите на $a$]{
  \begin{tikzpicture}[scale=0.9]
    % nodes
    \foreach \nodename/\x/\y/\value/\direction/\navigate/\color in { 
      a/0/0/0/above/north/green, 
      b/-1/1/\infty/left/west/black, 
      c/2.5/1/\infty/above/north/black, 
      d/2/-0.5/\infty/below/south/black,
      e/-1/-0.5/\infty/below/south/black, 
      f/0.5/2/\infty/above/north/black,
      g/0/-1.8/\infty/below/south/black, 
      h/2.5/-1.8/\infty/right/east/black}
    {
        \node[vertex, nodedecorate, fill=\color!25] (\nodename) at (\x,\y) {$\value$};
        \node [\direction] at (\nodename.\navigate) {$\nodename$};
      };
      % edges
      \path
      \foreach \startnode/\endnode/\direction/\angle/\weight in {
        e/b/left/15/5, e/g/below/-15/3, d/g/below/15/2, 
        g/a/left/15/2, a/g/right/30/9, f/c/below/-15/1,
        c/f/above/-30/5, c/h/right/15/2, c/d/above/-15/1,
        f/d/left/-15/4, a/b/below/0/2, b/f/above/0/1, 
        a/d/above/0/8}
      {
        (\startnode) edge[arrowdecorate,bend left=\angle] node[\direction] {$\weight$} (\endnode)
      };
    \end{tikzpicture}
  }
  \subfigure[$b$ е най-близко до $a$]{
    \begin{tikzpicture}[scale=0.9]
      %nodes
      \foreach \nodename/\x/\y/\value/\direction/\navigate/\color in { 
        a/0/0/0/above/north/green, 
        b/-1/1/2/left/west/yellow, 
        c/2.5/1/\infty/above/north/black, 
        d/2/-0.5/8/below/south/yellow,
        e/-1/-0.5/\infty/below/south/black, 
        f/0.5/2/\infty/above/north/black, 
        g/0/-1.8/9/below/south/yellow, 
        h/2.5/-1.8/\infty/right/east/black}
      {
        \node[vertex, nodedecorate, fill=\color!25] (\nodename) at (\x,\y) {$\value$};
        \node [\direction] at (\nodename.\navigate) {$\nodename$};
      };

      \path
      \foreach \startnode/\endnode/\direction/\angle in {
        a/g/right/30, a/b/above/0, a/d/above/0}
      {
        (\startnode) edge[selected edge,bend left=\angle] node[\direction] {} (\endnode)
      };
      %edges
      \path
      \foreach \startnode/\endnode/\direction/\angle/\weight in {
        e/b/left/15/5, e/g/below/-15/3, d/g/below/15/2, g/a/left/15/2, a/g/right/30/9, f/c/below/-15/1, c/f/above/-30/5,
        c/h/right/15/2, c/d/above/-15/1, f/d/left/-15/4, a/b/below/0/2, b/f/above/0/1,  a/d/above/0/8}
      {
        (\startnode) edge[arrowdecorate,bend left=\angle] node[\direction] {$\weight$} (\endnode)
      };
    \end{tikzpicture}
  }
  \subfigure[Най-къс път до $f$]{
    \begin{tikzpicture}[scale=0.9]
      %nodes
      \foreach \nodename/\x/\y/\value/\direction/\navigate/\color in { 
        a/0/0/0/above/north/green, 
        b/-1/1/2/left/west/green,
        c/2.5/1/\infty/above/north/black, 
        d/2/-0.5/8/below/south/yellow,
        e/-1/-0.5/\infty/below/south/black, 
        f/0.5/2/3/above/north/yellow, 
        g/0/-1.8/9/below/south/yellow, 
        h/2.5/-1.8/\infty/right/east/black}
      {
        \node[vertex, nodedecorate, fill=\color!25] (\nodename) at (\x,\y) {$\value$};
        \node [\direction] at (\nodename.\navigate) {$\nodename$};
      };
      \path
      (a) edge[sure edge] node[] {} (b);
      \path
      \foreach \startnode/\endnode/\direction/\angle in {
        a/g/right/30, a/d/above/0}
      {
        (\startnode) edge[selected edge,bend left=\angle] node[\direction] {} (\endnode)
      };
      \path
      \foreach \startnode/\endnode/\direction/\angle in {
         b/f/above/0}
      {
        (\startnode) edge[selected edge, bend left=\angle] node[\direction] {} (\endnode)
      };
      %edges
      \path
      \foreach \startnode/\endnode/\direction/\angle/\weight in {
        e/b/left/15/5, e/g/below/-15/3, d/g/below/15/2, g/a/left/15/2, a/g/right/30/9, f/c/below/-15/1, c/f/above/-30/5,
        c/h/right/15/2, c/d/above/-15/1, f/d/left/-15/4, a/b/below/0/2, b/f/above/0/1,  a/d/above/0/8}
      {
        (\startnode) edge[arrowdecorate,bend left=\angle] node[\direction] {$\weight$} (\endnode)
      };
    \end{tikzpicture}
  }
  \subfigure[По-къс път до $d$ и $c$]{
    \begin{tikzpicture}[scale=0.9]
      % nodes
      \foreach \nodename/\x/\y/\value/\direction/\navigate/\color in { 
        a/0/0/0/above/north/green,
        b/-1/1/2/left/west/green,
        c/2.5/1/4/above/north/yellow, 
        d/2/-0.5/7/below/south/yellow,
        e/-1/-0.5/\infty/below/south/black, 
        f/0.5/2/3/above/north/green, 
        g/0/-1.8/9/below/south/yellow, 
        h/2.5/-1.8/\infty/right/east/black}
      {
        \node[vertex, nodedecorate, fill=\color!25] (\nodename) at (\x,\y) {$\value$};
        \node [\direction] at (\nodename.\navigate) {$\nodename$};
      };
      \path
      (a) edge[sure edge] node[] {} (b)
      (b) edge[sure edge] node[] {} (f);
      
      \path
      \foreach \startnode/\endnode/\direction/\angle in {
        a/d/above/0}
      {
        (\startnode) edge[path edge,bend left=\angle] node[\direction] {} (\endnode)
      };
      \path
      \foreach \startnode/\endnode/\direction/\angle in {f/c/below/-15/1, f/d/left/-15/4, a/g/right/30}
      {
        (\startnode) edge[selected edge, bend left=\angle] node[\direction] {} (\endnode)
      };
      %edges
      \path
      \foreach \startnode/\endnode/\direction/\angle/\weight in {
        e/b/left/15/5, e/g/below/-15/3, d/g/below/15/2, g/a/left/15/2, a/g/right/30/9, f/c/below/-15/1, c/f/above/-30/5,
        c/h/right/15/2, c/d/above/-15/1, f/d/left/-15/4, a/b/below/0/2, b/f/above/0/1,  a/d/above/0/8}
      {
        (\startnode) edge[arrowdecorate,bend left=\angle] node[\direction] {$\weight$} (\endnode)
      };
    \end{tikzpicture}
  }
  \subfigure[По-къс път до $d$ и $h$]{
    \begin{tikzpicture}[scale=0.9]
      % nodes
      \foreach \nodename/\x/\y/\value/\direction/\navigate/\color in { 
        a/0/0/0/above/north/green,
        b/-1/1/2/left/west/green,
        c/2.5/1/4/above/north/green,
        d/2/-0.5/5/below/south/yellow,
        e/-1/-0.5/\infty/below/south/black, 
        f/0.5/2/3/above/north/green, 
        g/0/-1.8/9/below/south/yellow, 
        h/2.5/-1.8/6/right/east/yellow}
      {
        \node[vertex, nodedecorate, fill=\color!25] (\nodename) at (\x,\y) {$\value$};
        \node [\direction] at (\nodename.\navigate) {$\nodename$};
      };
      \path
      (a) edge[sure edge] node[] {} (b)
      (b) edge[sure edge] node[] {} (f)
      (f) edge[sure edge, bend left=-15] node[] {} (c);
      
      \path
      \foreach \startnode/\endnode/\direction/\angle in {f/d/left/-15/4, a/d/above/0
        }
      {
        (\startnode) edge[path edge,bend left=\angle] node[\direction] {} (\endnode)
      };
      \path
      \foreach \startnode/\endnode/\direction/\angle in {c/d/above/-15/1,  c/f/above/-30/5, c/h/right/15/2, a/g/right/30}
      {
        (\startnode) edge[selected edge, bend left=\angle] node[\direction] {} (\endnode)
      };
      %edges
      \path
      \foreach \startnode/\endnode/\direction/\angle/\weight in {
        e/b/left/15/5, e/g/below/-15/3, d/g/below/15/2, g/a/left/15/2, a/g/right/30/9, f/c/below/-15/1, c/f/above/-30/5,
        c/h/right/15/2, c/d/above/-15/1, f/d/left/-15/4, a/b/below/0/2, b/f/above/0/1,  a/d/above/0/8}
      {
        (\startnode) edge[arrowdecorate,bend left=\angle] node[\direction] {$\weight$} (\endnode)
      };
    \end{tikzpicture}
  }
  \subfigure[По-къс път до $g$]{
    \begin{tikzpicture}[scale=0.9]
      %nodes
      \foreach \nodename/\x/\y/\value/\direction/\navigate/\color in { 
        a/0/0/0/above/north/green, 
        b/-1/1/2/left/west/green,
        c/2.5/1/4/above/north/green,
        d/2/-0.5/5/below/south/green,
        e/-1/-0.5/\infty/below/south/black, 
        f/0.5/2/3/above/north/green, 
        g/0/-1.8/7/below/south/yellow,
        h/2.5/-1.8/6/right/east/yellow}
      {
        \node[vertex, nodedecorate, fill=\color!25] (\nodename) at (\x,\y) {$\value$};
        \node [\direction] at (\nodename.\navigate) {$\nodename$};
      };
      \path
      (a) edge[sure edge] node[] {} (b)
      (b) edge[sure edge] node[] {} (f)
      (f) edge[sure edge, bend left=-15] node[] {} (c)
      (c) edge[sure edge, bend left=-15] node[] {} (d);
      
      \path
      \foreach \startnode/\endnode/\direction/\angle in {f/d/left/-15, a/g/right/30, a/d/above/0, c/f/above/-30
        }
      {
        (\startnode) edge[path edge,bend left=\angle] node[] {} (\endnode)
      };
      \path
      \foreach \startnode/\endnode/\direction/\angle in {d/g/below/15/2, c/h/right/15}
      {
        (\startnode) edge[selected edge, bend left=\angle] node[\direction] {} (\endnode)
      };
      %edges
      \path
      \foreach \startnode/\endnode/\direction/\angle/\weight in {
        e/b/left/15/5, e/g/below/-15/3, d/g/below/15/2, g/a/left/15/2, a/g/right/30/9, f/c/below/-15/1, c/f/above/-30/5,
        c/h/right/15/2, c/d/above/-15/1, f/d/left/-15/4, a/b/below/0/2, b/f/above/0/1,  a/d/above/0/8}
      {
        (\startnode) edge[arrowdecorate,bend left=\angle] node[\direction] {$\weight$} (\endnode)
      };
    \end{tikzpicture}
  }
  \subfigure[$h$ е задънена улица]{
    \begin{tikzpicture}[scale=0.9]
      %nodes
      \foreach \nodename/\x/\y/\value/\direction/\navigate/\color in { 
        a/0/0/0/above/north/green, 
        b/-1/1/2/left/west/green,
        c/2.5/1/4/above/north/green, 
        d/2/-0.5/5/below/south/green,
        e/-1/-0.5/\infty/below/south/black, 
        f/0.5/2/3/above/north/green, 
        g/0/-1.8/7/below/south/yellow,
        h/2.5/-1.8/6/right/east/green}
      {
        \node[vertex, nodedecorate, fill=\color!25] (\nodename) at (\x,\y) {$\value$};
        \node [\direction] at (\nodename.\navigate) {$\nodename$};
      };
      \path
      (a) edge[sure edge] node[] {} (b)
      (b) edge[sure edge] node[] {} (f)
      (f) edge[sure edge, bend left=-15] node[] {} (c)
      (c) edge[sure edge, bend left=-15] node[] {} (d)
      (c) edge[sure edge, bend left=15] node[] {} (h);
      
      \path
      \foreach \startnode/\endnode/\direction/\angle in {f/d/left/-15, a/g/right/30, a/d/above/0, c/f/above/-30
        }
      {
        (\startnode) edge[path edge,bend left=\angle] node[] {} (\endnode)
      };
      \path
      \foreach \startnode/\endnode/\direction/\angle in {d/g/below/15/2}
      {
        (\startnode) edge[selected edge, bend left=\angle] node[\direction] {} (\endnode)
      };
      %edges
      \path
      \foreach \startnode/\endnode/\direction/\angle/\weight in {
        e/b/left/15/5, e/g/below/-15/3, d/g/below/15/2, g/a/left/15/2, a/g/right/30/9, f/c/below/-15/1, c/f/above/-30/5,
        c/h/right/15/2, c/d/above/-15/1, f/d/left/-15/4, a/b/below/0/2, b/f/above/0/1,  a/d/above/0/8}
      {
        (\startnode) edge[arrowdecorate,bend left=\angle] node[\direction] {$\weight$} (\endnode)
      };
    \end{tikzpicture}
  }
  \subfigure[$e$ не е достижим]{
    \begin{tikzpicture}[scale=0.9]
      %nodes
      \foreach \nodename/\x/\y/\value/\direction/\navigate/\color in { 
        a/0/0/0/above/north/green,
        b/-1/1/2/left/west/green,
        c/2.5/1/4/above/north/green, 
        d/2/-0.5/5/below/south/green,
        e/-1/-0.5/\infty/below/south/black, 
        f/0.5/2/3/above/north/green, 
        g/0/-1.8/7/below/south/green,
        h/2.5/-1.8/6/right/east/green}
      {
        \node[vertex, nodedecorate, fill=\color!25] (\nodename) at (\x,\y) {$\value$};
        \node [\direction] at (\nodename.\navigate) {$\nodename$};
      };
      \path
      (a) edge[sure edge] node[] {} (b)
      (b) edge[sure edge] node[] {} (f)
      (f) edge[sure edge, bend left=-15] node[] {} (c)
      (c) edge[sure edge, bend left=-15] node[] {} (d)
      (c) edge[sure edge, bend left=15] node[] {} (h)
      (d) edge[sure edge, bend left=15] node[] {} (g);
      
      \path
      \foreach \startnode/\endnode/\direction/\angle in {f/d/left/-15, a/g/right/30, a/d/above/0, c/f/above/-30
        }
      {
        (\startnode) edge[path edge,bend left=\angle] node[] {} (\endnode)
      };
      \path
      \foreach \startnode/\endnode/\direction/\angle in {g/a/left/15}
      {
        (\startnode) edge[selected edge, bend left=\angle] node[\direction] {} (\endnode)
      };
      %edges
      \path
      \foreach \startnode/\endnode/\direction/\angle/\weight in {
        e/b/left/15/5, e/g/below/-15/3, d/g/below/15/2, g/a/left/15/2, a/g/right/30/9, f/c/below/-15/1, c/f/above/-30/5,
        c/h/right/15/2, c/d/above/-15/1, f/d/left/-15/4, a/b/below/0/2, b/f/above/0/1,  a/d/above/0/8}
      {
        (\startnode) edge[arrowdecorate,bend left=\angle] node[\direction] {$\weight$} (\endnode)
      };
    \end{tikzpicture}
  }
  \subfigure[Краен резултат]{
    \begin{tikzpicture}[scale=0.9]
      %nodes
      \foreach \nodename/\x/\y/\value/\direction/\navigate/\color in { 
        a/0/0/0/above/north/green, b/-1/1/2/left/west/green,
        c/2.5/1/4/above/north/green, d/2/-0.5/5/below/south/green,
        e/-1/-0.5/\infty/below/south/black, f/0.5/2/3/above/north/green, 
        g/0/-1.8/7/below/south/green, h/2.5/-1.8/6/right/east/green}
      {
        \node[vertex, nodedecorate, fill=\color!25] (\nodename) at (\x,\y) {$\value$};
        \node [\direction] at (\nodename.\navigate) {$\nodename$};
      };
      \path
      (a) edge[sure edge] node[] {} (b)
      (b) edge[sure edge] node[] {} (f)
      (f) edge[sure edge, bend left=-15] node[] {} (c)
      (c) edge[sure edge, bend left=-15] node[] {} (d)
      (c) edge[sure edge, bend left=15] node[] {} (h)
      (d) edge[sure edge, bend left=15] node[] {} (g);
      
      \path
      \foreach \startnode/\endnode/\direction/\angle in {f/d/left/-15, a/g/right/30, a/d/above/0, c/f/above/-30, g/a/above/15
        }
      {
        (\startnode) edge[path edge,bend left=\angle] node[] {} (\endnode)
      };
      \path
      \foreach \startnode/\endnode/\direction/\angle in {}
      {
        (\startnode) edge[selected edge, bend left=\angle] node[\direction] {} (\endnode)
      };
      %edges
      \path
      \foreach \startnode/\endnode/\direction/\angle/\weight in {
        e/b/left/15/5, e/g/below/-15/3, d/g/below/15/2, g/a/left/15/2, a/g/right/30/9, f/c/below/-15/1, c/f/above/-30/5,
        c/h/right/15/2, c/d/above/-15/1, f/d/left/-15/4, a/b/below/0/2, b/f/above/0/1,  a/d/above/0/8}
      {
        (\startnode) edge[arrowdecorate,bend left=\angle] node[\direction] {$\weight$} (\endnode)
      };
    \end{tikzpicture}
  }

%%% Local Variables: 
%%% mode: latex
%%% TeX-master: "discrete-math"
%%% End: 

%   \caption{Алгоритъм на Дейкстра запазващ минималните пътища}
%   \label{fig:dijkstra-graph}
% \end{figure}



% \newpage

% \subsection{Алгоритъм на Белман-Форд}\index{Белман-Форд!алгоритъм}

% Алгоритъмът на Дейкстра работи само за графи $G = (V,E,w)$ с {\em положителни} тегла по ребрата.
% Сега ще разгледаме един алгоритъм, който работи и за графи с отрицателни тегла по ребрата.
% Задачата отново е да намерим минималните разстояния на пътищата с начало върха $s$, но
% искаме също така алгоритъмът да отговаря на въпроса дали има отрицателен цикъл в графа. 
% Ако такъв съществува, то няма решение на проблема. (Защо?)
% Ако отрицателен цикъл не съществува, то алгоритъмът намира пътища в графа с минимални тегла от върха $s$
% до всички достижими върхове в графа.


% \begin{algorithm}
%   \caption{Пътища с мин. тегло от върха $s$ (Белман-Форд)}
%   \label{alg:belman-ford}
  
%   \begin{algorithmic}[1]
%     \Procedure{Bellman-Ford}{$s$}
%     \State\Call{INIT}{$s$}
%     \For{$i:=1$ to $\abs{V}-1$}
%     \ForAll{$(u,v)\in E$}
%     \State\Call{UPDATE}{$u$,$v$}
%     % \ENSURE{$\texttt{dist}[v] \geq \delta(s,v)$}
%     \EndFor
%     \EndFor
    
%     \Comment{Проверка за отрицателен цикъл}
%     \ForAll{$(u,v)\in E$}
%     \If {$\texttt{dist}[v] > \texttt{dist}[u] + w(u,v)$}
%     \State Return \texttt{False}
%     \EndIf
%     \EndFor
%     \State Return \texttt{True}
%     \EndProcedure
%   \end{algorithmic}
% \end{algorithm}

% \begin{prop}
%   \label{prop:bellman-ford}
%   Нека графът $G$ няма отрицателни цикли, които са достижими от $s$.
%   Тогава след изпълнение на алгоритъма на Белман-Форд получаваме, че
%   за всички $v \in V$ достижими от $s$, 
%   \[\texttt{dist}[v] = \delta(s,v).\]
% \end{prop}
% \begin{proof}
%   Да разгледаме $s \stackrel{p}{\leadsto} v$, където $p = (v_0,\dots,v_k)$ е път с минимално тегло в $G$.
%   Понеже в пътища с минимална дължина няма цикли, то $k \leq \abs{V} - 1$.
%   Тогава според Твърдение \ref{prop:path-update}, 
%   след $i$-тата итерация на \texttt{for} цикъла (ред 3), $\texttt{dist}[v_i] = \delta(s,v_i)$.
%   Така получаваме, че най-накрая $\texttt{dist}[v] = \delta(s,v)$.
% \end{proof}
% \begin{cor}
%   \label{cor:bellman-ford}
%   При същите предположения за графа $G$,
%   за всяко $v \in V$, 
%   има път $s \leadsto v$ точно тогава, когато след приключване на алгоритъма е изпълнено $\texttt{dist}[v] < \infty$.
% \end{cor}
% \begin{proof}
%   Ако има път $p$, $s \stackrel{p}{\leadsto} v$, то
%   според твърдението $\texttt{dist}[v] = \delta(s,v) < \infty$.
%   За другата посока, нека $\texttt{dist}[v] < \infty$, но да допуснем, че няма път от $s$ до $v$.
%   Но тогава от Твърдение \ref{prop:no-path} следва, че $\texttt{dist}[v]  = \infty$,
%   което е противоречие.
% \end{proof}


% \begin{thm}
%   \label{th:bellman-ford}
%   Ако $G$ няма отрицателни цикли достижими от $s$, то
%   алгоритъмът на Белман-Форд връща TRUE, $(\forall v\in V)[dist[v] = \delta(s,v)]$,
%   и $G_{pred}$ е дърво с корен $s$, което съдържа пътища с минимални тегла.

%   Ако $G$ има отрицателни цикли достижими от $s$, то
%   алгоритъмът на Белман-Форд връща FALSE.
% \end{thm}
% \begin{proof}
%   \begin{enumerate}[a)]
%   \item 
%     Нека $G$ не съдържа цикъл с отрицателно тегло, достижим от $s$.
%     Ако $v$ е достижим от $s$, то според Твърдение \ref{prop:bellman-ford}, 
%     след изпълнение на алгоритъма
%     \[\texttt{dist}[v] = \delta(s,v).\]

%     Ако $v$ не е достижим от $s$, то според Твърдение \ref{prop:no-path},
%     след изпълнение на алгоритъма
%     \[\texttt{dist}[v] = \infty = \delta(s,v).\]

%     Понеже $(\forall v\in V)[\texttt{dist}[v] = \delta(s,v)]$, от Твърдение \ref{prop:tree-shortest-path} следва, че
%     $G_{pred}$ е дърво с корен $s$, което съдържа пътища с минимални тегла.
    
%     Като използваме Твърдение \ref{prop:triangle} лесно се вижда, че алгоритъмът връща TRUE.
%   \item
%     Нека $G$ съдържа цикъл с отрицателно тегло, достижим от $s$.
%     Нека един такъв цикъл е $\gamma = (v_0,\dots,v_k)$, $v_0 = v_k$.
%     Тогава
%     \[w(\gamma) = \sum^k_{i=1}w(v_{i-1},v_i) < 0.\]
%     Да допуснем, че алгоритъмът връща TRUE. Тогава за всяко $i = 1,\dots, k$, 
%     \[\texttt{dist}[v_i] \leq \texttt{dist}[v_{i-1}] + w(v_{i-1},v_i)\]
%     и като сумираме, 
%     \[\sum^{k}_{i=1} \texttt{dist}[v_i] \leq \sum^{k}_{i=1} \texttt{dist}[v_{i-1}] + \sum^{k}_{i=1}w(v_{i-1},v_i).\]
%     Тъй като $v_0 = v_k$, 
%     \[\sum^{k}_{i=1} \texttt{dist}[v_i] = \sum^{k}_{i=1}\texttt{dist}[v_{i-1}].\]
%     Получаваме, че \[0 \leq \sum^{k}_{i=1}w(v_{i-1},v_i) = w(\gamma),\] което е противоречие с отицателността на цикъла.
%   \end{enumerate}
% \end{proof}

% Фигура \ref{fig:bellman-ford-negative-cycle} илюстрира случая за цикъл с отрицателно тегло.
% Както забелязахме при алгоритъма на Дейкстра, и тук можем да намерим не само дължините на най-късите пътища, но
% и спъсъка на ребрата, участващи в тях. Фигура \ref{fig:bellman-ford-graph} илюстрира този проблем. 
% Останалите накрая оцветени в синьо ребра участват в най-късите пътища.



% \begin{figure}[!htbp]
%   \begin{subfigure}[b]{0.5\textwidth}
%     \begin{tikzpicture}
%       [nodedecorate/.style={shape=circle,inner sep=2pt,draw,thick},%
%       arrowdecorate/.style={->,>=stealth,thick}]
%       %% nodes or vertices
      
%       \foreach \nodename/\x/\y/\direction/\navigate in { a/0/0/below/south,
%         b/6/0/below/south, c/4.5/0/below/south, d/3/0/below/south, e/1.5/0/below/south}
%       {
%         \node (\nodename) at (\x,\y) [nodedecorate] {};
%         \node [\direction] at (\nodename.\navigate) {$\nodename$};
%       }
%       %% edges or lines
%       \path
%       \foreach \startnode/\endnode/\direction/\angle/\weight in {
%         a/e/above/15/1, d/e/above/-25/-1, e/d/below/-15/1,  d/c/above/15/1, c/b/above/15/1, b/e/below/60/-4}
%       {
%         (\startnode) edge[arrowdecorate,bend left=\angle] node[\direction] {$\weight$} (\endnode)
%       };
%       ;
%     \end{tikzpicture}
%     \caption{Граф с отрицателен цикъл}
%   \end{subfigure}
%   \quad
%   \begin{subtable}[b]{0.5\textwidth}
%     \begin{tabular}{|c|c|c|c|c|}
%       \hline
%       $\delta(a)$ & $\delta(b)$ & $\delta(c)$ & $\delta(d)$ & $\delta(e)$ \\
%       \hline
%       0 & $\infty$ & $\infty$ & $\infty$ & {\bf \framebox{1}}\\
%       \hline
%       \hline
%       $\colon$ & \framebox{$\infty$} & $\infty$ & $\infty$ & 1\\
%       $\colon$ & $\infty$ & \framebox{$\infty$} & $\infty$ & 1\\
%       $\colon$ & $\infty$ & $\infty$ & {\bf \framebox{2}} & 1\\
%       $\colon$ & $\infty$ & $\infty$ & 2 & \framebox{1}\\
%       \hline\hline
%       $\colon$ & \framebox{$\infty$} & $\infty$ & 2 & 1\\
%       $\colon$ & $\infty$ & {\bf \framebox{3}} & 2 & 1\\
%       $\colon$ & $\infty$ & 3 & \framebox{2} & 1\\
%       $\colon$ & $\infty$ & $\infty$ & 2 & \framebox{1}\\
%       \hline\hline
%       $\colon$ & {\bf \framebox{4}} & 3 & 2 & 1\\
%       $\colon$ & 4 & \framebox{3} & 2 & 1\\
%       $\colon$ & 4 & 3 & \framebox{2} & 1\\
%       $\colon$ & 4 & 3 & 2 & {\bf \framebox{0}}\\
%       \hline\hline
%     \end{tabular}
%     \caption{Изпълнение на алгоритъма}
%   \end{subtable}
%   \caption{Алгоритъм на Белман-Форд върху ориентиран граф с отрицателен цикъл,
%   като ребрата са подредени лексикографски: $\pair{a,e}, \pair{b,e}, \pair{c,b}, \pair{d,c}, \pair{d,e}, \pair{e, d}$}
%   \label{fig:bellman-ford-negative-cycle}
% \end{figure}


% % \begin{figure}[!htbp]
% %   
\begin{subfigure}[b]{0.3\textwidth}
    \begin{tikzpicture}[scale=0.9]
      %% nodes or vertices
      
      \foreach \nodename/\value/\x/\y/\direction/\navigate/\color in { 
        s/0/0/0/left/west/green, 
        x/\infty/3.5/1.5/above/north/black,
        y/\infty/1/-1.5/below/south/black,
        t/\infty/1/1.5/above/north/black, 
        z/\infty/3.5/-1.5/below/south/black}
      {
        \node[vertex, nodedecorate, fill=\color!25] (\nodename) at (\x,\y) {$\value$};
        \node [\direction] at (\nodename.\navigate) {$\nodename$};
      }
      % edges or lines
      \path
      \foreach \startnode/\endnode/\direction/\angle/\weight in {
        s/t/left/15/6, s/y/left/-25/7, y/z/below/0/9,  t/y/left/0/8, t/x/above/30/5, x/t/above/30/-2, z/x/right/0/7,
        z/s/below/0/2, y/x/below/-3, t/z/right/15/-4
      }
      {
        (\startnode) edge[arrowdecorate,bend left=\angle] node[\direction] {$\weight$} (\endnode)
      };
      ;
    \end{tikzpicture}
    \caption{Начален връх е $s$}
  \end{subfigure}
  \quad
  \begin{subfigure}[b]{0.3\textwidth}
    \begin{tikzpicture}[scale=0.9]
      \foreach \nodename/\value/\x/\y/\direction/\navigate/\color in { 
        s/0/0/0/left/west/green, 
        x/\infty/3.5/1.5/above/north/black,
        y/7/1/-1.5/below/south/red,
        t/6/1/1.5/above/north/red, 
        z/\infty/3.5/-1.5/below/south/black}
      {
        \node[vertex, nodedecorate, fill=\color!25] (\nodename) at (\x,\y) {$\value$};
        \node [\direction] at (\nodename.\navigate) {$\nodename$};
      }
    
    \path
    \foreach \startnode/\endnode/\angle in {
      s/t/15, s/y/-25}
    {
      (\startnode) edge[selected edge, bend left=\angle] node[] {} (\endnode)
    };
    
    %edges
    \path
    \foreach \startnode/\endnode/\direction/\angle/\weight in {
      s/t/left/15/6, s/y/left/-25/7, y/z/below/0/9,  t/y/left/0/8, t/x/above/30/5, x/t/above/30/-2, z/x/right/0/7,
      z/s/below/0/2, y/x/below/-3, t/z/right/15/-4}
    {
      (\startnode) edge[arrowdecorate,bend left=\angle] node[\direction] {$\weight$} (\endnode)
    };
    

  \end{tikzpicture}
  \caption{Започваме със съседите на $s$}
  \end{subfigure}
  \quad
  \begin{subfigure}[b]{0.3\textwidth}
    \begin{tikzpicture}[scale=0.9]
      
      \foreach \nodename/\value/\x/\y/\direction/\navigate/\color in { 
        s/0/0/0/left/west/green, 
        x/4/3.5/1.5/above/north/red,
        y/7/1/-1.5/below/south/blue,
        t/6/1/1.5/above/north/blue, 
        z/2/3.5/-1.5/below/south/red}
      {
        \node[vertex, nodedecorate, fill=\color!25] (\nodename) at (\x,\y) {$\value$};
        \node [\direction] at (\nodename.\navigate) {$\nodename$};
      }
    
    \path
    \foreach \startnode/\endnode/\angle in {
      s/t/15, s/y/-25}
    {
      (\startnode) edge[path edge, bend left=\angle] node[] {} (\endnode)
    };


    %edges or lines
    \path
      (y) edge[selected edge] node[below] {} (x)
      (t) edge[selected edge, bend left=15] node[left] {} (z);

    \path
    \foreach \startnode/\endnode/\direction/\angle/\weight in {
      s/t/left/15/6, s/y/left/-25/7, y/z/below/0/9,  t/y/left/0/8, t/x/above/30/5, x/t/above/30/-2, z/x/right/0/7,
      z/s/below/0/2, y/x/below/0/-3, t/z/right/15/-4
    }
    {
      (\startnode) edge[arrowdecorate,bend left=\angle] node[\direction] {$\weight$} (\endnode)
    };
    ;

  \end{tikzpicture}
  \caption{Продължаваме с $x$ и $z$}
  \end{subfigure}
\quad
\begin{subfigure}[b]{0.3\textwidth}
  \begin{tikzpicture}[scale=0.9]
    
    \foreach \nodename/\value/\x/\y/\direction/\navigate/\color in { 
      s/0/0/0/left/west/green, 
      x/4/3.5/1.5/above/north/blue,
      y/7/1/-1.5/below/south/blue,
      t/2/1/1.5/above/north/red, 
      z/2/3.5/-1.5/below/south/blue}
    {
      \node[vertex, nodedecorate, fill=\color!25] (\nodename) at (\x,\y) {$\value$};
      \node [\direction] at (\nodename.\navigate) {$\nodename$};
    }
    
    \path
    \foreach \startnode/\endnode/\angle in {
      s/y/-25, y/x/0, t/z/15}
    {
      (\startnode) edge[path edge, bend left=\angle] node[] {} (\endnode)
    };

    \path
    \foreach \startnode/\endnode/\angle in {
      x/t/30
    }
    {
      (\startnode) edge[selected edge,bend left=\angle] node[] {} (\endnode)
    };
    %edges or lines
    \path
    \foreach \startnode/\endnode/\direction/\angle/\weight in {
      s/t/left/15/6, s/y/left/-25/7, y/z/below/0/9,  t/y/left/0/8, t/x/above/30/5, x/t/above/30/-2, z/x/right/0/7,
      z/s/below/0/2, y/x/below/0/-3, t/z/right/15/-4
    }
    {
      (\startnode) edge[arrowdecorate,bend left=\angle] node[\direction] {$\weight$} (\endnode)
    };
    ;
  \end{tikzpicture}
  \caption{По-кратък път до $t$}
\end{subfigure}
\quad
\begin{subfigure}[b]{0.3\textwidth}
  \begin{tikzpicture}[scale=0.9]
    %% nodes or vertices
    \foreach \nodename/\value/\x/\y/\direction/\navigate/\color in { 
      s/0/0/0/left/west/green, 
      x/4/3.5/1.5/above/north/blue,
      y/7/1/-1.5/below/south/blue,
      t/2/1/1.5/above/north/blue, 
      z/-2/3.5/-1.5/below/south/red}
    {
      \node[vertex, nodedecorate, fill=\color!25] (\nodename) at (\x,\y) {$\value$};
      \node [\direction] at (\nodename.\navigate) {$\nodename$};
    }
    \path
    \foreach \startnode/\endnode/\angle in {
      s/y/-25, y/x/0, x/t/30}
    {
      (\startnode) edge[path edge, bend left=\angle] node[] {} (\endnode)
    };
    \path
    \foreach \startnode/\endnode/\angle in {
      t/z/15/
    }
    {
      (\startnode) edge[selected edge,bend left=\angle] node[] {} (\endnode)
    };
    %edges or lines
    \path
    \foreach \startnode/\endnode/\direction/\angle/\weight in {
      s/t/left/15/6, s/y/left/-25/7, y/z/below/0/9,  t/y/left/0/8, t/x/above/30/5, x/t/above/30/-2, z/x/right/0/7,
      z/s/below/0/2, y/x/below/0/-3, t/z/right/15/-4
    }
    {
      (\startnode) edge[arrowdecorate,bend left=\angle] node[\direction] {$\weight$} (\endnode)
    };
  \end{tikzpicture}
  \caption{По-кратък път до $z$}
  \end{subfigure}
\quad
\begin{subfigure}[b]{0.3\textwidth}
  \begin{tikzpicture}[scale=0.9]
    %% nodes or vertices
    \foreach \nodename/\value/\x/\y/\direction/\navigate/\color in { 
      s/0/0/0/left/west/green, 
      x/4/3.5/1.5/above/north/blue,
      y/7/1/-1.5/below/south/blue,
      t/2/1/1.5/above/north/blue, 
      z/-2/3.5/-1.5/below/south/blue}
    {
      \node[vertex, nodedecorate, fill=\color!25] (\nodename) at (\x,\y) {$\value$};
      \node [\direction] at (\nodename.\navigate) {$\nodename$};
    }
    \path
    \foreach \startnode/\endnode/\angle in {
      s/y/-25, y/x/0, x/t/30, t/z/15}
    {
      (\startnode) edge[path edge, bend left=\angle] node[] {} (\endnode)
    };
    %edges or lines
    \path
    \foreach \startnode/\endnode/\direction/\angle/\weight in {
      s/t/left/15/6, s/y/left/-25/7, y/z/below/0/9,  t/y/left/0/8, t/x/above/30/5, x/t/above/30/-2, z/x/right/0/7,
      z/s/below/0/2, y/x/below/0/-3, t/z/right/15/-4
    }
    {
      (\startnode) edge[arrowdecorate,bend left=\angle] node[\direction] {$\weight$} (\endnode)
    };
  \end{tikzpicture}
  \caption{Край на процедурата.}
\end{subfigure}


%%% Local Variables: 
%%% mode: latex
%%% TeX-master: "discrete-math"
%%% End: 

% %   \index{Белман-Форд!алгоритъм}
% %   \caption{Алгоритъм на Белман-Форд запазващ минималните пътища}
% %   \label{fig:bellman-ford-graph}
% % \end{figure}


% %% стр. 654
% \begin{problem}
%   Променете алгоритъма на Белман-Форд, така че $\delta(v) = -\infty$ за всеки връх $v$, 
%   за който има отрицателен цикъл по някой път от началния връх $s$ до $v$.
% \end{problem}



%%% Local Variables: 
%%% mode: latex
%%% TeX-master: "discrete-math"
%%% End: 


% \backmatter

\bibliographystyle{amsalpha}
\bibliography{discrete-math}

\printindex
% \listofalgorithms

\end{document}


%%% Local Variables: 
%%% mode: latex
%%% TeX-master: "discrete-math"
%%% End: 

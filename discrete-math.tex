%%%% NORMAL DIMENSIONS %%%%%%%%%%%%%%%%%
\documentclass[a4paper,10pt,fleqn]{report}
\setlength{\marginparsep}{0.5cm}
\setlength{\oddsidemargin}{0.3cm}
\setlength{\hoffset}{0cm}
\setlength{\marginparwidth}{110pt}
\let\oldmarginpar\marginpar

% \renewcommand\marginpar[1]{\-\oldmarginpar[\raggedleft\scriptsize #1]%
% {\raggedright\scriptsize #1}}

\renewcommand\marginpar[1]{\oldmarginpar{\raggedleft\scriptsize #1}}

%%%%%%%%%%%%%%%%%%%%%%%%%%%%%%%%%%%%%%%%

%%% KINDLE DIMENSIONS %%%%
% \documentclass[10pt]{extreport}
% \usepackage[papersize={3.6in,4.8in},hmargin=0.1in,vmargin={0.1in,0.1in}]{geometry}  % page geometry
% \renewcommand\marginpar[1]{}
% \usepackage{extsizes}
%%%%%%%%%%%%%%%%%%%%%%%%%%

% \setlength{\oddsidemargin}{4cm}
% \setlength{\evensidemargin}{4cm}
%\usepackage{ucs}

% \usepackage[bindingoffset=2cm]{geometry}

%%%%%%%%%%%%%%%%%%%%%%%%
%%    PACKAGES        %%
%%%%%%%%%%%%%%%%%%%%%%%%

\usepackage[english,bulgarian]{babel}
\usepackage[utf8]{inputenc}
\usepackage[]{hyperref}
\hypersetup{
  pdftitle={Exercises in Discrete mathematics},
  pdfauthor={Stefan Vatev},
  colorlinks=true,
  linkcolor=blue,
  pdfstartview=FitV,
  citecolor=green,
  pdfpagemode=UseOutlines,    % this is the option you were lookin for
  urlcolor=blue
}
% \usepackage[pdftex,colorlinks=true,linkcolor=blue,pdfstartview=FitV,citecolor=green,urlcolor=blue]{hyperref}
% \usepackage{hypcap}
\usepackage{pifont}
\usepackage{amssymb, amsmath, amsthm}
\usepackage{mathrsfs}
\usepackage{latexsym}
\usepackage{makeidx}
\usepackage{stmaryrd}
\usepackage{layout}
\usepackage{framed}
\usepackage{bussproofs}

\usepackage{paralist}
\usepackage[shortlabels]{enumitem}
\setlist{leftmargin=*}
% \newenvironment{compactenum}{%
%   \enumerate[topsep=0pt,partopsep=0pt,parsep=0pt,itemsep=0pt]%
% }{\endenumerate}
% \usepackage[inline]{enumitem}

\usepackage{algorithm}
\floatname{algorithm}{Алгоритъм}
%\usepackage{algorithmic}
\usepackage[noend]{algpseudocode}
%\usepackage{algpseudocode}

%%%%%%%%%%%%%%% TIKZ Package %%%%%%%%%%%%%%%%%%%%%%%
\usepackage{tikz}
\usepackage{pgf}
\usetikzlibrary{arrows,automata}
%%%%%%%%%%%%%%%%%%%%%%%%%%%%%%%%%%%%%%%%%%%%%%%%%%%%
\usepackage{caption}
\usepackage{subcaption}
\usepackage{color, soulutf8}
\newcommand{\newauthor}[2]{
\definecolor{#1}{rgb}{#2}
\expandafter\newcommand\csname #1\endcsname[1]{{\sethlcolor{#1}\hl{#1: ,,##1''}}}}
 
\definecolor{codegreen}{rgb}{0,0.6,0}
\definecolor{codegray}{rgb}{0.5,0.5,0.5}
\definecolor{codepurple}{rgb}{0.58,0,0.82}
\definecolor{backcolour}{rgb}{0.95,0.95,0.92}


\newauthor{Stefan}{0.6,0.6,0.6}
\newauthor{Stela}{0.8,0.8,0}




%%%%%%%%%%%%%%%%%
%% ENVIRONMENT %%
%%%%%%%%%%%%%%%%%

\theoremstyle{definition}
\newtheorem{thm}{Теорема}
\newtheorem{crl}{Следствие}
\newtheorem{cor}{Следствие}
\newtheorem{lemma}{Лема}
\newtheorem{prop}{Твърдение}
\newtheorem{dfn}{Определение}
\newtheorem{problem}{Задача}
\newtheorem{example}{Пример}
\newtheorem{question}{Въпрос}
\newtheorem*{remark}{Забележка}
\renewenvironment{proof}{\noindent{\bf Доказателство.}\hspace*{1em}}{\qed\par}
\newenvironment{solution}{\noindent{\bf Решение.}\hspace*{1em}}{\qed\par}
\newenvironment{hint}{\noindent{\bf Упътване.}\hspace*{1em}}{\qed\par}


%%%%%%%%%%%%%%%%%%
%%   HELPERS    %%
%%%%%%%%%%%%%%%%%%


\newcommand{\A}{\mathcal{A}}
\newcommand{\B}{\mathcal{B}}
\renewcommand{\C}{\mathcal{C}}
\newcommand{\M}{\mathcal{M}}
\renewcommand{\L}{\mathcal{L}}
\newcommand{\D}{\mathcal{D}}
\newcommand{\R}{\mathbb{R}}
\newcommand{\Z}{\mathbb{Z}}
\newcommand{\N}{\mathcal{N}}
\newcommand{\Q}{\mathbb{Q}}
\newcommand{\Ls}{\mathscr{L}}
\newcommand{\Fs}{\mathscr{F}}
\newcommand{\Rs}{\mathscr{R}}
\newcommand{\Ps}{\mathscr{P}}
\newcommand{\As}{\mathscr{A}}
\newcommand{\Bs}{\mathscr{B}}
\newcommand{\Es}{\mathscr{E}}
\newcommand{\Is}{\mathscr{I}}
\newcommand{\Ss}{\mathscr{S}}
\newcommand{\xn}{x_{1},\dots,x_{n}}

\newcommand{\Nat}{\mathbb{N}}
\newcommand{\Int}{\mathbb{Z}}
\newcommand{\Real}{\mathbb{R}}

\newcommand{\xs}{overline{x}}

\newcommand{\ys}{overline{y}}

\newcommand{\zs}{overline{z}}
\newcommand{\ov}[1]{\overline{#1}}
\newcommand{\abs}[1]{\lvert{#1}\rvert}
\newcommand{\pair}[1]{\langle{#1}\rangle}
\newcommand{\val}[1]{\llbracket{#1}\rrbracket}
\newcommand{\FA}{\langle{Q,\Sigma,s,\delta,F}\rangle}
\newcommand{\FAn}[1]{\langle{Q_#1,\Sigma,s_#1,\delta_#1,F_#1}\rangle}
\newcommand{\NFA}{\langle{Q,\Sigma,s,\Delta,F}\rangle}
\newcommand{\NFAn}[1]{\langle{Q_#1,\Sigma,s_#1,\Delta_#1,F_#1}\rangle}
\newcommand{\PDA}{\langle{Q,\Sigma,\Gamma,\#,s,\Delta,F}\rangle}
\newcommand{\PDAn}[1]{\langle{Q_#1,\Sigma,\Gamma,\#,s_#1,\Delta_#1,F_#1}\rangle}
\newcommand{\CFG}{\langle{V,\Sigma,R,S}\rangle}
\newcommand{\TM}{\langle{Q,\Sigma,\Gamma,\delta,s,\bot,F}\rangle}

\renewcommand{\iff}{\ \leftrightarrow\ }
\newcommand{\todo}{\ding{45} }
\newcommand{\dff}{\stackrel{\text{\footnotesize{дeф}}}{=}}
\newcommand{\Th}[1]{{\em Теорема~\ref{th:#1}}}
\newcommand{\Lem}[1]{{\em Лема~\ref{lem:#1}}}
\newcommand{\Prob}[1]{{\em Задача~\ref{pr:#1}}}
\newcommand{\Prop}[1]{{\em Твърдение~\ref{pr:#1}}}
\newcommand{\Eq}[1]{{\em Равенство~(\ref{eq:#1}})}


% \setlist[itemize]{leftmargin=*}



%\renewcommand\marginpar[1]{\oldmarginpar{\scriptsize #1}}

% \fontfamily{garamond}
\title{Записки по Дискретна математика}
\author{Стефан Вътев\thanks{ел. поща: \href{stefanv@fmi.uni-sofia.bg}{stefanv@fmi.uni-sofia.bg}}}
%, Факултет по математика и информатика, Софийски университет ,,Св. Климент Охридски''}}

% \usepackage{breqn}         % automatic equation breaking

\makeindex
\begin{document}
\maketitle
% \layout

\tableofcontents
\section*{Съждително смятане}

% Булевите алгебри носят името на Джордж Бул (1815 - 1864), който първи описва техните свойства в своята книга
% ``The Mathematical Analysis of Logic'' (1847).

% Висша математика, част 1, стр. 32
\begin{dfn}
  Логически израз начичаме съвкупността от съждения $p,q,r,\dots$, свързани със знаците за логически операции
  $\neg, \vee, \wedge, \rightarrow, \leftrightarrow$ и скоби, определящи реда на операциите.
\end{dfn}

\begin{dfn}
  Тъждествено верен (истинен) е този логически израз, който има верностна стойност {\bf 1} при всички възможни набори на
  верностните стойности на съждителните променливи в израза.
\end{dfn}

\begin{enumerate}[I)]
  \item
    Комутативен закон
    \[p\vee q \equiv q\vee p\] 
    \[p\ \wedge\ q \equiv q\ \wedge\ p\]
  \item
    Асоциативен закон
    \[(p\vee q)\vee r \equiv p\vee(q\vee r)\]
    \[(p\ \wedge\ q)\ \wedge\ r \equiv p\ \wedge\ (q\ \wedge\ r)\]
  \item
    Дистрибутивен закон
    \[p\ \wedge\ (q \vee r) \equiv (p\ \wedge q)\vee (p\ \wedge\ r)\]
    \[p\vee (q\ \wedge\ r) \equiv (p\vee q)\ \wedge\ (p\vee r)\]
  \item
    Закони на Де Морган
    \[\neg(p\ \wedge\ q) \equiv (\neg p \vee \neg q)\]
    \[\neg(p\vee q) \equiv (\neg p\ \wedge\ \neg q)\]
  \item
    Закон за контрапозицията
    \[p\rightarrow q \equiv \neg q \rightarrow \neg p\]
  \item
    Обобщен закон за контрапозицията
    \[(p\ \wedge\ q)\rightarrow r \equiv (p\ \wedge\ \neg r) \rightarrow \neg q\]
  \item
    Закон за изключеното трето
    \[p\vee \neg p \equiv {\mathbf 1}\]
  \item
    Закон за силогизма (транзитивност)
    \[[(p\rightarrow q)\ \wedge\ (q\rightarrow r)] \rightarrow [p\rightarrow r] \equiv {\mathbf 1}\]
\end{enumerate}

Лесно се проверява с таблиците за истинност, че законите са тъждествено верни.

\begin{problem}
  Проверете с таблици за истинност, че следните съждителни формули са тавтологии:
  \begin{enumerate}[a)]
  \item
    $(p\wedge q)\rightarrow p$;
  \item
    $p\rightarrow(p\vee q)$;
  \item
    $(p\rightarrow q) \iff (\neg q \rightarrow \neg p)$;
  \item
    $p\rightarrow q \equiv \neg p \vee q$
  \item
    $(p\ \wedge\ q) \rightarrow r \equiv p \rightarrow (q\rightarrow r)$
  \item
    $p\ \wedge\ q \equiv \neg(\neg p \vee \neg q)$
  \item
    $p \leftrightarrow q \equiv (p\rightarrow q)\ \wedge\ (q\rightarrow p)$
  \item
    $\neg(p\wedge q) \equiv (\neg p \vee \neg q)$;
  \item
    $\neg(p\vee q) \equiv (\neg p \wedge \neg q)$;
  \item
    $\neg(p\rightarrow q) \equiv (p\wedge \neg q)$;
\end{enumerate}
\end{problem}

Обърнете внимание, че $(p\rightarrow q)\rightarrow r$ не е еквивалентно на $p\rightarrow (q\rightarrow r)$, например
вземете $p = 0, q = 0, r = 0$.

% \begin{problem}
%   В един затвор имало трима арестанти - {\bf А}, {\bf Б} и {\bf В}, който дори бил и сляп.
%   Шерифът решил да пусне един от тях на свобода, затова завързал очите на {\bf А} и {\bf Б} и
%   им сложил по една шапка на главите. Шапките били общо пет, като три от тях били бели и две - черни.
%   След това шерифът им свалил превръзките на очите и им казал, че ще пусне този, който позне какъв
%   цвят е шапката му.
%   \begin{enumerate}[]
%   \item
%     {\bf А} казал, че не знае какъв цвят е шапката му и шерифът го върнал в ареста.
%   \item
%     {\bf Б} казал, че не знае какъв цвят е шапката му и шерифът го върнал в ареста.
%   \item
%     Слепият затворник {\bf В} отговорил правилно какъв цвят е шапката му и шерифът го пуснал.
%   \item
%     Какъв е цвета на шапката на {\bf В}?
%   \end{enumerate}
% \end{problem}
% \begin{proof}
%   Нека да означим с $а$ твърдението ``цветът на шапката на {\bf А} e черен'',
%   а съответно с $\ov{a}$ твърдението ``цветът на шапката на {\bf А} e бял'',
%   По аналогичен начин означаваме съждителните променливи $b$ и $c$ за арестантите {\bf Б} и {\bf В}.
%   Превеждаме твърденията в съждителни формули:
%   \begin{enumerate}[A)]
%   \item
%     Щом {\bf А} не знае какъв цвят е шапката му, то със сигурност шапките на главите на другите арестанти 
%     не са и двете черни. Получаваме
%     \[\ov{b}c\ \vee\ b\ov{c}\ \vee\ \ov{b}\ov{c}\]
%   \item
%     Щом и {\bf Б} не знае какъв цвят е шапката му, то шапката на {\bf В} не е черна, защото това ще означава, че
    
%   \end{enumerate}

  
%   Получаваме $(\ov{b}c\ \vee\ b\ov{c}\ \vee\ \ov{b}\ov{c})\ \wedge\ (a\ov{c}\ \vee\ \ov{a}\ov{c})$
% \end{proof}


\begin{problem}
  Да предположим, че сме на остров, който се обитава лъжци и благородници.
  Лъжците винаги лъжат, а благородниците винаги казват истината.
  Срещаме трима обитатели на този остров, наречени {\bf А}, {\bf Б} и {\bf В}.
  \begin{enumerate}[a)]
  \item
    \begin{enumerate}[]
    \item
      \marginpar{$a \iff \ov{a}\ov{b}\ov{c}$}
      {\bf А} казва "Всички сме лъжци".
    \item
      \marginpar{$b \iff (\ov{a}\ov{b}c\vee \ov{a}b\ov{c}\vee a\ov{b}\ov{c})$}
      {\bf Б} казва "Точно един от нас е благородник".
    \item
      Какви са {\bf А},{\bf Б} и {\bf В}?
    \end{enumerate}
  \item
    \begin{enumerate}[]
    \item
      \marginpar{$a \iff \ov{a}\ov{b}\ov{c}$}
      {\bf А} казва "Всички сме лъжци".
    \item
      \marginpar{$b \iff (\ov{a}bc\vee ab\ov{c}\vee a\ov{b}c)$}
      {\bf Б} казва "Точно един от нас е лъжец".
    \item
      Може ли да определим какъв е {\bf Б}?
    \item
      Може ли да определим какъв е {\bf В}?
    \end{enumerate}
  \item
    \begin{enumerate}[]
    \item
      \marginpar{$a \iff \ov{b}$}
      {\bf А} казва ``{\bf Б} е лъжец''.
    \item
      \marginpar{$b \iff (ac \vee \ov{a}\ov{c})$}
      {\bf Б} казва ``{\bf А} и {\bf В} са от един и същ тип, т.е. или и двамата са благородници, или и двамата са лъжци''.
    \item
      Какъв е {\bf В}?
    \end{enumerate}
  \end{enumerate}
\end{problem}
\begin{proof}
  \begin{enumerate}[a)]
  \item
    Нека съждителната променлива $a$ да има стойност {\bf 1}, ако {\bf A} е благородник и нека има стойност {\bf 0}, 
    ако {\bf A} е негодник.
    Тогава
    \begin{enumerate}[A)]
    \item
      \begin{align*}
        a \iff \ov{a}\ov{b}\ov{c} &\ \equiv\ (a \rightarrow \ov{a}\ov{b}\ov{c})\ \wedge\ (\ov{a}\rightarrow (a\vee b \vee c))\\
        & \equiv\ (\ov{a} \vee \ov{a}\ov{b}\ov{c}) \wedge (a \vee a\vee b \vee c)\\
        & \equiv\ \ov{a}\ \wedge\ (a \vee b \vee c)\\
        & \equiv\ \ov{a}a \vee \ov{a}b \vee \ov{a}c\\
        & \equiv\ \ov{a}b \vee \ov{a}c.
      \end{align*}
    \item
      \begin{align*}
        b \iff (\ov{a}\ov{b}c\vee \ov{a}b\ov{c}\vee a\ov{b}\ov{c}) &\ \equiv\ (b \rightarrow (\ov{a}\ov{b}c\vee \ov{a}b\ov{c}\vee a\ov{b}\ov{c}))\wedge (\ov{b}\rightarrow \ov{\ov{a}\ov{b}c\vee \ov{a}b\ov{c}\vee a\ov{b}\ov{c}})\\
        & \equiv\ (\ov{b} \vee \ov{a}\ov{b}c\vee \ov{a}b\ov{c}\vee a\ov{b}\ov{c}) \wedge (b\vee \ov{a}\ov{b}\ov{c}\vee abc \vee \ov{a}bc \vee a\ov{b}c \vee ab\ov{c})\\
        & \equiv\ (\ov{b} \vee \ov{a}b\ov{c}) \wedge (b \vee \ov{a}\ov{b}\ov{c} \vee  a\ov{b}c)\\
        & \equiv\  \ov{a}\ov{b}\ov{c} \vee  a\ov{b}c \vee \ov{a}b\ov{c}.
      \end{align*}
    \item
      Сега взимаме конюнкцията на А) и Б).
      \begin{align*}
        (\ov{a}b \vee \ov{a}c)\wedge (\ov{a}\ov{b}\ov{c} \vee  a\ov{b}c \vee \ov{a}b\ov{c}) &\ \equiv\ \ov{a}b\ov{c}.
      \end{align*}
    \end{enumerate}
  \end{enumerate}
\end{proof}

%%% Local Variables: 
%%% mode: latex
%%% TeX-master: "discrete-math"
%%% End: 

\section{Предикатно смятане}

\subsection*{Квантори}

Свойствата или отношенията на елементите в произволно множество се наричат {\bf предикати}.

Нека имаме един едноместен предикат $P(\cdot)$ дефиниран в множеството $A$.
\begin{enumerate}[(I)]
\item 
  {\bf Квантор за общност} $\forall x$.
  Записът $(\forall x \in A) P(x)$ означава, че за всеки елемент $a$ в $A$, 
  твърдението $P(a)$ има стойност истина.
  Например, $(\forall x\in\Nat)[(x+1)^2 = x^2+2x+1]$.
\item
  {\bf Квантор за съществуване} $\exists x$.
  Записът $(\exists x \in A) P(x)$ означава, че съществува елемент $a$ в $A$, 
  за който твърдението $P(a)$ има стойност истина.
  Например, $(\exists x \in\mathbf{C})[x^2 = -1]$.
\end{enumerate}

\begin{example}
  \begin{itemize}
  \item
    За всяко естествено число, съществува по-голямо от него:
    \[(\forall x\in\Nat)(\exists z\in\Nat)[x < z].\]
  \item
    Съществува естествено число, от което няма по-малко:
    \[(\exists x\in\Nat)(\forall y\in\Nat)[x < y \vee x = y].\]
    Нека да означим с $Zero(x)$ предиката, който казва, че $x$ е най-малкото число, т.е.
    \[(\forall y)[x < y \vee x =y].\]
  \item
    Нека $S(x,y)$ да бъде предиката, който казва, че $y = x+1$ в естествените числа:
    \[S(x,y) \equiv (x < y\ \wedge\ (\forall z\in\Nat)[x < z\ \rightarrow (z = y\ \vee\ y < z)].\]
  \item
    $O(x)$ - $x$ е числото $1$:
    \[(\exists y)[Z(y)\ \wedge\ S(y,x)].\]
  \item
    $Div(x,y)$ - $x$ се дели на $y$:
    \[(\exists z)[x = y.z].\]
  \item
    $Prime(x)$ - $x$ е просто число:
    \[x \geq 2\ \wedge\ (\forall y\in\Nat)[\neg (O(y)\ \wedge Z(y))\ \rightarrow\ \neg Div(x,y)].\]
  \end{itemize}
\end{example}


\subsection*{Закони на предикатното смятане}

\begin{enumerate}[(I)]
  \item
    $\neg\forall x P(x) \iff \exists x \neg P(x)$
  \item
    $\neg\exists x P(x) \iff \forall x \neg P(x)$
  \item
    $\forall x P(x) \iff \neg\exists x \neg P(x)$
  \item
    $\exists x P(x) \iff \neg\forall x \neg P(x)$
  \item
    $\forall x \forall y P(x) \iff \forall y\forall x P(x)$
  \item
    $\exists x\exists y P(x,y) \iff \exists y \exists x P(x)$  
  \item
    $\exists x\forall y P(x,y) \rightarrow \forall y \exists x P(x,y)$
\end{enumerate}


\begin{problem}
  Да означим с $K(x,y)$ твърдението ``$x$ познава $y$''.
  Изразете като формула следните твърдения.
  \begin{enumerate}[1)]
  \item
    \marginpar{$\forall x \exists y K(x,y)$}
    Всеки познава някого.
  \item
    \marginpar{$\exists x \forall y K(x,y)$}
    Някой познава всеки.
  \item
    \marginpar{$\exists x\forall y K(y,x)$}
    Някой е познаван от всички.
  \item
    \marginpar{$\forall x \exists y(K(x,y)\wedge \neg K(y,x)) $}
    Всеки знае някой, който не го познава.
  \item
    \marginpar{$\exists x \forall y(K(y,x)\ \rightarrow K(x,y))$}
    Има такъв, който знае всеки, който го познава.
  \end{enumerate}
\end{problem}

\begin{problem}
  Нека $G(x)$ означава, че човекът $x$ е добър.
  \begin{enumerate}[1)]
  \item
    Изразете с формула твърдението, че всички хора са добри {\em без}
    да използвате квантора $\forall$, а само квантора $\exists$ и логическите връзки.
  \item
    Изразете с формула твърдението, че {\em поне един} човек е добър {\em без}
    да използвате квантора $\exists$, а само квантора $\forall$ и логическите връзки.
  \end{enumerate}
\end{problem}

\begin{problem}
  \marginpar{(От \cite{smullyan})}
  На един остров живеели два вида обитатели - благородници и негодници.
  Благородниците винаги казвали истината, а негодниците винаги лъжели.
  Един пътешественик попаднал на този остров и искал да разбере повече за
  неговите обитатели. 
  Всеки обитател на острова му казал:
  \begin{enumerate}[a)]
  \item
    \marginpar{$\forall x(K(x) \leftrightarrow (\forall xK(x)\vee\forall x\neg K(x)))\rightarrow\ ?$}
    ``Всички тук сме от един и същ вид''.
    Какви са жителите на острова?
  \item
    \marginpar{$\forall x(K(x) \leftrightarrow (\exists xK(x)\wedge\exists x\neg K(x)))\rightarrow\ ?$}
    ``Някои от  нас са благородници и някои от нас са негодници''.
    Какви са жителите на острова?
  \end{enumerate}
\end{problem}
\begin{proof}
  Нека с $K(x)$ да означаваме, че жителят $x$ е благородник (от англ. Knight),
  и съответно с $\neg K(x)$ ще означаваме, че жителят $x$ е негодник.
  \begin{enumerate}[a)]
  \item 
    Tвърдението, което казва, че всички обитетели са от един и същ вид може да се преведе 
    на езика на предикатното смятане като:
    \[\forall x K(x) \vee \forall x \neg K(x).\]
    Тогава, понеже ако обитателят $x$ е благородник, той винаги казва истината, 
    следната формула е вярна:
    \[(\forall x)[K(x) \rightarrow (\forall x K(x) \vee \forall x \neg K(x))].\]
    Съответно ако $x$ е негодник, то той винаги казва лъжа, 
    \[(\forall x)[\neg K(x) \rightarrow \neg (\forall x K(x) \vee \forall x \neg K(x))].\]
    Така получаваме формулата
    \[(\forall x)[K(x) \iff (\forall x K(x) \vee \forall x \neg K(x))].\]
    \begin{itemize}
    \item 
      Ако има благородник на острова, то всички са благородници.
    \item
      Ако има негодник на острова, то има и благородник.
      Но щом има благородник, всички са благородници. Противоречие.
    \end{itemize}
  \item
    \begin{itemize}
    \item
      Ако има негодник на острова, то всички са негодници.
    \item
      Ако има благородник на острова, то има и негодник.
      Щом има негодник, то  всички са негодници. Противоречие.
    \end{itemize}
  \end{enumerate}
\end{proof}

\begin{problem}
  След това пътешественикът попаднал на друг остров, на който
  той силно се интересувал от това дали обитателите пушат.
  Жителите на този остров му отговорили така.
  \begin{enumerate}[a)]
  \item
    \marginpar{$\forall x(K(x) \leftrightarrow \forall y(K(y)\rightarrow S(y))) \rightarrow\ ?$}
    ``Всички благородници пушат.''
    Какви са жителите на острова и пушат ли ?
  \item
    \marginpar{$\forall x(K(x) \leftrightarrow \exists y(\neg K(y)\wedge S(y))) \rightarrow\ ?$}
    ``Някои негодници пушат.''
    Какви са жителите на острова и пушат ли ?
  \end{enumerate}
\end{problem}
\begin{solution}
  \begin{enumerate}[a)]
  \item 
    \begin{itemize}
    \item 
      Има благородник на острова. Тогава всички благородници пушат.
      Ако има също така и негодник на острова, то тогава има благородник, който не пуше.
      Това е  противоречие. Следователно, ако има поне един благородник, то всички обитатели са благородници.
    \item
      Има негодник на острова. Тогава има благородник на острова, който не пуше.
      Но тогава пък всички благородници пушат, което е противоречие.
    \end{itemize}
    Заключаваме, че всички обитателя на острова са благородни пушачи.
  \item
    Да допуснем, че има благородник на острова. Следователно има и негодник, който пуше.
    Но тогава $(\forall y)[K(y) \vee \neg S(y)]$, което означава, че
    всички негодници са непушачи. Достигнахме до противоречие.
    Следователно всички обитатели на острова са негодници, които са непушачи.
  \end{enumerate}
\end{solution}

\begin{problem}
  Пътешественикът отишъл и на трети остров, на който всички обитатели били от един и същ вид.
  \marginpar{Добавяме $\forall xK(x) \vee \forall x\neg K(x)$}
  \begin{enumerate}[a)]
  \item
    \marginpar{$\forall x(K(x) \leftrightarrow (S(x) \rightarrow \forall yS(y)))$}
    ``Ако аз пуша, то всички пушат.''
  \item
    \marginpar{$\forall x(K(x) \leftrightarrow (\exists yS(y) \rightarrow S(x)))$}
    ``Ако някой обитател на острова пуше, то и аз пуша.''
  \item
    \marginpar{$\forall x(K(x) \leftrightarrow (\exists yS(y) \wedge \neg S(x)))$}
    ``Някои пушат, но аз не.''
  \item
    \marginpar{$\forall x(K(x) \leftrightarrow \exists yS(y))\ \wedge$ $\forall x(K(x) \iff \neg S(x))$}
    ``Някои пушат.'' и след това добавили - ``Аз не пуша.''.
  \end{enumerate}
  Какво можем да кажем за обитателите на този остров?
\end{problem}
\begin{solution}
  \begin{enumerate}[a)]
  \item
    Ако всички негодници, то $(\forall x)[S(x)\ \wedge\ \exists y\neg S(y)]$,
    което е невъзможно.
    Следователно, всички са благородници. Тогава или всички пушат или никой не пуше.
  \item
    Ако всички са негодници, то $(\forall x)[\exists y S(y)\ \wedge\ \neg S(x)]$,
    което е невъзможно. Следователно, всички са благородници.
    Тогава или всички пушат или никой не пуше.
  \item
    Ако всички са благородници, то $(\forall x)[\exists y S(y)\ \wedge \neg S(x)]$,
    което е невъзможно. Следователно, всички  са негодници.
    Тогава $(\forall x)[\forall y\neg S(y)\ \vee S(x)]$, което означава, че
    или всички пушат или никой не пуше.
  \item
    Тази ситуация е невъзможна!
  \end{enumerate}
\end{solution}


%%% Local Variables: 
%%% mode: latex
%%% TeX-master: "discrete-math"
%%% End: 

% \chapter{Теория на множествата}

\begin{dfn}
  Множество е колекция от обекти.
  Използваме означението $x \in A$, 
  че обектът $x$ принадлежи на множеството $A$.
\end{dfn}

{\bf Празното множество} означаваме с $\emptyset$.
То има следното свойство:
\[(\forall x)[x \not \in \emptyset]\]
или еквивалентно,
\[\neg (\exists x)[x \in \emptyset].\]

\begin{example}
  Ето няколко примера за множества, които ще използваме често:
  \begin{align*}
    \Nat = & \{0,1,2,\dots\},\\
    \mathbb{Z} = & \{\dots,-2,-1,0,1,2,\dots\},\\
    \mathbb{Q} = & \{\frac{m}{n} \mid m,n \in \mathbb{Z}\ \&\ n \neq 0\}.
  \end{align*}

\end{example}


% Не му е  мястото на това тук! Да се премести.
% \section{Декартово произведение}
%   Въвеждаме операция наредена двойка $\langle{x,y}\rangle$, която искаме да има следните свойства:
%   \begin{enumerate}
%   \item
%     $\langle{x,y}\rangle = \langle{x',y'}\rangle \iff x = x' \ \&\ y = y'$;
%   \item
%     класът $A\times B = \{\langle{x,y}\rangle\ \mid\ x\in A\ \&\ x\in B\}$ е множество.
% \end{enumerate}


% \begin{dfn}[Куратовски]
%   Наредена двойка\index{наредена двойка} $\langle{x,y}\rangle = \{\{x\},\{x,y\}\}$
% \end{dfn}

% Първото свойство се проверява лесно.
% За второто свойство, достатъчно е да покажем, че за произволни множества $A,B$ можем да 
% изберем множество $C$, за което е изпълнено, че
% \[x\in A\ \&\ x\in B \rightarrow \{x,\{x,y\}\}\in C.\]
% Ако успеем да намерим такова множество $C$, то тогава от аксиомата за отделянето следва, че $A\times B$
% е множество, защото $A\times B = \{ z\in C\ \mid\ (\exists x\in A)(\exists y\in B)[z = \langle{x,y}\rangle]\}$ е множество.

% Лесно може да се провери, че $C = \Ps(\Ps(A\cup B))$ върши работа.

% Възможно е да се дадат и други дефиниции на наредена двойка.
% \begin{problem}
%   Проверете кои от следните операции отговарят на условията за наредена двойка.
%   \begin{enumerate}
%   \item
%     $\langle{x,y}\rangle_{1} = \{x,y\}$;
%   \item
%     $\langle{x,y}\rangle_{2} = \{x,\{y\}\}$;
%   \item
%     $\langle{x,y}\rangle_{3} = \{\{\emptyset,\{x\}\},\{\{y\}\}\}$;
%   \item
%     $\langle{x,y}\rangle_{4} = \{\{0,x\},\{1,y\}\}$, 
%     където $0,1$ са различни обекти.
% \end{enumerate}
% \end{problem}

% \begin{problem}
%   Проверете:
%   \begin{enumerate}
%   \item
%     $A\times B = \emptyset \iff A = \emptyset \vee B = \emptyset$
%   \item
%     $A\times(B\cup C) = (A\times B)\cup(A\times C)$
%   \item
%     $A\times(B\cap C) = (A\times B)\cap(A\times C)$ 
%   \item
%     $A\times(B\backslash C) = (A\times B)\backslash(A\times C)$
%   \item
%     $(A\cap B)\times (C\cap D) = (A\times C)\cap(B\times D)$
%   \item
%     $(A\cup B)\times (C\cup D) = (A\times C)\cup(B\times D)$
%   \item
%     $(A\backslash C)\times(B\backslash D)\subsetneq (A\times B)\backslash(C\times D)$
%   \end{enumerate}
% \end{problem}

\subsection*{Сравняване на множества}

Казваме, че едно множество $A$ {\bf се включва в} множеството $B$, което означаваме с $A \subseteq B$, 
ако всеки елемент, който принадлежи на $A$, принадлежи и на $B$, т.е.
\[(\forall x)[x \in A\ \rightarrow\ x \in B].\]
Обикновено ще казваме, че $A$ е {\bf подмножество} на $B$.
Ето няколко примера:
\begin{itemize}
\item 
  $\emptyset \subseteq A$, за всяко множество $A$.
\item
  $\{1,2\} \subseteq \{1,2,3\}$.
\item
  $\{\{\emptyset\}\} \subseteq \{\{\emptyset\},\emptyset\}$.
\item
  $\Nat \subseteq \mathbb{Z}$ и $\mathbb{Z} \subseteq \mathbb{Q}$.
\end{itemize}
Две множества $A$ и $B$ са {\bf равни}, ако
\[A = B \iff A \subseteq B\ \&\ B\subseteq A.\]

\section*{Операции върху множества}

\begin{dfn}
  Определяме следните операции върху произволни множества $A$ и $B$.
  \begin{enumerate}[{\bf (I)}]
  \item
    {\bf Сечение}
    \[A\cap B = \{x\ \mid\ x\in A\ \&\ x\in B\}.\]
    Казано малко по-формално, $A\cap B$ е множеството, за което е изпълнена формулата
    \[(\forall x)[x \in A\cap B \iff (x\in A\ \wedge\ x \in B)].\]
    Примери:
    \begin{itemize}
    \item 
      $A \cap \emptyset = \emptyset$, за всяко множество $A$.
    \item
      $\{1,2,\emptyset\} \cap \{\emptyset\} = \{\emptyset\}$.
    \item
      $\{1,2,\{1,2\}\} \cap \{1,\{1\}\} = \{1\}$.
    \end{itemize}
  \item
    {\bf Обединение}
    \[A\cup B = \{x\ \mid x\in A\ \vee\ x\in B\}.\]
    \[(\forall x)[x \in A\cup B \iff (x\in A\ \vee\ x \in B)].\]
    $A\cup B$ е множеството, за което е изпълнена формулата
    Примери:
    \begin{itemize}
    \item 
      $A \cup \emptyset = A$, за всяко множество $A$.
    \item
      $\{1,2,\emptyset\} \cup \{1,2,\{\emptyset\}\} = \{1,2,\emptyset,\{\emptyset\}\}$.
    \item
      $\{1,2,\{1,2\}\} \cup \{1,\{1\}\} = \{1,2,\{1\},\{1,2\}\}$.
    \end{itemize}
  \item
    {\bf Разлика}
    \[A\setminus B = \{x\ \mid\ x\in A\ \&\ x\not\in B\}.\]
    $A\setminus B$ е множеството, за което е изпълнена формулата
    \[(\forall x)[x \in A\setminus B \iff (x\in A\ \wedge\ x \not\in B)].\]
    Примери:
    \begin{itemize}
    \item 
      $A \setminus \emptyset = A$, за всяко множество $A$.
    \item 
      $\emptyset \setminus A = \emptyset$, за всяко множество $A$.
    \item
      $\{1,2,\emptyset\} \setminus \{1,2,\{\emptyset\}\} = \{\emptyset\}$.
    \item
      $\{1,2,\{1,2\}\} \setminus \{1,\{1\}\} = \{2,\{1,2\}\}$.
    \end{itemize}
  \item
    {\bf Симетрична разлика}
    \[A\triangle B = (A\backslash B)\cup (B\backslash A).\]
    $A\triangle B$ е множеството, за което е изпълнена формулата
    \[(\forall x)[x \in A\triangle B \iff [(x\in A\ \wedge\ x \not\in B) \vee (x \in B\ \wedge\ x\not\in A)]].\]
    Примери:
    \begin{itemize}
    \item 
      $A \triangle \emptyset = A$, за всяко множество $A$.
    \item
      $A \triangle A = \emptyset$, за всяко множество $A$.
    \item
      $A\triangle B = B \triangle A$, за всеки две множества $A$ и $B$.
    \item
      $\{1,2,\emptyset\} \triangle \{1,2,\{\emptyset\}\} = \{\emptyset\} \cup \{\{\emptyset\}\} = \{\emptyset,\{\emptyset\}\}$.
    \item
      $\{1,2,\{1,2\}\} \triangle \{1,\{1\}\} = \{2,\{1,2\}\} \cup \{\{1\}\} = \{2,\{1\},\{1,2\}\}$.
    \end{itemize}
  \item
    {\bf Степенно множество}
    \[\Ps(A) = \{x\mid x\subseteq A\}.\]
    $\Ps(A)$ е множеството, за което е изпълнена формулата
    \[(\forall x)[x \in \Ps(A) \iff (\forall y)[y\in x\rightarrow y \in A]].\]
    Примери:
    \begin{itemize}
    \item 
      $\Ps(\emptyset) = \{\emptyset\}$, $\Ps(\{\emptyset\}) = \{\emptyset,\{\emptyset\}\}$.
    \item
      $\Ps(\{\emptyset,\{\emptyset\}\}) = \{\emptyset,\{\emptyset\},\{\{\emptyset\}\},\{\emptyset,\{\emptyset\}\}\}$.
    \item
      $\Ps(\{1,2\}) = \{\emptyset,\{1\},\{2\},\{1,2\}\}$.
    \end{itemize}
  \end{enumerate}
  Нека имаме редица от множества $\{A_1,A_2,\dots,A_n\}$.
  Тогава имаме следните операции:
  \begin{enumerate}[{\bf (I)}]
  \item
    {\bf Обединение на редица от множества}
    \[\bigcup^{n}_{i=1} A_i = \{x \mid \exists i (1\leq i\leq n\ \&\ x\in A_i)\}.\]
    \[(\forall x)[x \in \bigcup^n_{i=1}A_i \iff (\exists i)[1 \leq i \leq n\ \wedge\ x \in A_i]].\]
  \item
    {\bf Сечение на редица от множества}
    \[\bigcap^{n}_{i=1} A_i = \{x \mid \forall i (1\leq i\leq n \rightarrow x\in A_i)\}.\]
    \[(\forall x)[x \in \bigcap^n_{i=1}A_i \iff (\forall i)[1 \leq i \leq n\ \rightarrow\ x \in A_i]].\]
  \end{enumerate}
\end{dfn}

% Това е много сложно!
% Тук имаме проблем с значението на $\bigcap\emptyset$.
% На пръв поглед изглежда, че $\bigcap\emptyset$ е множеството от всички множества $V$, 
% но ние знаем, че такова множество не съществува.
% Това в известен смисъл е аналог на делението на нула.
% Ние ще приемем, че $\bigcap\emptyset = \emptyset$.

\begin{example}
  Нека $A = \{x\in\N\mid x > 1\}$ и $B = \{x\in\N\mid x>3\}$. Тогава :
    \begin{enumerate}[]
    \item
      $A\cap B = \{x\in\N\mid x > 3\}$,
    \item
      $A\cup B = \{x\in\N\mid x > 1\}$,
    \item
      $A\setminus B = \{x\in\N\mid 1<x\leq 3\}$,
    \item
      $B\setminus A = \emptyset$,
    \item
      $A\triangle B = \{x\in\N\mid 1<x\leq 3\}$
    \end{enumerate}
\end{example}


\begin{problem}
  Нека $A = \{x\in\R\mid |x|\leq 1\}$ и $B = \{x\in\R\mid |x-1|\leq \frac{1}{2}\}$.
  Намерете множествата $A\cup B$, $A\cap B$, $A\setminus B$, $B\setminus A$, $A\triangle B$.
\end{problem}

% \begin{example}
%   \[\bigcap\{\{1,2,3,4\},\{2,4\},\{1,3,4\}\} = \{4\}\]
%   \[\bigcup\{\{3\},\{2,4\},\{1,4\}\} = \{1,2,3,4\}\]
%   \[\bigcap\{\{a\},\{a,b\}\} = \{a\}\cap\{a,b\} = \{a\}\]
%   \[\bigcup\bigcap\{\{a\},\{a,b\}\}  = \bigcup\{a\} = a\]
% \end{example}

% \begin{problem}
%   Нека $B = \{\{1,2\},\{2,3\}, \{1,3\}, \{\emptyset\}\}$.
%   Намерете $\bigcup{B}$, $\bigcap{B}$, $\bigcap\bigcup{B}$ и $\bigcup\bigcap{B}$.
% \end{problem}

% \begin{example}
  % Ето няколко примера, които показват действието на някои от операциите
  % \begin{enumerate}[]
%     \item
%       $\bigcup\{\emptyset\} = \emptyset$
%     \item
%       $\bigcup\{\emptyset,\{\emptyset\}\} = \{\emptyset\}$
%     \item      
%       $\bigcup\{\emptyset,\{\emptyset\},\{\{\emptyset\}\}, \{\emptyset,\{\emptyset\}\}\} = \{\emptyset,\{\emptyset\}\}$
%     \end{enumerate}
%   \item
%     $\bigcap\{\emptyset,\{\emptyset\}\} = \emptyset$
% \end{enumerate}
% \end{example}



% \begin{problem}
%   \begin{enumerate}
%   \item
%     Намерете двуелементно множество такова, че всеки елемент на множеството да е също и негово подмножество.
%   \item
%     Намерете триелементно множество такова, че всеки елемент на множеството да е също и негово подмножество.
%   \item
%     Намерете четириелементно множество такова, че всеки елемент на множеството да е също и негово подмножество.
% \end{enumerate}
% \end{problem}


% \begin{problem}
%   Докажете:
%   \begin{enumerate}
%   \item
%     $\bigcup\Ps A = A$;
%   \item
%     $A\subseteq\Ps\bigcup A$; кога имаме равенство?
%   \item
%     $\Ps A \cap \Ps B = \Ps(A\cap B)$;
%   \item
%     $\Ps A \cup \Ps B \subseteq\Ps(A\cup B)$; кога имаме равенство?
%   \item
%     съществуват множества $a$ и $B$, за които $a\in B$, но $\Ps{a}\not\subseteq\Ps{B}$;
%   \item
%     ако $a\in B$, то $\Ps{a}\in\Ps\Ps{B}$;
%   \item
%     $\{\emptyset,\{\emptyset\}\} \in \Ps\Ps{A}$, за всяко множество $A$.
%   \end{enumerate}
% \end{problem}

\begin{problem}
  Намерете $\Ps(A)$, където:
  \begin{enumerate}[a)]
  \item
    $A= \emptyset$.
  \item
    $A= \{\{1,2\}\}$.
  \item
    $A= \{\emptyset, \{\emptyset\}\}$.
  \item
    $A= \{\emptyset, \{1,2\}, 7\}$.
  \item
    $A= \{1,2,3,4\}$.
  \end{enumerate}
\end{problem}

\begin{problem}
  Проверете:
  \begin{enumerate}[a)]
  \item
    $A\subseteq B \iff A\setminus B = \emptyset \iff A\cup B = B \iff A\cap B = A$;
  \item
    $A\setminus \emptyset = A$, $\emptyset\setminus A=\emptyset$, $A\setminus B = B\setminus A$.
  \item
    $A\cap (B\cup A) = A \cap B$;
  \item
    $A\cup(B\cap C) = (A\cup B)\cap(A\cup C)$ и $A \cap (B \cup C) = (A \cup B) \cap (A \cup C)$;
  \item
    $C\subseteq A\ \&\ C\subseteq B \rightarrow C\subseteq A\cap B$;
  \item
    $A\subseteq C\ \&\ B\subseteq C \rightarrow A\cup B\subseteq C$;
  \item
    $(\bigcup^{n}_{i=1} A_i) \cap B = \bigcup^{n}_{i=1} (A_i \cap B)$;
  \item
    $(\bigcap^{n}_{i=1} A_i) \cup B = \bigcap^{n}_{i=1} (A_i \cup B)$;
  \item
    $A\backslash B = A \iff A\cap B = \emptyset$;
  \item
    $A\backslash B = A\backslash (A\cap B)$ и $A\backslash B = A\backslash (A\cup B)$;
  \item
    $(A\cup B)\setminus C = (A\setminus C) \cup (B\setminus C)$;
  \item
    $C\backslash (A\cup B) = (C\backslash A)\cap(C\backslash B)$ и $C \backslash (A\cap B) = (C\backslash A)\cup(C\backslash B)$
  \item
    $C\backslash(\bigcup^{n}_{i=1} A_i) = \bigcap^{n}_{i=1} (C\backslash A_i)$ и $C \backslash(\bigcap^{n}_{i=1} A_i) = \bigcup^{n}_{i=1} (C\backslash A_i)$;
  % \item
  %   $A\cup\bigcap B = \{A\cup X\mid X\in B\}$, за $B\neq\emptyset$
  % \item
  %   $A\cap\bigcup B = \{A\cap X\mid X\in B\}$
  \item
    $(A\backslash B)\backslash C = (A\backslash C)\backslash(B \backslash C)$ и $A\backslash (B\backslash C) = (A\backslash B) \cup (A\cap C)$;
  \item
    $A\triangle B = B\triangle A$
  \item
    $A\triangle(B\triangle C) = (A\triangle B)\triangle C$
  \item
    $A\backslash B = A\triangle(A\cap B)$
  \item
    $A\cap(B\triangle C) = (A\cup B)\triangle(A\cup C)$
  \item
    $A\cup B = (A\triangle B)\cup(A\cap B)$
  \item
    $A\triangle A = A$ и $A\triangle B = \emptyset \iff A = B$;
  \item
    $A\triangle B = C \iff B\triangle C = A \iff C\triangle A = B$;
  \item 
    $A\subseteq B \Rightarrow \Ps(A) \subseteq \Ps(B)$;
  \item
    $\Ps(A\cap B) = \Ps(A) \cap \Ps(B)$ и $\Ps(A\cup B) = \Ps(A) \cup \Ps(B)$;
  \end{enumerate}
\end{problem}

\begin{problem}
  Да се решат системите с променлива $X$:
  \begin{enumerate}[a)]
  \item
    \begin{tabular}{l c l}
      $\big|A\setminus X$ & $= $ & $ B$\\
      $\big|X\setminus A $ & $=$ & $ C$,
    \end{tabular}
    
    където са дадени множествата $A,B,C$ и $B\subseteq A$, $A\cap C = \emptyset$;
  \item
    \begin{tabular}{l c l}
      $\big|A\cap X$ & $= $ & $ B$\\
      $\big|A\cup X $ & $=$ & $ C$,
    \end{tabular}
    
    където са дадени множествата $A,B,C$ и $B\subseteq A\subseteq C$;
  \item
    \begin{tabular}{l c l}
      $\big|A\setminus X$ & $= $ & $ B$\\
      $\big|A\cup X $ & $=$ & $ C$,
    \end{tabular}

    където са дадени множествата $A,B,C$ и $B\subseteq A\subseteq C$.
  \end{enumerate}
\end{problem}

\begin{example}
  Нека съвкупността от обекти $D$ е определена като
  \[D = \{A\mid A\mbox{ е множество и } A\not\in A\}.\]
  Тогава:
  \begin{enumerate}[a)]
  \item
    Ако $D \in D$, то $D \not\in D$. Противоречие.
  \item
    Ако $D \not\in D$, то $D \in D$. Противоречие.
  \end{enumerate}
\end{example}


% \begin{problem}
%   Нека множеството $A$ е дефинирано по следния начин:
%   \begin{enumerate}
%   \item
%     $0\in A$
%   \item
%     Ако $x\in A$, то $2x+1 \in A$.
% \end{enumerate}
% Намерете $A$.
% \end{problem}
% \begin{proof}
%   $A = \{2^n - 1\ \mid n\in\N\}$.
% \end{proof}

% \begin{thm}
%   Нека множеството $A$ е дефинирано по следния начин:
%   \begin{enumerate}[(1)]
%   \item
%     $1\in A$
%   \item
%     Ако $m,n\in A$, то $2m+3n \in A$.
%   \item
%     Всички елементи на $A$ са добавени или по правило (1) или правило (2).
% \end{enumerate}
% Намерете $A$.
% \end{thm}
% \begin{proof}
%   Нека $B = \{n \mid n\equiv 1 (\mod 12)\ \vee n\equiv 5 (\mod 12) \}$.
%   Искаме да докажем, че $A = B$.
%   Първо ще докажем, че $A\subseteq B$.
%   За целта проверяваме, че $1\in B$ и ако $m,n \in B$, то $2m+3n \in B$.
  
%   За другата посока, т.е. $B\subseteq A$, трябва да докажем, че ако
%   за всяко $k\leq n$ е вярно, че $12k+1 \in B$ и $12k + 5 \in B$,
%   то е вярно, че $12(n+1)+1 \in B $ и $12(n+1) + 5 \in B$.
% \end{proof}




%%% Local Variables: 
%%% mode: latex
%%% TeX-master: "discrete-math"
%%% End: 

\chapter{Релации}
\index{релация}

За да дадем определение на понятието релация, трябва първо 
да въведем понятието декартово произведение на множества,
което пък от своя страна се основава на понятието наредена двойка.

\subsection*{Наредена двойка}

За два елемента $a$ и $b$ въвеждаме опрецията {\bf наредена двойка} $\pair{a,b}$.
Наредената двойка $\pair{a,b}$ има следното характеристичното свойство:
\[a_1 = a_2\ \wedge\ b_1 = b_2\ \iff\ \pair{a_1,b_1} = \pair{a_2,b_2}.\]
Понятието наредена двойка може да се дефинира по много начини, стига да изпълнява харектеристичното свойство.
Ето примери как това може да стане:
\begin{enumerate}[1)]
\item
  \marginpar{Norbert Wiener (1914)}
  Първото теоретико-множествено определение на понятието наредена двойка е
  дадено от Норберт Винер:
  \[\pair{a,b} = \{\{\{a\},\emptyset\},\{\{b\}\}\}.\]
\item
  \marginpar{Kazimierz Kuratowski (1921)}
  Определението на Куратовски се приема за ,,стандартно`` в наши дни:
  \[\pair{a,b} = \{\{a\},\{a,b\}\}.\]
\end{enumerate}

\begin{problem}
  Докажете, че горните дефиниции наистина изпълняват харектеристичното свойство за наредени двойки.
\end{problem}

\begin{remark}
  %\marginpar{Пример за рекурсивна дефиниция}
  Сега можем да въведем понятието наредена $n$-орка $\pair{a_1,\dots,a_n}$ за всяко естествено число $n \geq 1$:
  \begin{align*}
    & \pair{a_1} = a_1,\\
    & \pair{a_1,a_2,\dots,a_n} = \pair{a_1,\pair{a_2,\dots,a_n}}
  \end{align*}
\end{remark}
 
\section{Декартово произведение}
\marginpar{На англ. cartesian product}
\index{декартово произведение}
\marginpar{Считаме, че $(A\times B)\times C = A\times (B\times C) = A\times B \times C$}
За две множества $A$ и $B$, определяме тяхното декартово произведение като
\[A\times B = \{\pair{a,b}\mid a\in A\ \&\ b\in B\}.\]
За краен брой множества $A_1,A_2,\dots,A_n$, определяме
\[A_1\times A_2\times\cdots\times A_n = \{\pair{a_1,a_2,\dots,a_n}\mid a_1 \in A_1\ \&\ a_2\in A_2\ \&\ \dots\ \&\ a_n \in A_n\}.\]


\begin{problem}
  Преверете дали:
  \begin{enumerate}[a)]
  \item 
    $A\times(B\cup C) = (A\times B) \cup (A\times C)$;
  \item
    $(A\cup B)\times C = (A\times C)\cup (B\times C)$;
  \item 
    $A\times(B\cap C) = (A\times B) \cap (A\times C)$;
  \item
    $(A \cap B)\times C = (A \times C)\cap(B\times C)$;
  \item 
    $A\times(B\setminus C) = (A\times B) \setminus (A\times C)$;
  \item
    $(A\setminus B)\times C = (A\times C)\setminus (B\times C)$;
  \item
    $(A\triangle B)\times C = (A\times C)\triangle (B\times C)$;
  \item
    $\Ps(A \times B) = \Ps(A) \times \Ps(B)$;
  \end{enumerate}
\end{problem}

\section{Основни видове бинарни релации}
\index{бинарна релация}
% Подмножествата $R$ от вида $R \subseteq A\times A\times\cdots\times A$ се наричат релации.
Релациите от вида $R\ \subseteq\ A\times A$ са важен клас, който ще срещаме често.
Да разгледаме няколко основни вида релации от този клас:
\begin{enumerate}[I)]
\item
  {\bf рефликсивна}, ако
  \[(\forall x\in A)[\pair{x,x}\in R].\]
  Например, релацията $\leq\ \subseteq\ \Nat\times\Nat$ е рефлексивна, защото
  \[(\forall x\in \Nat)[x \leq x].\]
\item
  {\bf антирефлексивна}, ако
  \[(\forall x\in A)[\pair{x,x}\not\in R].\]
  Например, релацията $<\ \subseteq\ \Nat\times\Nat$ е антирефлексивна, защото
  \[(\forall x\in \Nat)[x \not< x].\]
\item
  {\bf транзитивна}, ако
  \[(\forall x,y,z\in A)[\pair{x,y}\in R\ \&\ \pair{y,z}\in R \rightarrow \pair{x,z}\in R].\]
  Например, релацията $\leq\ \subseteq\ \Nat\times\Nat$ е транзитивна, защото
  \[(\forall x,y,z\in A)[x \leq y\ \&\ y \leq z\ \rightarrow\ x\leq z].\]
\item
  {\bf симетрична}, ако
  \[(\forall x,y\in A)[\pair{x,y}\in R \rightarrow \pair{y,x}\in R].\]
  Например, релацията $=\ \subseteq\ \Nat\times\Nat$ е рефлексивна, защото
  \[(\forall x,y\in \Nat)[x = y\ \rightarrow\ y = x].\]
\item
  {\bf антисиметрична}, ако
  \[(\forall x,y\in A)[\pair{x,y}\in R\ \&\ \pair{y,x}\in R \rightarrow x = y].\]
  Например, релацията $\leq\ \subseteq\ \Nat\times\Nat$ е антисиметрична, защото
  \[(\forall x,y,z\in A)[x \leq y\ \&\ y \leq x\ \rightarrow\ x = y].\]
\item
  {\bf асиметрична}, ако
  \[(\forall x,y)[\pair{x,y}\in R \rightarrow \pair{y,x}\not\in R].\]
  Например, релацията $\leq\ \subseteq\ \Nat\times\Nat$ е асиметрична, защото
  \[(\forall x,y\in \Nat)[x < y\ \rightarrow\ y \not< x].\]
\end{enumerate}

\begin{remark}
  Добре е да запомните как се наричат тези основни видове релации, защото ще ги използваме често.
  Обърнете също внимание, че ако една релация {\em не} е рефлексивна, то това не означава, че тя е антирефлексивна.
  Също така, ако една релация {\em не} е симетрична, то това не означава, че тя е антисиметрична или асиметрична.
\end{remark}

\begin{example}
  Да обобщим примерите от по-горе.
  \begin{enumerate}[a)]
  \item
    Релацията $\leq\ \subseteq\ \Nat\times\Nat$ е рефлексивна, транзитивна и антисиметрична.
  \item
    Релацията $<\ \subseteq\ \Nat\times\Nat$ е антирефлексивна, транзитивна и асиметрична.
  \item
    Релацията $=\ \subseteq\ \Nat\times\Nat$ е рефлексивна, транзитивна и симетрична.
  \end{enumerate}
\end{example}

\begin{problem}
  Проверете кои от горе-изброените свойства притежава релацията $R$:
  \begin{enumerate}[a)]
  \item
    \marginpar{Озн. $\Nat^2 = \Nat\times\Nat$}
    $R\subseteq \Nat^2$ и е определна като 
    \marginpar{$a\vert b \iff (\exists k\in\Nat)(b = k\cdot a)$}
    \[(a,b) \in R \iff a | b.\]
  \item
    $R \subseteq \Z\times \Z$ е определена като
    \[(x,y)\in R \iff \mbox{НОД}(x,y) = 1\]% \iff \neg(\exists z > 1)[\ z\vert x\ \wedge\ z\vert y\ ]\]
  \item
    \marginpar{Озн. $\R$ - реалните числа}
    $R\subseteq \Int^2$ и е определена като
    \[(a,b) \in R \iff a\cdot b > 0.\]
  \item
    $R\subseteq \Int^2$ и е определена като 
    \[(a,b) \in R \iff a+b = 0.\]
  \item
    $R\subseteq \Int^2$ и е определена като
    \[(a,b) \in R \iff a+b = 5.\]
  \item
    $R\subseteq \Int^2$ и е определена като 
    \[(a,b) \in R \iff a+b\mbox{ е четно}.\]
  \item
    $R\subseteq (\Int^2)^2$ и е определена като
    \[(\pair{a,b}, \pair{c,d}) \in R \iff a+d = b+c.\]
  \item
    $R\subseteq (\Int^2)^2$ и е определена като
    \[(\pair{a,b},\pair{c,d}) \in R \iff a\cdot d = b\cdot c.\]
  \item
    $R\subseteq (\Int\times (\Int\setminus\{0\}))^2$ и е определена като
    \[(\pair{a,b},\pair{c,d}) \in R \iff a\cdot d = b\cdot c.\]
  \item
    \marginpar{Озн. $\Z$ - целите числа}
    $R_{m}\subseteq \Z^2, m\in \Z, m>0$ и е определена като
    \[\pair{a,b} \in R_m \iff m\mid (a - b).\]
  \item
    $R\subseteq \R^2$ и е определена като 
    \[(x,y) \in R \iff (x-y)\mbox{ е рационално число}.\]
  \item
    \marginpar{Озн. $\Q$ - рационалните числа}
    $R \subseteq \Q^2$ и е определена като
    \[(p,r) \in R\ \iff\ p-r \mbox{ е цяло число}.\]
  \item
    $R \subseteq \Nat^2$ и е определена като
    \[(a,b) \in R \iff a = b \vee a+1 = b.\]
  \item
    $R \subseteq \Nat^2$ и е определена като
    \[\pair{a,b}\in R \iff (\exists k\in\Nat)[a+k = b].\]
  \item
    Нека $\leq_1\ \subseteq\ A^2$ и $\leq_2\ \subseteq\ B^2$ са частични наредби.
    $R \subseteq A^2\times B^2$ е определена като
    \[(\pair{a,b}, \pair{c,d}) \in R \iff a\leq_{1}c\ \wedge\ b\leq_{2}d .\]
  \item
    Нека $\leq_1\ \subseteq\ A^2$ и $\leq_2\ \subseteq\ B^2$ са частични наредби.
    $R \subseteq A^2\times B^2$ е определена като
    \[(\pair{a,b}, \pair{c,d}) \in R \iff a\leq_{1}c\ \vee\ b\leq_{2}d .\]

  \end{enumerate}
\end{problem}

\begin{problem}
  Нека $R$ и $S$ са релации на еквивалентност върху множеството $A$.
  Какви свойства притежават следните релации:
  \begin{enumerate}[a)]
  \item
    $R \cap S$;
  \item
    $R \cup S$;
  \item
    $R \setminus S$;
  % \item
  %   $R \circ S$.
  \end{enumerate}
\end{problem}

\begin{problem}
  Нека $\Sigma = \{1,2,\dots,9\}$ и $n$ е ествествено число.
  Нека $R \subseteq \Sigma^n \times \Sigma^n$, където
  $(\pair{a_1,a_2,\dots,a_n},\pair{b_1,b_2,\dots,b_n}) \in R \iff a_i + 1 \equiv b_1 (\mod 2)$.
\end{problem}

\begin{problem}
  За едно число $x \in \Nat$, нека с $x_{(2)}$ да означаваме най-късия двоичния запис на $n$,
  а с $N_0(x)$ и $N_1(x)$ броя на $0$-лите и съответно броя на $1$-ците в $x_{(2)}$.
  Нека $R \subseteq \Nat\times\Nat$, където
  \begin{enumerate}
  \item 
    $(x,y) \in R$ iff $N_0(x) \leq N_0(y)\ \&\ N_1(x) \leq N_1(y)$.
  \item
    $(x,y) \in R$ iff $N_0(x) = N_0(y)\ \vee\ N_1(x) = N_1(y)$.
  \end{enumerate}
\end{problem}



%% Rosen Textbook
\begin{problem}
  Кои от следните бинарни релации върху множеството на функциите от $\Z$ в $\Z$
  са релации на еквивалентност? Опишете техните класове на еквивалентност.
  \begin{enumerate}[a)]
  \item
    $R = \{(f,g)\mid f(1) = g(1)\}$.
  \item
    $R = \{(f,g)\mid f(0) = g(0)\wedge f(1) = g(1)\}$.
  \item
    $R =\{(f,g)\mid (\forall x\in\Z)[f(x)-g(x) = 1]\}$.
  \item
    $R = \{(f,g)\mid (\exists c\in\Z)(\forall x\in\Z)[f(x)-g(x) = c]\}$.
  \item
    $R = \{(f,g)\mid f(0) = g(1)\wedge f(1) = g(0)\}$.
  \end{enumerate}
\end{problem}

\section{Релации над думи}

\begin{itemize}
\item
  Азбука е крайно множество $\Sigma = \{a_1,\dots,a_n\}$ като елементите $a_i$ на $\Sigma$ наричаме {\bf букви}.
\item
  Нека да фиксираме един елемент $\varepsilon \not\in \Sigma$.
  Сега ще определим {\bf думите} над азбуката $\Sigma$. Това са:
  \begin{itemize}
  \item
    \marginpar{наричаме $\varepsilon$ празната дума}
    $\varepsilon$ е дума над $\Sigma$;
  \item
    \marginpar{т.е. думите са крайни последователности от букви}
    Нека $\alpha$ е дума над $\Sigma$ и $a \in \Sigma$ е буква.
    Тогава $\alpha a$ е дума над $\Sigma$;
  \item
    няма други думи над $\Sigma$.
  \end{itemize}
\item
  Означаваме с $\Sigma^n$ множеството от всички думи с дължина $n$ над азбуката $\Sigma$, $\Sigma^0 = \{\varepsilon\}$,
  защото празната дума е единствената дума с дължина $0$.
\item
  Със $\Sigma^\star$ означаваме множеството от всички думи над азбуката $\Sigma$, т.е.
  \marginpar{$0 \in \Nat$}
  \[\Sigma^\star = \bigcup_{n\in\Nat} \Sigma^{n}.\]
\end{itemize}

Сега ще определим функцията {\bf дължина}\index{дума!дължина} на дума.
\marginpar{Функцията $\abs{\cdot}:\Sigma^\star\to\Nat$ е винаги сюрективна. Кога е биективна?}
Дължината $|\alpha|$ на думата $\alpha \in \Sigma^\star$ се определя с индукция по построението на $\alpha$.
\begin{enumerate}[(i)]
  \item
    Нека $\alpha = \varepsilon$. Тогава $|\alpha| = 0$.
  \item
    Нека $\alpha = \beta a$, за някоя дума $\beta\in \Sigma^\star$ и някоя буква $a\in X$.
    Тогава \[|\alpha| = |\beta| + 1.\]
\end{enumerate}

Определяме функцията {\bf конкатенация}\index{дума!конкатенация} $\cdot$, т.е.
слепване на две думи.
За всеки две думи $\alpha$ и $\beta$ от $\Sigma^\star$ определяме тяхната конкатенация с индукция по дължината $\beta$:
\marginpar{$\cdot:\Sigma^\star\times \Sigma^\star \to \Sigma^\star$ е винаги сюрективна. Може ли да бъде биективна?}
\begin{enumerate}[(i)]
  \item
    $|\beta| = 0$, т.е. $\beta = \varepsilon$.
    Тогава $\alpha\cdot\beta = \alpha$.
  \item
    $|\beta| = n+1$, т.е. $\beta = \gamma b$, за някоя дума $\gamma$, $|\gamma| = n$, и някоя буква $b\in\Sigma$.
    Тогава $\alpha\cdot\beta = (\alpha\cdot\gamma)\cdot b$.
\end{enumerate}

Казваме, че думата $\alpha$ е {\bf начало} на думата $\beta$, ако съществува дума $\gamma \in \Sigma^\star$ такава, че
$\beta = \alpha\cdot\gamma$. Обикновено означаваме $\alpha \preceq \beta$.
Аналогично дефинираме $\alpha$ да бъде {\bf край} на думата $\beta$, ако съществува $\gamma \in \Sigma^\star$ такава, че
$\beta = \gamma \cdot \alpha$.

\begin{problem}
  Нека $\Sigma = \{0,1\}$.
  Какви свойства имат следните бинарни релации над $\Sigma^\star$ ?
  Ако $R$ е релация на  еквивалентност, то опишете нейните класове на еквивалентност 
  и намерете техния брой.
  \begin{enumerate}[a)]
  \item
    $(\alpha,\beta) \in R \iff \alpha \preceq \beta$;
  \item
    $(\alpha,\beta) \in R  \iff (\exists\gamma\in \Sigma^\star)[\exists a,b\in \Sigma)(\gamma a \preceq \alpha\ \&\ \gamma b \preceq \beta\ \&\ a \neq b]$;
  \item
    $(\alpha,\beta) \in R  \iff (\exists\gamma\in \Sigma^\star)[\exists a,b\in \Sigma)(\gamma a \preceq \alpha\ \&\ \gamma b \preceq \beta\ \&\ a < b]$;
  \item
    $(\alpha,\beta) \in R  \iff \alpha \preceq \beta \vee (\exists\gamma\in \Sigma^\star)[\exists a,b\in \Sigma)(\gamma a \preceq \alpha\ \&\ \gamma b \preceq \beta\ \&\ a < b]$;
  \item
    $(\alpha,\beta) \in R \iff |\alpha| = |\beta|\ \& (\forall i \leq |\alpha|)[a_i \leq b_i]$;
  \item
    $(\alpha,\beta) \in R \iff (\forall i \leq \min\{|\alpha|,|\beta|\})[a_i \leq b_i]$;
  \item
    $(\alpha,\beta) \in R \iff (\exists \gamma_1,\gamma_2 \in \Sigma^\star)[\beta = \gamma_1 \alpha \gamma_2]$;
  \item
    за фиксирано число $n$,
    \[(\alpha,\beta) \in R \iff (\exists\gamma\in\Sigma^n)[\gamma\preceq\alpha\ \&\ \gamma\preceq\beta].\]
  \item
    за фиксирано число $n$,
    \[(\alpha,\beta) \in R \iff \alpha = \beta \vee (\alpha \neq \beta\ \&\ (\exists\gamma\in\Sigma^n)[\gamma\preceq\alpha\ \&\ \gamma\preceq\beta]).\]
  \item
    за фиксирано число $n$,
    \[(\alpha, \beta)\in R \iff |\alpha| = |\beta| > n\ \&\ (\forall i > n)[a_i = b_i].\]
  \end{enumerate}
\end{problem}

\begin{problem}
  Да фиксираме две множества $B \subseteq A$. Дефинираме бинарната релация $R$ върху $\Ps(A)$ по следния начин:
  \[R = \{(X,Y) \in \Ps(A)\times\Ps(A) \mid (X \triangle Y) \subseteq B\}.\]
  Докажете, че:
  \begin{enumerate}[a)]
  \item 
    $R$ е релация на еквивалентност;
  \item
    за всяко $X \in \Ps(A)$ съществува точно едно множество $Y \in [X]_R$, за което $Y \cap B = \emptyset$.
  \end{enumerate}
\end{problem}


\section{Операции върху релации}
\begin{enumerate}[I)]
\item
  {\bf Композиция} на две релации $S \subseteq B\times C$ и $T \subseteq A\times B$ е релацията $S\circ T \subseteq A\times C$,
  определена като:
  \[S\circ T = \{\langle{a,c}\rangle \mid (\exists b \in B)[\pair{a,b}\in T\ \&\ \pair{b,c} \in S]\}.\]
\item
  {\bf Обръщане} на релацията $R \subseteq A\times B$ е релацията $R^{-1}\subseteq B\times A$, 
  определена като:
  \[R^{-1} = \{\pair{x,y} \mid \pair{y,x} \in R\}.\]
  \item
  \marginpar{Очевидно е, че $P$ е рефлексивна релация, дори ако $R$ не е.}
  {\bf Рефлексивно затваряне} на релацията $R \subseteq A\times A$ е релацията
  \[P = R \cup \{\pair{a,a} \mid a\in A\}.\]
\item
  {\bf Итерация} на релацията $R \subseteq A\times A$ дефинираме като за всяко естествено число $n$,
  дефинираме релацията $R^n$ по следния начин:
  \marginpar{Лесно се вижда, че  $R^1 = R$}
  \begin{align*}
    R^0 & = A\times A\\
    R^{n+1} & = R^n \circ R.
  \end{align*}
\item
  \marginpar{\ding{45} Проверете, че $R^+$ е транзитивна релация!}
  {\bf Транзитивно затваряне} на $R \subseteq A\times A$ е релацията
  \[R^+ = \bigcup_{n\geq 1} R^n.\]
\end{enumerate}

За дадена релация $R$, с $R^\star$ ще означаваме нейното рефлексивно и транзитивно затваряне.
От дефинициите е ясно, че \[R^\star = \bigcup_{n\geq 0} R^n.\]

\begin{problem}
  Ако $P$ е множеството от всички хора, да разгледаме релациите
  \begin{align*}
    E & = \{\pair{x,y} \in P\times P \mid \text{$x$ е враг на $y$}\},\\
    F & = \{\pair{x,y} \in P\times P \mid \text{$x$ е приятел на $y$}\}.
  \end{align*}
  Ако приемем, че поговорката ,,врагът на моя враг е мой приятел'' е вярна, то 
  какво ни казва това за релациите $E$ и $F$ ?
\end{problem}

\begin{example}
  Да разгледаме релацията $R$ над $\Nat$, за която
  $R = \{\pair{x,y} \in \Nat^2 \mid x+1 = y\}$. Тогава:
  \begin{itemize}
  \item 
    $R^0 = \{\pair{x,x} \mid x\in \Nat\}$;
  \item
    $R^n = \{\pair{x,y} \in \Nat^2 \mid x+n = y\}$, за всяко $n \in \Nat$;
  \item
    $R^+ = \{\pair{x,y}\in\Nat^2 \mid (\exists n\geq 1)[x+n = y]\} = \{\pair{x,y} \in \Nat^2 \mid x < y\}$;
  \item
    $R^\star = \{\pair{x,y} \in \Nat^2 \mid x \leq y\}$.
  \end{itemize}
\end{example}


\begin{problem}
  Дайте пример за релации $R$ и $S$, за които
  \[R\circ S \neq S\circ R.\]
\end{problem}

  
\begin{problem}
  Докажете, че:
  \begin{enumerate}[a)]
  \item
    $R$ е симетрична тогава и само тогава, когато $R^{-1}\subseteq R$;
  \item
    $R$ е транзитивна тогава и само тогава, когато $R\circ R\subseteq R$;
  \item
    $R$ е транзитивна и симетрична тогава и само тогава, когато $R = R^{-1}\circ R$.
\end{enumerate}
\end{problem}
\begin{proof}
  \begin{enumerate}[a)]
  \item
    Задачата се разделя на две подзадачи.
    \begin{enumerate}[(i)]
    \item
      Нека $R$ да бъде симетрична. Ще докажем, че $R^{-1}\subseteq R$, т.е.
      \[(\forall x\forall y)[(x,y)\in R^{-1} \rightarrow (x,y)\in R].\]
      Нека $(x,y)\in R^{-1}$. Тогава по определение имаме, че $(y,x)\in R$ и следователно $(x,y)\in R$,
      защото $R$ е симетрична.
    \item
      Нека $R^{-1}\subseteq R$. Щe докажем, че $R$ е симетрична, т.е.
      \[(\forall x\forall y)[(x,y)\in R \rightarrow (y,x)\in R].\]
      Нека $(x,y)\in R$, следователно по определение $(y,x)\in R^{-1}$.
      Тогава от $R^{-1}\subseteq R$ следва, че $(y,x)\in R$.
    \end{enumerate}
  \item
    Тази задача е лесна.
  \item
    Нека $R^{-1}\circ R = R$. Ще докажем, че $R$
    е симетрична и транзитивна.
    \begin{enumerate}[(i)]
    \item
      Ще докажем, че $R$ e симетрична.
      За целта е достатъчно да вземем произволна двойка $\pair{x,y} \in R$
      и да покажем, че $\pair{y,x} \in R$.
      Нека 
      % \begin{prooftree}
      %   \AxiomC{$\pair{x,y} \in R$}
      %   \RightLabel{\scriptsize($R^{-1}\circ R = R$)}
      %   \UnaryInfC{$\pair{x,y} \in R^{-1}\circ R$}
      %   \UnaryInfC{$(\exists z)[\pair{x,z} \in R\ \wedge\ \pair{z,y} \in R^{-1}]$}
      %   \UnaryInfC{$(\exists z)[\pair{z,x} \in R^{-1}\ \wedge\ \pair{y,z} \in R]$}
      %   \UnaryInfC{$(\exists z)[\pair{y,z} \in R\ \wedge\ \pair{z,x} \in R^{-1}]$}
      %   \UnaryInfC{$\pair{y,x} \in R^{-1}\circ R$}
      %   \RightLabel{\scriptsize($R^{-1}\circ R = R$)}
      %   \UnaryInfC{$\pair{y,x} \in R$}
      % \end{prooftree}
      \begin{align*}
        \pair{x,y} \in R & \iff \pair{x,y} \in R^{-1}\circ R & (\text{имаме, че }R^{-1}\circ R = R)\\
        & \iff (\exists z)[\pair{x,z} \in R\ \wedge\ \pair{z,y} \in R^{-1}]\\
        & \iff (\exists z)[\pair{z,x} \in R^{-1}\ \wedge\ \pair{y,z} \in R]\\
        & \iff (\exists z)[\pair{y,z} \in R\ \wedge\ \pair{z,x} \in R^{-1}]\\
        & \iff \pair{y,x} \in R^{-1}\circ R\\
        & \iff \pair{y,x} \in R.
      \end{align*}
      
      Следователно,
      \[(\forall x,y \in A)[\pair{x,y} \in R\ \rightarrow\ \pair{y,x} \in R].\]
    \item
      Ще докажем, че $R$ e транзитивна.
      За целта е достатъчно да вземем произволни двойки $\pair{x,y} \in R$
      и $\pair{y,z} \in R$, то $\pair{x,z} \in R$.
      
      \begin{prooftree}
      \AxiomC{$\pair{x,y} \in R$}
      \AxiomC{$\pair{y,z} \in R$}
      \RightLabel{\scriptsize($R$ е симетрична)}
      \UnaryInfC{$\pair{z,y} \in R$}
      \UnaryInfC{$\pair{y,z} \in R^{-1}$}
      \BinaryInfC{$\pair{x,z} \in R^{-1}\circ R$}
      \RightLabel{\scriptsize($R^{-1}\circ R = R$)}
      \UnaryInfC{$\pair{x,z} \in R$}
      \end{prooftree}
      Следователно,
      \[(\forall x,y,z \in A)[(\pair{x,y} \in R\ \wedge\ \pair{y,z} \in R) \rightarrow\ \pair{x,z} \in R].\]
    \end{enumerate}
    Нека сега $R$ е транзитивна и симетрична.
    Ще докажем, че $R^{-1}\circ R = R$.
    
    \begin{enumerate}[(i)]
    \item 
      Първо да отбележим, че
      \begin{prooftree}
        \AxiomC{$\pair{x,y} \in R$}
        \AxiomC{$\pair{x,y} \in R$}
        \RightLabel{\scriptsize($R$ е симетрична) }
        \UnaryInfC{$\pair{y,x} \in R$}
        \RightLabel{\scriptsize($R$ е транзитивна) }
        \BinaryInfC{$\pair{x,x} \in R\ \wedge \pair{y,y}\in R$}
      \end{prooftree}
      Следователно,
      \[(\forall x \in Dom(R))[\pair{x,x} \in R].\]
    \item
      Ще докажем, че $R^{-1}\circ R \subseteq R$.
      
      \begin{prooftree}
        \AxiomC{$\pair{x,z}\in R$}
        \AxiomC{$\pair{x,y} \in R^{-1}\circ R$}
        \RightLabel{\scriptsize(Съществува $z$)}
        \UnaryInfC{$\pair{z,y} \in R^{-1}$}
        \UnaryInfC{$\pair{y,z} \in R$}
        \RightLabel{\scriptsize($R$ е симетрична)}
        \UnaryInfC{$\pair{z,y} \in R$}
        \RightLabel{\scriptsize($R$ е транзитивна)}
        \BinaryInfC{$\pair{x,y} \in R$}
      \end{prooftree}
    \item
      Ще докажем, че $R \subseteq R^{-1}\circ R$.
      \begin{prooftree}
        \AxiomC{$\pair{x,y}\in R$}
        \AxiomC{$\pair{x,y}\in R$}
        \RightLabel{\scriptsize(От (i))}
        \UnaryInfC{$\pair{y,y} \in R$}
        \UnaryInfC{$\pair{y,y} \in R^{-1}$}
        \RightLabel{\scriptsize(по деф.)}
        \BinaryInfC{$\pair{x,y} \in R^{-1}\circ R$}
      \end{prooftree}
    \end{enumerate}
  \end{enumerate}
  
\end{proof}

\begin{problem}
  Нека $\{\pair{a,b}\}\subseteq R$, за някои $a\neq b$.
  Докажете, че ако $R$ е симетрична, то $R$ не е антисиметрична.
\end{problem}
% \begin{proof}
%   Нека $R$ е симетрична.
% \end{proof}

\begin{problem}
  Нека $R$ да бъде релация на еквивалентност върху $B$ и $f:A\to B$.
  Дефинираме множеството 
  \[Q = \{((x,y)\in A\times A\mid (f(x),f(y))\in R\}.\]
  Докажете, че $Q$ е релация на еквивалентност.
\end{problem}

\begin{problem}
  Нека $R$ е релация върху $A$.
  Да определим релациите:
  \begin{itemize}
  \item 
    $S = \{\pair{x,y}\mid \pair{x,y} \in R\ \wedge\ \pair{y,x} \in R\}$;
  \item
    $T = \{\pair{x,y} \mid \pair{x,y}\in R\ \wedge\ \pair{y,x}\not\in R\}$.
  \end{itemize}
  Докажете, че:
  \begin{enumerate}[a)]
  \item 
    $S$ е симетрична и $T$ е антисиметрична.
  \item
    $\pair{x,y} \in R\ \iff\ (\pair{x,y}\in S \vee \pair{x,y} \in T)$;
  \item
    ако $R$ е транзитивна, то $S$ и $T$ са също транзитивни, но обратната посока не е вярна.
  \end{enumerate}
\end{problem}

Нека $R \subseteq A\times A$ е бинарна релация.
Да определим множеството $[x]_R$ като
\[[x]_R = \{y\in A\mid \pair{x,y} \in R\}.\]
\index{клас на еквивалентност}
Ако $R$ е релация на еквивалентност и $x\in Field(R)$, то наричаме множеството $[x]_R$ {\bf клас на еквивалентност} за $x$ относно релацията $R$.

\begin{example}
  Нека $S$ е множеството от всички студенти, които живеят в Студентски град.
  Да разгледаме следната бинарна релация над $S$:
  \[R = \{\pair{x,y} \in S \times S \mid \text{$x$ живее в същия блок като $y$}\}.\]
  Лесно се съобразява, че $R$ е релация на еквивалентност.
  Тогава всеки блок в Студентски град определя отделен клас на еквивалентност.
  Елементите на такъв клас са всички студенти, които живеят в съответния блок.
\end{example}

\begin{example}
  \marginpar{$x \equiv y \mod 4 \iff (\exists k \in \Z)(x = y + 4\cdot k)$}
  Нека $\sim_4\ \subseteq\ \Nat\times \Nat$ е бинарна релация, дефинирана като
  \[x\sim_4 y \iff x\equiv y \mod 4.\]
  $\sim_4$ е релация на еквивалентност и има четири класа на еквивалентност
  \[[0]_{\sim_4}, [1]_{\sim_4}, [2]_{\sim_4}, [3]_{\sim_4}.\]
\end{example}

\begin{problem}
  Да дефинираме релацията $R \subseteq \R\times \R$ като:
  \[\pair{x,y}\in R \iff (x-y)\in\Z.\]
  Намерете множествата $[1]_R$ и $[\frac{1}{2}]_R$.
\end{problem}

\newpage
\section{Наредби}
Обикновено се изучават релации притежаващи различни комбинации от горните свойства. 
Сега ще изброим няколко основни вида релации (понякога се наричат наредби).
Релацията $R \subseteq A\times A$ се нарича:
\begin{itemize}
\item
  \marginpar{На англ. partial order}
  {\bf частична наредба}, ако тя е рефлексивна, транзитивна и антисиметрична.
  Например, $\leq$ е частична наредба.
  Също така, релацията $\subseteq$ между множества е частична наредба.
\item 
  \marginpar{На англ. equivalence relation}
  {\bf релация на еквивалентност}, ако тя е рефлексивна, транзитивна и симетрична.
  Например, $=$ е релация на еквивалентност.
\item
  \marginpar{На англ. linear order}
  {\bf линейна наредба}\index{наредба!линейна}, ако $R$ е частична наредба, 
  и за всеки два елемента $x,y$ точно едно от $\pair{x,y} \in R$, $x = y$, $\pair{y,x}\in R$ е изпълнено.
\item
  \marginpar{На англ. well-founded order}
  {\bf фундирана наредба}\index{наредба!фундирана}, 
  ако всяко непразно подмножество $X\subseteq A$ притежава поне един {\em минимален} елемент, т.е.
  \[(\forall X\subseteq A)[X\neq\emptyset \rightarrow (\exists m\in X)\neg(\exists y\in X)[\pair{y,m} \in R]].\]
\item
  \marginpar{На англ. well-ordered relation}
  {\bf добра наредба}\index{наредба!добра}, ако всяко непразно подмножество $X\subseteq A$ има {\em най-малък} елемент , т.е.
  \[(\forall X\subseteq A)[X\neq\emptyset \rightarrow (\exists m\in X)(\forall y\in X)[\pair{m,y}\in R \vee m = y]].\]
\item
  \marginpar{На англ. lexicographical order}
  {\bf лексикографска наредба} върху частичната наредба $(X,<)$ ще наричаме 
  наредбата $(X\times X,\prec)$, където
  \[\pair{x,y}\prec\pair{x^\prime,y^\prime}\ \iff\ x<x^\prime\ \vee\ (x = x^\prime\ \wedge\ y < y^\prime).\]
\end{itemize}

\begin{remark}
  Обърнете внимание на разликата между понятията {\em минимален} и {\em най-малък} елемент относно релацията $R$.
  \begin{itemize}
  \item $x_0$ е {\bf минимален} елемент за множеството $X \subseteq A$ относно $R$,
    ако не съществуват елементи $y \in X$, за които $\pair{y,x}\in R$, т.е.
    \[(\forall y \in X)[\pair{y,x} \not\in R].\]
  \item $x_0$ е {\bf най-малък} елемент за множеството $X \subseteq A$ относно $R$,
    ако за всеки друг елемент $y \in X$ е изпълнено $\pair{x,y}\in R$, т.е.
    \[(\forall y \in X)[x\neq y \rightarrow \pair{x,y} \in R].\]
  \end{itemize}
\end{remark}

\begin{problem}
  Да се докаже, че $(\Nat^2,\prec)$ е добре наредено множество.
\end{problem}

\begin{problem}
  Докажете, че следните две условия за частично наредено множество $(X,<)$ са еквивалентни:
  \begin{enumerate}[a)]
  \item
    всяко непразно подмножество на $X$ има минимален елемент;
  \item
    не съществува строго намаляваща редица $x_1>x_2>x_3>\dots$ то елементи на $X$.
  \end{enumerate}
\end{problem}

\begin{problem}
  Да се докаже, че частично нареденото множество $(X,<)$ е добре наредено тогава и само тогава, когато 
  $(X,<)$ е фундирано и $<$ е линейна наредба върху $X$.
\end{problem}

\begin{problem}% от СЕП
  Кои от изброените множества са фундирани? Кои са добре наредени?
  \begin{enumerate}[a)]
  \item
    $(\Nat,<)$;
  \item
    $(\Z,<)$;
  \item
    $(X^*, <)$, където за $\alpha,\beta\in X^*$, $\alpha < \beta \iff \alpha\mbox{ е поддума на }\beta$;
  \item
    $(2^\Nat,\subsetneq)$;
  \item
    $(Fin(\Nat),\subsetneq)$, където $Fin(\Nat)$ е съвкупността от всички крайни подмножества на $\Nat$;
  \item
    $(\Nat^+,|)$, където $m|n \iff m\neq n\ \&\ m\mbox{ дели }n$.
  \end{enumerate}
\end{problem}

\begin{problem}
  Докажете, че подмножество на всяка антисиметрична релация е също антисиметрична.
\end{problem}





%%% Local Variables: 
%%% mode: latex
%%% TeX-master: "discrete-math"
%%% End: 

\chapter{Функции}
\index{функция}

\section{Основни свойства}

Релацията $R \subseteq A\times B$ се нарича {\bf тотална функция}\index{тотална функция} от $A$ в $B$, ако
\begin{enumerate}[i)]
\item
  $Dom(R) = A$, т.е.
  \[(\forall a\in A)(\exists b\in B)[(a,b)\in R].\]
\item
  За всеки елемент $a\in A$ съотвества {\em точно един} елемент $b \in B$, т.е.
  \[(\forall a\in A)(\forall b_1,b_2 \in B)((\langle{a,b_1}\rangle\in R\ \wedge\ \langle{a,b_2}\rangle\in R) \rightarrow b_1 = b_2).\]
\end{enumerate}
Обикновено означаваме функциите като $f:A\to B$ и
вместо $(a,b)\in f$ пишем $f(a) = b$.
Казваме, че функцията $f$ e
\begin{itemize}
\item
  \marginpar{или $f$ е {\bf обратима}}
  {\bf инекция}\index{функция!инекция}, ако 
  \[(\forall a_1,a_2\in A)[a_1\neq a_2 \rightarrow f(a_1)\neq f(a_2)],\]
  или еквивалентно,
  \[(\forall a_1,a_2\in A)[f(a_1) = f(a_2) \rightarrow a_1 = a_2].\]
\item
  \marginpar{или $f$ е {\bf върху} $B$}
  {\bf сюрекция}\index{функция!сюрекция}, ако 
  \[(\forall b\in B)(\exists a\in A)[f(a) = b].\]
\item
  {\bf биекция}\index{функция!биекция}, ако е инекция и сюрекция.
\end{itemize}

\begin{problem}
  Дайте примери за функция $f:\mathbb{N}\rightarrow\Z$, която е:
  \begin{enumerate}[a)]
  \item
    инективна;
  \item
    сюрективна;
  \item
    нито инективна, нито сюрективна;
  \item
    инективна, но не е сюрективна;
  \item
    сюрективна, но не е инективна;
  \item
    сюрективна и инективна.
  \end{enumerate}
\end{problem}


Да разгледаме функцията $f \subseteq (\Real\times\Real)\times\Real$, дефинирана като:
\[f(x,y) = \frac{x}{y}.\]
Както много добре знаем, за стойности от вида $(x,0)$, функцията $f$ не е дефинирана.
Такива функции ще наричаме {\bf частични}.
Официалната дефиниция е следната.
\marginpar{Тук нямаме условието $Dom(R) = A$}
Релацията $R \subseteq A\times B$ се нарича {\bf частична функция}\index{частична функция} от $A$ в $B$, ако
за всеки елемент $a\in A$ съотвества {\em най-много един} елемент $b \in B$, т.е.
\[(\forall x\in A)(\forall y_1,y_2 \in B)[(\langle{x,y_1}\rangle\in R\ \wedge\ \langle{x,y_2}\rangle\in R) \rightarrow y_1 = y_2].\]

\begin{problem}
  За всяка от следните тотални функции $f$ определете дали $f$ е
  инекция, сюрекция или биекция.
  \begin{enumerate}[a)]
  \item
    \marginpar{(биекция)}
    $f: \Real\rightarrow \Real$, $f(x) = 2x+3$.
  \item
    $f: \Real\rightarrow \Real$, $f(x) = x^2 - 4x +2$.
  \item 
    $f: \Real\rightarrow \Real$, $f(x) = x^3+7$.    
  \item
    $f: \Nat\rightarrow \Nat$, 
    $f(x) = 
      \begin{cases}
        x+1, & \mbox{ ако }x\mbox{ е четно}\\
        x-1, & \mbox{ ако }x\mbox{ е нечетно}\\
      \end{cases}$
    \item
    \marginpar{$rem(x,3)$ - остатък при деление на $3$}
    $f: \Nat\rightarrow \Nat$, $f(x) = rem(x,3)$.
  \item 
    \marginpar{НОД - най-голям общ делител}
    $f: \Nat\times\Nat \rightarrow \Nat$,
    $f(x, y) = \mbox{ НОД}(x,y)$.
  \item 
    \marginpar{НОК - най-малко общо кратно}
    $f: \Nat\times\Nat \rightarrow \Nat$,
    $f(x, y) = \mbox{ НОК}(x,y)$.
  \item 
    $f: \Nat \times \Nat\rightarrow \Nat$,
    $f(x, y) = 3x+2y$.
  \item 
    $f: \Nat \times \Nat\rightarrow \Nat$,
    $f(x, y) = 2^x(2y+1)-1$.
  \item 
    \marginpar{Канторово кодиране}
    $f: \Nat \times \Nat\rightarrow \Nat$,
    $f(x, y) = \frac{(x+y)(x+y+1)}{2} + x$.
  \item 
    $f: \Nat \times \Nat\rightarrow \Nat$,
    $f(x, y) = 2x(2y+1)$.
  \item
    $f: \Nat \times \Nat\rightarrow \Nat$,
    $f(x, y) = 2x(2^y+1)$.
  \item
    $f: \Nat \times \Nat\rightarrow \Nat$,
    $f(x, y) = 2^x3^y$.
  \item
    $f: \Nat \times \Nat\rightarrow \Nat$,
    $f(x, y) = 2^x6^y$.
  \item 
    $f: \Real\times\Real\rightarrow \Real$,
    $f(x, y) = x^2+y^2$.
  \end{enumerate}
\end{problem}

\begin{problem}
  Докажете:
  \begin{enumerate}[a)]
  \item
    Ако $f,g$ са функции, то $f\cap g$ е функция;
  \item
    Нека $f,g$ са функции и $(\forall x)[x\in Dom(f)\cap Dom(g)\rightarrow f(x) = g(x)]$.
    Докажете, че $f\cup g$ е функция.
  \end{enumerate}
\end{problem}

\section{Операции върху функции}


\subsection*{Образ и първообраз}
Нека е дадена функцията $f:A\to B$.
Ще разгледаме няколко основни операции върху функции.

\begin{itemize}
\item 
  {\bf Образ на множеството} $X\subseteq A$ под действието на функцията $f$, наричаме
  множеството: \[f(X) = \{b\in B \mid f(a) = b\ \wedge\ a \in X\}.\]
\item
  {\bf Първообраз на множеството} $Y\subseteq B$ под действието на функцията $f$, наричаме
  множеството: \[f^{-1}(Y) = \{a\in A \mid f(a) = b\ \wedge\ b \in Y\}.\]
\end{itemize}


\begin{example}
  Нека $f:\Real\to\Real$ е дефинирана като $f(x) = \abs{x+1}$ и нека $A = [-1,1)$.
  \begin{itemize}
  \item
    $f(A) = \{f(x) \mid x\in A\} = \{\abs{x+1} \mid x \in [-1,1)\} = [0, 2)$.
  \item
    $f^{-1}(A) = \{x\in\Real\mid f(x) \in A\} = \{x\in\Real\mid \abs{x+1} \in [-1,1)\} = [-1,0)$.
  \end{itemize}
\end{example}

\begin{problem}
  Нека е дадена произволна функция $f:A \to B$.
  Проверете:
  \begin{enumerate}[a)]
  \item
    $(\forall X,Y \subseteq B)[f^{-1}(X\cup Y) = f^{-1}(X)\cup f^{-1}(Y)$.
  \item
    $(\forall X,Y \subseteq B)[f^{-1}(X\cap Y) = f^{-1}(X)\cap f^{-1}(Y)$.
  \item
    $(\forall X,Y \subseteq B)[f^{-1}(X\backslash Y) = f^{-1}(X)\backslash f^{-1}(Y)]$.
  \item
    $(\forall X\subseteq A)(\forall Y\subseteq B)[f(X)\cap Y = f(X\cap f^{-1}(Y))]$.
  \item
    $(\forall X \subseteq A)(\forall Y \subseteq B)[f(X)\cap Y = \emptyset \iff X\cap f^{-1}(Y) = \emptyset]$.
  \item
    $(\forall X \subseteq A)(\forall Y \subseteq B)[f(X)\subseteq Y \iff X\subseteq f^{-1}(Y)]$.
  \item
    $(\forall X,Y \subseteq A)[f(X)\cup f(Y) = f(X\cup Y)]$;
  \item
    $f(\bigcup_{i\in I}X_i) = \bigcup_{i\in I}(X_i)$
  \item
    при какви условия за $f$,
    $(\forall X\subseteq A)[X =  f^{-1}(f(X))]$.
  \item
    при какви условия за $f$,
    $(\forall Y \subseteq B)[Y = f(f^{-1}(Y))]$.
  \item
    при какви условия за $f$,
    $(\forall X,Y \subseteq A)[f(X)\backslash f(Y) = f(X\backslash Y)]$.
  \end{enumerate}
\end{problem}

\newpage

\subsection*{Обратна функция}

За всяка биективна функция $f:A\to B$, определяме нейната обратна функция $g:B \to  A$ като:
\[(\forall a \in A)(\forall b \in Ran(f))[g(b) = a\ \iff\ f(a) = b].\]
Обикновено означаваме $g$ като $f^{-1}$.

\subsection*{Композиция}

Нека са дадени функциите $f:A\to B$ и $g:C\to A$.
{\em Композиция} на $f$ и $g$ е функцията $f\circ g: C \to B$ определена като
\[f\circ g = \{\pair{c,b}\mid (\exists a\in A)[g(c) = a\ \wedge\ f(a) = b]\}.\]
\marginpar{Най-напред прилагаме $g$ и след това $f$}
Композицията на $f$ и $g$ може да се запише и така:
\[(\forall c\in C)[(f\circ g)(c) = f(g(c))]\]

\begin{example}
  Нека $f(x) = 2x+1$, $g(x) = x^2$. Тогава:
  \begin{itemize}
  \item 
    $(f\circ g)(x) = 2x^2 + 1$;
  \item
    $(g\circ f)(x) = (2x+1)^2$.
  \end{itemize}
\end{example}


% \begin{enumerate}[I)]
% \item
%   {\bf Образ}
  

% \item
%   {\bf Първообраз}


% \item
%   {\bf Обратна функция}

% \item 
%   {\bf Рестрикция}

%   Нека $X\subseteq A$. {\em Рестрикция} на $f$ до множеството $X$, наричаме
%   множеството: \[f\upharpoonright X = \{\langle{x,y}\rangle\mid f(x) = y\ \wedge\ x\in X\} =  f\cap X\times B.\]
% \item
%   {\bf Затваряне}
  
%   Нека в този случай $f:A\to A$ и нека $X\subseteq A$.
%   За всяко $n \geq 0$ определяме $X_0 = X$ и $X_{n+1} = X_n \cup f(X_n)$.
%   {\em Затваряне} на множеството $X$ относно функцията $f$ е множеството
%   \[f[X] = \bigcup_{n\in\Nat} X_n. \]
  
% \item
%   {\bf Композиция}


% \end{enumerate}




\begin{problem}
  Нека $f: A\to B$, $g: B\to C$ са функции.
  Вярно ли е, че:
  \begin{enumerate}[a)]
  \item 
    Ако $f$ не е инекция, то $g\circ f$ не е инекция?
  \item
    Ако $g$ не е инекция, то $g\circ f$ не е инекция?
  \item 
    Ако $f$ не е сюрекция, то $g\circ f$ не е сюрекция?
  \item
    Ако $g$ не е сюрекция, то $g\circ f$ не е сюрекция?
  \item
    \marginpar{Да}
    $f,g$ са инективни, то $g\circ f$ е инективна?
  \item
    \marginpar{Да}
    $f,g$ са сюрективни, то $g\circ f$ е сюрективна?
  \item
    \marginpar{Да}
    $f,g$ са биективни, то $g\circ f$ е биективна?
  \item
    $g\circ f$ е сюрективна,  то $f,g$ са сюрективни ?
  \item
    $g\circ f$ е инективна, то $f,g$ са инективни ?
  \end{enumerate}
\end{problem}

\begin{problem}
  Нека $f: A\to B$, $g: B\to C$ са {\em биективни} функции.
  Докажете, че
  \[(g\circ f)^{-1} = f^{-1}\circ g^{-1}.\]
\end{problem}

\begin{problem}
  Нека а дадена произволна функция $f:A \to B$.
  Проверете:
  \begin{enumerate}[a)]
  % \item
  %   $(\forall X,Y \subseteq A)[f(X)\cup f(Y) = f(X\cup Y)]$;
  % \item
  %   $f(\bigcup_{i\in I}X_i) = \bigcup_{i\in I}(X_i)$
  % \item
  %   при какви условия за $f$,
  %   $(\forall X,Y \subseteq A)[f(X\cap Y) = f(X)\cap f(Y)]$.
  % \item
  %   $f(\bigcap_{i\in I}A_i) \subseteq \bigcap_{i\in I}f(A_i)$
  % \item
  %   при какви условия за $f$,
  %   $(\forall X,Y \subseteq A)[f(X)\backslash f(Y) = f(X\backslash Y)]$.
  \item
    \marginpar{Опр. $(\forall x\in X)\ id_X(x) = x$}
    при какви условия за $f$, $f\circ f^{-1} = id_{B}$.
  \item
     при какви условия за $f$, $f^{-1}\circ f = id_{A}$.
   % \item
     % $Dom(f\circ g) = g^{-1}(Dom(f))$, където $g$ е функция.
  % \item
  %   $(\forall X,Y \subseteq B)[f^{-1}(X\cup Y) = f^{-1}(X)\cup f^{-1}(Y)$.
  % \item
  %   $(\forall X,Y \subseteq B)[f^{-1}(X\cap Y) = f^{-1}(X)\cap f^{-1}(Y)$.
  % \item
  %   $(\forall X,Y \subseteq B)[f^{-1}(X\backslash Y) = f^{-1}(X)\backslash f^{-1}(Y)]$.
  % \item
  %   при какви условия за $f$,
  %   $(\forall X\subseteq A)[X =  f^{-1}(f(X))]$.
  % \item
  %   при какви условия за $f$,
  %   $(\forall Y \subseteq B)[Y = f(f^{-1}(Y))]$.
  % \item
  %   $(\forall X\subseteq A)(\forall Y\subseteq B)[f(X)\cap Y = f(X\cap f^{-1}(Y))]$.
  % \item
  %   $(\forall X \subseteq A)(\forall Y \subseteq B)[f(X)\cap Y = \emptyset \iff X\cap f^{-1}(Y) = \emptyset]$.
  % \item
  %   $(\forall X \subseteq A)(\forall Y \subseteq B)[f(X)\subseteq Y \iff X\subseteq f^{-1}(Y)]$.
  \end{enumerate}
\end{problem}
\newpage
\begin{problem}%Л.М. 18 / 23
  Нека $f,g$ са функции. При какви условия :
  \begin{enumerate}
  \item
    $f^{-1}$ е функция?
  \item
    $f\circ g$ е инективна функция?
  \end{enumerate}
\end{problem}

% \begin{problem}
%   Дайте примери за функция $f:\mathbb{N}\rightarrow\Z$, която е:
%   \begin{enumerate}
%   \item
%     нито инективна, нито сюрективна;
%   \item
%     инективна, но не е сюрективна;
%   \item
%     сюрективна, но не е инективна;
%   \item
%     сюрективна и инективна.
% \end{enumerate}
% \end{problem}


\begin{problem}
  Нека е дадена релацията $R\subseteq A\times B$.
  Докажете, че $R$ е биективна функция тогава и само тогава, когато $R\circ R^{-1} = id_A$ и $R^{-1}\circ R = id_B$.
\end{problem}

\begin{problem}
  Нека $f$ е инективна функция от $A$ в $B$ и $g:\Ps(A) \rightarrow \Ps(B)$, дефинирана като 
  \[(\forall X \subseteq A)[g(X) = f(X)].\]
  Докажете, че $g$ е инективна.
\end{problem}

\begin{problem}
  Нека $f:A\rightarrow B$ и $g:B\rightarrow\Ps(A)$, дефинирана като 
  \[(\forall b \in B)[g(b) = \{x\in A\mid f(x) = b\}].\]
  Докажете, че ако $f$ е сюрективна, то $g$ е инективна.
  Вярна ли е обратната посока?
\end{problem}

\section{Монотонни функции}



% \cite{hein}

%%% Local Variables: 
%%% mode: latex
%%% TeX-master: "discrete-math"
%%% End: 


\section{Мощности}


Казваме, че едно множество $A$ е {\em изброимо безкрайно}\index{множество!изброимо}, ако съществува 
биекция от $A$ въху $\N$.

Казваме, че едно множество $A$ е {\em неизброимо безкрайно}\index{множество!неизброимо}, ако $A$ е безкрайно и {\bf не} съществува 
биекция от $A$ въху $\N$.

\begin{dfn}
  Две множества $A$ и $B$ са равномощни, $|A| = |B|$, ако съществува биекция от $A$ върху $B$.
\end{dfn}

\begin{thm}
  Нека $A$ е множество и $\Ps(A)$ е множеството от всички подмножества на $A$.
  Докажете, че $|A| < |\Ps(A)|$.
\end{thm}


\begin{thm}[Кантор-Шрьодер-Бернщайн]\index{Кантор-Шрьодер-Бернщайн}\label{KSB}
  Ако $|A|\leq|B|\ \&\ |B|\leq|A|$, то $|A| = |B|$.
\end{thm}

\begin{thm}\label{countable_union}
  Нека $A = \bigcup_{i\in I}A_i$, където множествата $A_i$ са изброими и индексното множество $I$ е изброимо.
  Тогава $A$ е изброимо множество.
\end{thm}
\begin{crl}
  Ако $A$ е крайна или изброимо безкрайна азбука, то $A^*$ е изброимо безкрайно.
\end{crl}

\begin{problem}
  Множеството $\Ps(\N)$ е равномощно с това на затворения интервал от реални числа $[0,1]$.
\end{problem}

\begin{problem}
  Докажете, че отвореният интервал от реални числа $(0,1)$ е неизброимо множество.
\end{problem}

\begin{problem}
  Докажете, че множеството $^\N\N = \{f\mid f:\N\to\N\}$ е неизброимо.
\end{problem}

\begin{problem}
  Нека $A$ е крайна азбука.
  Докажете, че :
  \begin{enumerate}[1)]
  \item
    $A^*$ е изброимо множество.
  \item
    $\Ps(A^*)$ е неизброимо безкрайно.
  \end{enumerate}
\end{problem}

\begin{problem}
  Докажете, че следните множества са изброимо безкрайни.
  \begin{enumerate}[1)]
  \item
    $B$ е множеството от тези думи над азбуката $\{0,1\}$, които не започват с $0$, с изключение на 
    думата $0$, т.е. $B = \{0, 1, 10, 11, 100, 101, 110, 111, \dots\}$.
  \item
    $F(\N)$ е множеството от всички крайни подмножества от естествени числа.
  \item
    $F(A^*)$ е множеството от всички крайни подмножества от $A^*$, за произволна азбука $A$.
  \end{enumerate}
\end{problem}


\begin{problem}
  Докажете, че :
  \begin{enumerate}
  \item
    aко $g:A\rightarrow B$ е сюрекция, то $|A|\geq |B|$;
  \item
    множествата $\Z,\N\times\N,\Q$ са изброимо безкрайни;
  \item
    съвкупността от всички полиноми на една променлива с цели коефициенти е изброимо безкрайно множество.
  \item
    Съвкупността $\mathscr{K}$ от всички реални алгебрични числа (т.е. корени на полиноми с цели коефициенти) е изброима.
  \end{enumerate}
\end{problem}

\begin{problem}
  Докажете, че множеството от изброимо безкрайни последователности от естествени числа $\N^{\N}$ е равномощно на $\R$.
\end{problem}

\begin{problem}
  Докажете, че следните множества са равномощни:
  \begin{enumerate}[a)]
  \item
    $\R$;
  \item
    интервалът от реални числа $(0,1)$;
\item
    интервалът от реални числа $[0,1]$;
  \item
    интервалът от реални числа $(a,b)$, за $a<b$.
  \item
    $\Ps(\N)$;
  \item
    $^\N\N = \{f\ \mid\ f:\N\to\N\}$.
  \item
    $^\N2 = \{f\ \mid\ f:\N\to\{0,1\}\}$
\end{enumerate}
\end{problem}

\begin{problem}
  Нека $|A_1| = |A_2|$ и $|B_1| = |B_2|$.
  Докажете, че $|\{f\mid f:A_1\to B_1\}| = |f\mid f:A_2\to B_2|$.
\end{problem}


\subsection*{Допускане на противното}

\begin{prb}
  За всяко $a \in \Z$, ако $a^2$ е четно, то $a$ е четно.
\end{prb}
\begin{proof}
  Твърдението може да се запише като
  \[(\forall a\in\Z)[a^2\mbox{ е четно}\ \rightarrow\ a\mbox{ е четно}].\]
  Да допуснем противното, т.е.
  \[(\exists a\in\Z)[a^2\mbox{ е четно}\ \wedge\ a\mbox{ не е четно}].\]
  Да вземем едно такова $a$.
  Тогава $a = 2k+1$, за някое $k \in \Z$
  и \[a^2 = (2k+1)^2 = 4k^2 + 4k + 1,\]
  което очевидно е нечетно число.
  Но ние допуснахме, че $a^2$ е четно.
  Така достигаме до противоречие, следователно нашето допускане е грешно 
  и 
  \[(\forall a\in\Z)[a^2\mbox{ е четно}\ \rightarrow\ a\mbox{ е четно}].\]
\end{proof}

\begin{prb}
  Докажете, че:
  \begin{enumerate}[a)]
  \item
    $\sqrt{2},\sqrt{3},\sqrt{6}$ не са рационални числа.
  \item
    $\sqrt{pq}$ и $\sqrt{\frac{p}{q}}$ не са рационални числа, където
    $p$ и $q$ са прости числа.
  \item
    $log_23$ не е рационално.
  \end{enumerate}
\end{prb}
\begin{proof}
  \begin{enumerate}[a)]
  \item
    Да допуснем, че $\sqrt{2}$ е рационално число. Тогава  съществуват $a,b \in \Z$, такива че:
    \[\sqrt{2} = \frac{a}{b}.\]
    Без ограничение, можем да приемем, че $a$ и $b$ са естествени числа,
    които нямат общи делители, т.е. не можем да съкратим дробта $\frac{a}{b}$.
    Получаваме, че \[2b^2 = a^2.\]
    Тогава $a^2$ е четно число и следователно $a$ е четно число, защото 
    произведение на две нечетни числа е нечетно число.
    Нека $a = 2k$. Получаваме, че
    \[2b^2 = 4k^2,\]
    от което следва, че
    \[b^2 = 2k^2.\]
    Това означава, че $b$ също е четно число, $b = 2n$.
    Следователно, $a$ и $b$ са четни числа и имат общ делител $2$,
    което е противоречие с нашето допускане. Така достигаме до
    противоречие.
    Накрая заключаваме, че $\sqrt{2}$ не е рационално число.
  \end{enumerate}
\end{proof}

\subsection*{Индукция върху $\Nat$}

Доказателството с индукция по $\Nat$ представлява следната схема:
\begin{prooftree}
  \AxiomC{$P(0)$}
  \AxiomC{$(\forall x\in\Nat)[P(x)\rightarrow P(x+1)]$}
  \BinaryInfC{$(\forall x\in\Nat) P(x)$}
\end{prooftree}

Това означава, че ако искаме да докажем, че свойството $P(x)$ е вярно за всяко $x\in\Nat$,
то трябва да докажем първо, че $P(0)$ и след това ако $P(x)$ вярно, то също така е вярно $P(x+1)$.

\begin{prb}
  Докажете, че:
  \begin{enumerate}[a)]
  \item
    $3^n$ е нечетно;
  \item
    $n < 2^n$;
  \item
    $2^n < n!$ за $n \geq 4$;
  \item
    \marginpar{$a\vert b\ \iff\ (\exists c\in\Nat)(b = c\cdot a)$}
    $3 \vert (n^3 - n)$;
  \item
    $6 \vert (n^3 + 11n)$;
  \item
    $9 \vert (2^{2n} + 15n - 1)$;
  \item
    $57 \vert (7^{n+2} + 8^{2n+1})$;
  \item
    \marginpar{$\Ps(A) = \{B\mid B\subseteq A\}$}
    \marginpar{$\abs{A}$ - брой елементи на $A$}
    \marginpar{$\abs{\Ps(A)}$ - брой на подмножествата на $A$}
    за всяко крайно множество $A$,
    ако $\abs{A} = n$, то $\abs{\Ps(A)} = 2^n$;
  \item
    $C\setminus \bigcup^n_{i=0}A_i = \bigcap^n_{i=0}(C\setminus A_i)$;
  \item
    $C\setminus \bigcap^n_{i=0}A_i = \bigcup^n_{i=0}(C\setminus A_i)$;
  \item
    ако $(\forall i \leq n)[A_i \subseteq B_i]$, то
    $\bigcap^n_{i=0}A_i \subseteq \bigcap^n_{i=0}B_i$;
  \item
    ако $(\forall i \leq n)[A_i \subseteq B_i]$, то
    $\bigcup^n_{i=0}A_i \subseteq \bigcup^n_{i=0}B_i$;
  \item
    $(\bigcap^n_{i=0} A_i)\cup B = \bigcap^n_{i=0} (A_i\cup B)$;
  \item
    $(\bigcup^n_{i=0} A_i)\cap B = \bigcup^n_{i=0} (A_i\cap B)$;
  \item
    $\bigcap^n_{i=0} (A_i\setminus B) = (\bigcap^n_{i=0} A_i) \setminus B$;
  \item
    $\neg (p_1\vee p_2\vee\dots\vee p_n) \iff (\neg p_1 \wedge \neg p_2 \wedge\dots\wedge \neg p_n)$;
  \item
    $\neg (p_1\wedge p_2\wedge\dots\wedge p_n) \iff (\neg p_1 \vee \neg p_2 \vee \dots \vee \neg p_n)$;
  \item
    $\sum^n_{i=0} 2^i = 2^{n+1} - 1$;
  \item
    $\sum^n_{i=0} ar^i = \frac{ar^{n+1}-a}{r-1}$ за $r \neq 1$;
  \item
    $\sum^n_{i=1}i = \frac{n(n+1)}{2}$;
  \item
    $\sum^n_{i=1}i^2 = \frac{n(n+1)(2n+1)}{6}$;
  \item
    $\sum^n_{i=1}i^3 = \frac{n^2(n+1)^2}{4}$;
  \item
    $\sum^n_{i=1}i^4 = \frac{n(n+1)(2n+1)(3n^2+3n-1)}{30}$;
  \item
    $\sum^n_{i=0}(2i+1)^2 = \frac{(n+1)(2n+1)(2n+3)}{3}$;
  \item
    $\sum^n_{i=1}\frac{1}{i(i+1)} = \frac{n}{n+1}$;
  \item
    $\sum^n_{i=0}(-\frac{1}{2})^i = \frac{2^{n+1}+(-1)^n}{3\cdot 2^n}$;

  \end{enumerate}
\end{prb}

% \begin{prb}
%   \begin{enumerate}
%   \item
%     за $n > 1$ е изпълнено
%     \[\frac{1}{n+1} + \frac{1}{n+2} + \cdots + \cdots \frac{1}{2n} > \frac{13}{24};\]
%   \item
%     за произволни реални числа $a_1,\dots,a_k \geq 0$,
%     \[\sqrt[k]{a_1\cdots a_k}\leq \frac{a_1+\cdots +a_k}{k};\]
%   \item
%     за произволни реални числа $a_1,\dots,a_n \geq 0$,
%     \[(1+a_1)\cdots (1+a_n) \geq (1+ \sqrt[n]{a_1\cdots a_n}).\]
%   \item
%     $\sum^n_{i=1}\frac{1}{\sqrt{i}} > 2(\sqrt{n+1} - 1)$;
%   \end{enumerate}
% \end{prb}
% \begin{proof}
%   \begin{enumerate}[]
%   \item
%     Първо, доказва се с индукция, че твърдението е вярно за $n = 2k$.
%     Очевидно е вярно за $n = 2$.
%     Да допуснем, че $\sqrt[k]{\prod^{k}_{i=1} a_i}\leq \frac{\sum^{k}_{i=1} a_i}{k}$.
%     Ще докажем, че $\sqrt[2k]{\prod^{2k}_{i=1} a_i}\leq \frac{\sum^{2k}_{i=1} a_i}{2k}$.
%     \[
%     \begin{array}{lll}
%       \sqrt[2k]{\prod^{2k}_{i=1} a_i} & = & \sqrt[2]{\sqrt[k]{\prod^{k}_{i=1} a_i}\sqrt[k]{\prod^{2k}_{i=k+1} a_i}}\\
%       & \leq & \frac{\sqrt[k]{\prod^{k}_{i=1} a_i} + \sqrt[k]{\prod^{2k}_{i=k+1} a_i}}{2} \\
%       & \leq & \frac{\frac{\sum^{k}_{i=1} a_i}{k} + \frac{\sum^{2k}_{i=k+1} a_i}{k}}{2}\\
%       & = & \frac{\sum^{2k}_{i=1} a_i}{2k}\\
%     \end{array}
%     \]
    
%     Сега, ще докажем, че ако твърдението е вярно за $n = k$, то е вярно и за $n = k-1$.
%     Нека \[\sqrt[k]{\prod^{k}_{i=1}a_i} \leq \frac{\sum^{k}_{i=1} a_i}{k}.\]
%     Това е вярно за произволни $a_1,\dots,a_k$.
%     Нека да изберем $a_k$, така че
%     \[\frac{\sum^{k}_{i=1} a_i}{k} = \frac{\sum^{k-1}_{i=1} a_i}{k-1},\] т.е.
%     \[a_k = \frac{\sum^{k-1}_{i=1} a_i}{k-1}.\]
%     Получаваме, че:
%     \[
%     \begin{array}{rll}
%       \sqrt[k]{\prod^{k}_{i=1}a_i} & \leq & \frac{\sum^{k}_{i=1} a_i}{k}\\
%       \frac{(\prod^{k-1}_{i=1}a_i)(\sum^{k-1}_{i=1}a_i)}{k-1} & \leq & (\frac{\sum^{k-1}_{i=1}a_i}{k-1})^{k} \\
%       \prod^{k-1}_{i=1}a_i & \leq & (\frac{\sum^{k-1}_{i=1}a_i}{k-1})^{k-1}\\
%       \sqrt[k-1]{\prod^{k-1}_{i=1}a_i} & \leq & \frac{\sum^{k-1}_{i=1}a_i}{k-1}\\
%     \end{array}
%     \]
%   \end{enumerate}
% \end{proof}

% \begin{prb}
%   \marginpar{Нютонов бином}
%   Нека положим 
%   \[\binom{n}{m} = \frac{n!}{(n-i)!i!}\]
%   Проверете:
%   \begin{enumerate}[a)]
%   \item 
%     $2^n = \sum^n_{i=0}\binom{n}{i}$;
%   \item
%     $(x+y)^n = \sum^n_{i=0} \binom{n}{i}x^{n-i}y^{i}$;
%   \item
%     $\binom{n+1}{m+1} = \binom{n}{m} + \binom{n}{m+1}$;
%   \end{enumerate}
% \end{prb}

\begin{prb}
  \marginpar{Наричат се хармонични числа}
  Нека да положим
  $H_n = \sum^n_{i=1}\frac{1}{i}$.
  Проверете:
  \begin{enumerate}[a)]
  \item
    $\sum^n_{i=1}H_i = (n+1)H_n - n$.
  \item
    $H_{2^k} \geq 1+ k/2$, за всяко $k \geq 0$.
  \end{enumerate}
\end{prb}

\subsection*{Пълна индукция върху $\Nat$}

Доказателство с пълна индукция по $\Nat$ за свойството $P$ представлява следната схема:
\begin{prooftree}
  \AxiomC{$(\forall x\in\Nat)[(\forall y\in \Nat)[y < x\ \rightarrow P(y)]\rightarrow P(x)]$}
  \UnaryInfC{$(\forall x\in\Nat) P(x)$}
\end{prooftree}
Нека да проверим принципа за пълна индукция.
Да допуснем, че принципът не е верен, т.е. за някое свойство $P$ е изпълнено, че
\[(\forall x\in\Nat)[(\forall y\in \Nat)[y < x\ \rightarrow P(y)]\rightarrow P(x)]\ \wedge\ (\exists x\in\Nat) \neg P(x).\]
Да вземем най-малкия елемент $x_0$, за който $\neg P(x_0)$.
Тогава \[(\forall y\in \Nat)[y < x_0\ \rightarrow P(y)]\]
и следователно:
\begin{prooftree}
  \AxiomC{$(\forall y\in \Nat)[y < x_0\ \rightarrow P(y)]$}
  \AxiomC{$(\forall x\in\Nat)[(\forall y\in \Nat)[y < x\ \rightarrow P(y)]\rightarrow P(x)]$}
  \UnaryInfC{$(\forall y\in \Nat)[y < x_0\ \rightarrow P(y)]\rightarrow P(x_0)$}
  \BinaryInfC{$P(x_0)$}
\end{prooftree}
Така достигаме до противоречие, защото получаваме, че $P(x_0)\wedge \neg P(x_0)$.
% \begin{prop}
%   Двете форми на индукция са еквивалентни.
% \end{prop}
% \begin{proof}
%   \begin{enumerate}
%   \item 
%     Нека имаме схемата за ``обикновена'' индукция.
%     Ще докажем, че тогава имаме схемата за пълна индукция.
%     Нека \[(\forall x\in\Nat)[(\forall y\in \Nat)[y < x\ \rightarrow P(y)]\rightarrow P(x)].\]
%     Ще докажем с ``обикновена'' индукция, че $(\forall x\in\Nat)P(x)$.
    
%     Нека да положим $Q(x) = (\forall y\in\Nat)[y<x \rightarrow P(y)]$.
%     Очевидно е, че $Q(0)$ е изпълнено.
%     Освен това, 
%     \begin{prooftree}
%       \AxiomC{$(\forall x\in\Nat)[(\forall y\in \Nat)[y < x\ \rightarrow P(y)]\rightarrow P(x)]$}
%       \UnaryInfC{$(\forall x\in\Nat)[Q(x)\rightarrow P(x)]$}
%       \UnaryInfC{$(\forall x\in\Nat)[Q(x)\rightarrow Q(x)\wedge P(x)]$}
%       \AxiomC{$(\forall x\in\Nat)[Q(x)\wedge P(x)\ \rightarrow\ Q(x+1)]$}
%       \BinaryInfC{$(\forall x\in\Nat)[Q(x)\rightarrow Q(x+1)]$}
%     \end{prooftree}
%     Получаваме, че 
%     \begin{prooftree}
%       \AxiomC{$Q(0)$}
%       \AxiomC{$(\forall x\in\Nat)[Q(x)\rightarrow Q(x+1)]$}
%       \BinaryInfC{$(\forall x\in\Nat)Q(x)$}
%       \AxiomC{$(\forall x\in\Nat)Q(x)\ \rightarrow\ (\forall x\in\Nat)P(x)$}
%       \BinaryInfC{$(\forall x\in\Nat)P(x)$}
%     \end{prooftree}
%   \item
%     Нека сега имаме схемата за пълна индукция.
%     Ще докажем, че имаме и схемата за ``обикновена'' индукция.
%     За тази цел, нека имаме, че $P(0)$ и $(\forall x\in\Nat)[P(x) \rightarrow P(x+1)]$.
%     Достатъчно е да докажем, че
%     \[(\forall x\in\Nat)[(\forall y\in \Nat)[y < x \rightarrow P(y)] \rightarrow P(x)],\]
%     защото тогава ще приложим пълна индукция и ще получим, че \[(\forall x\in\Nat)P(x).\]
    
%     Да допуснем противното, т.е.
%     \[(\exists x\in\Nat)[(\forall y\in \Nat)[y < x \rightarrow P(y)] \wedge \neg P(x)].\]
%     Да разгледаме едно такова $x_0$, за което
%     \[(\forall y\in \Nat)[y < x_0 \rightarrow P(y)] \wedge \neg P(x_0).\]
%     \begin{itemize}
%     \item 
%       Ако $x_0 = 0$, то очевидно е изпълнено, че $(\forall y\in\Nat)[y < 0 \rightarrow P(y)]$.
%       Тогава $\neg P(0)$, което е противоречие.
%     \item
%       Ако $x_0 > 0$, тогава от
%       $(\forall y\in\Nat)[y < x_0 \rightarrow P(y)]\ \rightarrow\ P(x_0-1)$
%       следва, че $P(x_0-1)$.
%       Тогава 
%       \begin{prooftree}
%         \AxiomC{$(\forall x)[P(x)\rightarrow P(x+1)]$}
%         \AxiomC{$P(x_0-1)$}
%         \BinaryInfC{$P(x_0)$}
%       \end{prooftree}
%       Отново достигаме до противоречие.
%     \end{itemize}
%     Следователно нашето допускане е невярно и
%     \[(\forall x\in\Nat)[(\forall y\in \Nat)[y < x \rightarrow P(y)] \rightarrow P(x)].\]
%   \end{enumerate}
% \end{proof}

\begin{prb}
  Докажете, че за всяко $x,y\in\Nat$
  \[f(x,y) = x^y,\]
  където
  \begin{align*}
    f(x,y) = 
    \begin{cases}
      1, & x\neq 0\ \wedge\ y = 0\\
      f(x,y-1) * x, & x\neq 0\ \wedge y\mbox{ е нечетно}\\
      f(x,y/2) * f(x,y/2), & x\neq 0\ \wedge y\mbox{ е четно}
    \end{cases}
  \end{align*}
\end{prb}

\begin{prb}
  Всяко естествено число $n \geq 2$ може да се запише като произведение на прости числа.
\end{prb}
\begin{proof}
  \begin{enumerate}[a)]
  \item 
    За $n = 2$  е ясно.
  \item
    Ако $n+1$ е просто число, то всичко е ясно.
    Ако $n+1$ е съставно, то \[n + 1 = n_1\cdot n_2.\]
    Тогава $n_1 = p^{n_1}_1\cdots p^{n_k}_k$ и $n_2 = q^{m_1}_1\cdots q^{m_r}_r$,
    където $p_1,\dots,p_k$ и $q_1,\dots,q_r$ са прости числа.
    Тогава е ясно, че $n$ също е произведение на прости числа.
  \end{enumerate}
\end{proof}

\begin{prb}
  Нека фунцкията $f:\Nat\to\Nat$ е определена като
  \begin{align*}
    f(x) = x-10, & x > 100\\
    f(x) = f(f(x+11)), & x < 100.
  \end{align*}
  Докажете, че $(\forall x \leq 100)[f(x) = 91]$.
\end{prb}

\subsection*{Индукция върху $\Nat\times\Nat$}

\begin{dfn}
  Определяме лексикографската наредба $\prec$ върху $\Nat\times\Nat$ като
  \[\pair{x,y} \prec \pair{x^\prime,y^\prime}\ \iff\ x < x^\prime \vee (x = x^\prime\ \wedge\ y < y^\prime).\]
  Наричаме двойката $\pair{x_0,y_0}$ {\em минимална} за множеството $A \subseteq \Nat\times\Nat$, ако
  \[\pair{x_0,y_0}\in A\ \wedge\ (\forall \pair{x,y}\in A)[\pair{x,y}\not\prec\pair{x_0,y_0}].\]
\end{dfn}

\begin{prop}
  Всяко непразно подмножество $A\subseteq \Nat\times \Nat$ притежава поне един {\em минимален} елемент.
\end{prop}

\begin{prop}
  Не съществуват безкрайни строго намаляващи редици относно $\succ$ в $\Nat\times\Nat$, т.е.
  не съществува 
  \[\pair{x_0,y_0} \succ \pair{x_1,y_1}\succ \pair{x_2,y_2} \succ \dots \succ \pair{x_n,y_n}\succ\dots\]
\end{prop}

\begin{dfn}
  Доказателството с индукция върху $\Nat\times\Nat$ представлява следната схема:
  \begin{prooftree}
    \AxiomC{$(\forall\pair{x,y})[(\forall \pair{x^\prime,y^\prime})[\pair{x^\prime,y^\prime} \prec \pair{x,y}\ \rightarrow\ P(x^\prime,y^\prime)]\ \rightarrow P(x,y)]$}
    \UnaryInfC{$\forall \pair{x,y} P(x,y)$}
  \end{prooftree}  
\end{dfn}

Да проверим схемата.
Да допуснем, че тя не е вярна, т.е. за някое свойство $P$ е изпълнено, че
\[(\forall\pair{x,y})[(\forall \pair{x^\prime,y^\prime})[\pair{x^\prime,y^\prime} \prec \pair{x,y}\ \rightarrow\ P(x^\prime,y^\prime)]\ \rightarrow P(x,y)],\]
но \[\exists \pair{x,y} \neg P(x,y),\]
т.е. съществува $\pair{x,y} \in \Nat\times\Nat$, за което $\neg P(x,y)$.
Да разгледаме
\[A = \{\pair{x,y}\in \Nat\times\Nat\mid \neg P(x,y)\}.\]
Щом $A$ е непразно, то $A$ има минимален елемент $\pair{x_0,y_0}$.
Тогава
\[(\forall \pair{x^\prime,y^\prime})[\pair{x^\prime,y^\prime} \prec\pair{x_0,y_0}\ \rightarrow P(x^\prime,y^\prime)].\]
Но ние имаме, че 
\[(\forall \pair{x^\prime,y^\prime})[\pair{x^\prime,y^\prime} \prec \pair{x_0,y_0}\ \rightarrow\ P(x^\prime,y^\prime)]\ \rightarrow P(x_0,y_0).\]
Това означава, че $P(x_0,y_0)$, което е противоречие.

\begin{remark}
  За да докажем едно свойство $P$ с индукция по лексикографската наредба върху $\Nat\times\Nat$,
  първо доказваме $P$ за минималната двойка $\pair{0,0}$.
  След това доказваме, че ако $P$ е вярно за всички двойки $\pair{x^\prime,y^\prime}\prec \pair{x,y}$,
  то $P$ е вярно и за $\pair{x,y}$.
\end{remark}

\begin{prb}
  Докажете,  че $f(x,y) = \abs{x-y}$, където
  \begin{align*}
    f(x,y) = 
    \begin{cases}
      y, & x = 0\\
      x, & y = 0\\
      f(x-1,y-1), & \mbox{ иначе}
    \end{cases}
  \end{align*}
\end{prb}
\begin{proof}
  Индукция по $(\Nat^2,\prec)$, където $\prec$ е лексикографската наредба.
  Имаме един минимален елемент $(0,0)$.
  \[f(0,0) = 0 = \abs{0 - 0}.\]
  Да допуснем, че за всяко $(u,v) \prec (x,y)$, 
  \[f(u,v) = \abs{u - v}.\]
  Тогава ако $x > 0, y = 0$, то
  \[f(x,0) = x = \abs{x - 0}.\]
  Ако $x> 0, y > 0$, то
  \[f(x,y) = f(x-1,y-1) = \abs{x-1-y+1} =\abs{x-y}.\]
\end{proof}

\begin{prb}
  Докажете, че $f(x,y) = \mbox{НОД}(x,y)$, където
  за $x,y \in \Nat$,
  \begin{align*}
    f(x,y) = 
    \begin{cases}
      f(x-y,y), & x > y\\
      f(y,x), & x < y\\
      x, & x = y.
    \end{cases}
  \end{align*}
\end{prb}

\begin{prb}
  \marginpar{Числа на Фибоначи}
  Да определим следната редица:
  \[F_0 = 0,F_1 = 1,\dots,F_{n+2} = F_{n} + F_{n+1}.\]
  Проверете:
  \begin{enumerate}[a)]
  \item
    $\sum^n_{i=0} F^2_i = F_{n}F_{n+1}$;
  \item
    $\sum^n_{i=1} F_{2i-1} = F_{2n}$;
  \item
    $\sum^{2n}_{i=1}F_{i-1}F_{i} = F^2_{2n}$;
  \item
    единствено членовете от вида $F_{3n}$ са четни;
  \item
    за $n > 0$, $F_{n+1}F_{n-1} - F^2_n = (-1)^n$;
  \item
    $F_{m+n} = F_{m-1}\cdot F_{n} + F_m \cdot F_{n-1}$;
  \item
    ако $m\vert n$, то $F_m \vert F_n$.
  \item
    \marginpar{Използвайте, че $\phi^2 = \phi + 1$}
    ако $n\geq 3$, то $F_n > \phi^{n-2}$,
    където $\phi = \frac{1+\sqrt{5}}{2}$.
  \end{enumerate}
\end{prb}

\begin{prb}
  \marginpar{Нютонов бином}
  Нека положим $\binom{n}{m} = \frac{n!}{(n-i)!i!}$.
  Проверете:
  \begin{enumerate}[a)]
  \item 
    $2^n = \sum^n_{i=0}\binom{n}{i}$;
  \item
    $(x+y)^n = \sum^n_{i=0} \binom{n}{i}x^{n-i}y^{i}$;
  \item
    $\binom{n+1}{m+1} = \binom{n}{m} + \binom{n}{m+1}$;
  \end{enumerate}
\end{prb}

\begin{prb}
  Докажете, че за всеки $m,n\in\Nat$
  съществуват $p,q\in\Z$, такива че
  \[p\cdot m +  q\cdot n = \mbox{НОД}(m,n).\]
\end{prb}

\newpage
\subsection*{Фундирани множества}
\marginpar{Англ. well-founded sets}

Понякога се налага да правим индукция по по-сложни множества от това на естествените
числа.

\begin{dfn}
  Нека е дадена двойката $(A,R)$, където $A$ е множество, а $R\subseteq A^2$.
  Казваме, че $A$ е фундирано множество относно $R$, ако 
  \begin{itemize}
  \item 
    \marginpar{$R$ задава строга частична наредба върху $A$}
    $R$ е антирефлексивна, транзитивна, асиметрична.
  \item
    ако всяко непразно подмножество $X\subseteq A$ притежава поне един {\em минимален} елемент, т.е.
    \[(\forall X\subseteq A)[X\neq\emptyset \rightarrow (\exists m\in X)\neg(\exists y\in X)[\pair{y,m} \in R]].\]
  \end{itemize}
\end{dfn}

Обърнете внимание, че минималният елемент може да не е уникален.
\begin{example}
  Нека да определим $\prec$ върху $\N$ като
  \[\pair{x,y}\prec\pair{x^\prime,y^\prime}\ \iff\ x < y\ \wedge\ x^\prime < y^\prime.\]
  Тогава например $X = \{\pair{1,2},\pair{2,1},\pair{2,2},\pair{2,3}\}$
  има два минимални елемента - $\pair{1,2}$ и $\pair{2,1}$.
\end{example}


\begin{prop}
  Нека $\prec$ е строга частична наредба върху $A$.
  Следните твърдения са  еквивалентни:
  \begin{enumerate}[a)]
  \item
    ако всяко непразно подмножество $X\subseteq A$ притежава поне един {\em минимален} елемент, т.е.
    \[(\forall X\subseteq A)[X\neq\emptyset \rightarrow (\exists x\in X)\neg(\exists y\in X)[y \prec x]];\]
  \item
    не съществуват безкрайни редици от вида
    \[x_0 \succ x_1 \succ x_2 \succ \cdots \succ x_n \succ \cdots\]
  \end{enumerate}
\end{prop}
\begin{proof}
  \begin{enumerate}
  \item[а)$\ \to\ $б)]
    Да допуснем, че съществува безкрайно-намаляваща редица
    \[x_0 \succ x_1 \succ x_2 \succ \cdots \succ x_n \succ \cdots\]
    Нека $X = \{x_i\mid i \in \Nat\}$.
    Тогава лесно се вижда, че в $X$ няма минимален елемент, което е противоречие.
  \item[б)$\ \to\ $а)]
    Да допуснем, че съществува непразно множество $X \subseteq A$, което не притежава минимален елемент, т.е.
    \[(\forall x\in X)(\exists y\in X)[y \prec x].\]
    Ще построим безкрайно-намаляваща редица относно $\prec$.
    Да вземем произволен $x_0 \in X$. 
    Знаем, че съществува $y \in X, x_0 \prec y$.
    Нека да изберем едно такова $y\in X$ и да означим $x_1 = y$.
    По този начим можем да построим
    \[x_0 \prec x_1 \prec x_2 \prec \cdots \]
  \end{enumerate}
\end{proof}

\begin{prop}
  Нека $(A_1,\prec_1)$ и $(A_2,\prec_2)$ са фундирани.
  Тогава \[(A_1\times A_2, \prec)\]
  е фундирано множество, където
  \[\pair{a_1,a_2}\prec \pair{a^\prime_1,a^\prime_2}\ \iff\ a_1\prec_1 a^\prime_1\ \vee\ (a_1 = a^\prime_1\ \wedge\ a_2\prec_2 a^\prime_2)\]
\end{prop}
\begin{proof}
  Да допуснем, че съществува
  безкрайно намаляваща редица относно $\prec$:
  \[(x_0,y_0)\succ(x_1,y_1) \succ \cdots \succ (x_n,y_n)\succ\cdots\]
  Да разгледаме редицата само от първите компоненти :
  \[x_0 \succeq x_1 \succeq \cdots \succeq x_n \succeq \cdots\]
  Това означава, че съществува число $n_1$, такова че 
  \[(\forall k \geq n_1)[x_{n_1} = x_k].\]
  В противен случай ще получим безкрайно намаляваща редица, което ще бъде
  противоречие с фундираността на $A_1$.
  Аналогично, съществува $n_2$, такова че
  \[(\forall k \geq n_2)[y_{n_2} = y_k].\]
  В противен случай ще получим безкрайно намаляваща редица, което ще бъде
  противоречие с фундираността на $A_2$.
  Нека \[n = \max(n_1,n_2).\]
  Тогава 
  \[(\forall k\geq n)[(x_n,y_n) = (x_k,y_k)].\]
  Така достигаме до противоречие с 
  \[(\forall k \geq n)[(x_n,y_n) \succ (x_k,y_k)].\]
  Следователно $\prec$ задава фундирана наредба върху $A_1\times A_2$.
\end{proof}



\subsection*{Индукция по фундирани наредби}

Доказателството с индукция по фундираното множество $(A,\prec)$ представлява следната схема:
\begin{prooftree}
  \AxiomC{$(\forall x \in A)[(\forall y\in A)[y \prec x\ \rightarrow\ P(y)]\ \rightarrow P(x)]$}
  \UnaryInfC{$(\forall x \in A) P(x)$}
\end{prooftree}
Ако допуснем, че съществува $x \in A$, за което $\neg P(x)$, то да разгледаме
\[X = \{x\in A\mid \neg P(x)\}.\]
Щом това множество е непразно, то $X$ има поне един минимален елемент $x_0$.
Тогава
\[(\forall y\in A)[y \prec x_0\ \rightarrow P(y)].\]
Но това означава, че $P(x_0)$, което е противоречие.

\begin{prb}
  Докажете,  че $f(x,y) = \abs{x-y}$, където
  \begin{align*}
    f(x,y) = 
    \begin{cases}
      y, & x = 0\\
      x, & y = 0\\
      f(x-1,y-1), & \mbox{ иначе}
    \end{cases}
  \end{align*}
\end{prb}
\begin{proof}
  Индукция по $(\Nat^2,\prec)$, където $\prec$ е лексикографската наредба.
  Имаме един минимален елемент $(0,0)$.
  \[f(0,0) = 0 = \abs{0 - 0}.\]
  Да допуснем, че за всяко $(u,v) \prec (x,y)$, 
  \[f(u,v) = \abs{u - v}.\]
  Тогава ако $x > 0, y = 0$, то
  \[f(x,0) = x = \abs{x - 0}.\]
  Ако $x> 0, y > 0$, то
  \[f(x,y) = f(x-1,y-1) = \abs{x-1-y+1} =\abs{x-y}.\]
\end{proof}

\begin{prb}
  Докажете, че $f(x,y) = \mbox{НОД}(x,y)$, където
  за $x,y \in \Nat$,
  \begin{align*}
    f(x,y) = 
    \begin{cases}
      f(x-y,y), & x > y\\
      f(y,x), & x < y\\
      x, & x = y.
    \end{cases}
  \end{align*}
\end{prb}

\begin{prb}
  Докажете, че $f(x,y) = \binom{x}{y}$, където
  за $x \geq y, x,y\in\Nat$,
  \begin{align*}
    f(x,y) = 
    \begin{cases}
      1, & x = 0\ \vee\ y = 0\ \vee\ x = y\\
      f(x-1,y) + f(x-1,y-1), & \mbox{ иначе}
    \end{cases}
  \end{align*}
\end{prb}

% \begin{prb}
%   \begin{align*}
%     f(x,y) = 
%     \begin{cases}
%       y+1, & x = 0\\
%       f(x-1,1), & x > 0\ \wedge\ y = 0\\
%       f(x-1,f(x,y-1), & x > 0\ \wedge\ y > 0.
%     \end{cases}
%   \end{align*}
%   Докажете, че $f$ е тотална функция.

% \end{prb}

\begin{dfn}
  Да определим следната редица:
  \[F_0 = 0,F_1 = 1,\dots,F_{n+2} = F_{n} + F_{n+1}.\]
  Числата $F_n$ се наричат {\em числа на Фибоначи}.
\end{dfn}

\begin{prb}
  Докажете, че:
  \begin{enumerate}[a)]
  \item
    единствено членовете от вида $F_{3n}$ са четни;
  \item
    за $n > 0$, $F_{n+1}F_{n-1} - F^2_n = (-1)^n$;
  \item
    $F_{m+n} = F_{m-1}\cdot F_{n} + F_m \cdot F_{n-1}$;
  \item
    ако $m\vert n$, то $F_m \vert F_n$.
  \item
    ако $n\geq 3$, то $F_n > \phi^{n-2}$,
    където $\phi = \frac{1+\sqrt{5}}{2}$.
  \end{enumerate}
\end{prb}


\begin{prb}
  Докажете, че за всеки $m,n\in\Nat$
  съществуват $p,q\in\Z$, такива че
  \[p\cdot m +  q\cdot n = \mbox{НОД}(m,n).\]
\end{prb}


%%% Local Variables: 
%%% mode: latex
%%% TeX-master: "discrete-math"
%%% End: 

\chapter{Комбинаторика}

\section{Основни понятия}

\begin{description}
\item[(0+R+)]
  {\bf Конфигурации с подредба и с повторение.}
  Също така се наричат пермутации с повторение.
  Това е броят $P_r(n,k)$ на всички думи с дължина $k$ над $n$-елементна азбука.
  \[P_r(n,k) = n^k\]
  С тази формула можем да намираме всички $k$-буквени думи над азбука с $n$ букви.
  Например, всички 4-буквени думи над азбуката $\{a,b,c\}$ са $3^4$ на брой.
  Иначе казано, това са всички начини да изберем по една буква от всяка урна:
  \[
  \left(\begin{array}{c}
      a\\
      b\\
      c\\
      \end{array}
    \right)
  \left(\begin{array}{c}
      a\\
      b\\
      c\\
      \end{array}
    \right)
  \left(\begin{array}{c}
      a\\
      b\\
      c\\
      \end{array}
    \right)
  \left(\begin{array}{c}
      a\\
      b\\
      c\\
      \end{array}
    \right)
  \]
  и ги подреждаме в редица.
\item[(0+R--)]
  \marginpar{Тук $k \leq n$.}
  {\bf Конфигурации с подредба, но без повторение.}
  Съща така се наричат пермутации.
  Това е броят $P(n,k)$ на думите с дължина $k$ над азбука с $n$ букви, като нямаме повторения на буквите.
  \[P(n,k) = n(n-1)\cdots(n-k+1) = \frac{n!}{(n-k)!}.\]
  Например, всички 3-буквени думи {\em без повторения} над азбуката $\Sigma = \{a,b,c,d\}$
  са $4!3!2!$ на брой.
  Как можем да генерираме всички такива 3-буквени думи?
  Започваме с 3 пълни урни:
  \[
  \left(\begin{array}{c}
      a\\
      b\\
      c\\
      d
      \end{array}
    \right)
  \left(\begin{array}{c}
      a\\
      b\\
      c\\
      d
      \end{array}
    \right)
  \left(\begin{array}{c}
      a\\
      b\\
      c\\
      d
      \end{array}
    \right)
  \]
  От първата урна избираме произволен елемент измежду 4-те букви. Например $b$.
  Това ще бъди първият символ на нашата дума. Понеже той не може да се повтаря,
  ние премахваме $b$ от другите урни. Оставаме с втора и трета урна:
  \[
  \left(\begin{array}{c}
      a\\
      c\\
      d
      \end{array}
    \right)
  \left(\begin{array}{c}
      a\\
      c\\
      d
      \end{array}
    \right).
    \]
    От втората урна избираме произволен елемент измежду 3-те останали букви.
    Нека да изберем от втората урна $a$.
    Това означава, че нашата дума ще започва с $ba$.
    Отново, понеже не искаме $a$ да се повтаря, премахваме $a$ от третата урна. Оставаме само с третата урна:
      \[
      \left(\begin{array}{c}
          c\\
          d
        \end{array}
      \right).
    \]
    За третата буква от нашата дума избираме измежду $c$ и $d$.
    Нека да изберем $d$.
    Така генерирахме думата $bad$.
  \item[(0--R--)]
  {\bf Конфигурации без подредба и без повторение.}
  \marginpar{Тук също $k \leq n$.}
  \marginpar{Също така се наричат комбинации}
  Това е броят $C(n,k)$ на $k$-елементните подмножества (т.е. елементите {\em не са подредени}) на едно $n$-елементно множество.
  Имаме следната връзка с пермутации без повторение:
  \[P(n,k) = C(n,k)\cdot P(k,k),\] 
  т.е. за да получим всички думи с дължина $k$ {\em без повторения на буквите},
  можем първо да изберем едно множество от $k$ букви и след това да ги подредим тези $k$ на брой букви в една редица.
  Следователно,
  \marginpar{$\binom{n}{k}$ - чете се $n$ над $k$}
  \[C(n,k) =  \frac{P(n,k)}{P(k,k)} = \frac{n!}{(n-k)!k!} = \binom{n}{k}.\]
  Например, всички $3$ елементни подмножества на $\{1,2,3,4\}$ са
  \[\{1,2,3\},\{1,2,4\},\{1,3,4\},\{2,3,4\}.\]

  Като друг пример, броят на всички комбинации от правилно попълнени фишове в тото 6 от 49 са $\binom{49}{6}$.
  Всеки правилно попълнен фиш еднозначно се определя като множество от 6 елемента измежду числата $\{1,2,\dots,49\}$,
  защото не е важен реда на попълване на числата.
\item[(0-- R+)]
  {\bf Комбинации без подредба и с повторение.}
  Мултимножество е съвкупност от обекти, в които позволяваме повторение на елементи.
  Например, $\{3,1,1,2\}$ е мултимножество и $\{3,1,1,2\} = \{2,1,3,1\}$,
  но $\{3,1,2\} \neq \{3,1,1,2\}$.
  Броят на $n$-елементните мулти-подмножества на едно $k$-елементно множество е:
  \[C(n+k-1,k-1) = \binom{n+k-1}{k-1}.\]
  Нека да видим как можем да достигнем до тази формула като намерим всички 4-елементни мулти-подмножества
  на $\{a,b,c\}$. Ще видим, че на всяко такова мулти-множество можем да съпоставим редица от 6 кутии,
  като в две от тези кутии са отбелязани с $\star$, а в другите кутии са буквите от азбуката, избрани по следния начин - 
  в кутиите до първата $\star$ поставяме $a$; в кутиите между двете $\star$ поставяме $b$; и в кутиите след втората $\star$
  поставяме $c$.
  Например, на следната редица от кутии:
  
  \begin{tabular}{|l|l|l|l|l|l|}
    \hline
    a & a & $\star$ & $\star$ & c & c \\
    \hline
  \end{tabular}  
  съответства мулти-множеството $\{a,a,c,c\}$, а на редицата от кутии:

  \begin{tabular}{|l|l|l|l|l|l|}
    \hline
    a & $\star$ & b & $\star$ & c & c \\
    \hline
  \end{tabular}
  съответства мулти-множеството $\{a,b,c,c\}$
  
  Всяка такава подредба се определя еднозначно от позициите на двете $\star$.
  Следователно, всички мулти-множества са $\binom{4+3-1}{3-1} = \binom{6}{2}$.
\end{description}



\begin{problem}
  Отговорете на следните въпроси:
  \begin{enumerate}[a)]
  \item
    \marginpar{Отг. $2^8$}
    Колко битови низове с дължина един байт има ?
  \item
    Колко са всички подмножества на множеството $A$ с $8$ елемента ?
  \item 
    \marginpar{Отг. $2^5$}
    Колко битови низове с дължина един байт започват с 1 завършват с 00 ?
  \item
    \marginpar{Отг. $62! - 52!$}
    Всеки потребител на една компютърна система има парола, която е дълга между 6 и 8 символа.
    Всеки символ е малка или голяма буква, или цифра.
    Всяка парола трябва да съдържа поне една цифра.
    Колко такива пароли има?
  \item
    \marginpar{Отг. $4!$}
    По колко начина можем да подредим елементите $\{a,b,c,d\}$ ?
  \item 
    \marginpar{Отг. $5!$}
    Колко думи може да се образуват от буквите в $ABCDEFG$, които съдържат $ABC$.
  \item
    \marginpar{Отг. $\binom{11}{1}\binom{10}{4}\binom{6}{4}\binom{2}{2}$}
    Колко различни думи могат да се образуват като разместим буквите на думата $MISSISSIPPI$?
  \item
    Колко различни думи могат да се образуват като разместим буквите на думата $TENNESSEE$?
  \item
    Колко различни думи могат да се образуват като разместим буквите на думата $SUCCESS$?
  \item
    Колко различни думи могат да се образуват като разместим буквите на думата АБРАКАДАБРА?
  \item
    Колко различни думи могат да се образуват като разместим буквите на думата ПЕРПЕРИКОН?
  \item
    \marginpar{Отг. $10\cdot 9\cdot 8$}
    В състезание участват 10 отбора. 
    По колко начина могат да се разпределят златните, сребърните и бронзовите медали?
  \item
    \marginpar{Не искаме числата да започват с нула. Отг. $5! - 4!$}
    Колко различни петцифрени числа могат да се образуват чрез разместване на цифрите от 0,1,2,3,4?
  \item
    \marginpar{Отг. $\binom{8}{1}\binom{7}{3}\binom{4}{4}$}
    По колко различни начина могат да се настанят осем студенти в три стаи съответно с едно, три и четири легла?
  \item
    %\marginpar{Отг. $\binom{n}{1}\binom{n-1}{1}\binom{n-2}{1}\binom{n-3}{1}$}
    По колко различни начина четирима младежи могат да поканят на танц четири от $n$ девойки?
  \item
    %\marginpar{Отг. $\binom{6}{2}\binom{4}{2}\binom{2}{2}$}
    Шест различни предмета се боядисват по следния начин: два зелен, два червен, два син цвят.
    По колко различни начина могат да се боядисат предметите?  
  \item
    По колко различни начина могат да се разпределят 10 специалисти в 4 цеха така, че в тях да попаднат съответно по 1,2,3 и 4 души?
  \item
    \marginpar{Отг. $(n+1)! - n! - n!$}
    Иванчо и $n$ негови приятели отиват на кино.
    По колко различни начина могат всички да седнат заедно на един ред, така че Иванчо е винаги
    между двама негови приятели.
  \item
    \marginpar{Отг. $\binom{m}{k}\binom{N-M}{n-k}$}
    В партида от $N$ изделия, $M$ са бракувани.
    По колко различни начина могат да се вземат от партидата $n$ изделия, така че точно $k$ от тях да бъдат бракувани ($M\leq N, k\leq n\leq N$)?
  \item
    \marginpar{Отг. $\binom{4}{2}\binom{48}{4}$}
    От колода с 52 карти се изваждат 6 произволни карти без връщане.
    По колко различни начина могат да се извадят картите, така че две от тях да са дами?
  \item
    \marginpar{Отг. $\binom{4}{2}\binom{4}{2}\binom{44}{2}$}
    От колода с 52 карти се изваждат 6 произволни карти без връщане.
    По колко различни начина могат да се извадят картите, така че две от тях да са тройки и две осмици?
  \item
    \marginpar{Отг. $\binom{48}{24}\binom{4}{2}$}
    По колко различни начина може да се раздели колода от 52 карти на две пачки от по 26 карти така, че във всяка от тях да има по две дами?
  \item
    \marginpar{Отг. $\binom{8}{2}\binom{6}{2}\binom{4}{2}\binom{2}{2}$}
    По колко начина може да се разпределят 8 подаръка между 4 лица, така че всеки да получи по два подаръка?
  \item
    %\marginpar{Отг. $\binom{40}{1}\binom{39}{1}\binom{38}{5}$}
    Провежда се събрание с $40$ присъстващи.
    По колко начина може да се избере председател, секретар и 5 членна комисия?
  \end{enumerate}
\end{problem}


\begin{problem}
  От колода с $52$ карти се избират $11$. По колко различни начина могат да се изберат извадки, в които се срещат:
  \begin{enumerate}[a)]
  \item
    \marginpar{Отг. $\binom{48}{10}\binom{4}{1}$}
    точно $1$ ас;
  \item
    \marginpar{Отг. $\binom{52}{11} - \binom{48}{11} - \binom{48}{10}\binom{4}{1}$}
    поне $2$ валета;
  \item
    \marginpar{Отг. $\binom{39}{7}\binom{13}{4}$}
    точно $4$ пики;
  \item
    \marginpar{Отг. $\binom{52}{11} + \binom{39}{10}\binom{13}{1} + \binom{39}{9}\binom{13}{2}$}
    най-много $2$ кари;
  \item
    \marginpar{Отг. $\binom{3}{2}\binom{12}{2}\binom{36}{7} + \binom{3}{1}\binom{12}{1}\binom{36}{8}$}
    точно $2$ аса и $2$ точно трефи;
  \item
    точно $2$ аса и не повече от $2$ трефи;
  \end{enumerate}
\end{problem}

\begin{problem}
  \begin{enumerate}[a)]
  \item
    \marginpar{Отг. $\binom{6}{2}$}
    Колко е максималният брой прави, които могат да се прекарат през 6 точки?
  \item
    \marginpar{Отг. $\binom{10}{2} - 2$}
    Колко е максималният брой прави, които могат да се прекарат през 10 точки, три от които лежат на една права? 
  \item
    \marginpar{Отг. $\binom{7}{2}$}
    В колко точки се пресичат 7 прави от една равнина, никои три от които не минават през една точка и никои две не са успоредни?
  \item
    \marginpar{Отг. $\binom{n}{2}$}
    Колко е максималният брой точки, в които се пресичат $n$ прави от една равнина?
  \item
    \marginpar{$\binom{n-3}{2} + 3(n-3) = \binom{n}{2} - \binom{3}{2}$}
    Колко е максималният брой точки, в които се пресичат $n$ прави от една равнина, като три от тези прави са успоредни?
  \item
    \marginpar{$\binom{n-4}{2} + 4(n-4)+1 = \binom{n}{2} - (\binom{4}{2} - 1)$}
    Колко е максималният брой точки, в които се пресичат $n$ прави от една равнина, като четири прави минават през една и съща точка?
  \item
    \marginpar{$\binom{n}{2} - (\binom{4}{2} - 1) - \binom{3}{2}$}
    В колко точки се пресичат $n$ прави от една равнина, като три от тези прави са успоредни и четири други минават през една и съща точка.
\end{enumerate}
\end{problem}





% \begin{problem}
%   Докажете, че:
%   \begin{enumerate}[a)]
%   % \item
%   %   $(x+y)^n = \sum^{n}_{i=0}\binom{n}{i}x^iy^{n-i}$;
%   \item
%     $2^n = \sum^n_{k=0}\binom{n}{k}$;
%   \item
%     $3^n = \sum^n_{k=0}2^n\binom{n}{k}$;
%   \item
%     $\binom{n}{k} = \binom{n}{n-k}$;
%   \item
%     $\binom{n+1}{k} = \binom{n}{k} + \binom{n}{k-1}$
%   \item 
%     $\binom{n+m}{r} = \sum^r_{k=0}\binom{n}{r-k}\binom{m}{k}$;
%   \item
%     $\binom{2n}{n} = \sum^n_{k=0}\binom{n}{k}^2$;
%   \item
%     $\binom{n+1}{r+1} = \sum^n_{j=r}\binom{j}{r}$;
%   \item
%     $\binom{2n}{2} = 2\binom{n}{2} + n^2$;
%   \item
%     $\binom{n+r+1}{r} = \sum^r_{k=0}\binom{n+k}{k}$;
%   \item
%     $n2^{n-1} = \sum^{n}_{k=1} k\binom{n}{k}$;
%   \item
%     $n\binom{2n-1}{n-1} = \sum^{n}_{k=1}k\binom{n}{k}^2$;
%   \end{enumerate}
% \end{problem}

\begin{problem}
  \marginpar{Един метод за доказателство е с индукция по $n$. Ние ще разгледаме друг метод.}
  Докажете, че
  \[\sum^n_{k=0}\binom{n}{k}k = n2^{n-1}.\]
\end{problem}
\begin{proof}
  Нека за определеност да положим $A = \{a_1,\dots,a_n\}$.
  Ще докажем задачата като преброим по два пъти един и същ клас от обекти.% , а именно
  % $\Bs = \{\pair{a,B} \mid a\in A\ \&\ B \in \Ps(A)\ \&\ a \not\in B\}$.
  \marginpar{Да означим $\Ps_k(A) = \{B \subseteq A \mid \abs{B} = k\}$ и да напомним, че $\abs{\Ps_k(A)} = \binom{n}{k}$}
  \begin{itemize}
  \item 
    Дясната страна на тъждеството можем да представим по следния начин:
    \[\sum^n_{i=1}\abs{\{B \in \Ps(A) \mid a_i \not\in B\}} = \sum^{n}_{i=1}2^{n-1} = n2^{n-1}.\]
  \item
    Това ни подсказва, че за лявата страна, можем да използваме наблюдението, че
    \begin{align*}
      \sum^n_{i=1}\abs{\{B \in \Ps(A) \mid a_i \not\in B\}} & = \sum^n_{i=1}\sum^n_{k=0}\abs{\{B \in \Ps_k(A) \mid a_i \not\in B\}}\\
      & = \sum^n_{k=0}\sum^n_{i=1}\abs{\{B \in \Ps_k(A) \mid a_i \not\in B\}}.
    \end{align*}
    Да разгледаме всяко $k$ поотделно като използваме тъждеството:
    \[\Ps(A) = \bigcup^n_{k=0}\Ps_k(A).\]
    \begin{itemize}
    \item 
      \marginpar{$\Ps_0(A) = \{\emptyset\}$}
      За $k = 0$ получваме, че  
      \[\sum^n_{i=1}\abs{\{B \in \Ps_0(A) \mid a_i \not\in B\}} = \sum^n_{i=1}1 = \binom{n}{n}n.\]
    \item 
      \marginpar{Да напомним, че \[\binom{n}{k} = \binom{n}{n-k}\]}
      За $k = 1$ получваме, че
      \begin{align*}
        \sum^n_{i=1}\abs{\{B \in \Ps_1(A) \mid a_i \not\in B\}} & = \sum^n_{i=1}\binom{n-1}{1}\\
        & = n\binom{n-1}{n-2}\frac{n-1}{n-1}\\
        & = \binom{n}{n-1}(n-1).
      \end{align*}
    \item
      Да видим дали можем да получим за $k = 2$ формула сходна с тази за $k = 0$ и $k =1$.
      \begin{align*}
        \sum^n_{i=1}\abs{\{B \in \Ps_2(A) \mid a_i \not\in B\}} & = \sum^n_{i=1}\binom{n-1}{2}\\
        & = n\binom{n-1}{n-3}\frac{n-2}{n-2}\\
        & = \binom{n}{n-2}(n-2).
      \end{align*}
    \item
      Сега вече имаме добра идея какъв вид трябва да намерим за произволно $k < n$:
      \begin{align*}
        \sum^n_{i=1}\abs{\{B \in \Ps_k(A) \mid a_i \not\in B\}} & = \sum^n_{i=1}\binom{n-1}{k}\\
        & = n\binom{n-1}{n-1-k}\frac{n-k}{n-k}\\
        & = \binom{n}{n-k}(n-k).
      \end{align*}
    \item
      Да разгледаме накрая и случая $k = n$:
      \marginpar{$\Ps_n(A) = \{A\}$}
      \[\sum^n_{i=1}\abs{\{B \in \Ps_n(A) \mid a_i \not\in B\}} = \sum^n_{i=1}0 = 0 = \binom{n}{0}\cdot 0.\]
  \end{itemize}
  Да обединим всички случаи за $k$, получаваме лявата страна:
  \begin{align*}
    \sum^n_{k=0}\sum^n_{i=1}\abs{\{B \in \Ps_k(A) \mid a_i \not\in B\}} & = \sum^n_{k=0} \binom{n}{n-k}(n-k)\\
    & = \sum^{n}_{k=0}\binom{n}{k}k.
  \end{align*}
\end{itemize}
\end{proof}

\section{Принцип на включването и изключването}

\begin{prop}
  За две крайни множества $A$ и $B$,
  \begin{enumerate}[a)]
  \item 
    ако $A \cap B = \emptyset$, то $\abs{A\cup B} = \abs{A} + \abs{B}$.
  \item
    ако $A\subseteq B$, то $\abs{B\setminus A} = \abs{B} - \abs{A}$.
  \item
    $\abs{A\cup B} = \abs{A} + \abs{B} - \abs{A\cap B}$.
  \end{enumerate}
\end{prop}
\begin{proof}
  \begin{enumerate}[a)]
  % \item
  %   Индукция по броя на елементите на $B$.
  %   \begin{itemize}
  %   \item
  %     $\abs{B} = 0$, то $B = \emptyset$ и тогава за произволно множество $А$,
  %     \[\abs{A\cup B} = \abs{A} + \abs{B}.\]
  %   \item
  %     $\abs{B} = 1$, то $B = \{b\}$ и тогава за произволно крайно множество $A$, за което $b \not\in A$,
  %     е очевидно, че \[\abs{A\cup\{b\}} = \abs{A} + 1.\]
  %   \item
  %     $\abs{B} = n+1$, то $B = B^\prime \cup \{b\}$, $\abs{B^\prime} = n$ и 
  %     нека $A$ е произволно крайно множество, за което $A \cap B = \emptyset$.
  %     \begin{align*}
  %       \abs{A \cup B} = \abs{(A \cup B^\prime) \cup \{b\}} = \abs{A\cup B^\prime} + 1 = \abs{A} + \abs{B^\prime} + 1 = \abs{A} + \abs{B}.
  %     \end{align*}
  %   \end{itemize}
  % \item
  %   Ако $A \subseteq B$, то $B\setminus A\ \cup\ A = B$ и $B\setminus A\ \cap\ A = \emptyset$. Тогава от a):
  %   \begin{align*}
  %     \abs{B} = \abs{B\setminus A\ \cup\ A} = \abs{B\setminus A} + \abs{A}.
  %   \end{align*}
  %   Следователно,
  %   \[\abs{B\setminus A} = \abs{B} - \abs{A}.\]
  \item[в)]
    Имаме, че:
    \begin{align*}
      A\cup B & = A \setminus B\ \cup\ (A\cap B)\ \cup\ B\setminus A\\
      & = A\setminus (A\cap B)\ \cup\ (A\cap B)\ \cup\ B\setminus (A\cap B)
    \end{align*}
    Трите множества в дясната страна на равенството са непресичащи се.
    Тогава, използвайки а) и б), 
    \begin{align*}
      \abs{A \cup B} & = \abs{ A\setminus (A\cap B)} + \abs{A\cap B} + \abs{B\setminus (A\cap B)}\\
      & = \abs{A} - \abs{A\cap B} + \abs{A \cap B} + \abs{B} - \abs{A\cap B}\\
      & = \abs{A} + \abs{B} + \abs{A \cap B}
    \end{align*}
  \end{enumerate}
\end{proof}

\begin{prop}
  Докажете, че за всеки три крайни множества $A$, $B$ и $C$,
  \[\abs{A\cup B \cup C} = \abs{A}+\abs{B}+\abs{C} - \abs{A\cap B} - \abs{B\cap C} - \abs{A\cap C}+ \abs{A\cap B \cap C}.\]
\end{prop}
\begin{proof}
  \begin{align*}
    \abs{(A \cup B) \cup C} & = \abs{A \cup B} + \abs{C} - \abs{(A\cup B)\cap C}\\
    & = (\abs{A} + \abs{B} - \abs{A\cap B}) + \abs{C} - \abs{(A\cap C)\cup(B\cap C)}\\
    & = \abs{A} + \abs{B} + \abs{C} - \abs{A\cap B} - (\abs{A\cap C} + \abs{B\cap C} - \abs{(A\cap C) \cap (B\cap C)})\\
    & = \abs{A} + \abs{B} + \abs{C} - \abs{A\cap B} - \abs{A\cap C} - \abs{B\cap C} + \abs{A\cap B \cap C}
  \end{align*}
\end{proof}


\begin{framed}
\begin{thm}
  Нека $A_1\dots A_n$ са $n$ на брой крайни множества и $n\geq 2$. Тогава:
  \begin{align*}
    |A_1\cup A_2\cup \dots \cup A_n| = & \sum^n_{i=1} |A_i| - \sum_{i < j} |A_{i}\cap A_{j}| + \\
    & \sum_{i < j < k} |A_{i}\cap A_{j}\cap A_{k}|- \dots + (-1)^{n-1}|A_1 \cap A_2\dots \cap A_n|.    
  \end{align*}
\end{thm}
\end{framed}

\begin{problem}
  Колко решения в естествените числа имат уравненията:
  \begin{enumerate}[a)]
  \item
    $x_1+x_2+x_3 = 15$;
  \item
    $x_1 + x_2 + x_3 = 15$, като $x_2 < 3$;
  \item
    % \marginpar{Отг. $\binom{13}{2} - \binom{12}{1} - \binom{11}{1} - \binom{10}{1}$}
    $x_1 + x_2 + x_3 = 15$, като $x_2 \geq 3$;
  \item
    $x_1+x_2+x_3 = 15$, като $x_1 \geq 2\ \&\ x_2 \geq 3$;
  \item
    $x_1+x_2+x_3 = 15$, като $x_1 \geq 2\ \&\ x_2 \geq 3\ \&\ x_3 \leq 8$;
  \item
    $x_1+x_2+x_3+x_4 = 25$, като $x_1 < 2\ \vee\ x_2 \geq 3$;
  \item
    $x_1+x_2+x_3+x_4 = 25$, като $x_1 < 2\ \vee\ x_2 = 3$;
  \item
    $x_1+x_2+x_3+x_4 = 25$, като $x_1 < 2\ \vee\ x_2 < 3$;
  \end{enumerate}
\end{problem}
\begin{solution}
  \begin{enumerate}[a)]
  \item
    Търсим броят на елементите на 
    \[A = \{(x_1,x_2,x_3) \in \Nat^3\mid x_1 + x_2 + x_3 = 15\}.\]
    Това са всички $15$ елементни мултимножества на  $\{x_1,x_2,x_3\}$.
    Например, мултимножеството $\{x_1,x_1,x_3,x_2,x_1,x_3\}$ отговаря на решение на уравнението $x_1 + x_2 + x_3 = 6$,
    където $x_1 = 3$, $x_2 = 1$, $x_3 = 2$.
    Следователно,
    \[\abs{A} = \binom{15 + 3 - 1}{3-1}.\]
  \item
    Търсим броя на елементите на 
    \begin{align*}
      A_2 =\ &\{(x_1,x_2,x_3) \in \Nat^3\mid x_1 + x_2 + x_3 = 15\ \&\ x_2 < 3\}\\
      =\ & \{(x_1,0,x_3) \in \Nat^3\mid x_1 + 0 + x_3 = 15\}\ \cup \\ 
      & \{(x_1,1,x_3) \in \Nat^3\mid x_1 + 1 + x_3 = 15\}\ \cup \\ 
      & \{(x_1,2,x_3) \in \Nat^3\mid x_1 + 2 + x_3 = 15\}.
    \end{align*}
    Лесно се съобразява, че
    \[\abs{A_2} = \binom{16}{1} + \binom{15}{1} + \binom{14}{1}.\]
  \item
    Отговорът е
    \[\abs{A} - \abs{A_2} = \binom{17}{2} - \binom{16}{1} - \binom{15}{1} - \binom{14}{1}.\]
    Друг начин да решим задачата е като съобразим, че 
    \[x_2 \geq 3 \iff x_2 + 3 \geq 0.\]
    Това означава, че търсим броя на решенията на уравнението
    \[x_1 + (x_2 + 3) + x_3 = 15 \iff x_1 + x_2 + x_3 = 12.\]
    \marginpar{Проверете, че $\binom{14}{2} = \binom{17}{2} - \binom{16}{1} - \binom{15}{1} - \binom{14}{1}$}
    Това означава, че отговорът е $\binom{14}{2}$.
  \item
    Първо ще дадем едно доказателство, което е полезно за да упражним принципа за включването и изключването.
    След това ще видим, че по-лесно можем да решим задачата като използваме идеята от доказателството на в).
    Да разгледаме множествата:
    \begin{align*}
      A & = \{(x_1,x_2,x_3) \in \Nat^3\mid x_1 + x_2 + x_3 = 15\},\\
      A_1 & = \{(x_1,x_2,x_3) \in \Nat^3\mid x_1 + x_2 + x_3 = 15\ \&\ x_1 < 2\},\\
      A_2 & = \{(x_1,x_2,x_3) \in \Nat^3\mid x_1 + x_2 + x_3 = 15\ \&\ x_2 < 3\}.
    \end{align*}
    Ние търсим колко елемента има множеството 
    \[B = \{(x_1,x_2,x_3) \in \Nat^3\mid x_1 + x_2 + x_3 = 15\ \&\ x_1 \geq 2\ \&\ x_2 \geq 3\}.\]
    Понеже \[B = (A\setminus A_1) \cap (A\setminus A_2) = A \setminus (A_1 \cup A_2),\]
    трябва да намерим $\abs{A}$ и $\abs{A_1 \cup A_2}$. Тогава отговорът на задачата е:
    \[\abs{B} = \abs{A} - \abs{A_1 \cup A_2}.\]

    Лесно се вижда, че:
    \begin{align*}
      & \abs{A}  = \binom{17}{2} = 136\\
      & \abs{A_1} = 16 + 15 = 31\\
      & \abs{A_2} = 16 + 15 + 14 = 45\\
      & \abs{A_1\cap A_2} = 6.
    \end{align*}
    Освен това, от принципа за включването и изключването, 
    \[\abs{A_1 \cup A_2} = \abs{A_1} + \abs{A_2} - \abs{A_1\cap A_2} = 31 + 45 - 6 = 70.\]

    Следователно, отговорът е 
    \[\abs{B} = \abs{A} - \abs{A_1 \cup A_2} = 136 - 70 = 66.\]

    За второ решение, достатъчно е да намерим броя на решенията на уравнението:
    \[(x_1 + 2) + (x_2 + 3) + x_3 = 15 \iff x_1 + x_2 + x_3 = 10.\]
    Това означава, че отговорът е $\binom{12}{2} = 66$.
  % \item[д)]
  %   Използваме същата идея и означения както в горната задача.
  %   Да разгледаме множествата:
  %   \begin{align*}
  %     A & = \{(x_1,x_2,x_3) \in \Nat^3\mid x_1 + x_2 + x_3 = 15\},\\
  %     A_1 & = \{(x_1,x_2,x_3) \in \Nat^3\mid x_1 + x_2 + x_3 = 15\ \&\ x_1 < 2\},\\
  %     A_2 & = \{(x_1,x_2,x_3) \in \Nat^3\mid x_1 + x_2 + x_3 = 15\ \&\ x_2 < 3\},\\
  %     A_3 & = \{(x_1,x_2,x_3) \in \Nat^3\mid x_1 + x_2 + x_3 = 15\ \&\ x_3 > 8\}.
  %   \end{align*}
  %   Тук търсим колко елемента има множеството
  %   \[B = \{(x_1,x_2,x_3) \in \Nat^3\mid x_1 + x_2 + x_3 = 15\ \&\ x_1 \geq 2\ \&\ x_2 \geq 3\ \&\ x_3 \leq 8\}.\]
  %   Понеже
  %   \[B = (A \setminus A_1) \cap (A\setminus A_2) \cap (A\setminus A_3) = A \setminus (A_1 \cup A_2 \cup A_3),\]
  %   трябва да намерим $\abs{A}$ и $\abs{A_1 \cup A_2 \cup A_3}$.
  %   Тогава отговорът на задачата е 
  %   \[\abs{B} = \abs{A} - \abs{A_1 \cup A_2 \cup A_3}.\]
  %   Лесно се вижда, че:
  %   \begin{align*}
  %     & \abs{A}  = 136\\
  %     & \abs{A_1} = 16 + 15 = 31\\
  %     & \abs{A_2} = 16 + 15 + 14 = 45\\
  %     & \abs{A_3} = 7 + 6 + 5 + 4 + 3 + 2 + 1 = 28\\
  %     & \abs{A_1\cap A_2} = 2.3 = 6\\
  %     & \abs{A_1\cap A_3}  = 7 + 6 = 13\\
  %     & \abs{A_2\cap A_3}  = 7 + 6 + 5 = 18\\
  %     & \abs{A_1\cap A_2\cap A_3} = \abs{A_1 \cap A_2} = 6.
  %   \end{align*}
  %   Сега от принципа за включването и изключването, 
  %   \begin{align*}
  %     \abs{A_1 \cup A_2 \cup A_3} & = \abs{A_1} + \abs{A_2} + \abs{A_3} - \abs{A_1 \cap A_2} - \abs{A_1 \cap A_3} - \abs{A_2 \cap A_3} + \abs{A_1 \cap A_2 \cap A_3}\\
  %     & = 31 + 45 + 28 - 6 - 13 - 18 + 6 = 73
  %   \end{align*}
  %   Следователно, отговорът е:
  %   \[\abs{B} = \abs{A} - \abs{A_1\cup A_2 \cup A_3} = 136 - 73 = 63.\]
  \item[е)]
    Да означим
    \begin{align*}
      A_1 & = \{(x_1,x_2,x_3,x_4)\in\Nat^4 \mid x_1+x_2+x_3+x_4 = 25\ \&\ x_1 < 2\}\\
      A_2 & = \{(x_1,x_2,x_3,x_4)\in\Nat^4 \mid x_1+x_2+x_3+x_4 = 25\ \&\ x_2 \geq 3\}.
    \end{align*}
    Тогава от принципа за включването и изключването, отговорът е 
    \[|A_1 \cup A_2| = |A_1| + |A_2| - |A_1 \cap A_2|.\]
  \end{enumerate}
\end{solution}


% \begin{problem}
%   Дайте комбинаторно доказателство на 
%   \begin{enumerate}[1)]
%   \item
%     $n! = \sum^{n}_{k=0}(-1)^k\binom{n}{k}(n-k)^n$
%   \item
%     $\binom{x+a}{n} = \sum^{k}_{i=0}\binom{x}{i}\binom{a}{n-i}$
%   \end{enumerate}
% \end{problem}

% \begin{problem}
%   \begin{enumerate}[a)]
%   \item
%     \marginpar{Отг. $\binom{n+4-1}{4-1}$}
%     По колко начина могат да се изберат $n$ монети от купчина монети с номинал 5, 10, 20 и 50 стотинки?
%   % \item
%   %   %\marginpar{Отг. }
%   %   По колко начина могат да се изтеглят $13$ от $52$ карти, ако ги различаваме само по цвета?
%   \item
%     %\marginpar{Отг. $\binom{n+m-1}{m-1}$}
% Намерете броя на възможните начини за разпределение на $n$ {\bf неразличими} топки в $m$ различни кутии.
% \item
%     %\marginpar{Отг. $\binom{(n-m)+m-1}{m-1}$}
%     Намерете броя на възможните начини за разпределение на $n$ {\bf неразличими} топки в $m$ различни кутии, 
%     ако няма празна кутия.
%   \item
%     %\marginpar{Отг. $\binom{n+m-1}{m-1} - \binom{(n-m)+m-1}{m-1}$}
%     Намерете броя на възможните начини за разпределение на $n$ {\bf неразличими} топки в $m$ различни кутии,
%     ако съществува поне една празна кутия.
%   \item
%     %\marginpar{$\binom{n+m-1}{m-1}/m!$ ???}
%     Да се намери броя на възможните начини за разпределения на $n$ {\bf неразличими} топки в $m$ {\bf неразличими} кутии.
%   % \item
%   %   %\marginpar{}
%   %   Да се намери броя на възможните начини за разпределения на $n$ {\bf различими} топки в $m$ различни кутии.
%   \end{enumerate}
% \end{problem}

% \begin{problem}
%   Множеството от всички двоични вектори от $\{0,1\}^{n}$, които във фиксирани $n-k$ позиции имат равни значения,
%   ги наричаме $k$-равнини, за $k\leq n$.
%   \begin{enumerate}[a)]
%   \item
%     \marginpar{Отг. $2^{k}$}
%     Колко различни вектора има в една $k$-равнина?
%   \item
%     \marginpar{Отг. $2^{n-k}\binom{n}{n-k}$}
%     Колко различни $k$-равнини има в $\{0,1\}^{n}$?
%   \item
%     \marginpar{Отг. $\binom{n}{n-k}$}
%     Колко различни $k$-равнини съдържат даден фиксиран $n$-мерен вектор?
%   % \item
%   %   %\marginpar{Отг. $$}
%   %   Колко различни $k$-равнини съдържат дадена $l$-равнина, $0\leq l < k$.
%   \end{enumerate}
% \end{problem}



\begin{problem}
  Да фиксираме множеството $U = \{u_1,u_2\dots,u_n\}$.
  Намерете броя на елементите на следните множества:
  \begin{enumerate}[a)]
  \item 
    $S = \{(X,Y) \mid X \subseteq Y \subseteq U\}$;
  \item
    $S = \{(X,Y,Z) \mid X \subseteq Y \subseteq Z \subseteq U\}$;
  \item
    $S = \{(X,Y,Z) \mid X,Y,Z \subseteq U\ \&\ X \cap Y \cap Z = \emptyset\}$.
  \item
    $S = \{(X,Y,Z) \mid X,Y,Z \subseteq U\ \&\ X\cap Y = \emptyset\ \&\ X \cap Z = \emptyset\ \&\ Y \cap Z = \emptyset\}$;
  \end{enumerate}
\end{problem}
\begin{hint}
  \begin{enumerate}[a)]
  \item 
    За всяко $k = 0,\dots,n$, да разгледаме
    \[S_k = \{(X,Y) \mid X \subseteq Y \subseteq U\ \&\ \abs{X} = k\}.\]
    Съобразете, че:
    \begin{itemize}
    \item 
      за $k \neq k'$, $S_k \neq S_{k'}$;
    \item
      $\abs{S} = \abs{\bigcup^n_{k=0} S_k} = \sum^n_{k=0}\abs{S_k}$.
    \item
      $\abs{S_0} = 2^n$;
    \item
      $\abs{S_1} = n.2^{n-1}$;
    \item
      $\abs{S_k} = \binom{n}{k}2^{n-k}$, за всяко $k = 0,1,\dots,n$.
    \end{itemize}
    Тогава отговорът е
    \[\abs{S} = |\bigcup^n_{k=0} S_k| = \sum^n_{k=0} |S_k| = \sum^n_{k=0} \binom{n}{k}1^k.2^{n-k} = (1+2)^n = 3^n.\]
    
    Сега да видим, че тази задача може да се реши по-лесно ако интерпретираме всеки елемент от вида $(X,Y) \in S$
    като една дума над азбуката $\Sigma = \{XY$, $\bar{X}Y$, $X\bar{Y}$, $\bar{X}\bar{Y}\}$.
    Единственото нещо, което трябва да съобразим е, че нямаме букви от вида $X\bar{Y}$, защото това ще означава, че 
    $X \not\subseteq Y$. Така свеждаме задачата до въпроса колко са всички думи с дължина $n$ над азбука с три букви.
    Ясно е, че отговорър е $3^n$.
  \end{enumerate}
\end{hint}


\section{Комбинаторни задачи за функции}

\begin{problem}
  \marginpar{Отг. $\binom{11}{5}$}
  Нека $(a_1,a_2,\dots,a_{12})$ е пермутация на числата от 1 до 12, за които е изпълнено условието:
  \[a_1 > a_2 > a_3 > a_4 > a_5 > a_6 < a_7 < a_8 < a_9 < a_{10} < a_{11} < a_{12}.\]
  Намерете броя на тези пермутации.  
\end{problem}

\begin{problem}
  Да фиксираме естествените числа $m$ и $n$.
  Една функция \[f:\{1,\dots,n\}\to\{1,\dots,m\}\] е монотонно ненамаляваща, ако
  \[(\forall i\forall j)[1\leq i<j\leq n \rightarrow f(i)\leq f(j)].\]
  \begin{enumerate}[a)]
  \item
    \marginpar{Отг. $\binom{n+m-1}{m-1}$}
    Колко такива функции съществуват?
  \item
    \marginpar{Във всяка кутийка има топка. Отг. $\binom{(n-m)+m-1}{m-1}$}
    Колко от тези функции са сюрективни при $n\geq m$?
  \item
    \marginpar{Отг. $\binom{m}{n}$}
    Колко от тези функции са инективни при $n\leq m$?
  \end{enumerate}
\end{problem}

\begin{problem}
  Да разгледаме функциите от вида $f:A\to B$,
  където $\abs{A} = k$, $\abs{B} = n$.
  \begin{enumerate}[a)]
  \item 
    \marginpar{Отг. $n^k$}
    Колко са всички тези функции?
  \item
    \marginpar{Само ако $k\leq n$, $n!/(n-k)!$}
    Колко от тези функции са инективни?
  \item
    \marginpar{$\binom{n+k-1}{k-1}$}
    Колко от тези функции са монотонно ненамаляващи?
  \item
    \marginpar{Само ако $n = k$, $n!$}
    Колко от тези функции са биективни?
  \item
    \marginpar{Това е по-трудно.}
    Колко от тези функции са сюрективни?
  \end{enumerate}
\end{problem}


\begin{problem}
  Да разгледаме функциите от вида $f:A\to B$, където $\abs{A} = k$, $\abs{B} = n$.
  Колко от тези функции са сюрективни?
\end{problem}
\begin{solution}
  Нека $B = \{b_1,b_2,\dots,b_n\}$.
  Да означим с $F$ всички функции от вида $f:A\to B$.
  \marginpar{$Range(f) = \{f(a) \mid a\in A\}$}
  \[F_i = \{f:A\to B\mid b_i \not\in Range(f)\}.\]
  Да означим с $S$ сюрективните функции $f:A\to B$.
  Понеже сюрективните функции са тези, за които $Range(f) = B$, то
  \[S = F\setminus(F_1\cup F_2 \cup \dots \cup F_n).\]
  Лесно се съобразява, че имаме следните равенства:
  \begin{align*}
    & \abs{F} = n^k\\
    & \abs{F_i} = (n-1)^k\\
    & \abs{F_i \cap F_j} = (n-2)^k\\
    & \sum^n_{i=1}\abs{F_i} = n.(n-1)^k\\
    & \sum_{i < j}\abs{F_i \cap F_j} = \binom{n}{2}(n-2)^k\\
    & \sum_{i < j < l}\abs{F_i \cap F_j \cap F_l} = \binom{n}{3}(n-3)^k\\
    & \quad \vdots
  \end{align*}
  Прилагайки принципа на включването и изключването, получаваме:
  \begin{align*}
    \abs{S} & = \abs{F} - \sum_i \abs{F_i} + \sum_{i < j}\abs{F_i\cap F_j} - \sum_{i < j < l}\abs{F_i \cap F_j \cap F_l} + \dots \\
    & = n^k - \binom{n}{1}(n-1)^k + \binom{n}{2}(n-2)^k - \binom{n}{3}(n-3)^k + \dots\\
    & = n^k + (-1)^1\binom{n}{1}(n-1)^k + (-1)^2\binom{n}{2}(n-2)^k + (-1)^3\binom{n}{3}(n-3)^k + \dots\\
    & = \sum^n_{i=0}(-1)^i\binom{n}{i}(n-i)^k
  \end{align*}
\end{solution}

Сега ще видим едно приложение на горната задача.

\begin{problem}
  \marginpar{Колко са сюрективните функции $f:A\to B$, като $\abs{A} = 7$, $\abs{B}=3$?}
  Нека да имаме 7 топки с номер на всяка от тях и нека имаме 3 различни кутии, отново номерирани.
  По колко начина можем да поставим топките в кутиите, така че във всяка кутия да има поне по една топка ?
\end{problem}

\begin{remark}
  Да разгледаме множествата $A=\{1,2,\dots,n\}$ и $\Sigma = \{a_1,a_2,\dots,a_k\}$.
  Тогава имаме следните преводи между езика на функциите и езика на думите и азбуките.
  \newline
  \begin{tabular}{|l|l|}
    \hline
    функциите от вида $f:A\to \Sigma$ & думите с дължина $n$ над азбуката $\Sigma$ \\
    \hline
    \hline
    {\bf всички} такива функции & {\bf всички} такива думи\\
    \hline
    {\bf инективните} функции, $n \leq k$ & думите  {\bf без повторения на букви} \\
    \hline
    {\bf сюрективните} функции, $n \geq k$ & думите, в които {\bf всяка буква се среща} \\
    \hline
    {\bf биективните} функции, $n = k$ & думите, в които всяка буква от $\Sigma$ се \\
    & среща {\bf точно веднъж} \\
    \hline
  \end{tabular}

  % Да разгледаме две произволни множества $A = \{1,2,\dots,n\}$ и $B=\{1,2,\dots,k\}$.
  % Да напмним, че една функция $f:A \to B$ е {\bf монотонно ненамаляваща}, ако
  % \[(\forall i\forall j)[1\leq i<j\leq n \rightarrow f(i)\leq f(j)].\]
  % Тогава имаме следните еквивалентни преводи между езика на функциите и езика на кутиите и топките:
  % \newline
  % \begin{tabular}{|l|l|}
  %   \hline
  %   мон. ненамаляващите $f:A\to B$ & \\
  %   \hline
  %   \hline
  %   {\bf всички} такива функции & {\bf всички} такива думи\\
  %   \hline
  %   {\bf инективните} функции при $n \leq k$ & думите, в които {\bf няма повторения на букви} \\
  %   \hline
  %   {\bf сюрективните} функции при $n \geq k$ & във всяка кутия има поне една топка\\
  %   \hline
  %   {\bf биективните} функции при $n = k$ & {\bf всяка буква се среща точно веднъж} \\
  %   \hline
  % \end{tabular}

\end{remark}

% \subsection{Пълно разбъркване на множество}
% \marginpar{На англ. derangement}
% Нека $A = \{a_1,\dots a_n\}$ е произволно множество от $n$ елемента.
% Една пермутация $f$ на елементите на $A$ наричаме {\bf пълно разбъркване}, 
% ако $f$ няма неподвижни точки, т.е. $(\forall a\in A)[f(a)\neq a]$.
% % Можем да означим едно пълно разбъркване $f$ на $A$ 
% % като редицата $(a_{i_1},a_{i_2},\dots,a_{i_n})$,
% % където $f(j) = a_{i_j}$.
% \begin{example}
%   Нека $A = \{1,2,3\}$. Да изредим всички пълни разбърквания на $A$:
%   \[ (2,3,1), \quad  (3,1,2)\]
%   \end{align*}
%   Нека сега $A = \{1,2,3,4\}$. Пълните разбърквания на $A$ са:
%   \begin{align*}
%     & (2,1,4,3), \quad (2,4,1,3), \quad (2,3,4,1)\\
%     & (3,1,4,2), \quad (3,4,1,2), \quad (3,4,2,1)\\
%     & (4,1,2,3), \quad (4,3,2,1), \quad (4,3,1,2)\\
%   \end{align*}
% \end{example}

% \begin{problem}
%   Напишете програма, която намира всички пълни разбърквания на зададено като вход множество $A$.
% \end{problem}


% \begin{problem}
%   Означаваме с $D(n)$ всички пълни разбърквания на едно множество с $n$ елемента.
%   \begin{enumerate}[a)]
%   \item
%     Докажете, че $D(n) - nD(n-1) = (-1)^{n}$ за $n \geq 2$.
%   \item
%     Намерете $D(n)$.
% \end{enumerate}
% \end{problem}
% \begin{proof}
%   \begin{enumerate}[a)]
%   \item
%     Доказателството е с индукция по $n$.
%     Лесно се  съобразява, че имаме това свойство за $n=2$, 
%     защото $D(2) = 1$, а $D(1) = 0$.
    
%     Да приемем, че твърдението е вярно за $n > 2$, като примем, че то е вярно за $n-1$.
%     Нека $A = \{a_1,a_2,\dots,a_n\}$ и да фиксираме последния елемент $a_n$ на $A$.
%     Да означим $B = A\setminus\{a_n\}$.
%     %$B$ имат $D(n-1)$ разбърквания.
%     Да разгледаме едно пълно разбъркване на $B$, $\pi(i) = a_i$, $i = 1,\dots,n-1$.
% %    \[\pi = [a_{i_1}, a_{i_2},\dots, a_{i_{n-1}}].\]
%     Ако заменим $\pi(j)$ с $a_n$ и поставим $\pi(j)$ на $n$-та позиция, то получаваме разбъркване $\rho$ на $A$.
%     Това можем да направим за всяко $ 1 \leq j \leq n-1$ и всяко разбъркване на $B$ и получаваме всеки път ново разбъркване на $A$.
%     Това са общо $(n-1)D(n-1)$ разбърквания.

%     За съжаление, това не са всички пермутации, които ни трябват.
%     Нека сега $\pi_i$ е пермутация на $B$, която има точно една неподвижна точка и $\pi_i(i) = a_i$.
%     Като сменим $a_i$ с $a_n$ и поставим $a_i$ на $n$-та позиция, то получаваме ново разбъркване на $A$.
%     Лесно се съобразява, че разбъркванията получени по този начин са $(n-2)D(n-2)$ 
%     и с това се изчерпват начините за генериране на разбърквания на $A$ от пермутациите на $B$.   

%     Следователно, \[D(n) = (n-1)D(n-1) + (n-2)D(n-2).\]
%     От индукционното предположение знаем, че \[D(n-1) = (n-1)D(n-2) + (-1)^{n-1}.\]
%     Като заместим, получаваме \[D(n) = (n-1)D(n-1) + D(n-1) - (-1)^{n-1},\]
%     което ни дава крайния резултат \[D(n) = nD(n-1) + (-1)^{n}.\]
        
%   \item
%     Може да се реши и с индукция, използвайки предишната подточка.
%     Ще дадем директно рещение като приложим принципа за включване и изключване.
%     Да означим с $P$ броят на всички пермутации на $A$.
%     Да означим с $P_i$ броят на пермутациите, които запазват $i$-тия елемент от $A$.
%     Следователно,
%     \begin{align*}
%       D(n) =\ & |P\setminus{(P_1\cup P_2 \cup \dots \cup P_n)}|\\
%       =\ & |P| - |P_1\cup P_2 \cup \dots \cup P_n|\\
%       =\ & |P| - \sum^n_{i = 1}|P_i| + \sum_{i < j }|P_i\cap P_j| - \dots \\
%       =\ & n! - \sum^n_{i = 1}(n-1)! + \sum_{i<j} (n-2)! - \dots\\
%       =\ & \binom{n}{0}n! + (-1)\binom{n}{1}(n-1)! + (-1)^2\binom{n}{2}(n-2)! - \dots\\
%       =\ & \sum^n_{k=0}(-1)^k\binom{n}{k}(n-k)!
%     \end{align*}
%   \end{enumerate}
% \end{proof}

% \begin{problem}
%   Дадени са $n$ кутии и $m$ неразличими топки.
%   По колко начина могат да се разпределят всички топки в кутиите, така че нито в една кутия да няма повече от $r$ топки?
% \end{problem}
% \begin{proof}
%   Първо, нека отбележим, че $r.n \geq m$. В противен случай, няма да можем да разпределим всички топки в кутиите.
%   Имаме $S = \binom{n+m-1}{n-1}$ общо начини за разпределяне на топките, без ограничение за брой топки в една кутия.
 
%   Да разгледаме едно от тези разпределения на топките.
%   Нека при него във всяка кутия има $r_i$ топки и $\sum^{n}_{i=1} r_i = m$.
%   Нека $p = \lfloor{\frac{m}{r}}\rfloor$.
%   Ясно е, че не може да има повече от $p$ кутии с повече от $r$ топки.
 
%   \begin{enumerate}
%     \item
%       Ако в $i$-тата кутия има повече от $r$ топки, т.е. $r_i = (r+1)+r'_i$, то 
%       броят на различните начини за разпределения на $m-(r+1)$ топки в $n$ кутии е 
%       \[S_1 = \binom{n}{1}\binom{m-(r+1)+(n-1)}{n-1}.\]
%       $S_1$ е броят на разпределенията с поне една кутия с повече от $r$ топки.
%     \item
%       Ако в $i$-тата и $j$-тата кутии има повече от $r$ топки, т.е.
%       $r_i = (r+1)+r'_i, r_j = (r+1)+r'_j$.
%       Тогава броят на различните разпределения на $m-2(r+1)$ топки в $n$ кутии е
%       \[S_2 = \binom{n}{2}\binom{m-2(r+1)+(n-1)}{n-1}.\]
%       $S_2$ е броят на разпределенията с поне две кутии с повече от $r$ кутии.
%     \item
%       Продължаваме да дефинираме $S_i$ до $i=p$.
%   \end{enumerate}

%   Накрая от принципа за включването и изключването получаваме, че крайният резултат е $S + \sum^{p}_{i=1}(-1)^{i}S_i$.
% \end{proof}


\section{Принцип на Дирихле}
\begin{problem}%[\cite{rosen}, стр. 351]
  В един месец от 30 дни се провежда баскетболен турни, в който се играе поне един мач на ден, но всички мачове са не повече от 45.
  Покажете, че има период от последователни дни от месеца, в който се провеждат точно 14 мача.
\end{problem}
\begin{proof}
  Нека $a_j$ означава сумата на всички проведени мачове в първите $j$ дни на месеца.
  Търсим такива $i<j$, че $a_j - a_i = 14$.
  От условието следва, че редицата $a_1,a_2,\dots, a_{30}$ е строго монотонно растяща и
  \[(\forall j)[0\leq j\leq 30 \rightarrow a_j \leq 45].\]
  Ясно е също, че редицата $a_1+14,a_2+14,\dots,a_{30}+14$ е строго монотонно растяща.
  Образуваме редица от 60 елемента $a_1,\dots,a_{30},a_1+14,\dots,a_{30}+14$, като 
  всеки елемент на редицата приема стойност от 1 до 59.
  Тогава от принципа на Дирихле следва, че съществуват два елемента на редицата, които са равни.
  Първите 30 са различни са различни помежду си, вторите 30 елемента също са различни помежду си.
  Следователно, $a_i = a_j + 14$ за някои $i,j$.
  Тогава в дните от $j$ до $i$ са проведени точно 14 мача.
\end{proof}

\begin{problem}%[\cite{rosen}, стр. 351]
  Нека имаме редица $a_0,\dots,a_n$от $n+1$ произволни числа, ненадвишаващи $2n$.
  Покажете, че трябва да съществува $i$ такова, че $a_i\vert a_j$ за някое $j\neq i$.
\end{problem}
\begin{proof}
  Да представим всеки от елементите на редицата $a_j = 2^{k_j}q_j$, където $q_j$ е нечетно.
  Да разгледаме редицата от нечетни числа $q_0,\dots,q_n$, като имаме и условието $q_i \leq 2n$.
  Имаме само $n$ нечетни числа в интервала $[0,2n]$, следователно $q_i = q_j = q$, за някои $i,j$.
  Тогава $a_i = 2^{k_i}q$ и $a_j = 2^{k_j}q$ и е ясно, че или $a_i\vert a_j$ или $a_j\vert a_i$.  
\end{proof}

% \begin{lemma}
%   Нека имаме редица от $n^2 + 1$ различни реални числа $a_1,a_2,\dots a_{n^2+1}$.
%   Тогава съществува подредица с дължина $n+1$, която е или монотонно растяща или монотонно намаляваща.
% \end{lemma}
% \begin{proof}
%   Искаме да намерим индекси $1\leq i_1<i_2<\dots i_{n+1}\leq n^2+1$, за които или $a_{i_1}<a_{i_2}<\dots< a_{i_{n+1}}$ или
%   $a_{i_1}>a_{i_2}>\dots> a_{i_{n+1}}$.

%   Нека за всяко $1\leq i \leq n^2+1$ да означим с $\eta_i$ дължината на най-дългата монотонно растяща редица, която започва от $a_i$.
%   Ако имаме за някое $i$, $\eta_i \geq n+1$, то сме готови.
%   Иначе, от принципа на Дирихле, съществува $1\leq j \leq n$, редица $i_1<i_2<\dots<i_m$ за $m \geq \lceil{\frac{n^2+1}{n}}\rceil = n+1$ и
%   $\eta_{i_1} = \eta_{i_2} = \dots = \eta_{i_m} = j$.
%   Ако $a_{i_k} < a_{i{k+1}}$, то тогава $\eta_{i_k}\geq j+1$, което е противоречие.
%   Следователно, $a_{i_k} > a_{i{k+1}}$ и така получаваме монотонно намаляваща редица $a_{i_1} > a_{i_2} > \dots > a_{i_m}$, която има дължина поне $n+1$.
% \end{proof}

\begin{problem}
  Нека имаме редица от $n$ произволни, не непременно различни, естествени числа $a_1,\dots,a_n$.
  Тогава има подредица от последователни елементи $a_i,a_{i+1},\dots,a_{j}$, за които
  $n | (\sum^{j}_{k=i}a_k)$.
\end{problem}
\begin{proof}
  Да разгледаме редицата от $n+1$ елемента:
  \[\sum^0_{i=1}a_i,\sum^1_{i=1}a_i,\dots,\sum^n_{i=1}a_i.\]
  Тъй като има $n$ различни остатъка при деление на $n$, то от принципа на Дирихле следва, че 
  има поне два елемента $\sum^l_{i=1}a_i, \sum^k_{i=1}a_i$, за $l<k$, които дават един и същ остатък при деление на $n$.
  Получаваме, че \[n|(\sum^l_{i=1}a_i - \sum^k_{i=1}a_i)\ \Rightarrow\ n|(\sum^k_{i=l+1}a_i).\]
\end{proof}

\section{Допълнителни задачи}

\begin{problem}
  Отговорете на следните въпроси:
  \begin{enumerate}[a)]
  \item
    \marginpar{$2.10!10!$}
    По колко начина могат $n$ момчета и $n$ момичета да седнат на ред с $2n$ стола, като няма двама от един пол седящи един до друг?
  \item
    \marginpar{$2.10!10! - 2.7.3!3!$}
    По колко начина могат $n$ момчета и $n$ момичета да седнат на ред с $2n$ стола, като няма двама от един пол седящи един до друг и Иванчо и Марийка не седят един до друг? 
  \item
    \marginpar{Отг. $2.(n-1)!$}
    По колко различни начина могат да се подредят на рафт $n$ книги, така че две от тях, определени предварително, да са една до друга?
  \item
    \marginpar{Отг. $\frac{n!}{2n}$}
    Колко различни гердана могат да се направят от $n$ различни перли, като се използват всичките?
  \item
    \marginpar{Отг. $2.(n-2)(n-3)!$}
    На хоро в кръг са хванали общо $n$ души, между които и Иванчо и Марийка.
    Колко са възможните подредби, при които Иванчо и Марийка са един до друг?
 \item
    На хоро в кръг са хванали общо $n$ души, между които и Иванчо и Марийка.
    Колко са възможните подредби, при които Иванчо и Марийка не са един до друг?
  \item
    \marginpar{Отг. $2^n(n-1)!$}
    Имаме $n$ съпружески двойки, които седят на $2n$ места около една кръгла маса. 
    По колко начина могат да седнат всички двойки, ако ротациите се броят за едно и също подреждане, и
    всеки мъж седи до половинката си.
  \item
    \marginpar{Това е както при герданите}
    Две сядания на една кръгла маса не са различни, ако всеки от седящите има едни и същи съседи.
    По колко различни начина могат да седнат около една кръгла маса:
    \begin{enumerate}
    \item
      \marginpar{Отг. $\frac{(n-1)!}{2}$}
      $n (\geq 2)$ човека;
    \item
      \marginpar{Отг. $\frac{2n!n!}{2.2n} = \frac{n!(n-1)!}{2}$}
      $n$ мъже и $n$ жени, като двама души от един и същ пол не седят един до друг.
    \end{enumerate}
  \end{enumerate}
\end{problem}

\begin{problem}
  \marginpar{Тази задача мисля, че трябва да бъде при включването и изключването}
  Да разгледаме азбуката $\Sigma = \{a_1,\dots,a_n\}$.
  Да се намерят всички $k$-буквени думи над азбуката $\Sigma$, за $k \leq n$, където:
  \begin{enumerate}[a)]
  \item
    \marginpar{Отг. $\frac{n!}{(n-k)!}$}
    нито една буква не се повтаря;
  \item
    \marginpar{Отг. $n^{\lceil{\frac{k}{2}}\rceil}$}
    са палиндроми;
  \item
    \marginpar{Отг. $n(n-1)^{k-1}$}
    нямат две последователни еднакви букви;
  \item
    \marginpar{Отг. $n^k - n(n-1)^{k-1}$}
    имат две последователни еднакви букви;
  \item
    \marginpar{Отг. $\binom{k}{2}n\cdot\frac{(n-1)!}{(n-1-(k-2))!}$}
    съществува само една буква, която се среща точно два пъти;
  \item
    \marginpar{Отг. $n^k - \frac{n!}{(n-k)!}$}
    съществува буква, която се повтаря;
  % \item
  %   \marginpar{Отг. $\binom{k}{3}n(n-1)^{k-3}$}
  %   съществува буква, която се среща точно три пъти;
  % \item
  %   \marginpar{Отг. $2\binom{k}{2}(n-2)^{k-2}$}
  %   буквите $a, b\in \Sigma$ се срещат точно по веднъж;
  \end{enumerate}
\end{problem}


\begin{problem} % Гаврилов стр. 265, зад. 7
  Нека $U$ е множество от $n$ елемента, $n\geq 3$. За всяко множество $X\subseteq U$, с $\overline{X}$ означаваме $U\setminus X$.
  Също така, за множества $X$ и $Y$, понякога ще пишем $XY$ вместо $X \cap Y$.
  Намерете броя на елементите на следните множества:
  \begin{enumerate}[a)]
  \item
    \marginpar{Отг. $4^n$}
    $\{(X,Y) \mid X,Y\subseteq U\}$
  \item
    \marginpar{Отг. $2\binom{n}{1} 2^{n-1}$}
    $\{(X,Y) \mid X,Y\subseteq U\ \&\ \vert{X}\vert = 1\}$;
  \item
    \marginpar{Отг. $2^2\binom{n}{2}2^{n-2}$}
    $\{(X,Y) \mid X,Y\subseteq U\ \&\ \vert{X}\vert = 2\}$;
  \item
    \marginpar{Отг. $4^n - 3^{n}$}
    $\{(X,Y) \mid X,Y\subseteq U\ \&\ \vert{X}\vert \geq 1\}$;
  \item
    $\{(X,Y) \mid X,Y\subseteq U\ \&\ \vert{X}\vert = k\}$ за произволно $k \leq n$;
  \item
    \marginpar{Отг. $3^{n} + \binom{n}{1} 3^{n-1}$}
    $\{(X,Y) \mid X,Y\subseteq U\ \&\ \vert{X}\vert \leq 1\}$;
  \item
    \marginpar{Отг. $\binom{n}{1}\binom{n}{1} = n^2$}
    $\{(X,Y) \mid X,Y\subseteq U\ \&\ \vert{X}\vert = 1\ \&\ \vert{Y}\vert = 1\}$;
  \item
    \marginpar{Нямаме буква $XY$}
    $\{(X,Y) \mid X,Y\subseteq U\ \&\ X \cap Y = \emptyset\}$;
  \item
    \marginpar{Една буква $XY$. Отг. $\binom{n}{1}3^{n-1}$}
    $\{(X,Y) \mid X,Y\subseteq U\ \&\ \abs{X \cap Y} = 1\}$;
  \item
    $\{(X,Y) \mid X,Y\subseteq U\ \&\ \abs{X \cap Y} = k\}$ за произволно $k \leq n$;
  \item
    \marginpar{Отг. $2\binom{n}{1}2^{n-1} = n2^n$}
    $\{(X,Y) \mid X,Y\subseteq U\ \&\ \abs{(X\setminus Y) \cup (Y\setminus X)} = 1\}$;
  \item
    $\{(X,Y) \mid X,Y\subseteq U\ \&\ X\cap Y = \emptyset\ \&\ |X|\geq 1\ \&\ |Y|\geq 1\}$;
  \item
    $\{(X,Y) \mid X,Y\subseteq U\ \&\ X\cap Y = \emptyset\ \&\ |X|\geq 2\ \&\ |Y|\geq 2\}$;
  \item
    $\{(X,Y) \mid X,Y\subseteq U\ \&\ |X\setminus Y| = 1\}$;
  \item
    $\{(X,Y) \mid X,Y\subseteq U\ \&\ |X\setminus Y| = k\}$ за произволно $k \leq n$;
  \item
    $\{(X,Y) \mid X,Y\subseteq U\ \&\ |(X\setminus Y)\cup(Y\setminus X)| = 1\ \&\ |X|\geq 2\ \&\ |Y|\geq 2\}$;
  \item
    \marginpar{Използвайте принципа за вкл. и изкл.}
    $\{(X,Y) \mid X,Y\subseteq U\ \&\ X\cap Y = \emptyset\ \&\ |X|\geq 2\ \&\ |Y|\geq 3\}$;
  \item
    \marginpar{Използвайте принципа за вкл. и изкл.}
    $\{(X,Y) \mid X,Y\subseteq U\ \&\ |(X\setminus Y)\cup(Y\setminus X)| = 1\ \&\ X\cap Y = \emptyset\ \&\ |X|\geq 2\ \&\ |Y|\geq 3\}$;
  \item
    \marginpar{Отг. $8^n$}
    $\{(X,Y,Z) \mid X,Y,Z\subseteq U\}$;
  \item
    \marginpar{Отг. $6^n$}
    $\{(X,Y,Z) \mid X,Y,Z\subseteq U\ \&\ X \cap Y = \emptyset\}$;
  \item
    \marginpar{$U = X\overline{Y}Z\cup X\overline{YZ} \cup \overline{X}Y\overline{Z}$. Отг. $3^n$}
    $\{(X,Y,Z) \mid X,Y,Z\subseteq U\ \&\ X\cup Y\overline{Z} = \overline{X}\cup\overline{Y}\}$;
  \item
    $\{(X,Y,Z) \mid X,Y,Z\subseteq U\ \&\ Y\cup X = Z\cup\overline{Y}\}$;
  \item
    $\{(X,Y,Z) \mid X,Y,Z\subseteq U\ \&\ X\cup Y\overline{Z} = \overline{X}\cup\overline{Y}\ \&\ |Z| = 0\}$;
  \item
    $\{(X,Y,Z) \mid X,Y,Z \subseteq U\ \&\ X\cup Y\overline{Z} = \overline{X}\cup\overline{Y}\ \&\ |X|\geq 1\ \&\ |Y|\geq 1\ \&\ |Z| = 1\}$;
  \item
    $\{(X,Y,Z) \mid X,Y,Z\subseteq U\ \&\ X\cup Y\overline{Z} = \overline{X}\cup\overline{Y}\ \&\ |X|\geq 1\ \&\ |Y|\geq 1\ \&\ |Z|\leq 1\}$;
  \item
    $\{(X,Y,Z) \mid X,Y,Z\subseteq U\ \&\ X \cup YZ = \ov{X} \cup \ov{Z}\}$;
  \item
    $\{(X,Y,Z) \mid X,Y,Z\subseteq U\ \&\ X\ov{Y} \cup YZ = U\}$;
  \end{enumerate}
\end{problem}
\begin{hint}
  \begin{enumerate}[a)]
  \item
    Понеже всички подмножества на $U$ са $2^n$ и множествата $X$ и $Y$ са независими едно от друго, то
    лесно се съобразява, че броят на елементите на $\{(X,Y) \mid X,Y \in U\} = \Ps(U) \times \Ps(U)$ е $4^n$.
    Ще дадем и друго доказателство, което ще ни помогне да решаваме по-сложни задачи, в които имаме връзка между 
    елементите на $X$ и $Y$.
    
    Нека $U = \{u_1,\dots,u_n\}$ и да разгледаме азбуката $\Sigma = \{XY, X\bar{Y}, \bar{X}Y, \bar{X}\bar{Y}\}$.
    На всеки елемент на  $\{(X,Y) \mid X,Y \subseteq U\}$ можем еднозначно да съпоставим 
    дума  $\alpha = a_1\cdots a_n$ над азбуката $\Sigma$
    по следния начин:
    \marginpar{Всеки елемент $u_i \in U$ попада в един от четирите случаи}
    \begin{itemize}
    \item 
      ако $u_i \in X \cap Y$, то $a_i = XY$;
    \item 
      ако $u_i \in X \cap \bar{Y}$, то $a_i = X\bar{Y}$;
    \item 
      ако $u_i \in \bar{X} \cap Y$, то $a_i = \bar{X}Y$;
    \item 
      ако $u_i \in \bar{X} \cap \bar{Y}$, то $a_i = \bar{X}\bar{Y}$.
    \end{itemize}
    Да разгледаме няколко примера:
    \begin{itemize}
    \item
      на двойката $(\{u_1\},\{u_2\})$ съпоставяме думата $\alpha = a_1\cdots a_n$,
      където $a_1 = X\bar{Y}$, $a_2 = \bar{X}Y$, $a_i = \bar{X}\bar{Y}$ за $i \geq 3$.
    \item 
      на двойката $(U,\{u_2\})$ съпоставяме думата $\alpha = a_1\cdots a_n$,
      където $a_2 = XY$ и $a_i = X\bar{Y}$ за $i \neq 2$.
    \item
      на двойката $(\{u_1\},\{u_1,u_2\})$ съпоставяме думата $\alpha = a_1\cdots a_n$,
      където $a_1 = XY$ и $a_2 = \bar{X}Y$, $a_i = \bar{X}\bar{Y}$ за $i \geq 3$.
    \end{itemize}
    Понеже всички думи с дължина $n$ над азбука с $4$ букви са $4^n$, 
    то всички двойки $(X,Y)$ са също $4^n$.
  \item
    Трябва да намерим всичи думи с дължина $n$ над азбуката $\{XY,X\ov{Y},\ov{X}Y,\ov{X}\ov{Y}\}$,
    като буквите $XY$ и $X\ov{Y}$ се срещат общо веднъж.
    Това означава, че от $n$ позиции трябва да изберем една, в която да поставим $XY$ или $X\ov{Y}$,
    а в останалите $n-1$ позиции поставяме буквите $\ov{X}Y$ или $\ov{X}\ov{Y}$.
    Така получаваме като резултат \[2\binom{n}{1}2^{n-1} = n2^n.\]
  \item
    Тук разглеждаме тези думи с дължина $n$ над азбуката $\Sigma$, като
    има {\em точно едно} срещане на $XY$ или $X\ov{Y}$.
    Всички тези думи са $2\binom{n}{1}2^{n-1}$
  \item[ф)]
    Понеже $X \cap Y = \emptyset$, то в думите не се срещат буквите $XYZ$ и $XY\ov{Z}$.
    Така остават $6$ възможни букви и оттук веднага следва, че всички такива думи са $6^n$.    
  \item[х)]
    Да разгледаме какво означава $X\cup Y\ov{Z} = \ov{X} \cup \ov{Y}$.
    \begin{itemize}
    \item 
      Ако $x \in X$, то $x \in X \cup Y\ov{Z}$ и следователно $x \in \ov{X} \cup \ov{Y}$.
      Тогава е ясно, че $x \in \ov{Y}$. Това означава, че имаме буквите $X\ov{Y}Z$ и $X\ov{Y}\ov{Z}$.
    \item
      Ако $x \in \ov{X}$, то $X \in \ov{X} \cup \ov{Y}$ и следователно $x \in X \cup Y\ov{Z}$.
      Сега получаваме, че $x \in Y\ov{Z}$. Това означава, че имаме буквата $\ov{X}Y\ov{Z}$.
    \end{itemize}
    Видяхме, че с горното ограничение трябва да разгледаме само думите с дължина $n$ съставени от три букви.
    Те са общо $3^n$ на брой.
  \end{enumerate}
\end{hint}


%%% Local Variables: 
%%% mode: latex
%%% TeX-master: "discrete-math"
%%% End: 

\chapter{Булеви функции}

%Основен източник е %\cite{gavrilov}.

Да припомним таблицата за истинност на някои от основните булеви функции на два аргумента.
\marginpar{$x\oplus y$ - симетрична разлика}

\begin{tabular}{|c|c|c|c|c|c|c|c|c|c|}
  \hline
  $x$ & $y$ & $\overline{x}$ & $x \vee y$ & $xy$ & $x \rightarrow y$ & $\overline{x}\vee y$ & $x \iff y$ & $x \oplus y$ & $x\overline{y} \vee \overline{x}y$\\
  \hline
  \hline
  0 & 0 & 1 & 0 & 0 & 1 & 1 & 1 & 0 & 0 \\
  \hline
  0 & 1 & 1 & 1 & 0 & 1 & 1 & 0 & 1 & 1 \\
  \hline
  1 & 0 & 0 & 1 & 0 & 0 & 0 & 0 & 1 & 1 \\
  \hline
  1 & 1 & 0 & 1 & 1 & 1 & 1 & 1 & 0 & 0 \\
  \hline
\end{tabular}

\section{Основни свойства}

\marginpar{Често вместо $x\wedge y$ пишем $x\cdot y$ или $xy$. Също така, вместо $\neg x$ пишем $\overline{x}$}
\begin{enumerate}[1)]%% ДА се напишат всичките от Манев, стр. 189
\item
  Комутативни свойства
  \[xy = yx,\quad x\vee y = y\vee x,\quad x\oplus y = y\oplus x\]
\item
  Асоциативни свойства
  \[(xy)z = x(yz),\quad (x\vee y)\vee z = x\vee (y\vee z),\quad (x\oplus y)\oplus z = x\oplus (y\oplus z)\]
\item
  Лесно се проверява с таблиците за истинност, че:
  \[x\oplus y = x\ov{y}\vee \ov{x}y = (x\vee y)(\ov{x}\vee\ov{y})\]
\item
  Свойства на отрицанието
  \[x\ov{x} = 0, \quad x\vee\ov{x} = x\vee 1,\quad x\oplus\ov{x} = 1\]
\item
  Закон за двойното отрицание
  \[\ov{\ov{x}} = x\]
\item
  Свойства на константите
  \[x\cdot 0 = 0, \quad x\cdot 1 = x,\quad x\vee 0 = x,\quad x\vee 1 = 1,\quad x\oplus 0 = x, \quad x\oplus 1 = \ov{x}\]
\item
  Дистрибутивни свойства
  \begin{enumerate}[]
  \item
    $x(y\vee z) = xy \vee xz$,
  \item
    $xy \vee z = (x\vee z)(y\vee z)$,
  \item
    $(x\oplus y)z = xz \oplus yz$.
  \end{enumerate}
\item
  Идемпотентентни свойства
  \[xx = x, \quad x\vee x = x\]
\item
  Свойства на отрицанието
  \[x\ov{x} = 0, \quad x\vee\ov{x} = 1, \quad x\oplus\ov{x} = 1\]
\item
  Закони на Де Морган
  \[\ov{xy} = \ov{x}\vee\ov{y}, \quad \ov{x\vee y} = \ov{x}\cdot\ov{y}\]
\end{enumerate}

\begin{problem} %% Гаврилов, стр. 30
  Проверете еквивалентни ли са формулите $\varphi$ и $\psi$ като използвате еквивалентни преобразования на формулите.
  \begin{enumerate}[a)]
  \item
    $\varphi = (x\oplus yz)\rightarrow (\overline{x}\rightarrow (y\rightarrow z))$,
    $\psi = x\rightarrow ((y\rightarrow z)\rightarrow x)$;
  \item
    $\varphi = (\overline{x}\vee \overline{y}.z)\rightarrow ((x\rightarrow y)\rightarrow (y\vee z)\rightarrow\overline{x})$,
    $\psi = (x\rightarrow y)\rightarrow(\overline{y}\rightarrow\overline{x})$;
  \item
    $\varphi = (x.\overline{y}\vee \overline{x}.z)\oplus ((y\rightarrow z)\rightarrow \overline{x}.y)$,
    $\psi = (x.(\overline{y}.\overline{z})\oplus y)\oplus z$;
  \item
    $\varphi = x\rightarrow ((\ov{x}.\ov{y}\rightarrow(\ov{x}.\ov{z}\rightarrow y))\rightarrow y).z$,
    $\psi = \ov{x.(y\rightarrow\ov{z})}$.
  \item
    $\varphi = \ov{((x\vee y) \rightarrow y.z)\vee (y\rightarrow x.z)} \vee (x\rightarrow (\ov{y}\rightarrow z))$,
    $\psi = (x\rightarrow y)\vee z$.
  \end{enumerate}
\end{problem}
\begin{solution}
  \begin{enumerate}[a)]
  \item
    $\psi =  x\rightarrow ((y\rightarrow z)\rightarrow x)\ =\ \overline{x}\vee (\overline{y\rightarrow z}\vee x)\  =\ 1$.
    \begin{align*}
      \varphi =\ & (x\oplus yz)\rightarrow (\overline{x}\rightarrow (y\rightarrow z))\\
      =\ & \overline{(x\vee yz)(\overline{x}\vee\overline{yz})} \vee x\vee \overline{y}\vee z\\
      =\ & \overline{(x\vee yz)}\vee\overline{(\overline{x}\vee\overline{yz})} \vee x\vee \overline{y}\vee z\\
      =\ & \overline{x}.\overline{yz} \vee xyz \vee x\vee \overline{y}\vee z =\ \overline{x}(\overline{y}\vee\overline{z}) \vee x\vee \overline{y}\vee z\\
      =\ & \overline{x}.\overline{y} \vee \overline{x}.\overline{z} \vee x\vee \overline{y}\vee z =\ \overline{x}.\overline{z} \vee x\vee \overline{y}\vee z\\
      =\ & \overline{(x\vee z)} \vee (x\vee z)\vee \overline{y} = 1 \vee \ov{y} = 1.
    \end{align*}
  \item
    $\psi = (x\rightarrow y)\rightarrow(\overline{y}\rightarrow\overline{x}) = 
    \ov{\ov{x}\vee y} \vee y \vee \ov{x} = x\ov{y} \vee y \vee \ov{x} = x \vee y \vee \ov{x} = 1$
    
    \begin{tabular}{l c l}
      $\varphi $ & $=$ & $(\overline{x}\vee \overline{y}.z)\rightarrow ((x\rightarrow y)\rightarrow (y\vee z)\rightarrow\overline{x}) $\\
      & $ = $ & $\ov{\overline{x}\vee \overline{y}.z} \vee \ov{x\rightarrow y} \vee \ov{y\vee z} \vee \ov{x}$ \\
      & $=$ & $x(y\vee \ov{z})\vee \ov{\ov{x}\vee y} \vee \ov{y}.\ov{z} \vee \ov{x} = $\\
      & $=$ & $xy \vee x\ov{z} \vee x.\ov{y} \vee \ov{y}.\ov{z} \vee \ov{x} =$ \\
      & $=$ & $x(y\vee\ov{y}) \vee x\ov{z} \vee \ov{y}.\ov{z} \vee \ov{x} = $\\
      & $=$ & $x \vee \ov{x} \vee x\ov{z} \vee \ov{y}.\ov{z} = 1$
    \end{tabular}
    
  \item
    $\psi = (x.(\overline{y}.\overline{z})\oplus y)\oplus z = x(y\oplus 1)(z\oplus 1) \oplus y \oplus z = xyz \oplus xy \oplus xz \oplus x \oplus y \oplus z$
    
    \begin{tabular}{l c l}
      $\varphi$ & $=$ & $(x\overline{y}\vee \overline{x}z)\oplus ((y\rightarrow z)\rightarrow \overline{x}y)$\\
      & $=$ & $(x\ov{y}\vee \ov{x}z) \oplus (\ov{\ov{y}\vee z} \vee \ov{x}y)$ \\
      & $=$ & $x\ov{y}\oplus \ov{x}z \oplus (y\ov{z} \oplus \ov{x}y)$ \\
      & $=$ & $x\ov{y}\oplus \ov{x}z\oplus \ov{x}y\ov{z} \oplus y\ov{z} \oplus \ov{x}y$ \\
      & $=$ & $xy \oplus x \oplus xz \oplus z \oplus (x\oplus 1)y(z\oplus 1) \oplus yz\oplus y \oplus xy \oplus y$ \\
      & $=$ & $x \oplus xz \oplus z \oplus (x\oplus 1)y(z\oplus 1) \oplus yz\oplus $ \\
      & $=$ & $x \oplus xz \oplus z \oplus xyz \oplus yz \oplus xy \oplus y \oplus yz\oplus $ \\
      & $=$ & $x \oplus y\oplus z \oplus xz \oplus xy \oplus xyz$ \\
     \end{tabular}
     
  %  \item
  %    $\psi = \ov{x.(y\rightarrow\ov{z})} = \ov{x}\vee yz$.
     
  %    \begin{tabular}{l c l}
  %      $\varphi$ & $ = $ & $x\rightarrow ((\ov{x}.\ov{y}\rightarrow(\ov{x}.\ov{z}\rightarrow y))\rightarrow y).z$\\
  %      & $=$ & $\ov{x} \vee ((\ov{\ov{x}.\ov{y}} \vee \ov{\ov{x}.\ov{z}}\vee y)\rightarrow y).z$\\
  %      & $=$ & $\ov{x} \vee ((x\vee y \vee x\vee z \vee y)\rightarrow y).z$\\
  %      & $=$ & $\ov{x} \vee ((x\vee y \vee z)\rightarrow y).z$\\
  %      & $=$ & $\ov{x} \vee (\ov{x\vee y \vee z}\vee y).z$\\
  %      & $=$ & $\ov{x} \vee yz$
  %    \end{tabular}

     
  % \item
  %   $\psi = (x\rightarrow y)\vee z = \ov{x}\vee y \vee z$.
    
  %   \begin{tabular}{l c l}
  %     $\varphi $ & $ = $ & $\ov{((x\vee y) \rightarrow y.z)\vee (y\rightarrow x.z)} \vee (x\rightarrow (\ov{y}\rightarrow z)) $\\
  %     & $=$ & $\ov{\ov{x}.\ov{y}\vee yz \vee \ov{y}\vee xz} \vee \ov{x}\vee y \vee z$\\
  %     & $=$& $\ov{\ov{y}\vee yz \vee xz} \vee \ov{x}\vee y \vee z$ \\
  %     & $=$ & $\ov{\ov{y}\vee z \vee xz} \vee \ov{x}\vee y \vee z$\\
  %     & $=$ & $\ov{\ov{y}\vee z} \vee \ov{x}\vee y \vee z$ \\
  %     & $=$ & $y\ov{z} \vee \ov{x}\vee y \vee z$ \\
  %     & $=$ & $\ov{x}\vee y \vee z$.
  %   \end{tabular}
\end{enumerate}
\end{solution}




\section{Полином на Жегалкин}

\begin{itemize}
\item 
  Полином на Жегалкин на 2 променливи е формула от вида:
  \[a_0\oplus a_1x_1\oplus a_2x_2  \oplus a_{12}x_1x_2  ,\]
  където $a_0,a_1,a_2,a_{12}$ приемат стойности 0 или 1.
\item
  Полином на Жегалкин на 3 променливи е формула от вида:
  \[a_0\oplus a_1x_1\oplus a_2x_2 \oplus a_3x_3 \oplus a_{12}x_1x_2 \oplus a_{13}x_1x_3 \oplus a_{23} x_2x_3 \oplus a_{123}x_1x_2x_3,\]  
  където $a_0,a_1\dots,a_{123}$ приемат стойности 0 или 1.
\item
  Полином на Жегалкин на $n$ променливи е формула от вида:
  \[a_0 \oplus \bigoplus_{1\leq i\leq n}a_i x_i\oplus \bigoplus_{1\leq i<j \leq n}a_{ij} x_ix_j\oplus\dots  \oplus a_{12\dots n} x_1x_2\dots x_n,\]
\end{itemize}

\begin{thm}
  Всяка булева функция има единствен полином на Жегалкин.
\end{thm}
\begin{proof}
  Всеки полином на Жегалкин представя различна булева функция.
  Всички полиноми на Жегалкин на $n$ променливи са $2^{2^n}$.
  Всички булеви функции на $n$ променливи са $2^{2^n}$.
\end{proof}


\begin{problem}
  По метода на неопределените коефициенти, намерете полинома на Жегалкин на функцията 
  \begin{enumerate}[a)]
  \item
    $f(x,y) = x\vee y$;
  \item
    $f(x,y,z) = x\vee y \vee z$;
  \item
    $f(x,y,z) = x\rightarrow (y \rightarrow z)$;
  \item
    $f(x,y,z) = x(y\vee\overline{z})$.
  \end{enumerate}
\end{problem}
\begin{proof}
  \begin{enumerate}[a)]
  \item
    Общият вид на $f(x,y) = a_0\oplus a_1 x \oplus a_2 y \oplus a_3 xy $.
    \[
    \begin{array}{c c c}
      a_0\oplus a_1 0 \oplus a_2 0 \oplus a_3 0 & = 0 \vee 0  &  = 0\\
      a_0\oplus a_1 1 \oplus a_2 0 \oplus a_3 0 & = 1 \vee 0  &  = 1\\
      a_0\oplus a_1 0 \oplus a_2 1 \oplus a_3 0 & = 0 \vee 1  &  = 1\\
      a_0\oplus a_1 1 \oplus a_2 1 \oplus a_3 1 & = 1 \vee 1  &  = 1\\
    \end{array}
    \]
    Следователно, $x\vee y = x\oplus y\oplus xy$.
  \end{enumerate}
\end{proof}

\begin{problem}
  Използвайки еквивалентности от вида $\overline{A} = A\oplus 1$ и $A\vee B = AB\oplus A\oplus B$, 
  намерете полинома на Жегалкин на функцията:
  \begin{enumerate}[a)]
  \item
    \marginpar{$1 \oplus x \oplus xy$}
    $f(x,y) = x\rightarrow y$;
  \item
    \marginpar{$1 \oplus xy \oplus xyz$}
    $f(x,y,z) = (x\rightarrow (y\rightarrow z))$;
  \item
    \marginpar{$x\oplus z\oplus xy\oplus xz \oplus xyz$}
    $f(x,y,z) = ((x\rightarrow y)\rightarrow z)$;
  \item
    \marginpar{$x\oplus z\oplus xy\oplus xz \oplus xyz$}
    $f(x,y,z) = (x\rightarrow (y\rightarrow z)).((x\rightarrow y)\rightarrow z)$;
  \item
    $f(x,y,z,t) = (x\rightarrow y)\rightarrow (z\rightarrow xt)$;
  \item
    $f(x,y,z,t) = x\vee (y\rightarrow ((z\rightarrow y)\rightarrow t)$;
  \item
    $f(x,y,z,t) = (x\vee y\vee z)t \vee xyz$.
  \end{enumerate}
\end{problem}
% \begin{proof}
%   \begin{enumerate}[a)]
%   \item
%     $x\rightarrow y = \overline{x}\vee y = \overline{x}\oplus y \oplus \overline{x}y = x\oplus 1 \oplus y \oplus (x\oplus 1)y = 
%     x\oplus 1 \oplus y \oplus xy \oplus y = 1 \oplus x \oplus xy$.
%   \item
%     $1 \oplus xy \oplus xyz$
%   \item
%     $x\oplus z\oplus xy\oplus xz \oplus xyz$
%   \item
%     $x\oplus z\oplus xy\oplus xz \oplus xyz$

%   \end{enumerate}
% \end{proof}

\section{Дизюнктивна нормална форма}

\begin{itemize}
\item
  \index{конюнкт}
  {\bf Конюнкт} на променливите $x_1,x_2,\dots,x_n$ представлява съждителна формула от вида 
  \[x^{\sigma_1}_1x^{\sigma_2}_2 \cdots x^{\sigma_n}_n,\]
  където $x^{\sigma_i}_i = x_i$, ако $\sigma_i = 1$ и $x^{\sigma_i}_i = \overline{x}_i$, ако $\sigma_i = 0$.
\item
  \index{дизюнктивна нормална форма}
  Една съждителна формула $\Phi(x_1,\dots,x_n)$ е в {\bf дизюнктивна нормална форма (ДНФ)}, ако
  тя представлява дизюнкция от конюнкти на някои от променливите на $\Phi$.
  Например, формулата 
  \[\Phi(x,y) = \ov{x}y \vee x\ov{y}\]
  е в дизюнктивна формална форма.
\item
  \index{съвършена дизюнктивна нормална форма}
  Една съждителна формула $\Phi(x_1,\dots,x_n)$ е в {\bf съвършена дизюнктивна нормална форма (СДНФ)}, ако
  тя е е в ДНФ и всеки конюнкт участват всичките променливи $x_1,\dots,x_n$.
  За една булева функция $f(x_1,\dots,x_n)$, можем да намерим формула $\Phi(x_1,\dots,x_n)$ в СДНФ еквивалентна на нея по следния начин:
  \[\Phi(x_1,\dots,x_n) = \bigvee_{\stackrel{(\sigma_1\dots \sigma_n) \in \{0,1\}^n}{f(\sigma_1, \dots \sigma_n) = 1}}x_1^{\sigma_1}x_2^{\sigma_2}\dots x_n^{\sigma_n}.\]
\end{itemize}

\begin{problem} %% Гаврилов стр. 50
  С помощта на еквивалентни преобразувания постройте ДНФ на булевите функции
  \begin{enumerate}[a)]
  \item
    \marginpar{$xy\overline{z} \vee \overline{x}z \vee \overline{y}z$}
    $f(x,y,z) = (\ov{x}\vee\ov{y}\vee\ov{z})\cdot(xy\vee z)$;
  \item
    \marginpar{$\overline{x}y\overline{z} \vee xyz \vee x\overline{y}z$}
    $f(x,y,z) = (\overline{x}y\oplus z)\cdot(xz\rightarrow y)$;
  \item
    $f(x,y,z) = (x\vee y\overline{z})\cdot(x\ov{y}\vee\ov{z})\cdot(\ov{xy}\vee z)$;
  \item
    $f(x,y,z,t) = (x\vee y\ov{z}.\ov{t})((\ov{x}\vee t)\oplus yz)\vee \ov{y}\cdot(z\vee \ov{x\ov{t}})$;
  \item
    $f(x,y,z,t) = (x\rightarrow y).(y\rightarrow \ov{z}).(z\rightarrow x\ov{t})$;
  \end{enumerate}
\end{problem}

\begin{problem}% Гаврилов, стр. 50, 2.12
  По дадена ДНФ на булевата функция $f$ постройте нейната СДНФ.
  \begin{enumerate}[1)]
  \item
    \marginpar{}
    $f(x,y,z) = xy\vee\ov{z}$;
  \item
    \marginpar{}
    $f(x,y,z) = \ov{x}.\ov{y} \vee y\ov{z} \vee z\ov{z}$;
  \item
    $f(x,y,z) = x\vee yz \vee \ov{x}.\ov{z}$;
  \item
    $f(x,y,z) = x\vee \ov{y}\vee \ov{x}z$;
  \item
    $f(x,y,z,t) = xy\ov{z} \vee xz\ov{t}$;
  \item
    $f(x,y,z,t) = xy \vee \ov{y}t \vee z\ov{t}$.
  \end{enumerate}
\end{problem}

\begin{problem}
  Представете в СДНФ следните булеви функции:
  \begin{enumerate}[1)]
  \item
    $f(x,y,z) = (x\vee y)\rightarrow z$;
  \item
    $f(x,y,z) = (01010001)$;
  \item
    $f(x,y,z) = (11001010)$;
  \item
    $f(x,y,z,t) = (x\rightarrow yzt)(z\rightarrow x\ov{y})$;
  \item
    $f(x,y,z,t) = (x\oplus y)(z\rightarrow \ov{y}t)$;
  \end{enumerate}
\end{problem}

\section{Класовете $T_0$ и $T_1$}
\index{$T_0, T_1$}

\begin{itemize}
\item 
  Нека $c\in\{0,1\}$. 
  Казваме, че булевата функция $f(\xn)$ запазва константата $c$, ако $f(c,c,\dots,c) = c$.
\item
  Означаваме с $T_0$ функциите, които запазват константата $0$ и с $T_1$ тези, които запазват константата $1$.
\item
  С $T^n_0$ и $T^n_1$ означаваме тези функции, които са на $n$ променливи и принадлежат на $T_0$ или $T_1$ съответно.
\end{itemize}

\begin{problem}% Гаврилов, стр. 73
  Принадлежи ли функцията $f$ на множеството $T_1 \setminus T_0$ ?
  \begin{enumerate}[a)]
  \item
    \marginpar{Да}
    $f = (x\rightarrow y)(y\rightarrow z)(z\rightarrow x)$;
  \item
    \marginpar{Да}
    $f = x\rightarrow(y\rightarrow (z\rightarrow x))$;
  \item
    \marginpar{Да}
    $f = xyz \vee \ov{x}y \vee \ov{y}$;
  \end{enumerate}
\end{problem}

\begin{problem}
  При какви $n$ функцията $f(x_1,\dots, x_n)$ принадлежи на $T_0\setminus T_1$?
  \begin{enumerate}[1)]
  \item
    $f = x_1\oplus x_2 \oplus\dots\oplus x_n$;
  \item
    $f = (\bigoplus^{n-1}_{i=1} x_ix_{i+1})\oplus x_nx_1$;
  \end{enumerate}
\end{problem}

\begin{prop}
  Класовете $T_0$ и $T_1$ са затворени, т.е. $[T_0] = T_0$ и $[T_1] = T_1$.
\end{prop}


\section{Самодвойнствени функции}

\begin{itemize}
\item 
  Нека е дадена булевата функция $f(\xn)$. Дефинираме булевата функция $f^\star(\xn)$ като
  \[f^\star(\xn) = \overline{f}(\overline{x}_1,\dots,\overline{x}_n).\]
\item
  Ще наричаме $f^\star$ {\bf двойнствена} функция на $f$.
\item
  Ако $f = f^\star$, то ще наричаме $f$ {\bf самодвойнствена} функция.
\item
  Ще означаваме с $S$ множеството от всички самодвойнствени булеви функции, а с $S^n$ тези на $n$ променливи.
\end{itemize}

\begin{prop}
  Класът на самодвойнствените функции е затворен, т.е. $[S] = S$.
  Освен това, $S \subsetneqq \Fs_2$.
\end{prop}


\begin{problem} %% Гаврилов, стр. 31, зад. 1.25
  Проверете дали функцията $g$ е двойнствена на $f$.
  \begin{enumerate}[1)]
  \item
    \marginpar{Да}
    $f = x\rightarrow y$, $g = \overline{x}.y$;
  \item
    \marginpar{Не}
    $f = (\overline{x}\rightarrow\overline{y})\rightarrow(y\rightarrow x)$, $g = (x\rightarrow y).(\overline{y}\rightarrow\overline{x})$;
  \item
    \marginpar{Да}
    $f = x.y \rightarrow z$, $g = \overline{x}.\overline{y}.z$;
  \item
    $f = (x\vee y\vee z).t\vee x.y.z$, $g = (x\vee y\vee z).t\vee x.y.z$;
  \item
    $f = xy\vee yz\vee zt\vee tx$, $g = xz\vee yt$;
  \item
    $f = (x\rightarrow y).(z\rightarrow t)$, $g = (x\rightarrow\overline{z}).(x\rightarrow t).(\overline{y}\rightarrow\overline{z}).(\overline{y}\rightarrow t)$.
  \end{enumerate}
\end{problem}

\begin{problem}
  Проверете самодвойнствена ли е $f$.
  \begin{enumerate}[a)]
  \item
    \marginpar{Не}
    $f(x,y) = x\vee y$;
  \item
    \marginpar{Не}
    $f(x,y) = x\rightarrow y$;
  \item
    \marginpar{Не}
    $f(x,y) = x\oplus y$;
  \item
    \marginpar{Да}
    $f_4(x,y,z) = xy\vee yz\vee zx$;
  \item
    \marginpar{Да}
    $f_5(x,y,z) = x\oplus y\oplus z\oplus 1$;
  \item
    \marginpar{Да}
    $f_6(x,y,z) = xyz\oplus xy\ov{z}\oplus yz\oplus xz$.
  \item
    \marginpar{Не}
    $f_7(x,y,z) = xyz\oplus xy\oplus yz\oplus xz$;
  \item
    \marginpar{Не}
    $f(x,y,z) = (x\rightarrow y)\oplus (y\rightarrow z)\oplus (y\rightarrow x)$;
  \item
    \marginpar{Не}
    $f(x,y,z) = (x\rightarrow y)\oplus (y\rightarrow z)\oplus (z\rightarrow x)\oplus z$;
  \end{enumerate}
\end{problem}
\begin{proof}
  \begin{table}[H]
    \begin{subtable}{0.5\textwidth}
      \begin{tabular}[b]{|c||c|c|c|}
        \hline
        $xz$ & а) & б) & в)\\
        \hline
        $00$ & $0$ & $1$ & $0$ \\
        \hline
        $01$ & $1$ & $1$ & $1$\\
        \hline
        \hline
        $10$ & $1$ & $0$ & $1$\\
        \hline
        $11$ & $1$ & $1$ & $0$\\
        \hline
      \end{tabular}
    \end{subtable}
    \begin{subtable}{0.5\textwidth}
      \begin{tabular}[b]{|c||c|c|c|c|c|}
        \hline
        $xyz$ & г) & д) & е) & ж) & з)\\
        \hline
        $000$ & $0$ & $1$ & $0$ & $0$ & $1$\\
        \hline
        $001$ & $0$ & $0$ & $0$ & $0$ & $1$\\
        \hline
        $010$ & $0$ & $0$ & $0$ & $0$ & $1$\\
        \hline
        $011$ & $1$ & $1$ & $1$ & $1$ & $0$\\
        \hline
        \hline
        $100$ & $0$ & $0$ & $0$ & $0$ & $0$\\
        \hline
        $101$ & $1$ & $1$ & $1$ & $1$ & $0$\\
        \hline
        $110$ & $1$ & $1$ & $1$ & $1$ & $0$\\
        \hline
        $111$ & $1$ & $0$ & $1$ & $0$ & $1$\\
        \hline
      \end{tabular}
    \end{subtable}
  \end{table}
\end{proof}


\begin{problem}
  Проверете дали функцията $f$ е самодвойнствена, ако е зададена векторно:
  \begin{enumerate}[1)]
  \item
    \marginpar{Да}
    $\alpha_f = (01001101)$;
  \item
    \marginpar{Не}
    $\alpha_f = (01100110)$;
  \item
    $\alpha_f = (1100 1001 0110 1100)$;
  \item
    $\alpha_f = (1110 0111 0001 1000)$;
  \item
    $\alpha_f = (1100 0011 0011 1100)$;
  \item
    $\alpha_f = (1001 0110 1001 0110)$;
  \item
    $\alpha_f = (1100 0011 1010 0101)$;
  \end{enumerate}
\end{problem}

\begin{problem}
  Заменете $-$ в $\chi_f$ с $0$ или $1$ за да получите характеристичен вектор на самодвойнствена функция.\\
  \begin{inparaenum}[a)]
  \item
    $\chi_f = (1-0-)$;
  \item
    $\chi_f = (01-0-0--)$;
  \item
    $\chi_f = (--01--11)$;
  \end{inparaenum}
\end{problem}

\section{Линейни функции}
\index{линейна!булева функция}

\begin{itemize}
\item 
  Всяка булева функция $f(\xn)$ с полином на Жегалкин от вида 
  \[a_0\oplus a_1x_1 \oplus a_2x_2 \dots\oplus a_nx_n\] наричаме {\bf линейна}.
\item
  Ще означаваме с $L$ множеството от всички линейни булеви функции, а с $L^n$ тези на $n$ променливи.
\end{itemize}

\begin{problem}
  \marginpar{Отг. $2^n$}
  Колко са всички линейни булеви функции на $n$ променливи?
\end{problem}


\begin{prop}
  Класът на линейните функции е затворен, т.е. $[L] = L$.
  Освен това, $L \subsetneqq \Fs_2$.
\end{prop}


\begin{problem}
  Линейна ли е функцията $f$ с характеристичен вектор $\chi_f = (1001011010010110)$?
\end{problem}

\begin{problem}
  Заменете $-$ в $\chi_f = (-110---0)$ с $0$ или $1$, така че да получите $f$ линейна.
\end{problem}


\begin{problem}
  Проверете дали $f$ е линейна функция.
  \begin{enumerate}
  \item
    \marginpar{Не}
    $f = x\rightarrow y$;
  \item
    \marginpar{Да}
    $f = \ov{x\rightarrow y}\oplus \ov{x}y$;
  \item
    \marginpar{Не}
    $f = xy\vee \ov{x}.\ov{y}\vee z$;
  \item
    \marginpar{Не}
    $f = xy\ov{z}\vee x\ov{y}$;
  \item
    \marginpar{Да}
    $f = (x\vee yz)\oplus xyz$;
  \item
    $f = (x\vee yz)\oplus \ov{x}yz$;
  \item
    $\chi_f = (1100 0011)$;
  \item
    $\chi_f = (1001 0110 0110 1001)$;
  \end{enumerate}
\end{problem}

\begin{problem}
  Заменете $-$ в $\chi_f$ с $0$ или $1$, така че да получите $f$ линейна.
  \begin{enumerate}[a)]
  \item
    $\chi_f = (10-1)$;
  \item
    $\chi_f = (100-0---)$;
  \item
    $\chi_f = (-001--1-)$;
  \item
    $\chi_f = (11-0---1)$;
  \item
    $\chi_f = (-0-1--00)$;
  \item
    $\chi_f = (--10----0--1-110)$;
  \end{enumerate}
\end{problem}
\begin{proof}
  а) $(1001)$; б) $f = 1\oplus x \oplus y\oplus z$; в) $f = 1\oplus x\oplus y\oplus z$ ;
  г) $f = 1\oplus x\oplus y$; д) $f = x\oplus y$;
\end{proof}


\section{Монотонни функции}

\begin{itemize}
\item 
  Нека $\alpha$ и $\beta$ са два бинарни вектора с равна дължина.
  \marginpar{Това не е лексикографската наредба!}
  Дефинираме релацията $\preceq$ между тях по следния начин.
  \[\alpha \preceq \beta \iff \abs{\alpha} = \abs{\beta}\wedge (\forall i \leq \abs{\alpha})[a_i \leq b_i].\]
\item
  Булевата фунция $f(\xn)$ наричаме {\bf монотонна}\index{монотонна!функция}, ако 
  \[(\forall \alpha,\beta\in J^n_2 )[\alpha\preceq\beta \rightarrow f(\alpha) \leq f(\beta)].\]  
\item
  Ще означаваме с $M$ множеството от всички монотонни булеви функции, а с $M^n$ тези на $n$ променливи.
\end{itemize}

\begin{prop}
  Класът на монотонните функции е затворен, т.е. $[M] = M$.
  Освен това, $M \subsetneqq \Fs_2$.
\end{prop}


\begin{problem}
  Проверете монотонни ли са функциите:
  \begin{enumerate}[a)]
  \item
    \marginpar{Да}
    $f(x,y) = x\rightarrow (y\rightarrow x)$;
  \item
    \marginpar{Не}
    $f(x,y) = x\rightarrow (x\rightarrow y)$;
  \item
    \marginpar{Да}
    $f(x,y) = (x\oplus y)xy$;
  \item
    \marginpar{Да}
    $f(x,y,z) = xy\oplus yz \oplus zx$;
  \item
    \marginpar{Не}
    $f(x,y,z) = xy\oplus yz \oplus zx \oplus x$;
  \end{enumerate}
\end{problem}

\begin{problem}
  За немонотонните функции $f$, намерете съседни $\alpha$, $\beta$, такива че
  $\alpha \prec \beta$ и $f(\alpha) > f(\beta)$.
  \begin{enumerate}[a)]
  \item
    \marginpar{Отг. $\alpha = (010)$, $\beta = (110)$}
    $f = xyz \vee \ov{x}y$;
  \item
    \marginpar{Отг. $\alpha = (010)$, $\beta = (110)$}
    $f = x\oplus y\oplus z$;
  \item
    \marginpar{Отг. $\alpha = (110)$, $\beta = (111)$}
    $f = xy\oplus z$;
  \item
    \marginpar{Отг. $\alpha = (010)$, $\beta = (011)$}
    $f = x\vee y\ov{z}$;
  \item
    \marginpar{Отг. $\alpha = (0111)$, $\beta = (1111)$}
    $f = xz\oplus yt$;
  \item
    \marginpar{Отг. $\alpha = (1110)$, $\beta = (1111)$}
    $f(x,y,z,t) = (xyt\rightarrow yz)\oplus t$;
  \end{enumerate}
\end{problem}
% \begin{proof}
%   \begin{enumerate}[a)]
%   \item
%     $\alpha = (010)$, $\beta = (110)$;
%   \item
%     $\alpha = (010)$, $\beta = (110)$;
%   \item
%     $\alpha = (110)$, $\beta = (111)$;
%   \item
%     $\alpha = (010)$, $\beta = (011)$;
%   \item
%     $\alpha = (0111)$, $\beta = (1111)$;
%   \item
%     $\alpha = (1110)$, $\beta = (1111)$;
%   \end{enumerate}
% \end{proof}


\section{Пълнота и затворени класове}

\begin{dfn}
Нека $F\subseteq \Fs_2$ е множество от булеви функции. С
индукция дефинираме следната редица за всяко $n \in \Nat$:
\begin{align*}
  & F_0 = F\cup \{I^m_k \mid m,k\in\Nat, 1\leq k\leq m\}\\
  & F_{n+1} = F_n \cup \{h \mid (\exists f, g_1\dots g_m \in F_n)[h(x_1\dots x_k) =  f(g_1(x_1\dots x_k), \dots, g_m(x_1\dots x_k)]\},
\end{align*}

Затварянето на $F$ по отношение на суперпозиция наричаме
множеството:
\[[F] = \bigcup_{n\in \mathbb{N}}F_n.\]

\end{dfn}

% Така множеството $[F]$ се задава с индукция от базово множество
% $F\cup \{I^n_k \mid 1\leq k\leq n \}$ по правилото суперпозиция. От
% Тема 1, знаем, че това е най-малкото множество $X$, което съдържа
% базовото множество и е затворено относно суперпозиция.


\begin{dfn}
  Нека $F\subseteq \Fs_2$ е множество от булеви функции. 
  $F$ е пълно множество, ако $[F] = \Fs_2$.
  $F$ се нарича базис, ако не съществува $G \subsetneqq F$, за което $[G] = \Fs_2$.
\end{dfn}

\begin{thm}[Бул]
  Множеството $\{x\vee y,\ov{x},x\wedge y\}$ е пълно.
\end{thm}
\begin{proof}
  Ще докажем, че за всяка булева функция $f \in \Fs_2$, $f \in [\{x\vee y, \ov{x}, x \wedge y\}]$.
  Ще разгледаме два случая.
  
  Нека $f = \bf{0}$. Тогава $f(x_1,\dots,x_n) \equiv x_1\wedge\ov{x}_1$.
  
  Нека $f \neq \bf{0}$. Тогава лесно се съобразява, че
  \marginpar{$x^1 = x$, $x^0 = \ov{x}$}
  \[f(x_1,\dots,x_n) \equiv \bigvee_{\stackrel{a_1,\dots,a_n:}{f(a_1,\dots,a_n) = 1}} x^{a_1}_1\dots x^{a_n}_n\]
\end{proof}

\index{пълно множество}
\begin{thm}[Критерий за пълнота на Пост-Яблонский]
  Нека $P\subseteq \Fs_2$ е непразно множество от булеви функции. Множеството $P$ е {\em пълно} тогава и само тогава, когато то {\em не} е подмножество на 
  нито едно от множествата $T_0,T_1,S,M,L$.
\end{thm}
\begin{proof}
  Първо, нека $P$ е пълно множество, т.е. $[P] = \Fs_2$.
  Ако допуснем, че съществува $K \in \{T_0,T_1,S,M,L\}$, за което $P \subseteq K$, то
  \[\Fs_2 = [P] \subseteq [K] = K,\]
  от което следва, че $\Fs_2 = K$, което е противоречие.
  
  Нека сега имаме, че за всяко $K \in \{T_0,T_1,S,M,L\}$, $P \not\subseteq K$.
  Ще докажем, че $\{\ov{x}, x\vee y\} \subseteq [P]$. Тогава, от теоремата на Бул $\Fs_2 = [\{\ov{x}, x\vee y\}] \subseteq [P]$,
  ще следва, че $[P] = \Fs_2$.
  Да фиксираме функциите:
  \[f_0 \in P\setminus T_0,\ f_1 \in P \setminus T_1,\ f_S \in P\setminus S,\ f_M \in P\setminus M,\ f_L \in P\setminus L.\]
  
  Можем да приемем, че функциите $f_0$ и $f_1$ са едноаргументни,
  защото ако например $f_0$ е $n$-аргументна, то ще вземем 
  $g_0(x) = f_0(x,\dots,x) = f_0(I^1_1(x),\dots,I^1_1(x)) \in [P]$.
  Имаме, че $f_0(0) = 1$ и $f_1(1) = 0$. 
  Ще разгледаме четири случая за другите стойности на аргументите:
  \begin{enumerate}[1)]
  \item 
    ако $f_0(1) = 0$ и $f_1(0) = 0$, тогава $f_0 \equiv \ov{x}$, $f_1 \equiv {\bf 0}$ и $f_0(f_1(x)) = 1$;
  \item
    ако $f_0(1) = 0$ и $f_1(0) = 1$, тогава $f_0 \equiv f_1\equiv \ov{x}$;
  \item
    ако $f_0(1) = 1$ и $f_1(0) = 0$, тогава $f_0 \equiv {\bf 1}$ и $f_1 \equiv {\bf 0}$;
  \item
    ако $f_0(1) = 1$ и $f_1(0) = 1$, тогава $f_0 \equiv {\bf 1}$, $f_1 \equiv \ov{x}$ и $f_1(f_0(x)) = 0$.
  \end{enumerate}
  В случаите $1)$, $3)$, $4)$ получихме, че ${\bf 0}, {\bf 1} \in [P]$.
  Ще докажем, че и в случая $2)$ също имаме, че ${\bf 0}, {\bf 1} \in [P]$.
  За целта ще трябва да разгледаме и функцията $f_S \not\in S$.
  За нея знаем, че съществуват стойности на аргументите ѝ $a_1,\dots,a_n$, такива че:
  \[f_S(a_1,\dots,a_n) = f_S(\ov{a}_1,\dots,\ov{a}_n).\]
  Да разгледаме едноместната функция $g_S$, дефинирана като:
  \[g_S(x) = f_S(x^{a_1},\dots,x^{a_n}) \in [P].\]
  \begin{align*}
    g_S(0) & = f_S(0^{a_1},\dots,0^{a_n}) & (0^0 = 1, 0^1 = 0)\\
    & = f_S(\ov{a}_1,\dots,\ov{a}_n) = f_S(a_1,\dots,a_n) & (\text{защото } f_S\not\in S)\\
    & = f_S(1^{a_1},\dots,1^{a_n}) & (1^0 = 0, 1^1 = 1)\\
    & = g_S(1).
  \end{align*}
  Следователно $g_S$ е константа функция. 
  Понеже сме в случая $2)$, то $\ov{x} \in [P]$ и следователно $\ov{g}_S$ е другата константна функция.
  Заключаваме, че във всичките четири случая получаваме, че ${\bf 0}, {\bf 1} \in [P]$.
  
  Сега ще докажем, че имаме $\ov{x} \in [P]$.
  Да разгледаме $f_M \not\in M$. Това означава, че съществуват вектори $\alpha  \prec \beta$, които се различават
  само в една стойност и $f_M(\alpha) = 1$, $f_M(\beta) = 0$.
  Имаме, че  $\alpha = (a_1,\dots,a_{k-1},0,a_{k+1},\dots,a_n)$ и $\beta = (a_1,\dots,a_{k-1},1,a_{k+1},\dots,a_n)$.
  Да разгледаме едноместната булева функция $g_M$, дефинираната като:
  \[g_M(x) = f_M(a_1,\dots,a_{k-1},x,a_{k+1},\dots,a_n).\]
  Понеже вече доказахме, че ${\bf 0},{\bf 1} \in [P]$, то $g_M \in [P]$.
  Лесно се вижда, че $g_M(x) = \ov{x}$. 

  Остава да докажем, че $xy \in [P]$.
  Да разгледаме $f_L \not\in L$. Това означава, че $f_L$ има следния вид:
  \[f_L(x_1,\dots,x_n) = a_0 \oplus a_1x_1\oplus a_2x_2 \dots\oplus a_n x_n\oplus \dots \oplus a_lx_{i_1}x_{i_2}\dots x_{i_k}\oplus\dots \]
  Нека $x_{i_1}x_{i_2}\dots x_{i_k}$ е първият нелинеен член в записа на $f_L$.
  Да разгледаме двуместната булева функция $g_L$, дефинирана като:
  \[g_L(x_{i_1},x_{i_2}) = a_0 \oplus a_{i_1}x_{i_1} \oplus a_{i_2}x_{i_2} \oplus x_{i_1}x_{i_2},\]
  т.е. тя е получена от $f_L$ като $x_{j} = 1$ за $j \in \{i_3,\dots,i_k\}$ и $x_j = 0$ за $j \not\in\{i_1,i_2,\dots,i_k\}$.
  
  Можем да приемем, че $a_0 = 0$, защото ако $a_0 = 1$, то ще разгледаме функцията $g^\prime_L(x_{i_1},x_{i_2}) = \ov{g}_L(x_{i_1},x_{i_2}) \in [P]$.
  За нея, $g^\prime(0,0) = a_0 = 0$.
  И така, нека $g_L$ има вида
  \[g_L(x,y) = ax\oplus by \oplus xy.\]
  Отново трябва да разгледаме четири случая за стойностите на $a$ и $b$.
  \begin{itemize}
  \item 
    ако $a = 0$, $b = 0$, тогава $g_L(x,y) = xy$;
  \item 
    ако $a = 0$, $b = 1$, тогава $g_L(x,y) = y \oplus xy$ и 
    \[g_L(\ov{x},y) = y\oplus (x\oplus 1)y = y \oplus y \oplus xy = xy;\]
  \item 
    ако $a = 1$, $b = 0$, тогава $g_L(x,y) = x\oplus xy$ и
    \[g_L(x, \ov{y}) = x\oplus x(y\oplus 1) = xy;\]
  \item 
    ако $a = 1$, $b = 1$, тогава $g_L(x,y) = x \oplus y \oplus xy$ и
    \[\ov{g}_L(\ov{x},\ov{y}) = xy.\]
  \end{itemize}
  Във всички случаи получихме, че $xy \in [P]$.
\end{proof}

\begin{problem} %Гаврилов, стр. 83, зад. 6.1
  Пълна ли е системата от функции?
  \begin{enumerate}[1)]
  \item
    $A = \{xy, x\vee y, x\oplus y\oplus z\oplus 1\}$;
  \item
    $A = \{1, xy(x\oplus z)\}$;
  \item
    $A = \{x\rightarrow y, x\oplus y\}$;
  \item
    $A = \{0, \ov{x}, x(y\oplus z)\oplus yz\}$;
  \item
    $A = \{x\rightarrow y, \ov{x}\rightarrow \ov{y}x, x\oplus y\oplus z, 1\}$;
  \item
    $A = \{\ov{y}\rightarrow\ov{x}, \ov{x}\rightarrow \ov{y}x, x\oplus y\oplus z, 0\}$;
  \item
    $A = \{\ov{y}\rightarrow \ov{x}z, (y\vee \ov{x} \rightarrow x, x\oplus y\oplus z, 1\}$;
  \item
    $A = \{x\oplus z \oplus 1, x \to \ov{y}, x\oplus (y \vee z) \oplus 1\}$;
  \item
    $A = \{1,\ov{x}, x(y\leftrightarrow z)\oplus\ov{x}(y\oplus z), x\leftrightarrow y\}$;
  \item
    $A = \{\ov{x}, x(y\leftrightarrow z) \leftrightarrow yz, x \oplus y \oplus z\}$;
  \item
    $A = \{\ov{x}, x(y\leftrightarrow z) \leftrightarrow (y\vee z), x \oplus y \oplus z\}$;
  \item
    $A = \{\chi_{f_1} = (0110), \chi_{f_2} = (1100 0011), \chi_{f_3} = (1001 0110)\}$;
  \item
    $A = \{\chi_{f_1} = (11), \chi_{f_2} = (00), \chi_{f_3} = (0011 0101)\}$;
  \end{enumerate}
\end{problem}
\begin{solution}
  \begin{enumerate}[1)]
  \item
    \begin{tabular}[b]{|c|c|c|c|c|c|}
      \hline
      & $T_0$ & $T_1$ & $L$ & $S$ & $M$\\
      \hline
      $xy$ & $+$ & $+$ & $-$ & $-$ & $+$\\
      \hline
      $x\vee y$ & $+$ & $+$ & $-$ & $-$ & $+$\\
      \hline
      $x\oplus y\oplus z\oplus 1$ & $-$ & $-$ & $+$ & $+$ & $-$ \\
      \hline
    \end{tabular}
  \item
    \begin{tabular}[b]{|c|c|c|c|c|c|}
      \hline
      & $T_0$ & $T_1$ & $L$ & $S$ & $M$\\
      \hline
      $1$ & $-$ & $+$ & $+$ & $-$ & $+$\\
      \hline
      $xy(x\oplus z)$ & $+$ & $-$ & $-$ & $-$ & $-$\\
      \hline
    \end{tabular}
  \item
    \begin{tabular}[b]{|c|c|c|c|c|c|}
      \hline
      & $T_0$ & $T_1$ & $L$ & $S$ & $M$\\
      \hline
      $x\rightarrow y$ & $-$ & $+$ & $-$ & $-$ & $+$\\
      \hline
      $x\oplus y$ & $+$ & $-$ & $+$ & $-$ & $-$\\
      \hline
    \end{tabular}
  % \item
  %   \begin{tabular}[b]{|c|c|c|c|c|c|}
  %     \hline
  %     & $T_0$ & $T_1$ & $L$ & $S$ & $M$\\
  %     \hline
  %     $0$ & $+$ & $-$ & $+$ & $-$ & $+$\\
  %     \hline
  %     $x\oplus y$ & $+$ & $-$ & $+$ & $-$ & $-$\\
  %     \hline
  %     $x\rightarrow y$ & $-$ & $+$ & $-$ & $-$ & $-$\\
  %     \hline
  %     $xy \sim xz$ & $-$ & $+$ & $-$ & $-$ & $-$\\
  %     \hline
    % \end{tabular}
  \end{enumerate}
\end{solution}


\begin{problem} % Гаврилов, стр. 83
  Проверете пълно ли е множеството от булеви функции:
  \begin{enumerate}[a)]
  \item
    \marginpar{Да}
    $A = (S\cap M)\cup(L\setminus M)$;
  \item
    $A = ((L\cap M)\setminus T_1)\cup (S\cap T_1)$;
  \item
    $A = (L\cap M)\cup (S\setminus T_0)$;
  \item
    $A = (L\cap T_1)\cup (S\cap M)$;
  \item
    $A = (M\setminus S)\cup(L\cap S)$;
  \item
    $A = (M\setminus T_0)\cup (L\setminus S)$;
  \item
    $A = (M\setminus T_0) \cup (S\setminus L)$.
  \end{enumerate}
\end{problem}
\begin{solution}
  Във всяка една от задачите трябва да проверим дали
  $A \not\subseteq T_0$, 
  $A \not\subseteq T_1$, 
  $A \not\subseteq S$, 
  $A \not\subseteq M$ и 
  $A \not\subseteq L$.
  \begin{enumerate}[a)]
  \item 
    Нека $A = (S \cap M) \cup (L\setminus M)$.
    \begin{itemize}
    \item 
      Да разгледаме функцията 
      $f(x) = x \oplus 1$.
      Лесно се съобразява, че $f \in L\setminus M$, откъдето  следва, че $f \in A$.
      Обаче ние имаме, че $f \not\in T_0, T_1, M$.
      Следователно, $A \not\subseteq T_0, T_1, M$.
    \item
      Да разгледаме $g(x,y) = x\oplus y \oplus 1$.
      За нея имаме, че $g \in A$, защото $g \in L\setminus M$ и освен това $g \not\in S$.
    \item
      Остана да проверим, че $A \not\subseteq L$.
      Да разгледаме 
      \[h(x,y,z) = xy\oplus yz \oplus xz.\]
      Тогава $h \in A$, защото $h \in S\cap M$.
      Ясно е, че $h \not\in L$.
    \end{itemize}
  \item
    Нека $A = ((L\cap M)\setminus T_1)\cup (S\cap T_1)$.
    \begin{itemize}
    \item 
      Ако една функция е монотонна, но не запазва $1$-цата, то тогава 
      със сигурност тази функция е константата $0$, т.е.
      \[(L \cap M)\setminus T_1 = \{0\}.\]
      % $0 \not\in S$ и оттук $A \not\subseteq S$.
    \item
      Ако $f \in S \cap T_1$, то това означава $f(1,\dots,1) = 1$ и $f(0,\dots,0) = 0$.
      Следователно, $S \cap T_1 \subseteq T_0$.
      Получаваме, че $A =  \{0\} \cup (S \cap T_1) \subseteq T_0$.
    \end{itemize}    
    Следователно $A$ не е пълен клас.
  \item
    Нека $A = (L\cap M)\cup (S\setminus T_0)$.
    \begin{itemize}
    \item
      $0 \in L \cap M$ и следователно $A \not\subseteq T_1$;
    \item
      $1 \in L \cap M$ и следователно $A \not\subseteq T_0$;
    \item
      И двете константи са в $L \cap M$, но както знаем, те  не са самодвойнствени.
      Следователно $A \not\subseteq S$.
    \item
      Да разгледаме
      \[h(x,y,z) = xy\oplus xz \oplus yz \oplus 1.\]
      Лесно се съобразява, че $h \in S \setminus T_0$, но $h \not\in L$.
      Следователно, $A \not\subseteq L$.
      Освен това, $h \not \in M$. Следователно, $A \not\subseteq M$.
    \end{itemize}
  \item
    Нека $A = (L \cap T_1) \cup (S \cap M)$.
    Ако $f \in S \cap M$, то $f$ не е константа и $f(1,\dots,1) = 1$.
    Следователно, $f \in T_1$.
    Получаваме, че $S \cap M \subseteq T_1$.
    Заключаваме, че $A \subseteq T_1$.
  \item
    Нека $A = (M\setminus S)\cup(L\cap S)$.
    \begin{itemize}
    \item 
      $x\oplus 1 \in L \cap S$, но $x \oplus 1 \not\in T_0, T_1$.
      Следователно, $A \not\subseteq T_0,T_1$.
      Освен това, $x\oplus 1 \not\in M$.
      И така, $A \not\subseteq M$.
    \item
      Нека $f(x,y) \equiv x \vee y \equiv xy \oplus x \oplus y \oplus 1$.
      Имаме, че $f \in M \setminus S$ и $f \not\in L$.
      Следователно, $A \not\subseteq L$ и $A \not\subseteq S$.
    \end{itemize}
  \item
    Нека $A = (M \setminus T_0) \cup (L \setminus S)$.
    Имаме, че $M \setminus T_0 = \{1\}$.
    Следователно, $A \subseteq L$.
  \item
    Нека $A = (M \setminus T_0) \cup (S \setminus L)$.
    \begin{itemize}
    \item 
      Имаме, че $1 \in M\setminus T_0$.
      Следователно, $A \not\subseteq T_0$ и $A \not\subseteq S$.
    \item
      Нека $h(x,y,z) \equiv xy \oplus xz \oplus yz \oplus 1$.
      Тогава $h \in S\setminus L$ и $h \not\in T_1$, $h \not \in M$.
      Заключаваме, че $A \not\subseteq T_1, M, L$.
    \end{itemize}
  \end{enumerate}
\end{solution}

\begin{problem} % Гаврилов, стр. 84
  Проверете дали системата от функции $A$ е базис?
  \begin{enumerate}[a)]
  \item
    \marginpar{Не}
    $A = \{x\rightarrow y, x\oplus y, x\vee y\}$;
  \item
    \marginpar{Да}
    $A = \{x\oplus y\oplus z, x\vee y, 0, 1\}$;
  \item
    \marginpar{Не, $A \subseteq T_1$}
    $A = \{x\oplus y\oplus yz, x \oplus y \oplus 1\}$;
  \item
    \marginpar{Не}
    $A = \{xy \vee z, xy \oplus z, xy \iff z\}$;
  \end{enumerate}
\end{problem}

\begin{problem}
  Намерете всички базиси на класа $A$, където:
  \begin{enumerate}[a)]
  \item 
    $A = \{1, \ov{x}, xy(x\oplus y), x \oplus y \oplus xy \oplus yz \oplus zx\}$;
  \item
    $A = \{0, x\oplus y, x \to y, xy \iff xz\}$;
  \item
    $A = \{0,1,x\oplus y \oplus z, xy \oplus zx \oplus yz, xy \oplus z, x \vee y\}$;
  \item
    $A = \{xy, x\vee y, xy\vee z, x\oplus y, x \to y\}$.
  \end{enumerate}
\end{problem}

\begin{problem}
  Намерете броя на булевите функции на $n$ променливи, които принадлежат на следните класове:
  \begin{enumerate}[a)]
  \item
    \marginpar{$2^{2^n-1}$}
    $T_0$, $T_1$;
  \item
    \marginpar{$2^{2^n-2}$}
    $T_0 \cap T_1$
  \item
%    \marginpar{$3.2^{2^n-2}$}
    $T_0 \cup T_1$
  \item
 %   \marginpar{$2^{2^n-2}$}
    $T_0 \setminus T_1$;
  \item
    \marginpar{$2^{2^{n-1}}$}
    $S$;
  \item
    $T_0 \cap S$, $T_1 \cap S$;
  \item
    $T_0 \cap T_1 \cap S$;
  \item
    $S \setminus T_0$, $S \setminus (T_0 \cap T_1)$, $S \setminus (T_0 \cup T_1)$;
  \item
    \marginpar{$2^{n+1}$}
    $L$;
  \item
    $T_0 \cap L$, $T_1 \cap L$;
  \item
    $T_0 \cap T_1 \cap L$;
  \end{enumerate}
\end{problem}



%%% Local Variables: 
%%% mode: latex
%%% TeX-master: "discrete-math"
%%% End: 

% \chapter{Езици и автомати}

\section{Езици, които не са регулярни}
\begin{lemma}[за разрастването (регулярни езици)]
  % \index{лема за разрастването!регулярни езици}
  % \label{lem:pumping-reg}
  % \marginpar{На англ.\\ Pumping Lemma}
  Нека $\Ls$ да бъде безкраен регулярен език.
  Съществува число $n\geq 1$, зависещо само от $\Ls$, 
  за което за всяка дума $\alpha\in \Ls, \abs{\alpha}\geq n$ може да 
  бъде записана във вида $\alpha = xyz$ и 
  \begin{enumerate}
  \item
    $|y|\geq 1$;
  \item
    $|xy|\leq n$;
  \item
    % \marginpar{$i = 0\ \rightarrow\ xz \in \Ls$}
    $(\forall i\in\N)[xy^iz \in \Ls]$.
  \end{enumerate}
\end{lemma}

\begin{crl}
  Регулярният език $\Ls$, 
  разпознаван от КДА $M$ е непразен тoчно тогава, когато съдържа дума $\alpha, \abs{\alpha} \leq \abs{Q}$.
\end{crl}

\begin{problem}
  \marginpar{$c^+\{a^nb^n\mid n\in\N\}\cup (a\vert b)^\star$}
  Да се даде пример за език $L$, който {\bf не} е регулярен, но удовлетворява
  лемата за разрастването.
\end{problem}


% \section{Регулярни езици}
% \begin{problem}
%   Нека $\Sigma = \{0,1\}$.  Проверете дали $L$ е регулярен, където
%   \begin{enumerate}[1)]
%   \item
%     $L_1 = \{0^i1^i\ \mid\ i\geq 0\}$;
%   \item
%     $L_2 = \{0^i1^j\ \mid\ i > j\}$;
%   \item
%     $L_3 = \{0^{2n}\ \mid\ i\geq 1\}$;
%   \item
%     $L_4 = \{0^1m1^n0^{m+n}\ \mid\ m\geq 1\ \&\ n\geq 1\}$;
%   \item
%     $L_5 = \{0^n\ \mid\ n\mbox{ е просто }\}$;
%   \item
%     $L_6 = \{w\mid w\in\{0,1\}^\star\mbox{ има равен брой нули и единици}\}$;
%   \item
%     $L_7 = \{ww\mid w\in\{0,1\}^\star\}$;
%   \item
%     $L_8 = \{1^{n^2}\mid n\geq 0\}$;
%   \item
%     $L_{9} = \{0^n1^n2^n\mid n\geq 0\}$;
%   \item
%     $L_{10} = \{www\mid w\in \{0,1\}^\star\}$;
%   \item
%     $L_{11} = \{0^{2^n}\mid n\geq 0\}$;
%   \item
%     $L_{12} = \{0^m1^n\mid n\neq m\}$;
%   \end{enumerate}
% \end{problem}

\begin{problem}
  Нека $\Sigma = \{a,b\}$.  Проверете дали $L$ е регулярен, където
  \begin{enumerate}[1)]
  \item
    $L = \{\alpha^R \mid \alpha \in L_0\}$, където $L_0$ е регулярен;
  \item
    $L_1 = \{a^ib^i\ \mid\ i\geq 0\}$;
  \item
    $L_2 = \{a^ib^j\ \mid\ i > j\}$;
  \item
    $L_3 = \{a^{2n}\ \mid\ n\geq 1\}$;
  \item
    $L_4 = \{a^mb^na^{m+n}\ \mid\ m\geq 1\ \&\ n\geq 1\}$;
  \item
    $L_5 = \{a^n\ \mid\ n\mbox{ е просто число}\}$;
  \item
    $L = \{a^{n.m}\mid n,m\mbox{ са прости числа}\}$;
  \item
    $L_6 = \{w\mid w\in\{a,b\}^\star\mbox{ има равен брой нули и единици}\}$;
  \item
    $L_7 = \{ww\mid w\in\{a,b\}^\star\}$;
  \item
    $L_8 = \{ww^R\mid w\in\{a,b\}^\star\}$;
  \item
    $L_9= \{a^{n^2}\mid n\geq 0\}$;
  \item
    $L_{10} = \{a^nb^nc^n\mid n\geq 0\}$;
  \item
    $L_{11} = \{www\mid w\in \Sigma^\star\}$;
  \item
    $L_{12} = \{a^{2^n}\mid n\geq 0\}$;
  \item
    $L_{13} = \{a^mb^n\mid n\neq m\}$;
  \item
    $L_{14} = \{a^{n!}b^{n!}\mid n\neq 1\}$;
  \item
    $L_{15} = \{a^{f_n} \mid f_0 = f_1 = 1\ \&\ f_{n+2} = f_{n+1} + f_{n}\}$;
  \item
    $L = \{\alpha \in \{a,b\}^\star \mid \abs{n_a(\alpha) - n_b(\alpha)} \leq 2\}$;
  \item
    $L = \{\alpha \in \{a,b\}^\star \mid \alpha = vuv\ \&\ \abs{u} \leq \abs{v}\}$;
  \item
    $L = \{\alpha \in \{a,b\}^\star \mid \alpha = uvv^R\ \&\ \abs{u} \leq \abs{v}\}$;
  \item
    $L = \{c^ka^nb^m \mid k,m,n > 0\ \&\ n \neq m\}$;
  \item
    $L = \{c^ka^nb^n \mid k > 0\ \&\ n \geq 0\}\cup\{a,b\}^\star$;
  \item
    $L = \{c^ka^nb^m\mid k,n,m \in \N\ \&\ k = 1\implies m = n\}$; % p \geq 2, не става с p = 1
  \end{enumerate}
\end{problem}
% \begin{proof}
%   \begin{enumerate}[1)]
%   \item
%     Разгледайте $w = a^pb^p$.
%   \item
%     Разгледайте $w = a^{p+1}b^p$.
%   \item
%     Езикът е регулярен.
%   \item
%     Подобно на 1) се доказва, че езикът не е регулярен.
%   \item
%     Да допуснем, че $L_5$ е регулярен и нека $w \in L_5$, така че $\abs{w} > p+1$.
%     $w = xyz$ и понеже $\abs{xy} \leq p$, то $\abs{z} > 1$.
%     Имаме, че $xy^iz \in L_5$ и следователно $\abs{xy^iz} = \abs{xz} + i.\abs{y}$ е просто число, за всяко $i$.
%     Да изберем $i = \abs{xz} > 1$.
%     Но тогава $\abs{xy^iz} = (1 + \abs{y})\abs{xz}$ е съставно число, следователно 
%     достигнахме до противоречие.
%   \item
%     Разгледайте $w = 0^p1^p$. Проблем с условието $\abs{xy} \leq p$.
%     Друг начин е да се използва, че $0^\star1^\star \cap L_6 = L_1$.
%   \item
%     Разгледайте $0^p10^p1$.
%   \item
    
%   \item
%     Да разгледаме $w = 1^{p^2}$.
%     Да допуснем, че $L_9$ е регулярен и имаме, че
%     \[(\forall i)[\abs{xy^iz}\ \&\ \abs{xy^{i+1}z}\mbox{ са точни квадрати}].\]
%     Тогава съществува $n$, за което $n^2 = \abs{xy^iz}$, $\abs{xy^{i+1}z} \geq (n+1)^2$ и
%     \[\abs{y} = \abs{xy^{i+1}z} - \abs{xy^{i}z} \geq 2n + 1 = 2\sqrt{\abs{xy^iz}} + 1\]
%     Ясно е, че $\abs{y} \leq \abs{w} = p^2$.
%     Получаваме, че за всяко $i$,
%     \[\abs{y} \geq 2\sqrt{\abs{xy^iz}} + 1\]
%     Лесно стигаме до противоречие, например да вземем $i = p^2$.
%     Тогава $\abs{y} > p$, което е противоречие.
%     % Да вземем $i = p^4$.
%     % $\abs{y} \leq p^2 < 2\sqrt{p^4} + 1 \leq 2\sqrt{\abs{xy^iz}} + 1$,
%     което е противоречие.
%   \item
%   \item
%   \item
%     Аналогично на 9).
%   \item
%     Да допуснем, че $L_{13}$ е регулярен.
%     Но тогава $L^\prime = \{a,b\}^\star \setminus L_{13}$ е регулярен
%     и $L_1 = a^\star b^\star \cap L^\prime$ е регулярен.
%     Достигнахме до противоречие.
%   \end{enumerate}
% \end{proof}


\begin{problem}
  Да означим $\mbox{Half}(L) = \{w\mid (\exists x \in \Sigma^\star)[wx \in L\ \&\ \abs{w} = \abs{x}]\}$.
  Докажете, че ако $L$ е регулярен език, то $\mbox{Half}(L)$ също е регулярен език.
\end{problem}

\begin{problem}
  Да означим $\mbox{Pref}(L) = \{\alpha \mid (\exists \beta \in \Sigma^\star)[\alpha\beta \in L]\}$.
  Докажете, че ако $L$ е регулярен език, то $\mbox{Pref}(L)$ също е регулярен език.
\end{problem}
\begin{proof}
  Две доказателства. Едното е с индукция по построението регулярния израз за $L$.
  Другото е по автомата за $L$.
\end{proof}



%%% Local Variables: 
%%% mode: latex
%%% TeX-master: "discrete-math"
%%% End: 

% \chapter{Контекстно-свободни езици}

\section{Контекстно-свободни граматики}
% От Сипсер, същото е в слайдовете на Сашка
% Малко е тъпо, че в Пападимитриу дефиницията е различна. Там \Sigma \subseteq V
\begin{dfn}
  Контекстно-свободна граматика e четворка $G = (V,\Sigma,R,S)$,
  където
  \begin{itemize}
  \item
    $V$ е крайно множество от {\em променливи};
  \item
    $\Sigma$ е крайно множество от {\em терминали}, $\Sigma \cap V = \emptyset$;
  \item
    $R \subseteq V\times (V\cup\Sigma)^\star$, крайно множество от {\em правила};
  \item
    $S \in V$ е началната променлива. 
  \end{itemize}
  Обикновено ще пишем $A \rightarrow_G v$ вместо $(A,v) \in R$.
  Пишем $u \Rightarrow_G v$, ако съществуват думи $x,y\in (\Sigma\cup V)^\star$, $A\in V$,
  правило $A\rightarrow_G v^\prime$ и $u = xAy$, $v = xv^\prime y$.
  Езикът генериран от $G$, $L(G) = \{\alpha\in\Sigma^\star\mid S \Rightarrow^\star_G \alpha\}$.
\end{dfn}

\begin{thm}
  Контекстно-свободните езици са затворени относно
  операциите обединение, конкатенация и звезда на Клини.
\end{thm}

\begin{thm}
  Сечение на контекстно-свободен език с регулярен език е контекстно-свободен език.
\end{thm}

\section{Езици, които не са контекстно-свободни}

\begin{lemma}[за нарастването (контекстно-свободни езици)]
  \index{лема за нарастването!контекстно-свободни езици}
  \label{lem:pumping-context}
  За всеки КСЕ $L\neq\{\epsilon\}$ съществува $n>0$ такова,
  че ако $\alpha\in L, \abs{\alpha} \geq p$, то $\alpha=xyuvw$ и
  \begin{enumerate}
  \item
    $\abs{yv}\geq 1$,
  \item
    $\abs{yuv}\leq p$, и
  \item
    $(\forall i\geq 0)[xy^iuv^iw\in L]$.
\end{enumerate}
\end{lemma}

\begin{crl}
  Нека $G$ е контекстно-свободна граматика и $n$ е константата за разрастването на $G$.
  Тогава $\abs{L(G)} = \infty$ точно тогава, когато съществува $z \in L(G)$, за която $n \leq \abs{z} \leq 2n$.
\end{crl}

\begin{thm}
  Контекстно-свободните езици {\bf не} са затворени относно сечение и допълнение.
\end{thm}
% \begin{proof}
%   Езикът $\Ls_0 = \{a^nb^nc^n\mid n\in\N\}$ не е контекстно-свободен, докато езиците 
%   $\Ls_1 = \{a^nb^nc^m\mid n,m\in\N\}$ и $\Ls_2 = \{a^mb^nc^n\mid n,m\in\N\}$ 
%   са контекстно-свободни.
%   Понеже, $\Ls_0 = \Ls_1\cap\Ls_2$, то заключаваме, 
%   че контекстно-свободните езици не са затворени относно операцията сечение.
%   Понеже $\Ls_1\cap\Ls_2 = \ov{\ov{\Ls_1}\cup\ov{\Ls_2}}$,
%   контекстно-свободните езици не са затворени относно допълнение.
% \end{proof}


\begin{problem}
  Проверете дали следните езици са контекстно-свободни:
  \begin{enumerate}
  \item
    $L_1 = \{a^nb^nc^n\ \mid\ n\in\N\}$;
  \item
    $L_2 = \{a^ib^jc^k\ \mid\ 0 \leq i \leq j \leq k\}$;
  \item
    $L_3 = \{ww\mid w\in \{a,b\}^\star\}$;
  \item
    $L_3 = \{ww^R\mid w\in \{a,b\}^\star\}$;
  \item
    $L_4 = \{a^ib^j\ \mid i = j^2\}$;
  \item
    $L_5 = \{a^p\ \mid\ p\mbox{ е просто }\}$;
  \item
    $L_6 = \{a^nb^na^nb^n\mid n\geq 0\}$;
  \item
    $L_7 = \{a^n b a^{2n} b a^{3n}\mid n\geq 0\}$;
  \item
    $L_8 = \{w c x\mid w,x\in \{a,b\}^\star\ \&\ w\mbox{ е подниз на }x\}$;
  \item
    $L_9 = \{x_1 c x_2 c \dots c x_k\mid k\geq 2\ \&\ x_i\in\{a,b\}^\star\ \&\ (\exists i,j)[i \neq j\ \&\ x_i = x_j]\}$;
  \item
    $L_{10} = \{a^ib^jc^k\mid i,j,k\geq 0\ \&\ (i = j \vee j = k)\}$;
  \item
    $L_{11} = \{\alpha \in \{a,b,c\}^\star\mid n_a(\alpha) = n_b(\alpha) = n_c(\alpha)\}$;
  \end{enumerate}
\end{problem}
\begin{proof}
  \begin{enumerate}[1)]
  \item
    За думата $w = a^pb^pc^p = xyuvw$ разгледайте различните случаи за $y$ и $v$.
  \item
    Разгледайте $w = a^pb^pc^p$.
    \begin{enumerate}[a)]
    \item
      Знаем, че поне една от $y$ и $v$ не е празната дума.
      Имаме два случая за $y$, $v$.
      \begin{enumerate}[i)]
      \item
        $a$ не се среща в $y$ и $v$.
        Тогава $xy^0vu^0w$ съдържа повече $a$ от $b$ или $c$.
      \item
        $b$ не се среща в $y$ и $v$.
        Ако $a$ се среща в $y$ или $v$, тогава $xy^2uv^2w$ съдържа повече $a$ от $b$
        Ако $c$ се среща в $y$ или $v$, тогава $xy^0uv^0w$ съдържа по-малко $c$ от $b$.
      \item
        $c$ не се среща в $y$ и $v$.
        Тогава $xy^2uv^2w$ съдържа повече $a$ или $b$ от $c$.
      \end{enumerate}      
    \item
      $y$ или $v$ е съставена от две букви.
      Тук разглеждаме $xy^2uv^2w$ и съобразяваме, че редът на буквите е нарушен.
    \end{enumerate}
  \item
    \marginpar{Защо $\alpha = a^pba^pb$ не е добър кандидат?}
    Разгледайте $\alpha = a^pb^pa^pb^p$.
    \begin{enumerate}[a)]
    \item
      Ако $yuv$ е в първата част на думата, то 
      $xy^0uv^0w = a^ib^ja^pb^p \not\in L_3$.
      Аналогично ако $yuv$ е във втората част на думата.
    \item
      Ако $yuv$ е в двете части на думата, то 
      Но $xy^0uv^0w = a^pb^ia^jb^p \not\in L_3$.
    \end{enumerate}
  \item
  \item[10)]
    Контекстно-свободен е. Лесно може да се напише контекстно-свободна граматика за този език.
  \item[11)]
    Разгледайте $L_{11} \cap a^\star b^\star c^\star$.
  \end{enumerate}
\end{proof}


\section{Алгоритми}

\subsection{Нормална Форма на Чомски}

\begin{problem}
Използвайте обща конструкция, за да намерите всички нетерминали, от които в $\Gamma$ се извежда празната дума.
Принадлежи ли празната дума на езика на $\Gamma$? Обосновете се.

\begin{enumerate}
\item
$\Gamma=\langle\{S,A,B,C,D,E\},\{a,b\},S,\{S\rightarrow D,D\rightarrow AD|b,A\rightarrow ACB|BC|a, B\rightarrow ABCA|CEC,C\rightarrow \varepsilon|CA|a, E\rightarrow \varepsilon|aEb\}\rangle$;
\item
$\Gamma=\langle\{S,A,B,C,D,E\},\{a,b\},S,\{S \rightarrow aD, D\rightarrow \varepsilon|ABBA|ADD,A\rightarrow DEB|a,B\rightarrow DDD|DC|b,C\rightarrow CCE|a, E\rightarrow \varepsilon|bEa\}\rangle$;
\item
$\Gamma=\langle\{S,A,B,C,D,E\},\{a,b\},S,\{ S\rightarrow D,D\rightarrow AD|b,A\rightarrow AB|BC|a, B\rightarrow AB|CC, C\rightarrow \varepsilon|CA|a, E\rightarrow a|EB\}\rangle$;
\item
$\Gamma=\langle\{S,A,B,C,D,E\},\{a,b\},S,\{ S \rightarrow AD|a, D\rightarrow \varepsilon|BB|AD,A\rightarrow DB|a,  B\rightarrow DD|DC|b,C\rightarrow CE|a, E\rightarrow AB|b|EA\}\rangle$;
\item
$\Gamma=\langle\{a,b\},\{S,A,B,C\},S,\{S\rightarrow AS|SB|SS,B\rightarrow CA|b, C\rightarrow AA|a|BA,A\rightarrow \varepsilon|BS\}\rangle$;
\item
$\Gamma=\langle\{a,b\},\{S,A,B,C\},S,\{S\rightarrow AB|AC,B\rightarrow \varepsilon |BC|b,A\rightarrow BB|CC|a,
C\rightarrow CS|a\}\rangle$;

\item
$\Gamma=\langle\{S,A,B,C,D,E\},\{a,b\},S,\{S\rightarrow DD,D\rightarrow AD|b,A\rightarrow BC|a,B\rightarrow AB|CC, C\rightarrow \varepsilon|AC|a, E\rightarrow a|EB\}\rangle$;

\item
$\Gamma=\langle\{S,A,B,C,D,E\},\{a,b\},S,\{S \rightarrow DA|a, D\rightarrow \varepsilon|BD|AB,A\rightarrow DD|a, B\rightarrow CC|DC|b, C\rightarrow EC|a, E\rightarrow BA|b|AE\}\rangle$;
\end{enumerate}
\end{problem}


\begin{problem} Нека множеството от терминали в $\Gamma$, което извежда празната дума е $\mathcal{E}$. Използвайте обща конструкция, за да намерите граматика $\Gamma_1$ без $\varepsilon$-правила, за която $L(\Gamma_1)=L(\Gamma)\setminus\{\varepsilon\}$. Принадлежи ли празната дума на $L(\Gamma)$? Обосновете се!

\begin{enumerate}
\item
$\Gamma=\langle\{a,b\},\{S,A,B,C\},S,\{S\rightarrow AS|SB|SS,B\rightarrow AC|b, C\rightarrow A|a|AB,A\rightarrow \varepsilon|BS\}\rangle$, $\mathcal{E}=\{A,B,C\}$;

\item
$\Gamma=\langle\{a,b\},\{S,A,B,C\},S,\{S\rightarrow BA|CA,B\rightarrow \varepsilon |BC|b,A\rightarrow BB|CC|a,
C\rightarrow CS|a\}\rangle$, $\mathcal{E}=\{A,B,S\}$;

\item
$\Gamma=\langle\{a,b\},\{S,A,B,C\},S,\{S\rightarrow AS|b,A\rightarrow AC|BC|a, B\rightarrow BC|CC,C\rightarrow \varepsilon|CA|a\}\rangle$, $\mathcal{E}=\{A,B,C\}$;

\item
$\Gamma=\langle\{a,b\},\{S,A,B,C\},S,\{S\rightarrow \varepsilon|BA|AS,A\rightarrow SB|a,B\rightarrow SS|SC|b,
C\rightarrow CC|a\}\rangle$, $\mathcal{E}=\{A,B,S\}$; 

\end{enumerate} 
\end{problem}

\begin{problem}
Използвайте обща конструкция, за да премахнете "дългите" правила (т.е. с дължина поне 2, които не са в н.ф. на Чомски) от $ G_1$ като при това получите к.св. граматика $G$ с език $L(G)=L(G_1)$, където:
\begin{enumerate}
\item
$G_1=\langle\{S,T\},\{a,b\},S,\{S \rightarrow \epsilon|ab|aTba,T\rightarrow aTTb\}\rangle$;
\item
$G_1=\langle\{S,T\},\{a,b\},S,\{S \rightarrow \epsilon|ab|baTb,T\rightarrow TaTb\}\rangle$;
\item
$\Gamma=\langle\{a,b\},\{A,B,C,S\},S,\{A\rightarrow BSB|a,B\rightarrow ba|BC,C\rightarrow BaSA|a|b,S\rightarrow CC|b\}\rangle$;
\item
$\Gamma=\langle\{a,b\},\{A,B,C,S\},S,\{A\rightarrow BAS,B\rightarrow CB,C\rightarrow ab|ABbS,S\rightarrow CC|b\}\rangle$;
\item
$G_1=\langle\{S,T\},\{a,b\},S,\{S \rightarrow \epsilon|ab|aTba,T\rightarrow TabTT|a\}\rangle$;
\item
$G_1=\langle\{S,T\},\{a,b\},S,\{S \rightarrow \epsilon|ab|bTba,T\rightarrow aTaTb|b\}\rangle$;
\end{enumerate}
\end{problem}


\begin{problem}
Използвайте обща конструкция, за да премахнете преименуващите правила от граматиката $\Gamma$ като при това запазите езика.

\begin{enumerate}
\item
$\Gamma=\langle\{a,b\},\{A,B,C,S\},S,\{A\rightarrow B|S,B\rightarrow C|BC,C\rightarrow AB|a|b,S\rightarrow B|CC|b\}\rangle$;
\item
$\Gamma=\langle\{a,b\},\{A,B,C,S\},S,\{A\rightarrow B,B\rightarrow S|C|BC,C\rightarrow a|AB,S\rightarrow C|CC|b\}\rangle$;
\item
$\Gamma=\langle\{a,b\},\{A,B,C,S\},S,\{A\rightarrow B|CC|a,B\rightarrow S|AB,C\rightarrow SC|b,S\rightarrow A|CC|b\}\rangle$;
\item
$\Gamma=\langle\{a,b\},\{A,B,C,S\},S,\{A\rightarrow BB|b,B\rightarrow S|SS|b,C\rightarrow B|a,S\rightarrow C|AB|a\}\rangle$;
\end{enumerate}
\end{problem}

\begin{problem}
Построите к. св. г. $G$ в нормална форма на Чомски с език $L(G) = L(G_1)$, където:
\begin{enumerate}
\item
$G_1=\langle\{S,T\},\{a,b\},S,\{S \rightarrow \epsilon|ab|aTba,T\rightarrow aTTb\}\rangle$;
\item
$G_1=\langle\{S,T\},\{a,b\},S,\{S \rightarrow \epsilon|ab|baTb,T\rightarrow TaTb\}\rangle$;
\end{enumerate}
\end{problem}


\subsection{Проблемът за принадлежност}

\begin{problem}
Използвайте алгоритъма за динамично програмиране (CYK), за да проверите дали
думата $\alpha$ принадлежи на езика, определен от граматиката $\Gamma$.

\begin{enumerate}
\item
$\Gamma=\langle \{a,b\}, \{S,A,B,C\},S,\{S\rightarrow a| AB|AC, C\rightarrow SB|AS,A\rightarrow a, B\rightarrow b\}\rangle$, $\alpha=aaabb$;
\item
$\Gamma=\langle \{a,b\}, \{S,A,B,C\},S,\{S\rightarrow BA| CA|a, C\rightarrow BS|SA,A\rightarrow a, B\rightarrow b\}\rangle$, $\alpha=bbaaa$;
\item
$\Gamma=\langle \{a,b\}, \{S,A,B,C\},S,\{S\rightarrow AB|BC, A\rightarrow BA|a,B\rightarrow CC|b, C\rightarrow AB|a\}\rangle$, $\alpha=baaba$;
\item
$\Gamma=\langle \{a,b\}, \{S,A,B,C\},S,\{S\rightarrow AB, A\rightarrow AC|a|b,B\rightarrow CB|a, C\rightarrow a\}\rangle$, $\alpha=babaa$;
\item
$\Gamma=\langle \{S,A,B\}, \{a,b\},S,\{S\rightarrow BA|SS|b, A\rightarrow SA|a,B\rightarrow BS|b\}\rangle$, $bbbaa$;
\item
$\Gamma=\langle\{S,A,B\},\{a,b\},S,\{S\rightarrow AB|SS|a, A\rightarrow AS|a,B\rightarrow SB|b\}\rangle$, $aaabb$;
\item
$\Gamma=\langle \{a,b\}, \{S,A,B\},S,\{S\rightarrow AB| BS|b, A\rightarrow SS|a,B\rightarrow BA|b\}\rangle$, $\alpha=babab$;
\item
$\Gamma=\langle \{a,b\}, \{S,A,B\},S,\{S\rightarrow BA| AS|a, A\rightarrow AB|a,B\rightarrow SS|b\}\rangle$, $\alpha=ababa$;
\item
$\Gamma=\langle \{a,b\}, \{S,A,B\},S,\{S\rightarrow AB| BS|b, A\rightarrow SS|a,B\rightarrow BA|b\}\rangle$, $\alpha=babab$;
\item
$\Gamma=\langle \{a,b\}, \{S,A,B\},S,\{S\rightarrow AB|a, A\rightarrow BA|SS|a,B\rightarrow SS|b\}\rangle$, $\alpha=aabba$;
\end{enumerate}
\end{problem}


\section{Недетерминирани стекови автомати}

\index{автомат!недетерминиран стеков}
% \marginpar{На англ. Push-down automaton}
%Sipser p.102
\begin{dfn}[стр. 102 от \cite{sipser}]
  Недетерминиран краен стеков автомат е $P = \SA$, където 
  \begin{itemize}
  \item
    $Q$ е крайно множество от състояния;
  \item  
    $\Sigma$ е крайна входна азбука;
  \item
    $\Gamma$ е крайна стекова азбука;
  \item
    $q_{0}\in Q$ е начално състояние;
  \item
    $\Delta:Q\times\Sigma_\varepsilon\times\Gamma_\varepsilon\rightarrow \Ps(Q\times\Gamma_\varepsilon)$ 
    е {\em частична} функция на преходите;    
  \item
    $F\subseteq Q$ е множество от заключителни състояния.
  \end{itemize}
\end{dfn}

Нека $P$ е стеков автомат. Тогава
\begin{itemize}
\item
  $\Ls(P)$ е езика, който се разпознава от $P$ с финално състояние,
  \[\Ls(P) = \{w\mid (q_0,w,\varepsilon) \vdash^\star_P (q,\varepsilon,\alpha)\ \&\ q \in F\}.\]    
\item
  $\Ls_N(P)$ е езика, който се разпознава от $P$ с празен стек,
  \[\Ls_N(P) = \{w\mid (q_0,w,\varepsilon) \vdash^\star_P (q,\varepsilon,\varepsilon)\}.\]    
\end{itemize}

\begin{thm}
  Класът на езиците, които се разпознават от краен стеков автомат, съвпада с
  класа на контекстно-свободните езици.
\end{thm}



%%% Local Variables: 
%%% mode: latex
%%% TeX-master: "discrete-math"
%%% End: 

% \include{turing}
%\begin{thm}
  Ако $G$ е дърво, то $\varepsilon = \nu - 1$.
\end{thm}

\begin{crl}
  Всяко нетривиално дърво има поне два върха със степен 1.
\end{crl}

\begin{problem}
  Покажете, че ако $G$ е дърво и има връх със степен $\geq k$, то $G$ има поне $k$ върха със степен 1.
\end{problem}

\begin{problem}
  Нека $G$ е свързан граф.
  Докажете, че всеки два най-дълги пътя в свързан граф имат общ връх.
\end{problem}

\begin{problem}
  За $G$ прост граф, докажете, че $\varepsilon = \binom{\nu}{2}$ т.с.т.к. $G$ е пълен.
\end{problem}

\begin{problem}
  Докажете, че в граф с $\nu\geq 2$, има поне два върха с еднаква степен.
\end{problem}

\begin{problem}
  Докажете, че:
  \begin{enumerate}
  \item
    във всеки неориентиран граф броят на върховете с нечетна степен е четен;
  \item
    всеки регулярен граф с нечетна степен има четен брой върхове;
  \item
    всеки граф с $\varepsilon > \binom{\nu-1}{2}$ е свързан.
    Дайде пример за несвързан граф с $\varepsilon = \binom{\nu-1}{2}$.
  \item
    във граф всички върхове имат степен поне $d$.
    Докажете, че в графа има път с дължина $d$.
  \end{enumerate}
\end{problem}


\begin{problem} % зад. 1.22
  Да разгледаме графа $G$ (без примки и без кратни ребра) със $s$ компоненти на свързаност.
  Докажете, че $\nu - s \leq \varepsilon \leq \binom{\nu-s+1}{2}$.
\end{problem}

\begin{problem}
  Нега $G$ е граф с $n$ върха и в $G$ няма прост цикъл с дължина 3.
  Докажете, че $G$ има най-много $\lfloor{\frac{n^2}{4}}\rfloor$ ребра.
\end{problem}

\begin{problem}
  Нека $G$ е произволен граф без примки и кратни ребра, а $\overline{G}$ е неговото допълнение.
  Докажете, че поне един от графите $G$, $\overline{G}$ е свързан;
\end{problem}


\begin{problem}
  \begin{enumerate}
  \item
    Да се построят всички неизоморфни графи на 1,2,3 и 4 върха.
  \item
    Намерете броя на ребрата на граф без цикли с $n$ върха и $k$ компоненти.
  \end{enumerate}
\end{problem}

% \chapter {Алгоритми за графи}

\section{Обхождане на граф}

\subsection{Обхождане в дълбочина}

\subsection{Обхождане в широчина}

\begin{itemize}
\item
  
\item 
  Алгоритъмът работи както за ориентирани, така и за неориентирани графи.
\end{itemize}

\section{Минимално покриващо дърво на граф}

\begin{itemize}
\item
  Един граф $G = (V,E)$ се нарича {\bf свързан}, ако има път между всеки два $v,v^\prime \in V$.
\item 
  Един граф $G$ се нарича {\bf дърво}, ако $G$ е свързан неориентиран граф без цикли.
\item
  {\bf Покриващо дърво} за свъзан неориентиран граф $G = (V,E)$,
  е дърво $T = (V,E^\prime)$, $Е^\prime \subseteq E$.
\item
  {\bf Минимално покриващо дърво} свъзан неориентиран граф $G = (V,E,c)$
  е покриващо дърво $T$, за което числото $r = \sum_{e\in T} c(e)$ е минимално.
\end{itemize}

\subsection{Алгоритъм на Прим}

Нека е даден неориентиран свързан претеглен граф $G = (V,E,c)$
и да фиксираме един връх $r \in V$, който ще бъден корен покриващото дърво.

\begin{enumerate}
\item 
  Започваме от дървото $T_0 = (\{r\},\emptyset)$.
\item
  Нека сме построили дървото $T_i = (V_i,E_i)$.
  Търси реброто $(v,v^\prime)$ с минимално тегло, за което
  $v \in V_i$ и $v^\prime \in V \setminus V_i$.
  Образуваме \[T_{i+1} = (V_i\cup\{v^\prime\}, E_i \cup \{(v,v^\prime)\}).\]
\item
  Алгоритъмът завършва когато $V_i = V$.
\end{enumerate}

\subsection{Алгоритъм на Крускал}

Нека е даден неориентиран свързан претеглен граф $G = (V,E,c)$.
\begin{enumerate}
\item
  Нека $V = \{v_0,\dots,v_n\}$.
  Строим редица от дървета $\{T_i\}_{i \leq n}$, като
  $T_i = (\{v_i\},\emptyset)$, т.е. дърветата са съставени само от един връх и нямат ребра.
  Ще се стремим да обединяваме тези дървете и най-накрая да получим само едно дърво.
\item
  Сортираме ребрата $E$ на графа $G$ във възходящ ред относно тяхната цена.
\item
  За всяко ребро $(v,v^\prime) \in E$, нека $v$ принадлежи на върховете на $T$, а $v^\prime$ 
  принадлежи на върховете на $T^\prime$.
  \begin{itemize}
  \item 
    Ако $T \neq T^\prime$, то обединяваме $T$ и $T^\prime$ като добавяме към техните ребра $(v,v^\prime)$.
    Така получаваме с единица по-къса редица от дървета.
  \item
    Ако $T = T^\prime$, то не правим нищо.    
  \end{itemize}  
\item
  Алгоритъмът завършва или когато сме обходили всички ребра, или
  когато сме получили редица от дървета с дължина 1.
\end{enumerate}

\section{Минимални пътища от даден връх}

\begin{itemize}
\item
  С $u \stackrel{p}{\leadsto} v$ означаваме, че $p$ е път от $u$ до $v$.
\item
  Тук ще разглеждаме ориентирани графи $G = (V,E)$, като имаме и 
  функция $w: E\to \R$, която задава {\bf тегла} на ребрата на графа.
\item 
  {\bf Цена на път} $p = (v_0,\dots,v_k)$ в графа означаваме 
  \[w(p) = \sum_{i<k} w(v_i,v_{i+1}).\]
\item
  За всеки два върха $u,v \in V$, означаваме
  \begin{align*}
    \delta(u,v) = 
    \begin{cases}
      \min\{w(p)\mid u \stackrel{p}{\leadsto} v\}, \mbox{ ако има път от }u\mbox{ до }v\\
      \infty, \mbox{ иначе }
    \end{cases}
  \end{align*}
\item
  {\bf Минимален път} $p$ от $u$ до $v$ е такъв път, за който $w(p) = \delta(u,v)$.
\item
  Имаме следното важно свойство.
  Нека $u \stackrel{p}{\leadsto} v$ и $p$ е {\bf минимален път}.
  Да означим $p = (v_0,\dots,v_k)$ и $p_{ij} = (v_i,\dots,v_j)$ за $0\leq i \leq j \leq k$.
  Тогава за всеки $0\leq i \leq j \leq k$, 
  $p_{ij}$ е {\bf минимален път} от $v_i$ до $v_j$.
\item
  {\bf Цикъл} е път $p = (v_0,\dots,v_k)$, където $v_0 = v_k$.
\item
  Също така казваме, че по пътя $p = (v_0,\dots,v_k)$ има цикъл, ако
  за някои $0 \leq i < j \leq k$ имаме, че $v_i = v_j$.
\item
  Ако има цикъл с отрицателно тегло по някой път от $u$ до $v$, то 
  тогава пишем, че $\delta(u,v) = -\infty$.
\item
  Нека $u \stackrel{p}{\leadsto} v$ и $p$ е с минимално тегло.
  Тогава няма цикъл с положително тегло по $p$.
\item
  Нека $u \stackrel{p}{\leadsto} v$ и $p$ е с минимално тегло.
  Можем без ограничение на общността да приемем, че няма цикли с нулево тегло
  по $p$.
\item
  Важно свойство е, че броят на върховете по всички минални пътища е $\leq \abs{V}$.
\end{itemize}

\begin{prop}
  Нека $u \stackrel{p}{\leadsto} v$, където $p$ е с минимално тегло.
  Тогава няма цикъл с положително тегло по $p$.
\end{prop}
\begin{proof}
  Нека $p = (v_0,\dots,v_k)$, $v_0 = u$, $v_k = v$ и 
  $c = (v_i,\dots,v_j)$, за някои $0 \leq i < j \leq k$, 
  като $w(c) > 0$. Щом $c$ е цикъл, то $v_i = v_j$.
  Но тогава пътя $p^\prime = (v_0,\dots,v_i,v_{j+1},\dots,v_k)$ също е път от $u$ до $v$.
  Освен това, $w(p^\prime) = w(p) - w(c) < w(p)$.
  Получихме, че пътя $p^\prime$ има по-малко тегло от пътя $p$.
  Това е противоречие с минималността на теглото на пътя $p$.
\end{proof}




Нека да фиксираме един връх $s \in V$.
Нашата цел е да намерим минимални пътища от $s$ до всички достижими от $s$ върхове,
както и тяхната цена. 
Да отбележим, че ако имаме отрицателен по някой път $s \leadsto v$, то задачата не е добре 
дефинирана, защото тогава $\delta(s,v) = -\infty$.

За тази цел въвеждаме два масива, $dist$ и $pred$, с дължина $\abs{V}$.
\begin{itemize}
\item 
  $dist[v]$ - дава цена на минимален път от $s$ до $v$.
  Ако $dist[v] = \infty$, то не е намерен път $s\leadsto v$.
\item
  $pred[v]$ - дава предшественика на $v$ по този минимален път, т.е.
  ако $pred[v] = u$, то $s \leadsto u \to v$.
  Ако $pred[v] = NIL$, то не е намерен път $s \leadsto v$.
\item
  Сега като имаме масива $pred$, можем да образуваме графа на предшествениците $G_{pred} = (V_{pred},E_{pred})$, където
  \begin{enumerate}[]
  \item 
    $V_{pred} = \{u \in V \mid pred[u] \neq NIL \} \cup \{s\}$,
  \item
    $E_{pred} = \{(u,v) \in V\times V \mid pred[v] = u\}$.
  \end{enumerate}
\end{itemize}

% \begin{figure}[!ht]
%   \centering
% \begin{subfigure}{0.5\textwidth}
%   \centering
% % \begin{algorithm}
%   % \caption{INIT(s)}
%   % \label{alg:init}
%   \begin{algorithmic}[1]
%     \FORALL{$v \in V$}
%     \STATE $dist[v] := \infty$
%     \STATE $pred[v] := NIL$
%     \ENDFOR
%     \STATE $dist[s] := 0$
%   \end{algorithmic}
% % \end{algorithm}
% \end{subfigure}
% \hfill
% \begin{subfigure}{0.5\textwidth}
%   \centering
% % \begin{algorithm}
%   % \caption{UPDATE(u,v)}
%   % \label{alg:update}
%   \begin{algorithmic}[1]
%     \IF{$dist[v] > dist[u] + w(u,v)$}
%     \STATE $dist[v] := dist[u] + w(u,v)$
%     \STATE $pred[v] := u$
%     \ENDIF
%   \end{algorithmic}
% % \end{algorithm}
% \end{subfigure}
% \end{figure}

\begin{algorithm}
  \caption{INIT(s)}
  \label{alg:init}
  \begin{algorithmic}[1]
    \FORALL{$v \in V$}
    \STATE $dist[v] := \infty$
    \STATE $pred[v] := NIL$
    \ENDFOR
    \STATE $dist[s] := 0$
  \end{algorithmic}
\end{algorithm}

\begin{algorithm}
  \caption{UPDATE(u,v)}
  \label{alg:update}
  \begin{algorithmic}[1]
    \IF{$dist[v] > dist[u] + w(u,v)$}
    \STATE $dist[v] := dist[u] + w(u,v)$
    \STATE $pred[v] := u$
    \ENDIF
  \end{algorithmic}
\end{algorithm}

\subsection{Основни свойства}
  
\begin{prop}[Неравенство на триъгълника]
  \label{prop:triangle}
  За всяко $(u,v) \in E$,
  \[\delta(s,v) \leq \delta(s,u) + w(u,v).\]
\end{prop}

\begin{prop}
  \label{prop:upper-bound}
  Нека сме изпълнили INIT(s).
  Тогава имаме свойството \[(\forall v\in V)[dist[v] \geq \delta(s,v)].\]
  То се запазва и след прозволен брой изпълнения на UPDATE върху ребра на графа.
  
  Освен това, ако веднъж $dist[v] = \delta(s,v)$, то $dist[v]$
  повече не се променя.
\end{prop}
\begin{proof}
  Индукция по броя $i$ на изпълнения на UPDATE.
  За $i = 0$ е очевидно.
  Ще докажем твърдението за $i > 0$ изпълнения на UPDATE.
  Нека $dist[v] > dist[u] + w(u,v)$ и изпълним UPDATE(u,v).
  Тогава като използваме индукционното предположение и неравенството на триъгълника,
  \begin{align*}
    dist[v] & = dist[u] + w(u,v)\\
    & \geq \delta(s,u) + w(u,v)\\
    & \geq  \delta(s,v).
  \end{align*}

  Ясно е, че веднъж достигнали $dist[v] = \delta(s,v)$, $dist[v]$
  не може да се промени, защото тази стойност може само да намалява, а ние
  сме достигнали нейния минумум.
\end{proof}

\begin{prop}
  \label{prop:no-path}
  Нека сме изпълнили INIT(s)
  и нека няма път от $s$ до $v$.
  Тогава имаме свойството
  \[dist[v] = \delta(s,v) = \infty.\]
  То се запазва и след прозволен брой изпълнения на UPDATE върху ребра на графа.
\end{prop}
\begin{proof}
  Щом няма път от $s$ до $v$, то $\delta(s,v) = \infty$.
  От Твърдение \ref{prop:upper-bound}, $dist[v] \geq \delta(s,v) = \infty$.
  Следователно, $dist[v] = \infty$.
\end{proof}

% \begin{proof}
%   Индукция по броя на изпълнения $i$ на UPDATE.
  
%   За $i = 0$, то твърдението е очевидно.
%   Да приемем, че то е вярно за $\leq i$.
%   Нека $(i+1)$-то изпълнение е UPDATE(u,v), за някое $u \in V$,
%   само в този случай можем да променим стойността на $v$.
  
%   Ако $dist[u] = \infty$, то $dist[v] = dist[u] + w(u,v)$ и нищо не правим.
%   Ако $dist[u] < \infty$, то това означава, че има път от $s$ до $u$, 
%   защото от твърдение ...., $\infty > dist[u] \geq \delta(s,u)$.
%   \begin{itemize}
%   \item 
%     ако има ребро $(u,v) \in E$, то това означава, че има път $s \leadsto u \to v$,
%     което е противоречие с условието.
%   \item
%     ако няма ребро $(u,v)\not\in E$, $w(u,v) = \infty$ и $dist[v] = \infty$.
%   \end{itemize}
% \end{proof}


% \begin{prop}
%   Нека $(u,v) \in E$.
%   Тогава веднага след изпълнението на UPDATE(u,v) имаме, че
%   \[dist[v] \leq dist[u] + w(u,v).\]
% \end{prop}


\begin{prop}
  \label{prop:converge}
  Нека $s\leadsto u \to v$ е път с минимално тегло.
  Нека сме изпълнили INIT(s) и няколко на брой UPDATE, като измежду тях и UPDATE(u,v).
  Ако по някое време преди изпълнение на UPDATE(u,v) 
  имаме, че $dist[u] = \delta(s,u)$, то след това изпълнение
  $dist[v] = \delta(s,v)$ и повече не се променя.
\end{prop}
\begin{proof}
  Първо да отбележим, че за $(u,v) \in E$, веднага след изпълнението на UPDATE(u,v) имаме, че
  \[dist[v] \leq dist[u] + w(u,v).\]
  Ако $dist[u] = \delta(s,u)$, то от Твърдение \ref{prop:upper-bound} това равенство се запазва.
  Получаваме, че:
  \begin{align*}
    dist[v] & \leq dist[u] + w(u,v)\\
    & = \delta(s,u) + w(u,v)\\
    & = \delta(s,v),
  \end{align*}
  защото $s\leadsto u \to v$ е път с минимална дължина.
  Тогава $dist[v] \leq \delta(s,v)$ и следователно 
  \[dist[v] = \delta(s,v),\]
  защото пак от Твърдение \ref{prop:upper-bound}, винаги е изпълнено, че $dist[v] \geq \delta(s,v)$,
\end{proof}

\begin{prop}
  \label{prop:path-update}
  Да разгледаме пътя $p = (v_0,\dots,v_k)$, като $v_0 = s$.
  Нека сме изпълнили INIT(s) и след това няколко пъти UPDATE, като сме включили 
  UPDATE($v_{i}$,$v_{i+1}$), за всяко $0\leq i < k$, в този ред на изпълнение.
  Тогава най-накрая получаваме, че $dist[v_k] = \delta(s,v_k)$.
\end{prop}
\begin{proof}
  Индукция по $i$.
  В началото, $dist[v_0] = dist[s] = 0 = \delta(s,s)$.
  Ако $dist[v_{i-1}] = \delta(s,v_{i-1})$, то след изпълнение на UPDATE($v_{i-1}$,$v_{i}$),
  получаваме от Твърдение \ref{prop:converge}, че $dist[v_i] = \delta(s,v_{i})$.
\end{proof}

\begin{prop}
  \label{prop:tree-shortest-path}
  Нека сега да приемем, че в нашия граф няма цикли с отрицателни тегла, достижими от $s$
  и нека сме изпълнили INIT(s) и произволен брой пъти UPDATE.
  Тогава:
  \begin{enumerate}[1)]
  \item 
    $G_{pred}$ е дърво с корен $s$.
  \item
    ако $(\forall v\in V)[dist[v] = \delta(s,v)]$, то $G_{pred}$ е дърво на пътищата с минимални тегла с корен $s$.
  \end{enumerate}
\end{prop}
\begin{proof}
  \begin{enumerate}[1)]
  \item 
    Първо ще докажем, че $G_{pred}$ е насочен ацикличен граф и след
    това, че няма пътища $p \neq p^\prime$ от вида $s\stackrel{p}{\leadsto} v$ и $s\stackrel{p^\prime}{\leadsto} v$.
    \begin{itemize}
    \item 
      Да допуснем, че $G_{pred}$ е цикличен граф.
      Нека $c = (v_0,\dots,v_k)$ е цикъл, $v_0 = v_k$, който се е получил точно след изпълнение на 
      UPDATE($v_{k-1}$,$v_k$).

      Да разгледаме ситуацията преди изпълнението на UPDATE($v_{k-1}$,$v_{k}$).
      Имаме, че 
      \[(\forall i < k)[pred[v_{i+1}] = v_{i}]\]
      от което следва, че
      \[(\forall i < k-1)[dist[v_{i+1}] \geq dist[v_i] + w(v_i,v_{i+1})].\]
      Щом още нямаме цикъл, преди изпълнението на UPDATE($v_{k-1}$,$v_k$) имаме, че
      \[dist[v_{k}] > dist[v_{k-1}] + w(v_{k-1},v_k).\]
      Получаваме, че
      \begin{align*}
        \sum^{k}_{i = 1} dist[v_{i}] & > \sum^{k-1}_{i=0} (dist[v_i] + w(v_{i},v_{i+1}))\\
        & = \sum^{k-1}_{i=0} dist[v_i] + w(c),
      \end{align*}
      но понеже $v_0 = v_k$, 
      \[\sum^{k-1}_{i=0} dist[v_i] = \sum^{k}_{i=1} dist[v_i]\]
      и тогава
      \[0 > w(c).\]
      Получаваме, че цикълът $c$ има отрицателно тегло, което е противоречие.
    \item
      Да допуснем, че има $p \neq p^\prime$  и  $s\stackrel{p}{\leadsto} v$ и $s\stackrel{p^\prime}{\leadsto} v$.
      Това означава, че съществуват $x \neq y$,
      $s \leadsto u \leadsto x \to z \leadsto v$ и $s \leadsto u \leadsto y \to z \leadsto v$.
      По определение, $pred(z) = x \neq y = pred(z)$. Противоречие.
    \end{itemize}
  \item
    \begin{itemize}
    \item 
      Лесно се съобразява, че $V_{pred}$ съдържа точно върховете достижими от $s$,
      защото $v$ е достижим от $s$ точно когато $\delta(s,v) = dist[v] < \infty$, 
      но тогава $pred[v] \neq NIL$.
    \item
      Вече доказахме в 1), че $G_{pred}$ е дърво с корен $s$.
    \item
      Остана да докажем, че ако имаме $s\stackrel{p}{\leadsto} v$ в $G_{pred}$, 
      то $p$ е път с минимално тегло в $G$.
      Нека $p = (v_0,\dots,v_k)$, $v_0 = s$, $v_k = v$.
      По условие, 
      \[(\forall i < k)[dist[v_i] = \delta(s,v_i)],\]
      а от факта, че $(\forall i < k)[pred[v_i] = v_{i-1}]$ следва, че
      \[(\forall i < k)[dist[v_i] \geq dist[v_{i-1}] + w(v_{i-1},v_{i})].\]
      Като обединим горните две неравенства, получваме, че
      \[(\forall i < k)[w(v_{i-1},v_{i}) \leq \delta(s,v_{i}) - \delta(s,v_{i-1})],\]
      Тогава
      \begin{align*}
        w(p) & = \sum^k_{i=1} w(v_{i-1},v_{i})\\
        & \leq \sum^k_{i=1} (\delta(s,v_i)- \delta(s,v_{i-1}))\\
        & = \delta(s,v_k) - \delta(s,v_0)\\
        & = \delta(s,v_k) - \delta(s,s)\\
        & = \delta(s,v_k) - 0\\
        & = \delta(s,v_k).
      \end{align*}
      Следователно, 
      \[w(p) \leq \delta(s,v_k).\]
      Понеже $\delta(s,v_k)$ е минималното тегло на път от $s$ до $v_k$,
      то $w(p) = \delta(s,v_k)$.
      Следователно, $p$ е път с минимално тегло.
    \end{itemize}
  \end{enumerate}
\end{proof}




\subsection{Алгоритъм на Дейкстра}
\index{Дейкстра!алгоритъм}

В този алгоритъм, разглеждаме ориентирани графи $G = (V,E,c)$ с {\em положителни} тегла (или цени) по ребрата, т.е. 
имаме функция $c:E\to\R^+$. 
Целта на алгоритъма е да построим следните функции:
\begin{itemize}
\item 
  $\delta: V\to\R\cup\{\infty\}$, 
  която да дава минималната цена на път от $s$ до $v$. Ако няма път от $s$ до $v$, то $\delta(v)$
  ще приема стойност $\infty$.
\item
  $\pi:V\to V\cup\{\infty\}$, която да дава предшественика на $v$ по път с минимална цена от $s$ до $v$,
  ако такъв път съществува, иначе ще дава като резултат $\infty$.
\end{itemize}

\begin{algorithm}
\caption{Алгоритъм на Дейкстра}
\label{alg:dijkstra}

\begin{algorithmic}[1]
  \REQUIRE{$w:E\to \R^+$}
  \STATE INIT(s)
  \STATE $V^\prime := V$
  \WHILE{$V^\prime\neq\emptyset$}
  % \AND $(\exists u\in V^\prime)[dist[u] < \infty]$}
  \STATE Избираме $u_0\in V^\prime$, за който $ dist[u_0] = min\{dist[v] \mid v\in V^\prime\} $
  \STATE $V^\prime := V^\prime\setminus\{u_0\}$
  \FORALL{ $v\in V^\prime $ }
  \IF{$(u_0,v)\in E$}
  \STATE UPDATE($u_0$,$v$)
    % \AND $\delta(v) > \delta(u_0) + c(u_0,v)$}
  % \STATE $\delta(v):= \delta(u_0)+c(u_0,v)$
  % \STATE $\pi(v) := u_0$
  \ENDIF
  \ENDFOR
  \ENDWHILE
  % \RETURN $\delta$
\end{algorithmic}
\end{algorithm}

\begin{thm}
  Нека $G$ е ориентиран граф с неотрицателни тегла по ребрата.
  След изпълнението на алгоритъма на Дейкстра с начален връх $s$,
  \[(\forall v \in V)[dist[v] = \delta(s,v)].\]
\end{thm}
\begin{proof}
  Ще докажем, че на всяка итерация на while-цикъла, 
  \[(\forall v\in V\setminus V^\prime)[dist[v] = \delta(s,v)].\]
  Първоначално $V\setminus V^\prime = \emptyset$.
  Ще докажем, че на всяка итерация на while-цикъла, за върха $u$, който сме премахнали от $V^\prime$,
  е изпълнено, че $dist[u] = \delta(s,u)$.
  За целта да допуснем противното и нека $u$ е първия връх, който е премахнат от $V^\prime$,
  за който $dist[u] \neq \delta(s,u)$.
  Лесно се съобразява, че $u \neq s$.
  Освен това, трябва $s \leadsto u$, защото иначе $dist[u] = \delta(s,u) = \infty$ според Твърдение \ref{prop:no-path}.
  Нека $s \stackrel{p}{\leadsto} u$ и $p$ е път с минимално тегло.
  Да разбием пътя $p$ по следния начин:
  \[s \stackrel{p_1}{\leadsto}x\to y\stackrel{p_2}{\leadsto}u,\]
  където $y$ е първия връх по пътя $p$, за който $y\not\in V^\prime$.
  Ясно е, че тогава $x \in V^\prime$ и тогава $dist[x] = \delta(s,x)$, 
  защото ние избрахме $u$ да бъде първия връх, за който $dist[u] \neq \delta(s,u)$.
  На итерацията на while-цикъла, на която добавяме $x$ към $V^\prime$, 
  ние изпълняваме UPDATE(x,y) и според Твърдение \ref{prop:converge}, $dist[y] = \delta(s,y)$.
  Но понеже $y$ е преди $u$ по път с минимално тегло и при положение, че няма ребра с отрицателни тегла,
  \[\delta(s,y) \leq \delta(s,u).\]
  Тогава
  \begin{align*}
    dist[y] & = \delta(s,y) \\
    & \leq \delta(s,u)\\
    & \leq dist[u], \mbox{според Твърдение \ref{prop:upper-bound}}.
  \end{align*}
  Но понеже $y,v \not\in V^\prime$ и сме избрали $u$ вместо $y$, то това означава, че
  \[dist[u] \leq dist[y].\]
  Следователно, 
  \[dist[y] = dist[u]\]
  и тогава 
  \[dist[u] = \delta(s,u),\]
  с което достигаме до противоречие.
\end{proof}
\begin{cor}
  $G_{pred}$ е дърво на минималните пътища с корен $s$.
\end{cor}


Ако във $V^\prime$ има останали върхове $v$, то те имат $\delta(v) = \infty$, т.е.
те са недостижими от $s$ и следователно пътят от $s$ до $v$ има дължина $\infty$.

Фигура \ref{fig:dijkstra-table} илюстрира как се променя функцията $\delta$ по време на изпълнението на алгоритъма.
Освен това, с лека модификация на горния алгоритъм, можем да намерим не само стойността на най-късите пътища, но
и списък с ребрата, които участват във всеки от тях. Фигура \ref{fig:dijkstra-graph} илюстрира това.
Ребрата, оцветени в зелено, са тези, които участват в най-късите пътища.
Жълти ребра са тези, които са кандидати да участват в най-късите пътища.
Червени са тези ребра, които са били вече обходени и са отхвърлени като част от най-къс път.

\tikzstyle{weight} = [font=\small]
\tikzstyle{value} = [font=\small]
\tikzstyle{edge} = [draw,thick,-]
\tikzstyle{nodedecorate}=[shape=circle,draw,thick,font=\small]
\tikzstyle{arrowdecorate}=[->,>=stealth,thick]

% Rename: selected --> current
\tikzstyle{selected vertex}=[vertex, fill=yellow!50]
\tikzstyle{selected edge} = [draw,line width=5pt,-,yellow!50]

\tikzstyle{vertex}=[circle,minimum size=15pt,inner sep=0pt]
\tikzstyle{sure vertex} = [vertex, fill=green!30]

\tikzstyle{path edge} = [draw,line width=5pt,-,red!50]

\tikzstyle{sure edge} = [draw,line width=5pt,-,green!30]
% \tikzstyle{ignored edge} = [draw,line width=5pt,-,black!20]


\begin{figure}[!htbp]
  \begin{subfigure}[b]{0.5\textwidth}
    \begin{tikzpicture}[]
      
      \foreach \nodename/\x/\y/\direction/\navigate in { a/1/1/above/north,
        b/0/0/left/west, c/1/-1.5/below/south, d/3/1/above/north, e/3/-1.5/below/south, f/5/0.5/right/east, g/5/2.5/right/east}
      {
        \node (\nodename) at (\x,\y) [nodedecorate] {};
        \node [\direction] at (\nodename.\navigate) {$\nodename$};
      }
      %% edges or lines
      \path
      \foreach \startnode/\endnode/\direction/\weight in {b/a/above/7,
        b/c/below/2, c/a/left/4, a/d/below/4, c/e/below/5, d/c/left/8, e/d/right/3}
      {
        (\startnode) edge[arrowdecorate] node[\direction] {$\weight$} (\endnode)
      }
      
      \foreach \startnode/\endnode/\direction/\angle/\weight in {
        a/g/above/15/10, d/f/above/15/5, d/g/above/-15/2, f/d/below/15/1, g/f/right/15/6, e/f/below/-15/7}
      {
        (\startnode) edge[arrowdecorate,bend left=\angle] node[\direction] {$\weight$} (\endnode)
      };
    \end{tikzpicture}
    \caption{Пример за насочен граф с тегла по ребрата}
  \end{subfigure}
 \quad
 \begin{subtable}[b]{0.5\textwidth}
   \begin{tabular}[b]{|c|c|c|c|c|c|c|c|c|}
     \hline
     $\delta(a)$ & $\delta(b)$ & $\delta(c)$ & $\delta(d)$ & $\delta(e)$ & $\delta(f)$ & $\delta(g)$\\
     \hline
     $\infty$ & {\bf \framebox{0}} & $\infty$ & $\infty$ & $\infty$ & $\infty$ & $\infty$ \\
     \hline
     7 & $\colon$ & {\bf \framebox{2}} & $\infty$ & $\infty$ & $\infty$ & $\infty$ \\
     \hline
      {\bf \framebox{6}} & $\colon$ & $\colon$ & $\infty$ & 7 & $\infty$ & $\infty$ \\
      \hline
      $\colon$ & $\colon$ & $\colon$ & 10 & {\bf \framebox{7}} & $\infty$ & 16 \\
      \hline
      $\colon$ & $\colon$ & $\colon$ & {\bf \framebox{10}} & $\colon$ & 14 & {\bf 12} \\
      \hline
      $\colon$ & $\colon$ & $\colon$ & $\colon$ & $\colon$ & 14 & {\bf \framebox{12}} \\
      \hline
      $\colon$ & $\colon$ & $\colon$ & $\colon$ & $\colon$ & {\bf \framebox{14}} & $\colon$ \\
      \hline
      $\colon$ & $\colon$ & $\colon$ & $\colon$ & $\colon$ & $\colon$ & $\colon$ \\
      \hline
    \end{tabular}
    \caption{Разстояния с начален връх $b$}
  \end{subtable}
  \caption{Алгоритъм на Дейкстра}
  \label{fig:dijkstra-table}
\end{figure}


\begin{figure}[!htbp]
  \index{Дейкстра!алгоритъм}
  

\subfigure[Започваме от съседите на $a$]{
  \begin{tikzpicture}[scale=0.9]
    % nodes
    \foreach \nodename/\x/\y/\value/\direction/\navigate/\color in { 
      a/0/0/0/above/north/green, 
      b/-1/1/\infty/left/west/black, 
      c/2.5/1/\infty/above/north/black, 
      d/2/-0.5/\infty/below/south/black,
      e/-1/-0.5/\infty/below/south/black, 
      f/0.5/2/\infty/above/north/black,
      g/0/-1.8/\infty/below/south/black, 
      h/2.5/-1.8/\infty/right/east/black}
    {
        \node[vertex, nodedecorate, fill=\color!25] (\nodename) at (\x,\y) {$\value$};
        \node [\direction] at (\nodename.\navigate) {$\nodename$};
      };
      % edges
      \path
      \foreach \startnode/\endnode/\direction/\angle/\weight in {
        e/b/left/15/5, e/g/below/-15/3, d/g/below/15/2, 
        g/a/left/15/2, a/g/right/30/9, f/c/below/-15/1,
        c/f/above/-30/5, c/h/right/15/2, c/d/above/-15/1,
        f/d/left/-15/4, a/b/below/0/2, b/f/above/0/1, 
        a/d/above/0/8}
      {
        (\startnode) edge[arrowdecorate,bend left=\angle] node[\direction] {$\weight$} (\endnode)
      };
    \end{tikzpicture}
  }
  \subfigure[$b$ е най-близко до $a$]{
    \begin{tikzpicture}[scale=0.9]
      %nodes
      \foreach \nodename/\x/\y/\value/\direction/\navigate/\color in { 
        a/0/0/0/above/north/green, 
        b/-1/1/2/left/west/yellow, 
        c/2.5/1/\infty/above/north/black, 
        d/2/-0.5/8/below/south/yellow,
        e/-1/-0.5/\infty/below/south/black, 
        f/0.5/2/\infty/above/north/black, 
        g/0/-1.8/9/below/south/yellow, 
        h/2.5/-1.8/\infty/right/east/black}
      {
        \node[vertex, nodedecorate, fill=\color!25] (\nodename) at (\x,\y) {$\value$};
        \node [\direction] at (\nodename.\navigate) {$\nodename$};
      };

      \path
      \foreach \startnode/\endnode/\direction/\angle in {
        a/g/right/30, a/b/above/0, a/d/above/0}
      {
        (\startnode) edge[selected edge,bend left=\angle] node[\direction] {} (\endnode)
      };
      %edges
      \path
      \foreach \startnode/\endnode/\direction/\angle/\weight in {
        e/b/left/15/5, e/g/below/-15/3, d/g/below/15/2, g/a/left/15/2, a/g/right/30/9, f/c/below/-15/1, c/f/above/-30/5,
        c/h/right/15/2, c/d/above/-15/1, f/d/left/-15/4, a/b/below/0/2, b/f/above/0/1,  a/d/above/0/8}
      {
        (\startnode) edge[arrowdecorate,bend left=\angle] node[\direction] {$\weight$} (\endnode)
      };
    \end{tikzpicture}
  }
  \subfigure[Най-къс път до $f$]{
    \begin{tikzpicture}[scale=0.9]
      %nodes
      \foreach \nodename/\x/\y/\value/\direction/\navigate/\color in { 
        a/0/0/0/above/north/green, 
        b/-1/1/2/left/west/green,
        c/2.5/1/\infty/above/north/black, 
        d/2/-0.5/8/below/south/yellow,
        e/-1/-0.5/\infty/below/south/black, 
        f/0.5/2/3/above/north/yellow, 
        g/0/-1.8/9/below/south/yellow, 
        h/2.5/-1.8/\infty/right/east/black}
      {
        \node[vertex, nodedecorate, fill=\color!25] (\nodename) at (\x,\y) {$\value$};
        \node [\direction] at (\nodename.\navigate) {$\nodename$};
      };
      \path
      (a) edge[sure edge] node[] {} (b);
      \path
      \foreach \startnode/\endnode/\direction/\angle in {
        a/g/right/30, a/d/above/0}
      {
        (\startnode) edge[selected edge,bend left=\angle] node[\direction] {} (\endnode)
      };
      \path
      \foreach \startnode/\endnode/\direction/\angle in {
         b/f/above/0}
      {
        (\startnode) edge[selected edge, bend left=\angle] node[\direction] {} (\endnode)
      };
      %edges
      \path
      \foreach \startnode/\endnode/\direction/\angle/\weight in {
        e/b/left/15/5, e/g/below/-15/3, d/g/below/15/2, g/a/left/15/2, a/g/right/30/9, f/c/below/-15/1, c/f/above/-30/5,
        c/h/right/15/2, c/d/above/-15/1, f/d/left/-15/4, a/b/below/0/2, b/f/above/0/1,  a/d/above/0/8}
      {
        (\startnode) edge[arrowdecorate,bend left=\angle] node[\direction] {$\weight$} (\endnode)
      };
    \end{tikzpicture}
  }
  \subfigure[По-къс път до $d$ и $c$]{
    \begin{tikzpicture}[scale=0.9]
      % nodes
      \foreach \nodename/\x/\y/\value/\direction/\navigate/\color in { 
        a/0/0/0/above/north/green,
        b/-1/1/2/left/west/green,
        c/2.5/1/4/above/north/yellow, 
        d/2/-0.5/7/below/south/yellow,
        e/-1/-0.5/\infty/below/south/black, 
        f/0.5/2/3/above/north/green, 
        g/0/-1.8/9/below/south/yellow, 
        h/2.5/-1.8/\infty/right/east/black}
      {
        \node[vertex, nodedecorate, fill=\color!25] (\nodename) at (\x,\y) {$\value$};
        \node [\direction] at (\nodename.\navigate) {$\nodename$};
      };
      \path
      (a) edge[sure edge] node[] {} (b)
      (b) edge[sure edge] node[] {} (f);
      
      \path
      \foreach \startnode/\endnode/\direction/\angle in {
        a/d/above/0}
      {
        (\startnode) edge[path edge,bend left=\angle] node[\direction] {} (\endnode)
      };
      \path
      \foreach \startnode/\endnode/\direction/\angle in {f/c/below/-15/1, f/d/left/-15/4, a/g/right/30}
      {
        (\startnode) edge[selected edge, bend left=\angle] node[\direction] {} (\endnode)
      };
      %edges
      \path
      \foreach \startnode/\endnode/\direction/\angle/\weight in {
        e/b/left/15/5, e/g/below/-15/3, d/g/below/15/2, g/a/left/15/2, a/g/right/30/9, f/c/below/-15/1, c/f/above/-30/5,
        c/h/right/15/2, c/d/above/-15/1, f/d/left/-15/4, a/b/below/0/2, b/f/above/0/1,  a/d/above/0/8}
      {
        (\startnode) edge[arrowdecorate,bend left=\angle] node[\direction] {$\weight$} (\endnode)
      };
    \end{tikzpicture}
  }
  \subfigure[По-къс път до $d$ и $h$]{
    \begin{tikzpicture}[scale=0.9]
      % nodes
      \foreach \nodename/\x/\y/\value/\direction/\navigate/\color in { 
        a/0/0/0/above/north/green,
        b/-1/1/2/left/west/green,
        c/2.5/1/4/above/north/green,
        d/2/-0.5/5/below/south/yellow,
        e/-1/-0.5/\infty/below/south/black, 
        f/0.5/2/3/above/north/green, 
        g/0/-1.8/9/below/south/yellow, 
        h/2.5/-1.8/6/right/east/yellow}
      {
        \node[vertex, nodedecorate, fill=\color!25] (\nodename) at (\x,\y) {$\value$};
        \node [\direction] at (\nodename.\navigate) {$\nodename$};
      };
      \path
      (a) edge[sure edge] node[] {} (b)
      (b) edge[sure edge] node[] {} (f)
      (f) edge[sure edge, bend left=-15] node[] {} (c);
      
      \path
      \foreach \startnode/\endnode/\direction/\angle in {f/d/left/-15/4, a/d/above/0
        }
      {
        (\startnode) edge[path edge,bend left=\angle] node[\direction] {} (\endnode)
      };
      \path
      \foreach \startnode/\endnode/\direction/\angle in {c/d/above/-15/1,  c/f/above/-30/5, c/h/right/15/2, a/g/right/30}
      {
        (\startnode) edge[selected edge, bend left=\angle] node[\direction] {} (\endnode)
      };
      %edges
      \path
      \foreach \startnode/\endnode/\direction/\angle/\weight in {
        e/b/left/15/5, e/g/below/-15/3, d/g/below/15/2, g/a/left/15/2, a/g/right/30/9, f/c/below/-15/1, c/f/above/-30/5,
        c/h/right/15/2, c/d/above/-15/1, f/d/left/-15/4, a/b/below/0/2, b/f/above/0/1,  a/d/above/0/8}
      {
        (\startnode) edge[arrowdecorate,bend left=\angle] node[\direction] {$\weight$} (\endnode)
      };
    \end{tikzpicture}
  }
  \subfigure[По-къс път до $g$]{
    \begin{tikzpicture}[scale=0.9]
      %nodes
      \foreach \nodename/\x/\y/\value/\direction/\navigate/\color in { 
        a/0/0/0/above/north/green, 
        b/-1/1/2/left/west/green,
        c/2.5/1/4/above/north/green,
        d/2/-0.5/5/below/south/green,
        e/-1/-0.5/\infty/below/south/black, 
        f/0.5/2/3/above/north/green, 
        g/0/-1.8/7/below/south/yellow,
        h/2.5/-1.8/6/right/east/yellow}
      {
        \node[vertex, nodedecorate, fill=\color!25] (\nodename) at (\x,\y) {$\value$};
        \node [\direction] at (\nodename.\navigate) {$\nodename$};
      };
      \path
      (a) edge[sure edge] node[] {} (b)
      (b) edge[sure edge] node[] {} (f)
      (f) edge[sure edge, bend left=-15] node[] {} (c)
      (c) edge[sure edge, bend left=-15] node[] {} (d);
      
      \path
      \foreach \startnode/\endnode/\direction/\angle in {f/d/left/-15, a/g/right/30, a/d/above/0, c/f/above/-30
        }
      {
        (\startnode) edge[path edge,bend left=\angle] node[] {} (\endnode)
      };
      \path
      \foreach \startnode/\endnode/\direction/\angle in {d/g/below/15/2, c/h/right/15}
      {
        (\startnode) edge[selected edge, bend left=\angle] node[\direction] {} (\endnode)
      };
      %edges
      \path
      \foreach \startnode/\endnode/\direction/\angle/\weight in {
        e/b/left/15/5, e/g/below/-15/3, d/g/below/15/2, g/a/left/15/2, a/g/right/30/9, f/c/below/-15/1, c/f/above/-30/5,
        c/h/right/15/2, c/d/above/-15/1, f/d/left/-15/4, a/b/below/0/2, b/f/above/0/1,  a/d/above/0/8}
      {
        (\startnode) edge[arrowdecorate,bend left=\angle] node[\direction] {$\weight$} (\endnode)
      };
    \end{tikzpicture}
  }
  \subfigure[$h$ е задънена улица]{
    \begin{tikzpicture}[scale=0.9]
      %nodes
      \foreach \nodename/\x/\y/\value/\direction/\navigate/\color in { 
        a/0/0/0/above/north/green, 
        b/-1/1/2/left/west/green,
        c/2.5/1/4/above/north/green, 
        d/2/-0.5/5/below/south/green,
        e/-1/-0.5/\infty/below/south/black, 
        f/0.5/2/3/above/north/green, 
        g/0/-1.8/7/below/south/yellow,
        h/2.5/-1.8/6/right/east/green}
      {
        \node[vertex, nodedecorate, fill=\color!25] (\nodename) at (\x,\y) {$\value$};
        \node [\direction] at (\nodename.\navigate) {$\nodename$};
      };
      \path
      (a) edge[sure edge] node[] {} (b)
      (b) edge[sure edge] node[] {} (f)
      (f) edge[sure edge, bend left=-15] node[] {} (c)
      (c) edge[sure edge, bend left=-15] node[] {} (d)
      (c) edge[sure edge, bend left=15] node[] {} (h);
      
      \path
      \foreach \startnode/\endnode/\direction/\angle in {f/d/left/-15, a/g/right/30, a/d/above/0, c/f/above/-30
        }
      {
        (\startnode) edge[path edge,bend left=\angle] node[] {} (\endnode)
      };
      \path
      \foreach \startnode/\endnode/\direction/\angle in {d/g/below/15/2}
      {
        (\startnode) edge[selected edge, bend left=\angle] node[\direction] {} (\endnode)
      };
      %edges
      \path
      \foreach \startnode/\endnode/\direction/\angle/\weight in {
        e/b/left/15/5, e/g/below/-15/3, d/g/below/15/2, g/a/left/15/2, a/g/right/30/9, f/c/below/-15/1, c/f/above/-30/5,
        c/h/right/15/2, c/d/above/-15/1, f/d/left/-15/4, a/b/below/0/2, b/f/above/0/1,  a/d/above/0/8}
      {
        (\startnode) edge[arrowdecorate,bend left=\angle] node[\direction] {$\weight$} (\endnode)
      };
    \end{tikzpicture}
  }
  \subfigure[$e$ не е достижим]{
    \begin{tikzpicture}[scale=0.9]
      %nodes
      \foreach \nodename/\x/\y/\value/\direction/\navigate/\color in { 
        a/0/0/0/above/north/green,
        b/-1/1/2/left/west/green,
        c/2.5/1/4/above/north/green, 
        d/2/-0.5/5/below/south/green,
        e/-1/-0.5/\infty/below/south/black, 
        f/0.5/2/3/above/north/green, 
        g/0/-1.8/7/below/south/green,
        h/2.5/-1.8/6/right/east/green}
      {
        \node[vertex, nodedecorate, fill=\color!25] (\nodename) at (\x,\y) {$\value$};
        \node [\direction] at (\nodename.\navigate) {$\nodename$};
      };
      \path
      (a) edge[sure edge] node[] {} (b)
      (b) edge[sure edge] node[] {} (f)
      (f) edge[sure edge, bend left=-15] node[] {} (c)
      (c) edge[sure edge, bend left=-15] node[] {} (d)
      (c) edge[sure edge, bend left=15] node[] {} (h)
      (d) edge[sure edge, bend left=15] node[] {} (g);
      
      \path
      \foreach \startnode/\endnode/\direction/\angle in {f/d/left/-15, a/g/right/30, a/d/above/0, c/f/above/-30
        }
      {
        (\startnode) edge[path edge,bend left=\angle] node[] {} (\endnode)
      };
      \path
      \foreach \startnode/\endnode/\direction/\angle in {g/a/left/15}
      {
        (\startnode) edge[selected edge, bend left=\angle] node[\direction] {} (\endnode)
      };
      %edges
      \path
      \foreach \startnode/\endnode/\direction/\angle/\weight in {
        e/b/left/15/5, e/g/below/-15/3, d/g/below/15/2, g/a/left/15/2, a/g/right/30/9, f/c/below/-15/1, c/f/above/-30/5,
        c/h/right/15/2, c/d/above/-15/1, f/d/left/-15/4, a/b/below/0/2, b/f/above/0/1,  a/d/above/0/8}
      {
        (\startnode) edge[arrowdecorate,bend left=\angle] node[\direction] {$\weight$} (\endnode)
      };
    \end{tikzpicture}
  }
  \subfigure[Краен резултат]{
    \begin{tikzpicture}[scale=0.9]
      %nodes
      \foreach \nodename/\x/\y/\value/\direction/\navigate/\color in { 
        a/0/0/0/above/north/green, b/-1/1/2/left/west/green,
        c/2.5/1/4/above/north/green, d/2/-0.5/5/below/south/green,
        e/-1/-0.5/\infty/below/south/black, f/0.5/2/3/above/north/green, 
        g/0/-1.8/7/below/south/green, h/2.5/-1.8/6/right/east/green}
      {
        \node[vertex, nodedecorate, fill=\color!25] (\nodename) at (\x,\y) {$\value$};
        \node [\direction] at (\nodename.\navigate) {$\nodename$};
      };
      \path
      (a) edge[sure edge] node[] {} (b)
      (b) edge[sure edge] node[] {} (f)
      (f) edge[sure edge, bend left=-15] node[] {} (c)
      (c) edge[sure edge, bend left=-15] node[] {} (d)
      (c) edge[sure edge, bend left=15] node[] {} (h)
      (d) edge[sure edge, bend left=15] node[] {} (g);
      
      \path
      \foreach \startnode/\endnode/\direction/\angle in {f/d/left/-15, a/g/right/30, a/d/above/0, c/f/above/-30, g/a/above/15
        }
      {
        (\startnode) edge[path edge,bend left=\angle] node[] {} (\endnode)
      };
      \path
      \foreach \startnode/\endnode/\direction/\angle in {}
      {
        (\startnode) edge[selected edge, bend left=\angle] node[\direction] {} (\endnode)
      };
      %edges
      \path
      \foreach \startnode/\endnode/\direction/\angle/\weight in {
        e/b/left/15/5, e/g/below/-15/3, d/g/below/15/2, g/a/left/15/2, a/g/right/30/9, f/c/below/-15/1, c/f/above/-30/5,
        c/h/right/15/2, c/d/above/-15/1, f/d/left/-15/4, a/b/below/0/2, b/f/above/0/1,  a/d/above/0/8}
      {
        (\startnode) edge[arrowdecorate,bend left=\angle] node[\direction] {$\weight$} (\endnode)
      };
    \end{tikzpicture}
  }

%%% Local Variables: 
%%% mode: latex
%%% TeX-master: "discrete-math"
%%% End: 

  \caption{Алгоритъм на Дейкстра запазващ минималните пътища}
  \label{fig:dijkstra-graph}
\end{figure}

\begin{problem}
  Дайте пример за насочен граф с отрицателен цикъл, при който алгоритъмът на Дейкстра не дава правилния резултат.
\end{problem}

\newpage
\subsection{Алгоритъм на Белман-Форд}\index{Белман-Форд!алгоритъм}

Алгоритъмът на Дейкстра работи само за графи $G = (V,E,c)$ с {\em положителни} тегла по ребрата.
Сега ще разгледаме един алгоритъм, който работи и за графи с отрицателни тегла по ребрата.
Задачата отново е да намерим минималните разстояния на пътищата с начало върха $s$, но
искаме също така алгоритъмът да отговаря на въпроса дали има отрицателен цикъл в графа. 
Ако такъв съществува, то няма решение на проблема. (Защо?)
Ако отрицателен цикъл не съществува, то алгоритъмът намира пътища в графа с минимални тегла от върха $s$
до всички достижими върхове в графа.


\begin{algorithm}
  \caption{Белман-Форд}
  \label{alg:belman-ford}
  
  \begin{algorithmic}[1]
    \STATE INIT(s)
    \FOR{$i:=1$ to $\abs{V}-1$}
    \FORALL{$(u,v)\in E$}
    \STATE UPDATE(u,v)
    % \ENSURE{$dist[v] \geq \delta(s,v)$}
    \ENDFOR
    \ENDFOR
    
    \COMMENT{Проверка за отрицателен цикъл}
    \FORALL{$(u,v)\in E$}
    \IF {$dist[v] > dist[u] + w(u,v)$}
    \RETURN \FALSE
    \ENDIF
    \ENDFOR
    \RETURN \TRUE
    
  \end{algorithmic}
\end{algorithm}

% \begin{enumerate}[1)]
%   \item
%     Дефинираме функция $\delta:V\to\R^+\cup\{\infty\}$ като
%     \[\delta(u)=
%     \begin{cases}
%       0 & \text{, ако } u = s\\
%       \infty & \text{, иначе}
%     \end{cases}
%     \]
%   Тъй като $V$ е крайно множество, можем да представим $\delta$ като масив.
%   \item
%     За всяко ребро $(u,v)\in E$,
%         \[\delta(v) :=
%         \begin{cases}
%           \delta(u) + c(u,v) & \text{, ако } \delta(v) > \delta(u) + c(u,v)\\
%           \delta(v) & \text{, иначе}
%         \end{cases}
%         \]
%   \item
%     Стъпка 2) се изпълнява общо $\nu- 1$ пъти.
%   \item
%     За всяко ребро $(u,v)\in E$, проверяваме дали $\delta(v) > \delta(u) + c(u,v)$.
%     Ако намерим такова ребро, то процедурата връща $FALSE$, иначе връща $TRUE$.
% \end{enumerate}

\begin{prop}
  \label{prop:bellman-ford}
  Нека графът $G$ няма отрицателни цикли, които са достижими от $s$.
  Тогава след изпълнение на алгоритъма на Белман-Форд получаваме, че
  за всички $v \in V$ достижими от $s$, 
  \[dist[v] = \delta(s,v).\]
\end{prop}
\begin{proof}
  Да разгледаме $s \stackrel{p}{\leadsto} v$, където $p = (v_0,\dots,v_k)$ е път с минимално тегло в $G$.
  Понеже в пътища с минимална дължина няма цикли, то $k \leq \abs{V} - 1$.
  Тогава според Твърдение \ref{prop:path-update}, 
  след $i$-тата итерация на FOR цикъла, $dist[v_i] = \delta(s,v_i)$.
  Така получаваме, че най-накрая $dist[v] = \delta(s,v)$.
\end{proof}
\begin{cor}
  \label{cor:bellman-ford}
  При същите предположения за графа $G$,
  за всяко $v \in V$, 
  има път $s \leadsto v$ точно тогава, когато след приключване на алгоритъма е изпълнено $dist[v] < \infty$.
\end{cor}
\begin{proof}
  Ако има път $p$, $s \stackrel{p}{\leadsto} v$, то
  според твърдението $dist[v] = \delta(s,v) < \infty$.
  За другата посока, нека $dist[v] < \infty$, но да допуснем, че няма път от $s$ до $v$.
  Но тогава от Твърдение \ref{prop:no-path} следва, че $dist[v]  = \infty$,
  което е противоречие.
\end{proof}


\begin{thm}
  \label{th:bellman-ford}
  Ако $G$ няма отрицателни цикли достижими от $s$, то
  алгоритъмът на Белман-Форд връща TRUE, $(\forall v\in V)[dist[v] = \delta(s,v)]$,
  и $G_{pred}$ е дърво с корен $s$, което съдържа пътища с минимални тегла.

  Ако $G$ има отрицателни цикли достижими от $s$, то
  алгоритъмът на Белман-Форд връща FALSE.
\end{thm}
\begin{proof}
  \begin{enumerate}[a)]
  \item 
    Нека $G$ не съдържа цикъл с отрицателно тегло, достижим от $s$.
    Ако $v$ е достижим от $s$, то според Твърдение \ref{prop:bellman-ford}, 
    след изпълнение на алгоритъма
    \[dist[v] = \delta(s,v).\]

    Ако $v$ не е достижим от $s$, то според Твърдение \ref{prop:no-path},
    след изпълнение на алгоритъма
    \[dist[v] = \infty = \delta(s,v).\]

    Понеже $(\forall v\in V)[dist[v] = \delta(s,v)]$, от Твърдение \ref{prop:tree-shortest-path} следва, че
    $G_{pred}$ е дърво с корен $s$, което съдържа пътища с минимални тегла.
    
    Като използваме Твърдение \ref{prop:triangle} лесно се вижда, че алгоритъмът връща TRUE, 
  \item
    Нека $G$ съдържа цикъл с отрицателно тегло, достижим от $s$.
    Нека един такъв цикъл е $c = (v_0,\dots,v_k)$, $v_0 = v_k$.
    Тогава
    \[w(c) = \sum^k_{i=1}w(v_{i-1},v_i) < 0.\]
    Да допуснем, че алгоритъмът връща TRUE. Тогава за всяко $i = 1,\dots, k$, 
    \[dist[v_i] \leq dist[v_{i-1}] + w(v_{i-1},v_i)\]
    и като сумираме, 
    \[\sum^{k}_{i=1} dist[v_i] \leq \sum^{k}_{i=1} dist[v_{i-1}] + \sum^{k}_{i=1}w(v_{i-1},v_i).\]
    Тъй като $v_0 = v_k$, 
    \[\sum^{k}_{i=1} dist[v_i] = \sum^{k}_{i=1}dist[v_{i-1}].\]
    Получаваме, че \[0 \leq \sum^{k}_{i=1}w(v_{i-1},v_i),\] което е противоречие с отицателността на цикъла.
  \end{enumerate}
\end{proof}

Фигура \ref{fig:bellman-ford-negative-cycle} илюстрира случая за цикъл с отрицателно тегло.
Както забелязахме при алгоритъма на Дейкстра, и тук можем да намерим не само дължините на най-късите пътища, но
и спъсъка на ребрата, участващи в тях. Фигура \ref{fig:bellman-ford-graph} илюстрира този проблем. 
Останалите накрая оцветени в синьо ребра участват в най-късите пътища.



\begin{figure}[!htbp]
  \begin{subfigure}[b]{0.5\textwidth}
    \begin{tikzpicture}
      [nodedecorate/.style={shape=circle,inner sep=2pt,draw,thick},%
      arrowdecorate/.style={->,>=stealth,thick}]
      %% nodes or vertices
      
      \foreach \nodename/\x/\y/\direction/\navigate in { a/0/0/below/south,
        b/6/0/below/south, c/4.5/0/below/south, d/3/0/below/south, e/1.5/0/below/south}
      {
        \node (\nodename) at (\x,\y) [nodedecorate] {};
        \node [\direction] at (\nodename.\navigate) {$\nodename$};
      }
      %% edges or lines
      \path
      \foreach \startnode/\endnode/\direction/\angle/\weight in {
        a/e/above/15/1, d/e/above/-25/-1, e/d/below/-15/1,  d/c/above/15/1, c/b/above/15/1, b/e/below/60/-4}
      {
        (\startnode) edge[arrowdecorate,bend left=\angle] node[\direction] {$\weight$} (\endnode)
      };
      ;
    \end{tikzpicture}
    \caption{Граф с отрицателен цикъл}
  \end{subfigure}
  \quad
  \begin{subtable}[b]{0.5\textwidth}
    \begin{tabular}{|c|c|c|c|c|}
      \hline
      $\delta(a)$ & $\delta(b)$ & $\delta(c)$ & $\delta(d)$ & $\delta(e)$ \\
      \hline
      0 & $\infty$ & $\infty$ & $\infty$ & {\bf \framebox{1}}\\
      \hline
      \hline
      $\colon$ & \framebox{$\infty$} & $\infty$ & $\infty$ & 1\\
      $\colon$ & $\infty$ & \framebox{$\infty$} & $\infty$ & 1\\
      $\colon$ & $\infty$ & $\infty$ & {\bf \framebox{2}} & 1\\
      $\colon$ & $\infty$ & $\infty$ & 2 & \framebox{1}\\
      \hline\hline
      $\colon$ & \framebox{$\infty$} & $\infty$ & 2 & 1\\
      $\colon$ & $\infty$ & {\bf \framebox{3}} & 2 & 1\\
      $\colon$ & $\infty$ & 3 & \framebox{2} & 1\\
      $\colon$ & $\infty$ & $\infty$ & 2 & \framebox{1}\\
      \hline\hline
      $\colon$ & {\bf \framebox{4}} & 3 & 2 & 1\\
      $\colon$ & 4 & \framebox{3} & 2 & 1\\
      $\colon$ & 4 & 3 & \framebox{2} & 1\\
      $\colon$ & 4 & 3 & 2 & {\bf \framebox{0}}\\
      \hline\hline
    \end{tabular}
    \caption{Изпълнение на алгоритъма}
  \end{subtable}
  \caption{Алгоритъм на Белман-Форд върху ориентиран граф с отрицателен цикъл,
  като ребрата са подредени лексикографски: $\pair{a,e}, \pair{b,e}, \pair{c,b}, \pair{d,c}, \pair{d,e}, \pair{e, d}$}
  \label{fig:bellman-ford-negative-cycle}
\end{figure}


\begin{figure}[!htbp]
  
\begin{subfigure}[b]{0.3\textwidth}
    \begin{tikzpicture}[scale=0.9]
      %% nodes or vertices
      
      \foreach \nodename/\value/\x/\y/\direction/\navigate/\color in { 
        s/0/0/0/left/west/green, 
        x/\infty/3.5/1.5/above/north/black,
        y/\infty/1/-1.5/below/south/black,
        t/\infty/1/1.5/above/north/black, 
        z/\infty/3.5/-1.5/below/south/black}
      {
        \node[vertex, nodedecorate, fill=\color!25] (\nodename) at (\x,\y) {$\value$};
        \node [\direction] at (\nodename.\navigate) {$\nodename$};
      }
      % edges or lines
      \path
      \foreach \startnode/\endnode/\direction/\angle/\weight in {
        s/t/left/15/6, s/y/left/-25/7, y/z/below/0/9,  t/y/left/0/8, t/x/above/30/5, x/t/above/30/-2, z/x/right/0/7,
        z/s/below/0/2, y/x/below/-3, t/z/right/15/-4
      }
      {
        (\startnode) edge[arrowdecorate,bend left=\angle] node[\direction] {$\weight$} (\endnode)
      };
      ;
    \end{tikzpicture}
    \caption{Начален връх е $s$}
  \end{subfigure}
  \quad
  \begin{subfigure}[b]{0.3\textwidth}
    \begin{tikzpicture}[scale=0.9]
      \foreach \nodename/\value/\x/\y/\direction/\navigate/\color in { 
        s/0/0/0/left/west/green, 
        x/\infty/3.5/1.5/above/north/black,
        y/7/1/-1.5/below/south/red,
        t/6/1/1.5/above/north/red, 
        z/\infty/3.5/-1.5/below/south/black}
      {
        \node[vertex, nodedecorate, fill=\color!25] (\nodename) at (\x,\y) {$\value$};
        \node [\direction] at (\nodename.\navigate) {$\nodename$};
      }
    
    \path
    \foreach \startnode/\endnode/\angle in {
      s/t/15, s/y/-25}
    {
      (\startnode) edge[selected edge, bend left=\angle] node[] {} (\endnode)
    };
    
    %edges
    \path
    \foreach \startnode/\endnode/\direction/\angle/\weight in {
      s/t/left/15/6, s/y/left/-25/7, y/z/below/0/9,  t/y/left/0/8, t/x/above/30/5, x/t/above/30/-2, z/x/right/0/7,
      z/s/below/0/2, y/x/below/-3, t/z/right/15/-4}
    {
      (\startnode) edge[arrowdecorate,bend left=\angle] node[\direction] {$\weight$} (\endnode)
    };
    

  \end{tikzpicture}
  \caption{Започваме със съседите на $s$}
  \end{subfigure}
  \quad
  \begin{subfigure}[b]{0.3\textwidth}
    \begin{tikzpicture}[scale=0.9]
      
      \foreach \nodename/\value/\x/\y/\direction/\navigate/\color in { 
        s/0/0/0/left/west/green, 
        x/4/3.5/1.5/above/north/red,
        y/7/1/-1.5/below/south/blue,
        t/6/1/1.5/above/north/blue, 
        z/2/3.5/-1.5/below/south/red}
      {
        \node[vertex, nodedecorate, fill=\color!25] (\nodename) at (\x,\y) {$\value$};
        \node [\direction] at (\nodename.\navigate) {$\nodename$};
      }
    
    \path
    \foreach \startnode/\endnode/\angle in {
      s/t/15, s/y/-25}
    {
      (\startnode) edge[path edge, bend left=\angle] node[] {} (\endnode)
    };


    %edges or lines
    \path
      (y) edge[selected edge] node[below] {} (x)
      (t) edge[selected edge, bend left=15] node[left] {} (z);

    \path
    \foreach \startnode/\endnode/\direction/\angle/\weight in {
      s/t/left/15/6, s/y/left/-25/7, y/z/below/0/9,  t/y/left/0/8, t/x/above/30/5, x/t/above/30/-2, z/x/right/0/7,
      z/s/below/0/2, y/x/below/0/-3, t/z/right/15/-4
    }
    {
      (\startnode) edge[arrowdecorate,bend left=\angle] node[\direction] {$\weight$} (\endnode)
    };
    ;

  \end{tikzpicture}
  \caption{Продължаваме с $x$ и $z$}
  \end{subfigure}
\quad
\begin{subfigure}[b]{0.3\textwidth}
  \begin{tikzpicture}[scale=0.9]
    
    \foreach \nodename/\value/\x/\y/\direction/\navigate/\color in { 
      s/0/0/0/left/west/green, 
      x/4/3.5/1.5/above/north/blue,
      y/7/1/-1.5/below/south/blue,
      t/2/1/1.5/above/north/red, 
      z/2/3.5/-1.5/below/south/blue}
    {
      \node[vertex, nodedecorate, fill=\color!25] (\nodename) at (\x,\y) {$\value$};
      \node [\direction] at (\nodename.\navigate) {$\nodename$};
    }
    
    \path
    \foreach \startnode/\endnode/\angle in {
      s/y/-25, y/x/0, t/z/15}
    {
      (\startnode) edge[path edge, bend left=\angle] node[] {} (\endnode)
    };

    \path
    \foreach \startnode/\endnode/\angle in {
      x/t/30
    }
    {
      (\startnode) edge[selected edge,bend left=\angle] node[] {} (\endnode)
    };
    %edges or lines
    \path
    \foreach \startnode/\endnode/\direction/\angle/\weight in {
      s/t/left/15/6, s/y/left/-25/7, y/z/below/0/9,  t/y/left/0/8, t/x/above/30/5, x/t/above/30/-2, z/x/right/0/7,
      z/s/below/0/2, y/x/below/0/-3, t/z/right/15/-4
    }
    {
      (\startnode) edge[arrowdecorate,bend left=\angle] node[\direction] {$\weight$} (\endnode)
    };
    ;
  \end{tikzpicture}
  \caption{По-кратък път до $t$}
\end{subfigure}
\quad
\begin{subfigure}[b]{0.3\textwidth}
  \begin{tikzpicture}[scale=0.9]
    %% nodes or vertices
    \foreach \nodename/\value/\x/\y/\direction/\navigate/\color in { 
      s/0/0/0/left/west/green, 
      x/4/3.5/1.5/above/north/blue,
      y/7/1/-1.5/below/south/blue,
      t/2/1/1.5/above/north/blue, 
      z/-2/3.5/-1.5/below/south/red}
    {
      \node[vertex, nodedecorate, fill=\color!25] (\nodename) at (\x,\y) {$\value$};
      \node [\direction] at (\nodename.\navigate) {$\nodename$};
    }
    \path
    \foreach \startnode/\endnode/\angle in {
      s/y/-25, y/x/0, x/t/30}
    {
      (\startnode) edge[path edge, bend left=\angle] node[] {} (\endnode)
    };
    \path
    \foreach \startnode/\endnode/\angle in {
      t/z/15/
    }
    {
      (\startnode) edge[selected edge,bend left=\angle] node[] {} (\endnode)
    };
    %edges or lines
    \path
    \foreach \startnode/\endnode/\direction/\angle/\weight in {
      s/t/left/15/6, s/y/left/-25/7, y/z/below/0/9,  t/y/left/0/8, t/x/above/30/5, x/t/above/30/-2, z/x/right/0/7,
      z/s/below/0/2, y/x/below/0/-3, t/z/right/15/-4
    }
    {
      (\startnode) edge[arrowdecorate,bend left=\angle] node[\direction] {$\weight$} (\endnode)
    };
  \end{tikzpicture}
  \caption{По-кратък път до $z$}
  \end{subfigure}
\quad
\begin{subfigure}[b]{0.3\textwidth}
  \begin{tikzpicture}[scale=0.9]
    %% nodes or vertices
    \foreach \nodename/\value/\x/\y/\direction/\navigate/\color in { 
      s/0/0/0/left/west/green, 
      x/4/3.5/1.5/above/north/blue,
      y/7/1/-1.5/below/south/blue,
      t/2/1/1.5/above/north/blue, 
      z/-2/3.5/-1.5/below/south/blue}
    {
      \node[vertex, nodedecorate, fill=\color!25] (\nodename) at (\x,\y) {$\value$};
      \node [\direction] at (\nodename.\navigate) {$\nodename$};
    }
    \path
    \foreach \startnode/\endnode/\angle in {
      s/y/-25, y/x/0, x/t/30, t/z/15}
    {
      (\startnode) edge[path edge, bend left=\angle] node[] {} (\endnode)
    };
    %edges or lines
    \path
    \foreach \startnode/\endnode/\direction/\angle/\weight in {
      s/t/left/15/6, s/y/left/-25/7, y/z/below/0/9,  t/y/left/0/8, t/x/above/30/5, x/t/above/30/-2, z/x/right/0/7,
      z/s/below/0/2, y/x/below/0/-3, t/z/right/15/-4
    }
    {
      (\startnode) edge[arrowdecorate,bend left=\angle] node[\direction] {$\weight$} (\endnode)
    };
  \end{tikzpicture}
  \caption{Край на процедурата.}
\end{subfigure}


%%% Local Variables: 
%%% mode: latex
%%% TeX-master: "discrete-math"
%%% End: 

  \index{Белман-Форд!алгоритъм}
  \caption{Алгоритъм на Белман-Форд запазващ минималните пътища}
  \label{fig:bellman-ford-graph}
\end{figure}


%% стр. 654
\begin{problem}
  Променете алгоритъма на Белман-Форд, така че $\delta(v) = -\infty$ за всеки връх $v$, 
  за който има отрицателен цикъл по някой път от началния връх $s$ до $v$.
\end{problem}



%%% Local Variables: 
%%% mode: latex
%%% TeX-master: "discrete-math"
%%% End: 


% \backmatter

% \bibliographystyle{amsalpha}
\bibliographystyle{amsplain}
\bibliography{discrete-math}

\printindex
% \listofalgorithms



\end{document}


%%% Local Variables: 
%%% mode: latex
%%% TeX-master: "discrete-math"
%%% End: 

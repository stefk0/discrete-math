\documentclass[a4paper,twoside,openright]{book}
\usepackage[margin=4cm]{geometry}
\usepackage[english,bulgarian]{babel}
%\usepackage{ucs}
\usepackage[utf8]{inputenc}

\usepackage{amssymb}
\usepackage{amsmath}
\usepackage{mathrsfs}
\usepackage{latexsym}
\usepackage{amsthm}
\usepackage{paralist}
\usepackage{enumerate}
\usepackage{makeidx}

% \usepackage{algorithm}
% \usepackage{algorithmic}

%%%%%%%%%%%%%%% TIKZ Package %%%%%%%%%%%%%%%%%%%%%%%
% \usepackage{tikz}
% \usepackage{pgf}
% \usetikzlibrary{arrows,automata}
%%%%%%%%%%%%%%%%%%%%%%%%%%%%%%%%%%%%%%%%%%%%%%%%%%%%

% \usepackage{subfigure}

\setlength{\parskip}{2.3ex}            % vertical space between paragraphs
\setlength{\parindent}{0in}            % amount of indentation of paragraph
%this package allows for hyperlinks within the pdf document
\usepackage[colorlinks=true, linkcolor=blue,pdfstartview=FitV,
citecolor=green, urlcolor=blue]{hyperref}


\newtheorem{thm}{Теорема}
\newtheorem{crl}{Следствие}
\newtheorem{lemma}{Лема}
\newtheorem{dfn}{Дефиниция}
\newtheorem{problem}{Задача}
\newtheorem{example}{Пример}
\newtheorem{remark}{Забележка}
\renewenvironment{proof}{\noindent{\bf Док.}\hspace*{1em}}{\qed\par}

\newcommand{\A}{\mathfrak{A}}
\newcommand{\B}{\mathfrak{B}}
\renewcommand{\C}{\mathfrak{C}}
\newcommand{\D}{\mathfrak{D}}
\newcommand{\R}{\mathbb{R}}
\newcommand{\Z}{\mathbb{Z}}
\newcommand{\N}{\mathbb{N}}
\newcommand{\Q}{\mathbb{Q}}
\newcommand{\Ls}{\mathscr{L}}
\newcommand{\Fs}{\mathscr{F}}
\newcommand{\Rs}{\mathscr{R}}
\newcommand{\Ps}{\mathscr{P}}
\newcommand{\As}{\mathscr{A}}
\newcommand{\Bs}{\mathscr{B}}
\newcommand{\Is}{\mathscr{I}}
\newcommand{\Ss}{\mathscr{S}}
\newcommand{\xn}{x_{1},\dots,x_{n}}

\newcommand{\xs}{overline{x}}

\newcommand{\ys}{overline{y}}

\newcommand{\zs}{overline{z}}
\newcommand{\ov}[1]{\overline{#1}}
\newcommand{\abs}[1]{\vert{#1}\vert}
\newcommand{\pair}[1]{\langle{#1}\rangle}

\newcommand{\SA}{\langle{Q,\Sigma,\Gamma,q_{0},\Delta,F}\rangle}
\newcommand{\CFG}{\langle{V,\Sigma,R,S}\rangle}


\setlength{\marginparwidth}{1.2in}
\let\oldmarginpar\marginpar

\renewcommand\marginpar[1]{\-\oldmarginpar[\raggedleft\footnotesize #1]%
{\raggedright\footnotesize #1}}

\makeindex
\begin{document}


\author{Stefan Vatev\thanks{foo}}
\tableofcontents
\chapter{Теория на множествата}


\section{Декартово произведение}
  Въвеждаме операция наредена двойка $\langle{x,y}\rangle$, която искаме да има следните свойства:
  \begin{enumerate}
  \item
    $\langle{x,y}\rangle = \langle{x',y'}\rangle \iff x = x' \ \&\ y = y'$;
  \item
    класът $A\times B = \{\langle{x,y}\rangle\ \mid\ x\in A\ \&\ x\in B\}$ е множество.
\end{enumerate}


\begin{dfn}[Куратовски]
  Наредена двойка\index{наредена двойка} $\langle{x,y}\rangle = \{\{x\},\{x,y\}\}$
\end{dfn}

Първото свойство се проверява лесно.
За второто свойство, достатъчно е да покажем, че за произволни множества $A,B$ можем да 
изберем множество $C$, за което е изпълнено, че
\[x\in A\ \&\ x\in B \rightarrow \{x,\{x,y\}\}\in C.\]
Ако успеем да намерим такова множество $C$, то тогава от аксиомата за отделянето следва, че $A\times B$
е множество, защото $A\times B = \{ z\in C\ \mid\ (\exists x\in A)(\exists y\in B)[z = \langle{x,y}\rangle]\}$ е множество.

Лесно може да се провери, че $C = \Ps(\Ps(A\cup B))$ върши работа.

Възможно е да се дадат и други дефиниции на наредена двойка.
\begin{problem}
  Проверете кои от следните операции отговарят на условията за наредена двойка.
  \begin{enumerate}
  \item
    $\langle{x,y}\rangle_{1} = \{x,y\}$;
  \item
    $\langle{x,y}\rangle_{2} = \{x,\{y\}\}$;
  \item
    $\langle{x,y}\rangle_{3} = \{\{\emptyset,\{x\}\},\{\{y\}\}\}$;
  \item
    $\langle{x,y}\rangle_{4} = \{\{0,x\},\{1,y\}\}$, 
    където $0,1$ са различни обекти.
\end{enumerate}
\end{problem}



\begin{problem}
  Проверете:
  \begin{enumerate}
  \item
    $A\times B = \emptyset \iff A = \emptyset \vee B = \emptyset$
  \item
    $A\times(B\cup C) = (A\times B)\cup(A\times C)$
  \item
    $A\times(B\cap C) = (A\times B)\cap(A\times C)$ 
  \item
    $A\times(B\backslash C) = (A\times B)\backslash(A\times C)$
  \item
    $(A\cap B)\times (C\cap D) = (A\times C)\cap(B\times D)$
  \item
    $(A\cup B)\times (C\cup D) = (A\times C)\cup(B\times D)$
  \item
    $(A\backslash C)\times(B\backslash D)\subsetneq (A\times B)\backslash(C\times D)$
  \end{enumerate}
\end{problem}

\section{Операции върху множества}

Дефинираме следните операции върху множества:
\begin{enumerate}[(i)]
  \item
    Сечение, $A\cap B = \{x\ \mid\ x\in A\ \&\ x\in B\}$;
  \item
    Обединение, $A\cup B = \{x\ \mid x\in A\ \vee\ x\in B\}$
  \item
    $\bigcup^{n}_{i=1} A_i = \{x \mid \exists i (1\leq i\leq n\ \&\ x\in A_i \}$;
  \item
    $\bigcap^{n}_{i=1} A_i = \{x \mid \forall i (1\leq i\leq n \rightarrow x\in A_i)\}$;
  \item
    Разлика, $A\setminus B = \{x\ \mid\ x\in A\ \&\ x\not\in B\}$;
  \item
    Симетрична разлика, $A\triangle B = (A\backslash B)\cup (B\backslash A)$;
  \item
    $\bigcup A = \{x\mid (\exists y\in A)[x\in y]\}$;
  \item
    $\bigcap A = \{x\mid (\forall y\in A)[x\in y]\}$;
  \item
    Степенно множество, $\Ps A = \{x\mid x\subseteq A\}$.
\end{enumerate}

Тук имаме проблем с значението на $\bigcap\emptyset$.
На пръв поглед изглежда, че $\bigcap\emptyset$ е множеството от всички множества $V$, 
но ние знаем, че такова множество не съществува.
Това в известен смисъл е аналог на делението на нула.
Ние ще приемем, че $\bigcap\emptyset = \emptyset$.


\begin{example}
  Нека $A = \{x\in\N\mid x > 1\}$ и $B = \{x\in\N\mid x>3\}$. Тогава :
    \begin{enumerate}[]
    \item
      $A\cap B = \{x\in\N\mid x > 3\}$,
    \item
      $A\cup B = \{x\in\N\mid x > 1\}$,
    \item
      $A\setminus B = \{x\in\N\mid 1<x\leq 3\}$,
    \item
      $B\setminus A = \emptyset$,
    \item
      $A\triangle B = \{x\in\N\mid 1<x\leq 3\}$
    \end{enumerate}
\end{example}


\begin{problem}
  Нека $A = \{x\in\R\mid |x|\leq 1\}$ и $B = \{x\in\R\mid |x-1|\leq \frac{1}{2}\}$.
  Намерете $A\cup B$, $A\cap B$, $A\setminus B$, $B\setminus A$, $A\triangle B$.
\end{problem}



\begin{example}
  \[\bigcap\{\{1,2,3,4\},\{2,4\},\{1,3,4\}\} = \{4\}\]
  \[\bigcup\{\{3\},\{2,4\},\{1,4\}\} = \{1,2,3,4\}\]
  \[\bigcap\{\{a\},\{a,b\}\} = \{a\}\cap\{a,b\} = \{a\}\]
  \[\bigcup\bigcap\{\{a\},\{a,b\}\}  = \bigcup\{a\} = a\]
\end{example}


\begin{problem}
  Нека $B = \{\{1,2\},\{2,3\}, \{1,3\}, \{\emptyset\}\}$.
  Намерете $\bigcup{B}$, $\bigcap{B}$, $\bigcap\bigcup{B}$ и $\bigcup\bigcap{B}$.
\end{problem}


\begin{example}
  Ето няколко примера, които показват действието на някои от операциите
  \begin{enumerate}[1)]
  \item
    \begin{enumerate}[]
    \item
      $\Ps\emptyset = \{\emptyset\}$
    \item
      $\Ps\{\emptyset\} = \{\emptyset,\{\emptyset\}\}$
    \item
      $\Ps\{\emptyset,\{\emptyset\}\} = \{\emptyset,\{\emptyset\},\{\{\emptyset\}\}, \{\emptyset,\{\emptyset\}\}\}$
    \end{enumerate}
  \item
    \begin{enumerate}[]
    \item
      $\bigcup\{\emptyset\} = \emptyset$
    \item
      $\bigcup\{\emptyset,\{\emptyset\}\} = \{\emptyset\}$
    \item      
      $\bigcup\{\emptyset,\{\emptyset\},\{\{\emptyset\}\}, \{\emptyset,\{\emptyset\}\}\} = \{\emptyset,\{\emptyset\}\}$
    \end{enumerate}
  \item
    $\bigcap\{\emptyset,\{\emptyset\}\} = \emptyset$
\end{enumerate}
\end{example}



\begin{problem}
  \begin{enumerate}
  \item
    Намерете двуелементно множество такова, че всеки елемент на множеството да е също и негово подмножество.
  \item
    Намерете триелементно множество такова, че всеки елемент на множеството да е също и негово подмножество.
  \item
    Намерете четириелементно множество такова, че всеки елемент на множеството да е също и негово подмножество.
\end{enumerate}
\end{problem}


\begin{problem}
  Докажете:
  \begin{enumerate}
  \item
    $\bigcup\Ps A = A$;
  \item
    $A\subseteq\Ps\bigcup A$; кога имаме равенство?
  \item
    $\Ps A \cap \Ps B = \Ps(A\cap B)$;
  \item
    $\Ps A \cup \Ps B \subseteq\Ps(A\cup B)$; кога имаме равенство?
  \item
    съществуват множества $a$ и $B$, за които $a\in B$, но $\Ps{a}\not\subseteq\Ps{B}$;
  \item
    ако $a\in B$, то $\Ps{a}\in\Ps\Ps{B}$;
  \item
    $\{\emptyset,\{\emptyset\}\} \in \Ps\Ps{A}$, за всяко множество $A$.
  \end{enumerate}
\end{problem}



\begin{problem}
  Проверете:
\begin{enumerate}[a)]
  \item
    $A\cup(B\cap C) = (A\cup B)\cap(A\cup C)$
  \item
    $X\subseteq A\ \&\ X\subseteq B \rightarrow X\subseteq A\cap B$
  \item
    $A\subseteq X\ \&\ B\subseteq X \rightarrow A\cup B\subseteq X$
  \item
    $(\bigcup^{n}_{i=1} A_i) \cap B = \bigcup^{n}_{i=1} (A_i \cap B)$
  \item
    $(\bigcap^{n}_{i=1} A_i) \cup B = \bigcap^{n}_{i=1} (A_i \cup B)$
  \item
    $A\subseteq B \iff A\setminus B = \emptyset \iff A\cup B = B \iff A\cap B = A$
  \item
    $A\backslash B = A \iff A\cap B = \emptyset$
  \item
    $A\backslash B = A\backslash (A\cap B)$
  \item
    $(A\cup B)\setminus C = (A\setminus C) \cup (B\setminus C)$
  \item
    $X\backslash (A\cup B) = (X\backslash A)\cap(X\backslash B)$
  \item
    $X\backslash(\bigcup^{n}_{i=1} A_i) = \bigcap^{n}_{i=1} (X\backslash A_i)$
  \item
    $X\backslash (A\cap B) = (X\backslash A)\cup(X\backslash B)$
  \item
    $X\backslash(\bigcap^{n}_{i=1} A_i) = \bigcup^{n}_{i=1} (X\backslash A_i)$
  \item
    $A\cup\bigcap B = \{A\cup X\mid X\in B\}$, за $B\neq\emptyset$
  \item
    $A\cap\bigcup B = \{A\cap X\mid X\in B\}$
  \item
    $(A\backslash B)\backslash C = (A\backslash C)\backslash(B \backslash C)$
  \item
    $A\backslash (B\backslash C) = (A\backslash B) \cup (A\cap C)$
  \item
    $A\triangle B = B\triangle A$
  \item
    $A\triangle(B\triangle C) = (A\triangle B)\triangle C$
  \item
    $A\backslash B = A\triangle(A\cap B)$
  \item
    $A\cap(B\triangle C) = (A\cup B)\triangle(A\cup C)$
  \item
    $A\cup B = (A\triangle B)\cup(A\cap B)$
  \item
    $A\triangle B = \emptyset \iff A = B$
  \item
    $A\triangle B = C \iff B\triangle C = A \iff C\triangle A = B$
  \end{enumerate}
\end{problem}

\begin{problem}
  Да се решат системите с променлива $X$:
  \begin{enumerate}[(a)]
  \item
    \begin{tabular}{l c l}
      $\big|A\setminus X$ & $= $ & $ B$\\
      $\big|X\setminus A $ & $=$ & $ C$
    \end{tabular}, където са дадени множествата $A,B,C$ и $B\subseteq A$, $A\cap C = \emptyset$;
  \item
    \begin{tabular}{l c l}
      $\big|A\cap X$ & $= $ & $ B$\\
      $\big|A\cup X $ & $=$ & $ C$
    \end{tabular}, където са дадени множествата $A,B,C$ и $B\subseteq A\subseteq C$;
  \item
    \begin{tabular}{l c l}
      $\big|A\setminus X$ & $= $ & $ B$\\
      $\big|A\cup X $ & $=$ & $ C$
    \end{tabular}, където са дадени множествата $A,B,C$ и $B\subseteq A\subseteq C$.
  \end{enumerate}
\end{problem}



\begin{problem}
  Нека множеството $A$ е дефинирано по следния начин:
  \begin{enumerate}
  \item
    $0\in A$
  \item
    Ако $x\in A$, то $2x+1 \in A$.
\end{enumerate}
Намерете $A$.
\end{problem}
\begin{proof}
  $A = \{2^n - 1\ \mid n\in\N\}$.
\end{proof}

\begin{thm}
  Нека множеството $A$ е дефинирано по следния начин:
  \begin{enumerate}[(1)]
  \item
    $1\in A$
  \item
    Ако $m,n\in A$, то $2m+3n \in A$.
  \item
    Всички елементи на $A$ са добавени или по правило (1) или правило (2).
\end{enumerate}
Намерете $A$.
\end{thm}
\begin{proof}
  Нека $B = \{n \mid n\equiv 1 (mod 12)\ \vee n\equiv 5 (mod 12) \}$.
  Искаме да докажем, че $A = B$.
  Първо ще докажем, че $A\subseteq B$.
  За целта проверяваме, че $1\in B$ и ако $m,n \in B$, то $2m+3n \in B$.
  
  За другата посока, т.е. $B\subseteq A$, трябва да докажем, че ако
  за всяко $k\leq n$ е вярно, че $12k+1 \in B$ и $12k + 5 \in B$,
  то е вярно, че $12(n+1)+1 \in B $ и $12(n+1) + 5 \in B$.
\end{proof}



\section{Релации}


\begin{dfn}
  Една релация $R \subseteq A^2$ е:
  \begin{enumerate}[1)]
  \item
    антирефлексивна, ако
    $(\forall x\in A)[(x,x)\not\in R]$
  \item
    рефликсивна, ако
    $(\forall x\in A)((x,x)\in R)$;
  \item
    транзитивна, ако
    $(\forall x,y,z\in A)((x,y)\in R\ \&\ (y,z)\in R \rightarrow (x,z)\in R)$;
  \item
    симетрична, ако
    $(\forall x,y\in A)((x,y)\in R \rightarrow (y,x)\in R)$;
  \item
    антисиметрична, ако
    $(\forall x,y\in A)((x,y)\in R\ \&\ (y,x)\in R \rightarrow x = y)$;
  \item
    асиметрична, ако
    $(\forall x,y)[(x,y)\in R \rightarrow (y,x)\not\in R]$.
\end{enumerate}
\end{dfn}

\begin{enumerate}[(i)]
\item
  Композиция на релации
  $S\circ T = \{\langle{x,y}\rangle \mid (\exists z)[\langle{x,z}\rangle\in T\ \&\ \langle{z,y}\rangle \in S]\}$;
\item
  $R^{-1} = \{\langle{x,y}\rangle \mid \langle{y,x}\rangle \in R\}$;
\item
  $\overline{R} = \{\langle{x,y}\rangle \in A\times B \mid\langle{x,y}\rangle\not\in R\}$;
\end{enumerate}

  
\begin{problem}
  Докажете, че:
  \begin{enumerate}[a)]
  \item
    $R$ е симетрична тогава и само тогава, когато $R^{-1}\subseteq R$;
  \item
    $R$ е транзитивна тогава и само тогава, когато $R\circ R\subseteq R$;
  \item
    $R$ е транзитивна и симетрична тогава и само тогава, когато $R = R^{-1}\circ R$.
\end{enumerate}
\end{problem}
\begin{proof}
  \begin{enumerate}[a)]
  \item
    Задачата се разделя на две подзадачи.
    \begin{enumerate}[(i)]
    \item
      Нека $R$ да бъде симетрична. Ще докажем, че $R^{-1}\subseteq R$, т.е.
      \[(\forall x\forall y)[(x,y)\in R^{-1} \rightarrow (y,x)\in R].\]
      Нека $(x,y)\in R^{-1}$. Тогава имаме, че $(y,x)\in R$ и следователно $(x,y)\in R$,
      защото $R$ е симетрична.
    \item
      Нека $R^{-1}\subseteq R$. Щe докажем, че $R$ е симетрична, т.е.
      \[(\forall x\forall y)[(x,y)\in R \rightarrow (y,x)\in R].\]
      Нека $(x,y)\in R$, следователно $(y,x)\in R^{-1}$.
      Тогава от $R^{-1}\subseteq R$ следва, че $(y,x)\in R$.
    \end{enumerate}
  \end{enumerate}
\end{proof}


\begin{problem}
  Нека $R$ да бъде релация на еквивалентност върху $B$ и $f:A\to B$.
  Дефинираме множеството \[Q = \{((x,y)\in A\times A\mid (f(x),f(y))\in R\}.\]
  Докажете, че $Q$ е релация на еквивалентност.
\end{problem}

\begin{problem}
  Нека $\{(a,b)\}\subseteq R$, за някои $a\neq b$.
  Докажете, че ако $R$ е симетрична, то $R$ не е антисиметрична.
\end{problem}
\begin{proof}
  Нека $R$ е симетрична.
\end{proof}



\begin{problem}
  Проверете за $R$ дали е рефлексивна, транзитивна, симетрична, антисиметрична или асиметрична релация.
  \begin{enumerate}[a)]
  \item
    $R\subsetneq \mathbb{N}^2, aRb \iff a | b$ 
  \item
    $R\subseteq \R^2 , aRb \iff a.b > 0$ 
  \item
    $R\subseteq \R^2, aRb \iff a+b = 0$
  \item
    $R\subseteq \R^2, aRb \iff a+b = 5$ 
  \item
    $R\subseteq \R^2, aRb \iff a+b\mbox{ е четно }$ 
  \item
    $R\subseteq (\R^2)^2, \langle{a,b}\rangle R \langle{c,d}\rangle \iff a+d = b+c$ 
  \item
    $R\subseteq (\R^2)^2, \langle{a,b}\rangle R \langle{c,d}\rangle \iff a.d = b.c$ 
  \item
    $R_{m}\subseteq \Z^2, m\in \Z, m>0, aR_{m}b \iff m\mid (a - b)$ 
  \item
    $R\subseteq \R^2, xRy \iff (x-y)\mbox{ е рационално число}$ 
  \item
    $aRb \iff a,b\in\N\ \&\ (a = b \vee a+1 = b)$ 
  \item
    $aRb \iff a,b\in\N\ \&\ (\exists k\in\N)(a+k = b)$
  \item
    Нека $\leq_1$ е ч.н. върху $A$, $\leq_2$ е ч.н. върху $B$.
    $\langle{a,b}\rangle R\langle{c,d}\rangle \iff a\leq_{1}c\ \&\ b\leq_{2}d$ 
  \item
    Нека $\leq_1$ е ч.н. върху $A$, $\leq_2$ е ч.н. върху $B$.
    $\langle{a,b}\rangle R\langle{c,d}\rangle \iff a\leq_{1}c\ \vee\ b\leq_{2}d$
  \item
    $f:X\rightarrow Y$, $R\subseteq (2^{X})^{2}, ARB \iff f(A) = f(B)$ 
  \end{enumerate}
\end{problem}


\begin{dfn}
  \begin{enumerate}
  \item
    Азбука е крайно множество $X = \{a_1,\dots,a_n\}, \varepsilon\not\in X$.
    Елементите на $X$ наричаме букви.
  \item
    Думи над азбуката $X$ са:
    \begin{enumerate}
    \item
      $\varepsilon$ е дума над $X$, наричаме я празната дума.
    \item
      Нека $\alpha$ е дума над $X$. 
      Тогава за всяко $i\leq n$ имаме, че $\alpha a_i$ е дума над $X$;
    \item
      няма други думи над $X$.
    \end{enumerate}
  \item
    Нека $\alpha=a_{i_1}\dots a_{i_m}$ и $\beta=b_{j_1}\dots b_{j_n}$ са думи над $X$.
    $\alpha$ е начало на $\beta$, ако $m\leq n$ и $(\forall k\leq m)(a_{i_k} = b_{j_k})$.
    $\alpha$ е край на $\beta$, ако $m\leq n$ и $(\forall k\leq m)(a_{i_{(m-k)}} = b_{j_{(n-k)}})$.
  \end{enumerate}
  Означаваме с $X^n$ множеството от всички думи с дължина $n$ над азбуката $X$, $X^0 = \{\varepsilon\}$.
  С $X^{*}$ означаваме множеството от всички думи над азбуката $X$, т.е. $X^{*} = \bigcup_{0\leq n} X^{n}$.
\end{dfn}

Дефинираме дължината $|\alpha|$ на думите $\alpha \in X^*$ с индукция по построението на $\alpha$.
\begin{enumerate}[(i)]
  \item
    Ако $\alpha = \varepsilon$, то $|\alpha| = 0$;
  \item
    Ако $\alpha = \beta a$, за някоя дума $\beta\in X^*$ и някоя буква $a\in X$. Тогава \[|\alpha| = |\beta| + 1.\]
\end{enumerate}

Дефинираме операцията {\em конкатенация}\index{конкатенация} $\cdot$ на две думи $\alpha$ и $\beta$ от $X^*$ с индукция по дължината $\beta$:
\begin{enumerate}[(i)]
  \item
    $|\beta| = 0$, т.е. $\beta = \varepsilon$.
    Тогава \[\alpha\cdot\beta = \alpha.\]
  \item
    $|\beta| = n+1$, т.е. $\beta = \gamma b$, за някоя дума $\gamma$, $|\gamma| = n$, и някоя буква $b\in X$.
    Тогава \[\alpha\cdot\beta = (\alpha\cdot\gamma)\cdot b.\]
\end{enumerate}

Сега можем да даден алтернативни дефиниции на понятията начало и край на дума.
Казваме, че думата $\alpha$ е начало на думата $\beta$, ако съществува дума $\gamma$ такава, че
$\beta = \alpha\cdot\gamma$.
Аналогично дефинираме $\alpha$ да бъде край на думата $\beta$.


\begin{problem}
  Определете релациите:
  \begin{enumerate}[a)]
  \item
    $\alpha R \beta \iff \alpha \mbox{ е начало на }\beta$ 
  \item
    $\alpha R \beta \iff \alpha \mbox{ е край на }\beta$
  \item
    $\alpha R \beta \iff \alpha \mbox{ е начало на }\beta \vee (\exists\alpha_1\in X^{*})(\exists a,b\in X)(\alpha_1 a \mbox{ е начало на }\alpha\ \&\ \alpha_1 b \mbox{ е начало на } \beta)$
  \item
    $R\subseteq (\{0,1\}^{n})^{2}, \langle{a_1,\dots,a_n}\rangle R \langle{b_1,\dots,b_n}\rangle \iff a_1\leq b_1\ \&\dots\ \&\ a_n\leq b_n$
    Забележете, че това не е лексикографската наредба.
  \item
    $R\subseteq (\{0,1\}^{n})^{2}, \langle{a_1,\dots,a_n}\rangle R \langle{b_1,\dots,b_n}\rangle \iff (\exists i : 1\leq i\leq n)((\forall j < i)(a_j = b_j)\ \&\ a_i \leq b_i)$
\end{enumerate}
\end{problem}
\begin{proof}
  \begin{enumerate}[1)]
  \item[в)]
    \begin{enumerate}[(i)]
    \item
      Ще проверим дали $R$ е рефлексивна, т.е. дали $(\forall\gamma\in X^{*})[\gamma R\gamma]$.
      Очевидно $R$ е рефлексивна, защото всяка дума е начало на самата себе си ($\alpha = \alpha\cdot\varepsilon$).
    \item
      Ще проверим дали $R$ е транзитивна, т.е. дали $(\forall\alpha,\beta,\gamma\in X^*)[\alpha R\beta\wedge\beta R\gamma\rightarrow\alpha R\gamma]$.
      Тук трябва да разгледаме четири случая:
      \begin{enumerate}[a)]
      \item
        $\beta = \alpha\cdot\delta$ и $\gamma = \beta\cdot\rho$, за някои $\delta$ и $\rho$.
        Тогава $\gamma = \alpha\cdot(\delta\cdot\rho)$, следователно $\alpha$ е начало на $\gamma$.
      \item
        $\beta = \alpha\cdot\delta$ и $\beta = \rho\cdot a\cdot \beta'$ и $\gamma = \rho\cdot b\cdot\gamma'$.
        Ако $\alpha$ е начало на $\rho$, то $\alpha R \gamma$.
        Ако $\rho\cdot a$ е начало на $\alpha$, тогава $\alpha = \rho\cdot c\cdot\alpha'$ и $\alpha R \gamma$.
      \item
        $\alpha = \rho\cdot a\cdot \alpha'$ и $\beta = \rho\cdot b\cdot\beta'$ и $\gamma = \beta\cdot\delta$ се разглежда аналогично.
      \item
        $\alpha = \rho\cdot a\cdot \alpha'$ и $\beta = \rho\cdot b\cdot\beta'$ и $\beta = \delta\cdot c\cdot \beta'$ и $\gamma = \delta\cdot d\cdot\gamma'$.
        Ако $\rho\cdot b$ е начало на $\delta$, то $\delta = \rho\cdot b\cdot\nu$ и $\gamma = \rho\cdot b\cdot \gamma''$. Тогава $\alpha R \gamma$.
        Ако $\delta\cdot c$ е начало на $\rho$, то $\rho = \delta\cdot c\cdot\nu$ и $\alpha = \delta\cdot c\cdot\alpha''$. Получаваме, че 
        $\alpha R \gamma$.
      \end{enumerate}
    \item
      Ще проверим дали $R$ е симетрична, т.е. дали $(\forall\alpha,\beta\in X^*)[\alpha R\beta \rightarrow \beta R\alpha]$.
      Ако $\alpha = \varepsilon$, то $\alpha R\beta$, но нямаме $\beta R\alpha$.
      Следователно, релацията не е симетрична.
    \item
      Лесно се вижда също, че $R$ не е антисиметрична.
    \end{enumerate}
    
  \end{enumerate}
  
\end{proof}


\begin{problem}
  За всяко естествено число $n$, дефинираме релацията $R_n \subseteq (X^*)^2$ като
  \[\alpha R_n \beta \iff [|\alpha| > n\ \wedge |\beta| > n\wedge (\forall i < n)[a_i = b_i]].\]
  Докажете, че $R_n$ е релация на еквивалентност и намерете броя на класовете на еквивалентност.
\end{problem}

\begin{problem}
  За всяко естествено число $n$, дефинираме релацията $R_n \subseteq (X^*)^2$ като
  \[\alpha R_n \beta \iff \alpha = \beta \vee [|\alpha| > n\ \wedge |\beta| > n\wedge (\forall i < n)[a_i = b_i]].\]
  Докажете, че $R_n$ е релация на еквивалентност и намерете броя на класовете на еквивалентност.
\end{problem}
\begin{proof}
  Броят на класовете на еквивалентност е $\frac{|X|^n - 1}{|X| - 1}$.
\end{proof}


\begin{problem}
  За всяко естествено число $n$, дефинираме релацията $R_n \subseteq (X^*)^2$ като
  \[\alpha R_n \beta \iff |\alpha| = |\beta| > n\wedge (\forall i > n)[a_i = b_i].\]
  Докажете, че $R_n$ е релация на еквивалентност.
\end{problem}

\begin{problem}
  Нека $R$ е релация върху паметта на един компютър и е дефинирана като
  \[xRy \iff x,y\mbox{ са указатели в паметта и }*x = *y.\]
  Докажете, че $R$ е релация на еквивалентност.
\end{problem}



\begin{dfn}
  Нека $R$ е релация.
  Множеството $[x]_R$ се дефинира като
  \[[x]_R = \{t\mid xRt\}.\]
  Ако $R$ е релация на еквивалентност и $x\in Field(R)$, то $[x]_R$ клас на еквивалентност за $x$ (по модул $R$).
\end{dfn}

\begin{example}
  Нека $\sim$ е бинарна релация върху $\N$, дефинирана като
  \[x\sim y \iff x\equiv y\ (mod\ 4).\]
  $\sim$ е релация на еквивалентност и има четири класове на еквивалентност
  \[[0]_\sim, [1]_\sim, [2]_\sim, [3]_\sim.\]
\end{example}

\begin{problem}
  Да дефинираме релацията $R$ върху реалните числа, като:
  \[xRy \iff (x-y)\in\Z.\]
  Намерете $[1]_R$ и $[\frac{1}{2}]_R$.
\end{problem}


\begin{lemma}
  Нека $R$ е релация на еквивалентност върху $A$ и $x,y\in A$. Тогава \[[x]_R = [y]_R \iff xRy.\]
\end{lemma}


\begin{problem}
  Покажете, че за всяка релация $R$ и $x$, $[x]_R = R[\{x\}]$.
\end{problem}

%% Rosen Textbook
\begin{problem}
  Кои от следните релации върху множеството от функциите от $\Z$ в $\Z$.
  \begin{enumerate}[a)]
  \item
    $\{(f,g)\mid f(1) = g(1)\}$.
  \item
    $\{(f,g)\mid f(0) = g(0)\wedge f(1) = g(1)\}$.
  \item
    $\{(f,g)\mid (\forall x\in\Z)[f(x)-g(x) = 1]\}$.
  \item
    $\{(f,g)\mid (\exists c\in\Z)(\forall x\in\Z)[f(x)-g(x) = c]\}$.
  \item
    $\{(f,g)\mid f(0) = g(1)\wedge f(1) = g(0)\}$.
  \end{enumerate}
\end{problem}


\begin{problem}
  Нека $A$ е непразно множество и $f$ е функция с $Domain(f) = A$.
  Дефинираме $R$ върху $A$ като:
  \[\{(x,y)\mid x,y\in A\wedge f(x) = f(y)\}.\]
  Докажете, че
  \begin{enumerate}[(i)]
  \item
    $R$ е релация на еквивалентност.
  \item
    Определете класовете на еквивалентност на $R$.
\end{enumerate}

\begin{problem}
  Нека $R_1$ и $R_2$ са симетрични релации.
  Проверете дали $\overline{R_1}$, $R_1\cap R_2$ и $R_1 \cup R_2$ са симетрични.
\end{problem}

\begin{problem}
  Докажете, че подмножество на всяка антисиметрична релация е също антисиметрична.
\end{problem}

\end{problem}

\section*{Функции}
\index{функция}

Релацията $R \subseteq A\times B$ се нарича {\bf функция}\index{функция} от $A$ в $B$, ако
\begin{enumerate}[i)]
  \item
    $Dom(R) = A$, т.е.
    \[(\forall a\in A)(\exists b\in B)[(a,b)\in R].\]
  \item
    За всеки елемент $a\in A$ съотвества {\em точно един} елемент $b \in B$, т.е.
    \[(\forall x\in A)(\forall y_1,y_2 \in B)(\langle{x,y_1}\rangle\in R\ \wedge\ \langle{x,y_2}\rangle\in R \rightarrow y_1 = y_2).\]
\end{enumerate}
Обикновено означаваме функциите като $f:A\to B$ и
вместо $(a,b)\in f$ пишем $f(a) = b$.
Казваме, че функцията $f$ e
\begin{itemize}
\item
  \marginpar{също казваме, че $f$ е обратима}
  {\bf инекция}\index{функция!инекция}, ако 
  \[(\forall x_1,x_2\in A)[x_1\neq x_2 \rightarrow f(x_1)\neq f(x_2)],\]
  или еквивалентно,
  \[(\forall x_1,x_2\in A)[f(x_1) = f(x_2) \rightarrow x_1 = x_2].\]
\item
  \marginpar{също казваме, че $f$ е върху $B$}
  {\bf сюрекция}\index{функция!сюрекция}, ако 
  \[(\forall y\in B)(\exists x\in A)[f(x) = y].\]
\item
  {\bf биекция}\index{функция!биекция}, ако е инекция и сюрекция.
\end{itemize}

\begin{problem}
  Дайте примери за функция $f:\mathbb{N}\rightarrow\Z$, която е:
  \begin{enumerate}
  \item
    нито инективна, нито сюрективна;
  \item
    инективна, но не е сюрективна;
  \item
    сюрективна, но не е инективна;
  \item
    сюрективна и инективна.
  \end{enumerate}
\end{problem}

\begin{problem}
  Докажете:
  \begin{enumerate}[a)]
  \item
    Ако $f,g$ са функции, то $f\cap g$ е функция;
  \item
    Нека $f,g$ са функции и $(\forall x)[x\in Dom(f)\cap Dom(g)\rightarrow f(x) = g(x)]$.
    Докажете, че $f\cup g$ е функция.
  \end{enumerate}
\end{problem}

\begin{prb}
  За всяка от следните  функции $f$ определете дали $f$ е
  инекция, сюрекция или биекция.
  \begin{enumerate}[a)]
  \item
    $f: \mathbb{R}\rightarrow \mathbb{R}$, $f(x) = 2x+3$.
  \item
    $f: \mathbb{R}\rightarrow \mathbb{R}$, $f(x) = x^2 - 4x +2$.
  \item 
    $f: \mathbb{R}\rightarrow \mathbb{R}$, $f(x) = x^3+7$.    
  \item
    $f: \mathbb{N}\rightarrow \mathbb{N}$, 
    \begin{align*}
      f(x) = 
      \begin{cases}
        x+1, & \mbox{ ако }x\mbox{ е четно}\\
        x-1, & \mbox{ ако }x\mbox{ е нечетно}\\
      \end{cases}
    \end{align*}
  \item
    \marginpar{$rem(x,3)$ - остатък при деление на $3$}
    $f: \mathbb{N}\rightarrow \mathbb{N}$, $f(x) = rem(x,3)$.
  \item 
    \marginpar{НОД - най-голям общ делител}
    $f: \mathbb{N} \times \mathbb{N}\rightarrow \mathbb{N}$,
    $f(x, y) = \mbox{ НОД}(x,y)$.
  \item 
    $f: \mathbb{N} \times \mathbb{N}\rightarrow \mathbb{N}$,
    $f(x, y) = 3x+2y$.
  \item 
    $f: \Nat \times \Nat\rightarrow \Nat$,
    $f(x, y) = 2^x(2y+1)-1$.
  \item 
    $f: \Nat \times \Nat\rightarrow \Nat$,
    $f(x, y) = 2x(2y+1)$.
  \item
    $f: \Nat \times \Nat\rightarrow \Nat$,
    $f(x, y) = 2x(2^y+1)$.
  \item
    $f: \Nat \times \Nat\rightarrow \Nat$,
    $f(x, y) = 2^x3^y$.
  \item
    $f: \Nat \times \Nat\rightarrow \Nat$,
    $f(x, y) = 2^x6^y$.
  \item 
    $f: \mathbb{R} \times \mathbb{R}\rightarrow \mathbb{R}$,
    $f(x, y) = x^2+y^2$.
  \item
    
  \end{enumerate}
\end{prb}


% \item 
%   $f: \mathbb{Q}\rightarrow \mathbb{Q}$, $f(x) =
%   \cstwo{0}{$x=0$}{\frac{1}{x}}{$x\neq 0$}$.
% \item
%   $f: \mathbb{R} \rightarrow \mathbb{R}$, $f(x) = |x|+1$.
% \item 
%   $f: (-\frac{\pi}{2}, \frac{\pi}{2})\rightarrow \mathbb{R}$,
%   $f(x) = tg x$.
% \item 
%   $f: \mathbb{N} \times \mathbb{N}\rightarrow \mathbb{N}$,
%   $f(x, y) = 3^x.5^y$.
% \item
%   $f: \mathbb{R}^+ \rightarrow \mathbb{N}$, $f(x) = \lfloor x
%   \rfloor$. (най-голямото естествено  число, по-малко или равно на
%   $x$. )
% \item
%   $f: \mathbb{N} \rightarrow \mathbb{N}$, $f(x) = (x+1)$-вото
%   просто число.
% \end{enumerate}


\subsection*{Операции върху функции}

Нека е дадена функцията $f:A\to B$.
Ще разгледаме няколко основни операции върху функции.
\begin{enumerate}[I)]
\item
  {\bf Образ}
  
  Нека $X\subseteq A$. {\em Образ на множеството} $X$ под действието на функцията $f$, наричаме
  множеството: \[f(X) = \{b\in B \mid f(a) = b\ \wedge\ a \in X\}.\]
\item
  {\bf Първообраз}

  Нека $Y\subseteq B$. {\em Първообраз на множеството} $Y$ под действието на функцията $f$, наричаме
  множеството: \[f^{-1}(Y) = \{a\in A \mid f(a) = b\ \wedge\ b \in Y\}.\]
\item
  {\bf Обратна функция}
  За всяка биективна функция $f:A\to B$, определяме нейната обратна функция $g:B \to  A$ като:
  \[(\forall a \in A)(\forall b \in Ran(f))[g(b) = a\ \iff\ f(a) = b].\]
  Обикновено означаваме $g$ като $f^{-1}$.
% \item 
%   {\bf Рестрикция}

%   Нека $X\subseteq A$. {\em Рестрикция} на $f$ до множеството $X$, наричаме
%   множеството: \[f\upharpoonright X = \{\langle{x,y}\rangle\mid f(x) = y\ \wedge\ x\in X\} =  f\cap X\times B.\]
% \item
%   {\bf Затваряне}
  
%   Нека в този случай $f:A\to A$ и нека $X\subseteq A$.
%   За всяко $n \geq 0$ определяме $X_0 = X$ и $X_{n+1} = X_n \cup f(X_n)$.
%   {\em Затваряне} на множеството $X$ относно функцията $f$ е множеството
%   \[f[X] = \bigcup_{n\in\Nat} X_n. \]
  
\item
  {\bf Композиция}

  Нека са дадени функциите $f:A\to B$ и $g:C\to A$.
  {\em Композиция} на $f$ и $g$ е функцията $f\circ g: C \to B$ определена като
  \[f\circ g = \{\pair{c,b}\mid (\exists a\in A)[g(c) = a\ \wedge\ f(a) = b]\}.\]
  \marginpar{Най-напред прилагаме $g$ и след това $f$}
  Композицията на $f$ и $g$ може да се запише и така:
  \[(\forall c\in C)[(f\circ g)(c) = f(g(c))]\]
\end{enumerate}

\begin{prb}
  Нека $f: A\to B$, $g: B\to C$ са функции.
  Вярно ли е, че:
  \begin{enumerate}
  \item 
    Ако $f$  не е инекция, то $g\circ f$ не е инекция?
  \item
    Ако $g$  не е инекция, то $g\circ f$ не е инекция?
  \item 
    Ако $f$  не е сюрекция, то $g\circ f$ не е сюрекция?
  \item
    Ако $g$  не е сюрекция, то $g\circ f$ не е сюрекция?
  \end{enumerate}
\end{prb}

\begin{prb}
  Нека $f: A\to B$, $g: B\to C$ са функции.
  Вярно ли е, че:
  \begin{enumerate}[1)]
  \item
    $f,g$ са инективни, то $g\circ f$ е инективна?
  \item
    $f,g$ са сюрективни, то $g\circ f$ е сюрективна?
  \item
    $f,g$ са биективни, то $g\circ f$ е биективна?
  \item
    $g\circ f$ е сюрективна,  то $f,g$ са сюрективни ?
  \item
    $g\circ f$ е инективна, то $f,g$ са инективни ?
  \end{enumerate}
\end{prb}

\begin{prb}
  Нека $f: A\to B$, $g: B\to C$ са {\em биективни} функции.
  Докажете, че
  \[(g\circ f)^{-1} = f^{-1}\circ g^{-1}.\]
\end{prb}

\begin{prb}
  $f(x) = \pair{g(x),h(x)}$.
\end{prb}


% \begin{dfn}
%   Дефинираме следните операции върху релацията $R\subseteq A\times{B}$:
%   \begin{enumerate}
%   \item
%     Дефиниционна област
%     $Domain(R) = \{x\mid (\exists y)\langle{x,y}\rangle\in R \}$;
%   \item
%     Област от стойности
%     $Range(R) = \{y\mid (\exists x)[\langle{x,y}\rangle\in R]\}$;
%   \item
%     Поле
%     $Field(R) = Domain(R) \cup Range(R)$;
%   \item
%     Рестрикция
%     $R\upharpoonright{C} = \{\langle{x,y}\rangle\mid \langle{x,y}\rangle\in R\ \&\ x\in C\}$;
%   \item
%     Образ
%     $R[C] = \{ y \mid (\exists x)[ x\in C\ \&\ \langle{x,y}\rangle\in R]\} = Range(R\upharpoonright{C})$.


% \end{enumerate}
% \end{dfn}

% \begin{example}
%   Нека да разгледаме релацията \[F = \{\langle{\emptyset, a}\rangle,\langle{\{\emptyset\}, b}\rangle\}.\]
%   Лесно се вижда, че $F$ е функция.
%   Имаме, че \[F^{-1} = \{\langle{a,\emptyset}\rangle,\langle{b, \{\emptyset\}}\rangle\}\] е функция тогава и само тогава, когато  $a\neq b$.
%   Обърнете внимание, че
%   \[F\upharpoonright{\emptyset} = \emptyset \mbox{, но } F\upharpoonright\{\emptyset\} = \{\langle{\emptyset,a}\rangle\}.\]
%   Освен това, $F(\{\emptyset\}) = \{a\}$ и $F(\{\emptyset\}) = b$.
% \end{example}

% Нека $f$ е функция и $A$ е множество.
% \begin{enumerate}[(i)]
% \item
%   $f(A) = \{y \mid (\exists x\in A)(f(x) = y)\}$
% \item
%   $f^{-1}(A) = \{x \mid f(x)\in A\}$
% \end{enumerate}


\begin{problem}
  Нека а дадена произволна функция $f:A \to B$.
  Проверете:
  \begin{enumerate}[a)]
  \item
    $(\forall X,Y \subseteq A)[f(X)\cup f(Y) = f(X\cup Y)]$;
  \item
    $f(\bigcup_{i\in I}X_i) = \bigcup_{i\in I}(X_i)$
  \item
    при какви условия за $f$,
    $(\forall X,Y \subseteq A)[f(X\cap Y) = f(X)\cap f(Y)]$.
  % \item
  %   $f(\bigcap_{i\in I}A_i) \subseteq \bigcap_{i\in I}f(A_i)$
  \item
    при какви условия за $f$,
    $(\forall X,Y \subseteq A)[f(X)\backslash f(Y) = f(X\backslash Y)]$.
  \item
    \marginpar{Опр. $(\forall x\in X)\ id_X(x) = x$}
    при какви условия за $f$, $f\circ f^{-1} = id_{B}$.
  \item
     при какви условия за $f$, $f^{-1}\circ f = id_{A}$.
   % \item
     % $Dom(f\circ g) = g^{-1}(Dom(f))$, където $g$ е функция.
  \item
    $(\forall X,Y \subseteq B)[f^{-1}(X\cup Y) = f^{-1}(X)\cup f^{-1}(Y)$.
  \item
    $(\forall X,Y \subseteq B)[f^{-1}(X\cap Y) = f^{-1}(X)\cap f^{-1}(Y)$.
  \item
    $(\forall X,Y \subseteq B)[f^{-1}(X\backslash Y) = f^{-1}(X)\backslash f^{-1}(Y)]$.
  \item
    при какви условия за $f$,
    $(\forall X\subseteq A)[X =  f^{-1}(f(X))]$.
  \item
    при какви условия за $f$,
    $(\forall Y \subseteq B)[Y = f(f^{-1}(Y))]$.
  \item
    $(\forall X\subseteq A)(\forall Y\subseteq B)[f(X)\cap Y = f(X\cap f^{-1}(Y))]$.
  \item
    $(\forall X \subseteq A)(\forall Y \subseteq B)[f(X)\cap Y = \emptyset \iff X\cap f^{-1}(Y) = \emptyset]$.
  \item
    $(\forall X \subseteq A)(\forall Y \subseteq B)[f(X)\subseteq Y \iff X\subseteq f^{-1}(Y)]$.
  \end{enumerate}
\end{problem}
\newpage
\begin{problem}%Л.М. 18 / 23
  Нека $f,g$ са функции. При какви условия :
  \begin{enumerate}
  \item
    $f^{-1}$ е функция?
  \item
    $f\circ g$ е инективна функция?
  \end{enumerate}
\end{problem}

% \begin{problem}
%   Дайте примери за функция $f:\mathbb{N}\rightarrow\Z$, която е:
%   \begin{enumerate}
%   \item
%     нито инективна, нито сюрективна;
%   \item
%     инективна, но не е сюрективна;
%   \item
%     сюрективна, но не е инективна;
%   \item
%     сюрективна и инективна.
% \end{enumerate}
% \end{problem}


\begin{problem}
  Нека е дадена релацията $R\subseteq A\times B$.
  Докажете, че $R$ е биективна функция тогава и само тогава, когато $R\circ R^{-1} = id_A$ и $R^{-1}\circ R = id_B$.
\end{problem}

\begin{problem}
  Нека $f$ е инективна функция от $A$ в $B$ и $g:\Ps(A) \rightarrow \Ps(B)$, дефинирана като 
  \[(\forall X \subseteq A)[g(X) = f(X)].\]
  Докажете, че $g$ е инективна.
\end{problem}

\begin{problem}
  Нека $f:A\rightarrow B$ и $g:B\rightarrow\Ps(A)$, дефинирана като 
  \[(\forall b \in B)[g(b) = \{x\in A\mid f(x) = b\}].\]
  Докажете, че ако $f$ е сюрективна, то $g$ е инективна.
  Вярна ли е обратната посока?
\end{problem}

\cite{hein}

%%% Local Variables: 
%%% mode: latex
%%% TeX-master: "discrete-math"
%%% End: 

\chapter{Мощност на множества}

\section{Основни понятия}

\begin{itemize}
\item 
  Казваме, че едно множество $A$ е {\bf изброимо безкрайно}\index{множество!изброимо безкрайно}, ако съществува 
  биекция от $A$ върху $\Nat$.
\item
  Едно множество е {\bf изброимо}, ако е или крайно или безкрайно изброимо.
\item
  Казваме, че едно множество $A$ е {\bf неизброимо}\index{множество!неизброимо}, ако $A$ е безкрайно и {\bf не} съществува 
  биекция от $A$ върху $\Nat$.
\item
  Казваме, че мощността на едно множество $A$ е не по-голяма от мощността на множеството $B$, 
  което записваме като $\abs{A} \leq \abs{B}$, ако съществува {\em инекция} $f:A \to B$.
  Възможно е да използваме и означението $A \preceq B$.
\item
  Когато множеството $A$ е крайно, например $A = \{a_1,\dots,a_n\}$, 
  ще записваме $\abs{A} = n$.
\item
  Две множества $A$ и $B$ са равномощни, $|A| = |B|$, ако съществува биекция от $A$ върху $B$.
  Алтернативен запис е $A \sim B$.
\item
  Записваме $\abs{A} < \abs{B}$, ако $\abs{A} \leq \abs{B}$ и $\abs{A} \neq \abs{B}$.
  Алтернативен запис е $A \precneqq B$, т.е. $A \preceq B$ и $A \not\sim B$.
\end{itemize}


\section{Сравняване на мощности}


\begin{framed}
\begin{thm}[Кантор-Шрьодер-Бернщайн]
  \index{Кантор-Шрьодер-Бернщайн}
  \label{th:ksb}
  За всеки две множества $A$ и $B$,
  \[A \preceq B\ \&\ B \preceq A \implies A \sim B.\]
\end{thm}
\end{framed}
\begin{proof}
  \marginpar{Според уикипедия това доказателство е на Гюла Кьониг(1906), синът на Денеш Кьониг}
  Без ограничение на общността, нека $A\cap B = \emptyset$.
  Нека също така да фиксираме инективни функции $f:A\rightarrow B$ и $g:B\rightarrow A$.
  Ще построим биективна функция $h:A\rightarrow B$.
  
  Понеже $g$ е инективна, то $g^{-1}$ също е (частична) функция. За $a\in A$, имаме следното:
  \[
  g^{-1}(\{a\}) = 
  \begin{cases}
    \emptyset, & a \not\in Range(g)\\
    \{b\}, & g(a) = b 
  \end{cases}
  \]
  Ако $g^{-1}(\{a\}) = \{b\}$, то наричаме $b$ {\em наследник} на $a$.
  Аналогично, понеже $f$ е инективна, то $f^{-1}$ също е (частична) функция и за $b\in B$:
  \[
  f^{-1}(\{b\}) = 
  \begin{cases}
    \emptyset, & a \not\in Range(f)\\
    \{a\}, & f(b) = a 
  \end{cases}
  \]
  Ако $f^{-1}(\{b\}) = \{a\}$, то казваме, че $a$ е {\em наследник} на $b$.
  Продължавайки същата схема, можем да се опитаме да намерим наследника на $a$ и т.н.
  За елемента $a$ имаме три възможни изхода от тази процедура:
  \begin{enumerate}[i)]
  \item
    $a$ има като последен наследник някой елемент от $A$;
  \item
    $a$ има като последен наследник някой елемент от $B$;
  \item
    $a$ има безкрайно много наследника.
\end{enumerate}
Например, следната верига
\[a \stackrel{g^{-1}}{\longrightarrow} b \stackrel{f^{-1}}{\longrightarrow}a_1 \stackrel{g^{-1}}{\longrightarrow} b_1 \stackrel{f^{-1}}{\longrightarrow}a_2\stackrel{g^{-1}}{\longrightarrow}\emptyset\]
показва, че $a \in A_1$, защото последния наследник на $a$ е елемента $a_2 \in A$.
Да означим множествата:
\begin{align*}
  A_1 = & \{a\in A \mid \mbox{ веригата с начало $a$ завършва с елемент от } A\}\\
  A_2 = & \{a\in A \mid \mbox{ веригата с начало $a$ завършва с елемент от } B\}\\
  A_3 = & \{a\in A \mid \mbox{ веригата с начало $a$ е безкрайна} \}.
\end{align*}
Лесно се съобразява, че $A_1\cup A_2\cup A_3 = A$ и 
множествата $A_1$, $A_2$ и $A_3$ нямат общи елементи.
Аналогично дефинираме:
\begin{align*}
  B_1 = & \{b\in B \mid \mbox{ веригата с начало $b$ завършва с елемент от} A\}\\
  B_2 = & \{b\in B \mid \mbox{ веригата с начало $b$ завършва с елемент от } B\}\\
  B_3 = & \{b\in B \mid \mbox{ веригата с начало $b$ е безкрайна} \}.
\end{align*}
Отново $B_1\cup B_2\cup B_3 = B$ и множествата $B_1$, $B_2$ и $B_3$ нямат общи елементи.

Да разгледаме функциите $f_i = f\upharpoonright{A_i}$ и $g_i = g\upharpoonright{B_i}$, $i = 1,2,3$. 
Лесно се съобразява, че 
\begin{align*}
  f_i:& A_i\to B_i,\\
  g_i:& B_i\to A_i.
\end{align*}
Например, нека $a \in A_1$ и $b = f(a)$. Да съобразим, че наистина $b \in B_1$.
Имаме, че $f^{-1}(\{b\}) = \{a\}$, т.е. $a$ е {\em наследник} на $b$.
Получаваме веригата:
\[b \stackrel{f^{-1}}{\longrightarrow}a \stackrel{g^{-1}}{\longrightarrow} b_1 \stackrel{f^{-1}}{\longrightarrow} a_1 \stackrel{g^{-1}}{\longrightarrow}\dots \stackrel{g^{-1}}{\longrightarrow}a' \in A\]
Това означава, че наистина $b \in B_1$.

Ясно е, че всички тези функции $f_i$, $g_i$ са инективни, $i = 1,2,3$.
За да построим биективна функция $h:A\to B$ е достатъчно да докажем, че 
поне една функция във всяка от двойките $(f_i,g_i)$, $i = 1,2,3$ е биективна.
Тогава ще получим $h$ като ,,слепим'' три такива биекции.
Кои от тях са биективни? Достатъчно е да проверим кои от тях са сюрективни.
Ще разгледаме всички шест функции.
\begin{enumerate}[i)]
\item 
  Да разгледаме $b \in B_1$. Това означава, че веригата започваща с $b$
  завършва в $A$. Следователно, съществува 
  $a \in A_1$, за който $a = f^{-1}(b)$.
  Заключаваме, че $f_1$ е сюрективна и следователно биективна.
\item
  Да разгледаме $b \in B_2$. Това означава, че веригата започваща с $b$
  завършва в $B$.
  Обаче може $b$ изобщо да няма наследници, т.е.
  възможно е $f^{-1}(\{b\}) = \emptyset$.
  Това означава, че може $Range(f_2) \subsetneqq B_2$ и
  нямаме гаранция, че $f_2$ е сюрективна.
\item
  Да разгледаме $b \in B_3$. Това означава, че веригата за $b$
  е безкрайно дълга.
  Следователно, съществува $a \in A_3$, за което $f^{-1}(b) = a$.
  Заключаваме, че $f_3$ е сюрективна и следователно биективна.
\item
  Да разгледаме $a \in A_1$. Това означава, че веригата за $a$
  завършва в $A$. 
  Обаче пак както в {\em ii)} може $a$ изобщо да няма наследници, т.е.
  $g^{-1}(\{a\}) = \emptyset$.
  Това означава, че може $Range(g_1) \subsetneqq A_1$ и 
  може $g_1$ да не е сюрективна.
\item
  Да разгледаме $a \in A_2$. Това означава, че веригата за $a$ завършва в  $B$.
  Следователно, съществува $b \in B_2$, за който $b = g^{-1}(a)$.
  Заключаваме, че $g_2$ е сюрективна и следователно биективна.
\item
  Да разгледаме $a \in A_3$. Това означава, че веригата с начало $a$ е безкрайно дълга.
  Следователно, съществува $b \in B_3$, за което $g^{-1}(a) = b$.
  Заключаваме, че $g_3$ е сюрективна и следователно.
\end{enumerate}

Накрая получаваме, че функциите $f_1,f_3$ и $g_2,g_3$ са биективни.

Да определим биекция $h:A\rightarrow B$ по следния начин:
\[
h(a) =
\begin{cases}
  f_1(a),     & \quad \text{ако $a\in A_1$}\\
  g^{-1}_2(a), & \quad \text{ако $a\in A_2$}\\
  f_3(a),     & \quad \text{ако $a\in A_3$}\\
\end{cases}
\]
Така доказахме, че множествата $A$ и $B$ са {\bf равномощни}, т.е. $A \sim B$.
\end{proof}

\begin{cor}
  Ако $A \subseteq B \subseteq C$ и $A \sim C$, то $B \sim C$.
\end{cor}
% \begin{proof}
%   Понеже $B \subseteq C$, то $\abs{B} \leq \abs{C}$.
%   Понеже $\abs{C} = \abs{A}$, то съществува инекция $g:C \to A$ и $g:C\to B$ също е инекция.
%   Получаваме, че $\abs{C} \leq \abs{B}$.
%   Тогава от \Th{ksb} следва, че $\abs{B} = \abs{C}$.
% \end{proof}

\begin{example}
  Не е лесно да се докаже, че $(0,1)_\Real \sim (0,1]_\Real$ като се посочи биекция.
  Обаче с \Th{ksb} това не е толкова трудно.
  \begin{enumerate}[1)]
  \item 
    Очевидно е, че има инекция $(0,1)_\Real$ в $(0,1]_\Real$;
  \item
    Можем да дефинираме инекция $f:(0,1]_\Real \to (0,1)_\Real$
    като $f(x) = \frac{x}{2}$.    
  \end{enumerate}
  Като имаме 1) и 2), от \Th{ksb} следва, че двете множества са равномощни.
  Макар и да не сме посочили такава биекция, то от теоремата знаем, че тя съществува.  
\end{example}

% \begin{problem}
%   Докажете, че ако $g:A\rightarrow B$ е сюрекция, то $|A|\geq |B|$;
% \end{problem}
% \begin{proof}
%   Понеже $g$ е сюрективна, то $g^{-1}(\{b\}) \neq \emptyset$, за всяко $b \in B$.
%   Да отбележим също, че $b \neq b'$, то $g^{-1}(\{b\}) \cap g^{-1}(\{b'\}) = \emptyset$.
%   Всяка функция $f:B\to A$, която изпълнява свойството, че $f(b) \in g^{-1}(\{b\})$,
%   е инективна и следователно $|B| \leq |A|$.
% \end{proof}

\begin{framed}
\begin{remark}
  Напълно възможно е за две множества $A$ и $B$ да имаме, че  $B \subsetneqq A$, но $A \sim B$.
  Например, нека $A = \Nat$ и $B = \{2n\mid n\in\Nat\}$.
\end{remark}
\end{framed}

\begin{thm}
  \marginpar{Това означава, че можем да образуваме все по-неизброими множества. Например, $\Ps(\Real)$ има по-голяма мощност от $\Real$}
  Нека $A$ е множество и $\Ps(A)$ е множеството от всички подмножества на $A$.
  Докажете, че $A \precneqq \Ps(A)$.
\end{thm}
\begin{proof}
  Функцията $h:A \to \Ps(A)$ определена като $h(a) = \{a\}$ е инекция.
  Следователно, $A \preceq \Ps(A)$.

  Да допуснем, че $A \sim \Ps(A)$, т.е. 
  съществува биекция $f:A\rightarrow \Ps(A)$.
  Да разгледаме множеството \[B=\{a\in A\ \mid a\notin f(a)\}\in\Ps(A).\]
  Щом $f$ е биекция, съществува {\em единствено} $a_0\in A: f(a_0) = B$.
  Но тогава имаме следното:
  \begin{itemize}
  \item
    ако $a_0\in B$, то $a_0 \not\in f(a_0)$ и тогава $a_0\not\in B$;
  \item
    ако $a_0\not\in B$, то $a_0 \in f(a_0)$ и тогава $a_0\in B$.
  \end{itemize}
  И в двата случая достигаме до противоречие.
  Следователно {\bf не съществува биекция} от $A$ върху $\Ps(A)$.
  Накрая заключаваме, че $A \precneqq \Ps(A)$.
\end{proof}

\section{Изброими множества}

\begin{prop}
  Множеството $\Nat\times\Nat$ е изброимо безкрайно.
\end{prop}
\begin{hint}
  Целта е да намерим биекция от $\Nat\times \Nat$ върху $\Nat$.
  Съществуват много такива функции.
  \begin{enumerate}[a)]
  \item 
    \marginpar{Нарича се Канторово кодиране. Има удобно графично представяне}
    Разгледайте функцията 
    \[\pi(x,y) = \frac{1}{2}((x+y)^2+3x+y).\]
  \item
    Разгледайте функцията
    \[\pi(x,y) = 2^x(2y+1)-1.\]
  \end{enumerate}
\end{hint}

\begin{prop}
  \label{pr:pi-k}
  За всяко $k$, множеството $\Nat^k$ е изброимо безкрайно.
\end{prop}
\begin{hint}
  Индукция по $k \geq 2$.
  \begin{itemize}
  \item 
    За $k = 2$, от предишното твърдение имаме биекцията $\pi:\Nat^2 \to \Nat$.
    Да положим $\pi_2 = \pi$.
  \item
    Нека $k = m+1$.
    Тогава \[\pi_{m+1}(n_1,\dots,n_{m+1}) = \pi_m(f(n_1,\dots,n_m),n_{m+1}),\]
    където сме използвали биекцията $\pi_m:\Nat^m \to \Nat$, която имаме от И.П.
  \end{itemize}
\end{hint}

\begin{prop}
  Ако $A$ е изброимо безкрайно множество, то $A^k$ също е изброимо безкрайно множество,
  където $k \geq 2$ е естествено число.
\end{prop}

\begin{prop}
  Да разгледаме редица от изброимо безкрайни множества $A_0,A_1,\dots$ със свойството, че $i \neq j \implies A_i \cap A_j = \emptyset$.
  Тогава множеството 
  \[B = \bigcup^\infty_{i=0}A_i\] е изброимо безкрайно.
\end{prop}

\begin{framed}
  \begin{thm}[Кантор 1874]
    Множеството на рационалните числа $\Q$ е изброимо безкрайно.
  \end{thm}
\end{framed}
\begin{hint}
  Разгледайте за $n = 1,2,3\dots$ множествата 
  \[Q_n = \left\{\frac{m}{n} \mid m \in \Int\ \&\ \text{НОД}(m,n)=1\right\}.\]
  Всяко от тези множества е изброимо безкрайно.
  Тогава 
  \[\Q = \bigcup_{n\geq 1}Q_n\]
  е изброимо безкрайно множество.
\end{hint}

\begin{problem}
  Докажете, че следните множества са изброимо безкрайни:
  \begin{enumerate}[a)]
  \item 
    $A \cup B$, където поне едното от $A$ и $B$ е изброимо безкрайно;
  \item
    $\bigcup_{i\in\Nat} A_i$, където всяко от множествата $A_i$ да е изброимо безкрайно, за $i = 0,1,2,\dots$;
  \item
    $A \times B$, където поне едно от множествата $A$ и $B$ е изброимо безкрайно;
  \item
    \marginpar{Озн. $A^\star = \bigcup_{n\in\Nat}A^n$}
    $A^\star$, където $A$ е крайна азбука;
  \item
    $A^\star$, където $A$ е изброимо безкрайна азбука;
  \item
    $B$ - множеството от тези думи над азбуката $\{0,1\}$, които не започват с $0$, с изключение на 
    думата $0$, т.е. $B = \{0, 1, 10, 11, 100, 101, 110, 111, \dots\}$;
  \item
    $\Ps_{fin}(\Nat)$ - множеството от всички крайни подмножества от естествени числа;
  \item
    $\Ps_{fin}(A^*)$ - множеството от всички крайни подмножества на $A^*$, за произволна азбука 
    крайна или изброимо безкрайна азбука $A$;
  \item
    съвкупността от всички полиноми на една променлива с цели коефициенти;
  \item
    съвкупността от всички реални алгебрични числа (т.е. корени на полиноми с цели коефициенти).
  \item
    $[0,1]_{\mathbb{Q}} = \{q \in \mathbb{Q} \mid 0 \leq q \leq 1\}$;
  \item
    $[a,b]_{\mathbb{Q}} = \{q \in \mathbb{Q} \mid a \leq q \leq b\}$, за произволни рационални числа $a < b$;
  \end{enumerate}
\end{problem}
\begin{proof}
  \begin{enumerate}[a)]
  % \item
  %   Щом $A \subseteq A\cup B$, то $\abs{A} \leq \abs{A\cup B}$ и следователно
  %   $\abs{\Nat} \leq \abs{A\cup B}$.
  %   Нека $f:A\to \Nat$ и $g:B\to\Nat$ са инекции.
  %   Функцията $h:A\cup B\to \Nat$, определена като:
  %   \begin{align*}
  %     h(x) = 
  %     \begin{cases}
  %       2f(x), & x \in A\setminus B\\
  %       2g(x) + 1, & \mbox{ иначе}
  %     \end{cases}
  %   \end{align*}
  %   също е инекция.
  %   Тогава $\abs{A\cup B} \leq \abs{\Nat}$ и следователно, 
  %   $A\cup B$ е изброимо множество. 
  %   Съобразете, че $A\cup B$ е също така и безкрайно.
  \item[г)]
    Нека $A = \{a_1,\dots,a_k\}$.
    Лесно се съобразява, че $\abs{A^n} = k^n$.
    За някое $n$, да разгледаме множеството от думи 
    \[A^n = \{\alpha^n_{1},\alpha^n_{2},\dots, \alpha^n_{{k^n}}\}.\]
    \marginpar{\todo Докажете, че $f$ е биекция!}
    Можем да дефинираме инективната функция $f_n : A^n \to \Nat$ като
    \[f_n(\alpha^n_{i}) = \sum_{i<n} k^i + i.\]
    Понеже $A^n \cap A^{n+1} = \emptyset$, 
    то $f = \bigcup_n f_n : A^\star \to \Nat$ е функция.
  \item[д)]
    Нека $\kappa:\Nat \to A$ е биекция.
    Да изброим всички букви като $a_i = \kappa(i)$ за $i = 0,1,2,\dots$.
    \marginpar{\todo Докажете, че $f$ е биекция!}
    Дефинираме биекция $f:A^\star \to \Nat$ по следния начин:
    \[f(a_0,\dots,a_n) = \pi(n, \pi_{n+1}(a_0,\dots,a_n)),\]
    където използваме функциите дефинирани в Твърдение \ref{pr:pi-k}.
  \item[ж)]
    Нека на крайното множество от естествени числа
    \[D = \{n_0 < n_1 < \cdots < n_k\}\]
    да съпоставим числото $v = 2^{n_0} + 2^{n_1} + \cdots + 2^{n_k}$, което ще наричаме код на $D$.
    \marginpar{Ако $D = \{1,3,4\}$, то $v = (11010)_2 = 26$}
    С $D_v$ ще означаваме крайното множество с код $v$.
    Разгледайте $f:\Nat \to \Ps_{fin}(\Nat)$ дефинирана като
    \marginpar{\todo Докажете, че $f$ е биекция!}
    $f(v) = D_v$.
  % \item[к)]
  %   Всеки помином може да се представи като дума над азбуката $\Nat$.
  % \item[м)]
  %   Да разгледаме инективната функцията $f:\Nat \to [0,1]_{\mathbb{Q}}$ определена като:
  %   \[f(n) = \frac{1}{2^n}.\]
  %   Използвайте теоремата на Кантор-Шрьодер-Бернщайн.
  % \item[н)]
  %   Да фиксираме $a < a_1 < b_1 < b$.
  %   Дефинираме $f:\Nat \to [a,b]_{\mathbb{Q}}$ по следния начин:
  %   \begin{align*}
  %     & f(0) = \frac{a_1+b_1}{2}\\
  %     & f(n+1) = \frac{a_1+f(n)}{2}.
  %   \end{align*}
  \end{enumerate}
\end{proof}

\begin{problem}
  Нека $A$ е изброимо безкрайно множество.
  Докажете, че всяко $I \subseteq A$ е изброимо безкрайно или крайно.
\end{problem}
\begin{hint}
  Достатъчно е да разгледаме случая $A = \Nat$.
  Да разгледаме безкрайното подмножество $I \subseteq \Nat$.
  За да докажем, че то е {\em изброимо}, ще построим биекция $f:\Nat \to I$.
  Нека
  \begin{align*}
    f(0)   & = \min\{i \mid i \in I\}\\
    f(n+1) &= \min\{i \mid i \in I \setminus\{f(0),\dots,f(n)\}\}.
  \end{align*}
  Докажете, че $f$ е биекция от $\Nat$ върху $I$.
\end{hint}

\section{Неизброими множества}

\begin{problem}
  Докажете, че $\Nat^\Nat = \{f \mid f:\Nat \to \Nat\}$ е неизброимо.
\end{problem}
\begin{proof}
  Ще приложим метода на диагонализацията. 
  Да допуснем, че $\Nat^\Nat$ е изброимо.
  Тогава можем да подредим в редица всички функции \[f_0,f_1,f_2,\dots.\]
  Дефинираме функция $\kappa$, като $\kappa(i) = f_i(i)+1$.
  Да допуснем, че $\kappa = f_n$, за някое $n$.
  Но $\kappa(n) = f_n(n)+1 \neq f_n(n)$, следователно стигаме до противоречие.
  Заключаваме, че $\Nat^\Nat$ е неизброимо.
\end{proof}

% \begin{thm}[Кантор]
%   Интервалът от реални числа $[0,1]$ е неизброим.
% \end{thm}
% \begin{proof}
%   Да допуснем, че
%   \[[0,1] = \{r_1,r_2,\dots,r_n,\dots\},\]
%   т.е. можем да изброим всички реални числа в интервала $[0,1]$.
%   Нека $I_0 = [0, 1]$, $a_0 = 0$, $b_0 = 1$.
%   Да разгледаме интервалите $[0,1/3]$, $[1/3,2/3]$ и $[2/3,1]$ 
%   и да означим като $I_1 = [a_1,b_1]$ един от тях, за който $r_1 \not\in I_1$.
%   Ясно е, че $b_1-a_1 = 1/3$ и $[a_1,b_1] \subset [a_0,b_0]$.
%   Продължаваме процедурата като разгледаме интервала $[a_1,b_1]$ разделен пак на три равни части.
%   Избираме една от тези части $I_2 = [a_2,b_2]$, за която $r_2 \not\in [a_2,b_2]$.
%   Ясно е, че $b_2-a_2 = 1/3^2$ и $[a_2,b_2] \subset [a_1,b_1]$.
%   По този начин продължаваме процедурата като на стъпка $n+1$ 
%   избираме интервал 
%   \[I_{n+1} = [a_{n+1},b_{n+1}],\] за който $r_{n+1} \not\in [a_{n+1},b_{n+1}]$.
%   Тогава $b_{n+1}-a_{n+1} = 1/3^{n+1}$ и $I_{n+1} \subset I_n$.

%   Накрая получаваме безкрайна редица $\{I_n\}_{n\in\Nat}$, като имаме свойствата:
%   \[(\forall n)[I_{n+1}\subset I_n],\]
%   \[0\leq a_n \leq a_{n+1} < b_{n+1} \leq b_n \leq 1.\]
%   Редиците $\{a_n\}$ и $\{b_n\}$ са монотонни и ограничени, следователно са сходящи 
%   (т.е. съществуват $\lim_{n\to\infty} a_n$ и $\lim_{n\to\infty} b_n$).
%   Освен това, от \[(\forall n)[b_n-a_n \leq 1/3^n]\] следва, че 
%   \[\lim_{n\to\infty}(b_n-a_n) = 0\] и тогава съществува реално число 
%   \[r = \lim_n a_n = \lim_n b_n.\]
%   За това число $r$,
%   \[(\forall n)[r \neq r_n],\]
%   защото $r \in I_n$, но $r_n \not\in I_n$.
%   Достигаме до противоречие.
%   Следователно заключаваме, че не можем да подредим всички реални числа в интервала $[0,1]$
%   в една редица.
% \end{proof}

Представяме и друго доказателство на горната теорема.
\begin{problem}
  Докажете, че отвореният интервал от реални числа $(0,1)_\Real$ е неизброимо множество.
\end{problem}
\begin{proof}
  Да допуснем, че интервалът $(0,1)_\Real$ е изброим. Това означава, че можем да подредим всички елементи на $(0,1)_\Real$ в редица.
  Да представим всяко реално число в интервала $(0,1)_\Real$ в неговата десетична форма.
  Някои реални числа могат да имат по две десетични форми.
  Например, 
  \[0.2 = 0.1999999\dots.\]
  Нека винаги избираме тази, която започва с по-малко число, например избираме $0.1999\dots$ вместо $0.2$.
  Да подредим всички реални числа в интервала $(0,1)_\Real$ в редица:
  \begin{align*}
    r_0 & = 0.d_{00}d_{01}d_{02}\dots\\
    r_1 & = 0.d_{10}d_{11}d_{12}\dots\\
    \vdots\\
    r_n & = 0.d_{n01}d_{n1}d_{n2}\dots\\
    \vdots\\
  \end{align*}

  Да изберем две различни числа между 1 и 9. Например, 5 и 7.
  Нека 
  \begin{align*}
    k_i = 
    \begin{cases}
      5, & \mbox{ ако } d_{ii} = 7,\\
      7, & \mbox{ ако } d_{ii} \neq 7.
    \end{cases}
  \end{align*}
  Дефинираме $\kappa$ като реалното число, което има десетично представяне $0.k_0k_1k_2\dots$.
  Ясно е, че $\kappa \in (0,1)_\Real$. Ще покажем, че $\kappa \neq r_n$ за всяко $n$.
  Да отбележим, че понеже в $\kappa$ не участват редици от $0$-ли или $9$-ки, то със сигурност 
  реалното число $\kappa$ има единствено десетично представяне.
  Да допуснем, че $\kappa = r_n$, за някое $n$.
  Но $k_{nn} \neq d_{nn}$, следователно стигаме до противоречие.
  Заключаваме, че $(0,1)_\Real$ е {\bf неизброимо}.
\end{proof}

\begin{remark}
  От последната задача директно следва, че множеството $\Real$ е {\bf неизброимо} безкрайно.
\end{remark}


В следващата задача ще видим, че е удобно да можем да представяме 
всяко реално число в двоична бройна система.
Например,
\begin{align*}
  5.375 & = (1.2^2 + 0.2^1 + 1.2^0).(0.2^{-1} + 1.2^{-2} + 1.2^{-3} + 0.2^{-4} + 0.2^{-5} + \dots)\\
  & = (101.011)_2\\
  1 & = 0.(1.2^{-1} + 1.2^{-2} + 1.2^{-3} + 1.2^{-4} + 1.2^{-5} + \dots)\\
  & = (0.11111\dots)_2.
\end{align*}
Естествено, много реални числа ще имат безкраен запис в двоична бройна система.
Да разгледаме $\pi = 3.14159\dots$ и да видим как можем да намираме все по-добри негови 
апроксимации в двоична бройна система.
Да умножим $\pi$ по $2^3$. Получаваме число между $25$ и $26$. 
$25 = (11001)_2$ и следователно двоичният запис на $\pi$ започва с $(11.001)_2$, т.е.
преместваме двоичната точка $3$ места наляво. Да проверим дали това наистина е така.
Сега ако умножим $\pi$ по $2^6$, получаваме число между $201$ и $202$.
$201 = (11001001)_2$. Наистина, двоичният запис на $\pi$ започва с $(11.001001)_2$,
т.е. преместване двоичната точка $6$ места наляво.
% \end{framed}

% \begin{problem}
%   Множеството $\Ps(\Nat)$ е равномощно с това на затворения интервал от реални числа $[0,1]$.
% \end{problem}
% \begin{proof}
%   Ще използваме Теорема \ref{th:ksb}. За целта, първо ще докажем $|\Ps(\Nat)|\leq |[0,1]|$ и след това $|[0,1]|\leq |\Ps(\Nat)|$.
%   \begin{enumerate}[1)]
%   \item 
%     \marginpar{На практика доказваме, че $|\Ps(\Nat)| \leq |(0,1)|$}
%     Ще докажем $|\Ps(\Nat)|\leq |[0,1]|$.
%     Дефинираме $h:\Ps(\Nat)\to [0,1]$ като на всяко подмножество от естествени числа съпоставяме реално число в десетичен запис.
%     \[h(S) = 0.d_0d_1d_2\dots,\mbox{ където } d_i = 1,\mbox{ ако }i\in S\mbox{, иначе } d_i = 0.\]
%     Например,
%     \begin{enumerate}[]
%     \item
%       $h(\emptyset) = 0.0000\dots$
%     \item
%       $h(\{0\}) = 0.100000\dots$
%     \item
%       $h(\{1,2\}) = 0.011000000\dots$
%     \item
%       $h(\Nat) = 0.11111111111\dots$
%     \end{enumerate}
%     Лесно се вижда, че $h$ е инекция, следователно $|\Ps(\Nat)|\leq|[0,1]|$.
%   \item
%     Ще докажем $\abs{[0,1]} \leq \abs{\Ps(\Nat)}$.
%     Сега ще построим инекция $g:[0,1]\to\Ps(\Nat)$, като
%     за всеки елемент $b\in[0,1]$ избираме едно негово {\em двоично представяне} (може да има повече от едно)
%     $b = (0.b_0b_1b_2\dots)_2$ и дефинираме \[g(b) = \{i\in\Nat\mid b_i = 1\}.\]
%   \end{enumerate}
  
%   Ето няколко примера:
%   \begin{enumerate}[]
%   \item
%     $g(0) = g((0.00000\dots)_2) = \emptyset$.
%   \item
%     $g(1/4) = g((0.01000\dots)_2) = \{1\}$.
%   \item
%     $g(1/2) = g((0.10000\dots)_2) = \{0\}$.
%   \item
%     $g(3/4) = g((0.1100000\dots)_2) = \{0,1\}$.
%   \item
%     $g(3/8) = g((0.01100000\dots)_2) = \{1,2\}$.
%   \item
%     $g(1) = g((0.1111111\dots)_2) = \Nat$.
%   \end{enumerate}
  
%   Едно число може да има две представяния, но ние сме сигурни, че различни числа имат различни преставяния.
% \end{proof}

\begin{problem}
  Докажете, че следните множества са равномощни и следователно са {\bf неизброими}.
  \begin{enumerate}[a)]
  \item
    Множеството на реалните числа $\Real$;
  \item
    $(0,1)_\Real = \{x \in \Real \mid 0 < x < 1\}$;
  \item
    $[0,1]_\Real = \{x \in \Real \mid 0 \leq x \leq 1\}$;
  \item
    $(a,b)_\Real = \{x \in \Real \mid a < x < b\}$, където $a<b$ са произволни реални числа.
  \end{enumerate}
\end{problem}
\begin{hint}
  \begin{description}
  \item[а) $\leftrightarrow$ б)]
    Един начин е да използваме \Th{ksb}. Това означава, че е достатъчно да дефинираме инекция $f:\Real \to (0,1)_\Real$.

    % За определеност, ако $r$ има повече от едно представяния, нека сме избрали това, което е по-голямо относно лексикографската наредба.
    % Например, ако $r = 31.999999\cdots = 32.00000\cdots$, то избираме $32.0000000\cdots$.
    % На всяко реално число \[r=r_1r_2\cdots r_n.p_1p_2\cdots p_k \cdots,\]
    % където $r_1 \neq 0$, съпоставяме реалното число в интервала $(0,1)_\Real$:
    % \[r' = 0.\underbrace{7\cdots 7}_{n+1} r_1r_2\cdots r_np_1p_2\cdots p_k \cdots.\]
    % Докажете, че функцията $f(r) = r'$ е инективна.
    Да разгледаме следните функции:
    \begin{itemize}
    \item 
      $f_1: (1,\infty)_\Real \to (0,\frac{1}{4})_\Real$ е инекция дефинирана като
      \[f_1(x) = \frac{1}{4x}.\]
    \item
      $f_2: [0,1]_\Real \to [\frac{1}{4},\frac{1}{2}]_\Real$ е инекция дефинирана като
      \[f_2(x) = \frac{1}{4} + \frac{x}{4}.\]
    \item
      $f_3: (-1,0)_\Real \to (\frac{1}{2},\frac{3}{4})_\Real$ е инекция дефинирана като
      \[f_3(x) = \frac{1}{2} - \frac{x}{4}.\]
    \item
      $f_4: (-\infty,-1]_\Real \to (\frac{3}{4},1]_\Real$ е инекция дефинирана като
      \[f_4(x) = \frac{3}{4} - \frac{1}{4x}.\]
    \end{itemize}

    За друга инекция, нека да разглеждаме реалните числа в двоичен запис.
    \marginpar{Например, $(7,25)_{10} = (111,01)_2$}
    Ако едно реално число има две представяния в двоичен запис, то взимаме по-голямото от двете в лексикографската наредба.
    Сега на реалното число $r$ съпоставяме една крайна дума $r_1\cdots r_n$ и (потенциално безкрайна) дума $p_1\cdots p_k\cdots$, такива че
    \[(r)_{10} = (r_1\cdots r_n,p_1p_2\cdots p_n \cdots)_2,\]
    и трябва да помним дали числото е положително или отрицателно.
    Сега да разгледаме следната (потенциално безкрайна) дума над азбуката $\{0,1,2\}$:
    \[\hat{r} = \underbrace{0\cdots 0}_{i}\underbrace{2\cdots 2}_{n+1}r_1r_2\cdots r_n p_1p_2\cdots, \]
    където $i = 0$ ако $r$ е положително, $i = 1$ ако $r$ е отрицателно, и $i = 2$ ако $r = 0$.
    \marginpar{\todo Проверете дали това е инекция!}
    Функцията $f(r) = (0.\hat{r})_{10}$ е инективна функция от $\Real$ в $(0,1)_\Real$.
    
    Трето решение ще бъде директно, без позоваване на \Th{ksb}.
    Знаем, че $\tan: (-\pi/2,\pi/2) \to \Real$ е биекция.
    Освен това, $f: (0,1) \to (-\pi/2,\pi/2)$, дефинирана като
    \[f(x) = \pi/2 - \pi x\] също е биекция.
    Тогава функцията $\tan\circ f : (0,1) \to \Real$ е биекция.    
    
    % Като четвърто решение, да разгледаме $f:(0,1)_\Real \to \Real$, където
    % \[f(x) = \frac{1}{x} + \frac{1}{1-x}.\]
    % Докажете, че $f$ е биективна.
  \item[a) $\leftrightarrow$ в)]
    Използвайте \Th{ksb} с едно от първите две решения на а) $\leftrightarrow$ б).
  \item[б) $\leftrightarrow$ г)]
    \marginpar{\todo Проверете, че $f$ е биекция!}
    Разгледайте функцията $f: (0,1)_\Real \to (a,b)_\Real$, където
    \[f(x) = a + (b-a)x.\]
  \end{description}
\end{hint}


\begin{problem}
  Докажете, че следните множества са равномощни и следователно са {\bf неизброими}.
  \begin{enumerate}[a)]
  \item
    $\Ps(\Nat) = \{A \mid A \subseteq \Nat\}$;
  \item
    $(0,1)_\Real = \{r \in \Real \mid 0 < r < 1\}$;
  \item
    $\Nat^\Nat = \{f\ \mid\ f:\Nat\to\Nat\text{ тотална}\}$.
  \item
    $2^\Nat = \{f\ \mid\ f:\Nat\to\{0,1\}\text{ тотална}\}$
  \end{enumerate}
\end{problem}
\begin{hint}
  \begin{description}
  \item[а) $\rightarrow$ б)]
    \marginpar{За а) $\rightarrow$ б) и б) $\rightarrow$ а) строим две инекции. След това използваме \Th{ksb} за да получим биекция между множествата от  а) и б)}
    Дефинираме $h:\Ps(\Nat)\to (0,1)$ като на всяко подмножество от естествени числа съпоставяме реално число в десетичен запис.
    \[h(S) = 0.d_0d_1d_2\dots,\mbox{ където } d_i = 1,\mbox{ ако }i\in S\mbox{, иначе } d_i = 0.\]
    Например,
    \begin{itemize}
    \item
      $h(\emptyset) = 0.0000\dots$
    \item
      $h(\{0\}) = 0.100000\dots$
    \item
      $h(\{1,2\}) = 0.011000000\dots$
    \item
      $h(\Nat) = 0.11111111111\dots$
    \end{itemize}
    Лесно се вижда, че $h$ е инекция, следователно $|\Ps(\Nat)|\leq|(0,1)|$.

  \item[б) $\rightarrow$ а)]
    Ще построим инекция $g:(0,1)_\Real\to\Ps(\Nat)$, като
    за всеки елемент $b\in (0,1)_\Real$ избираме едно негово {\em двоично представяне} (може да има повече от едно)
    $b = (0.b_0b_1b_2\dots)_2$ и дефинираме \[g(b) = \{i\in\Nat\mid b_i = 1\}.\]
    За определеност, ако едно реално число има повече от едно представяния, избираме това, което е най-голямо относно лексикографската наредба.
    Например, 
    \[1/2 = (0,100000\dots)_2 = (0,011111\dots)_2.\]

    Ето няколко примера:
    \begin{itemize}
    \item
      $g(0) = g((0.00000\dots)_2) = \emptyset$.
    \item
      $g(1/4) = g((0.01000\dots)_2) = \{1\}$.
    \item
      $g(1/2) = g((0.10000\dots)_2) = \{0\}$.
    \item
      $g(3/4) = g((0.1100000\dots)_2) = \{0,1\}$.
    \item
      $g(3/8) = g((0.01100000\dots)_2) = \{1,2\}$.
    \item
      $g(1) = g((0.1111111\dots)_2) = \Nat$.
    \end{itemize}
    
    Едно число може да има две представяния, но ние сме сигурни, че различни числа имат различни представяния.
  \item[а) $\rightarrow$ г)]
    Да разгледаме едно множество $A \subseteq \Nat$.
    Съпоставяме на $A$ функцията $f_A:\Nat \to \{0,1\}$  по следния начин:
    \marginpar{$f_A$ се нарича характеристична функция за $A$}
    \[
    f_A(n) = 
    \begin{cases}
      1, & n \in A\\
      0, & n\not\in A.
    \end{cases}
    \]
    Проверете какви свойства има функцията $h:\Ps(A) \to 2^\Nat$ дефинирана като $h(A) = f_A$.
  \item[г) $\rightarrow$ а)]
    Да разгледаме функцията $f:\Nat\to\{0,1\}$.
    На нея съпоставяме множеството $A_f = \{n \in \Nat \mid f(n) = 1\}$.
    Проверете какви свойства има функцията $h:2^\Nat \to \Ps(A)$ дефинирана като $h(f) = A_f$.
  \item[б) $\rightarrow$ в)]
    Да разгледаме една функция $f:\Nat\to\Nat$.
    На нея съпоставяме реалното число $r_f \in (0,1)_\Real$, където
    \[r_f = 0,\underbrace{00\dots 0}_{f(0)+1}1\underbrace{00\dots 0}_{f(1)+1}1\dots\underbrace{00\dots 0}_{f(2)+1}1\dots\]
    Проверете какви свойства има функцията $h:\Nat^\Nat \to (0,1)_\Real$ дефинирана като $h(f) = r_f$.
  \item[в) $\rightarrow$ б)]
    Да разгледаме едно реално число $r\in (0,1)_\Real$, където
    \[r = 0, r_0r_1r_2r_3\dots\]
    На това число съпоставяме функцията $f_r:\Nat\to\Nat$ като $f_r(n) = r_n$.
    Проверете какви свойства има функцията $h:(0,1)_\Real \to \Nat^\Nat$ дефинирана като $h(r) = f_r$.
  % \item[б) $\rightarrow$ е)]
  %   Всяко реално число $r$ може да се представи като безкрайна редица от рационални числа $\{q_n\}$,
  %   за която $\lim_n q_n = r$.
  %   Понеже има биекция между рационалните и естествените числа, то на всяко реално $r$ число може да се съпостави
  %   функция $f_r:\Nat\to \mathbb{Q}$, за която $\lim_n f_r(n) = r$.
  %   Следователно, имаме сюрекция от $\Nat^\Nat$ върху $\Real$, т.е. $\abs{\Real} \leq \abs{\Nat^\Nat}$.
  \end{description}
\end{hint}

\begin{problem}
  \marginpar{Озн. $B^A = \{f \mid f:A\to B\}$}
  Нека $A \sim B$ и $C \sim D$. Докажете, че $B^A \sim D^C$.
\end{problem}

% \section*{Библиография}
% \begin{enumerate}[]
% \item 
%   Много задачи могат да се намерят в \cite[§1.4]{lavrov-maksimova}.
% \item
%   Добро изложение може да се намери в \cite[Глава 7]{prove-it}.
% \end{enumerate}

%%% Local Variables: 
%%% mode: latex
%%% TeX-master: "discrete-math"
%%% End: 

%%%%%%%%%%%%%%%%%%%%%%%%%
% Combinatorics: part 1 %
%%%%%%%%%%%%%%%%%%%%%%%%%

\begin{problem}
  \begin{enumerate}[a)]
  \item
    Колко ралични думи могат да се образуват като разместим буквите на думата $MISSISSIPPI$?
  \item
    Колко ралични думи могат да се образуват като разместим буквите на думата $TENNESSEE$?
  \item
    В състезание участват 10 отбора. 
    По колко начина могат да се разпределят златните, сребърните и бронзовите медали?
  \item
    Колко различни петцифрени числа могат да се образуват чрез разместване на цифрите от 0,1,2,3,4?
  \item
    По колко различни начина могат да се настанят осем студенти в три стаи съответно с едно, три и четири легла?
  \item
    По колко различни начина четирима младежи могат да поканят на танц четири от шест девойки?
  \item
    Шест различни предмета се боядисват по следния начин: два зелен, два червен, два син цвят.
    По колко различни начина могат да се боядисат предметите?  
  \item
    По колко различни начина могат да се разпределят 10 специалисти в 4 цеха така, че в тях да попаднат съответно по 1,2,3 и 4 души?
  \end{enumerate}
\end{problem}



\begin{problem}
  \begin{enumerate}[a)]
  \item
    По колко начина могат $n$ момчета и $n$ момичета да седнат на ред с $2n$ стола, като няма двама от един пол седящи един до друг?
  \item
    По колко начина могат $n$ момчета и $n$ момичета да седнат на ред с $2n$ стола, като няма двама от един пол седящи един до друг и Иванчо и Марийка не седят един до друг? 
  \item
    По колко различни начина могат да се подредят на рафт $n$ книги, така че две от тях, определени предварително, да са една до друга?
  \item
    Колко различни гердана могат да се направят от $n$ различни перли, като се използват всичките?
  \item
    На хоро в кръг са хванали общо $n$ души, между които и Иванчо и Марийка.
    Колко са възможните подредби, при които Иванчо и Марийка са един до друг?
 \item
    На хоро в кръг са хванали общо $n$ души, между които и Иванчо и Марийка.
    Колко са възможните подредби, при които Иванчо и Марийка не са един до друг?
  \item
    Две сядания на една кръгла маса не са различни, ако всеки от седящите има едни и същи съседи.
    По колко различни начина могат да седнат около една кръгла маса:
    \begin{enumerate}
    \item
      $n (\geq 2)$ човека;
    \item
      $n$ мъже и $n$ жени, като две лица от един и същ пол не седят един до друг.
    \end{enumerate}
  \end{enumerate}
\end{problem}


\begin{problem}
  \begin{enumerate}[a)]
  \item
    Иванчо и $n$ негови приятели отиват на кино.
    По колко различни начина могат всички да седнат заедно на един ред, така че Иванчо е винаги
    между двама негови приятели.
  \item
    В магазин продават $k$ вида ябълки.
    Колко различни покупки на $n$ ябълки могат да се направят, без да се купуват повече от две ябълки от един и същ вид?
  \item
    В магазин продават $k$ вида ябълки.
    Колко различни покупки на $n$ ябълки могат да се направят, като се купи поне по една от всеки вид и $n\geq k$?
  \item
    Имаме $n$ съпружески двойки, които седят на $2n$ места около една кръгла маса. 
    По колко начина могат да седнат всички двойки, ако ротациите се броят за едно и също подреждане, и
    всеки мъж седи до половинката си.
\end{enumerate}
\end{problem}


\begin{problem}
  \begin{enumerate}[a)]
  \item
    В един жилищен блок живеят $n$ семейства всяко с поне двама души.
    По колко различни начина може да се състави комисия от $k$ души от живущите в блока, ако
    в комисията не могат да влизат членове на едно семейство?
  \item
    В партида от $N$ изделия $M$ са бракувани.
    По колко различни начина могат да се вземат от партидата $n$ изделия, така че точно $k$ от тях да бъдат бракувани ($M\leq N, k\leq n\leq N$)?
  \item
    От колода с 52 карти се изваждат 6 произволни карти без връщане.
    По колко различни начина могат да се извадят картите, така че две от тях да са тройки и две осмици?
  \item
    По колко различни начина може да се раздели колода от 52 карти на две пачки от по 26 карти така, че във всяка от тях да има по две дами?
  \item
    По колко начина може да се разпределят 8 подаръка между 4 лица, така че всеки да получи по два подаръка?
  \item
    Провежда се събрание с n присъстващи.
    По колко начина може да се избере председател, секретар и 5 членна комисия?
\end{enumerate}
\end{problem}


\begin{problem}
  \begin{enumerate}
    От колода с $52$ карти се избират $11$. По колко различни начина могат да се изберат извадки, в които се срещат:
  \item
    точно $1$ ас;
  \item
    поне $2$ валета;
  \item
    точно $4$ пики;
  \item
    най-много $5$ кари;
  \item
    точно $2$ аса и $2$ точно трефи;
  \item
    точно $2$ аса и не повече от $2$ трефи;
\end{enumerate}
\end{problem}

\begin{problem}
  \begin{enumerate}[a)]
  \item
    По колко начина могат да се изберат $n$ монети да се изберат от купчина монети с номинал 5, 10, 20 и 50 стотинки?
  \item
    По колко начина могат да се изтеглят $13$ от $52$ карти, ако ги различаваме само по цвета?
  \item
    Намерете броя на възможните начини за разпределение на $n$ {\bf неразличими} топки в $m$ различни кутии, ако всяка кутия може да побере
    всичките $n$ топки.
  \item
    Намерете броя на възможните начини за разпределение на $n$ {\bf неразличими} топки в $m$ различни кутии, ако всяка кутия може да побере
    всичките $n$ топки и съществува поне една празна кутия.
  \item
    Да се намери броя на възможните начини за разпределения на $n$ {\bf различими} топки в $m$ различни кутии, ако всяка кутия може да побере всичките $n$ топки.
  % \item
  %   Да се намери броя на възможните начини за разпределения на $n$ {\bf неразличими} топки в $m$ {\bf неразлични} кутии, ако всяка кутия може да побере всичките $n$ топки.
  \end{enumerate}
\end{problem}



\begin{problem}
  Да се намерят всички $k$-буквени думи от азбука с $n$ букви, $k\leq n$,които:
  \begin{enumerate}[a)]
  \item
    нито една буква не се повтаря;
  \item
    са симетрични;
  \item
    имат две последователни еднакви букви;
  \item
    нямат две последователни еднакви букви;
  \item
    съществува буква, която се среща точно два пъти;
  \item
    съществува буква, която се повтаря;
  \item
    поне две букви се повтарят;
  \item
    точно една буква се повтаря;
  \item
    съществува единствена буква, която се среща точно два пъти;
\end{enumerate}
\end{problem}


\begin{problem}
  Множеството от всички двоични вектори от $\{0,1\}^{n}$, които във фиксирани $n-k$ позиции имат равни значения,
  ги наричаме $k$-равнини, за $k\leq n$.
  \begin{enumerate}
  \item
    Колко различни вектора има в една $k$-равнина?
  \item
    Колко различни $k$-равнини има в $\{0,1\}^{n}$?
  \item
    Колко различни $k$-равнини съдържат даден фиксиран вектор?
  \item
    Колко различни $k$-равнини съдържат дадена $l$-равнина, $0\leq l < k$.
  \end{enumerate}
\end{problem}


\begin{problem}
  Да фиксираме естествените числа $m$ и $n$.
  Една функция $f:\{1,\dots,n\}\to\{1,\dots,m\}$ е монотонно ненамаляваща, ако
  $(\forall i\forall j)[1\leq i<j\leq n \rightarrow f(i)\leq f(j) ]$.
  \begin{enumerate}
  \item
    Колко такива функции съществуват?
  \item
    Колко от тези функции са сюрективни при $n\geq m$?
  \item
    Колко от тези функции са инективни при $n\leq m$?
\end{enumerate}
\end{problem}


% \section{Принцип на Включване и Изключване}

% \begin{problem}
%   Дадени са $n$ кутии и $m$ неразличими топки.
%   По колко начина могат да се разпределят всички топки в кутиите, така че нито в една кутия да няма повече от $r$ топки?
% \end{problem}

% \begin{problem}
%   Колко решения в естествените числа имат уравненията:
%   \begin{enumerate}
%   \item
%     $x_1+x_2+x_3 = 11$;
%   \item
%     $x_1 + x_2 + x_3 = 11, x_2 \geq 3$;
%   \item
%     $x_1+x_2+x_3 = 11, x_2 \leq 3$;
%   \item
%     $x_1+x_2+x_3 = 11, x_1 \geq 2, x_2 \geq 3$;
%   \item
%     $x_1+x_2+x_3 = 11, x_1 \geq 2, x_2 \geq 3, x_3 \leq 8$;
%   \end{enumerate}
% \end{problem}

% \begin{problem}
%   $x_1 + x_2 + x_3 = n$, където
%   $0\leq x_i \leq k$, за всяко $1\leq i \leq 3$ и $n < 3k$.
% \end{problem}

\begin{prb} % Гаврилов стр. 265, зад. 7
  Нека $U$ е множество от $n (n\geq 3)$ елемента. За всяко множество $X\subseteq U$, с $\overline{X}$ означаваме $U\setminus X$.
  \begin{enumerate}[a)]
  \item
    Намерете броя на двойките $(X,Y)$ за $X,Y\subseteq U$;
  \item
    Намерете броя на двойките $(X,Y)$ за $X,Y\subseteq U$, за които $\vert{X}\vert = 1$;
  \item
    Намерете броя на двойките $(X,Y)$ за $X,Y\subseteq U$, за които $\vert{X}\vert = 2$;
  \item
    Намерете броя на двойките $(X,Y)$ за $X,Y\subseteq U$, за които $\vert{X}\vert = k$ и $k < n$;
  \item
    Намерете броя на двойките $(X,Y)$ за $X,Y\subseteq U$, за които $\vert{(X\setminus{Y})\cup (X\setminus\overline{Y})}\vert = k$ и $k < n$;
  \item
    Намерете броя на двойките $(X,Y)$ за $X,Y\subseteq U$, за които $\vert{X\setminus Y}\vert = 1$;
  \item
    Намерете броя на двойките $(X,Y)$ за $X,Y\subseteq U$, за които $\vert{X\setminus Y}\vert = k$ и $k < n$;
  \item
    Намерете броя на двойките $(X,Y)$ за $X,Y\subseteq U$, за които $X\cap Y = \emptyset$;
  \item
    Намерете броя на двойките $(X,Y)$ за $X,Y\subseteq U$, за които $\vert{X\cap Y}\vert = k$ и $k < n$;
  \item
    Намерете броя на двойките $(X,Y), X,Y\subseteq U$, за които $|(X\setminus Y)\cup(Y\setminus X)| = 1$;
  \item
    Намерете броя на двойките $(X,Y), X,Y\subseteq U$, за които $|(X\setminus Y)\cup(Y\setminus X)| = k$ и $k < n$;
  \item
    Намерете броя на тройките $(X,Y,Z), X,Y,Z\subseteq U$, за които $X\cup Y\overline{Z} = \overline{X}\cup\overline{Y}$;
  \item
    Намерете броя на тройките $(X,Y,Z), X,Y,Z\subseteq U$, за които $Y\cup X = Z\cup\overline{Y}$;
  \item
    Намерете броя на двойките $(X,Y), X,Y\subseteq U$, за които $X\cap Y = \emptyset$ и
    $|X|\geq 1$, $|Y|\geq 1$;
  \item
    Намерете броя на двойките $(X,Y), X,Y\subseteq U$, за които $X\cap Y = \emptyset$ и 
    $|X|\geq 2, |Y|\geq 2$;
  \item
    Намерете броя на двойките $(X,Y), X,Y\subseteq U$, за които $|(X\setminus Y)\cup(Y\setminus X)| = 1$ и 
    $|X|\geq 2, |Y|\geq 2$;
  \item
    Намерете броя на тройките $(X,Y,Z), X,Y,Z\subseteq U$, за които $X\cup Y\overline{Z} = \overline{X}\cup\overline{Y}$ и
    $|Z| = 0$.
  \item
    Намерете броя на тройките $(X,Y,Z), X,Y,Z\subseteq U$, за които $X\cup Y\overline{Z} = \overline{X}\cup\overline{Y}$ и
    $|X|\geq 1, |Y|\geq 1, |Z| = 1$.
  \item
    Намерете броя на тройките $(X,Y,Z), X,Y,Z\subseteq U$, за които $X\cup Y\overline{Z} = \overline{X}\cup\overline{Y}$ и
    $|X|\geq 1, |Y|\geq 1, |Z|\leq 1$.
  \end{enumerate}
\end{prb}


\chapter{Езици и автомати}

\begin{lemma}[за разрастването (регулярни езици)]
  % \index{лема за разрастването!регулярни езици}
  % \label{lem:pumping-reg}
  % \marginpar{На англ.\\ Pumping Lemma}
  Нека $\Ls$ да бъде безкраен регулярен език.
  Съществува число $n\geq 1$, зависещо само от $\Ls$, 
  за което за всяка дума $\alpha\in \Ls, \abs{\alpha}\geq n$ може да 
  бъде записана във вида $\alpha = xyz$ и 
  \begin{enumerate}
  \item
    $|y|\geq 1$;
  \item
    $|xy|\leq n$;
  \item
    % \marginpar{$i = 0\ \rightarrow\ xz \in \Ls$}
    $(\forall i\in\N)[xy^iz \in \Ls]$.
  \end{enumerate}
\end{lemma}

\begin{crl}
  Регулярният език $\Ls$, 
  разпознаван от КДА $M$ е непразен тoчно тогава, когато съдържа дума $\alpha, \abs{\alpha} \leq \abs{Q}$.
\end{crl}

\section{Регулярни езици}
\begin{problem}
  Нека $\Sigma = \{0,1\}$.  Проверете дали $L$ е регулярен, където
  \begin{enumerate}[1)]
  \item
    $L_1 = \{0^i1^i\ \mid\ i\geq 0\}$;
  \item
    $L_2 = \{0^i1^j\ \mid\ i > j\}$;
  \item
    $L_3 = \{0^{2n}\ \mid\ i\geq 1\}$;
  \item
    $L_4 = \{0^1m1^n0^{m+n}\ \mid\ m\geq 1\ \&\ n\geq 1\}$;
  \item
    $L_5 = \{0^n\ \mid\ n\mbox{ е просто }\}$;
  \item
    $L_6 = \{w\mid w\in\{0,1\}^\star\mbox{ има равен брой нули и единици}\}$;
  \item
    $L_7 = \{ww\mid w\in\{0,1\}^\star\}$;
  \item
    $L_8 = \{1^{n^2}\mid n\geq 0\}$;
  \item
    $L_{9} = \{0^n1^n2^n\mid n\geq 0\}$;
  \item
    $L_{10} = \{www\mid w\in \{0,1\}^\star\}$;
  \item
    $L_{11} = \{0^{2^n}\mid n\geq 0\}$;
  \item
    $L_{12} = \{0^m1^n\mid n\neq m\}$;
  \end{enumerate}
\end{problem}


%%% Local Variables: 
%%% mode: latex
%%% TeX-master: "discrete-math"
%%% End: 
?

\chapter{Контекстно-свободни езици}

\section{Свойства}

Някои свойства на контекстно-свободните езици:
\begin{itemize}
\item 
  те са затворени относно операцията обединение, т.е.
  ако $L_1, L_2$ са контекстно-свободни, то езикът $L_1 \cup L_2$ е контекстно-свободен; 
\item
  те са затворени относно операцията конкатенация, т.е.
  ако $L_1, L_2$ са контекстно-свободни, то езикът $L_1 \cdot L_2$ е контекстно-свободен; 
\item
  те са затворени относно операцията звезда на Клини, т.е.
  ако $L$ е контекстно-свободен, то езикът $L^\star = \bigcup_{n\in\Nat} L^n$ е контекстно-свободен; 
\item
  те {\bf не} са затворени относно операцията сечение, т.е.
  съществуват контекстно-свободни езици $L_1, L_2$, за които езикът $L_1 \cap L_2$ {\bf не} е контекстно-свободен; 
\item
  те са затворени относно сечение с регулярен език, т.е.
  ако $L$ е контекстно-свободен език и $R$ е регулярен език, то езикът $L = L \cap R$ е контекстно-свободен; 
\item
  те {\bf не} са затворени относно операцията допълнение, т.е.
  съществува контекстно-свободен език $L$, за който езикът $\Sigma^\star\setminus L$ {\bf не} е контекстно-свободен; 
\item
  те са затворени относно хомоморфизми, т.е.
  ако $L \subseteq \Sigma^\star_1$ е контекстно-свободен език и $h:\Sigma_1\to\Sigma^\star_2$ е хомоморфизъм, 
  то езикът $h(L) = \{h(\alpha) \in \Sigma^\star_2 \mid \alpha \in L\}$
  е контекстно-свободен.
\item
  те са затворени относно обратни хомоморфизми, т.е.
  ако $L\subseteq \Sigma^\star_2$ е контекстно-свободен език и $h:\Sigma_1\to\Sigma^\star_2$ е хомоморфизъм, 
  то езикът $h^{-1}(L) = \{\alpha \in \Sigma^\star_1 \mid h(\alpha) \in L\}$
  е контекстно-свободен.
\end{itemize}

\section{Контекстно-свободни граматики}
% От Сипсер, същото е в слайдовете на Сашка
% Малко е тъпо, че в Пападимитриу дефиницията е различна. Там \Sigma \subseteq V
\begin{dfn}
  \marginpar{На англ. context-free grammar}
  Контекстно-свободна граматика e четворка \[G = (V,\Sigma,R,S),\]
  където
  \begin{itemize}
  \item
    $V$ е крайно множество от {\em променливи};
  \item
    $\Sigma$ е крайно множество от {\em терминали}, $\Sigma \cap V = \emptyset$;
  \item
    $R \subseteq V\times (V\cup\Sigma)^\star$, крайно множество от {\em правила};
  \item
    $S \in V$ е началната променлива. 
  \end{itemize}
  Обикновено ще пишем $A \rightarrow_G v$ вместо $(A,v) \in R$.
  Пишем $u \Rightarrow_G v$, ако съществуват думи $x,y\in (\Sigma\cup V)^\star$, $A\in V$,
  правило $A\rightarrow_G v^\prime$ и $u = xAy$, $v = xv^\prime y$.
  Езикът генериран от $G$, $L(G) = \{\alpha\in\Sigma^\star\mid S \Rightarrow^\star_G \alpha\}$.
\end{dfn}

\begin{problem}
  Да се докаже, че езикът $\{\alpha \in \{a,b\}^\star\mid n_a(\alpha) = n_b(\alpha)\}$ 
  е контекстно свободен.
\end{problem}
\begin{proof}
  $S \rightarrow aSbS\vert bSaS \vert\varepsilon$  и да се докаже, че
  ако $\alpha = a\alpha^\prime$ и $n_a(\alpha) = n_b(\alpha)$,
  то съществуват $\alpha_1, \alpha_2$, $\alpha = a\alpha_1b\alpha_2$ и
  $n_a(\alpha_1) = n_b(\alpha_1)$, $n_a(\alpha_2) = n_b(\alpha_2)$.
  Аналогично е, ако $\alpha = b\alpha^\prime$.

  Алтернативна граматика е $S\rightarrow aB\vert bA, A\rightarrow a\vert aS\vert bAA, B\rightarrow b\vert bS\vert aBB$.
  
  Да се обясни защо граматиката $S\rightarrow aB\vert Ba\vert \varepsilon, B\rightarrow bS\vert Sb$ не работи.
\end{proof}

\begin{problem}
  Докажете, че следните езици са контекстно-свободни:
  \begin{enumerate}[1)]
  \item
    $\{ww^R \mid w \in \{a,b\}^\star\}$;
  \item
    $\{w \in \{a,b\}^\star \mid w = w^R\}$;
  \item
    $\{a^nb^{2n} \mid n \in \Nat\}$;
  \item
    $\{a^nb^k \mid n > k\}$;
  \item
    $\{a^nb^k \mid n \geq 2k\}$;
  \item
    $\{a^mb^nc^k\mid m+n \geq k\}$;
  \item
    $\{a^nb^mc^{n+m}\mid n,m \in \Nat\}$;
  \end{enumerate}
\end{problem}
\begin{proof}
  За 2), $S \rightarrow aSa \vert bSb \vert a\vert b \vert \varepsilon$.
  Докажете, че ако $w = w^R$, то $\abs{w} = 2n$, $w = w_1w^R_1$ и ако $\abs{w} = 2n+1$, то $w = w_1aw^R_1$, $w = w_1bw^R_1$.
  За 6), $S \rightarrow aSc\vert aS \vert B, B\vert bSc\vert bS\vert\varepsilon$.
\end{proof}


\section{Езици, които не са контекстно-свободни}

\begin{lemma}[за нарастването (контекстно-свободни езици)]
  \index{лема за нарастването!контекстно-свободни езици}
  \label{lem:pumping-context} 
  За всеки КСЕ $L\neq\{\epsilon\}$ съществува $n>0$ такова,
  че ако $\alpha\in L, \abs{\alpha} \geq p$, то $\alpha=xyuvw$ и
  \begin{enumerate}
  \item
    $\abs{yv}\geq 1$,
  \item
    $\abs{yuv}\leq p$, и
  \item
    $(\forall i\geq 0)[xy^iuv^iw\in L]$.
\end{enumerate}
\end{lemma}

\begin{crl}
  Нека $G$ е контекстно-свободна граматика и $n$ е константата за разрастването на $G$.
  Тогава $\abs{L(G)} = \infty$ точно тогава, когато съществува $z \in L(G)$, за която $n \leq \abs{z} \leq 2n$.
\end{crl}

\begin{problem}
  Да се даде пример за език $L$, който {\bf не} е контекстно-свободен, но удовлетворява
  лемата за разрастването.
\end{problem}

\begin{example}
  Приложете лемата за нарастването за да докажете, че 
  следните езици не са контекстно-свободни:
  \begin{itemize}
  \item
    $L_1 = \{a^ib^jc^k\ \mid\ 0 \leq i \leq j \leq k\}$;
  \item
    $L_2 = \{ww\mid w\in \{a,b\}^\star\}$;
  \end{itemize}
\end{example}
\begin{proof}
  \begin{enumerate}[1)]
  \item
    Разгледайте $w = a^pb^pc^p$.
    \begin{enumerate}[a)]
    \item
      Знаем, че поне една от $y$ и $v$ не е празната дума.
      Имаме три случая за поддумите $y$ и $v$.
      \begin{enumerate}[i)]
      \item
        $a$ не се среща в $y$ и $v$.
        Тогава $xy^0vu^0w$ съдържа повече $a$ от $b$ или $c$.
      \item
        $b$ не се среща в $y$ и $v$.
        Ако $a$ се среща в $y$ или $v$, тогава $xy^2uv^2w$ съдържа повече $a$ от $b$
        Ако $c$ се среща в $y$ или $v$, тогава $xy^0uv^0w$ съдържа по-малко $c$ от $b$.
      \item
        $c$ не се среща в $y$ и $v$.
        Тогава $xy^2uv^2w$ съдържа повече $a$ или $b$ от $c$.
      \end{enumerate}      
    \item
      $y$ или $v$ е съставена от две букви.  Контекстно-свободните езици {\bf не} са затворени относно сечение и допълнение.
      Тук разглеждаме $xy^2uv^2w$ и съобразяваме, че редът на буквите е нарушен.
    \end{enumerate}
  \item
    \marginpar{Защо $\alpha = a^pba^pb$ не е добър кандидат?}
    Разгледайте $\alpha = a^pb^pa^pb^p$.
    \begin{enumerate}[a)]
    \item
      Ако $yuv$ е в първата част на думата, то 
      $xy^0uv^0w = a^ib^ja^pb^p \not\in L_3$.
      Аналогично ако $yuv$ е във втората част на думата.
    \item
      Ако $yuv$ е в двете части на думата, то 
      Но $xy^0uv^0w = a^pb^ia^jb^p \not\in L_3$.
    \end{enumerate}    
  \end{enumerate}
\end{proof}


\begin{problem}
  Проверете дали следните езици са контекстно-свободни:
  \begin{enumerate}[a)]
  \item
    $\{a^nb^{2n}c^{3n}\ \mid\ n\in\N\}$;
  \item
    $\{a^mb^n\mid\ m \neq n\}$;
  \item
    $\{www\mid w\in \{a,b\}^\star\}$;
  \item
    $\{ww^R\mid w\in \{a,b\}^\star\}$;
  \item
    $\{a^{n^2}b^n\ \mid n \in \Nat\}$;
  \item
    $\{a^p\ \mid\ p\mbox{ е просто }\}$;
  \item
    $\{a^nb^na^nb^n\mid n\geq 0\}$;
  \item
    $\{w \in \{a,b\}^\star \mid w = w^R\}$;
  \item
    % Дефиниция на подниз
    $\{w c x\mid w,x\in \{a,b\}^\star\ \&\ w\mbox{ е подниз на }x\}$;
  \item
    $\{x_1 c x_2 c \dots c x_k\mid k\geq 2\ \&\ x_i\in\{a,b\}^\star\ \&\ (\exists i,j)[i \neq j\ \&\ x_i = x_j]\}$;
  \item
    $\{a^ib^jc^k\mid i,j,k\geq 0\ \&\ (i = j \vee j = k)\}$;
  \item
    $\{\alpha \in \{a,b,c\}^\star\mid n_a(\alpha) = n_b(\alpha) = n_c(\alpha)\}$;
  \item
    $\{a,b\}^\star \setminus \{a^nb^n\mid n\in \Nat\}$;
  \end{enumerate}
\end{problem}
% \begin{proof}
%   \begin{enumerate}
  % \item
    % За думата $w = a^pb^pc^p = xyuvw$ разгледайте различните случаи за $y$ и $v$.
  % \item[2)]
  %   Разгледайте $w = a^pb^pc^p$.
  %   \begin{enumerate}[a)]
  %   \item
  %     Знаем, че поне една от $y$ и $v$ не е празната дума.
  %     Имаме три случая за поддумите $y$ и $v$.
  %     \begin{enumerate}[i)]
  %     \item
  %       $a$ не се среща в $y$ и $v$.
  %       Тогава $xy^0vu^0w$ съдържа повече $a$ от $b$ или $c$.
  %     \item
  %       $b$ не се среща в $y$ и $v$.
  %       Ако $a$ се среща в $y$ или $v$, тогава $xy^2uv^2w$ съдържа повече $a$ от $b$
  %       Ако $c$ се среща в $y$ или $v$, тогава $xy^0uv^0w$ съдържа по-малко $c$ от $b$.
  %     \item
  %       $c$ не се среща в $y$ и $v$.
  %       Тогава $xy^2uv^2w$ съдържа повече $a$ или $b$ от $c$.
  %     \end{enumerate}      
  %   \item
  %     $y$ или $v$ е съставена от две букви.  Контекстно-свободните езици {\bf не} са затворени относно сечение и допълнение.
  %     Тук разглеждаме $xy^2uv^2w$ и съобразяваме, че редът на буквите е нарушен.
  %   \end{enumerate}
  % \item[3)]
  %   \marginpar{Защо $\alpha = a^pba^pb$ не е добър кандидат?}
  %   Разгледайте $\alpha = a^pb^pa^pb^p$.
  %   \begin{enumerate}[a)]
  %   \item
  %     Ако $yuv$ е в първата част на думата, то 
  %     $xy^0uv^0w = a^ib^ja^pb^p \not\in L_3$.
  %     Аналогично ако $yuv$ е във втората част на думата.
  %   \item
  %     Ако $yuv$ е в двете части на думата, то 
  %     Но $xy^0uv^0w = a^pb^ia^jb^p \not\in L_3$.
  %   \end{enumerate}
%   \item[10)]
%     Контекстно-свободен е. Лесно може да се напише контекстно-свободна граматика за този език.
%   \item[12)]
%     Разгледайте езика $L = L_{12} \cap a^\star b^\star c^\star$.
%   \end{enumerate}
% \end{proof}

\begin{problem}
  Проверете кои от следните езици са контекстно-свободни:
  \begin{enumerate}[a)]
  \item
    $\{a^mb^nc^k\mid m = n \vee n = k \vee m = k\}$;
  \item
    $\{a^mb^nc^k\mid m \neq n \vee n \neq k \vee m \neq k\}$;
  \item
    $\{a^mb^nc^k\mid m = n \wedge n = k \wedge m = k\}$;
  \item
    $\{w \in \{a,b,c\}^\star\mid n_a(w) \neq n_b(w) \vee n_a(w) \neq n_c(w) \vee n_b(w) \neq n_c(w)\}$.
  \end{enumerate}
\end{problem}


\section{Алгоритми}

\subsection{Нормална Форма на Чомски}

\begin{problem}
  Нека е дадена граматиката  $G = \pair{\{S,A,B,C,D,E\}, \{a,b\},S, R}$.
  \begin{enumerate}[a)]
  \item
    Намерете всички нетерминали, от които в $G$ се извежда празната дума.
  \item
    Принадлежи ли празната дума на $L(G)$?
  \item
    Постройте граматика $G_1$ без $\varepsilon$-правила, за която $L(G_1)=L(G)\setminus\{\varepsilon\}$.
  \end{enumerate}
  Множеството от правила $R$ на граматиката $G$ е зададено като:
  \begin{enumerate}
  \item
    $R = \{S\rightarrow D,D\rightarrow AD|b,A\rightarrow ACB|BC|a, B\rightarrow ABCA|CEC,C\rightarrow \varepsilon|CA|a, E\rightarrow \varepsilon|aEb\}$;
  \item
    $R = \{S \rightarrow aD, D\rightarrow \varepsilon|ABBA|ADD,A\rightarrow DEB|a,B\rightarrow DDD|DC|b,C\rightarrow CCE|a, E\rightarrow \varepsilon|bEa\}$;
  \item
    $R = \{ S\rightarrow D,D\rightarrow AD|b,A\rightarrow AB|BC|a, B\rightarrow AB|CC, C\rightarrow \varepsilon|CA|a, E\rightarrow a|EB\}$;
  \item
    $R = \{ S \rightarrow AD|a, D\rightarrow \varepsilon|BB|AD,A\rightarrow DB|a,B\rightarrow DD|DC|b,C\rightarrow CE|a, E\rightarrow AB|b|EA\}$;
  \item
    $R =\{S\rightarrow AS|SB|SS,B\rightarrow CA|b, C\rightarrow AA|a|BA,A\rightarrow \varepsilon|BS\}$;
  \item
    $R = \{S\rightarrow AB|AC,B\rightarrow \varepsilon |BC|b,A\rightarrow BB|CC|a,C\rightarrow CS|a\}$;
  \item
    $R = \{S\rightarrow AS|SB|SS,B\rightarrow AC|b, C\rightarrow A|a|AB,A\rightarrow \varepsilon|BS\}$;
  \item
    $R = \{S\rightarrow BA|CA,B\rightarrow \varepsilon |BC|b,A\rightarrow BB|CC|a, C\rightarrow CS|a\}$;
  \item
    $R = \{S\rightarrow AS|b,A\rightarrow AC|BC|a, B\rightarrow BC|CC,C\rightarrow \varepsilon|CA|a\}$;
  \item
    $R = \{S\rightarrow \varepsilon|BA|AS,A\rightarrow SB|a,B\rightarrow SS|SC|b,
    C\rightarrow CC|a\}$; 
  \end{enumerate}
\end{problem}

\begin{problem}
  Нека е дадена граматиката  $G = \pair{\{S,A,B,C\}, \{a,b\}, S, R}$.
  Използвайте обща конструкция, за да премахнете "дългите" правила 
  (т.е. с дължина поне 2, които не са в н.ф. на Чомски) от $ G$ като при това получите к.св. граматика $G_1$ 
  с език $L(G)=L(G_1)$, където:
  \begin{enumerate}
  \item
    $R = \{S \rightarrow \varepsilon|ab|aAba, A\rightarrow aBCb, B\rightarrow bbb, C\rightarrow aC\vert aCaC\}\rangle$;
  \item
    $R = \{S \rightarrow \epsilon|ab|baAb, A\rightarrow BaBb,B\rightarrow b,C\rightarrow AbA\vert aCCa\}$;
  \item
    $R = \{A\rightarrow BSB|a,B\rightarrow ba|BC,C\rightarrow BaSA|a|b,S\rightarrow CC|b\}$;
  \item
    $R = \{A\rightarrow BAS,B\rightarrow CB,C\rightarrow ab|ABbS,S\rightarrow CC|b\}$;
  \end{enumerate}
\end{problem}


\begin{problem}
  Използвайте обща конструкция, за да премахнете преименуващите правила от граматиката $G$ като при това запазите езика,
  където $G = \pair{\{A,B,C,S\},\{a,b\}, S, R}$ и
  \begin{enumerate}
  \item
    $R = \{A\rightarrow B|S,B\rightarrow C|BC,C\rightarrow AB|a|b,S\rightarrow B|CC|b\}$;
  \item
    $R = \{A\rightarrow B,B\rightarrow S|C|BC,C\rightarrow a|AB,S\rightarrow C|CC|b\}$;
  \item
    $R = \{A\rightarrow B|CC|a,B\rightarrow S|AB,C\rightarrow SC|b,S\rightarrow A|CC|b\}$;
  \item
    $R = \{A\rightarrow BB|b,B\rightarrow S|SS|b,C\rightarrow B|a,S\rightarrow C|AB|a\}$;
  \item
    $R = \{S\rightarrow A|a,A\rightarrow B|C|b, B\rightarrow AB, C\rightarrow CC|a\}$;
  \item
    $R = \{S\rightarrow A|B, A\rightarrow a|C|AB, B\rightarrow b|C, C\rightarrow CS|a|b\}$;
  \end{enumerate}
\end{problem}

\begin{problem}
  Намерете контекстно-свободна граматика в нормална форма на Чомски за езиците от задача 6.
  
\end{problem}


\subsection{Проблемът за принадлежност}

\begin{problem}
  Нека е дадена граматиката $\Gamma=\pair{\{a,b\}, \{S,A,B,C\},S,R}$.
  Използвайте алгоритъма за динамично програмиране (CYK), за да проверите дали
  думата $\alpha$ принадлежи на $L(G)$, където правилата на граматиката $R$ и думата $\alpha$
  са зададени като:
  \begin{enumerate}
  \item
    $R =\{S\rightarrow a| AB|AC, C\rightarrow SB|AS,A\rightarrow a, B\rightarrow b\}$, $\alpha=aaabb$;
  \item
    $R = \{S\rightarrow BA| CA|a, C\rightarrow BS|SA,A\rightarrow a, B\rightarrow b\}$, $\alpha=bbaaa$;
  \item
    $R =\{S\rightarrow AB|BC, A\rightarrow BA|a,B\rightarrow CC|b, C\rightarrow AB|a\}$, $\alpha=baaba$;
  \item
    $R = \{S\rightarrow AB, A\rightarrow AC|a|b,B\rightarrow CB|a, C\rightarrow a\}$, $\alpha=babaa$;
  \item
    $R = \{S\rightarrow BA|SS|b, A\rightarrow SA|a,B\rightarrow BS|b\}$, $\alpha = bbbaa$;
  \item
    $R = \{S\rightarrow AB| BS|b, A\rightarrow SS|a,B\rightarrow BA|b\}$, $\alpha = babab$;
  \item
    $R = \{S\rightarrow BA| AS|a, A\rightarrow AB|a,B\rightarrow SS|b\}$, $\alpha = ababa$;
  \item
    $R = \{S\rightarrow AB|a, A\rightarrow BA|SS|a,B\rightarrow SS|b\}$, $\alpha = aabba$.
  \end{enumerate}
\end{problem}


\section{Недетерминирани стекови автомати}

\index{автомат!недетерминиран стеков}
\marginpar{На англ. Push-down automaton}
%Sipser p.102
\begin{dfn}[стр. 102 от \cite{sipser}]
  Недетерминиран краен стеков автомат е \[P = \PDA,\] където 
  \begin{itemize}
  \item
    $Q$ е крайно множество от състояния;
  \item  
    $\Sigma$ е крайна входна азбука;
  \item
    $\Gamma$ е крайна стекова азбука;
  \item
    $\# \in \Gamma$ е символ за дъно на стека;
  \item
    $s\in Q$ е начално състояние;
  \item
    $\Delta:Q\times\Sigma_\varepsilon\times\Gamma\rightarrow \Ps_{fin}(Q\times\Gamma^\star)$ 
    е {\em частична} функция на преходите;    
  \item
    $F\subseteq Q$ е множество от заключителни състояния.
  \end{itemize}
\end{dfn}

Нека $P$ е стеков автомат. Тогава
\begin{itemize}
\item
  $\Ls_F(P)$ е езика, който се разпознава от $P$ {\bfс финално състояние},
  \[\Ls_F(P) = \{w\mid (q_0,w,\#) \vdash^\star_P (q,\varepsilon,\alpha)\ \&\ q \in F\}.\]    
\item
  $\Ls_S(P)$ е езика, който се разпознава от $P$  {\bf с празен стек},
  \[\Ls_S(P) = \{w\mid (q_0,w,\#) \vdash^\star_P (q,\varepsilon,\varepsilon)\}.\]    
\end{itemize}

\begin{thm}
  Класът на езиците, които се разпознават от краен стеков автомат, съвпада с
  класа на контекстно-свободните езици.
\end{thm}

\begin{problem}
  Нека е дадена граматиката $G = \pair{\{S,A,B\},\{a,b\},S,R\}}$.
  Постройте стеков автомат $P = \PDA$, такъв че $L_S(P) = L(G)$, където правилата $R$ на граматиката $G$ са зададени като:
  \begin{enumerate} 
    % За едно тези двете да се даде пример как става 
  \item
    $R = \{S\rightarrow ASB\vert \varepsilon, A\rightarrow aAa\vert a, B\rightarrow bBb\vert b\}$;
  \item
    $R = \{S\rightarrow ASB\vert \varepsilon, A\rightarrow aA\vert a, B\rightarrow Bb\vert b\}$;
  \item
    $R =\{S\rightarrow SA|\varepsilon,A\rightarrow BSa|B, B\rightarrow b|BS|ab\}$;
  \item
    $R = \{S\rightarrow AS|\varepsilon,A\rightarrow SaBB|A, B\rightarrow b|BBbS|AA\}$;
  % \item
  %   $\Gamma=\langle\{S,A,B,C,D,E\},\{a,b\},S,$\\
  %   $\{S \rightarrow aD, D\rightarrow ab|ABBA|ADD,A\rightarrow DEB|a,B\rightarrow DDD|DC|b,C\rightarrow CCE|a, E\rightarrow ba|bEa\}\rangle$;
  % \item
  %   $\Gamma=\langle\{S,A,B,C,D,E\},\{a,b\},S,$\\
  %   $\{S\rightarrow D,D\rightarrow AD|b,A\rightarrow ACB|BC|a, B\rightarrow ABCA|CEC, C\rightarrow \varepsilon|CA|a, E\rightarrow ab|aEb\}\rangle$;
  % \item
  %   $\Gamma=\langle\{S,A,B,C,D,E\},\{a,b\},S,$\\
  %   $\{S \rightarrow aD, D\rightarrow \varepsilon|ABBA|ADD,A\rightarrow DEB|a,B\rightarrow DDD|DC|b,C\rightarrow CCE|a, E\rightarrow \varepsilon|bEa\}\rangle$;
  % \item
  %   $\Gamma=\langle\{S,A,B,C,D,E\},\{a,b\},S,$\\
  %   $\{S\rightarrow D,D\rightarrow AD|b,A\rightarrow ACB|BC|a, B\rightarrow ABCA|CEC, C\rightarrow \varepsilon|CA|a, E\rightarrow \varepsilon|aEb\}\rangle$;

  % \item
  %   $\Gamma=\langle\{S,A,B,C,D,E\},\{a,b\},S,$\\
  %   $\{S\rightarrow DD,D\rightarrow DDA|b,A\rightarrow CAB|a, B\rightarrow BCA|CCE, C\rightarrow \varepsilon|CA|a, E\rightarrow \varepsilon|EaE\}\rangle$;
  % \item
  %   $\Gamma=\langle\{S,A,B,C,D,E\},\{a,b\},S,$\\
  %   $\{S\rightarrow DD,D\rightarrow DA|b,A\rightarrow CAB|a, B\rightarrow BCA|CCE, C\rightarrow \varepsilon|CA|a, E\rightarrow \varepsilon|EaE\}\rangle$;
  \end{enumerate}
\end{problem}

\section{Въпроси}
Вярно ли е, че:
\begin{itemize}
\item
  \marginpar{Да} 
  ако $L$ е контекстно-свободен език, то езикът $L \cap \{a^{2n}b^{2k}\mid n,k\in\Nat\}$ е контекстно-свободен ?
\item
  \marginpar{Да}
  ако $L$ е безкраен контекстно-свободен език, то съществува безкрайна редица от регулярни езици $L_1,L_2,\dots$,
  за които $L = \bigcup_{i\in\Nat}L_i$ ?
\item
  \marginpar{Не}
  за всяка безкрайна редица от регулярни езици $L_1,L_2,\dots$, то 
  езикът $L = \bigcup_{i\in\Nat}L_i$ е контекстно-свободен ?
\item
  за всеки регулярен език $R$ и всеки контекстно-свободен език $L$, то $L \cap R$ е контекстно-свободен ?
\item
  за всеки регулярен език $R$ и всеки контекстно-свободен език $L$, то $L \cup R$ е контекстно-свободен ?
\item
  за всеки регулярен език $R$ и всеки контекстно-свободен език $L$, то $L \setminus R$ е контекстно-свободен ?
\item
  за всеки регулярен език $R$ и всеки контекстно-свободен език $L$, то $R \setminus L$ е контекстно-свободен ?
\item
  съществува регулярен език $R$ и контекстно-свободен език $L$, за които $L \cap R$ не е контекстно-свободен ?
\item
  съществува регулярен език $R$ и нерегулярен, но контекстно-свободен език $L$, за които $L \cap R$ е регулярен ?
\item
  за всеки два нерегулярни, но контекстно-свободни езика $L_1,L_2$, то $L_1\cup L_2$ е регулярен ?
\item
  съществуват два нерегулярни, но контекстно-свободни езика $L_1,L_2$, за които $L_1\setminus L_2$ е регулярен ?
\item
  съществуват два нерегулярни, но контекстно-свободни езика $L_1,L_2$, за които $L_1\cap L_2$ е регулярен ?
\item
  съществуват два нерегулярни, но контекстно-свободни езика $L_1,L_2$, за които $L_1\cup L_2$ е регулярен ?
\item
  съществува регулярен език $R$, който може да се представи като $R = L_1 \cup L_2$, където
  $L_1 \cap L_2 = \emptyset$, $L_1,L_2$ са нерегулярни, но контекстно-свободни ?
\item
  езикът $\{a,b\}^\star \setminus \{a^nb^n \mid n\in\Nat\}$ е регулярен ?
\item
  езикът $\{a,b\}^\star \setminus \{a^nb^{2k+1} \mid n,k\in\Nat\}$ е регулярен ?
\item
  езикът $\{a,b\}^\star \setminus \{a^nb^{k} \mid n > k\}$ е регулярен ?
\item
  езикът $\{a,b\}^\star \setminus \{a^nbba^{n} \mid n \in \Nat\}$ е регулярен ?
\item
  езикът $\{a,b\}^\star \setminus \{a^nb^n \mid n\in\Nat\}$ е контекстно-свободен ?
\item
  езикът $\{a,b,c\}^\star \setminus \{a^nb^mc^k \mid m < n\ \&\ m < k\}$ е контекстно-свободен ?
\end{itemize}

Нека е дадена контекстно-свободната граматика $G$ с правила \[S\rightarrow a\vert AB \vert AC, A \rightarrow a, B\rightarrow b, C\rightarrow SB.\]
Вярно ли е, че ако приложим CYK алгоритъма върху думата $\alpha$, където
\begin{itemize}
\item 
  $\alpha = aabb$, то $N[1,1] = \{S\}$.
\item 
  $\alpha = aabb$, то $N[3,3] = \{B\}$.
\item 
  $\alpha = aabb$, то $N[1,4] = \{\}$.
\item
  $\alpha = baab$, то $N[2,4] = \{\}$.
\item
  $\alpha = baab$, то $N[1,3] = \{\}$.
\end{itemize}



%%% Local Variables: 
%%% mode: latex
%%% TeX-master: "discrete-math"
%%% End: 

\chapter{Булева алгебра}

\begin{enumerate}[1)]%% ДА се напишат всичките от Манев, стр. 189
\item
  Комутативни свойства\\
  $xy = yx$, $x\vee y = y\vee x$, $x\oplus y = y\oplus x$;
\item
  Асоциативни свойства\\
  $(xy)z = x(yz)$, $(x\vee y)\vee z = x\vee (y\vee z)$, $(x\oplus y)\oplus z = x\oplus (y\oplus z)$;
\item
  $x\oplus y = x.\ov{y}\vee \ov{x}.y = (x\vee y).(\ov{x}\vee\ov{y})$;
\item
  Свойства на отрицанието\\
  $x\ov{x} = 0$, $x\vee\ov{x} = x\vee 1$, $x\oplus\ov{x} = 1$;
\item
  Закон за двойното отрицание\\
  $\ov{\ov{x}} = x$;
\item
  Свойства на константите\\
  $x.0 = 0$,$x.1 = x$, $x\vee 0 = x$, $x\vee 1 = 1$, $x\oplus 0 = x$, $x\oplus 1 = \ov{x}$;
\item
  Дистрибутивни свойства
  \begin{enumerate}[]
  \item
    $x(y\vee z) = xy \vee xz$,
  \item
    $xy \vee z = (x\vee z)(y\vee z)$,
  \item
    $(x\oplus y)z = xz \oplus yz$.
  \end{enumerate}
\item
  Идемпотентентни свойства\\
  $xx = x$, $x\vee x = x$.
\item
  Свойства на отрицанието\\
  $x\ov{x} = 0$, $x\vee\ov{x} = 1$, $x\oplus\ov{x} = 1$;
\item
  Закони на Де Морган\\
  $\ov{xy} = \ov{x}\vee\ov{y}$, $\ov{x\vee y} = \ov{x}.\ov{y}$;
\end{enumerate}

\begin{problem} %% Гаврилов, стр. 30
  Проверете еквивалентни ли са формулите $\varphi$ и $\psi$ като използвате еквивалентни преобразования на формулите.
  \begin{enumerate}[a)]
  \item
    $\varphi = (x\oplus y.z)\rightarrow (\overline{x}\rightarrow (y\rightarrow z))$,
    $\psi = x\rightarrow ((y\rightarrow z)\rightarrow x)$;
  \item
    $\varphi = (\overline{x}\vee \overline{y}.z)\rightarrow ((x\rightarrow y)\rightarrow (y\vee z)\rightarrow\overline{x})$,
    $\psi = (x\rightarrow y)\rightarrow(\overline{y}\rightarrow\overline{x})$;
  \item
    $\varphi = (x.\overline{y}\vee \overline{x}.z)\oplus ((y\rightarrow z)\rightarrow \overline{x}.y)$,
    $\psi = (x.(\overline{y}.\overline{z})\oplus y)\oplus z$;
  \item
    $\varphi = x\rightarrow ((\ov{x}.\ov{y}\rightarrow(\ov{x}.\ov{z}\rightarrow y))\rightarrow y).z$,
    $\psi = \ov{x.(y\rightarrow\ov{z})}$.
  \item
    $\varphi = \ov{((x\vee y) \rightarrow y.z)\vee (y\rightarrow x.z)} \vee (x\rightarrow (\ov{y}\rightarrow z))$,
    $\psi = (x\rightarrow y)\vee z$.
  \end{enumerate}
\end{problem}

\begin{problem}
  По метода на неопределените коефициенти, намерете полинома на Жегалкин на функцията 
  \begin{enumerate}[a)]
  \item
    $f(x,y) = x\vee y$;
  \item
    $f(x,y,z) = x\vee y \vee z$;
  \item
    $f(x,y,z) = x\rightarrow (y \rightarrow z)$;
  \item
    $f(x,y,z) = x(y\vee\overline{z})$.
  \end{enumerate}
\end{problem}

\begin{problem}
  Използвайки еквивалентности от вида $\overline{A} = A\oplus 1$ и $A\vee B = AB\oplus A\oplus B$, 
  намерете полинома на Жегалкин на 
  \begin{enumerate}[a)]
  \item
    $f(x,y) = x\rightarrow y$;
  \item
    $f(x,y,z) = (x\rightarrow (y\rightarrow z))$;
  \item
    $f(x,y,z) = ((x\rightarrow y)\rightarrow z)$;
  \item
    $f(x,y,z) = (x\rightarrow (y\rightarrow z)).((x\rightarrow y)\rightarrow z)$;
  \item
    $f(x,y,z,t) = (x\rightarrow y)\rightarrow (z\rightarrow xt)$;
  \item
    $f(x,y,z,t) = x\vee (y\rightarrow ((z\rightarrow y)\rightarrow t)$;
  \item
    $f(x,y,z,t) = (x\vee y\vee z)t \vee xyz$.
  \end{enumerate}
\end{problem}


\begin{problem} %% Гаврилов стр. 50
  С помощта на еквивалентни преобразования постройте ДНФ на булевите функции
  \begin{enumerate}[a)]
  \item
    $f(x,y,z) = (\ov{x}\vee\ov{y}\vee\ov{z}).(xy\vee z)$;
  \item
    $f(x,y,z) = (\overline{x}y\oplus z).(xz\rightarrow y)$;
  \item
    $f(x,y,z) = (x\vee y\overline{z}).(x\ov{y}\vee\ov{z}).(\ov{xy}\vee z)$;
  \item
    $f(x,y,z,t) = (x\vee y\ov{z}.\ov{t})((\ov{x}\vee t)\oplus yz)\vee \ov{y}.(z\vee \ov{x\ov{t}})$;
  \item
    $f(x,y,z,t) = (x\rightarrow y).(y\rightarrow \ov{z}).(z\rightarrow x\ov{t})$;
  \end{enumerate}
\end{problem}

\begin{problem}% Гаврилов, стр. 50, 2.12
  По дадена ДНФ на булевата функция $f$ постройте нейната СДНФ.
  \begin{enumerate}[1)]
  \item
    $f(x,y,z) = xy\vee\ov{z}$;
  \item
    $f(x,y,z) = \ov{x}.\ov{y} \vee y\ov{z} \vee z\ov{z}$;
  \item
    $f(x,y,z) = x\vee yz \vee \ov{x}.\ov{z}$;
  \item
    $f(x,y,z) = x\vee \ov{y}\vee \ov{x}z$;
  \item
    $f(x,y,z,t) = xy\ov{z} \vee xz\ov{t}$;
  \item
    $f(x,y,z,t) = xy \vee \ov{y}t \vee z\ov{t}$.
  \end{enumerate}
\end{problem}


\begin{problem}
  Представете в СДНФ следните булеви функции:
  \begin{enumerate}[1)]
  \item
    $f(x,y,z) = (x\vee y)\rightarrow z$;
  \item
    $f(x,y,z) = (01010001)$;
  \item
    $f(x,y,z) = (11001010)$;
  \item
    $f(x,y,z,t) = (x\rightarrow yzt)(z\rightarrow x\ov{y})$;
  \item
    $f(x,y,z,t) = (x\oplus y)(z\rightarrow \ov{y}t)$;
  \end{enumerate}
\end{problem}

Нека е дадена булевата функция $f(\xn)$. Дефинираме булевата функция $f^\star(\xn)$ като
\[f^\star(\xn) = \overline{f}(\overline{x_1},\dots,\overline{x_n}).\]
Ще наричаме $f^\star$ двойнствена функция на $f$.

\begin{problem} %% Гаврилов, стр. 31, зад. 1.25
  Проверете дали функцията $g$ е двойнствена на $f$.
  \begin{enumerate}[1)]
  \item
    $f = x\rightarrow y$, $g = \overline{x}.y$;
  \item
    $f = (\overline{x}\rightarrow\overline{y})\rightarrow(y\rightarrow x)$, $g = (x\rightarrow y).(\overline{y}\rightarrow\overline{x})$;
  \item
    $f = x.y \rightarrow z$, $g = \overline{x}.\overline{y}.z$;
  \item
    $f = (x\vee y\vee z).t\vee x.y.z$, $g = (x\vee y\vee z).t\vee x.y.z$;
  \item
    $f = xy\vee yz\vee zt\vee tx$, $g = xz\vee yt$;
  \item
    $f = (x\rightarrow y).(z\rightarrow t)$, $g = (x\rightarrow\overline{z}).(x\rightarrow t).(\overline{y}\rightarrow\overline{z}).(\overline{y}\rightarrow t)$.
  \end{enumerate}
\end{problem}

\begin{problem}
  Проверете самодвойнствена ли е $f$.
  \begin{enumerate}[1)]
  \item
    $f(x,y) = x\vee y$;
  \item
    $f(x,y) = x\rightarrow y$;
  \item
    $f(x,y) = x\oplus y$;
  \item
    $f_4(x,y,z) = xy\vee yz\vee zx$;
  \item
    $f_5(x,y,z) = x\oplus y\oplus z\oplus 1$;
  \item
    $f_6(x,y,z) = xyz\oplus xy\ov{z}\oplus yz\oplus xz$.
  \item
    $f_7(x,y,z) = xyz\oplus xy\oplus yz\oplus xz$;
  \item
    $f(x,y,z) = (x\rightarrow y)\oplus (y\rightarrow z)\oplus (y\rightarrow x)$;
  \item
    $f(x,y,z) = (x\rightarrow y)\oplus (y\rightarrow z)\oplus (z\rightarrow x)\oplus z$;
  \end{enumerate}
\end{problem}

\begin{problem}
  Проверете дали функцията $f$ е самодвойнствена, ако е зададена векторно:
  \begin{enumerate}[1)]
  \item
    $\alpha_f = (01001101)$;
  \item
    $\alpha_f = (01100110)$;
  \item
    $\alpha_f = (1100 1001 0110 1100)$;
  \item
    $\alpha_f = (1110 0111 0001 1000)$;
  \item
    $\alpha_f = (1100 0011 0011 1100)$;
  \item
    $\alpha_f = (1001 0110 1001 0110)$;
  \item
    $\alpha_f = (1100 0011 1010 0101)$;
  \end{enumerate}
\end{problem}


%%% Local Variables: 
%%% mode: latex
%%% TeX-master: "discrete-math"
%%% End: 

\begin{problem}
  Заменете $-$ в $\chi_f$ с $0$ или $1$ за да получите характеристичен вектор на самодвойнствена функция.\\
  \begin{inparaenum}[a)]
  \item
    $\chi_f = (1-0-)$;
  \item
    $\chi_f = (01-0-0--)$;
  \item
    $\chi_f = (--01--11)$;
  \end{inparaenum}
\end{problem}

\subsection{Линейни функции}

Всяка булева функция $f(\xn)$ с полином на Жегалкин от вида 
$a_0\oplus a_1x_1 \oplus a_2x_2 \dots\oplus a_nx_n$ наричаме {\em линейна}.\index{линейна!булева функция}
Ще означаваме с $L$ множеството от всички линейни булеви функции, а с $L^n$ тези на $n$ променливи.

\begin{problem}
  Линейна ли е функцията $f$ с характеристичен вектор $\chi_f = (1001011010010110)$ ?
\end{problem}

\begin{problem}
  Заменете $-$ в $\chi_f = (-110---0)$ с $0$ или $1$, така че да получите $f$ линейна.
\end{problem}


\begin{problem}
  Проверете дали $f$ е линейна функция.\\
  \begin{inparaenum}[a)]
  \item
    $f = x\rightarrow y$;
  \item
    $f = \ov{x\rightarrow y}\oplus \ov{x}y$;
  \item
    $f = xy\vee \ov{x}.\ov{y}\vee z$;
  \item
    $f = xy\ov{z}\vee x\ov{y}$;
  \item
    $f = (x\vee yz)\oplus xyz$;
  \item
    $f = (x\vee yz)\oplus \ov{x}yz$;
  \item
    $\chi_f = (1100 0011)$;
  \item
    $\chi_f = (1001 0110 0110 1001)$;
  \end{inparaenum}
\end{problem}

\begin{problem}
  Заменете $-$ в $\chi_f$ с $0$ или $1$, така че да получите $f$ линейна.\\  
  \begin{inparaenum}[a)]
  \item
    $\chi_f = (10-1)$;
  \item
    $\chi_f = (100-0---)$;
  \item
    $\chi_f = (-001--1-)$;
  \item
    $\chi_f = (11-0---1)$;
  \item
    $\chi_f = (-0-1--00)$;
  \item
    $\chi_f = (--10----0--1-110)$;
  \end{inparaenum}
\end{problem}


\subsection{Монотонни функции}

Нека $\alpha$ и $\beta$ са два бинарни вектора с равна дължина.
Дефинираме релацията $\preceq$ между тях по следния начин.
\[\alpha \preceq \beta \iff \abs{\alpha} = \abs{\beta}\wedge (\forall i \leq \abs{\alpha})[a_i \leq b_i].\]
Булевата фунция $f(\xn)$ наричаме {\em монотонна}\index{монотонна!функция}, ако 
\[(\forall \alpha,\beta\in J^n_2 )[\alpha\preceq\beta \rightarrow f(\alpha) \leq f(\beta)].\]
Ще означаваме с $M$ множеството от всички монотонни булеви функции, а с $M^n$ тези на $n$ променливи.

\begin{problem}
  Проверете монотонни ли са функциите:\\
  \begin{inparaenum}[a)]
  \item
    $f = x\rightarrow (y\rightarrow x)$;
  \item
    $f = x\rightarrow (x\rightarrow y)$;
  \item
    $f = (x\oplus y)xy$;
  \item
    $f = xy\oplus yz \oplus zx$;
  \item
    $f = xy\oplus yz \oplus zx \oplus x$;
  \end{inparaenum}
\end{problem}

\begin{problem}
  За немонотонните функции $f$, намерете съседни $\alpha$, $\beta$, такива че
  $\alpha \prec \beta$ и $f(\alpha) > f(\beta)$.\\
  \begin{inparaenum}[a)]
  \item
    $f = xyz \vee \ov{x}y$;
  \item
    $f = x\oplus y\oplus z$;
  \item
    $f = xy\oplus z$;
  \item
    $f = x\vee y\ov{z}$;
  \item
    $f = xz\oplus yt$;
  \item
    $f(x,y,z,t) = (xyt\rightarrow yz)\oplus t$;
  \end{inparaenum}
\end{problem}
\begin{proof}
  \begin{enumerate}[a)]
  \item
    $\alpha = (010)$, $\beta = (110)$;
  \item
    $\alpha = (010)$, $\beta = (110)$;
  \item
    $\alpha = (110)$, $\beta = (111)$;
  \item
    $\alpha = (010)$, $\beta = (011)$;
  \item
    $\alpha = (0111)$, $\beta = (1111)$;
  \item
    $\alpha = (1110)$, $\beta = (1111)$;
  \end{enumerate}
\end{proof}



\subsection{Пълнота и затворени класове}

\begin{thm}[Критерий за пълнота на Пост-Яблонский]
  Нека $P\subseteq F_2$. Множеството $P$ е пълно тогава и само тогава, когато то {\em не} е подмножество на 
  нито едно от множествата $T_0,T_1,S,M,L$.
\end{thm}


\begin{problem} %Гаврилов, стр. 83, зад. 6.1
  Пълна ли е системата от функции?
  \begin{enumerate}[1)]
  \item
    $A = \{xy, x\vee y, x\oplus y\oplus z\oplus 1\}$;
  \item
    $A = \{1, xy(x\oplus z)\}$;
  \item
    $A = \{x\rightarrow y, x\oplus y\}$;
  \item
    $A = \{0, \ov{x}, x(y\oplus z)\oplus yz\}$;
  \item
    $A = \{x\rightarrow y, \ov{x}\rightarrow \ov{y}x, x\oplus y\oplus z, 1\}$;
  \item
    $A = \{\chi_{f_1} = (0110), \chi_{f_2} = (1100 0011), \chi_{f_3} = (1001 0110)\}$;
  \item
    $A = \{\chi_{f_1} = (11), \chi_{f_2} = (00), \chi_{f_3} = (0011 0101)\}$;
  \end{enumerate}
\end{problem}


\begin{problem} % Гаврилов, стр. 84
  Проверете пълно ли е множеството от булеви функции:
  \begin{enumerate}[a)]
  \item
    $A = (S\cap M)\cup(L\setminus M)$;
  \item
    $A = ((L\cap M)\setminus T_1)\cup (S\cap T_1)$.
  \item
    $A = (L\cap M)\cup (S\setminus T_0)$;
  \item
    $A = (L\cap T_1)\cup (S\cap M)$;
  \item
    $A = (M\setminus S)\cup(L\cap S)$;
  \item
    $A = (M\setminus T_0)\cup (L\setminus S)$;
  \end{enumerate}
\end{problem}

\begin{problem}
  Проверете дали системата от функции $A$ е базис?
  \begin{enumerate}[a)]
  \item
    $A = \{x\rightarrow y, x\oplus y, x\vee y\}$;
  \item
    $A = \{x\oplus y\oplus z, x\vee y, 0, 1\}$;
  \item
    $A = \{xy\oplus yz\oplus zx, 0, 1, x\vee y\}$;
  \end{enumerate}
\end{problem}


% \chapter{Теория на Графите}

\section{Неориентирани графи}

\begin{dfn}
  Неориентиран граф\index{неориентиран!граф} $G$ е наредена тройка $(V,E,\psi_G)$, където
  $V$ е непразно множество, $V,E$ са непресичащи се множества, и $\psi_G$ асоциира с всеки елемент $e\in E$
  ненаредена двойка от елементи на $V$.
  Елементите на $V$ наричаме върхове, а елементите на $E$ ребра.
\end{dfn}

Нека да въведем някои означения.
Под прост {\bf граф}\index{прост!граф} $G$ ще разбираме неориентиран граф без примки и повтарящи се ребра.
С $\delta(G)$\index{$\delta(G)$} ще означаваме минималната степен в графа $G$, а с $\Delta(G)$\index{$\Delta(G)$} - максималната.
Означаваме $\nu(G) = |V|, \epsilon(G) = |E|$.
Броят на свързаните компоненти на $G$ означаваме с $\omega(G)$.

\begin{problem}
  Докажете следното неравенство:
  \[\delta \leq \frac{2\varepsilon}{\nu} \leq \Delta.\]
\end{problem}
\begin{proof}
  \[\nu\delta \leq\sum_{1\leq i \leq\nu} deg_G(v_i) \leq \nu\Delta.\]
\end{proof}


\section{Дървета}

\begin{dfn}
  Дърво е свързан граф без цикли.
\end{dfn}

\begin{thm}
  В дърво, между всеки два върха има единствен път.
\end{thm}

\begin{thm}
  Ако $G$ е дърво, то $\varepsilon(G) = \nu(G) - 1$.
\end{thm}
\begin{proof}
  С индукция по $\nu$. За $\nu = 1$ е ясно.
  Да допуснем за $G$ с брой ребра $<\nu$ и ще докажем за $G$.
  Нека $uv\in E$ и $G'$ се получава като изтрием това ребро.
  Получаваме два свързани ациклични графа $G_1, G_2$.
  Следователно те са дървета и от и.п. 
  \[\varepsilon(G_1) = \nu(G_1) - 1\ \&\ \varepsilon(G_2) = \nu(G_2) - 1.\]
  Получаваме, че 
  \[\varepsilon(G - uv) = \varepsilon(G_1) + \varepsilon(G_2) = \nu(G_1) + \nu(G_2) - 2.\]
  Накрая, $\varepsilon(G) = \varepsilon(G-uv) + 1 = \nu(G_1) + \nu(G_2) - 2 + 1= \nu(G) - 1$.
\end{proof}

\begin{crl}
  Всяко нетривиално дърво има поне два върха със степен 1.
\end{crl}
\begin{proof}
  Ясно е, че $(\forall v\in V)[d(v) \geq 1]$.
  Знаем, че \[\sum_{v\in V}d(v) = 2\varepsilon = 2\nu - 2,\] от където следва, че има поне два върха със степен 1.
\end{proof}

\subsection{Покриващи дървета}

Означаваме с $\tau(G)$ броят на покриващите дървета на $G$ (не неизоморфните, а всички).

Нека $u\neq v, e = (u,v)\in E$. С $G.e = (E',V')$ означаваме графа получен от $G$, като премахваме реброто $e$ и 
съединяваме краищата $u,v$. Ясно е, че $\nu(G.e) = \nu(G) - 1, \varepsilon(G.e) = \varepsilon(G) - 1, \omega(G.e) = \omega(G)$.

\begin{thm}
  Ако $e\in E$ не е примка, то
  $\tau(G) = \tau(G-e) + \tau(G.e)$.
\end{thm}
\begin{proof}
  $\tau(G) = n + m$, където $n$ е броят на покриващите дървета на $G$, в които не участва $e$
  и $m$ е броят на покриващите дървета, в които участва $e$.
  Ясно е, че $n = \tau(G-e)$.
  На всяко покриващо дърво $T$, в което участва $e$, съответства покриващо дърво $T.e$ на $G.e$.
  Това съответствие е взаимно-еднозначно, следователно $m = \tau(G.e)$.
\end{proof}

\begin{problem}
  Колко на брой са всички изоморфни и неизоморфни покриващи дървета на $K_3,K_4$ ?
\end{problem}


\begin{problem}
  Покажете, че ако $G$ е дърво и има връх със степен $\geq k$, то $G$ има поне $k$ върха със степен 1.
\end{problem}

\begin{problem}
  Нека $G$ е свързан граф.
  Докажете, че всеки два най-дълги пътя в свързан граф имат общ връх.
\end{problem}

\begin{problem}
  За $G$ прост граф, докажете, че $\varepsilon = \binom{\nu}{2}$ т.с.т.к. $G$ е пълен.
\end{problem}

\begin{problem}
  Докажете, че в граф с $\nu\geq 2$, има поне два върха с еднаква степен.
\end{problem}

\begin{problem}
  Докажете, че:
  \begin{enumerate}
  \item
    във всеки неориентиран граф броят на върховете с нечетна степен е четен;
  \item
    всеки регулярен граф с нечетна степен има четен брой върхове;
  \item
    всеки граф с $\varepsilon > \binom{\nu-1}{2}$ е свързан.
    Дайде пример за несвързан граф с $\varepsilon = \binom{\nu-1}{2}$.
  \item
    във граф всички върхове имат степен поне $d$.
    Докажете, че в графа има път с дължина $d$.
  \end{enumerate}
\end{problem}


\begin{problem} % зад. 1.22
  Да разгледаме графа $G$ (без примки и без кратни ребра) със $s$ компоненти на свързаност.
  Докажете, че $\nu - s \leq \varepsilon \leq \binom{\nu-s+1}{2}$.
\end{problem}

\begin{problem}
  Нега $G$ е граф с $n$ върха и в $G$ няма прост цикъл с дължина 3.
  Докажете, че $G$ има най-много $\lfloor{\frac{n^2}{4}}\rfloor$ ребра.
\end{problem}

\begin{problem}
  Нека $G$ е произволен граф без примки и кратни ребра, а $\overline{G}$ е неговото допълнение.
  Докажете, че поне един от графите $G$, $\overline{G}$ е свързан;
\end{problem}


\begin{problem}
  \begin{enumerate}
  \item
    Да се построят всички неизоморфни графи на 1,2,3 и 4 върха.
  \item
    Намерете броя на ребрата на граф без цикли с $n$ върха и $k$ компоненти.
  \end{enumerate}
\end{problem}

\section{Ориентирани графи}

%%% Local Variables: 
%%% mode: latex
%%% TeX-master: "discrete-math"
%%% End: 

% \chapter {Алгоритми за графи}

\section{Обхождане на граф}

Нека е даден ориентирания граф $G = (V,E)$.
При обхождането на граф, всеки връх може да бъде в едно от три състояния, или както сме ги означили тук, в един от три цвята.
Ако един връх $v$ е
\begin{itemize}
\item 
  бял, то той още не е срещнат.
\item
  сив, то той е срещнат, но още не е напълно обработен.
\item
  черен, то той е напълно обработен.
\end{itemize}


\subsection{Обхождане в широчина}


\begin{algorithm}
  \caption{Инициализация}
  \label{alg:bfs-init}
  \begin{algorithmic}[1]
    \Procedure{BFS-INIT}{$G$,$r$}
    \ForAll{$v \in V \setminus \{r\}$}
    \State $\texttt{color}[v] := \texttt{WHITE}$
    \State $\texttt{dist}[v] := \infty$
    \State $\texttt{pred}[v] := \texttt{NIL}$
    \EndFor
    \State $\texttt{color}[r]:= \texttt{GRAY}$
    \State $\texttt{dist}[r]:= 0$
    \State $\texttt{pred}[r]:= \texttt{NIL}$
    \EndProcedure
  \end{algorithmic}
\end{algorithm}

За този алгоритъм най-удобно е да имаме масив $\texttt{Adj}$ с дължина $\abs{V}$,
като $\texttt{Adj}[u]$ дава списък с преките наследници на $u$, т.е.
\[\texttt{Adj}[u] = \{v \in V \mid (u,v) \in E\}.\]

\begin{algorithm}[H]
  \caption{Обхождане на граф в широчина}
  \label{alg:bfs}
  \begin{algorithmic}[1]
    \Procedure{BFS}{$G$,$r$}
    \State \Call{BFS-INIT}{$G$,$r$}
    \State $Q := \emptyset$\Comment{Опашката $Q$ съдържа точно сивите върхове}
    \State put($Q$,$r$)
    \While {$Q \neq \emptyset$}
    \State $u := get(Q)$\Comment{$u$ е премахнат от опашката}
    \ForAll{$v \in \texttt{Adj}[u]$}
    \If{$\texttt{WHITE} = \texttt{color}[v]$}
    \State $put(Q,v)$
    \State $\texttt{color}[v] := \texttt{GRAY}$
    \State $\texttt{pred}[v] := u$
    \State $\texttt{dist}[v] := \texttt{dist}[u] + 1$
    \EndIf
    \EndFor
    \State $\texttt{color}[u] := \texttt{BLACK}$
    \EndWhile
    \EndProcedure
  \end{algorithmic}
\end{algorithm}

\begin{itemize}
\item 
  Алгоритъмът работи както за ориентирани, така и за неориентирани графи.
\item
  {\bf Дължина на път} (без цикли) е броят на ребрата, които участват в пътя.
  Например, за пътя $p = (v_0,\dots,v_k)$ в графа $G$,
  неговата дължина е $k$, защото ребрата, които участват в $p$
  са $\{(v_0,v_1),(v_1,v_2),\dots,(v_{k-1},v_k)\}$ и са общо $k$ на брой.
  Обикновено ще означаваме дължината на пътя $p$ с $\abs{p}$ и пишем $v_0 \stackrel{p}{\leadsto} v_k$.
\item
  Нека да означим за всеки два върха $u,v \in V$,
  \begin{align*}
    \delta(u,v) =
    \begin{cases}
      \min\{\abs{p} \mid u\stackrel{p}{\leadsto}v\}, & \text{ ако има път между }u, v \\
      \infty, & \text{ ако няма път}
    \end{cases}
  \end{align*}
\item
  Имаме свойството, че за граф $G = (V,E)$ и един връх $s \in V$,
  ако $(u,v) \in E$, то
  \[\delta(s,v) \leq \delta(s,u) + 1.\]
\item
  За граф $G = (V,E)$, и фиксиран връх $r\in V$, означаваме
  \[G_{pred} = (V_{pred},E_{pred}),\]
  където
  \begin{align*}
    V_{pred} & = \{v \in V\mid \texttt{pred}[v] \neq \texttt{NIL}\} \cup \{s\},\\
    E_{pred} & = \{(u,v) \in E \mid \texttt{pred}[v] = u\}.
  \end{align*}
  След изпълнение на BFS($G$,$r$), $G_{pred}$ представлява дърво с корен $r$, 
  за всеки достижим в $G$ от $r$ връх $v$, $G_{pred}$ съдържа единствен прост път $r \stackrel{p}{\leadsto} v$, 
  като $p$ е най-къс измежду всички пътища свързващи $r$ с $v$ в $G$.
\end{itemize}

\begin{thm}
  Нека е даден граф $G = (V,E)$ и един връх $r \in V$.
  След изпълнение на BFS($G$,$r$) получаваме, че
  \[(\forall v \in V)[\texttt{dist}[v] = \delta(r,v)].\]
\end{thm}

\subsection{Обхождане в дълбочина}

\begin{algorithm}[H]
  \caption{Обхождане в дълбочина}
  \label{alg:dfs-visit}
  \begin{algorithmic}[1]
    \Procedure{DFS-VISIT}{$G$,$u$}
    % \State $t := t+1$
    % \State $in[u] := t$
    \State $\texttt{color}[u] := \texttt{GRAY}$\Comment{Върхът $u$ е посетен, но не е обработен}
    \ForAll{$v \in \texttt{Adj}[u]$}
    \If {$\texttt{WHITE} = \texttt{color}[v]$}
    \State $\texttt{pred}[v] := u$
    \State \Call{DFS-VISIT}{$v$}
    \EndIf
    \EndFor
    % \State $t := t+1$
    % \State $out[u] := t$
    \State $\texttt{color}[u] := \texttt{BLACK}$\Comment{Приключили сме с $u$}
    \EndProcedure
    \Statex
    \Procedure{DFS}{$G$}
    \ForAll{$v \in V$}\Comment{Инициализация}
    \State $\texttt{color}[v] := \texttt{WHITE}$
    \State $\texttt{pred}[v] := \texttt{NIL}$
    \EndFor    
    % \State $t := 0$
    \ForAll{$v \in V$}
    \If{$\texttt{WHITE} = \texttt{color}[v]$}
    \State\Call{DFS-VISIT}{$G$,$v$}
    \EndIf
    \EndFor
    \EndProcedure
  \end{algorithmic}
\end{algorithm}


\section{Минимално покриващо дърво на граф}


\begin{itemize}
\item
  Тук ще разглеждаме само {\bf неориентирани} графи $G = (V,E,w)$ с тегла по ребрата
  зададени с функцията $w:E\to\R$.
\item
  Един граф $G = (V,E)$ се нарича {\bf свързан}, ако има път между всеки два $v,v^\prime \in V$.
\item 
  Един неориентиран граф $G$ се нарича {\bf дърво}, ако $G$ е свързан и без цикли.
\item
  {\bf Покриващо дърво} за свъзан неориентиран граф $G = (V,E)$,
  е дърво $T = (V,E^\prime)$, $E^\prime \subseteq E$.
\item
  Тегло на едно подмножество от ребра $U \subseteq E$ е числото
  \[w(U) = \sum_{e \in U} w(e).\]
\item
  {\bf Минимално покриващо дърво} за свъзан неориентиран претеглен граф $G = (V,E,w)$
  е покриващо дърво $T$, за което 
  \[w(T) = \min\{w(T^\prime) \mid T^\prime\mbox{ е покриващо дърво за }G\}.\]
\end{itemize}

\subsection{Алгоритъм на Прим}

Нека е даден неориентиран претеглен {\bf свързан} граф $G = (V,E,w)$.

\begin{algorithm}[H]
  \caption{Намиране на покриващо дърво (Прим)}

  \begin{algorithmic}[1]
    \Procedure{PRIM}{$G,r$}
    \State $U := \{r\}$\Comment{Започваме от дърво с корен $r$ и без ребра}
    \State $S := \emptyset$
    \While{$(\exists (x,y)\in E)[x\in U\ \&\ y \in V\setminus U]$}
    \State Избираме $(u,v) \in E$, за което
    \State $w(u,v) = \min\{w(x,y) \mid x\in U\ \&\ y \in V\setminus U\ \&\ (x,y) \in E\}$
    \State $U := U\cup\{v\}$
    \State $S := S \cup\{(u,v)\}$
    \EndWhile
    \State \textbf{return} $(U,S)$\Comment{Връщаме като резултат полученото дърво}
    \EndProcedure
  \end{algorithmic}
\end{algorithm}

% \begin{enumerate}
% \item 
%   Започваме от дървото $T_0 = (\{r\},\emptyset)$.
% \item
%   Нека сме построили дървото $T_i = (V_i,E_i)$.
%   Избираме ребро $(v,v^\prime) \in E$ такова ,че
%   \[w(v,v^\prime) = \min\{w(x,y) \mid x\in V_i\ \&\ y \in V\setminus V_i\ \&\ (x,y) \in E\}.\]
%   Образуваме \[T_{i+1} = (V_i\cup\{v^\prime\}, E_i \cup \{(v,v^\prime)\}).\]
% \item
%   Алгоритъмът завършва когато $V_i = V$.
% \end{enumerate}

\subsection{Алгоритъм на Крускал}

Нека е даден неориентиран {\bf свързан} претеглен граф $G = (V,E,w)$.

\begin{algorithm}[h!]
  \caption{Намиране на покриващо дърво (Крускал)}
  
  \begin{algorithmic}[1]
    \Procedure{KRUSKAL}{$G$}
    \State $X = \emptyset$\Comment{$X$ ще бъде колекция от дървета}
    \For{$v \in V$}
    \State Добавяме дървото $T = (\{v\},\emptyset)$ към колекцията $X$
    \EndFor
    \State$E^\prime := $\Call{Sort}{$E$,$w$}\Comment{Сортираме $E$ във възходящ ред относно тегла им}
    \Statex
    \ForAll{$(u,v) \in E^\prime$}
    \State Нека $u$ е връх в дървото $T_u \in X$
    \State Нека $v$ е връх в дървото $T_v \in X$
    \If{$T_u \neq T_v$}
    \State $W := V_u\cup V_v$
    \State $R := E_u\cup E_v\cup\{(u,v)\}$
    \State $T := (W,R)$
    \State Премахваме $T_u$ и $T_v$ от колекцията $X$
    \State Добавяме дървото $T$ към $X$
    \EndIf
    \EndFor
    \State \Return единственото дърво останало в $X$
    \EndProcedure
  \end{algorithmic}
\end{algorithm}

\section{Минимални пътища от даден връх}

\begin{itemize}
\item
  С $u \stackrel{p}{\leadsto} v$ означаваме, че $p$ е път от $u$ до $v$.
\item
  Тук ще разглеждаме {\bf ориентирани} графи $G = (V,E)$, като имаме и 
  функция $w: E\to \R$, която задава {\bf тегла} на ребрата на графа.
\item 
  {\bf Цена на път} $p = (v_0,\dots,v_k)$ в графа означаваме 
  \[w(p) = \sum_{i<k} w(v_i,v_{i+1}).\]
\item
  За всеки два върха $u,v \in V$, означаваме
  \begin{align*}
    \delta(u,v) = 
    \begin{cases}
      \min\{w(p)\mid u \stackrel{p}{\leadsto} v\}, \mbox{ ако има път от }u\mbox{ до }v\\
      \infty, \mbox{ иначе }
    \end{cases}
  \end{align*}
\item
  {\bf Минимален път} $p$ от $u$ до $v$ е такъв път, за който $w(p) = \delta(u,v)$.
\item
  Имаме следното важно свойство.
  Нека $u \stackrel{p}{\leadsto} v$ и $p$ е {\bf минимален път}.
  Да означим $p = (v_0,\dots,v_k)$ и $p_{ij} = (v_i,\dots,v_j)$ за $0\leq i \leq j \leq k$.
  Тогава за всеки $0\leq i \leq j \leq k$, 
  $p_{ij}$ е {\bf минимален път} от $v_i$ до $v_j$.
\item
  {\bf Цикъл} е път $p = (v_0,\dots,v_k)$, където $v_0 = v_k$.
\item
  Също така казваме, че по пътя $p = (v_0,\dots,v_k)$ има цикъл, ако
  за някои $0 \leq i < j \leq k$ имаме, че $v_i = v_j$.
\item
  Ако има цикъл с отрицателно тегло по някой път от $u$ до $v$, то 
  тогава пишем, че $\delta(u,v) = -\infty$.
\item
  Нека $u \stackrel{p}{\leadsto} v$ и $p$ е с минимално тегло.
  Тогава няма цикъл с положително тегло по $p$.
\item
  Нека $u \stackrel{p}{\leadsto} v$ и $p$ е с минимално тегло.
  Можем без ограничение на общността да приемем, че няма цикли с нулево тегло
  по $p$.
\item
  Важно свойство е, че броят на върховете по всички минални пътища е $\leq \abs{V}$.
\end{itemize}

Нека да фиксираме един връх $s \in V$.
Целта ни е да намерим минимални пътища от $s$ до всички достижими от $s$ върхове,
както и тяхната цена. 
Да отбележим, че ако имаме отрицателен по някой път $s \leadsto v$, то задачата не е добре 
дефинирана, защото тогава $\delta(s,v) = -\infty$.

За тази цел въвеждаме два масива, $\texttt{dist}$ и $\texttt{pred}$, с дължина $\abs{V}$.
\begin{itemize}
\item 
  $\texttt{dist}[v]$ - дава цена на минимален път от $s$ до $v$.
  Ако $\texttt{dist}[v] = \infty$, то не е намерен път $s\leadsto v$.
\item
  $\texttt{pred}[v]$ - дава предшественика на $v$ по този минимален път, т.е.
  ако $\texttt{pred}[v] = u$, то $s \leadsto u \to v$.
  Ако $\texttt{pred}[v] = \texttt{NIL}$, то не е намерен път $s \leadsto v$.
\end{itemize}

\begin{algorithm}[h!]
  \caption{Инициализация}
  \label{alg:init}
  \begin{algorithmic}[1]
    \Procedure{INIT}{$s$}
    \ForAll{$v \in V$}
    \State $\texttt{dist}[v] := \infty$
    \State $\texttt{pred}[v] := \texttt{NIL}$
    \EndFor
    \State $\texttt{dist}[s] := 0$
    \EndProcedure
  \end{algorithmic}
\end{algorithm}

\begin{algorithm}[h!]
  \caption{Търсене на по-добър кандидат}
  \label{alg:update}
  \begin{algorithmic}[1]
    \Procedure{UPDATE}{$u$,$v$}
    \If{$\texttt{dist}[v] > \texttt{dist}[u] + w(u,v)$}
    \State $\texttt{dist}[v] := \texttt{dist}[u] + w(u,v)$
    \State $\texttt{pred}[v] := u$
    \EndIf
    \EndProcedure
  \end{algorithmic}
\end{algorithm}

\subsection{Основни свойства}
  
\begin{prop}[Неравенство на триъгълника]
  \label{prop:triangle}
  За всяко $(u,v) \in E$,
  \[\delta(s,v) \leq \delta(s,u) + w(u,v).\]
\end{prop}

\begin{prop}
  \label{prop:upper-bound}
  Нека сме изпълнили INIT(s).
  Тогава имаме свойството \[(\forall v\in V)[\texttt{dist}[v] \geq \delta(s,v)].\]
  То се запазва и след прозволен брой изпълнения на UPDATE върху ребра на графа.
  Освен това, ако веднъж $\texttt{dist}[v] = \delta(s,v)$, то стойността на $\texttt{dist}[v]$
  повече не се променя.
\end{prop}
\begin{proof}
  Индукция по броя $i$ на изпълнения на UPDATE.
  За $i = 0$ е очевидно.
  Ще докажем твърдението за $i > 0$ изпълнения на UPDATE.
  Нека $\texttt{dist}[v] > \texttt{dist}[u] + w(u,v)$ и изпълним UPDATE(u,v).
  Тогава като използваме индукционното предположение и неравенството на триъгълника,
  \begin{align*}
    \texttt{dist}[v] & = \texttt{dist}[u] + w(u,v)\\
    & \geq \delta(s,u) + w(u,v)\\
    & \geq  \delta(s,v).
  \end{align*}

  Ясно е, че веднъж достигнали $\texttt{dist}[v] = \delta(s,v)$, $\texttt{dist}[v]$
  не може да се промени, защото тази стойност може само да намалява, а ние
  сме достигнали нейния минумум.
\end{proof}

\begin{prop}
  \label{prop:no-path}
  Нека сме изпълнили INIT($s$) и нека няма път от $s$ до $v$.
  Тогава имаме свойството
  \[\texttt{dist}[v] = \delta(s,v) = \infty.\]
  То се запазва и след прозволен брой изпълнения на UPDATE върху ребра на графа.
\end{prop}
\begin{proof}
  Щом няма път от $s$ до $v$, то $\delta(s,v) = \infty$.
  От Твърдение \ref{prop:upper-bound}, $\texttt{dist}[v] \geq \delta(s,v) = \infty$.
  Следователно, $\texttt{dist}[v] = \infty$.
\end{proof}

\begin{prop}
  \label{prop:converge}
  Нека $s\leadsto u \to v$ е път с минимално тегло.
  Нека сме изпълнили INIT(s) и няколко пъти $\texttt{UPDATE}$, като измежду тях и UPDATE($u$,$v$).
  Ако преди изпълнението на UPDATE($u$,$v$) 
  имаме, че $\texttt{dist}[u] = \delta(s,u)$, то след това изпълнение
  $\texttt{dist}[v] = \delta(s,v)$ и стойността на $\texttt{dist}[v]$ повече не се променя.
\end{prop}
\begin{proof}
  Първо да отбележим, че за $(u,v) \in E$, веднага след изпълнението на UPDATE(u,v) имаме, че
  \[\texttt{dist}[v] \leq \texttt{dist}[u] + w(u,v).\]
  Ако $\texttt{dist}[u] = \delta(s,u)$, то от Твърдение \ref{prop:upper-bound} това равенство се запазва.
  Получаваме, че:
  \begin{align*}
    \texttt{dist}[v] & \leq \texttt{dist}[u] + w(u,v)\\
    & = \delta(s,u) + w(u,v)\\
    & = \delta(s,v),
  \end{align*}
  защото $s\leadsto u \to v$ е път с минимална дължина.
  Тогава $\texttt{dist}[v] \leq \delta(s,v)$ и следователно 
  \[\texttt{dist}[v] = \delta(s,v),\]
  защото пак от Твърдение \ref{prop:upper-bound}, винаги е изпълнено, че $\texttt{dist}[v] \geq \delta(s,v)$,
\end{proof}

% \begin{prop}
%   \label{prop:path-update}
%   Да разгледаме пътя $p = (v_0,\dots,v_k)$, като $v_0 = s$.
%   Нека сме изпълнили INIT(s) и след това няколко пъти UPDATE, като сме включили 
%   UPDATE($v_{i}$,$v_{i+1}$), за всяко $0\leq i < k$, в този ред на изпълнение.
%   Тогава най-накрая получаваме, че $\texttt{dist}[v_k] = \delta(s,v_k)$.
% \end{prop}
% \begin{proof}
%   Индукция по $i$.
%   В началото, $\texttt{dist}[v_0] = \texttt{dist}[s] = 0 = \delta(s,s)$.
%   Ако $\texttt{dist}[v_{i-1}] = \delta(s,v_{i-1})$, то след изпълнение на UPDATE($v_{i-1}$,$v_{i}$),
%   получаваме от Твърдение \ref{prop:converge}, че $\texttt{dist}[v_i] = \delta(s,v_{i})$.
% \end{proof}

% \begin{prop}
%   \label{prop:tree-shortest-path}
%   Нека сега да приемем, че в нашия граф няма цикли с отрицателни тегла, достижими от $s$
%   и нека сме изпълнили INIT(s) и произволен брой пъти UPDATE.
%   Тогава:
%   \begin{enumerate}[1)]
%   \item 
%     $G_{pred}$ е дърво с корен $s$.
%   \item
%     ако $(\forall v\in V)[\texttt{dist}[v] = \delta(s,v)]$, то $G_{pred}$ е дърво на пътищата с минимални тегла с корен $s$.
%   \end{enumerate}
% \end{prop}
% \begin{proof}
%   \begin{enumerate}[1)]
%   \item 
%     Първо ще докажем, че $G_{pred}$ е насочен ацикличен граф и след
%     това, че няма пътища $p \neq p^\prime$ от вида $s\stackrel{p}{\leadsto} v$ и $s\stackrel{p^\prime}{\leadsto} v$.
%     \begin{itemize}
%     \item 
%       Да допуснем, че $G_{pred}$ е цикличен граф.
%       Нека $\gamma = (v_0,\dots,v_k)$ е цикъл, $v_0 = v_k$, който се е получил точно след изпълнение на 
%       UPDATE($v_{k-1}$,$v_k$).

%       Да разгледаме ситуацията точно преди това изпълнение на UPDATE($v_{k-1}$,$v_{k}$).
%       Имаме, че 
%       \[(\forall i < k-1)[\texttt{pred}[v_{i+1}] = v_{i}]\]
%       от което следва, че
%       \begin{equation}
%         \label{ineq}
%         (\forall i < k-1)[\texttt{dist}[v_{i+1}] \geq \texttt{dist}[v_i] + w(v_i,v_{i+1})].
%       \end{equation}
%       Щом още нямаме цикъл преди изпълнението на UPDATE($v_{k-1}$,$v_k$),
%       то стойността на $\texttt{pred}[v_k]$ се променя при извикването на UPDATE($v_{k-1}$,$v_k$).
%       Оттук следва, че
%       \[\texttt{dist}[v_{k}] > \texttt{dist}[v_{k-1}] + w(v_{k-1},v_k).\]
%       Комбинирайки с Неравенство (\ref{ineq}) получаваме, че
%       \begin{align*}
%         \sum^{k}_{i = 1} \texttt{dist}[v_{i}] & > \sum^{k-1}_{i=0} (\texttt{dist}[v_i] + w(v_{i},v_{i+1}))\\
%         & = \sum^{k-1}_{i=0} \texttt{dist}[v_i] + w(\gamma),
%       \end{align*}
%       но понеже $v_0 = v_k$, 
%       \[\sum^{k-1}_{i=0} \texttt{dist}[v_i] = \sum^{k}_{i=1} \texttt{dist}[v_i]\]
%       и тогава
%       \[0 > w(\gamma).\]
%       Получаваме, че цикълът $\gamma$ има отрицателно тегло, което е противоречие.
%     \item
%       Да допуснем, че в $G_{pred}$ има $p \neq p^\prime$  и  $s\stackrel{p}{\leadsto} v$ и $s\stackrel{p^\prime}{\leadsto} v$.
%       Това означава, че съществуват $x \neq y$,
%       $s \leadsto u \leadsto x \to z \leadsto v$ и $s \leadsto u \leadsto y \to z \leadsto v$.
%       По определение, $pred(z) = x \neq y = pred(z)$. Противоречие.
%     \end{itemize}
%   \item
%     \begin{itemize}
%     \item 
%       Лесно се съобразява, че $V_{pred}$ съдържа точно върховете достижими от $s$,
%       т.е. ако $v \in V_{pred}$, то съществува път $p = (s,\dots,v)$.
%       % защото $v$ е достижим от $s$ точно когато $\delta(s,v) = \texttt{dist}[v] < \infty$, 
%       % но тогава $\texttt{pred}[v] \neq NIL$.
%     \item
%       Вече доказахме в 1), че $G_{pred}$ е дърво с корен $s$.
%     \item
%       Остана да докажем, че ако имаме $s\stackrel{p}{\leadsto} v$ в $G_{pred}$, 
%       то $p$ е път с минимално тегло в $G$.
%       Нека $p = (v_0,\dots,v_k)$, $v_0 = s$, $v_k = v$.
%       По условие, 
%       \[(\forall i < k)[\texttt{dist}[v_i] = \delta(s,v_i)],\]
%       а от факта, че $(\forall i < k)[\texttt{pred}[v_i] = v_{i-1}]$ следва, че
%       \[(\forall i < k)[\texttt{dist}[v_i] \geq \texttt{dist}[v_{i-1}] + w(v_{i-1},v_{i})].\]
%       Като обединим горните две неравенства, получваме, че
%       \begin{equation}
%         \label{eq:weight}
%         (\forall i < k)[w(v_{i-1},v_{i}) \leq \delta(s,v_{i}) - \delta(s,v_{i-1})],
%       \end{equation}
%       Тогава
%       \begin{align*}
%         w(p) & = \sum^k_{i=1} w(v_{i-1},v_{i}) & (\text{по деф.})\\
%         & \leq \sum^k_{i=1} (\delta(s,v_i)- \delta(s,v_{i-1})) & (\ref{eq:weight})\\
%         & = \delta(s,v_k) - \delta(s,v_0)\\
%         & = \delta(s,v_k) - \delta(s,s) & (v_0 = s)\\
%         & = \delta(s,v_k) - 0\\
%         & = \delta(s,v_k).
%       \end{align*}
%       Следователно, 
%       \[w(p) \leq \delta(s,v_k).\]
%       Понеже $\delta(s,v_k)$ е минималното тегло на път от $s$ до $v_k$,
%       то $w(p) = \delta(s,v_k)$.
%       Следователно, $p$ е път с минимално тегло.
%     \end{itemize}
%   \end{enumerate}
% \end{proof}




\subsection{Алгоритъм на Дейкстра}
\index{Дейкстра!алгоритъм}

В този алгоритъм, разглеждаме ориентирани графи $G = (V,E,w)$ с {\em положителни} тегла (или цени) по ребрата, т.е. 
за всяко $(u,v) \in E$, $w(u,v) \geq 0$.

\begin{algorithm}[H]
  \caption{Пътища с мин. тегло от върха $s$ (Дейкстра)}
  \label{alg:dijkstra}
  
  \begin{algorithmic}[1]
    \Require{$w:E\to \R^+$}
    \Procedure{DIJKSTRA}{$s$}
    \State \Call{INIT}{$s$}
    \State $U := V$
    \While{$U \neq\emptyset$}
    \State Избираме $u_0\in U$, за който $\texttt{dist}[u_0] = \min\{\texttt{dist}[v] \mid v\in U\} $
    \State $U := V^\prime\setminus\{u_0\}$
    \ForAll{ $v\in Adj[u_0]$ }
    \State\Call{UPDATE}{$u_0$,$v$}
    \EndFor
    \EndWhile
    \EndProcedure
    % \Return $\delta$
  \end{algorithmic}
\end{algorithm}

\begin{thm}
  Нека $G$ е ориентиран граф с неотрицателни тегла по ребрата.
  След изпълнението на алгоритъма на Дейкстра с начален връх $s$,
  \[(\forall v \in V)[dist[v] = \delta(s,v)].\]
\end{thm}
\begin{proof}
  Ще докажем, че на всяка итерация на while-цикъла, 
  \[(\forall v\in V\setminus V^\prime)[dist[v] = \delta(s,v)].\]
  Първоначално $V\setminus V^\prime = \emptyset$.
  Ще докажем, че на всяка итерация на while-цикъла, за върха $u$, който сме премахнали от $V^\prime$,
  е изпълнено, че $dist[u] = \delta(s,u)$.
  За целта да допуснем противното и нека $u$ е първия връх, който е премахнат от $V^\prime$,
  за който $dist[u] \neq \delta(s,u)$.
  Лесно се съобразява, че $u \neq s$.
  Освен това, трябва $s \leadsto u$, защото иначе $dist[u] = \delta(s,u) = \infty$ според Твърдение \ref{prop:no-path}.
  Нека $s \stackrel{p}{\leadsto} u$ и $p$ е път с минимално тегло.
  Да разбием пътя $p$ по следния начин:
  \[s \stackrel{p_1}{\leadsto}x\to y\stackrel{p_2}{\leadsto}u,\]
  където $y$ е първия връх по пътя $p$, за който $y\not\in V^\prime$.
  Ясно е, че тогава $x \in V^\prime$ и тогава $dist[x] = \delta(s,x)$, 
  защото ние избрахме $u$ да бъде първия връх, за който $dist[u] \neq \delta(s,u)$.
  На итерацията на while-цикъла, на която добавяме $x$ към $V^\prime$, 
  ние изпълняваме UPDATE(x,y) и според Твърдение \ref{prop:converge}, $dist[y] = \delta(s,y)$.
  Но понеже $y$ е преди $u$ по път с минимално тегло и при положение, че няма ребра с отрицателни тегла,
  \[\delta(s,y) \leq \delta(s,u).\]
  Тогава
  \begin{align*}
    dist[y] & = \delta(s,y) \\
    & \leq \delta(s,u)\\
    & \leq dist[u], \mbox{според Твърдение \ref{prop:upper-bound}}.
  \end{align*}
  Но понеже $y,v \not\in V^\prime$ и сме избрали $u$ вместо $y$, то това означава, че
  \[dist[u] \leq dist[y].\]
  Следователно, 
  \[dist[y] = dist[u]\]
  и тогава 
  \[dist[u] = \delta(s,u),\]
  с което достигаме до противоречие.
\end{proof}
\begin{cor}
  $G_{pred}$ е дърво на минималните пътища с корен $s$.
\end{cor}


Ако във $V^\prime$ има останали върхове $v$, то те имат $\delta(v) = \infty$, т.е.
те са недостижими от $s$ и следователно пътят от $s$ до $v$ има дължина $\infty$.

Фигура \ref{fig:dijkstra-table} илюстрира как се променя функцията $\delta$ по време на изпълнението на алгоритъма.
Освен това, можем да намерим не само стойността на най-късите пътища, но
и списък с ребрата, които участват във всеки от тях.
% Фигура \ref{fig:dijkstra-graph} илюстрира това.
% Ребрата, оцветени в зелено, са тези, които участват в най-късите пътища.
% Жълти ребра са тези, които са кандидати да участват в най-късите пътища.
% Червени са тези ребра, които са били вече обходени и са отхвърлени като част от най-къс път.

\tikzstyle{weight} = [font=\small]
\tikzstyle{value} = [font=\small]
\tikzstyle{edge} = [draw,thick,-]
\tikzstyle{nodedecorate}=[shape=circle,draw,thick,font=\small]
\tikzstyle{arrowdecorate}=[->,>=stealth,thick]

% Rename: selected --> current
\tikzstyle{selected vertex}=[vertex, fill=yellow!50]
\tikzstyle{selected edge} = [draw,line width=5pt,-,yellow!50]

\tikzstyle{vertex}=[circle,minimum size=15pt,inner sep=0pt]
\tikzstyle{sure vertex} = [vertex, fill=green!30]

\tikzstyle{path edge} = [draw,line width=5pt,-,red!50]

\tikzstyle{sure edge} = [draw,line width=5pt,-,green!30]
% \tikzstyle{ignored edge} = [draw,line width=5pt,-,black!20]


\begin{figure}[!htbp]
  \begin{subfigure}[b]{0.5\textwidth}
    \begin{tikzpicture}[]
      
      \foreach \nodename/\x/\y/\direction/\navigate in { a/1/1/above/north,
        b/0/0/left/west, c/1/-1.5/below/south, d/3/1/above/north, e/3/-1.5/below/south, f/5/0.5/right/east, g/5/2.5/right/east}
      {
        \node (\nodename) at (\x,\y) [nodedecorate] {};
        \node [\direction] at (\nodename.\navigate) {$\nodename$};
      }
      %% edges or lines
      \path
      \foreach \startnode/\endnode/\direction/\weight in {b/a/above/7,
        b/c/below/2, c/a/left/4, a/d/below/4, c/e/below/5, d/c/left/8, e/d/right/3}
      {
        (\startnode) edge[arrowdecorate] node[\direction] {$\weight$} (\endnode)
      }
      
      \foreach \startnode/\endnode/\direction/\angle/\weight in {
        a/g/above/15/10, d/f/above/15/5, d/g/above/-15/2, f/d/below/15/1, g/f/right/15/6, e/f/below/-15/7}
      {
        (\startnode) edge[arrowdecorate,bend left=\angle] node[\direction] {$\weight$} (\endnode)
      };
    \end{tikzpicture}
    \caption{По-долу ще приложим алгоритъма на Дейкстра върху този граф}
    \label{subfig:dijkstra}
  \end{subfigure}
  \qquad
  \begin{subfigure}[b]{0.5\textwidth}
  \begin{tikzpicture}[]
    
    \foreach \nodename/\x/\y/\direction/\navigate in { a/1/1/above/north,
      s/0/0/left/west, b/1/-1.5/below/south}
      {
        \node (\nodename) at (\x,\y) [nodedecorate] {};
        \node [\direction] at (\nodename.\navigate) {$\nodename$};
      }
      %% edges or lines
      \path
      \foreach \startnode/\endnode/\direction/\weight in {s/a/above/3,
        s/b/below/2, a/b/right/-2}
      {
        (\startnode) edge[arrowdecorate] node[\direction] {$\weight$} (\endnode)
      };
    \end{tikzpicture}
    \caption{Пример, за който алгоритъмът на Дейкстра не дава верен резултат (Защо?)}
  \end{subfigure}
  \caption{}
  \end{figure}

  \begin{figure}[!htbp]
    \begin{subtable}[b]{0.5\textwidth}
      \begin{tabular}[b]{|c|c|c|c|c|c|c|c|c|}
        \hline
        $a$ & $b$ & $c$ & $d$ & $e$ & $f$ & $g$\\
        \hline
        $\infty$ & {\bf \framebox{0}} & $\infty$ & $\infty$ & $\infty$ & $\infty$ & $\infty$ \\
        \hline
        7 & $\colon$ & {\bf \framebox{2}} & $\infty$ & $\infty$ & $\infty$ & $\infty$ \\
        \hline
        {\bf \framebox{6}} & $\colon$ & $\colon$ & $\infty$ & 7 & $\infty$ & $\infty$ \\
        \hline
        $\colon$ & $\colon$ & $\colon$ & 10 & {\bf \framebox{7}} & $\infty$ & 16 \\
        \hline
        $\colon$ & $\colon$ & $\colon$ & {\bf \framebox{10}} & $\colon$ & 14 & {\bf 16} \\
        \hline
        $\colon$ & $\colon$ & $\colon$ & $\colon$ & $\colon$ & 14 & {\bf \framebox{12}} \\
        \hline
        $\colon$ & $\colon$ & $\colon$ & $\colon$ & $\colon$ & {\bf \framebox{14}} & $\colon$ \\
        \hline
        $\colon$ & $\colon$ & $\colon$ & $\colon$ & $\colon$ & $\colon$ & $\colon$ \\
        \hline
      \end{tabular}
      \caption{Масива $\texttt{dist}$ за начален връх $b$}
    \end{subtable}
    \qquad
    \begin{subtable}[b]{0.5\textwidth}
      \begin{tabular}[b]{|c|c|c|c|c|c|c|c|c|}
        \hline
        $a$ & $b$ & $c$ & $d$ & $e$ & $f$ & $g$\\
        \hline
        $\texttt{NIL}$ & \framebox{\texttt{NIL}} & $\texttt{NIL}$ & $\texttt{NIL}$ & $\texttt{NIL}$ & $\texttt{NIL}$ & $\texttt{NIL}$ \\
        \hline
        $b$ & $\colon$ & \framebox{$b$} & $\texttt{NIL}$ & $\texttt{NIL}$ & $\texttt{NIL}$ & $\texttt{NIL}$ \\
        \hline
        {\bf \framebox{c}} & $\colon$ & $\colon$ & $\texttt{NIL}$ & c & $\texttt{NIL}$ & $\texttt{NIL}$ \\
        \hline
        $\colon$ & $\colon$ & $\colon$ & a & {\bf \framebox{c}} & $\texttt{NIL}$ & a \\
        \hline
        $\colon$ & $\colon$ & $\colon$ & {\bf \framebox{a}} & $\colon$ & c & a \\
        \hline
        $\colon$ & $\colon$ & $\colon$ & $\colon$ & $\colon$ & c & {\bf \framebox{d}} \\
        \hline
        $\colon$ & $\colon$ & $\colon$ & $\colon$ & $\colon$ & {\bf \framebox{c}} & $\colon$ \\
        \hline
        $\colon$ & $\colon$ & $\colon$ & $\colon$ & $\colon$ & $\colon$ & $\colon$ \\
        \hline
      \end{tabular}
      \caption{Масива $\texttt{pred}$ за начален връх $b$}
    \end{subtable}
    \caption{Алгоритъм на Дейкстра с начален връх $b$ за графа от (\ref{subfig:dijkstra})}
  \label{fig:dijkstra-table}
\end{figure}

% \begin{figure}[!htbp]
%   \index{Дейкстра!алгоритъм}
%   

\subfigure[Започваме от съседите на $a$]{
  \begin{tikzpicture}[scale=0.9]
    % nodes
    \foreach \nodename/\x/\y/\value/\direction/\navigate/\color in { 
      a/0/0/0/above/north/green, 
      b/-1/1/\infty/left/west/black, 
      c/2.5/1/\infty/above/north/black, 
      d/2/-0.5/\infty/below/south/black,
      e/-1/-0.5/\infty/below/south/black, 
      f/0.5/2/\infty/above/north/black,
      g/0/-1.8/\infty/below/south/black, 
      h/2.5/-1.8/\infty/right/east/black}
    {
        \node[vertex, nodedecorate, fill=\color!25] (\nodename) at (\x,\y) {$\value$};
        \node [\direction] at (\nodename.\navigate) {$\nodename$};
      };
      % edges
      \path
      \foreach \startnode/\endnode/\direction/\angle/\weight in {
        e/b/left/15/5, e/g/below/-15/3, d/g/below/15/2, 
        g/a/left/15/2, a/g/right/30/9, f/c/below/-15/1,
        c/f/above/-30/5, c/h/right/15/2, c/d/above/-15/1,
        f/d/left/-15/4, a/b/below/0/2, b/f/above/0/1, 
        a/d/above/0/8}
      {
        (\startnode) edge[arrowdecorate,bend left=\angle] node[\direction] {$\weight$} (\endnode)
      };
    \end{tikzpicture}
  }
  \subfigure[$b$ е най-близко до $a$]{
    \begin{tikzpicture}[scale=0.9]
      %nodes
      \foreach \nodename/\x/\y/\value/\direction/\navigate/\color in { 
        a/0/0/0/above/north/green, 
        b/-1/1/2/left/west/yellow, 
        c/2.5/1/\infty/above/north/black, 
        d/2/-0.5/8/below/south/yellow,
        e/-1/-0.5/\infty/below/south/black, 
        f/0.5/2/\infty/above/north/black, 
        g/0/-1.8/9/below/south/yellow, 
        h/2.5/-1.8/\infty/right/east/black}
      {
        \node[vertex, nodedecorate, fill=\color!25] (\nodename) at (\x,\y) {$\value$};
        \node [\direction] at (\nodename.\navigate) {$\nodename$};
      };

      \path
      \foreach \startnode/\endnode/\direction/\angle in {
        a/g/right/30, a/b/above/0, a/d/above/0}
      {
        (\startnode) edge[selected edge,bend left=\angle] node[\direction] {} (\endnode)
      };
      %edges
      \path
      \foreach \startnode/\endnode/\direction/\angle/\weight in {
        e/b/left/15/5, e/g/below/-15/3, d/g/below/15/2, g/a/left/15/2, a/g/right/30/9, f/c/below/-15/1, c/f/above/-30/5,
        c/h/right/15/2, c/d/above/-15/1, f/d/left/-15/4, a/b/below/0/2, b/f/above/0/1,  a/d/above/0/8}
      {
        (\startnode) edge[arrowdecorate,bend left=\angle] node[\direction] {$\weight$} (\endnode)
      };
    \end{tikzpicture}
  }
  \subfigure[Най-къс път до $f$]{
    \begin{tikzpicture}[scale=0.9]
      %nodes
      \foreach \nodename/\x/\y/\value/\direction/\navigate/\color in { 
        a/0/0/0/above/north/green, 
        b/-1/1/2/left/west/green,
        c/2.5/1/\infty/above/north/black, 
        d/2/-0.5/8/below/south/yellow,
        e/-1/-0.5/\infty/below/south/black, 
        f/0.5/2/3/above/north/yellow, 
        g/0/-1.8/9/below/south/yellow, 
        h/2.5/-1.8/\infty/right/east/black}
      {
        \node[vertex, nodedecorate, fill=\color!25] (\nodename) at (\x,\y) {$\value$};
        \node [\direction] at (\nodename.\navigate) {$\nodename$};
      };
      \path
      (a) edge[sure edge] node[] {} (b);
      \path
      \foreach \startnode/\endnode/\direction/\angle in {
        a/g/right/30, a/d/above/0}
      {
        (\startnode) edge[selected edge,bend left=\angle] node[\direction] {} (\endnode)
      };
      \path
      \foreach \startnode/\endnode/\direction/\angle in {
         b/f/above/0}
      {
        (\startnode) edge[selected edge, bend left=\angle] node[\direction] {} (\endnode)
      };
      %edges
      \path
      \foreach \startnode/\endnode/\direction/\angle/\weight in {
        e/b/left/15/5, e/g/below/-15/3, d/g/below/15/2, g/a/left/15/2, a/g/right/30/9, f/c/below/-15/1, c/f/above/-30/5,
        c/h/right/15/2, c/d/above/-15/1, f/d/left/-15/4, a/b/below/0/2, b/f/above/0/1,  a/d/above/0/8}
      {
        (\startnode) edge[arrowdecorate,bend left=\angle] node[\direction] {$\weight$} (\endnode)
      };
    \end{tikzpicture}
  }
  \subfigure[По-къс път до $d$ и $c$]{
    \begin{tikzpicture}[scale=0.9]
      % nodes
      \foreach \nodename/\x/\y/\value/\direction/\navigate/\color in { 
        a/0/0/0/above/north/green,
        b/-1/1/2/left/west/green,
        c/2.5/1/4/above/north/yellow, 
        d/2/-0.5/7/below/south/yellow,
        e/-1/-0.5/\infty/below/south/black, 
        f/0.5/2/3/above/north/green, 
        g/0/-1.8/9/below/south/yellow, 
        h/2.5/-1.8/\infty/right/east/black}
      {
        \node[vertex, nodedecorate, fill=\color!25] (\nodename) at (\x,\y) {$\value$};
        \node [\direction] at (\nodename.\navigate) {$\nodename$};
      };
      \path
      (a) edge[sure edge] node[] {} (b)
      (b) edge[sure edge] node[] {} (f);
      
      \path
      \foreach \startnode/\endnode/\direction/\angle in {
        a/d/above/0}
      {
        (\startnode) edge[path edge,bend left=\angle] node[\direction] {} (\endnode)
      };
      \path
      \foreach \startnode/\endnode/\direction/\angle in {f/c/below/-15/1, f/d/left/-15/4, a/g/right/30}
      {
        (\startnode) edge[selected edge, bend left=\angle] node[\direction] {} (\endnode)
      };
      %edges
      \path
      \foreach \startnode/\endnode/\direction/\angle/\weight in {
        e/b/left/15/5, e/g/below/-15/3, d/g/below/15/2, g/a/left/15/2, a/g/right/30/9, f/c/below/-15/1, c/f/above/-30/5,
        c/h/right/15/2, c/d/above/-15/1, f/d/left/-15/4, a/b/below/0/2, b/f/above/0/1,  a/d/above/0/8}
      {
        (\startnode) edge[arrowdecorate,bend left=\angle] node[\direction] {$\weight$} (\endnode)
      };
    \end{tikzpicture}
  }
  \subfigure[По-къс път до $d$ и $h$]{
    \begin{tikzpicture}[scale=0.9]
      % nodes
      \foreach \nodename/\x/\y/\value/\direction/\navigate/\color in { 
        a/0/0/0/above/north/green,
        b/-1/1/2/left/west/green,
        c/2.5/1/4/above/north/green,
        d/2/-0.5/5/below/south/yellow,
        e/-1/-0.5/\infty/below/south/black, 
        f/0.5/2/3/above/north/green, 
        g/0/-1.8/9/below/south/yellow, 
        h/2.5/-1.8/6/right/east/yellow}
      {
        \node[vertex, nodedecorate, fill=\color!25] (\nodename) at (\x,\y) {$\value$};
        \node [\direction] at (\nodename.\navigate) {$\nodename$};
      };
      \path
      (a) edge[sure edge] node[] {} (b)
      (b) edge[sure edge] node[] {} (f)
      (f) edge[sure edge, bend left=-15] node[] {} (c);
      
      \path
      \foreach \startnode/\endnode/\direction/\angle in {f/d/left/-15/4, a/d/above/0
        }
      {
        (\startnode) edge[path edge,bend left=\angle] node[\direction] {} (\endnode)
      };
      \path
      \foreach \startnode/\endnode/\direction/\angle in {c/d/above/-15/1,  c/f/above/-30/5, c/h/right/15/2, a/g/right/30}
      {
        (\startnode) edge[selected edge, bend left=\angle] node[\direction] {} (\endnode)
      };
      %edges
      \path
      \foreach \startnode/\endnode/\direction/\angle/\weight in {
        e/b/left/15/5, e/g/below/-15/3, d/g/below/15/2, g/a/left/15/2, a/g/right/30/9, f/c/below/-15/1, c/f/above/-30/5,
        c/h/right/15/2, c/d/above/-15/1, f/d/left/-15/4, a/b/below/0/2, b/f/above/0/1,  a/d/above/0/8}
      {
        (\startnode) edge[arrowdecorate,bend left=\angle] node[\direction] {$\weight$} (\endnode)
      };
    \end{tikzpicture}
  }
  \subfigure[По-къс път до $g$]{
    \begin{tikzpicture}[scale=0.9]
      %nodes
      \foreach \nodename/\x/\y/\value/\direction/\navigate/\color in { 
        a/0/0/0/above/north/green, 
        b/-1/1/2/left/west/green,
        c/2.5/1/4/above/north/green,
        d/2/-0.5/5/below/south/green,
        e/-1/-0.5/\infty/below/south/black, 
        f/0.5/2/3/above/north/green, 
        g/0/-1.8/7/below/south/yellow,
        h/2.5/-1.8/6/right/east/yellow}
      {
        \node[vertex, nodedecorate, fill=\color!25] (\nodename) at (\x,\y) {$\value$};
        \node [\direction] at (\nodename.\navigate) {$\nodename$};
      };
      \path
      (a) edge[sure edge] node[] {} (b)
      (b) edge[sure edge] node[] {} (f)
      (f) edge[sure edge, bend left=-15] node[] {} (c)
      (c) edge[sure edge, bend left=-15] node[] {} (d);
      
      \path
      \foreach \startnode/\endnode/\direction/\angle in {f/d/left/-15, a/g/right/30, a/d/above/0, c/f/above/-30
        }
      {
        (\startnode) edge[path edge,bend left=\angle] node[] {} (\endnode)
      };
      \path
      \foreach \startnode/\endnode/\direction/\angle in {d/g/below/15/2, c/h/right/15}
      {
        (\startnode) edge[selected edge, bend left=\angle] node[\direction] {} (\endnode)
      };
      %edges
      \path
      \foreach \startnode/\endnode/\direction/\angle/\weight in {
        e/b/left/15/5, e/g/below/-15/3, d/g/below/15/2, g/a/left/15/2, a/g/right/30/9, f/c/below/-15/1, c/f/above/-30/5,
        c/h/right/15/2, c/d/above/-15/1, f/d/left/-15/4, a/b/below/0/2, b/f/above/0/1,  a/d/above/0/8}
      {
        (\startnode) edge[arrowdecorate,bend left=\angle] node[\direction] {$\weight$} (\endnode)
      };
    \end{tikzpicture}
  }
  \subfigure[$h$ е задънена улица]{
    \begin{tikzpicture}[scale=0.9]
      %nodes
      \foreach \nodename/\x/\y/\value/\direction/\navigate/\color in { 
        a/0/0/0/above/north/green, 
        b/-1/1/2/left/west/green,
        c/2.5/1/4/above/north/green, 
        d/2/-0.5/5/below/south/green,
        e/-1/-0.5/\infty/below/south/black, 
        f/0.5/2/3/above/north/green, 
        g/0/-1.8/7/below/south/yellow,
        h/2.5/-1.8/6/right/east/green}
      {
        \node[vertex, nodedecorate, fill=\color!25] (\nodename) at (\x,\y) {$\value$};
        \node [\direction] at (\nodename.\navigate) {$\nodename$};
      };
      \path
      (a) edge[sure edge] node[] {} (b)
      (b) edge[sure edge] node[] {} (f)
      (f) edge[sure edge, bend left=-15] node[] {} (c)
      (c) edge[sure edge, bend left=-15] node[] {} (d)
      (c) edge[sure edge, bend left=15] node[] {} (h);
      
      \path
      \foreach \startnode/\endnode/\direction/\angle in {f/d/left/-15, a/g/right/30, a/d/above/0, c/f/above/-30
        }
      {
        (\startnode) edge[path edge,bend left=\angle] node[] {} (\endnode)
      };
      \path
      \foreach \startnode/\endnode/\direction/\angle in {d/g/below/15/2}
      {
        (\startnode) edge[selected edge, bend left=\angle] node[\direction] {} (\endnode)
      };
      %edges
      \path
      \foreach \startnode/\endnode/\direction/\angle/\weight in {
        e/b/left/15/5, e/g/below/-15/3, d/g/below/15/2, g/a/left/15/2, a/g/right/30/9, f/c/below/-15/1, c/f/above/-30/5,
        c/h/right/15/2, c/d/above/-15/1, f/d/left/-15/4, a/b/below/0/2, b/f/above/0/1,  a/d/above/0/8}
      {
        (\startnode) edge[arrowdecorate,bend left=\angle] node[\direction] {$\weight$} (\endnode)
      };
    \end{tikzpicture}
  }
  \subfigure[$e$ не е достижим]{
    \begin{tikzpicture}[scale=0.9]
      %nodes
      \foreach \nodename/\x/\y/\value/\direction/\navigate/\color in { 
        a/0/0/0/above/north/green,
        b/-1/1/2/left/west/green,
        c/2.5/1/4/above/north/green, 
        d/2/-0.5/5/below/south/green,
        e/-1/-0.5/\infty/below/south/black, 
        f/0.5/2/3/above/north/green, 
        g/0/-1.8/7/below/south/green,
        h/2.5/-1.8/6/right/east/green}
      {
        \node[vertex, nodedecorate, fill=\color!25] (\nodename) at (\x,\y) {$\value$};
        \node [\direction] at (\nodename.\navigate) {$\nodename$};
      };
      \path
      (a) edge[sure edge] node[] {} (b)
      (b) edge[sure edge] node[] {} (f)
      (f) edge[sure edge, bend left=-15] node[] {} (c)
      (c) edge[sure edge, bend left=-15] node[] {} (d)
      (c) edge[sure edge, bend left=15] node[] {} (h)
      (d) edge[sure edge, bend left=15] node[] {} (g);
      
      \path
      \foreach \startnode/\endnode/\direction/\angle in {f/d/left/-15, a/g/right/30, a/d/above/0, c/f/above/-30
        }
      {
        (\startnode) edge[path edge,bend left=\angle] node[] {} (\endnode)
      };
      \path
      \foreach \startnode/\endnode/\direction/\angle in {g/a/left/15}
      {
        (\startnode) edge[selected edge, bend left=\angle] node[\direction] {} (\endnode)
      };
      %edges
      \path
      \foreach \startnode/\endnode/\direction/\angle/\weight in {
        e/b/left/15/5, e/g/below/-15/3, d/g/below/15/2, g/a/left/15/2, a/g/right/30/9, f/c/below/-15/1, c/f/above/-30/5,
        c/h/right/15/2, c/d/above/-15/1, f/d/left/-15/4, a/b/below/0/2, b/f/above/0/1,  a/d/above/0/8}
      {
        (\startnode) edge[arrowdecorate,bend left=\angle] node[\direction] {$\weight$} (\endnode)
      };
    \end{tikzpicture}
  }
  \subfigure[Краен резултат]{
    \begin{tikzpicture}[scale=0.9]
      %nodes
      \foreach \nodename/\x/\y/\value/\direction/\navigate/\color in { 
        a/0/0/0/above/north/green, b/-1/1/2/left/west/green,
        c/2.5/1/4/above/north/green, d/2/-0.5/5/below/south/green,
        e/-1/-0.5/\infty/below/south/black, f/0.5/2/3/above/north/green, 
        g/0/-1.8/7/below/south/green, h/2.5/-1.8/6/right/east/green}
      {
        \node[vertex, nodedecorate, fill=\color!25] (\nodename) at (\x,\y) {$\value$};
        \node [\direction] at (\nodename.\navigate) {$\nodename$};
      };
      \path
      (a) edge[sure edge] node[] {} (b)
      (b) edge[sure edge] node[] {} (f)
      (f) edge[sure edge, bend left=-15] node[] {} (c)
      (c) edge[sure edge, bend left=-15] node[] {} (d)
      (c) edge[sure edge, bend left=15] node[] {} (h)
      (d) edge[sure edge, bend left=15] node[] {} (g);
      
      \path
      \foreach \startnode/\endnode/\direction/\angle in {f/d/left/-15, a/g/right/30, a/d/above/0, c/f/above/-30, g/a/above/15
        }
      {
        (\startnode) edge[path edge,bend left=\angle] node[] {} (\endnode)
      };
      \path
      \foreach \startnode/\endnode/\direction/\angle in {}
      {
        (\startnode) edge[selected edge, bend left=\angle] node[\direction] {} (\endnode)
      };
      %edges
      \path
      \foreach \startnode/\endnode/\direction/\angle/\weight in {
        e/b/left/15/5, e/g/below/-15/3, d/g/below/15/2, g/a/left/15/2, a/g/right/30/9, f/c/below/-15/1, c/f/above/-30/5,
        c/h/right/15/2, c/d/above/-15/1, f/d/left/-15/4, a/b/below/0/2, b/f/above/0/1,  a/d/above/0/8}
      {
        (\startnode) edge[arrowdecorate,bend left=\angle] node[\direction] {$\weight$} (\endnode)
      };
    \end{tikzpicture}
  }

%%% Local Variables: 
%%% mode: latex
%%% TeX-master: "discrete-math"
%%% End: 

%   \caption{Алгоритъм на Дейкстра запазващ минималните пътища}
%   \label{fig:dijkstra-graph}
% \end{figure}



% \newpage

% \subsection{Алгоритъм на Белман-Форд}\index{Белман-Форд!алгоритъм}

% Алгоритъмът на Дейкстра работи само за графи $G = (V,E,w)$ с {\em положителни} тегла по ребрата.
% Сега ще разгледаме един алгоритъм, който работи и за графи с отрицателни тегла по ребрата.
% Задачата отново е да намерим минималните разстояния на пътищата с начало върха $s$, но
% искаме също така алгоритъмът да отговаря на въпроса дали има отрицателен цикъл в графа. 
% Ако такъв съществува, то няма решение на проблема. (Защо?)
% Ако отрицателен цикъл не съществува, то алгоритъмът намира пътища в графа с минимални тегла от върха $s$
% до всички достижими върхове в графа.


% \begin{algorithm}
%   \caption{Пътища с мин. тегло от върха $s$ (Белман-Форд)}
%   \label{alg:belman-ford}
  
%   \begin{algorithmic}[1]
%     \Procedure{Bellman-Ford}{$s$}
%     \State\Call{INIT}{$s$}
%     \For{$i:=1$ to $\abs{V}-1$}
%     \ForAll{$(u,v)\in E$}
%     \State\Call{UPDATE}{$u$,$v$}
%     % \ENSURE{$\texttt{dist}[v] \geq \delta(s,v)$}
%     \EndFor
%     \EndFor
    
%     \Comment{Проверка за отрицателен цикъл}
%     \ForAll{$(u,v)\in E$}
%     \If {$\texttt{dist}[v] > \texttt{dist}[u] + w(u,v)$}
%     \State Return \texttt{False}
%     \EndIf
%     \EndFor
%     \State Return \texttt{True}
%     \EndProcedure
%   \end{algorithmic}
% \end{algorithm}

% \begin{prop}
%   \label{prop:bellman-ford}
%   Нека графът $G$ няма отрицателни цикли, които са достижими от $s$.
%   Тогава след изпълнение на алгоритъма на Белман-Форд получаваме, че
%   за всички $v \in V$ достижими от $s$, 
%   \[\texttt{dist}[v] = \delta(s,v).\]
% \end{prop}
% \begin{proof}
%   Да разгледаме $s \stackrel{p}{\leadsto} v$, където $p = (v_0,\dots,v_k)$ е път с минимално тегло в $G$.
%   Понеже в пътища с минимална дължина няма цикли, то $k \leq \abs{V} - 1$.
%   Тогава според Твърдение \ref{prop:path-update}, 
%   след $i$-тата итерация на \texttt{for} цикъла (ред 3), $\texttt{dist}[v_i] = \delta(s,v_i)$.
%   Така получаваме, че най-накрая $\texttt{dist}[v] = \delta(s,v)$.
% \end{proof}
% \begin{cor}
%   \label{cor:bellman-ford}
%   При същите предположения за графа $G$,
%   за всяко $v \in V$, 
%   има път $s \leadsto v$ точно тогава, когато след приключване на алгоритъма е изпълнено $\texttt{dist}[v] < \infty$.
% \end{cor}
% \begin{proof}
%   Ако има път $p$, $s \stackrel{p}{\leadsto} v$, то
%   според твърдението $\texttt{dist}[v] = \delta(s,v) < \infty$.
%   За другата посока, нека $\texttt{dist}[v] < \infty$, но да допуснем, че няма път от $s$ до $v$.
%   Но тогава от Твърдение \ref{prop:no-path} следва, че $\texttt{dist}[v]  = \infty$,
%   което е противоречие.
% \end{proof}


% \begin{thm}
%   \label{th:bellman-ford}
%   Ако $G$ няма отрицателни цикли достижими от $s$, то
%   алгоритъмът на Белман-Форд връща TRUE, $(\forall v\in V)[dist[v] = \delta(s,v)]$,
%   и $G_{pred}$ е дърво с корен $s$, което съдържа пътища с минимални тегла.

%   Ако $G$ има отрицателни цикли достижими от $s$, то
%   алгоритъмът на Белман-Форд връща FALSE.
% \end{thm}
% \begin{proof}
%   \begin{enumerate}[a)]
%   \item 
%     Нека $G$ не съдържа цикъл с отрицателно тегло, достижим от $s$.
%     Ако $v$ е достижим от $s$, то според Твърдение \ref{prop:bellman-ford}, 
%     след изпълнение на алгоритъма
%     \[\texttt{dist}[v] = \delta(s,v).\]

%     Ако $v$ не е достижим от $s$, то според Твърдение \ref{prop:no-path},
%     след изпълнение на алгоритъма
%     \[\texttt{dist}[v] = \infty = \delta(s,v).\]

%     Понеже $(\forall v\in V)[\texttt{dist}[v] = \delta(s,v)]$, от Твърдение \ref{prop:tree-shortest-path} следва, че
%     $G_{pred}$ е дърво с корен $s$, което съдържа пътища с минимални тегла.
    
%     Като използваме Твърдение \ref{prop:triangle} лесно се вижда, че алгоритъмът връща TRUE.
%   \item
%     Нека $G$ съдържа цикъл с отрицателно тегло, достижим от $s$.
%     Нека един такъв цикъл е $\gamma = (v_0,\dots,v_k)$, $v_0 = v_k$.
%     Тогава
%     \[w(\gamma) = \sum^k_{i=1}w(v_{i-1},v_i) < 0.\]
%     Да допуснем, че алгоритъмът връща TRUE. Тогава за всяко $i = 1,\dots, k$, 
%     \[\texttt{dist}[v_i] \leq \texttt{dist}[v_{i-1}] + w(v_{i-1},v_i)\]
%     и като сумираме, 
%     \[\sum^{k}_{i=1} \texttt{dist}[v_i] \leq \sum^{k}_{i=1} \texttt{dist}[v_{i-1}] + \sum^{k}_{i=1}w(v_{i-1},v_i).\]
%     Тъй като $v_0 = v_k$, 
%     \[\sum^{k}_{i=1} \texttt{dist}[v_i] = \sum^{k}_{i=1}\texttt{dist}[v_{i-1}].\]
%     Получаваме, че \[0 \leq \sum^{k}_{i=1}w(v_{i-1},v_i) = w(\gamma),\] което е противоречие с отицателността на цикъла.
%   \end{enumerate}
% \end{proof}

% Фигура \ref{fig:bellman-ford-negative-cycle} илюстрира случая за цикъл с отрицателно тегло.
% Както забелязахме при алгоритъма на Дейкстра, и тук можем да намерим не само дължините на най-късите пътища, но
% и спъсъка на ребрата, участващи в тях. Фигура \ref{fig:bellman-ford-graph} илюстрира този проблем. 
% Останалите накрая оцветени в синьо ребра участват в най-късите пътища.



% \begin{figure}[!htbp]
%   \begin{subfigure}[b]{0.5\textwidth}
%     \begin{tikzpicture}
%       [nodedecorate/.style={shape=circle,inner sep=2pt,draw,thick},%
%       arrowdecorate/.style={->,>=stealth,thick}]
%       %% nodes or vertices
      
%       \foreach \nodename/\x/\y/\direction/\navigate in { a/0/0/below/south,
%         b/6/0/below/south, c/4.5/0/below/south, d/3/0/below/south, e/1.5/0/below/south}
%       {
%         \node (\nodename) at (\x,\y) [nodedecorate] {};
%         \node [\direction] at (\nodename.\navigate) {$\nodename$};
%       }
%       %% edges or lines
%       \path
%       \foreach \startnode/\endnode/\direction/\angle/\weight in {
%         a/e/above/15/1, d/e/above/-25/-1, e/d/below/-15/1,  d/c/above/15/1, c/b/above/15/1, b/e/below/60/-4}
%       {
%         (\startnode) edge[arrowdecorate,bend left=\angle] node[\direction] {$\weight$} (\endnode)
%       };
%       ;
%     \end{tikzpicture}
%     \caption{Граф с отрицателен цикъл}
%   \end{subfigure}
%   \quad
%   \begin{subtable}[b]{0.5\textwidth}
%     \begin{tabular}{|c|c|c|c|c|}
%       \hline
%       $\delta(a)$ & $\delta(b)$ & $\delta(c)$ & $\delta(d)$ & $\delta(e)$ \\
%       \hline
%       0 & $\infty$ & $\infty$ & $\infty$ & {\bf \framebox{1}}\\
%       \hline
%       \hline
%       $\colon$ & \framebox{$\infty$} & $\infty$ & $\infty$ & 1\\
%       $\colon$ & $\infty$ & \framebox{$\infty$} & $\infty$ & 1\\
%       $\colon$ & $\infty$ & $\infty$ & {\bf \framebox{2}} & 1\\
%       $\colon$ & $\infty$ & $\infty$ & 2 & \framebox{1}\\
%       \hline\hline
%       $\colon$ & \framebox{$\infty$} & $\infty$ & 2 & 1\\
%       $\colon$ & $\infty$ & {\bf \framebox{3}} & 2 & 1\\
%       $\colon$ & $\infty$ & 3 & \framebox{2} & 1\\
%       $\colon$ & $\infty$ & $\infty$ & 2 & \framebox{1}\\
%       \hline\hline
%       $\colon$ & {\bf \framebox{4}} & 3 & 2 & 1\\
%       $\colon$ & 4 & \framebox{3} & 2 & 1\\
%       $\colon$ & 4 & 3 & \framebox{2} & 1\\
%       $\colon$ & 4 & 3 & 2 & {\bf \framebox{0}}\\
%       \hline\hline
%     \end{tabular}
%     \caption{Изпълнение на алгоритъма}
%   \end{subtable}
%   \caption{Алгоритъм на Белман-Форд върху ориентиран граф с отрицателен цикъл,
%   като ребрата са подредени лексикографски: $\pair{a,e}, \pair{b,e}, \pair{c,b}, \pair{d,c}, \pair{d,e}, \pair{e, d}$}
%   \label{fig:bellman-ford-negative-cycle}
% \end{figure}


% % \begin{figure}[!htbp]
% %   
\begin{subfigure}[b]{0.3\textwidth}
    \begin{tikzpicture}[scale=0.9]
      %% nodes or vertices
      
      \foreach \nodename/\value/\x/\y/\direction/\navigate/\color in { 
        s/0/0/0/left/west/green, 
        x/\infty/3.5/1.5/above/north/black,
        y/\infty/1/-1.5/below/south/black,
        t/\infty/1/1.5/above/north/black, 
        z/\infty/3.5/-1.5/below/south/black}
      {
        \node[vertex, nodedecorate, fill=\color!25] (\nodename) at (\x,\y) {$\value$};
        \node [\direction] at (\nodename.\navigate) {$\nodename$};
      }
      % edges or lines
      \path
      \foreach \startnode/\endnode/\direction/\angle/\weight in {
        s/t/left/15/6, s/y/left/-25/7, y/z/below/0/9,  t/y/left/0/8, t/x/above/30/5, x/t/above/30/-2, z/x/right/0/7,
        z/s/below/0/2, y/x/below/-3, t/z/right/15/-4
      }
      {
        (\startnode) edge[arrowdecorate,bend left=\angle] node[\direction] {$\weight$} (\endnode)
      };
      ;
    \end{tikzpicture}
    \caption{Начален връх е $s$}
  \end{subfigure}
  \quad
  \begin{subfigure}[b]{0.3\textwidth}
    \begin{tikzpicture}[scale=0.9]
      \foreach \nodename/\value/\x/\y/\direction/\navigate/\color in { 
        s/0/0/0/left/west/green, 
        x/\infty/3.5/1.5/above/north/black,
        y/7/1/-1.5/below/south/red,
        t/6/1/1.5/above/north/red, 
        z/\infty/3.5/-1.5/below/south/black}
      {
        \node[vertex, nodedecorate, fill=\color!25] (\nodename) at (\x,\y) {$\value$};
        \node [\direction] at (\nodename.\navigate) {$\nodename$};
      }
    
    \path
    \foreach \startnode/\endnode/\angle in {
      s/t/15, s/y/-25}
    {
      (\startnode) edge[selected edge, bend left=\angle] node[] {} (\endnode)
    };
    
    %edges
    \path
    \foreach \startnode/\endnode/\direction/\angle/\weight in {
      s/t/left/15/6, s/y/left/-25/7, y/z/below/0/9,  t/y/left/0/8, t/x/above/30/5, x/t/above/30/-2, z/x/right/0/7,
      z/s/below/0/2, y/x/below/-3, t/z/right/15/-4}
    {
      (\startnode) edge[arrowdecorate,bend left=\angle] node[\direction] {$\weight$} (\endnode)
    };
    

  \end{tikzpicture}
  \caption{Започваме със съседите на $s$}
  \end{subfigure}
  \quad
  \begin{subfigure}[b]{0.3\textwidth}
    \begin{tikzpicture}[scale=0.9]
      
      \foreach \nodename/\value/\x/\y/\direction/\navigate/\color in { 
        s/0/0/0/left/west/green, 
        x/4/3.5/1.5/above/north/red,
        y/7/1/-1.5/below/south/blue,
        t/6/1/1.5/above/north/blue, 
        z/2/3.5/-1.5/below/south/red}
      {
        \node[vertex, nodedecorate, fill=\color!25] (\nodename) at (\x,\y) {$\value$};
        \node [\direction] at (\nodename.\navigate) {$\nodename$};
      }
    
    \path
    \foreach \startnode/\endnode/\angle in {
      s/t/15, s/y/-25}
    {
      (\startnode) edge[path edge, bend left=\angle] node[] {} (\endnode)
    };


    %edges or lines
    \path
      (y) edge[selected edge] node[below] {} (x)
      (t) edge[selected edge, bend left=15] node[left] {} (z);

    \path
    \foreach \startnode/\endnode/\direction/\angle/\weight in {
      s/t/left/15/6, s/y/left/-25/7, y/z/below/0/9,  t/y/left/0/8, t/x/above/30/5, x/t/above/30/-2, z/x/right/0/7,
      z/s/below/0/2, y/x/below/0/-3, t/z/right/15/-4
    }
    {
      (\startnode) edge[arrowdecorate,bend left=\angle] node[\direction] {$\weight$} (\endnode)
    };
    ;

  \end{tikzpicture}
  \caption{Продължаваме с $x$ и $z$}
  \end{subfigure}
\quad
\begin{subfigure}[b]{0.3\textwidth}
  \begin{tikzpicture}[scale=0.9]
    
    \foreach \nodename/\value/\x/\y/\direction/\navigate/\color in { 
      s/0/0/0/left/west/green, 
      x/4/3.5/1.5/above/north/blue,
      y/7/1/-1.5/below/south/blue,
      t/2/1/1.5/above/north/red, 
      z/2/3.5/-1.5/below/south/blue}
    {
      \node[vertex, nodedecorate, fill=\color!25] (\nodename) at (\x,\y) {$\value$};
      \node [\direction] at (\nodename.\navigate) {$\nodename$};
    }
    
    \path
    \foreach \startnode/\endnode/\angle in {
      s/y/-25, y/x/0, t/z/15}
    {
      (\startnode) edge[path edge, bend left=\angle] node[] {} (\endnode)
    };

    \path
    \foreach \startnode/\endnode/\angle in {
      x/t/30
    }
    {
      (\startnode) edge[selected edge,bend left=\angle] node[] {} (\endnode)
    };
    %edges or lines
    \path
    \foreach \startnode/\endnode/\direction/\angle/\weight in {
      s/t/left/15/6, s/y/left/-25/7, y/z/below/0/9,  t/y/left/0/8, t/x/above/30/5, x/t/above/30/-2, z/x/right/0/7,
      z/s/below/0/2, y/x/below/0/-3, t/z/right/15/-4
    }
    {
      (\startnode) edge[arrowdecorate,bend left=\angle] node[\direction] {$\weight$} (\endnode)
    };
    ;
  \end{tikzpicture}
  \caption{По-кратък път до $t$}
\end{subfigure}
\quad
\begin{subfigure}[b]{0.3\textwidth}
  \begin{tikzpicture}[scale=0.9]
    %% nodes or vertices
    \foreach \nodename/\value/\x/\y/\direction/\navigate/\color in { 
      s/0/0/0/left/west/green, 
      x/4/3.5/1.5/above/north/blue,
      y/7/1/-1.5/below/south/blue,
      t/2/1/1.5/above/north/blue, 
      z/-2/3.5/-1.5/below/south/red}
    {
      \node[vertex, nodedecorate, fill=\color!25] (\nodename) at (\x,\y) {$\value$};
      \node [\direction] at (\nodename.\navigate) {$\nodename$};
    }
    \path
    \foreach \startnode/\endnode/\angle in {
      s/y/-25, y/x/0, x/t/30}
    {
      (\startnode) edge[path edge, bend left=\angle] node[] {} (\endnode)
    };
    \path
    \foreach \startnode/\endnode/\angle in {
      t/z/15/
    }
    {
      (\startnode) edge[selected edge,bend left=\angle] node[] {} (\endnode)
    };
    %edges or lines
    \path
    \foreach \startnode/\endnode/\direction/\angle/\weight in {
      s/t/left/15/6, s/y/left/-25/7, y/z/below/0/9,  t/y/left/0/8, t/x/above/30/5, x/t/above/30/-2, z/x/right/0/7,
      z/s/below/0/2, y/x/below/0/-3, t/z/right/15/-4
    }
    {
      (\startnode) edge[arrowdecorate,bend left=\angle] node[\direction] {$\weight$} (\endnode)
    };
  \end{tikzpicture}
  \caption{По-кратък път до $z$}
  \end{subfigure}
\quad
\begin{subfigure}[b]{0.3\textwidth}
  \begin{tikzpicture}[scale=0.9]
    %% nodes or vertices
    \foreach \nodename/\value/\x/\y/\direction/\navigate/\color in { 
      s/0/0/0/left/west/green, 
      x/4/3.5/1.5/above/north/blue,
      y/7/1/-1.5/below/south/blue,
      t/2/1/1.5/above/north/blue, 
      z/-2/3.5/-1.5/below/south/blue}
    {
      \node[vertex, nodedecorate, fill=\color!25] (\nodename) at (\x,\y) {$\value$};
      \node [\direction] at (\nodename.\navigate) {$\nodename$};
    }
    \path
    \foreach \startnode/\endnode/\angle in {
      s/y/-25, y/x/0, x/t/30, t/z/15}
    {
      (\startnode) edge[path edge, bend left=\angle] node[] {} (\endnode)
    };
    %edges or lines
    \path
    \foreach \startnode/\endnode/\direction/\angle/\weight in {
      s/t/left/15/6, s/y/left/-25/7, y/z/below/0/9,  t/y/left/0/8, t/x/above/30/5, x/t/above/30/-2, z/x/right/0/7,
      z/s/below/0/2, y/x/below/0/-3, t/z/right/15/-4
    }
    {
      (\startnode) edge[arrowdecorate,bend left=\angle] node[\direction] {$\weight$} (\endnode)
    };
  \end{tikzpicture}
  \caption{Край на процедурата.}
\end{subfigure}


%%% Local Variables: 
%%% mode: latex
%%% TeX-master: "discrete-math"
%%% End: 

% %   \index{Белман-Форд!алгоритъм}
% %   \caption{Алгоритъм на Белман-Форд запазващ минималните пътища}
% %   \label{fig:bellman-ford-graph}
% % \end{figure}


% %% стр. 654
% \begin{problem}
%   Променете алгоритъма на Белман-Форд, така че $\delta(v) = -\infty$ за всеки връх $v$, 
%   за който има отрицателен цикъл по някой път от началния връх $s$ до $v$.
% \end{problem}



%%% Local Variables: 
%%% mode: latex
%%% TeX-master: "discrete-math"
%%% End: 


% \backmatter

% \bibliographystyle{plain}
% \bibliography{discrete-math}

\printindex

\end{document}


%%% Local Variables: 
%%% mode: latex
%%% TeX-master: "discrete-math"
%%% End: 

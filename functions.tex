\chapter{Функции}
\index{функция}

\section{Основни свойства}

Релацията $R \subseteq A\times B$ се нарича {\bf тотална функция}\index{тотална функция} от $A$ в $B$, ако
\begin{enumerate}[i)]
\item
  $Dom(R) = A$, т.е.
  \[(\forall a\in A)(\exists b\in B)[(a,b)\in R].\]
\item
  За всеки елемент $a\in A$ съотвества {\em точно един} елемент $b \in B$, т.е.
  \[(\forall a\in A)(\forall b_1,b_2 \in B)((\langle{a,b_1}\rangle\in R\ \wedge\ \langle{a,b_2}\rangle\in R) \rightarrow b_1 = b_2).\]
\end{enumerate}
Обикновено означаваме функциите като $f:A\to B$ и
вместо $(a,b)\in f$ пишем $f(a) = b$.
Казваме, че функцията $f$ e
\begin{itemize}
\item
  \marginpar{или $f$ е {\bf обратима}}
  {\bf инекция}\index{функция!инекция}, ако 
  \[(\forall a_1,a_2\in A)[a_1\neq a_2 \rightarrow f(a_1)\neq f(a_2)],\]
  или еквивалентно,
  \[(\forall a_1,a_2\in A)[f(a_1) = f(a_2) \rightarrow a_1 = a_2].\]
\item
  \marginpar{или $f$ е {\bf върху} $B$}
  {\bf сюрекция}\index{функция!сюрекция}, ако 
  \[(\forall b\in B)(\exists a\in A)[f(a) = b].\]
\item
  {\bf биекция}\index{функция!биекция}, ако е инекция и сюрекция.
\end{itemize}

\begin{problem}
  Дайте примери за функция $f:\mathbb{N}\rightarrow\Z$, която е:
  \begin{enumerate}[a)]
  \item
    инективна;
  \item
    сюрективна;
  \item
    нито инективна, нито сюрективна;
  \item
    инективна, но не е сюрективна;
  \item
    сюрективна, но не е инективна;
  \item
    сюрективна и инективна.
  \end{enumerate}
\end{problem}


Да разгледаме функцията $f \subseteq (\Real\times\Real)\times\Real$, дефинирана като:
\[f(x,y) = \frac{x}{y}.\]
Както много добре знаем, за стойности от вида $(x,0)$, функцията $f$ не е дефинирана.
Такива функции ще наричаме {\bf частични}.
Официалната дефиниция е следната.
\marginpar{Тук нямаме условието $Dom(R) = A$}
Релацията $R \subseteq A\times B$ се нарича {\bf частична функция}\index{частична функция} от $A$ в $B$, ако
за всеки елемент $a\in A$ съотвества {\em най-много един} елемент $b \in B$, т.е.
\[(\forall x\in A)(\forall y_1,y_2 \in B)[(\langle{x,y_1}\rangle\in R\ \wedge\ \langle{x,y_2}\rangle\in R) \rightarrow y_1 = y_2].\]

\begin{problem}
  За всяка от следните тотални функции $f$ определете дали $f$ е
  инекция, сюрекция или биекция.
  \begin{enumerate}[a)]
  \item
    \marginpar{(биекция)}
    $f: \Real\rightarrow \Real$, $f(x) = 2x+3$.
  \item
    $f: \Real\rightarrow \Real$, $f(x) = x^2 - 4x +2$.
  \item 
    $f: \Real\rightarrow \Real$, $f(x) = x^3+7$.    
  \item
    $f: \Nat\rightarrow \Nat$, 
    $f(x) = 
      \begin{cases}
        x+1, & \mbox{ ако }x\mbox{ е четно}\\
        x-1, & \mbox{ ако }x\mbox{ е нечетно}\\
      \end{cases}$
    \item
    \marginpar{$rem(x,3)$ - остатък при деление на $3$}
    $f: \Nat\rightarrow \Nat$, $f(x) = rem(x,3)$.
  \item 
    \marginpar{НОД - най-голям общ делител}
    $f: \Nat\times\Nat \rightarrow \Nat$,
    $f(x, y) = \mbox{ НОД}(x,y)$.
  \item 
    \marginpar{НОК - най-малко общо кратно}
    $f: \Nat\times\Nat \rightarrow \Nat$,
    $f(x, y) = \mbox{ НОК}(x,y)$.
  \item 
    $f: \Nat \times \Nat\rightarrow \Nat$,
    $f(x, y) = 3x+2y$.
  \item 
    $f: \Nat \times \Nat\rightarrow \Nat$,
    $f(x, y) = 2^x(2y+1)-1$.
  \item 
    \marginpar{Канторово кодиране}
    $f: \Nat \times \Nat\rightarrow \Nat$,
    $f(x, y) = \frac{(x+y)(x+y+1)}{2} + x$.
  \item 
    $f: \Nat \times \Nat\rightarrow \Nat$,
    $f(x, y) = 2x(2y+1)$.
  \item
    $f: \Nat \times \Nat\rightarrow \Nat$,
    $f(x, y) = 2x(2^y+1)$.
  \item
    $f: \Nat \times \Nat\rightarrow \Nat$,
    $f(x, y) = 2^x3^y$.
  \item
    $f: \Nat \times \Nat\rightarrow \Nat$,
    $f(x, y) = 2^x6^y$.
  \item 
    $f: \Real\times\Real\rightarrow \Real$,
    $f(x, y) = x^2+y^2$.
  \end{enumerate}
\end{problem}

\begin{problem}
  Докажете:
  \begin{enumerate}[a)]
  \item
    Ако $f,g$ са функции, то $f\cap g$ е функция;
  \item
    Нека $f,g$ са функции и $(\forall x)[x\in Dom(f)\cap Dom(g)\rightarrow f(x) = g(x)]$.
    Докажете, че $f\cup g$ е функция.
  \end{enumerate}
\end{problem}

\section{Операции върху функции}


\subsection*{Образ и първообраз}
Нека е дадена функцията $f:A\to B$.
Ще разгледаме няколко основни операции върху функции.

\begin{itemize}
\item 
  {\bf Образ на множеството} $X\subseteq A$ под действието на функцията $f$, наричаме
  множеството: \[f(X) = \{b\in B \mid f(a) = b\ \wedge\ a \in X\}.\]
\item
  {\bf Първообраз на множеството} $Y\subseteq B$ под действието на функцията $f$, наричаме
  множеството: \[f^{-1}(Y) = \{a\in A \mid f(a) = b\ \wedge\ b \in Y\}.\]
\end{itemize}


\begin{example}
  Нека $f:\Real\to\Real$ е дефинирана като $f(x) = \abs{x+1}$.
  Тогава имаме, че:  
  \begin{enumerate}[a)]
  \item
    $f([-1,1)) = \{\abs{x+1} \mid -1 \leq x < 1\} = [0, 2)$;
  \item
    $f^{-1}([-1,1)) = \{x\in\Real\mid -1 \leq \abs{x+1} < 1\} = [-1,0)$;
  \item
    $f^{-1}([-1,0)) = \{x \in \Real \mid -1 \leq \abs{x+1} < 0\} = \emptyset$;
  \item
    $f([-1,0)) = \{\abs{x+1} \mid -1 \leq x < 0\} = [0, 1)$;
  \item
    $f^{-1}([0,2)) = \{x \in \Real \mid 0 \leq \abs{x+1} < 2\} = (-3, 1)$.
  \end{enumerate}
  Да положим $A = [-1,1)$. 
  От б) и г) виждаме, че $f(f^{-1}(A)) \neq A$, а от 
  а) и д) виждаме, че $f^{-1}(f(A)) \neq A$.
\end{example}

\begin{problem}
  Нека е дадена произволна функция $f:A \to B$.
  Проверете:
  \begin{enumerate}[a)]
  \item
    $(\forall X,Y \subseteq B)[f^{-1}(X\cup Y) = f^{-1}(X)\cup f^{-1}(Y)]$;
  \item
    $(\forall X,Y \subseteq B)[f^{-1}(X\cap Y) = f^{-1}(X)\cap f^{-1}(Y)]$;
  \item
    $(\forall X,Y \subseteq B)[f^{-1}(X\backslash Y) = f^{-1}(X)\backslash f^{-1}(Y)]$;
  \item
    $(\forall X\subseteq A)(\forall Y\subseteq B)[f(X)\cap Y = f(X\cap f^{-1}(Y))]$;
  \item
    $(\forall X \subseteq A)(\forall Y \subseteq B)[f(X)\cap Y = \emptyset \iff  X~\cap~f^{-1}(Y)~=~\emptyset]$;
  \item
    $(\forall X \subseteq A)(\forall Y \subseteq B)[f(X)\subseteq Y \iff X\subseteq f^{-1}(Y)]$;
  \item
    $(\forall X,Y \subseteq A)[f(X)\cup f(Y) = f(X\cup Y)]$;
  \item
    $f(\bigcup_{i\in I}X_i) = \bigcup_{i\in I}(X_i)$;
  \item
    при какви условия за $f$,
    $(\forall X\subseteq A)[X =  f^{-1}(f(X))]$;
  \item
    при какви условия за $f$,
    $(\forall Y \subseteq B)[Y = f(f^{-1}(Y))]$;
  \item
    при какви условия за $f$,
    $(\forall X,Y \subseteq A)[f(X)\backslash f(Y) = f(X\backslash Y)]$.
  \end{enumerate}
\end{problem}

\subsection*{Обратна функция}

За всяка биективна функция $f:A\to B$, определяме нейната обратна функция $g:B \to  A$ като:
\[(\forall a \in A)(\forall b \in Ran(f))[g(b) = a\ \iff\ f(a) = b].\]
Обикновено означаваме $g$ като $f^{-1}$.

\subsection*{Композиция}

Нека са дадени функциите $f:A\to B$ и $g:C\to A$.
{\em Композиция} на $f$ и $g$ е функцията $f\circ g: C \to B$ определена като
\[f\circ g = \{\pair{c,b}\mid (\exists a\in A)[g(c) = a\ \wedge\ f(a) = b]\}.\]
\marginpar{Най-напред прилагаме $g$ и след това $f$}
Композицията на $f$ и $g$ може да се запише и така:
\[(\forall c\in C)[(f\circ g)(c) = f(g(c))]\]

\begin{example}
  Нека $f(x) = 2x+1$, $g(x) = x^2$. Тогава:
  \begin{itemize}
  \item 
    $(f\circ g)(x) = 2x^2 + 1$;
  \item
    $(g\circ f)(x) = (2x+1)^2$.
  \end{itemize}
\end{example}

\begin{example}
  \marginpar{Среща се студенти да пишат, че $f^{-1}(x) = \frac{1}{x}$, което е безумно!}
  Ако дефинираме $f:A\to A$ като $f(x) = x$, 
  то за $f^{-1}:A\to A$ имаме също $f^{-1}(x) = x$.
\end{example}

% \begin{enumerate}[I)]
% \item
%   {\bf Образ}
  

% \item
%   {\bf Първообраз}


% \item
%   {\bf Обратна функция}

% \item 
%   {\bf Рестрикция}

%   Нека $X\subseteq A$. {\em Рестрикция} на $f$ до множеството $X$, наричаме
%   множеството: \[f\upharpoonright X = \{\langle{x,y}\rangle\mid f(x) = y\ \wedge\ x\in X\} =  f\cap X\times B.\]
% \item
%   {\bf Затваряне}
  
%   Нека в този случай $f:A\to A$ и нека $X\subseteq A$.
%   За всяко $n \geq 0$ определяме $X_0 = X$ и $X_{n+1} = X_n \cup f(X_n)$.
%   {\em Затваряне} на множеството $X$ относно функцията $f$ е множеството
%   \[f[X] = \bigcup_{n\in\Nat} X_n. \]
  
% \item
%   {\bf Композиция}


% \end{enumerate}




\begin{problem}
  Нека $f: A\to B$, $g: B\to C$ са функции.
  Вярно ли е, че:
  \begin{enumerate}[a)]
  \item 
    Ако $f$ не е инекция, то $g\circ f$ не е инекция?
  \item
    Ако $g$ не е инекция, то $g\circ f$ не е инекция?
  \item 
    Ако $f$ не е сюрекция, то $g\circ f$ не е сюрекция?
  \item
    Ако $g$ не е сюрекция, то $g\circ f$ не е сюрекция?
  \item
    \marginpar{Да}
    $f,g$ са инективни, то $g\circ f$ е инективна?
  \item
    \marginpar{Да}
    $f,g$ са сюрективни, то $g\circ f$ е сюрективна?
  \item
    \marginpar{Да}
    $f,g$ са биективни, то $g\circ f$ е биективна?
  \item
    $g\circ f$ е сюрективна,  то $f,g$ са сюрективни ?
  \item
    $g\circ f$ е инективна, то $f,g$ са инективни ?
  \end{enumerate}
\end{problem}


\begin{problem}
  Нека а дадена произволна функция $f:A \to B$.
  Проверете:
  \begin{enumerate}[a)]
  % \item
  %   $(\forall X,Y \subseteq A)[f(X)\cup f(Y) = f(X\cup Y)]$;
  % \item
  %   $f(\bigcup_{i\in I}X_i) = \bigcup_{i\in I}(X_i)$
  % \item
  %   при какви условия за $f$,
  %   $(\forall X,Y \subseteq A)[f(X\cap Y) = f(X)\cap f(Y)]$.
  % \item
  %   $f(\bigcap_{i\in I}A_i) \subseteq \bigcap_{i\in I}f(A_i)$
  % \item
  %   при какви условия за $f$,
  %   $(\forall X,Y \subseteq A)[f(X)\backslash f(Y) = f(X\backslash Y)]$.
  \item
    \marginpar{Опр. $(\forall x\in X)\ id_X(x) = x$}
    при какви условия за $f$, $f\circ f^{-1} = id_{B}$.
  \item
     при какви условия за $f$, $f^{-1}\circ f = id_{A}$.
   % \item
     % $Dom(f\circ g) = g^{-1}(Dom(f))$, където $g$ е функция.
  % \item
  %   $(\forall X,Y \subseteq B)[f^{-1}(X\cup Y) = f^{-1}(X)\cup f^{-1}(Y)$.
  % \item
  %   $(\forall X,Y \subseteq B)[f^{-1}(X\cap Y) = f^{-1}(X)\cap f^{-1}(Y)$.
  % \item
  %   $(\forall X,Y \subseteq B)[f^{-1}(X\backslash Y) = f^{-1}(X)\backslash f^{-1}(Y)]$.
  % \item
  %   при какви условия за $f$,
  %   $(\forall X\subseteq A)[X =  f^{-1}(f(X))]$.
  % \item
  %   при какви условия за $f$,
  %   $(\forall Y \subseteq B)[Y = f(f^{-1}(Y))]$.
  % \item
  %   $(\forall X\subseteq A)(\forall Y\subseteq B)[f(X)\cap Y = f(X\cap f^{-1}(Y))]$.
  % \item
  %   $(\forall X \subseteq A)(\forall Y \subseteq B)[f(X)\cap Y = \emptyset \iff X\cap f^{-1}(Y) = \emptyset]$.
  % \item
  %   $(\forall X \subseteq A)(\forall Y \subseteq B)[f(X)\subseteq Y \iff X\subseteq f^{-1}(Y)]$.
  \end{enumerate}
\end{problem}

\begin{problem}%Л.М. 18 / 23
  Нека $f,g$ са функции. При какви условия :
  \begin{enumerate}
  \item
    $f^{-1}$ е функция?
  \item
    $f\circ g$ е инективна функция?
  \end{enumerate}
\end{problem}

% \begin{problem}
%   Дайте примери за функция $f:\mathbb{N}\rightarrow\Z$, която е:
%   \begin{enumerate}
%   \item
%     нито инективна, нито сюрективна;
%   \item
%     инективна, но не е сюрективна;
%   \item
%     сюрективна, но не е инективна;
%   \item
%     сюрективна и инективна.
% \end{enumerate}
% \end{problem}


\begin{problem}
  Нека е дадена релацията $R\subseteq A\times B$.
  Докажете, че $R$ е биективна функция тогава и само тогава, когато $R\circ R^{-1} = id_A$ и $R^{-1}\circ R = id_B$.
\end{problem}

\begin{problem}
  Нека $f$ е инективна функция от $A$ в $B$ и $g:\Ps(A) \rightarrow \Ps(B)$, дефинирана като 
  \[(\forall X \subseteq A)[g(X) = f(X)].\]
  Докажете, че $g$ е инективна.
\end{problem}

\begin{problem}
  Нека $f:A\rightarrow B$ и $g:B\rightarrow\Ps(A)$, дефинирана като 
  \[(\forall b \in B)[g(b) = \{x\in A\mid f(x) = b\}].\]
  Докажете, че ако $f$ е сюрективна, то $g$ е инективна.
  Вярна ли е обратната посока?
\end{problem}

\begin{problem}
  Нека $\Fs = \{f| f:A\to B\}$ и $R$ е бинарна релация върху $B$.
  Да разгледаме бинарната релация $S$ върху $\Fs$:
  \[S = \{\pair{f,g}\in \Fs\times \Fs \mid (\forall a\in A)[\pair{f(a),g(a)} \in R]\}.\]
  Докажете или дайте контра-пример за следните твърдения:
  \begin{enumerate}[a)]
  \item 
    ако $R$ е рефликсивна, то $S$ е рефликсивна?
  \item 
    ако $R$ е симетрична, то $S$ е симетрична?
  \item 
    ако $R$ е транзитивна, то $S$ е транзитивна?
  \end{enumerate}
\end{problem}

% p. 243
\begin{problem}
  Нека $A = \Real \setminus \{1\}$.
  Да рагледаме функцията $f: A \to A$ определена като
  \[f(x) = \frac{x+1}{x-1}.\]
  Докажете, че:
  \begin{enumerate}[a)]
  \item 
    $f$ е инективна и сюрективна;
  \item
    $f\circ f = id_A$.
  \end{enumerate}
\end{problem}

\begin{problem}
  Нека $A = \Real \setminus \{0\}$ и $B = \Real \setminus \{2\}$.
  Да разгледаме $f:A \to B$, където
  \[f(x) = \frac{1}{x} + 2.\]
  \marginpar{$f^{-1}(x) = \frac{1}{x-2}$}
  Проверете дали $f$ е инективна и сюрективна.
  Ако да, то намерете $f^{-1}$.
\end{problem}

\begin{problem}
  Нека $A = \Real$ и $B = \{x \in \Real \mid x \geq 0\}$.
  Да разгледаме $f:A \to B$, където
  \[f(x) = x^2.\]
  \marginpar{$f^{-1}$ не е инективна}
  Проверете дали $f$ е инективна и сюрективна.
  Ако да, то намерете $f^{-1}$.
\end{problem}
\begin{hint}
  За да бъде една функция $g$ обратната на $f$,
  трабва да имаме свойствата $g\circ f = id_A$ и $f \circ g = id_B$.
  Не е достатъчно да имаме само едното.
  Ако вземем $g(x) = \sqrt{x}$,  то 
  $g(f(-3)) = \sqrt{(-3)^2} = 3 \neq -3$.
\end{hint}

\begin{problem}
  Нека $f:\Real \to \Real$ е определена като
  \[f(x) = \frac{2x+5}{3}.\]
  Покажете, че $f$ е инективна и сюрективна.
  Намерете $f^{-1}$.
\end{problem}

\begin{problem}
  Нека $f:\Real \to \Real$ е определена като 
  \[f = \frac{x+7}{5}.\]
  \begin{enumerate}[a)]
  \item 
    Докажете, че $f$ е биективна и намерете $f^{-1}$.
  \item
    Покажете, че $f\circ f^{-1} = id_\Real$ и $f^{-1}\circ f = id_\Real$.
  \end{enumerate}
  Да разгледаме сега следните функции:
  \[f_1(x) = x+7, \quad f_2(x) = \frac{x}{5}.\]
  Проверете, че:
  \begin{enumerate}[a)]
  \item 
    $f = f_2 \circ f_1$;
  \item
    $f^{-1} = f^{-1}_1 \circ f^{-1}_2$.
  \end{enumerate}
\end{problem}

Предишната задача ни подсказва, че трябва да имаме следното свойство.
\begin{problem}
  Нека $f: A\to B$, $g: B\to C$ са {\em биективни} функции.
  Докажете, че
  \[(g\circ f)^{-1} = f^{-1}\circ g^{-1}.\]
\end{problem}

\begin{problem}
  За дадена функция $f:A \to B$, определяме бинарна релация $R_f$ по следния начин:
  \begin{enumerate}[a)]
  \item
    $R_f = \{\pair{X,Y} \in \Ps(A) \times \Ps(A) \mid f(X) = f(Y)\}$;
  \item
    $R_f = \{\pair{X,Y} \in \Ps(A) \times \Ps(A) \mid f(X) \subseteq f(Y)\}$;
  \item
    $R_f = \{\pair{X,Y} \in \Ps(B) \times \Ps(A) \mid f^{-1}(X) = f^{-1}(Y)\}$;
  \item
    $R_f = \{\pair{X,Y} \in \Ps(B) \times \Ps(A) \mid f^{-1}(X) \subseteq f^{-1}(Y)\}$.
  \end{enumerate}
  В зависимост от това какви свойства има релацията $f$, определете свойствата на $R_f$.
\end{problem}

\begin{problem}
  Нека $A$ е непразно множество и тоталната функция $f:A \to B$.
  Дефинираме бинарната релация $R$ над $A$ като:
  \[R = \{(x,y) \in A\times A\mid f(x) = f(y)\}.\]
  Докажете, че:
  \begin{enumerate}[a)]
  \item
    $R$ е релация на еквивалентност.
  \item
    Определете класовете на еквивалентност на $R$.
\end{enumerate}
\end{problem}

\begin{problem}
  \marginpar{Възможно ли е да са частични функции?}
  Да разгледаме функциите $f:A\to B$ и $g:B\to A$.
  \begin{enumerate}[a)]
  \item 
    Ако $f$ е сюрективна и $g\circ f = id_A$, то докажете, че $g = f^{-1}$.
  \item
    Ако $f$ е инективна и $f\circ g = id_B$, то докажете, че $g = f^{-1}$.
  \end{enumerate}
\end{problem}
\begin{proof}
  \begin{enumerate}[a)]
  \item 
    \marginpar{Да напомним, че $id_A(a) =a$ за всяко $a\in A$}
    Възможно ли е $g$ да не е тотална функция, т.е. да съществува $b\in B$, 
    за което $g(b)$ не е дефинирана ?
    Да допуснем, че съществува такова $b$. 
    Понеже $f$ е сюрективна, съществува поне едно $a \in A$, за което $f(a) = b$.
    Това означава, че $(g\circ f)(a)$ не е дефинирана. Това е противоречие с условието, че $g\circ f = id_A$.
    
    Сега ще проверим, че $f$ е инективна. От това ще следва, че $f^{-1}$ е функция.
    Нека да разгледаме $a, a' \in A$,  за които $f(a) = b = f(a')$. Ще докажем, че $a = a'$.
    Щом $g\circ f = id_A$, то $g(f(a)) = g(b) = a$  и $g(f(a'))= g(b) = a'$.
    Ясно е, че $a = a'$, защото $g$ е функция.

    Вече знаем, че $f$ е биективна и следователно $f^{-1}$ е биективна функция. 
    Понеже $g$ е тотална, достатъчно е да докажем, че $g \subseteq f^{-1}$, т.е.
    \[(\forall a\in A)(\forall b\in B)[g(b) = a \implies f(a) = b].\]
    Нека $g(b) = a$ и $f(a') = b$. Знаем, че такова $a'$ съществува, защото $f$ е сюрективна.
    Но тогава $(g\circ f)(a') = a$ и следователно $a = a'$, защото $g\circ f = id_A$.
  \item    
    \marginpar{Да напомним, че $id_B(b)=b$ за всяко $b\in B$}
    Отново, възможно ли е $g$ да не е тотална функция, т.е. да съществува $b\in B$, за което $g(b)$ не е дефинирана?
    Но тогава е ясно, че $(f\circ g)(b)$ също няма да е дефинирана, което е противоречие с условието, че $f\circ g = id_B$.
    \marginpar{Добре е да отбележим, че понеже $f$ е инективна ние знаем, че $f^{-1}$ е функция.}
    Щом $g$ е тотална, за да докажем, че $g = f^{-1}$ е достатъчно да проверим, че $g\subseteq f^{-1}$, т.е.
    \[(\forall a\in A)(\forall b\in B)[g(b) = a \implies f(a) = b].\]
    \marginpar{Вярно ли е, че $f$ е сюрктивна?}
    Това е лесно да се провери. Нека $g(b) = a$. Понеже $(f\circ g)(b) = b$, то $f(g(b)) = f(a) = b$.
  \end{enumerate}
\end{proof}


% \section{Монотонни функции}



% \cite{hein}

%%% Local Variables: 
%%% mode: latex
%%% TeX-master: "discrete-math"
%%% End: 

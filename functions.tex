\section*{Функции}
\index{функция}

Релацията $R \subseteq A\times B$ се нарича {\bf функция}\index{функция} от $A$ в $B$, ако
\begin{enumerate}[i)]
  \item
    $Dom(R) = A$, т.е.
    \[(\forall a\in A)(\exists b\in B)[(a,b)\in R].\]
  \item
    За всеки елемент $a\in A$ съотвества {\em точно един} елемент $b \in B$, т.е.
    \[(\forall x\in A)(\forall y_1,y_2 \in B)(\langle{x,y_1}\rangle\in R\ \wedge\ \langle{x,y_2}\rangle\in R \rightarrow y_1 = y_2).\]
\end{enumerate}
Обикновено означаваме функциите като $f:A\to B$ и
вместо $(a,b)\in f$ пишем $f(a) = b$.
Казваме, че функцията $f$ e
\begin{itemize}
\item
  \marginpar{също казваме, че $f$ е обратима}
  {\bf инекция}\index{функция!инекция}, ако 
  \[(\forall x_1,x_2\in A)[x_1\neq x_2 \rightarrow f(x_1)\neq f(x_2)],\]
  или еквивалентно,
  \[(\forall x_1,x_2\in A)[f(x_1) = f(x_2) \rightarrow x_1 = x_2].\]
\item
  \marginpar{също казваме, че $f$ е върху $B$}
  {\bf сюрекция}\index{функция!сюрекция}, ако 
  \[(\forall y\in B)(\exists x\in A)[f(x) = y].\]
\item
  {\bf биекция}\index{функция!биекция}, ако е инекция и сюрекция.
\end{itemize}

\begin{problem}
  Дайте примери за функция $f:\mathbb{N}\rightarrow\Z$, която е:
  \begin{enumerate}
  \item
    нито инективна, нито сюрективна;
  \item
    инективна, но не е сюрективна;
  \item
    сюрективна, но не е инективна;
  \item
    сюрективна и инективна.
  \end{enumerate}
\end{problem}

\begin{problem}
  Докажете:
  \begin{enumerate}[a)]
  \item
    Ако $f,g$ са функции, то $f\cap g$ е функция;
  \item
    Нека $f,g$ са функции и $(\forall x)[x\in Dom(f)\cap Dom(g)\rightarrow f(x) = g(x)]$.
    Докажете, че $f\cup g$ е функция.
  \end{enumerate}
\end{problem}

\begin{prb}
  За всяка от следните  функции $f$ определете дали $f$ е
  инекция, сюрекция или биекция.
  \begin{enumerate}[a)]
  \item
    $f: \mathbb{R}\rightarrow \mathbb{R}$, $f(x) = 2x+3$.
  \item
    $f: \mathbb{R}\rightarrow \mathbb{R}$, $f(x) = x^2 - 4x +2$.
  \item 
    $f: \mathbb{R}\rightarrow \mathbb{R}$, $f(x) = x^3+7$.    
  \item
    $f: \mathbb{N}\rightarrow \mathbb{N}$, 
    \begin{align*}
      f(x) = 
      \begin{cases}
        x+1, & \mbox{ ако }x\mbox{ е четно}\\
        x-1, & \mbox{ ако }x\mbox{ е нечетно}\\
      \end{cases}
    \end{align*}
  \item
    \marginpar{$rem(x,3)$ - остатък при деление на $3$}
    $f: \mathbb{N}\rightarrow \mathbb{N}$, $f(x) = rem(x,3)$.
  \item 
    \marginpar{НОД - най-голям общ делител}
    $f: \mathbb{N} \times \mathbb{N}\rightarrow \mathbb{N}$,
    $f(x, y) = \mbox{ НОД}(x,y)$.
  \item 
    $f: \mathbb{N} \times \mathbb{N}\rightarrow \mathbb{N}$,
    $f(x, y) = 3x+2y$.
  \item 
    $f: \Nat \times \Nat\rightarrow \Nat$,
    $f(x, y) = 2^x(2y+1)-1$.
  \item 
    $f: \Nat \times \Nat\rightarrow \Nat$,
    $f(x, y) = 2x(2y+1)$.
  \item
    $f: \Nat \times \Nat\rightarrow \Nat$,
    $f(x, y) = 2x(2^y+1)$.
  \item
    $f: \Nat \times \Nat\rightarrow \Nat$,
    $f(x, y) = 2^x3^y$.
  \item
    $f: \Nat \times \Nat\rightarrow \Nat$,
    $f(x, y) = 2^x6^y$.
  \item 
    $f: \mathbb{R} \times \mathbb{R}\rightarrow \mathbb{R}$,
    $f(x, y) = x^2+y^2$.
  \item
    
  \end{enumerate}
\end{prb}


% \item 
%   $f: \mathbb{Q}\rightarrow \mathbb{Q}$, $f(x) =
%   \cstwo{0}{$x=0$}{\frac{1}{x}}{$x\neq 0$}$.
% \item
%   $f: \mathbb{R} \rightarrow \mathbb{R}$, $f(x) = |x|+1$.
% \item 
%   $f: (-\frac{\pi}{2}, \frac{\pi}{2})\rightarrow \mathbb{R}$,
%   $f(x) = tg x$.
% \item 
%   $f: \mathbb{N} \times \mathbb{N}\rightarrow \mathbb{N}$,
%   $f(x, y) = 3^x.5^y$.
% \item
%   $f: \mathbb{R}^+ \rightarrow \mathbb{N}$, $f(x) = \lfloor x
%   \rfloor$. (най-голямото естествено  число, по-малко или равно на
%   $x$. )
% \item
%   $f: \mathbb{N} \rightarrow \mathbb{N}$, $f(x) = (x+1)$-вото
%   просто число.
% \end{enumerate}


\subsection*{Операции върху функции}

Нека е дадена функцията $f:A\to B$.
Ще разгледаме няколко основни операции върху функции.
\begin{enumerate}[I)]
\item
  {\bf Образ}
  
  Нека $X\subseteq A$. {\em Образ на множеството} $X$ под действието на функцията $f$, наричаме
  множеството: \[f(X) = \{b\in B \mid f(a) = b\ \wedge\ a \in X\}.\]
\item
  {\bf Първообраз}

  Нека $Y\subseteq B$. {\em Първообраз на множеството} $Y$ под действието на функцията $f$, наричаме
  множеството: \[f^{-1}(Y) = \{a\in A \mid f(a) = b\ \wedge\ b \in Y\}.\]
\item
  {\bf Обратна функция}
  За всяка биективна функция $f:A\to B$, определяме нейната обратна функция $g:B \to  A$ като:
  \[(\forall a \in A)(\forall b \in Ran(f))[g(b) = a\ \iff\ f(a) = b].\]
  Обикновено означаваме $g$ като $f^{-1}$.
% \item 
%   {\bf Рестрикция}

%   Нека $X\subseteq A$. {\em Рестрикция} на $f$ до множеството $X$, наричаме
%   множеството: \[f\upharpoonright X = \{\langle{x,y}\rangle\mid f(x) = y\ \wedge\ x\in X\} =  f\cap X\times B.\]
% \item
%   {\bf Затваряне}
  
%   Нека в този случай $f:A\to A$ и нека $X\subseteq A$.
%   За всяко $n \geq 0$ определяме $X_0 = X$ и $X_{n+1} = X_n \cup f(X_n)$.
%   {\em Затваряне} на множеството $X$ относно функцията $f$ е множеството
%   \[f[X] = \bigcup_{n\in\Nat} X_n. \]
  
\item
  {\bf Композиция}

  Нека са дадени функциите $f:A\to B$ и $g:C\to A$.
  {\em Композиция} на $f$ и $g$ е функцията $f\circ g: C \to B$ определена като
  \[f\circ g = \{\pair{c,b}\mid (\exists a\in A)[g(c) = a\ \wedge\ f(a) = b]\}.\]
  \marginpar{Най-напред прилагаме $g$ и след това $f$}
  Композицията на $f$ и $g$ може да се запише и така:
  \[(\forall c\in C)[(f\circ g)(c) = f(g(c))]\]
\end{enumerate}

\begin{prb}
  Нека $f: A\to B$, $g: B\to C$ са функции.
  Вярно ли е, че:
  \begin{enumerate}
  \item 
    Ако $f$  не е инекция, то $g\circ f$ не е инекция?
  \item
    Ако $g$  не е инекция, то $g\circ f$ не е инекция?
  \item 
    Ако $f$  не е сюрекция, то $g\circ f$ не е сюрекция?
  \item
    Ако $g$  не е сюрекция, то $g\circ f$ не е сюрекция?
  \end{enumerate}
\end{prb}

\begin{prb}
  Нека $f: A\to B$, $g: B\to C$ са функции.
  Вярно ли е, че:
  \begin{enumerate}[1)]
  \item
    $f,g$ са инективни, то $g\circ f$ е инективна?
  \item
    $f,g$ са сюрективни, то $g\circ f$ е сюрективна?
  \item
    $f,g$ са биективни, то $g\circ f$ е биективна?
  \item
    $g\circ f$ е сюрективна,  то $f,g$ са сюрективни ?
  \item
    $g\circ f$ е инективна, то $f,g$ са инективни ?
  \end{enumerate}
\end{prb}

\begin{prb}
  Нека $f: A\to B$, $g: B\to C$ са {\em биективни} функции.
  Докажете, че
  \[(g\circ f)^{-1} = f^{-1}\circ g^{-1}.\]
\end{prb}

\begin{prb}
  $f(x) = \pair{g(x),h(x)}$.
\end{prb}


% \begin{dfn}
%   Дефинираме следните операции върху релацията $R\subseteq A\times{B}$:
%   \begin{enumerate}
%   \item
%     Дефиниционна област
%     $Domain(R) = \{x\mid (\exists y)\langle{x,y}\rangle\in R \}$;
%   \item
%     Област от стойности
%     $Range(R) = \{y\mid (\exists x)[\langle{x,y}\rangle\in R]\}$;
%   \item
%     Поле
%     $Field(R) = Domain(R) \cup Range(R)$;
%   \item
%     Рестрикция
%     $R\upharpoonright{C} = \{\langle{x,y}\rangle\mid \langle{x,y}\rangle\in R\ \&\ x\in C\}$;
%   \item
%     Образ
%     $R[C] = \{ y \mid (\exists x)[ x\in C\ \&\ \langle{x,y}\rangle\in R]\} = Range(R\upharpoonright{C})$.


% \end{enumerate}
% \end{dfn}

% \begin{example}
%   Нека да разгледаме релацията \[F = \{\langle{\emptyset, a}\rangle,\langle{\{\emptyset\}, b}\rangle\}.\]
%   Лесно се вижда, че $F$ е функция.
%   Имаме, че \[F^{-1} = \{\langle{a,\emptyset}\rangle,\langle{b, \{\emptyset\}}\rangle\}\] е функция тогава и само тогава, когато  $a\neq b$.
%   Обърнете внимание, че
%   \[F\upharpoonright{\emptyset} = \emptyset \mbox{, но } F\upharpoonright\{\emptyset\} = \{\langle{\emptyset,a}\rangle\}.\]
%   Освен това, $F(\{\emptyset\}) = \{a\}$ и $F(\{\emptyset\}) = b$.
% \end{example}

% Нека $f$ е функция и $A$ е множество.
% \begin{enumerate}[(i)]
% \item
%   $f(A) = \{y \mid (\exists x\in A)(f(x) = y)\}$
% \item
%   $f^{-1}(A) = \{x \mid f(x)\in A\}$
% \end{enumerate}


\begin{problem}
  Нека а дадена произволна функция $f:A \to B$.
  Проверете:
  \begin{enumerate}[a)]
  \item
    $(\forall X,Y \subseteq A)[f(X)\cup f(Y) = f(X\cup Y)]$;
  \item
    $f(\bigcup_{i\in I}X_i) = \bigcup_{i\in I}(X_i)$
  \item
    при какви условия за $f$,
    $(\forall X,Y \subseteq A)[f(X\cap Y) = f(X)\cap f(Y)]$.
  % \item
  %   $f(\bigcap_{i\in I}A_i) \subseteq \bigcap_{i\in I}f(A_i)$
  \item
    при какви условия за $f$,
    $(\forall X,Y \subseteq A)[f(X)\backslash f(Y) = f(X\backslash Y)]$.
  \item
    \marginpar{Опр. $(\forall x\in X)\ id_X(x) = x$}
    при какви условия за $f$, $f\circ f^{-1} = id_{B}$.
  \item
     при какви условия за $f$, $f^{-1}\circ f = id_{A}$.
   % \item
     % $Dom(f\circ g) = g^{-1}(Dom(f))$, където $g$ е функция.
  \item
    $(\forall X,Y \subseteq B)[f^{-1}(X\cup Y) = f^{-1}(X)\cup f^{-1}(Y)$.
  \item
    $(\forall X,Y \subseteq B)[f^{-1}(X\cap Y) = f^{-1}(X)\cap f^{-1}(Y)$.
  \item
    $(\forall X,Y \subseteq B)[f^{-1}(X\backslash Y) = f^{-1}(X)\backslash f^{-1}(Y)]$.
  \item
    при какви условия за $f$,
    $(\forall X\subseteq A)[X =  f^{-1}(f(X))]$.
  \item
    при какви условия за $f$,
    $(\forall Y \subseteq B)[Y = f(f^{-1}(Y))]$.
  \item
    $(\forall X\subseteq A)(\forall Y\subseteq B)[f(X)\cap Y = f(X\cap f^{-1}(Y))]$.
  \item
    $(\forall X \subseteq A)(\forall Y \subseteq B)[f(X)\cap Y = \emptyset \iff X\cap f^{-1}(Y) = \emptyset]$.
  \item
    $(\forall X \subseteq A)(\forall Y \subseteq B)[f(X)\subseteq Y \iff X\subseteq f^{-1}(Y)]$.
  \end{enumerate}
\end{problem}
\newpage
\begin{problem}%Л.М. 18 / 23
  Нека $f,g$ са функции. При какви условия :
  \begin{enumerate}
  \item
    $f^{-1}$ е функция?
  \item
    $f\circ g$ е инективна функция?
  \end{enumerate}
\end{problem}

% \begin{problem}
%   Дайте примери за функция $f:\mathbb{N}\rightarrow\Z$, която е:
%   \begin{enumerate}
%   \item
%     нито инективна, нито сюрективна;
%   \item
%     инективна, но не е сюрективна;
%   \item
%     сюрективна, но не е инективна;
%   \item
%     сюрективна и инективна.
% \end{enumerate}
% \end{problem}


\begin{problem}
  Нека е дадена релацията $R\subseteq A\times B$.
  Докажете, че $R$ е биективна функция тогава и само тогава, когато $R\circ R^{-1} = id_A$ и $R^{-1}\circ R = id_B$.
\end{problem}

\begin{problem}
  Нека $f$ е инективна функция от $A$ в $B$ и $g:\Ps(A) \rightarrow \Ps(B)$, дефинирана като 
  \[(\forall X \subseteq A)[g(X) = f(X)].\]
  Докажете, че $g$ е инективна.
\end{problem}

\begin{problem}
  Нека $f:A\rightarrow B$ и $g:B\rightarrow\Ps(A)$, дефинирана като 
  \[(\forall b \in B)[g(b) = \{x\in A\mid f(x) = b\}].\]
  Докажете, че ако $f$ е сюрективна, то $g$ е инективна.
  Вярна ли е обратната посока?
\end{problem}

\cite{hein}

%%% Local Variables: 
%%% mode: latex
%%% TeX-master: "discrete-math"
%%% End: 

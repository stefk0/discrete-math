
\documentclass[a4paper]{article}
\usepackage{geometry}
\geometry{margin=1in}
\usepackage[english,bulgarian]{babel}
\usepackage{amssymb}
\usepackage{amsmath}
\usepackage{mathrsfs}
\usepackage{latexsym}
\usepackage{amsthm}
\usepackage{enumerate}
\setlength{\parskip}{2.3ex}            % vertical space between paragraphs
\setlength{\parindent}{0in}            % amount of indentation of paragraph
%this package allows for hyperlinks within the pdf document
\usepackage[colorlinks=true, linkcolor=blue,pdfstartview=FitV,
citecolor=green, urlcolor=blue]{hyperref}

\newtheorem{thm}{Theorem}
\newtheorem{lemma}{Lemma}
\newtheorem{corollary}{Corollary}
\newtheorem{prp}{Proposition}
\newtheorem{example}{Example}
\newtheorem{dfn}{Defintion}
\newtheorem{question}{Question}
\newtheorem{remark}{Remark}
\newtheorem{claim}{Claim}
\newtheorem{problem}{Задача}
\newcommand{\A}{\mathfrak{A}}
\newcommand{\B}{\mathfrak{B}}
\renewcommand{\C}{\mathfrak{C}}
\newcommand{\D}{\mathfrak{D}}

\newcommand{\Ls}{\mathscr{L}}
\newcommand{\Fs}{\mathscr{F}}
\newcommand{\Rs}{\mathscr{R}}
\newcommand{\As}{\mathscr{A}}
\newcommand{\Bs}{\mathscr{B}}
\newcommand{\Is}{\mathscr{I}}
\newcommand{\Ss}{\mathscr{S}}
\newcommand{\Ps}{\mathscr{P}}

\newcommand{\xn}{x_{1},\dots,x_{n}}

\newcommand{\xs}{\overline{x}}
\newcommand{\ys}{\overline{y}}
\newcommand{\zs}{\overline{z}}
\newcommand{\forces}{\Vdash}
\renewcommand{\iff}{\leftrightarrow}
\begin{document}
\author{Stefan Vatev}

\begin{thm}
  Ако $G$ е дърво, то $\varepsilon = \nu - 1$.
\end{thm}

\begin{corollary}
  Всяко нетривиално дърво има поне два върха със степен 1.
\end{corollary}

\begin{problem}
  Покажете, че ако $G$ е дърво и има връх със степен $\geq k$, то $G$ има поне $k$ върха със степен 1.
\end{problem}

\begin{problem}
  Нека $G$ е свързан граф.
  Докажете, че всеки два най-дълги пътя в свързан граф имат общ връх.
\end{problem}

\begin{problem}
  За $G$ прост граф, докажете, че $\varepsilon = \binom{\nu}{2}$ т.с.т.к. $G$ е пълен.
\end{problem}

\begin{problem}
  Докажете, че в граф с $\nu\geq 2$, има поне два върха с еднаква степен.
\end{problem}

\begin{problem}
  Докажете, че:
  \begin{enumerate}
  \item
    във всеки неориентиран граф броят на върховете с нечетна степен е четен;
  \item
    всеки регулярен граф с нечетна степен има четен брой върхове;
  \item
    всеки граф с $\varepsilon > \binom{\nu-1}{2}$ е свързан.
    Дайде пример за несвързан граф с $\varepsilon = \binom{\nu-1}{2}$.
  \item
    във граф всички върхове имат степен поне $d$.
    Докажете, че в графа има път с дължина $d$.
  \end{enumerate}
\end{problem}


\begin{problem} % зад. 1.22
  Да разгледаме графа $G$ (без примки и без кратни ребра) със $s$ компоненти на свързаност.
  Докажете, че $\nu - s \leq \varepsilon \leq \binom{\nu-s+1}{2}$.
\end{problem}

\begin{problem}
  Нега $G$ е граф с $n$ върха и в $G$ няма прост цикъл с дължина 3.
  Докажете, че $G$ има най-много $\lfloor{\frac{n^2}{4}}\rfloor$ ребра.
\end{problem}

\begin{problem}
  Нека $G$ е произволен граф без примки и кратни ребра, а $\overline{G}$ е неговото допълнение.
  Докажете, че поне един от графите $G$, $\overline{G}$ е свързан;
\end{problem}


\begin{problem}
  \begin{enumerate}
  \item
    Да се построят всички неизоморфни графи на 1,2,3 и 4 върха.
  \item
    Намерете броя на ребрата на граф без цикли с $n$ върха и $k$ компоненти.
  \end{enumerate}
\end{problem}
\end{document}


%%% Local Variables: 
%%% mode: latex
%%% TeX-master: t
%%% End: 

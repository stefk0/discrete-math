\chapter{Булеви функции}

Да припомним таблицата за истинност на някои от основните булеви функции на два аргумента.
\marginpar{$x\oplus y$ - симетрична разлика}

\begin{tabular}{|c|c|c|c|c|c|c|c|c|c|}
  \hline
  $x$ & $y$ & $\overline{x}$ & $x \vee y$ & $xy$ & $x \rightarrow y$ & $\overline{x}\vee y$ & $x \iff y$ & $x \oplus y$ & $x\overline{y} \vee \overline{x}y$\\
  \hline
  \hline
  0 & 0 & 1 & 0 & 0 & 1 & 1 & 1 & 0 & 0 \\
  \hline
  0 & 1 & 1 & 1 & 0 & 1 & 1 & 0 & 1 & 1 \\
  \hline
  1 & 0 & 0 & 1 & 0 & 0 & 0 & 0 & 1 & 1 \\
  \hline
  1 & 1 & 0 & 1 & 1 & 1 & 1 & 1 & 0 & 0 \\
  \hline
\end{tabular}

Когато две булеви формули $\varphi$ и $\psi$ са тъждествено еквивалентни (т.е. имат еднакви стълбове), то ще пишем $\varphi \equiv \psi$.

\section{Основни свойства}

\marginpar{Често вместо $x\wedge y$ пишем $x\cdot y$ или $xy$. Също така, вместо $\neg x$ пишем $\overline{x}$}
\begin{enumerate}[1)]
\item
  Комутативни свойства
  \[xy \equiv yx,\quad x\vee y \equiv y\vee x,\quad x\oplus y \equiv y\oplus x\]
\item
  Асоциативни свойства
  \[(xy)z \equiv x(yz),\quad (x\vee y)\vee z \equiv x\vee (y\vee z),\quad (x\oplus y)\oplus z \equiv x\oplus (y\oplus z)\]
\item
  Лесно се проверява с таблиците за истинност, че:
  \[x\oplus y \equiv x\ov{y}\vee \ov{x}y \equiv (x\vee y)(\ov{x}\vee\ov{y})\]
\item
  Свойства на отрицанието
  \[x\ov{x} \equiv 0, \quad x\vee\ov{x} \equiv x\vee 1,\quad x\oplus\ov{x} \equiv 1\]
\item
  Закон за двойното отрицание
  \[\ov{\ov{x}} \equiv x\]
\item
  Свойства на константите
  \[x\cdot 0 \equiv 0, \quad x\cdot 1 \equiv x,\quad x\vee 0 \equiv x,\quad x\vee 1 \equiv 1,\quad x\oplus 0 \equiv x, \quad x\oplus 1 \equiv \ov{x}\]
\item
  Дистрибутивни свойства
  \begin{enumerate}[]
  \item
    $x(y\vee z) \equiv xy \vee xz$,
  \item
    $xy \vee z \equiv (x\vee z)(y\vee z)$,
  \item
    $(x\oplus y)z \equiv xz \oplus yz$.
  \end{enumerate}
\item
  Идемпотентентни свойства
  \[xx \equiv x, \quad x\vee x \equiv x\]
\item
  Свойства на отрицанието
  \[x\ov{x} \equiv 0, \quad x\vee\ov{x} \equiv 1, \quad x\oplus\ov{x} \equiv 1\]
\item
  Закони на Де Морган
  \[\ov{xy} \equiv \ov{x}\vee\ov{y}, \quad \ov{x\vee y} \equiv \ov{x}\cdot\ov{y}\]
\end{enumerate}

\begin{problem}
  \marginpar{\cite[стр. 30]{gavrilov}}
  Проверете еквивалентни ли са формулите $\varphi$ и $\psi$ като използвате еквивалентни преобразования на формулите.
  \begin{enumerate}[a)]
  \item
    $\varphi = (x\oplus yz)\rightarrow (\overline{x}\rightarrow (y\rightarrow z))$,
    $\psi = x\rightarrow ((y\rightarrow z)\rightarrow x)$;
  \item
    $\varphi = (\overline{x}\vee \overline{y}.z)\rightarrow ((x\rightarrow y)\rightarrow (y\vee z)\rightarrow\overline{x})$,
    $\psi = (x\rightarrow y)\rightarrow(\overline{y}\rightarrow\overline{x})$;
  \item
    $\varphi = (x.\overline{y}\vee \overline{x}.z)\oplus ((y\rightarrow z)\rightarrow \overline{x}.y)$,
    $\psi = (x.(\overline{y}.\overline{z})\oplus y)\oplus z$;
  \item
    $\varphi = x\rightarrow ((\ov{x}.\ov{y}\rightarrow(\ov{x}.\ov{z}\rightarrow y))\rightarrow y).z$,
    $\psi = \ov{x.(y\rightarrow\ov{z})}$.
  \item
    $\varphi = \ov{((x\vee y) \rightarrow y.z)\vee (y\rightarrow x.z)} \vee (x\rightarrow (\ov{y}\rightarrow z))$,
    $\psi = (x\rightarrow y)\vee z$.
  \end{enumerate}
\end{problem}
\begin{solution}
  \begin{enumerate}[a)]
  \item
    Ще направим еквивалентни преобразувания върху двете формули докато получим един и същ резултат.
    \begin{align*}
      \psi =\ &  x\rightarrow ((y\rightarrow z)\rightarrow x)\\
      \equiv\ & \overline{x}\vee (\overline{y\rightarrow z}\vee x)\  \equiv\ 1\\
      \varphi =\ & (x\oplus yz)\rightarrow (\overline{x}\rightarrow (y\rightarrow z))\\
      \equiv\ & \overline{(x\vee yz)(\overline{x}\vee\overline{yz})} \vee x\vee \overline{y}\vee z\\
      \equiv\ & \overline{(x\vee yz)}\vee\overline{(\overline{x}\vee\overline{yz})} \vee x\vee \overline{y}\vee z\\
      \equiv\ & \overline{x}.\overline{yz} \vee xyz \vee x\vee \overline{y}\vee z \equiv\ \overline{x}(\overline{y}\vee\overline{z}) \vee x\vee \overline{y}\vee z\\
      \equiv\ & \overline{x}.\overline{y} \vee \overline{x}.\overline{z} \vee x\vee \overline{y}\vee z \equiv\ \overline{x}.\overline{z} \vee x\vee \overline{y}\vee z\\
      \equiv\ & \overline{(x\vee z)} \vee (x\vee z)\vee \overline{y} \equiv 1 \vee \ov{y} \equiv 1.
    \end{align*}
  \item[в)]
    Правим отново същото.
    \begin{align*}
      \psi =\ & (x.(\overline{y}.\overline{z})\oplus y)\oplus z\\
      \equiv\ & x(y\oplus 1)(z\oplus 1) \oplus y \oplus z\\
      \equiv\ & xyz \oplus xy \oplus xz \oplus x \oplus y \oplus z\\
      \varphi =\ & (x\overline{y}\vee \overline{x}z)\oplus ((y\rightarrow z)\rightarrow \overline{x}y)\\
      \equiv\ & (x\ov{y}\vee \ov{x}z) \oplus (\ov{\ov{y}\vee z} \vee \ov{x}y)\\
      \equiv\ & x\ov{y}\oplus \ov{x}z \oplus (y\ov{z} \oplus \ov{x}y) \\
      \equiv\ & x\ov{y}\oplus \ov{x}z\oplus \ov{x}y\ov{z} \oplus y\ov{z} \oplus \ov{x}y \\
      \equiv\ & xy \oplus x \oplus xz \oplus z \oplus (x\oplus 1)y(z\oplus 1) \oplus yz\oplus y \oplus xy \oplus y \\
      \equiv\ & x \oplus xz \oplus z \oplus (x\oplus 1)y(z\oplus 1) \oplus yz\oplus  \\
      \equiv\ & x \oplus xz \oplus z \oplus xyz \oplus yz \oplus xy \oplus y \oplus yz\oplus\\
      \equiv\ & xyz \oplus xy \oplus xz \oplus x \oplus y \oplus z.
     \end{align*}
\end{enumerate}
\end{solution}

\section{Дизюнктивна нормална форма}

\begin{itemize}
\item
  \index{конюнкт}
  {\bf Конюнкт} на променливите $x_1,x_2,\dots,x_n$ представлява съждителна формула от вида 
  \[x^{\sigma_1}_1x^{\sigma_2}_2 \cdots x^{\sigma_n}_n,\]
  където $x^{\sigma_i}_i = x_i$, ако $\sigma_i = 1$ и $x^{\sigma_i}_i = \overline{x}_i$, ако $\sigma_i = 0$.
\item
  \index{дизюнктивна нормална форма}
  Една съждителна формула $\Phi(x_1,\dots,x_n)$ е в {\bf дизюнктивна нормална форма (ДНФ)}, ако
  тя представлява дизюнкция от конюнкти на някои от променливите на $\Phi$.
  Например, формулата 
  \[\Phi(x,y,z) = \ov{x}y \vee z\ov{y}\]
  е в дизюнктивна формална форма.
\item
  \index{съвършена дизюнктивна нормална форма}
  Една съждителна формула $\Phi(x_1,\dots,x_n)$ е в {\bf съвършена дизюнктивна нормална форма (СДНФ)}, ако
  тя е е в ДНФ и всеки конюнкт участват всичките променливи $x_1,\dots,x_n$.
  За една булева функция $f(x_1,\dots,x_n)$, можем да намерим формула $\Phi(x_1,\dots,x_n)$ в СДНФ еквивалентна на нея по следния начин:
  \[\Phi(x_1,\dots,x_n) = \bigvee_{\stackrel{(\sigma_1\dots \sigma_n) \in \{0,1\}^n}{f(\sigma_1, \dots \sigma_n) = 1}}x_1^{\sigma_1}x_2^{\sigma_2}\dots x_n^{\sigma_n}.\]
\end{itemize}

\begin{problem}
  \marginpar{\cite[стр. 50]{gavrilov}}
  С помощта на еквивалентни преобразувания постройте ДНФ на булевите функции
  \begin{enumerate}[a)]
  \item
    \marginpar{$xy\overline{z} \vee \overline{x}z \vee \overline{y}z$}
    $f(x,y,z) = (\ov{x}\vee\ov{y}\vee\ov{z})\cdot(xy\vee z)$;
  \item
    \marginpar{$\overline{x}y\overline{z} \vee xyz \vee x\overline{y}z$}
    $f(x,y,z) = (\overline{x}y\oplus z)\cdot(xz\rightarrow y)$;
  \item
    $f(x,y,z) = (x\vee y\overline{z})\cdot(x\ov{y}\vee\ov{z})\cdot(\ov{xy}\vee z)$;
  \item
    $f(x,y,z,t) = (x\vee y\ov{z}.\ov{t})((\ov{x}\vee t)\oplus yz)\vee \ov{y}\cdot(z\vee \ov{x\ov{t}})$;
  \item
    $f(x,y,z,t) = (x\rightarrow y).(y\rightarrow \ov{z}).(z\rightarrow x\ov{t})$;
  \end{enumerate}
\end{problem}

\begin{problem}% Гаврилов, стр. 50, 2.12
  По дадена ДНФ на булевата функция $f$ постройте нейната СДНФ.
  \begin{enumerate}[1)]
  \item
    \marginpar{}
    $f(x,y,z) = xy\vee\ov{z}$;
  \item
    \marginpar{}
    $f(x,y,z) = \ov{x}.\ov{y} \vee y\ov{z} \vee z\ov{z}$;
  \item
    $f(x,y,z) = x\vee yz \vee \ov{x}.\ov{z}$;
  \item
    $f(x,y,z) = x\vee \ov{y}\vee \ov{x}z$;
  \item
    $f(x,y,z,t) = xy\ov{z} \vee xz\ov{t}$;
  \item
    $f(x,y,z,t) = xy \vee \ov{y}t \vee z\ov{t}$.
  \end{enumerate}
\end{problem}

\begin{problem}
  Представете в СДНФ следните булеви функции:
  \begin{enumerate}[1)]
  \item
    $f(x,y,z) = (x\vee y)\rightarrow z$;
  \item
    $f(x,y,z) = (01010001)$;
  \item
    $f(x,y,z) = (11001010)$;
  \item
    $f(x,y,z,t) = (x\rightarrow yzt)(z\rightarrow x\ov{y})$;
  \item
    $f(x,y,z,t) = (x\oplus y)(z\rightarrow \ov{y}t)$;
  \end{enumerate}
\end{problem}


\section{Класовете $T_0$ и $T_1$}
\index{$T_0, T_1$}

\begin{itemize}
\item 
  Нека $c\in\{0,1\}$. 
  Казваме, че булевата функция $f(\xn)$ запазва константата $c$, ако $f(c,c,\dots,c) = c$.
\item
  Означаваме с $T_0$ функциите, които запазват константата $0$ и с $T_1$ тези, които запазват константата $1$.
\item
  С $T^n_0$ и $T^n_1$ означаваме тези функции, които са на $n$ променливи и принадлежат на $T_0$ или $T_1$ съответно.
\end{itemize}

\begin{problem}% Гаврилов, стр. 73
  Принадлежи ли функцията $f$ на множеството $T_1 \setminus T_0$ ?
  \begin{enumerate}[a)]
  \item
    \marginpar{Да}
    $f(x,y,z) = (x\rightarrow y)(y\rightarrow z)(z\rightarrow x)$;
  \item
    \marginpar{Да}
    $f(x,y,z) = x\rightarrow(y\rightarrow (z\rightarrow x))$;
  \item
    \marginpar{Да}
    $f(x,y,z) = xyz \vee \ov{x}y \vee \ov{y}$;
  \end{enumerate}
\end{problem}

\begin{problem}
  При какви $n$ функцията $f(x_1,\dots, x_n)$ принадлежи на $T_0\setminus T_1$?
  \begin{enumerate}[1)]
  \item
    $f(\xn) = x_1\oplus x_2 \oplus\dots\oplus x_n$;
  \item
    $f(\xn) = (\bigoplus^{n-1}_{i=1} x_ix_{i+1})\oplus x_nx_1$;
  \end{enumerate}
\end{problem}

\begin{prop}
  Класовете $T_0$ и $T_1$ са затворени, т.е. $[T_0] = T_0$ и $[T_1] = T_1$.
\end{prop}


\section{Самодвойнствени булеви функции}
\index{булева функция!самодвойнствена}
\begin{itemize}
\item 
  Нека е дадена булевата функция $f(\xn)$. Дефинираме булевата функция $f^\star(\xn)$ като
  \[f^\star(\xn) = \overline{f}(\overline{x}_1,\dots,\overline{x}_n).\]
\item
  Ще наричаме $f^\star$ {\bf двойнствена} функция на $f$.
\item
  Ако $f = f^\star$, то ще наричаме $f$ {\bf самодвойнствена} функция.
\item
  Ще означаваме с $S$ множеството от всички самодвойнствени булеви функции, а с $S^n$ тези на $n$ променливи.
\end{itemize}

\begin{prop}
  Класът на самодвойнствените функции е затворен, т.е. $[S] = S$.
  Освен това, $S \subsetneqq \Fs_2$.
\end{prop}


\begin{problem} %% Гаврилов, стр. 31, зад. 1.25
  Проверете дали функцията $g$ е двойнствена на $f$.
  \begin{enumerate}[1)]
  \item
    \marginpar{Да}
    $f(x,y) = x\rightarrow y$, $g(x,y) = \overline{x}.y$;
  \item
    \marginpar{Не}
    $f(x,y) = (\overline{x}\rightarrow\overline{y})\rightarrow(y\rightarrow x)$, \\
    $g(x,y) = (x\rightarrow y).(\overline{y}\rightarrow\overline{x})$;
  \item
    \marginpar{Да}
    $f(x,y,z) = xy \rightarrow z$,\\
    $g(x,y,z) = \overline{x}.\overline{y}.z$;
  \item
    $f(x,y,z,t) = (x\vee y\vee z).t\vee x.y.z$, \\
    $g(x,y,z,t) = (x\vee y\vee z).t\vee x.y.z$;
  \item
    $f(x,y,z,t) = xy\vee yz\vee zt\vee tx$, \\
    $g(x,y,z,t) = xz\vee yt$;
  \item
    $f(x,y,z,t) = (x\rightarrow y).(z\rightarrow t)$, \\
    $g(x,y,z,t) = (x\rightarrow\overline{z}).(x\rightarrow t).(\overline{y}\rightarrow\overline{z}).(\overline{y}\rightarrow t)$.
  \end{enumerate}
\end{problem}

\begin{problem}
  Проверете самодвойнствена ли е $f$.
  \begin{enumerate}[a)]
  \item
    \marginpar{Не}
    $f(x,y) = x\vee y$;
  \item
    \marginpar{Не}
    $f(x,y) = x\rightarrow y$;
  \item
    \marginpar{Не}
    $f(x,y) = x\oplus y$;
  \item
    \marginpar{Да}
    $f_4(x,y,z) = xy\vee yz\vee zx$;
  \item
    \marginpar{Да}
    $f_5(x,y,z) = x\oplus y\oplus z\oplus 1$;
  \item
    \marginpar{Да}
    $f_6(x,y,z) = xyz\oplus xy\ov{z}\oplus yz\oplus xz$.
  \item
    \marginpar{Не}
    $f_7(x,y,z) = xyz\oplus xy\oplus yz\oplus xz$;
  \item
    \marginpar{Не}
    $f(x,y,z) = (x\rightarrow y)\oplus (y\rightarrow z)\oplus (y\rightarrow x)$;
  \item
    \marginpar{Не}
    $f(x,y,z) = (x\rightarrow y)\oplus (y\rightarrow z)\oplus (z\rightarrow x)\oplus z$;
  \end{enumerate}
\end{problem}
\begin{proof}
  \begin{table}[H]
    \begin{subtable}{0.5\textwidth}
      \begin{tabular}[b]{|c||c|c|c|}
        \hline
        $xz$ & а) & б) & в)\\
        \hline
        $00$ & $0$ & $1$ & $0$ \\
        \hline
        $01$ & $1$ & $1$ & $1$\\
        \hline
        \hline
        $10$ & $1$ & $0$ & $1$\\
        \hline
        $11$ & $1$ & $1$ & $0$\\
        \hline
      \end{tabular}
    \end{subtable}
    \begin{subtable}{0.5\textwidth}
      \begin{tabular}[b]{|c||c|c|c|c|c|}
        \hline
        $xyz$ & г) & д) & е) & ж) & з)\\
        \hline
        $000$ & $0$ & $1$ & $0$ & $0$ & $1$\\
        \hline
        $001$ & $0$ & $0$ & $0$ & $0$ & $1$\\
        \hline
        $010$ & $0$ & $0$ & $0$ & $0$ & $1$\\
        \hline
        $011$ & $1$ & $1$ & $1$ & $1$ & $0$\\
        \hline
        \hline
        $100$ & $0$ & $0$ & $0$ & $0$ & $0$\\
        \hline
        $101$ & $1$ & $1$ & $1$ & $1$ & $0$\\
        \hline
        $110$ & $1$ & $1$ & $1$ & $1$ & $0$\\
        \hline
        $111$ & $1$ & $0$ & $1$ & $0$ & $1$\\
        \hline
      \end{tabular}
    \end{subtable}
  \end{table}
\end{proof}


\begin{problem}
  Проверете дали функцията $f$ е самодвойнствена, ако е зададена векторно:
  \begin{enumerate}[1)]
  \item
    \marginpar{Да}
    $\alpha_f = (01001101)$;
  \item
    \marginpar{Не}
    $\alpha_f = (01100110)$;
  \item
    $\alpha_f = (1100 1001 0110 1100)$;
  \item
    $\alpha_f = (1110 0111 0001 1000)$;
  \item
    $\alpha_f = (1100 0011 0011 1100)$;
  \item
    $\alpha_f = (1001 0110 1001 0110)$;
  \item
    $\alpha_f = (1100 0011 1010 0101)$;
  \end{enumerate}
\end{problem}

\begin{problem}
  Заменете $-$ в $\chi_f$ с $0$ или $1$ за да получите характеристичен вектор на самодвойнствена функция.\\
  \begin{inparaenum}[a)]
  \item
    $\chi_f = (1-0-)$;
  \item
    $\chi_f = (01-0-0--)$;
  \item
    $\chi_f = (--01--11)$;
  \end{inparaenum}
\end{problem}

\section{Полином на Жегалкин}
\index{полином на Жегалкин}
\begin{itemize}
\item 
  Полином на Жегалкин на 2 променливи е формула от вида:
  \[a_0\oplus a_1x_1\oplus a_2x_2  \oplus a_{12}x_1x_2  ,\]
  където $a_0,a_1,a_2,a_{12}$ приемат стойности 0 или 1.
\item
  Полином на Жегалкин на 3 променливи е формула от вида:
  \[a_0\oplus a_1x_1\oplus a_2x_2 \oplus a_3x_3 \oplus a_{12}x_1x_2 \oplus a_{13}x_1x_3 \oplus a_{23} x_2x_3 \oplus a_{123}x_1x_2x_3,\]  
  където $a_0,a_1\dots,a_{123}$ приемат стойности 0 или 1.
\item
  Полином на Жегалкин на $n$ променливи е формула от вида:
  \[a_0 \oplus \bigoplus_{1\leq i\leq n}a_i x_i\oplus \bigoplus_{1\leq i<j \leq n}a_{ij} x_ix_j\oplus \bigoplus_{1\leq i<j<k \leq n}a_{ijk} x_ix_jx_k \dots  \oplus a_{12\dots n} x_1x_2\dots x_n,\]
\end{itemize}

\begin{thm}
  Всяка булева функция има единствен полином на Жегалкин.
\end{thm}
\begin{hint}
  Всеки полином на Жегалкин представя различна булева функция.
  Всички полиноми на Жегалкин на $n$ променливи са $2^{2^n}$.
  Всички булеви функции на $n$ променливи са $2^{2^n}$.
\end{hint}


\begin{problem}
  По метода на неопределените коефициенти, намерете полинома на Жегалкин на функцията 
  \begin{enumerate}[a)]
  \item
    $f(x,y) = x\vee y$;
  \item
    $f(x,y,z) = x\vee y \vee z$;
  \item
    $f(x,y,z) = x\rightarrow (y \rightarrow z)$;
  \item
    $f(x,y,z) = x(y\vee\overline{z})$.
  \end{enumerate}
\end{problem}
\begin{proof}
  \begin{enumerate}[a)]
  \item
    Понеже общият вид на булевата функция е $f(x,y) = a_0\oplus a_1 x \oplus a_2 y \oplus a_3 xy $,
    трябва да намерим коефициентите $a_0,a_1,a_2,a_3$.
    \begin{align*}
      & a_0\oplus a_1 0 \oplus a_2 0 \oplus a_3 0 = 0 \vee 0 = 0\\
      & a_0\oplus a_1 1 \oplus a_2 0 \oplus a_3 0 = 1 \vee 0 = 1\\
      & a_0\oplus a_1 0 \oplus a_2 1 \oplus a_3 0 = 0 \vee 1 = 1\\
      & a_0\oplus a_1 1 \oplus a_2 1 \oplus a_3 1 = 1 \vee 1 = 1.
    \end{align*}
    Следователно, $x\vee y \equiv x\oplus y\oplus xy$.
  \end{enumerate}
\end{proof}

\begin{problem}
  Използвайки еквивалентности от вида $\overline{A} = A\oplus 1$ и $A\vee B = AB\oplus A\oplus B$, 
  намерете полинома на Жегалкин на функцията:
  \begin{enumerate}[a)]
  \item
    \marginpar{$1 \oplus x \oplus xy$}
    $f(x,y) = x\rightarrow y$;
  \item
    \marginpar{$1 \oplus xy \oplus xyz$}
    $f(x,y,z) = (x\rightarrow (y\rightarrow z))$;
  \item
    \marginpar{$x\oplus z\oplus xy\oplus xz \oplus xyz$}
    $f(x,y,z) = ((x\rightarrow y)\rightarrow z)$;
  \item
    \marginpar{$x\oplus z\oplus xy\oplus xz \oplus xyz$}
    $f(x,y,z) = (x\rightarrow (y\rightarrow z)).((x\rightarrow y)\rightarrow z)$;
  \item
    $f(x,y,z,t) = (x\rightarrow y)\rightarrow (z\rightarrow xt)$;
  \item
    $f(x,y,z,t) = x\vee (y\rightarrow ((z\rightarrow y)\rightarrow t)$;
  \item
    $f(x,y,z,t) = (x\vee y\vee z)t \vee xyz$.
  \end{enumerate}
\end{problem}

\section{Линейни функции}
\index{булева функция!линейна}

\begin{itemize}
\item
  Знаем, че всяка булева функция може да се представи {\em по единствен начин}
  с полином на Жегалкин.
\item
  Всяка булева функция $f(\xn)$ с полином на Жегалкин от вида 
  \[a_0\oplus a_1x_1 \oplus a_2x_2 \dots\oplus a_nx_n\] наричаме {\bf линейна}.
\item
  Ще означаваме с $L$ множеството от всички линейни булеви функции, а с $L^n$ тези на $n$ променливи.
\end{itemize}

\begin{problem}
  \marginpar{Отг. $2^n$}
  Колко са всички линейни булеви функции на $n$ променливи?
\end{problem}


\begin{prop}
  Класът на линейните функции е затворен, т.е. $[L] = L$.
  Освен това, $L \subsetneqq \Fs_2$.
\end{prop}


\begin{problem}
  Линейна ли е функцията $f$ с характеристичен вектор $\chi_f = (1001011010010110)$?
\end{problem}

\begin{problem}
  Заменете $-$ в $\chi_f = (-110---0)$ с $0$ или $1$, така че да получите $f$ линейна.
\end{problem}


\begin{problem}
  Проверете дали $f$ е линейна функция.
  \begin{enumerate}
  \item
    \marginpar{Не}
    $f = x\rightarrow y$;
  \item
    \marginpar{Да}
    $f = \ov{x\rightarrow y}\oplus \ov{x}y$;
  \item
    \marginpar{Не}
    $f = xy\vee \ov{x}.\ov{y}\vee z$;
  \item
    \marginpar{Не}
    $f = xy\ov{z}\vee x\ov{y}$;
  \item
    \marginpar{Да}
    $f = (x\vee yz)\oplus xyz$;
  \item
    $f = (x\vee yz)\oplus \ov{x}yz$;
  \item
    $\chi_f = (1100 0011)$;
  \item
    $\chi_f = (1001 0110 0110 1001)$;
  \end{enumerate}
\end{problem}

\begin{problem}
  Заменете $-$ в $\chi_f$ с $0$ или $1$, така че да получите $f$ линейна.
  \begin{enumerate}[a)]
  \item
    $\chi_f = (10-1)$;
  \item
    $\chi_f = (100-0---)$;
  \item
    $\chi_f = (-001--1-)$;
  \item
    $\chi_f = (11-0---1)$;
  \item
    $\chi_f = (-0-1--00)$;
  \item
    $\chi_f = (--10----0--1-110)$;
  \end{enumerate}
\end{problem}
\begin{proof}
  а) $(1001)$; б) $f = 1\oplus x \oplus y\oplus z$; в) $f = 1\oplus x\oplus y\oplus z$ ;
  г) $f = 1\oplus x\oplus y$; д) $f = x\oplus y$;
\end{proof}


\section{Монотонни функции}
\index{булева функция!монотонна}
\begin{itemize}
\item 
  Нека $\alpha = (a_1,a_1,\dots,a_n)$ и $\beta = (b_1,b_2,\dots,b_n)$ са два булеви вектора с равна дължина.
  \marginpar{Това не е лексикографската наредба!}
  Дефинираме релацията $\preceq$ между тях по следния начин.
  \[\alpha \preceq \beta \iff \abs{\alpha} = \abs{\beta}\wedge (\forall i \leq \abs{\alpha})[a_i \leq b_i].\]
  Ето няколко примера:
  \begin{itemize}
  \item 
    $(0,1,0) \preceq (0,1,1)$;
  \item
    $(0,1,0) \not\preceq (1,0,1)$;
  \item
    $(1,0,1) \not\preceq (0,1,0)$.
  \end{itemize}
\item
  Булевата фунция $f(\xn)$ наричаме {\bf монотонна}, ако 
  \[(\forall \alpha,\beta\in J^n_2 )[\alpha\preceq\beta \rightarrow f(\alpha) \leq f(\beta)].\]  
\item
  Ще означаваме с $M$ множеството от всички монотонни булеви функции, а с $M^n$ тези на $n$ променливи.
\end{itemize}

\begin{prop}
  Класът на монотонните функции е затворен, т.е. $[M] = M$.
  Освен това, $M \subsetneqq \Fs_2$.
\end{prop}

\begin{problem}
  Проверете монотонни ли са функциите:
  \begin{enumerate}[a)]
  \item
    \marginpar{Да}
    $f(x,y) = x\rightarrow (y\rightarrow x)$;
  \item
    \marginpar{Не}
    $f(x,y) = x\rightarrow (x\rightarrow y)$;
  \item
    \marginpar{Да}
    $f(x,y) = (x\oplus y)xy$;
  \item
    \marginpar{Да}
    $f(x,y,z) = xy\oplus yz \oplus zx$;
  \item
    \marginpar{Не}
    $f(x,y,z) = xy\oplus yz \oplus zx \oplus x$;
  \end{enumerate}
\end{problem}

\begin{problem}
  За немонотонните функции $f$, намерете съседни $\alpha$, $\beta$, такива че
  $\alpha \prec \beta$ и $f(\alpha) > f(\beta)$.
  \begin{enumerate}[a)]
  \item
    \marginpar{Отг. $\alpha = (010)$, $\beta = (110)$}
    $f = xyz \vee \ov{x}y$;
  \item
    \marginpar{Отг. $\alpha = (010)$, $\beta = (110)$}
    $f = x\oplus y\oplus z$;
  \item
    % \marginpar{Отг. $\alpha = (110)$, $\beta = (111)$}
    $f = xy\oplus z$;
  \item
     % \marginpar{Отг. $\alpha = (010)$, $\beta = (011)$}
    $f = x\vee y\ov{z}$;
  \item
    % \marginpar{Отг. $\alpha = (0111)$, $\beta = (1111)$}
    $f = xz\oplus yt$;
  \item
    % \marginpar{Отг. $\alpha = (1110)$, $\beta = (1111)$}
    $f(x,y,z,t) = (xyt\rightarrow yz)\oplus t$;
  \end{enumerate}
\end{problem}
% \begin{proof}
%   \begin{enumerate}[a)]
%   \item
%     $\alpha = (010)$, $\beta = (110)$;
%   \item
%     $\alpha = (010)$, $\beta = (110)$;
%   \item
%     $\alpha = (110)$, $\beta = (111)$;
%   \item
%     $\alpha = (010)$, $\beta = (011)$;
%   \item
%     $\alpha = (0111)$, $\beta = (1111)$;
%   \item
%     $\alpha = (1110)$, $\beta = (1111)$;
%   \end{enumerate}
% \end{proof}


\section{Пълнота и затворени класове}

\begin{itemize}
\item 
  Нека $F\subseteq \Fs_2$ е множество от булеви функции.
  С индукция дефинираме следната редица за всяко $n \in \Nat$:
  \begin{align*}
    F_0 = & F\cup \{I^m_k \mid m,k\in\Nat, 1\leq k\leq m\}\\
    F_{n+1} = & F_n\ \cup\\
    & \{h \mid (\exists f, g_1\dots g_m \in F_n)[h(x_1\dots x_k) =  f(g_1(x_1\dots x_k), \dots, g_m(x_1\dots x_k)]\},
  \end{align*}
  Затварянето на $F$ по отношение на суперпозиция наричаме множеството:
  \[[F] = \bigcup_{n\in \mathbb{N}}F_n.\]
\end{itemize}

% Така множеството $[F]$ се задава с индукция от базово множество
% $F\cup \{I^n_k \mid 1\leq k\leq n \}$ по правилото суперпозиция. От
% Тема 1, знаем, че това е най-малкото множество $X$, което съдържа
% базовото множество и е затворено относно суперпозиция.


\begin{dfn}
  \index{пълно множество}
  Нека $F\subseteq \Fs_2$ е множество от булеви функции. 
  $F$ е {\bf пълно} множество, ако $[F] = \Fs_2$.
  Това означава, че всяка булева функция може да се представи като суперпозиция на функции от множеството $F$.
  $F$ се нарича базис, ако не съществува $G \subsetneqq F$, за което $[G] = \Fs_2$.
\end{dfn}

\begin{thm}[Бул]
  Множеството $\{x\vee y,\ov{x},x\wedge y\}$ е пълно.
\end{thm}
\begin{hint}
  Ще докажем, че за всяка булева функция $f \in \Fs_2$ е изпълнено, че
  $f \in [\{x\vee y, \ov{x}, x \wedge y\}]$.
  Ще разгледаме два случая.
  \begin{itemize}
  \item 
    Нека $f = \bf{0}$. Тогава $f(x_1,\dots,x_n) \equiv x_1\wedge\ov{x}_1$.
  \item
    Нека $f \neq \bf{0}$. Тогава лесно се съобразява, че
    \marginpar{Да напомним, че $x^1 = x$, $x^0 = \ov{x}$}
    \[f(x_1,\dots,x_n) \equiv \bigvee_{\stackrel{a_1,\dots,a_n:}{f(a_1,\dots,a_n) = 1}} x^{a_1}_1\dots x^{a_n}_n.\]
  \end{itemize}
\end{hint}

\begin{example}
  Нека да фиксираме булевите функции $c(x,y) = x\wedge y$, $d(x,y) = x\vee y$, $n(x) = \ov{x}$.
  Булевата функция \[f(x,y,z) = \ov{\ov{x}y \vee z}\] се изразява чрез суперпозиция на функциите $c(x,y)$, $d(x,y)$ и $n(x)$
  по следния начин:
  \[f(x,y,z) \equiv n(d(c(n(x),y),z)).\]
\end{example}


\begin{framed}
  \begin{thm}[Критерий за пълнота на Пост-Яблонский]
    \index{Пост-Яблонский}
    Нека $P\subseteq \Fs_2$ е непразно множество от булеви функции. Множеството $P$ е {\em пълно} тогава и само тогава, когато то не е подмножество на 
    нито едно от множествата $T_0,T_1,S,M,L$.
  \end{thm}
\end{framed}

% \begin{proof}
%   Първо, нека $P$ е пълно множество, т.е. $[P] = \Fs_2$.
%   Ако допуснем, че съществува $K \in \{T_0,T_1,S,M,L\}$, за което $P \subseteq K$, то
%   \[\Fs_2 = [P] \subseteq [K] = K,\]
%   от което следва, че $\Fs_2 = K$, което е противоречие.
  
%   Нека сега имаме, че за всяко $K \in \{T_0,T_1,S,M,L\}$, $P \not\subseteq K$.
%   Ще докажем, че $\{\ov{x}, x\vee y\} \subseteq [P]$. Тогава, от теоремата на Бул $\Fs_2 = [\{\ov{x}, x\vee y\}] \subseteq [P]$,
%   ще следва, че $[P] = \Fs_2$.
%   Да фиксираме функциите:
%   \[f_0 \in P\setminus T_0,\ f_1 \in P \setminus T_1,\ f_S \in P\setminus S,\ f_M \in P\setminus M,\ f_L \in P\setminus L.\]
  
%   Можем да приемем, че функциите $f_0$ и $f_1$ са едноаргументни,
%   защото ако например $f_0$ е $n$-аргументна, то ще вземем 
%   $g_0(x) = f_0(x,\dots,x) = f_0(I^1_1(x),\dots,I^1_1(x)) \in [P]$.
%   Имаме, че $f_0(0) = 1$ и $f_1(1) = 0$. 
%   Ще разгледаме четири случая за другите стойности на аргументите:
%   \begin{enumerate}[1)]
%   \item 
%     ако $f_0(1) = 0$ и $f_1(0) = 0$, тогава $f_0 \equiv \ov{x}$, $f_1 \equiv {\bf 0}$ и $f_0(f_1(x)) = 1$;
%   \item
%     ако $f_0(1) = 0$ и $f_1(0) = 1$, тогава $f_0 \equiv f_1\equiv \ov{x}$;
%   \item
%     ако $f_0(1) = 1$ и $f_1(0) = 0$, тогава $f_0 \equiv {\bf 1}$ и $f_1 \equiv {\bf 0}$;
%   \item
%     ако $f_0(1) = 1$ и $f_1(0) = 1$, тогава $f_0 \equiv {\bf 1}$, $f_1 \equiv \ov{x}$ и $f_1(f_0(x)) = 0$.
%   \end{enumerate}
%   В случаите $1)$, $3)$, $4)$ получихме, че ${\bf 0}, {\bf 1} \in [P]$.
%   Ще докажем, че и в случая $2)$ също имаме, че ${\bf 0}, {\bf 1} \in [P]$.
%   За целта ще трябва да разгледаме и функцията $f_S \not\in S$.
%   За нея знаем, че съществуват стойности на аргументите ѝ $a_1,\dots,a_n$, такива че:
%   \[f_S(a_1,\dots,a_n) = f_S(\ov{a}_1,\dots,\ov{a}_n).\]
%   Да разгледаме едноместната функция $g_S$, дефинирана като:
%   \[g_S(x) = f_S(x^{a_1},\dots,x^{a_n}) \in [P].\]
%   \begin{align*}
%     g_S(0) & = f_S(0^{a_1},\dots,0^{a_n}) & (0^0 = 1, 0^1 = 0)\\
%     & = f_S(\ov{a}_1,\dots,\ov{a}_n) = f_S(a_1,\dots,a_n) & (\text{защото } f_S\not\in S)\\
%     & = f_S(1^{a_1},\dots,1^{a_n}) & (1^0 = 0, 1^1 = 1)\\
%     & = g_S(1).
%   \end{align*}
%   Следователно $g_S$ е константа функция. 
%   Понеже сме в случая $2)$, то $\ov{x} \in [P]$ и следователно $\ov{g}_S$ е другата константна функция.
%   Заключаваме, че във всичките четири случая получаваме, че ${\bf 0}, {\bf 1} \in [P]$.
  
%   Сега ще докажем, че имаме $\ov{x} \in [P]$.
%   Да разгледаме $f_M \not\in M$. Това означава, че съществуват вектори $\alpha  \prec \beta$, които се различават
%   само в една стойност и $f_M(\alpha) = 1$, $f_M(\beta) = 0$.
%   Имаме, че  $\alpha = (a_1,\dots,a_{k-1},0,a_{k+1},\dots,a_n)$ и $\beta = (a_1,\dots,a_{k-1},1,a_{k+1},\dots,a_n)$.
%   Да разгледаме едноместната булева функция $g_M$, дефинираната като:
%   \[g_M(x) = f_M(a_1,\dots,a_{k-1},x,a_{k+1},\dots,a_n).\]
%   Понеже вече доказахме, че ${\bf 0},{\bf 1} \in [P]$, то $g_M \in [P]$.
%   Лесно се вижда, че $g_M(x) = \ov{x}$. 

%   Остава да докажем, че $xy \in [P]$.
%   Да разгледаме $f_L \not\in L$. Това означава, че $f_L$ има следния вид:
%   \[f_L(x_1,\dots,x_n) = a_0 \oplus a_1x_1\oplus a_2x_2 \dots\oplus a_n x_n\oplus \dots \oplus a_lx_{i_1}x_{i_2}\dots x_{i_k}\oplus\dots \]
%   Нека $x_{i_1}x_{i_2}\dots x_{i_k}$ е първият нелинеен член в записа на $f_L$.
%   Да разгледаме двуместната булева функция $g_L$, дефинирана като:
%   \[g_L(x_{i_1},x_{i_2}) = a_0 \oplus a_{i_1}x_{i_1} \oplus a_{i_2}x_{i_2} \oplus x_{i_1}x_{i_2},\]
%   т.е. тя е получена от $f_L$ като $x_{j} = 1$ за $j \in \{i_3,\dots,i_k\}$ и $x_j = 0$ за $j \not\in\{i_1,i_2,\dots,i_k\}$.
  
%   Можем да приемем, че $a_0 = 0$, защото ако $a_0 = 1$, то ще разгледаме функцията $g^\prime_L(x_{i_1},x_{i_2}) = \ov{g}_L(x_{i_1},x_{i_2}) \in [P]$.
%   За нея, $g^\prime(0,0) = a_0 = 0$.
%   И така, нека $g_L$ има вида
%   \[g_L(x,y) = ax\oplus by \oplus xy.\]
%   Отново трябва да разгледаме четири случая за стойностите на $a$ и $b$.
%   \begin{itemize}
%   \item 
%     ако $a = 0$, $b = 0$, тогава $g_L(x,y) = xy$;
%   \item 
%     ако $a = 0$, $b = 1$, тогава $g_L(x,y) = y \oplus xy$ и 
%     \[g_L(\ov{x},y) = y\oplus (x\oplus 1)y = y \oplus y \oplus xy = xy;\]
%   \item 
%     ако $a = 1$, $b = 0$, тогава $g_L(x,y) = x\oplus xy$ и
%     \[g_L(x, \ov{y}) = x\oplus x(y\oplus 1) = xy;\]
%   \item 
%     ако $a = 1$, $b = 1$, тогава $g_L(x,y) = x \oplus y \oplus xy$ и
%     \[\ov{g}_L(\ov{x},\ov{y}) = xy.\]
%   \end{itemize}
%   Във всички случаи получихме, че $xy \in [P]$.
% \end{proof}

\begin{example}
  Да проверим дали следното множество от булеви функции е пълно
  $A = \{xy, x\vee y, x\oplus y\oplus z\oplus 1\}$.
  Пресмятаме за всяка от функциите на кои от петте класа принадлежат:
  \newline
  \newline
  \begin{tabular}[b]{|c|c|c|c|c|c|}
    \hline
    & $T_0$ & $T_1$ & $L$ & $S$ & $M$\\
    \hline
    $xy$ & $+$ & $+$ & $-$ & $-$ & $+$\\
    \hline
    $x\vee y$ & $+$ & $+$ & $-$ & $-$ & $+$\\
    \hline
    $x\oplus y\oplus z\oplus 1$ & $-$ & $-$ & $+$ & $+$ & $-$ \\
    \hline
  \end{tabular}
  \newline
  
  От таблицата имаме, че:
  \begin{itemize}
  \item 
    $x \oplus y \oplus z \oplus 1 \not\in T_0$. Следователно, $A \not\subseteq T_0$.
  \item 
    $x \oplus y \oplus z \oplus 1 \not\in T_1$. Следователно, $A \not\subseteq T_1$.
  \item
    $xy \not\in L$. Следователно, $A \not\subseteq L$.
  \item
    $xy \not\in S$. Следователно, $A \not\subseteq S$.
  \item
    $x \oplus y \oplus z \oplus 1 \not\in M$. Следователно, $A \not\subseteq M$.
  \end{itemize}
  Според критерия на Пост-Яблонский, множеството $A$ е пълно.
\end{example}


\begin{problem} %Гаврилов, стр. 83, зад. 6.1
  Пълна ли е системата от функции?
  \begin{enumerate}[a)]
  \item
    $A = \{1, xy(x\oplus z)\}$;
  \item
    $A = \{x\rightarrow y, x\oplus y\}$;
  \item
    $A = \{0, \ov{x}, x(y\oplus z)\oplus yz\}$;
  \item
    $A = \{x\rightarrow y, \ov{x}\rightarrow \ov{y}x, x\oplus y\oplus z, 1\}$;
  \item
    $A = \{\ov{y}\rightarrow\ov{x}, \ov{x}\rightarrow \ov{y}x, x\oplus y\oplus z, 0\}$;
  \item
    $A = \{\ov{y}\rightarrow \ov{x}z, (y\vee \ov{x} \rightarrow x, x\oplus y\oplus z, 1\}$;
  \item
    $A = \{x\oplus z \oplus 1, x \to \ov{y}, x\oplus (y \vee z) \oplus 1\}$;
  \item
    $A = \{1,\ov{x}, x(y\leftrightarrow z)\oplus\ov{x}(y\oplus z), x\leftrightarrow y\}$;
  \item
    $A = \{\ov{x}, x(y\leftrightarrow z) \leftrightarrow yz, x \oplus y \oplus z\}$;
  \item
    $A = \{\ov{x}, x(y\leftrightarrow z) \leftrightarrow (y\vee z), x \oplus y \oplus z\}$;
  \item
    $A = \{\chi_{f_1} = (0110), \chi_{f_2} = (1100 0011), \chi_{f_3} = (1001 0110)\}$;
  \item
    $A = \{\chi_{f_1} = (11), \chi_{f_2} = (00), \chi_{f_3} = (0011 0101)\}$;
  \end{enumerate}
\end{problem}
\begin{solution}
  \begin{figure}[H]
    \begin{subfigure}[b]{0.5\textwidth}
      \begin{tabular}[b]{|c|c|c|c|c|c|}
        \hline
        & $T_0$ & $T_1$ & $L$ & $S$ & $M$\\
        \hline
        $1$ & $-$ & $+$ & $+$ & $-$ & $+$\\
        \hline
        $xy(x\oplus z)$ & $+$ & $-$ & $-$ & $-$ & $-$\\
        \hline
      \end{tabular}
      \caption{}      
    \end{subfigure}
    \begin{subfigure}[b]{0.5\textwidth}
      \begin{tabular}[b]{|c|c|c|c|c|c|}
        \hline
        & $T_0$ & $T_1$ & $L$ & $S$ & $M$\\
        \hline
        $x\rightarrow y$ & $-$ & $+$ & $-$ & $-$ & $+$\\
        \hline
        $x\oplus y$ & $+$ & $-$ & $+$ & $-$ & $-$\\
        \hline
      \end{tabular}
      \caption{}
    \end{subfigure}
  \end{figure}
\end{solution}

\begin{problem} % Гаврилов, стр. 83
  Проверете пълно ли е множеството от булеви функции:
  \begin{enumerate}[a)]
  \item
    \marginpar{Да}
    $A = (S\cap M)\cup(L\setminus M)$;
  \item
    $A = ((L\cap M)\setminus T_1)\cup (S\cap T_1)$;
  \item
    $A = (L\cap M)\cup (S\setminus T_0)$;
  \item
    $A = (L\cap T_1)\cup (S\cap M)$;
  \item
    $A = (M\setminus S)\cup(L\cap S)$;
  \item
    $A = (M\setminus T_0)\cup (L\setminus S)$;
  \item
    $A = (M\setminus T_0) \cup (S\setminus L)$.
  \end{enumerate}
\end{problem}
\begin{solution}
  Във всяка една от задачите трябва да проверим дали
  $A \not\subseteq T_0$, 
  $A \not\subseteq T_1$, 
  $A \not\subseteq S$, 
  $A \not\subseteq M$ и 
  $A \not\subseteq L$.
  \begin{enumerate}[a)]
  \item 
    Нека $A = (S \cap M) \cup (L\setminus M)$.
    \begin{itemize}
    \item 
      Да разгледаме функцията 
      $f(x) = x \oplus 1$.
      Лесно се съобразява, че $f \in L\setminus M$, откъдето  следва, че $f \in A$.
      Обаче ние имаме, че $f \not\in T_0, T_1, M$.
      Следователно, $A \not\subseteq T_0, T_1, M$.
    \item
      Да разгледаме $g(x,y) = x\oplus y \oplus 1$.
      За нея имаме, че $g \in A$, защото $g \in L\setminus M$ и освен това $g \not\in S$.
    \item
      Остана да проверим, че $A \not\subseteq L$.
      Да разгледаме 
      \[h(x,y,z) = xy\oplus yz \oplus xz.\]
      Тогава $h \in A$, защото $h \in S\cap M$.
      Ясно е, че $h \not\in L$.
    \end{itemize}
  \item
    Нека $A = ((L\cap M)\setminus T_1)\cup (S\cap T_1)$.
    \begin{itemize}
    \item 
      Ако една функция е монотонна, но не запазва $1$-цата, то тогава 
      със сигурност тази функция е константата $0$, т.е.
      \[(L \cap M)\setminus T_1 = \{0\}.\]
      % $0 \not\in S$ и оттук $A \not\subseteq S$.
    \item
      Ако $f \in S \cap T_1$, то това означава $f(1,\dots,1) = 1$ и $f(0,\dots,0) = 0$.
      Следователно, $S \cap T_1 \subseteq T_0$.
      Получаваме, че $A =  \{0\} \cup (S \cap T_1) \subseteq T_0$.
    \end{itemize}    
    Следователно $A$ не е пълен клас.
  \item
    Нека $A = (L\cap M)\cup (S\setminus T_0)$.
    \begin{itemize}
    \item
      $0 \in L \cap M$ и следователно $A \not\subseteq T_1$;
    \item
      $1 \in L \cap M$ и следователно $A \not\subseteq T_0$;
    \item
      И двете константи са в $L \cap M$, но както знаем, те  не са самодвойнствени.
      Следователно $A \not\subseteq S$.
    \item
      Да разгледаме
      \[h(x,y,z) = xy\oplus xz \oplus yz \oplus 1.\]
      Лесно се съобразява, че $h \in S \setminus T_0$, но $h \not\in L$.
      Следователно, $A \not\subseteq L$.
      Освен това, $h \not \in M$. Следователно, $A \not\subseteq M$.
    \end{itemize}
  \item
    Нека $A = (L \cap T_1) \cup (S \cap M)$.
    Ако $f \in S \cap M$, то $f$ не е константа и $f(1,\dots,1) = 1$.
    Следователно, $f \in T_1$.
    Получаваме, че $S \cap M \subseteq T_1$.
    Заключаваме, че $A \subseteq T_1$.
  \item
    Нека $A = (M\setminus S)\cup(L\cap S)$.
    \begin{itemize}
    \item 
      $x\oplus 1 \in L \cap S$, но $x \oplus 1 \not\in T_0, T_1$.
      Следователно, $A \not\subseteq T_0,T_1$.
      Освен това, $x\oplus 1 \not\in M$.
      И така, $A \not\subseteq M$.
    \item
      Нека $f(x,y) \equiv x \vee y \equiv xy \oplus x \oplus y \oplus 1$.
      Имаме, че $f \in M \setminus S$ и $f \not\in L$.
      Следователно, $A \not\subseteq L$ и $A \not\subseteq S$.
    \end{itemize}
  \item
    Нека $A = (M \setminus T_0) \cup (L \setminus S)$.
    Имаме, че $M \setminus T_0 = \{1\}$.
    Следователно, $A \subseteq L$.
  \item
    Нека $A = (M \setminus T_0) \cup (S \setminus L)$.
    \begin{itemize}
    \item 
      Имаме, че $1 \in M\setminus T_0$.
      Следователно, $A \not\subseteq T_0$ и $A \not\subseteq S$.
    \item
      Нека $h(x,y,z) \equiv xy \oplus xz \oplus yz \oplus 1$.
      Тогава $h \in S\setminus L$ и $h \not\in T_1$, $h \not \in M$.
      Заключаваме, че $A \not\subseteq T_1, M, L$.
    \end{itemize}
  \end{enumerate}
\end{solution}

\begin{problem} % Гаврилов, стр. 84
  Проверете дали системата от функции $A$ е базис?
  \begin{enumerate}[a)]
  \item
    \marginpar{Не}
    $A = \{x\rightarrow y, x\oplus y, x\vee y\}$;
  \item
    \marginpar{Да}
    $A = \{x\oplus y\oplus z, x\vee y, 0, 1\}$;
  \item
    \marginpar{Не, $A \subseteq T_1$}
    $A = \{x\oplus y\oplus yz, x \oplus y \oplus 1\}$;
  \item
    \marginpar{Не}
    $A = \{xy \vee z, xy \oplus z, xy \iff z\}$;
  \end{enumerate}
\end{problem}

\begin{problem}
  Намерете всички базиси на класа $A$, където:
  \begin{enumerate}[a)]
  \item 
    $A = \{1, \ov{x}, xy(x\oplus y), x \oplus y \oplus xy \oplus yz \oplus zx\}$;
  \item
    $A = \{0, x\oplus y, x \to y, xy \iff xz\}$;
  \item
    $A = \{0,1,x\oplus y \oplus z, xy \oplus zx \oplus yz, xy \oplus z, x \vee y\}$;
  \item
    $A = \{xy, x\vee y, xy\vee z, x\oplus y, x \to y\}$.
  \end{enumerate}
\end{problem}

\begin{problem}
  Намерете броя на булевите функции на $n$ променливи, които принадлежат на следните класове:
  \begin{enumerate}[a)]
  \item
    \marginpar{$2^{2^n-1}$}
    $T_0$, $T_1$;
  \item
    \marginpar{$2^{2^n-2}$}
    $T_0 \cap T_1$
  \item
%    \marginpar{$3.2^{2^n-2}$}
    $T_0 \cup T_1$
  \item
 %   \marginpar{$2^{2^n-2}$}
    $T_0 \setminus T_1$;
  \item
    \marginpar{$2^{2^{n-1}}$}
    $S$;
  \item
    $T_0 \cap S$, $T_1 \cap S$;
  \item
    $T_0 \cap T_1 \cap S$;
  \item
    $S \setminus T_0$, $S \setminus (T_0 \cap T_1)$, $S \setminus (T_0 \cup T_1)$;
  \item
    \marginpar{$2^{n+1}$}
    $L$;
  \item
    $T_0 \cap L$, $T_1 \cap L$;
  \item
    $T_0 \cap T_1 \cap L$;
  \item
    $M \setminus T_1$;
  \item
    $M \setminus T_0$;
  \end{enumerate}
\end{problem}



%%% Local Variables: 
%%% mode: latex
%%% TeX-master: "discrete-math"
%%% End: 

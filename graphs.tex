\begin{thm}
  Ако $G$ е дърво, то $\varepsilon = \nu - 1$.
\end{thm}

\begin{crl}
  Всяко нетривиално дърво има поне два върха със степен 1.
\end{crl}

\begin{problem}
  Покажете, че ако $G$ е дърво и има връх със степен $\geq k$, то $G$ има поне $k$ върха със степен 1.
\end{problem}

\begin{problem}
  Нека $G$ е свързан граф.
  Докажете, че всеки два най-дълги пътя в свързан граф имат общ връх.
\end{problem}

\begin{problem}
  За $G$ прост граф, докажете, че $\varepsilon = \binom{\nu}{2}$ т.с.т.к. $G$ е пълен.
\end{problem}

\begin{problem}
  Докажете, че в граф с $\nu\geq 2$, има поне два върха с еднаква степен.
\end{problem}

\begin{problem}
  Докажете, че:
  \begin{enumerate}
  \item
    във всеки неориентиран граф броят на върховете с нечетна степен е четен;
  \item
    всеки регулярен граф с нечетна степен има четен брой върхове;
  \item
    всеки граф с $\varepsilon > \binom{\nu-1}{2}$ е свързан.
    Дайде пример за несвързан граф с $\varepsilon = \binom{\nu-1}{2}$.
  \item
    във граф всички върхове имат степен поне $d$.
    Докажете, че в графа има път с дължина $d$.
  \end{enumerate}
\end{problem}


\begin{problem} % зад. 1.22
  Да разгледаме графа $G$ (без примки и без кратни ребра) със $s$ компоненти на свързаност.
  Докажете, че $\nu - s \leq \varepsilon \leq \binom{\nu-s+1}{2}$.
\end{problem}

\begin{problem}
  Нега $G$ е граф с $n$ върха и в $G$ няма прост цикъл с дължина 3.
  Докажете, че $G$ има най-много $\lfloor{\frac{n^2}{4}}\rfloor$ ребра.
\end{problem}

\begin{problem}
  Нека $G$ е произволен граф без примки и кратни ребра, а $\overline{G}$ е неговото допълнение.
  Докажете, че поне един от графите $G$, $\overline{G}$ е свързан;
\end{problem}


\begin{problem}
  \begin{enumerate}
  \item
    Да се построят всички неизоморфни графи на 1,2,3 и 4 върха.
  \item
    Намерете броя на ребрата на граф без цикли с $n$ върха и $k$ компоненти.
  \end{enumerate}
\end{problem}

\chapter{Теория на Графите}

\section{Неориентирани графи}

\begin{dfn}
  Неориентиран граф\index{неориентиран!граф} $G$ е наредена тройка $(V,E,\psi_G)$, където
  $V$ е непразно множество, $V,E$ са непресичащи се множества, и $\psi_G$ асоциира с всеки елемент $e\in E$
  ненаредена двойка от елементи на $V$.
  Елементите на $V$ наричаме върхове, а елементите на $E$ ребра.
\end{dfn}

Нека да въведем някои означения.
Под прост граф\index{прост!граф} $G$ ще разбираме неориентиран граф без примки и повтарящи се ребра.
С $\delta(G)$\index{$\delta(G)$} ще означаваме минималната степен в графа $G$, а с $\Delta(G)$\index{$\Delta(G)$} - максималната.
Означаваме $\nu(G) = |V|, \epsilon(G) = |E|$.
Броят на свързаните компоненти на $G$ означаваме с $\omega(G)$.

\begin{problem}
  Докажете следното неравенство:
  \[\delta \leq \frac{2\varepsilon}{\nu} \leq \Delta.\]
\end{problem}
\begin{proof}
  \[\nu\delta \leq\sum_{1\leq i \leq\nu} deg_G(v_i) \leq \nu\Delta.\]
\end{proof}


\section{Дървета}

\begin{dfn}
  Дърво е свързан граф без цикли.
\end{dfn}

\begin{thm}
  В дърво, между всеки два върха има единствен път.
\end{thm}

\begin{thm}
  Ако $G$ е дърво, то $\varepsilon(G) = \nu(G) - 1$.
\end{thm}
\begin{proof}
  С индукция по $\nu$. За $\nu = 1$ е ясно.
  Да допуснем за $G$ с брой ребра $<\nu$ и ще докажем за $G$.
  Нека $uv\in E$ и $G'$ се получава като изтрием това ребро.
  Получаваме два свързани ациклични графа $G_1, G_2$.
  Следователно те са дървета и от и.п. 
  \[\varepsilon(G_1) = \nu(G_1) - 1\ \&\ \varepsilon(G_2) = \nu(G_2) - 1.\]
  Получаваме, че 
  \[\varepsilon(G - uv) = \varepsilon(G_1) + \varepsilon(G_2) = \nu(G_1) + \nu(G_2) - 2.\]
  Накрая, $\varepsilon(G) = \varepsilon(G-uv) + 1 = \nu(G_1) + \nu(G_2) - 2 + 1= \nu(G) - 1$.
\end{proof}

\begin{crl}
  Всяко нетривиално дърво има поне два върха със степен 1.
\end{crl}
\begin{proof}
  Ясно е, че $(\forall v\in V)[d(v) \geq 1]$.
  Знаем, че \[\sum_{v\in V}d(v) = 2\varepsilon = 2\nu - 2,\] от където следва, че има поне два върха със степен 1.
\end{proof}

\subsection{Покриващи дървета}

Означаваме с $\tau(G)$ броят на покриващите дървета на $G$ (не неизоморфните, а всички).

Нека $u\neq v, e = (u,v)\in E$. С $G.e = (E',V')$ означаваме графа получен от $G$, като премахваме реброто $e$ и 
съединяваме краищата $u,v$. Ясно е, че $\nu(G.e) = \nu(G) - 1, \varepsilon(G.e) = \varepsilon(G) - 1, \omega(G.e) = \omega(G)$.

\begin{thm}
  Ако $e\in E$ не е примка, то
  $\tau(G) = \tau(G-e) + \tau(G.e)$.
\end{thm}
\begin{proof}
  $\tau(G) = n + m$, където $n$ е броят на покриващите дървета на $G$, в които не участва $e$
  и $m$ е броят на покриващите дървета, в които участва $e$.
  Ясно е, че $n = \tau(G-e)$.
  На всяко покриващо дърво $T$, в което участва $e$, съответства покриващо дърво $T.e$ на $G.e$.
  Това съответствие е взаимно-еднозначно, следователно $m = \tau(G.e)$.
\end{proof}

\begin{problem}
  Колко на брой са всички изоморфни и неизоморфни покриващи дървета на $K_3,K_4$ ?
\end{problem}


\begin{problem}
  Покажете, че ако $G$ е дърво и има връх със степен $\geq k$, то $G$ има поне $k$ върха със степен 1.
\end{problem}

\begin{problem}
  Нека $G$ е свързан граф.
  Докажете, че всеки два най-дълги пътя в свързан граф имат общ връх.
\end{problem}

\begin{problem}
  За $G$ прост граф, докажете, че $\varepsilon = \binom{\nu}{2}$ т.с.т.к. $G$ е пълен.
\end{problem}

\begin{problem}
  Докажете, че в граф с $\nu\geq 2$, има поне два върха с еднаква степен.
\end{problem}

\begin{problem}
  Докажете, че:
  \begin{enumerate}
  \item
    във всеки неориентиран граф броят на върховете с нечетна степен е четен;
  \item
    всеки регулярен граф с нечетна степен има четен брой върхове;
  \item
    всеки граф с $\varepsilon > \binom{\nu-1}{2}$ е свързан.
    Дайде пример за несвързан граф с $\varepsilon = \binom{\nu-1}{2}$.
  \item
    във граф всички върхове имат степен поне $d$.
    Докажете, че в графа има път с дължина $d$.
  \end{enumerate}
\end{problem}


\begin{problem} % зад. 1.22
  Да разгледаме графа $G$ (без примки и без кратни ребра) със $s$ компоненти на свързаност.
  Докажете, че $\nu - s \leq \varepsilon \leq \binom{\nu-s+1}{2}$.
\end{problem}

\begin{problem}
  Нега $G$ е граф с $n$ върха и в $G$ няма прост цикъл с дължина 3.
  Докажете, че $G$ има най-много $\lfloor{\frac{n^2}{4}}\rfloor$ ребра.
\end{problem}

\begin{problem}
  Нека $G$ е произволен граф без примки и кратни ребра, а $\overline{G}$ е неговото допълнение.
  Докажете, че поне един от графите $G$, $\overline{G}$ е свързан;
\end{problem}


\begin{problem}
  \begin{enumerate}
  \item
    Да се построят всички неизоморфни графи на 1,2,3 и 4 върха.
  \item
    Намерете броя на ребрата на граф без цикли с $n$ върха и $k$ компоненти.
  \end{enumerate}
\end{problem}

\section{Ориентирани графи}

%%% Local Variables: 
%%% mode: latex
%%% TeX-master: "discrete-math"
%%% End: 

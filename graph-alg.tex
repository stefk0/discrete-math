\chapter {Алгоритми за графи}

\section{Обхождане на граф}

Нека е даден ориентирания граф $G = (V,E)$.
При обхождането на граф, всеки връх може да бъде в едно от три състояния, или както сме ги означили тук, в един от три цвята.
Ако един връх $v$ е
\begin{itemize}
\item 
  бял, то той още не е срещнат.
\item
  сив, то той е срещнат, но още не е напълно обработен.
\item
  черен, то той е напълно обработен.
\end{itemize}


\subsection{Обхождане в широчина}


\begin{algorithm}
  \caption{Инициализация}
  \label{alg:bfs-init}
  \begin{algorithmic}[1]
    \Procedure{BFS-INIT}{$G$,$r$}
    \ForAll{$v \in V \setminus \{r\}$}
    \State $\texttt{color}[v] := \texttt{WHITE}$
    \State $\texttt{dist}[v] := \infty$
    \State $\texttt{pred}[v] := \texttt{NIL}$
    \EndFor
    \State $\texttt{color}[r]:= \texttt{GRAY}$
    \State $\texttt{dist}[r]:= 0$
    \State $\texttt{pred}[r]:= \texttt{NIL}$
    \EndProcedure
  \end{algorithmic}
\end{algorithm}

За този алгоритъм най-удобно е да имаме масив $\texttt{Adj}$ с дължина $\abs{V}$,
като $\texttt{Adj}[u]$ дава списък с преките наследници на $u$, т.е.
\[\texttt{Adj}[u] = \{v \in V \mid (u,v) \in E\}.\]

\begin{algorithm}[H]
  \caption{Обхождане на граф в широчина}
  \label{alg:bfs}
  \begin{algorithmic}[1]
    \Procedure{BFS}{$G$,$r$}
    \State \Call{BFS-INIT}{$G$,$r$}
    \State $Q := \emptyset$\Comment{Опашката $Q$ съдържа точно сивите върхове}
    \State put($Q$,$r$)
    \While {$Q \neq \emptyset$}
    \State $u := get(Q)$\Comment{$u$ е премахнат от опашката}
    \ForAll{$v \in \texttt{Adj}[u]$}
    \If{$\texttt{WHITE} = \texttt{color}[v]$}
    \State $put(Q,v)$
    \State $\texttt{color}[v] := \texttt{GRAY}$
    \State $\texttt{pred}[v] := u$
    \State $\texttt{dist}[v] := \texttt{dist}[u] + 1$
    \EndIf
    \EndFor
    \State $\texttt{color}[u] := \texttt{BLACK}$
    \EndWhile
    \EndProcedure
  \end{algorithmic}
\end{algorithm}

\begin{itemize}
\item 
  Алгоритъмът работи както за ориентирани, така и за неориентирани графи.
\item
  {\bf Дължина на път} (без цикли) е броят на ребрата, които участват в пътя.
  Например, за пътя $p = (v_0,\dots,v_k)$ в графа $G$,
  неговата дължина е $k$, защото ребрата, които участват в $p$
  са $\{(v_0,v_1),(v_1,v_2),\dots,(v_{k-1},v_k)\}$ и са общо $k$ на брой.
  Обикновено ще означаваме дължината на пътя $p$ с $\abs{p}$ и пишем $v_0 \stackrel{p}{\leadsto} v_k$.
\item
  Нека да означим за всеки два върха $u,v \in V$,
  \begin{align*}
    \delta(u,v) =
    \begin{cases}
      \min\{\abs{p} \mid u\stackrel{p}{\leadsto}v\}, & \text{ ако има път между }u, v \\
      \infty, & \text{ ако няма път}
    \end{cases}
  \end{align*}
\item
  Имаме свойството, че за граф $G = (V,E)$ и един връх $s \in V$,
  ако $(u,v) \in E$, то
  \[\delta(s,v) \leq \delta(s,u) + 1.\]
\item
  За граф $G = (V,E)$, и фиксиран връх $r\in V$, означаваме
  \[G_{pred} = (V_{pred},E_{pred}),\]
  където
  \begin{align*}
    V_{pred} & = \{v \in V\mid \texttt{pred}[v] \neq \texttt{NIL}\} \cup \{s\},\\
    E_{pred} & = \{(u,v) \in E \mid \texttt{pred}[v] = u\}.
  \end{align*}
  След изпълнение на BFS($G$,$r$), $G_{pred}$ представлява дърво с корен $r$, 
  за всеки достижим в $G$ от $r$ връх $v$, $G_{pred}$ съдържа единствен прост път $r \stackrel{p}{\leadsto} v$, 
  като $p$ е най-къс измежду всички пътища свързващи $r$ с $v$ в $G$.
\end{itemize}

\begin{thm}
  Нека е даден граф $G = (V,E)$ и един връх $r \in V$.
  След изпълнение на BFS($G$,$r$) получаваме, че
  \[(\forall v \in V)[\texttt{dist}[v] = \delta(r,v)].\]
\end{thm}

\subsection{Обхождане в дълбочина}

\begin{algorithm}[H]
  \caption{Обхождане в дълбочина}
  \label{alg:dfs-visit}
  \begin{algorithmic}[1]
    \Procedure{DFS-VISIT}{$G$,$u$}
    % \State $t := t+1$
    % \State $in[u] := t$
    \State $\texttt{color}[u] := \texttt{GRAY}$\Comment{Върхът $u$ е посетен, но не е обработен}
    \ForAll{$v \in \texttt{Adj}[u]$}
    \If {$\texttt{WHITE} = \texttt{color}[v]$}
    \State $\texttt{pred}[v] := u$
    \State \Call{DFS-VISIT}{$v$}
    \EndIf
    \EndFor
    % \State $t := t+1$
    % \State $out[u] := t$
    \State $\texttt{color}[u] := \texttt{BLACK}$\Comment{Приключили сме с $u$}
    \EndProcedure
    \Statex
    \Procedure{DFS}{$G$}
    \ForAll{$v \in V$}\Comment{Инициализация}
    \State $\texttt{color}[v] := \texttt{WHITE}$
    \State $\texttt{pred}[v] := \texttt{NIL}$
    \EndFor    
    % \State $t := 0$
    \ForAll{$v \in V$}
    \If{$\texttt{WHITE} = \texttt{color}[v]$}
    \State\Call{DFS-VISIT}{$G$,$v$}
    \EndIf
    \EndFor
    \EndProcedure
  \end{algorithmic}
\end{algorithm}


\section{Минимално покриващо дърво на граф}


\begin{itemize}
\item
  Тук ще разглеждаме само {\bf неориентирани} графи $G = (V,E,w)$ с тегла по ребрата
  зададени с функцията $w:E\to\R$.
\item
  Един граф $G = (V,E)$ се нарича {\bf свързан}, ако има път между всеки два $v,v^\prime \in V$.
\item 
  Един неориентиран граф $G$ се нарича {\bf дърво}, ако $G$ е свързан и без цикли.
\item
  {\bf Покриващо дърво} за свъзан неориентиран граф $G = (V,E)$,
  е дърво $T = (V,E^\prime)$, $E^\prime \subseteq E$.
\item
  Тегло на едно подмножество от ребра $U \subseteq E$ е числото
  \[w(U) = \sum_{e \in U} w(e).\]
\item
  {\bf Минимално покриващо дърво} за свъзан неориентиран претеглен граф $G = (V,E,w)$
  е покриващо дърво $T$, за което 
  \[w(T) = \min\{w(T^\prime) \mid T^\prime\mbox{ е покриващо дърво за }G\}.\]
\end{itemize}

\subsection{Алгоритъм на Прим}

Нека е даден неориентиран претеглен {\bf свързан} граф $G = (V,E,w)$.

\begin{algorithm}[H]
  \caption{Намиране на покриващо дърво (Прим)}

  \begin{algorithmic}[1]
    \Procedure{PRIM}{$G,r$}
    \State $U := \{r\}$\Comment{Започваме от дърво с корен $r$ и без ребра}
    \State $S := \emptyset$
    \While{$(\exists (x,y)\in E)[x\in U\ \&\ y \in V\setminus U]$}
    \State Избираме $(u,v) \in E$, за което
    \State $w(u,v) = \min\{w(x,y) \mid x\in U\ \&\ y \in V\setminus U\ \&\ (x,y) \in E\}$
    \State $U := U\cup\{v\}$
    \State $S := S \cup\{(u,v)\}$
    \EndWhile
    \State \textbf{return} $(U,S)$\Comment{Връщаме като резултат полученото дърво}
    \EndProcedure
  \end{algorithmic}
\end{algorithm}

% \begin{enumerate}
% \item 
%   Започваме от дървото $T_0 = (\{r\},\emptyset)$.
% \item
%   Нека сме построили дървото $T_i = (V_i,E_i)$.
%   Избираме ребро $(v,v^\prime) \in E$ такова ,че
%   \[w(v,v^\prime) = \min\{w(x,y) \mid x\in V_i\ \&\ y \in V\setminus V_i\ \&\ (x,y) \in E\}.\]
%   Образуваме \[T_{i+1} = (V_i\cup\{v^\prime\}, E_i \cup \{(v,v^\prime)\}).\]
% \item
%   Алгоритъмът завършва когато $V_i = V$.
% \end{enumerate}

\subsection{Алгоритъм на Крускал}

Нека е даден неориентиран {\bf свързан} претеглен граф $G = (V,E,w)$.

\begin{algorithm}[h!]
  \caption{Намиране на покриващо дърво (Крускал)}
  
  \begin{algorithmic}[1]
    \Procedure{KRUSKAL}{$G$}
    \State $X = \emptyset$\Comment{$X$ ще бъде колекция от дървета}
    \For{$v \in V$}
    \State Добавяме дървото $T = (\{v\},\emptyset)$ към колекцията $X$
    \EndFor
    \State$E^\prime := $\Call{Sort}{$E$,$w$}\Comment{Сортираме $E$ във възходящ ред относно тегла им}
    \Statex
    \ForAll{$(u,v) \in E^\prime$}
    \State Нека $u$ е връх в дървото $T_u \in X$
    \State Нека $v$ е връх в дървото $T_v \in X$
    \If{$T_u \neq T_v$}
    \State $W := V_u\cup V_v$
    \State $R := E_u\cup E_v\cup\{(u,v)\}$
    \State $T := (W,R)$
    \State Премахваме $T_u$ и $T_v$ от колекцията $X$
    \State Добавяме дървото $T$ към $X$
    \EndIf
    \EndFor
    \State \Return единственото дърво останало в $X$
    \EndProcedure
  \end{algorithmic}
\end{algorithm}

\section{Минимални пътища от даден връх}

\begin{itemize}
\item
  С $u \stackrel{p}{\leadsto} v$ означаваме, че $p$ е път от $u$ до $v$.
\item
  Тук ще разглеждаме {\bf ориентирани} графи $G = (V,E)$, като имаме и 
  функция $w: E\to \R$, която задава {\bf тегла} на ребрата на графа.
\item 
  {\bf Цена на път} $p = (v_0,\dots,v_k)$ в графа означаваме 
  \[w(p) = \sum_{i<k} w(v_i,v_{i+1}).\]
\item
  За всеки два върха $u,v \in V$, означаваме
  \begin{align*}
    \delta(u,v) = 
    \begin{cases}
      \min\{w(p)\mid u \stackrel{p}{\leadsto} v\}, \mbox{ ако има път от }u\mbox{ до }v\\
      \infty, \mbox{ иначе }
    \end{cases}
  \end{align*}
\item
  {\bf Минимален път} $p$ от $u$ до $v$ е такъв път, за който $w(p) = \delta(u,v)$.
\item
  Имаме следното важно свойство.
  Нека $u \stackrel{p}{\leadsto} v$ и $p$ е {\bf минимален път}.
  Да означим $p = (v_0,\dots,v_k)$ и $p_{ij} = (v_i,\dots,v_j)$ за $0\leq i \leq j \leq k$.
  Тогава за всеки $0\leq i \leq j \leq k$, 
  $p_{ij}$ е {\bf минимален път} от $v_i$ до $v_j$.
\item
  {\bf Цикъл} е път $p = (v_0,\dots,v_k)$, където $v_0 = v_k$.
\item
  Също така казваме, че по пътя $p = (v_0,\dots,v_k)$ има цикъл, ако
  за някои $0 \leq i < j \leq k$ имаме, че $v_i = v_j$.
\item
  Ако има цикъл с отрицателно тегло по някой път от $u$ до $v$, то 
  тогава пишем, че $\delta(u,v) = -\infty$.
\item
  Нека $u \stackrel{p}{\leadsto} v$ и $p$ е с минимално тегло.
  Тогава няма цикъл с положително тегло по $p$.
\item
  Нека $u \stackrel{p}{\leadsto} v$ и $p$ е с минимално тегло.
  Можем без ограничение на общността да приемем, че няма цикли с нулево тегло
  по $p$.
\item
  Важно свойство е, че броят на върховете по всички минални пътища е $\leq \abs{V}$.
\end{itemize}

Нека да фиксираме един връх $s \in V$.
Целта ни е да намерим минимални пътища от $s$ до всички достижими от $s$ върхове,
както и тяхната цена. 
Да отбележим, че ако имаме отрицателен по някой път $s \leadsto v$, то задачата не е добре 
дефинирана, защото тогава $\delta(s,v) = -\infty$.

За тази цел въвеждаме два масива, $\texttt{dist}$ и $\texttt{pred}$, с дължина $\abs{V}$.
\begin{itemize}
\item 
  $\texttt{dist}[v]$ - дава цена на минимален път от $s$ до $v$.
  Ако $\texttt{dist}[v] = \infty$, то не е намерен път $s\leadsto v$.
\item
  $\texttt{pred}[v]$ - дава предшественика на $v$ по този минимален път, т.е.
  ако $\texttt{pred}[v] = u$, то $s \leadsto u \to v$.
  Ако $\texttt{pred}[v] = \texttt{NIL}$, то не е намерен път $s \leadsto v$.
\end{itemize}

\begin{algorithm}[h!]
  \caption{Инициализация}
  \label{alg:init}
  \begin{algorithmic}[1]
    \Procedure{INIT}{$s$}
    \ForAll{$v \in V$}
    \State $\texttt{dist}[v] := \infty$
    \State $\texttt{pred}[v] := \texttt{NIL}$
    \EndFor
    \State $\texttt{dist}[s] := 0$
    \EndProcedure
  \end{algorithmic}
\end{algorithm}

\begin{algorithm}[h!]
  \caption{Търсене на по-добър кандидат}
  \label{alg:update}
  \begin{algorithmic}[1]
    \Procedure{UPDATE}{$u$,$v$}
    \If{$\texttt{dist}[v] > \texttt{dist}[u] + w(u,v)$}
    \State $\texttt{dist}[v] := \texttt{dist}[u] + w(u,v)$
    \State $\texttt{pred}[v] := u$
    \EndIf
    \EndProcedure
  \end{algorithmic}
\end{algorithm}

\subsection{Основни свойства}
  
\begin{prop}[Неравенство на триъгълника]
  \label{prop:triangle}
  За всяко $(u,v) \in E$,
  \[\delta(s,v) \leq \delta(s,u) + w(u,v).\]
\end{prop}

\begin{prop}
  \label{prop:upper-bound}
  Нека сме изпълнили INIT(s).
  Тогава имаме свойството \[(\forall v\in V)[\texttt{dist}[v] \geq \delta(s,v)].\]
  То се запазва и след прозволен брой изпълнения на UPDATE върху ребра на графа.
  Освен това, ако веднъж $\texttt{dist}[v] = \delta(s,v)$, то стойността на $\texttt{dist}[v]$
  повече не се променя.
\end{prop}
\begin{proof}
  Индукция по броя $i$ на изпълнения на UPDATE.
  За $i = 0$ е очевидно.
  Ще докажем твърдението за $i > 0$ изпълнения на UPDATE.
  Нека $\texttt{dist}[v] > \texttt{dist}[u] + w(u,v)$ и изпълним UPDATE(u,v).
  Тогава като използваме индукционното предположение и неравенството на триъгълника,
  \begin{align*}
    \texttt{dist}[v] & = \texttt{dist}[u] + w(u,v)\\
    & \geq \delta(s,u) + w(u,v)\\
    & \geq  \delta(s,v).
  \end{align*}

  Ясно е, че веднъж достигнали $\texttt{dist}[v] = \delta(s,v)$, $\texttt{dist}[v]$
  не може да се промени, защото тази стойност може само да намалява, а ние
  сме достигнали нейния минумум.
\end{proof}

\begin{prop}
  \label{prop:no-path}
  Нека сме изпълнили INIT($s$) и нека няма път от $s$ до $v$.
  Тогава имаме свойството
  \[\texttt{dist}[v] = \delta(s,v) = \infty.\]
  То се запазва и след прозволен брой изпълнения на UPDATE върху ребра на графа.
\end{prop}
\begin{proof}
  Щом няма път от $s$ до $v$, то $\delta(s,v) = \infty$.
  От Твърдение \ref{prop:upper-bound}, $\texttt{dist}[v] \geq \delta(s,v) = \infty$.
  Следователно, $\texttt{dist}[v] = \infty$.
\end{proof}

\begin{prop}
  \label{prop:converge}
  Нека $s\leadsto u \to v$ е път с минимално тегло.
  Нека сме изпълнили INIT(s) и няколко пъти $\texttt{UPDATE}$, като измежду тях и UPDATE($u$,$v$).
  Ако преди изпълнението на UPDATE($u$,$v$) 
  имаме, че $\texttt{dist}[u] = \delta(s,u)$, то след това изпълнение
  $\texttt{dist}[v] = \delta(s,v)$ и стойността на $\texttt{dist}[v]$ повече не се променя.
\end{prop}
\begin{proof}
  Първо да отбележим, че за $(u,v) \in E$, веднага след изпълнението на UPDATE(u,v) имаме, че
  \[\texttt{dist}[v] \leq \texttt{dist}[u] + w(u,v).\]
  Ако $\texttt{dist}[u] = \delta(s,u)$, то от Твърдение \ref{prop:upper-bound} това равенство се запазва.
  Получаваме, че:
  \begin{align*}
    \texttt{dist}[v] & \leq \texttt{dist}[u] + w(u,v)\\
    & = \delta(s,u) + w(u,v)\\
    & = \delta(s,v),
  \end{align*}
  защото $s\leadsto u \to v$ е път с минимална дължина.
  Тогава $\texttt{dist}[v] \leq \delta(s,v)$ и следователно 
  \[\texttt{dist}[v] = \delta(s,v),\]
  защото пак от Твърдение \ref{prop:upper-bound}, винаги е изпълнено, че $\texttt{dist}[v] \geq \delta(s,v)$,
\end{proof}

% \begin{prop}
%   \label{prop:path-update}
%   Да разгледаме пътя $p = (v_0,\dots,v_k)$, като $v_0 = s$.
%   Нека сме изпълнили INIT(s) и след това няколко пъти UPDATE, като сме включили 
%   UPDATE($v_{i}$,$v_{i+1}$), за всяко $0\leq i < k$, в този ред на изпълнение.
%   Тогава най-накрая получаваме, че $\texttt{dist}[v_k] = \delta(s,v_k)$.
% \end{prop}
% \begin{proof}
%   Индукция по $i$.
%   В началото, $\texttt{dist}[v_0] = \texttt{dist}[s] = 0 = \delta(s,s)$.
%   Ако $\texttt{dist}[v_{i-1}] = \delta(s,v_{i-1})$, то след изпълнение на UPDATE($v_{i-1}$,$v_{i}$),
%   получаваме от Твърдение \ref{prop:converge}, че $\texttt{dist}[v_i] = \delta(s,v_{i})$.
% \end{proof}

% \begin{prop}
%   \label{prop:tree-shortest-path}
%   Нека сега да приемем, че в нашия граф няма цикли с отрицателни тегла, достижими от $s$
%   и нека сме изпълнили INIT(s) и произволен брой пъти UPDATE.
%   Тогава:
%   \begin{enumerate}[1)]
%   \item 
%     $G_{pred}$ е дърво с корен $s$.
%   \item
%     ако $(\forall v\in V)[\texttt{dist}[v] = \delta(s,v)]$, то $G_{pred}$ е дърво на пътищата с минимални тегла с корен $s$.
%   \end{enumerate}
% \end{prop}
% \begin{proof}
%   \begin{enumerate}[1)]
%   \item 
%     Първо ще докажем, че $G_{pred}$ е насочен ацикличен граф и след
%     това, че няма пътища $p \neq p^\prime$ от вида $s\stackrel{p}{\leadsto} v$ и $s\stackrel{p^\prime}{\leadsto} v$.
%     \begin{itemize}
%     \item 
%       Да допуснем, че $G_{pred}$ е цикличен граф.
%       Нека $\gamma = (v_0,\dots,v_k)$ е цикъл, $v_0 = v_k$, който се е получил точно след изпълнение на 
%       UPDATE($v_{k-1}$,$v_k$).

%       Да разгледаме ситуацията точно преди това изпълнение на UPDATE($v_{k-1}$,$v_{k}$).
%       Имаме, че 
%       \[(\forall i < k-1)[\texttt{pred}[v_{i+1}] = v_{i}]\]
%       от което следва, че
%       \begin{equation}
%         \label{ineq}
%         (\forall i < k-1)[\texttt{dist}[v_{i+1}] \geq \texttt{dist}[v_i] + w(v_i,v_{i+1})].
%       \end{equation}
%       Щом още нямаме цикъл преди изпълнението на UPDATE($v_{k-1}$,$v_k$),
%       то стойността на $\texttt{pred}[v_k]$ се променя при извикването на UPDATE($v_{k-1}$,$v_k$).
%       Оттук следва, че
%       \[\texttt{dist}[v_{k}] > \texttt{dist}[v_{k-1}] + w(v_{k-1},v_k).\]
%       Комбинирайки с Неравенство (\ref{ineq}) получаваме, че
%       \begin{align*}
%         \sum^{k}_{i = 1} \texttt{dist}[v_{i}] & > \sum^{k-1}_{i=0} (\texttt{dist}[v_i] + w(v_{i},v_{i+1}))\\
%         & = \sum^{k-1}_{i=0} \texttt{dist}[v_i] + w(\gamma),
%       \end{align*}
%       но понеже $v_0 = v_k$, 
%       \[\sum^{k-1}_{i=0} \texttt{dist}[v_i] = \sum^{k}_{i=1} \texttt{dist}[v_i]\]
%       и тогава
%       \[0 > w(\gamma).\]
%       Получаваме, че цикълът $\gamma$ има отрицателно тегло, което е противоречие.
%     \item
%       Да допуснем, че в $G_{pred}$ има $p \neq p^\prime$  и  $s\stackrel{p}{\leadsto} v$ и $s\stackrel{p^\prime}{\leadsto} v$.
%       Това означава, че съществуват $x \neq y$,
%       $s \leadsto u \leadsto x \to z \leadsto v$ и $s \leadsto u \leadsto y \to z \leadsto v$.
%       По определение, $pred(z) = x \neq y = pred(z)$. Противоречие.
%     \end{itemize}
%   \item
%     \begin{itemize}
%     \item 
%       Лесно се съобразява, че $V_{pred}$ съдържа точно върховете достижими от $s$,
%       т.е. ако $v \in V_{pred}$, то съществува път $p = (s,\dots,v)$.
%       % защото $v$ е достижим от $s$ точно когато $\delta(s,v) = \texttt{dist}[v] < \infty$, 
%       % но тогава $\texttt{pred}[v] \neq NIL$.
%     \item
%       Вече доказахме в 1), че $G_{pred}$ е дърво с корен $s$.
%     \item
%       Остана да докажем, че ако имаме $s\stackrel{p}{\leadsto} v$ в $G_{pred}$, 
%       то $p$ е път с минимално тегло в $G$.
%       Нека $p = (v_0,\dots,v_k)$, $v_0 = s$, $v_k = v$.
%       По условие, 
%       \[(\forall i < k)[\texttt{dist}[v_i] = \delta(s,v_i)],\]
%       а от факта, че $(\forall i < k)[\texttt{pred}[v_i] = v_{i-1}]$ следва, че
%       \[(\forall i < k)[\texttt{dist}[v_i] \geq \texttt{dist}[v_{i-1}] + w(v_{i-1},v_{i})].\]
%       Като обединим горните две неравенства, получваме, че
%       \begin{equation}
%         \label{eq:weight}
%         (\forall i < k)[w(v_{i-1},v_{i}) \leq \delta(s,v_{i}) - \delta(s,v_{i-1})],
%       \end{equation}
%       Тогава
%       \begin{align*}
%         w(p) & = \sum^k_{i=1} w(v_{i-1},v_{i}) & (\text{по деф.})\\
%         & \leq \sum^k_{i=1} (\delta(s,v_i)- \delta(s,v_{i-1})) & (\ref{eq:weight})\\
%         & = \delta(s,v_k) - \delta(s,v_0)\\
%         & = \delta(s,v_k) - \delta(s,s) & (v_0 = s)\\
%         & = \delta(s,v_k) - 0\\
%         & = \delta(s,v_k).
%       \end{align*}
%       Следователно, 
%       \[w(p) \leq \delta(s,v_k).\]
%       Понеже $\delta(s,v_k)$ е минималното тегло на път от $s$ до $v_k$,
%       то $w(p) = \delta(s,v_k)$.
%       Следователно, $p$ е път с минимално тегло.
%     \end{itemize}
%   \end{enumerate}
% \end{proof}




\subsection{Алгоритъм на Дейкстра}
\index{Дейкстра!алгоритъм}

В този алгоритъм, разглеждаме ориентирани графи $G = (V,E,w)$ с {\em положителни} тегла (или цени) по ребрата, т.е. 
за всяко $(u,v) \in E$, $w(u,v) \geq 0$.

\begin{algorithm}[H]
  \caption{Пътища с мин. тегло от върха $s$ (Дейкстра)}
  \label{alg:dijkstra}
  
  \begin{algorithmic}[1]
    \Require{$w:E\to \R^+$}
    \Procedure{DIJKSTRA}{$s$}
    \State \Call{INIT}{$s$}
    \State $U := V$
    \While{$U \neq\emptyset$}
    % \AND $(\exists u\in V^\prime)[dist[u] < \infty]$}
    \State Избираме $u_0\in U$, за който $\texttt{dist}[u_0] = \min\{\texttt{dist}[v] \mid v\in U\} $
    \State $U := V^\prime\setminus\{u_0\}$
    \ForAll{ $v\in Adj[u_0]$ }
    % \If{$(u_0,v)\in E$}
    \State\Call{UPDATE}{$u_0$,$v$}
    % \AND $\delta(v) > \delta(u_0) + c(u_0,v)$}
    % \State $\delta(v):= \delta(u_0)+c(u_0,v)$
    % \State $\pi(v) := u_0$
    % \EndIf
    \EndFor
    \EndWhile
    \EndProcedure
    % \Return $\delta$
  \end{algorithmic}
\end{algorithm}

\begin{thm}
  Нека $G$ е ориентиран граф с неотрицателни тегла по ребрата.
  След изпълнението на алгоритъма на Дейкстра с начален връх $s$,
  \[(\forall v \in V)[dist[v] = \delta(s,v)].\]
\end{thm}
\begin{proof}
  Ще докажем, че на всяка итерация на while-цикъла, 
  \[(\forall v\in V\setminus V^\prime)[dist[v] = \delta(s,v)].\]
  Първоначално $V\setminus V^\prime = \emptyset$.
  Ще докажем, че на всяка итерация на while-цикъла, за върха $u$, който сме премахнали от $V^\prime$,
  е изпълнено, че $dist[u] = \delta(s,u)$.
  За целта да допуснем противното и нека $u$ е първия връх, който е премахнат от $V^\prime$,
  за който $dist[u] \neq \delta(s,u)$.
  Лесно се съобразява, че $u \neq s$.
  Освен това, трябва $s \leadsto u$, защото иначе $dist[u] = \delta(s,u) = \infty$ според Твърдение \ref{prop:no-path}.
  Нека $s \stackrel{p}{\leadsto} u$ и $p$ е път с минимално тегло.
  Да разбием пътя $p$ по следния начин:
  \[s \stackrel{p_1}{\leadsto}x\to y\stackrel{p_2}{\leadsto}u,\]
  където $y$ е първия връх по пътя $p$, за който $y\not\in V^\prime$.
  Ясно е, че тогава $x \in V^\prime$ и тогава $dist[x] = \delta(s,x)$, 
  защото ние избрахме $u$ да бъде първия връх, за който $dist[u] \neq \delta(s,u)$.
  На итерацията на while-цикъла, на която добавяме $x$ към $V^\prime$, 
  ние изпълняваме UPDATE(x,y) и според Твърдение \ref{prop:converge}, $dist[y] = \delta(s,y)$.
  Но понеже $y$ е преди $u$ по път с минимално тегло и при положение, че няма ребра с отрицателни тегла,
  \[\delta(s,y) \leq \delta(s,u).\]
  Тогава
  \begin{align*}
    dist[y] & = \delta(s,y) \\
    & \leq \delta(s,u)\\
    & \leq dist[u], \mbox{според Твърдение \ref{prop:upper-bound}}.
  \end{align*}
  Но понеже $y,v \not\in V^\prime$ и сме избрали $u$ вместо $y$, то това означава, че
  \[dist[u] \leq dist[y].\]
  Следователно, 
  \[dist[y] = dist[u]\]
  и тогава 
  \[dist[u] = \delta(s,u),\]
  с което достигаме до противоречие.
\end{proof}
\begin{cor}
  $G_{pred}$ е дърво на минималните пътища с корен $s$.
\end{cor}


Ако във $V^\prime$ има останали върхове $v$, то те имат $\delta(v) = \infty$, т.е.
те са недостижими от $s$ и следователно пътят от $s$ до $v$ има дължина $\infty$.

Фигура \ref{fig:dijkstra-table} илюстрира как се променя функцията $\delta$ по време на изпълнението на алгоритъма.
Освен това, можем да намерим не само стойността на най-късите пътища, но
и списък с ребрата, които участват във всеки от тях.
% Фигура \ref{fig:dijkstra-graph} илюстрира това.
% Ребрата, оцветени в зелено, са тези, които участват в най-късите пътища.
% Жълти ребра са тези, които са кандидати да участват в най-късите пътища.
% Червени са тези ребра, които са били вече обходени и са отхвърлени като част от най-къс път.

\tikzstyle{weight} = [font=\small]
\tikzstyle{value} = [font=\small]
\tikzstyle{edge} = [draw,thick,-]
\tikzstyle{nodedecorate}=[shape=circle,draw,thick,font=\small]
\tikzstyle{arrowdecorate}=[->,>=stealth,thick]

% Rename: selected --> current
\tikzstyle{selected vertex}=[vertex, fill=yellow!50]
\tikzstyle{selected edge} = [draw,line width=5pt,-,yellow!50]

\tikzstyle{vertex}=[circle,minimum size=15pt,inner sep=0pt]
\tikzstyle{sure vertex} = [vertex, fill=green!30]

\tikzstyle{path edge} = [draw,line width=5pt,-,red!50]

\tikzstyle{sure edge} = [draw,line width=5pt,-,green!30]
% \tikzstyle{ignored edge} = [draw,line width=5pt,-,black!20]


\begin{figure}[!htbp]
  \begin{subfigure}[b]{0.5\textwidth}
    \begin{tikzpicture}[]
      
      \foreach \nodename/\x/\y/\direction/\navigate in { a/1/1/above/north,
        b/0/0/left/west, c/1/-1.5/below/south, d/3/1/above/north, e/3/-1.5/below/south, f/5/0.5/right/east, g/5/2.5/right/east}
      {
        \node (\nodename) at (\x,\y) [nodedecorate] {};
        \node [\direction] at (\nodename.\navigate) {$\nodename$};
      }
      %% edges or lines
      \path
      \foreach \startnode/\endnode/\direction/\weight in {b/a/above/7,
        b/c/below/2, c/a/left/4, a/d/below/4, c/e/below/5, d/c/left/8, e/d/right/3}
      {
        (\startnode) edge[arrowdecorate] node[\direction] {$\weight$} (\endnode)
      }
      
      \foreach \startnode/\endnode/\direction/\angle/\weight in {
        a/g/above/15/10, d/f/above/15/5, d/g/above/-15/2, f/d/below/15/1, g/f/right/15/6, e/f/below/-15/7}
      {
        (\startnode) edge[arrowdecorate,bend left=\angle] node[\direction] {$\weight$} (\endnode)
      };
    \end{tikzpicture}
    \caption{По-долу ще приложим алгоритъма на Дейкстра върху този граф}
    \label{subfig:dijkstra}
  \end{subfigure}
  \qquad
  \begin{subfigure}[b]{0.5\textwidth}
  \begin{tikzpicture}[]
    
    \foreach \nodename/\x/\y/\direction/\navigate in { a/1/1/above/north,
      s/0/0/left/west, b/1/-1.5/below/south}
      {
        \node (\nodename) at (\x,\y) [nodedecorate] {};
        \node [\direction] at (\nodename.\navigate) {$\nodename$};
      }
      %% edges or lines
      \path
      \foreach \startnode/\endnode/\direction/\weight in {s/a/above/3,
        s/b/below/2, a/b/right/-2}
      {
        (\startnode) edge[arrowdecorate] node[\direction] {$\weight$} (\endnode)
      };
    \end{tikzpicture}
    \caption{Пример, за който алгоритъмът на Дейкстра не дава верен резултат (Защо?)}
  \end{subfigure}
  \caption{}
  \end{figure}

  \begin{figure}[!htbp]
    \begin{subtable}[b]{0.5\textwidth}
      \begin{tabular}[b]{|c|c|c|c|c|c|c|c|c|}
        \hline
        $a$ & $b$ & $c$ & $d$ & $e$ & $f$ & $g$\\
        \hline
        $\infty$ & {\bf \framebox{0}} & $\infty$ & $\infty$ & $\infty$ & $\infty$ & $\infty$ \\
        \hline
        7 & $\colon$ & {\bf \framebox{2}} & $\infty$ & $\infty$ & $\infty$ & $\infty$ \\
        \hline
        {\bf \framebox{6}} & $\colon$ & $\colon$ & $\infty$ & 7 & $\infty$ & $\infty$ \\
        \hline
        $\colon$ & $\colon$ & $\colon$ & 10 & {\bf \framebox{7}} & $\infty$ & 16 \\
        \hline
        $\colon$ & $\colon$ & $\colon$ & {\bf \framebox{10}} & $\colon$ & 14 & {\bf 16} \\
        \hline
        $\colon$ & $\colon$ & $\colon$ & $\colon$ & $\colon$ & 14 & {\bf \framebox{12}} \\
        \hline
        $\colon$ & $\colon$ & $\colon$ & $\colon$ & $\colon$ & {\bf \framebox{14}} & $\colon$ \\
        \hline
        $\colon$ & $\colon$ & $\colon$ & $\colon$ & $\colon$ & $\colon$ & $\colon$ \\
        \hline
      \end{tabular}
      \caption{Масива $\texttt{dist}$ за начален връх $b$}
    \end{subtable}
    \qquad
    \begin{subtable}[b]{0.5\textwidth}
      \begin{tabular}[b]{|c|c|c|c|c|c|c|c|c|}
        \hline
        $a$ & $b$ & $c$ & $d$ & $e$ & $f$ & $g$\\
        \hline
        $\texttt{NIL}$ & \framebox{\texttt{NIL}} & $\texttt{NIL}$ & $\texttt{NIL}$ & $\texttt{NIL}$ & $\texttt{NIL}$ & $\texttt{NIL}$ \\
        \hline
        $b$ & $\colon$ & \framebox{$b$} & $\texttt{NIL}$ & $\texttt{NIL}$ & $\texttt{NIL}$ & $\texttt{NIL}$ \\
        \hline
        {\bf \framebox{c}} & $\colon$ & $\colon$ & $\texttt{NIL}$ & c & $\texttt{NIL}$ & $\texttt{NIL}$ \\
        \hline
        $\colon$ & $\colon$ & $\colon$ & a & {\bf \framebox{c}} & $\texttt{NIL}$ & a \\
        \hline
        $\colon$ & $\colon$ & $\colon$ & {\bf \framebox{a}} & $\colon$ & c & a \\
        \hline
        $\colon$ & $\colon$ & $\colon$ & $\colon$ & $\colon$ & c & {\bf \framebox{d}} \\
        \hline
        $\colon$ & $\colon$ & $\colon$ & $\colon$ & $\colon$ & {\bf \framebox{c}} & $\colon$ \\
        \hline
        $\colon$ & $\colon$ & $\colon$ & $\colon$ & $\colon$ & $\colon$ & $\colon$ \\
        \hline
      \end{tabular}
      \caption{Масива $\texttt{pred}$ за начален връх $b$}
    \end{subtable}
    \caption{Алгоритъм на Дейкстра с начален връх $b$ за графа от (\ref{subfig:dijkstra})}
  \label{fig:dijkstra-table}
\end{figure}

% \begin{figure}[!htbp]
%   \index{Дейкстра!алгоритъм}
%   

\subfigure[Започваме от съседите на $a$]{
  \begin{tikzpicture}[scale=0.9]
    % nodes
    \foreach \nodename/\x/\y/\value/\direction/\navigate/\color in { 
      a/0/0/0/above/north/green, 
      b/-1/1/\infty/left/west/black, 
      c/2.5/1/\infty/above/north/black, 
      d/2/-0.5/\infty/below/south/black,
      e/-1/-0.5/\infty/below/south/black, 
      f/0.5/2/\infty/above/north/black,
      g/0/-1.8/\infty/below/south/black, 
      h/2.5/-1.8/\infty/right/east/black}
    {
        \node[vertex, nodedecorate, fill=\color!25] (\nodename) at (\x,\y) {$\value$};
        \node [\direction] at (\nodename.\navigate) {$\nodename$};
      };
      % edges
      \path
      \foreach \startnode/\endnode/\direction/\angle/\weight in {
        e/b/left/15/5, e/g/below/-15/3, d/g/below/15/2, 
        g/a/left/15/2, a/g/right/30/9, f/c/below/-15/1,
        c/f/above/-30/5, c/h/right/15/2, c/d/above/-15/1,
        f/d/left/-15/4, a/b/below/0/2, b/f/above/0/1, 
        a/d/above/0/8}
      {
        (\startnode) edge[arrowdecorate,bend left=\angle] node[\direction] {$\weight$} (\endnode)
      };
    \end{tikzpicture}
  }
  \subfigure[$b$ е най-близко до $a$]{
    \begin{tikzpicture}[scale=0.9]
      %nodes
      \foreach \nodename/\x/\y/\value/\direction/\navigate/\color in { 
        a/0/0/0/above/north/green, 
        b/-1/1/2/left/west/yellow, 
        c/2.5/1/\infty/above/north/black, 
        d/2/-0.5/8/below/south/yellow,
        e/-1/-0.5/\infty/below/south/black, 
        f/0.5/2/\infty/above/north/black, 
        g/0/-1.8/9/below/south/yellow, 
        h/2.5/-1.8/\infty/right/east/black}
      {
        \node[vertex, nodedecorate, fill=\color!25] (\nodename) at (\x,\y) {$\value$};
        \node [\direction] at (\nodename.\navigate) {$\nodename$};
      };

      \path
      \foreach \startnode/\endnode/\direction/\angle in {
        a/g/right/30, a/b/above/0, a/d/above/0}
      {
        (\startnode) edge[selected edge,bend left=\angle] node[\direction] {} (\endnode)
      };
      %edges
      \path
      \foreach \startnode/\endnode/\direction/\angle/\weight in {
        e/b/left/15/5, e/g/below/-15/3, d/g/below/15/2, g/a/left/15/2, a/g/right/30/9, f/c/below/-15/1, c/f/above/-30/5,
        c/h/right/15/2, c/d/above/-15/1, f/d/left/-15/4, a/b/below/0/2, b/f/above/0/1,  a/d/above/0/8}
      {
        (\startnode) edge[arrowdecorate,bend left=\angle] node[\direction] {$\weight$} (\endnode)
      };
    \end{tikzpicture}
  }
  \subfigure[Най-къс път до $f$]{
    \begin{tikzpicture}[scale=0.9]
      %nodes
      \foreach \nodename/\x/\y/\value/\direction/\navigate/\color in { 
        a/0/0/0/above/north/green, 
        b/-1/1/2/left/west/green,
        c/2.5/1/\infty/above/north/black, 
        d/2/-0.5/8/below/south/yellow,
        e/-1/-0.5/\infty/below/south/black, 
        f/0.5/2/3/above/north/yellow, 
        g/0/-1.8/9/below/south/yellow, 
        h/2.5/-1.8/\infty/right/east/black}
      {
        \node[vertex, nodedecorate, fill=\color!25] (\nodename) at (\x,\y) {$\value$};
        \node [\direction] at (\nodename.\navigate) {$\nodename$};
      };
      \path
      (a) edge[sure edge] node[] {} (b);
      \path
      \foreach \startnode/\endnode/\direction/\angle in {
        a/g/right/30, a/d/above/0}
      {
        (\startnode) edge[selected edge,bend left=\angle] node[\direction] {} (\endnode)
      };
      \path
      \foreach \startnode/\endnode/\direction/\angle in {
         b/f/above/0}
      {
        (\startnode) edge[selected edge, bend left=\angle] node[\direction] {} (\endnode)
      };
      %edges
      \path
      \foreach \startnode/\endnode/\direction/\angle/\weight in {
        e/b/left/15/5, e/g/below/-15/3, d/g/below/15/2, g/a/left/15/2, a/g/right/30/9, f/c/below/-15/1, c/f/above/-30/5,
        c/h/right/15/2, c/d/above/-15/1, f/d/left/-15/4, a/b/below/0/2, b/f/above/0/1,  a/d/above/0/8}
      {
        (\startnode) edge[arrowdecorate,bend left=\angle] node[\direction] {$\weight$} (\endnode)
      };
    \end{tikzpicture}
  }
  \subfigure[По-къс път до $d$ и $c$]{
    \begin{tikzpicture}[scale=0.9]
      % nodes
      \foreach \nodename/\x/\y/\value/\direction/\navigate/\color in { 
        a/0/0/0/above/north/green,
        b/-1/1/2/left/west/green,
        c/2.5/1/4/above/north/yellow, 
        d/2/-0.5/7/below/south/yellow,
        e/-1/-0.5/\infty/below/south/black, 
        f/0.5/2/3/above/north/green, 
        g/0/-1.8/9/below/south/yellow, 
        h/2.5/-1.8/\infty/right/east/black}
      {
        \node[vertex, nodedecorate, fill=\color!25] (\nodename) at (\x,\y) {$\value$};
        \node [\direction] at (\nodename.\navigate) {$\nodename$};
      };
      \path
      (a) edge[sure edge] node[] {} (b)
      (b) edge[sure edge] node[] {} (f);
      
      \path
      \foreach \startnode/\endnode/\direction/\angle in {
        a/d/above/0}
      {
        (\startnode) edge[path edge,bend left=\angle] node[\direction] {} (\endnode)
      };
      \path
      \foreach \startnode/\endnode/\direction/\angle in {f/c/below/-15/1, f/d/left/-15/4, a/g/right/30}
      {
        (\startnode) edge[selected edge, bend left=\angle] node[\direction] {} (\endnode)
      };
      %edges
      \path
      \foreach \startnode/\endnode/\direction/\angle/\weight in {
        e/b/left/15/5, e/g/below/-15/3, d/g/below/15/2, g/a/left/15/2, a/g/right/30/9, f/c/below/-15/1, c/f/above/-30/5,
        c/h/right/15/2, c/d/above/-15/1, f/d/left/-15/4, a/b/below/0/2, b/f/above/0/1,  a/d/above/0/8}
      {
        (\startnode) edge[arrowdecorate,bend left=\angle] node[\direction] {$\weight$} (\endnode)
      };
    \end{tikzpicture}
  }
  \subfigure[По-къс път до $d$ и $h$]{
    \begin{tikzpicture}[scale=0.9]
      % nodes
      \foreach \nodename/\x/\y/\value/\direction/\navigate/\color in { 
        a/0/0/0/above/north/green,
        b/-1/1/2/left/west/green,
        c/2.5/1/4/above/north/green,
        d/2/-0.5/5/below/south/yellow,
        e/-1/-0.5/\infty/below/south/black, 
        f/0.5/2/3/above/north/green, 
        g/0/-1.8/9/below/south/yellow, 
        h/2.5/-1.8/6/right/east/yellow}
      {
        \node[vertex, nodedecorate, fill=\color!25] (\nodename) at (\x,\y) {$\value$};
        \node [\direction] at (\nodename.\navigate) {$\nodename$};
      };
      \path
      (a) edge[sure edge] node[] {} (b)
      (b) edge[sure edge] node[] {} (f)
      (f) edge[sure edge, bend left=-15] node[] {} (c);
      
      \path
      \foreach \startnode/\endnode/\direction/\angle in {f/d/left/-15/4, a/d/above/0
        }
      {
        (\startnode) edge[path edge,bend left=\angle] node[\direction] {} (\endnode)
      };
      \path
      \foreach \startnode/\endnode/\direction/\angle in {c/d/above/-15/1,  c/f/above/-30/5, c/h/right/15/2, a/g/right/30}
      {
        (\startnode) edge[selected edge, bend left=\angle] node[\direction] {} (\endnode)
      };
      %edges
      \path
      \foreach \startnode/\endnode/\direction/\angle/\weight in {
        e/b/left/15/5, e/g/below/-15/3, d/g/below/15/2, g/a/left/15/2, a/g/right/30/9, f/c/below/-15/1, c/f/above/-30/5,
        c/h/right/15/2, c/d/above/-15/1, f/d/left/-15/4, a/b/below/0/2, b/f/above/0/1,  a/d/above/0/8}
      {
        (\startnode) edge[arrowdecorate,bend left=\angle] node[\direction] {$\weight$} (\endnode)
      };
    \end{tikzpicture}
  }
  \subfigure[По-къс път до $g$]{
    \begin{tikzpicture}[scale=0.9]
      %nodes
      \foreach \nodename/\x/\y/\value/\direction/\navigate/\color in { 
        a/0/0/0/above/north/green, 
        b/-1/1/2/left/west/green,
        c/2.5/1/4/above/north/green,
        d/2/-0.5/5/below/south/green,
        e/-1/-0.5/\infty/below/south/black, 
        f/0.5/2/3/above/north/green, 
        g/0/-1.8/7/below/south/yellow,
        h/2.5/-1.8/6/right/east/yellow}
      {
        \node[vertex, nodedecorate, fill=\color!25] (\nodename) at (\x,\y) {$\value$};
        \node [\direction] at (\nodename.\navigate) {$\nodename$};
      };
      \path
      (a) edge[sure edge] node[] {} (b)
      (b) edge[sure edge] node[] {} (f)
      (f) edge[sure edge, bend left=-15] node[] {} (c)
      (c) edge[sure edge, bend left=-15] node[] {} (d);
      
      \path
      \foreach \startnode/\endnode/\direction/\angle in {f/d/left/-15, a/g/right/30, a/d/above/0, c/f/above/-30
        }
      {
        (\startnode) edge[path edge,bend left=\angle] node[] {} (\endnode)
      };
      \path
      \foreach \startnode/\endnode/\direction/\angle in {d/g/below/15/2, c/h/right/15}
      {
        (\startnode) edge[selected edge, bend left=\angle] node[\direction] {} (\endnode)
      };
      %edges
      \path
      \foreach \startnode/\endnode/\direction/\angle/\weight in {
        e/b/left/15/5, e/g/below/-15/3, d/g/below/15/2, g/a/left/15/2, a/g/right/30/9, f/c/below/-15/1, c/f/above/-30/5,
        c/h/right/15/2, c/d/above/-15/1, f/d/left/-15/4, a/b/below/0/2, b/f/above/0/1,  a/d/above/0/8}
      {
        (\startnode) edge[arrowdecorate,bend left=\angle] node[\direction] {$\weight$} (\endnode)
      };
    \end{tikzpicture}
  }
  \subfigure[$h$ е задънена улица]{
    \begin{tikzpicture}[scale=0.9]
      %nodes
      \foreach \nodename/\x/\y/\value/\direction/\navigate/\color in { 
        a/0/0/0/above/north/green, 
        b/-1/1/2/left/west/green,
        c/2.5/1/4/above/north/green, 
        d/2/-0.5/5/below/south/green,
        e/-1/-0.5/\infty/below/south/black, 
        f/0.5/2/3/above/north/green, 
        g/0/-1.8/7/below/south/yellow,
        h/2.5/-1.8/6/right/east/green}
      {
        \node[vertex, nodedecorate, fill=\color!25] (\nodename) at (\x,\y) {$\value$};
        \node [\direction] at (\nodename.\navigate) {$\nodename$};
      };
      \path
      (a) edge[sure edge] node[] {} (b)
      (b) edge[sure edge] node[] {} (f)
      (f) edge[sure edge, bend left=-15] node[] {} (c)
      (c) edge[sure edge, bend left=-15] node[] {} (d)
      (c) edge[sure edge, bend left=15] node[] {} (h);
      
      \path
      \foreach \startnode/\endnode/\direction/\angle in {f/d/left/-15, a/g/right/30, a/d/above/0, c/f/above/-30
        }
      {
        (\startnode) edge[path edge,bend left=\angle] node[] {} (\endnode)
      };
      \path
      \foreach \startnode/\endnode/\direction/\angle in {d/g/below/15/2}
      {
        (\startnode) edge[selected edge, bend left=\angle] node[\direction] {} (\endnode)
      };
      %edges
      \path
      \foreach \startnode/\endnode/\direction/\angle/\weight in {
        e/b/left/15/5, e/g/below/-15/3, d/g/below/15/2, g/a/left/15/2, a/g/right/30/9, f/c/below/-15/1, c/f/above/-30/5,
        c/h/right/15/2, c/d/above/-15/1, f/d/left/-15/4, a/b/below/0/2, b/f/above/0/1,  a/d/above/0/8}
      {
        (\startnode) edge[arrowdecorate,bend left=\angle] node[\direction] {$\weight$} (\endnode)
      };
    \end{tikzpicture}
  }
  \subfigure[$e$ не е достижим]{
    \begin{tikzpicture}[scale=0.9]
      %nodes
      \foreach \nodename/\x/\y/\value/\direction/\navigate/\color in { 
        a/0/0/0/above/north/green,
        b/-1/1/2/left/west/green,
        c/2.5/1/4/above/north/green, 
        d/2/-0.5/5/below/south/green,
        e/-1/-0.5/\infty/below/south/black, 
        f/0.5/2/3/above/north/green, 
        g/0/-1.8/7/below/south/green,
        h/2.5/-1.8/6/right/east/green}
      {
        \node[vertex, nodedecorate, fill=\color!25] (\nodename) at (\x,\y) {$\value$};
        \node [\direction] at (\nodename.\navigate) {$\nodename$};
      };
      \path
      (a) edge[sure edge] node[] {} (b)
      (b) edge[sure edge] node[] {} (f)
      (f) edge[sure edge, bend left=-15] node[] {} (c)
      (c) edge[sure edge, bend left=-15] node[] {} (d)
      (c) edge[sure edge, bend left=15] node[] {} (h)
      (d) edge[sure edge, bend left=15] node[] {} (g);
      
      \path
      \foreach \startnode/\endnode/\direction/\angle in {f/d/left/-15, a/g/right/30, a/d/above/0, c/f/above/-30
        }
      {
        (\startnode) edge[path edge,bend left=\angle] node[] {} (\endnode)
      };
      \path
      \foreach \startnode/\endnode/\direction/\angle in {g/a/left/15}
      {
        (\startnode) edge[selected edge, bend left=\angle] node[\direction] {} (\endnode)
      };
      %edges
      \path
      \foreach \startnode/\endnode/\direction/\angle/\weight in {
        e/b/left/15/5, e/g/below/-15/3, d/g/below/15/2, g/a/left/15/2, a/g/right/30/9, f/c/below/-15/1, c/f/above/-30/5,
        c/h/right/15/2, c/d/above/-15/1, f/d/left/-15/4, a/b/below/0/2, b/f/above/0/1,  a/d/above/0/8}
      {
        (\startnode) edge[arrowdecorate,bend left=\angle] node[\direction] {$\weight$} (\endnode)
      };
    \end{tikzpicture}
  }
  \subfigure[Краен резултат]{
    \begin{tikzpicture}[scale=0.9]
      %nodes
      \foreach \nodename/\x/\y/\value/\direction/\navigate/\color in { 
        a/0/0/0/above/north/green, b/-1/1/2/left/west/green,
        c/2.5/1/4/above/north/green, d/2/-0.5/5/below/south/green,
        e/-1/-0.5/\infty/below/south/black, f/0.5/2/3/above/north/green, 
        g/0/-1.8/7/below/south/green, h/2.5/-1.8/6/right/east/green}
      {
        \node[vertex, nodedecorate, fill=\color!25] (\nodename) at (\x,\y) {$\value$};
        \node [\direction] at (\nodename.\navigate) {$\nodename$};
      };
      \path
      (a) edge[sure edge] node[] {} (b)
      (b) edge[sure edge] node[] {} (f)
      (f) edge[sure edge, bend left=-15] node[] {} (c)
      (c) edge[sure edge, bend left=-15] node[] {} (d)
      (c) edge[sure edge, bend left=15] node[] {} (h)
      (d) edge[sure edge, bend left=15] node[] {} (g);
      
      \path
      \foreach \startnode/\endnode/\direction/\angle in {f/d/left/-15, a/g/right/30, a/d/above/0, c/f/above/-30, g/a/above/15
        }
      {
        (\startnode) edge[path edge,bend left=\angle] node[] {} (\endnode)
      };
      \path
      \foreach \startnode/\endnode/\direction/\angle in {}
      {
        (\startnode) edge[selected edge, bend left=\angle] node[\direction] {} (\endnode)
      };
      %edges
      \path
      \foreach \startnode/\endnode/\direction/\angle/\weight in {
        e/b/left/15/5, e/g/below/-15/3, d/g/below/15/2, g/a/left/15/2, a/g/right/30/9, f/c/below/-15/1, c/f/above/-30/5,
        c/h/right/15/2, c/d/above/-15/1, f/d/left/-15/4, a/b/below/0/2, b/f/above/0/1,  a/d/above/0/8}
      {
        (\startnode) edge[arrowdecorate,bend left=\angle] node[\direction] {$\weight$} (\endnode)
      };
    \end{tikzpicture}
  }

%%% Local Variables: 
%%% mode: latex
%%% TeX-master: "discrete-math"
%%% End: 

%   \caption{Алгоритъм на Дейкстра запазващ минималните пътища}
%   \label{fig:dijkstra-graph}
% \end{figure}



% \newpage

% \subsection{Алгоритъм на Белман-Форд}\index{Белман-Форд!алгоритъм}

% Алгоритъмът на Дейкстра работи само за графи $G = (V,E,w)$ с {\em положителни} тегла по ребрата.
% Сега ще разгледаме един алгоритъм, който работи и за графи с отрицателни тегла по ребрата.
% Задачата отново е да намерим минималните разстояния на пътищата с начало върха $s$, но
% искаме също така алгоритъмът да отговаря на въпроса дали има отрицателен цикъл в графа. 
% Ако такъв съществува, то няма решение на проблема. (Защо?)
% Ако отрицателен цикъл не съществува, то алгоритъмът намира пътища в графа с минимални тегла от върха $s$
% до всички достижими върхове в графа.


% \begin{algorithm}
%   \caption{Пътища с мин. тегло от върха $s$ (Белман-Форд)}
%   \label{alg:belman-ford}
  
%   \begin{algorithmic}[1]
%     \Procedure{Bellman-Ford}{$s$}
%     \State\Call{INIT}{$s$}
%     \For{$i:=1$ to $\abs{V}-1$}
%     \ForAll{$(u,v)\in E$}
%     \State\Call{UPDATE}{$u$,$v$}
%     % \ENSURE{$\texttt{dist}[v] \geq \delta(s,v)$}
%     \EndFor
%     \EndFor
    
%     \Comment{Проверка за отрицателен цикъл}
%     \ForAll{$(u,v)\in E$}
%     \If {$\texttt{dist}[v] > \texttt{dist}[u] + w(u,v)$}
%     \State Return \texttt{False}
%     \EndIf
%     \EndFor
%     \State Return \texttt{True}
%     \EndProcedure
%   \end{algorithmic}
% \end{algorithm}

% \begin{prop}
%   \label{prop:bellman-ford}
%   Нека графът $G$ няма отрицателни цикли, които са достижими от $s$.
%   Тогава след изпълнение на алгоритъма на Белман-Форд получаваме, че
%   за всички $v \in V$ достижими от $s$, 
%   \[\texttt{dist}[v] = \delta(s,v).\]
% \end{prop}
% \begin{proof}
%   Да разгледаме $s \stackrel{p}{\leadsto} v$, където $p = (v_0,\dots,v_k)$ е път с минимално тегло в $G$.
%   Понеже в пътища с минимална дължина няма цикли, то $k \leq \abs{V} - 1$.
%   Тогава според Твърдение \ref{prop:path-update}, 
%   след $i$-тата итерация на \texttt{for} цикъла (ред 3), $\texttt{dist}[v_i] = \delta(s,v_i)$.
%   Така получаваме, че най-накрая $\texttt{dist}[v] = \delta(s,v)$.
% \end{proof}
% \begin{cor}
%   \label{cor:bellman-ford}
%   При същите предположения за графа $G$,
%   за всяко $v \in V$, 
%   има път $s \leadsto v$ точно тогава, когато след приключване на алгоритъма е изпълнено $\texttt{dist}[v] < \infty$.
% \end{cor}
% \begin{proof}
%   Ако има път $p$, $s \stackrel{p}{\leadsto} v$, то
%   според твърдението $\texttt{dist}[v] = \delta(s,v) < \infty$.
%   За другата посока, нека $\texttt{dist}[v] < \infty$, но да допуснем, че няма път от $s$ до $v$.
%   Но тогава от Твърдение \ref{prop:no-path} следва, че $\texttt{dist}[v]  = \infty$,
%   което е противоречие.
% \end{proof}


% \begin{thm}
%   \label{th:bellman-ford}
%   Ако $G$ няма отрицателни цикли достижими от $s$, то
%   алгоритъмът на Белман-Форд връща TRUE, $(\forall v\in V)[dist[v] = \delta(s,v)]$,
%   и $G_{pred}$ е дърво с корен $s$, което съдържа пътища с минимални тегла.

%   Ако $G$ има отрицателни цикли достижими от $s$, то
%   алгоритъмът на Белман-Форд връща FALSE.
% \end{thm}
% \begin{proof}
%   \begin{enumerate}[a)]
%   \item 
%     Нека $G$ не съдържа цикъл с отрицателно тегло, достижим от $s$.
%     Ако $v$ е достижим от $s$, то според Твърдение \ref{prop:bellman-ford}, 
%     след изпълнение на алгоритъма
%     \[\texttt{dist}[v] = \delta(s,v).\]

%     Ако $v$ не е достижим от $s$, то според Твърдение \ref{prop:no-path},
%     след изпълнение на алгоритъма
%     \[\texttt{dist}[v] = \infty = \delta(s,v).\]

%     Понеже $(\forall v\in V)[\texttt{dist}[v] = \delta(s,v)]$, от Твърдение \ref{prop:tree-shortest-path} следва, че
%     $G_{pred}$ е дърво с корен $s$, което съдържа пътища с минимални тегла.
    
%     Като използваме Твърдение \ref{prop:triangle} лесно се вижда, че алгоритъмът връща TRUE.
%   \item
%     Нека $G$ съдържа цикъл с отрицателно тегло, достижим от $s$.
%     Нека един такъв цикъл е $\gamma = (v_0,\dots,v_k)$, $v_0 = v_k$.
%     Тогава
%     \[w(\gamma) = \sum^k_{i=1}w(v_{i-1},v_i) < 0.\]
%     Да допуснем, че алгоритъмът връща TRUE. Тогава за всяко $i = 1,\dots, k$, 
%     \[\texttt{dist}[v_i] \leq \texttt{dist}[v_{i-1}] + w(v_{i-1},v_i)\]
%     и като сумираме, 
%     \[\sum^{k}_{i=1} \texttt{dist}[v_i] \leq \sum^{k}_{i=1} \texttt{dist}[v_{i-1}] + \sum^{k}_{i=1}w(v_{i-1},v_i).\]
%     Тъй като $v_0 = v_k$, 
%     \[\sum^{k}_{i=1} \texttt{dist}[v_i] = \sum^{k}_{i=1}\texttt{dist}[v_{i-1}].\]
%     Получаваме, че \[0 \leq \sum^{k}_{i=1}w(v_{i-1},v_i) = w(\gamma),\] което е противоречие с отицателността на цикъла.
%   \end{enumerate}
% \end{proof}

% Фигура \ref{fig:bellman-ford-negative-cycle} илюстрира случая за цикъл с отрицателно тегло.
% Както забелязахме при алгоритъма на Дейкстра, и тук можем да намерим не само дължините на най-късите пътища, но
% и спъсъка на ребрата, участващи в тях. Фигура \ref{fig:bellman-ford-graph} илюстрира този проблем. 
% Останалите накрая оцветени в синьо ребра участват в най-късите пътища.



% \begin{figure}[!htbp]
%   \begin{subfigure}[b]{0.5\textwidth}
%     \begin{tikzpicture}
%       [nodedecorate/.style={shape=circle,inner sep=2pt,draw,thick},%
%       arrowdecorate/.style={->,>=stealth,thick}]
%       %% nodes or vertices
      
%       \foreach \nodename/\x/\y/\direction/\navigate in { a/0/0/below/south,
%         b/6/0/below/south, c/4.5/0/below/south, d/3/0/below/south, e/1.5/0/below/south}
%       {
%         \node (\nodename) at (\x,\y) [nodedecorate] {};
%         \node [\direction] at (\nodename.\navigate) {$\nodename$};
%       }
%       %% edges or lines
%       \path
%       \foreach \startnode/\endnode/\direction/\angle/\weight in {
%         a/e/above/15/1, d/e/above/-25/-1, e/d/below/-15/1,  d/c/above/15/1, c/b/above/15/1, b/e/below/60/-4}
%       {
%         (\startnode) edge[arrowdecorate,bend left=\angle] node[\direction] {$\weight$} (\endnode)
%       };
%       ;
%     \end{tikzpicture}
%     \caption{Граф с отрицателен цикъл}
%   \end{subfigure}
%   \quad
%   \begin{subtable}[b]{0.5\textwidth}
%     \begin{tabular}{|c|c|c|c|c|}
%       \hline
%       $\delta(a)$ & $\delta(b)$ & $\delta(c)$ & $\delta(d)$ & $\delta(e)$ \\
%       \hline
%       0 & $\infty$ & $\infty$ & $\infty$ & {\bf \framebox{1}}\\
%       \hline
%       \hline
%       $\colon$ & \framebox{$\infty$} & $\infty$ & $\infty$ & 1\\
%       $\colon$ & $\infty$ & \framebox{$\infty$} & $\infty$ & 1\\
%       $\colon$ & $\infty$ & $\infty$ & {\bf \framebox{2}} & 1\\
%       $\colon$ & $\infty$ & $\infty$ & 2 & \framebox{1}\\
%       \hline\hline
%       $\colon$ & \framebox{$\infty$} & $\infty$ & 2 & 1\\
%       $\colon$ & $\infty$ & {\bf \framebox{3}} & 2 & 1\\
%       $\colon$ & $\infty$ & 3 & \framebox{2} & 1\\
%       $\colon$ & $\infty$ & $\infty$ & 2 & \framebox{1}\\
%       \hline\hline
%       $\colon$ & {\bf \framebox{4}} & 3 & 2 & 1\\
%       $\colon$ & 4 & \framebox{3} & 2 & 1\\
%       $\colon$ & 4 & 3 & \framebox{2} & 1\\
%       $\colon$ & 4 & 3 & 2 & {\bf \framebox{0}}\\
%       \hline\hline
%     \end{tabular}
%     \caption{Изпълнение на алгоритъма}
%   \end{subtable}
%   \caption{Алгоритъм на Белман-Форд върху ориентиран граф с отрицателен цикъл,
%   като ребрата са подредени лексикографски: $\pair{a,e}, \pair{b,e}, \pair{c,b}, \pair{d,c}, \pair{d,e}, \pair{e, d}$}
%   \label{fig:bellman-ford-negative-cycle}
% \end{figure}


% % \begin{figure}[!htbp]
% %   
\begin{subfigure}[b]{0.3\textwidth}
    \begin{tikzpicture}[scale=0.9]
      %% nodes or vertices
      
      \foreach \nodename/\value/\x/\y/\direction/\navigate/\color in { 
        s/0/0/0/left/west/green, 
        x/\infty/3.5/1.5/above/north/black,
        y/\infty/1/-1.5/below/south/black,
        t/\infty/1/1.5/above/north/black, 
        z/\infty/3.5/-1.5/below/south/black}
      {
        \node[vertex, nodedecorate, fill=\color!25] (\nodename) at (\x,\y) {$\value$};
        \node [\direction] at (\nodename.\navigate) {$\nodename$};
      }
      % edges or lines
      \path
      \foreach \startnode/\endnode/\direction/\angle/\weight in {
        s/t/left/15/6, s/y/left/-25/7, y/z/below/0/9,  t/y/left/0/8, t/x/above/30/5, x/t/above/30/-2, z/x/right/0/7,
        z/s/below/0/2, y/x/below/-3, t/z/right/15/-4
      }
      {
        (\startnode) edge[arrowdecorate,bend left=\angle] node[\direction] {$\weight$} (\endnode)
      };
      ;
    \end{tikzpicture}
    \caption{Начален връх е $s$}
  \end{subfigure}
  \quad
  \begin{subfigure}[b]{0.3\textwidth}
    \begin{tikzpicture}[scale=0.9]
      \foreach \nodename/\value/\x/\y/\direction/\navigate/\color in { 
        s/0/0/0/left/west/green, 
        x/\infty/3.5/1.5/above/north/black,
        y/7/1/-1.5/below/south/red,
        t/6/1/1.5/above/north/red, 
        z/\infty/3.5/-1.5/below/south/black}
      {
        \node[vertex, nodedecorate, fill=\color!25] (\nodename) at (\x,\y) {$\value$};
        \node [\direction] at (\nodename.\navigate) {$\nodename$};
      }
    
    \path
    \foreach \startnode/\endnode/\angle in {
      s/t/15, s/y/-25}
    {
      (\startnode) edge[selected edge, bend left=\angle] node[] {} (\endnode)
    };
    
    %edges
    \path
    \foreach \startnode/\endnode/\direction/\angle/\weight in {
      s/t/left/15/6, s/y/left/-25/7, y/z/below/0/9,  t/y/left/0/8, t/x/above/30/5, x/t/above/30/-2, z/x/right/0/7,
      z/s/below/0/2, y/x/below/-3, t/z/right/15/-4}
    {
      (\startnode) edge[arrowdecorate,bend left=\angle] node[\direction] {$\weight$} (\endnode)
    };
    

  \end{tikzpicture}
  \caption{Започваме със съседите на $s$}
  \end{subfigure}
  \quad
  \begin{subfigure}[b]{0.3\textwidth}
    \begin{tikzpicture}[scale=0.9]
      
      \foreach \nodename/\value/\x/\y/\direction/\navigate/\color in { 
        s/0/0/0/left/west/green, 
        x/4/3.5/1.5/above/north/red,
        y/7/1/-1.5/below/south/blue,
        t/6/1/1.5/above/north/blue, 
        z/2/3.5/-1.5/below/south/red}
      {
        \node[vertex, nodedecorate, fill=\color!25] (\nodename) at (\x,\y) {$\value$};
        \node [\direction] at (\nodename.\navigate) {$\nodename$};
      }
    
    \path
    \foreach \startnode/\endnode/\angle in {
      s/t/15, s/y/-25}
    {
      (\startnode) edge[path edge, bend left=\angle] node[] {} (\endnode)
    };


    %edges or lines
    \path
      (y) edge[selected edge] node[below] {} (x)
      (t) edge[selected edge, bend left=15] node[left] {} (z);

    \path
    \foreach \startnode/\endnode/\direction/\angle/\weight in {
      s/t/left/15/6, s/y/left/-25/7, y/z/below/0/9,  t/y/left/0/8, t/x/above/30/5, x/t/above/30/-2, z/x/right/0/7,
      z/s/below/0/2, y/x/below/0/-3, t/z/right/15/-4
    }
    {
      (\startnode) edge[arrowdecorate,bend left=\angle] node[\direction] {$\weight$} (\endnode)
    };
    ;

  \end{tikzpicture}
  \caption{Продължаваме с $x$ и $z$}
  \end{subfigure}
\quad
\begin{subfigure}[b]{0.3\textwidth}
  \begin{tikzpicture}[scale=0.9]
    
    \foreach \nodename/\value/\x/\y/\direction/\navigate/\color in { 
      s/0/0/0/left/west/green, 
      x/4/3.5/1.5/above/north/blue,
      y/7/1/-1.5/below/south/blue,
      t/2/1/1.5/above/north/red, 
      z/2/3.5/-1.5/below/south/blue}
    {
      \node[vertex, nodedecorate, fill=\color!25] (\nodename) at (\x,\y) {$\value$};
      \node [\direction] at (\nodename.\navigate) {$\nodename$};
    }
    
    \path
    \foreach \startnode/\endnode/\angle in {
      s/y/-25, y/x/0, t/z/15}
    {
      (\startnode) edge[path edge, bend left=\angle] node[] {} (\endnode)
    };

    \path
    \foreach \startnode/\endnode/\angle in {
      x/t/30
    }
    {
      (\startnode) edge[selected edge,bend left=\angle] node[] {} (\endnode)
    };
    %edges or lines
    \path
    \foreach \startnode/\endnode/\direction/\angle/\weight in {
      s/t/left/15/6, s/y/left/-25/7, y/z/below/0/9,  t/y/left/0/8, t/x/above/30/5, x/t/above/30/-2, z/x/right/0/7,
      z/s/below/0/2, y/x/below/0/-3, t/z/right/15/-4
    }
    {
      (\startnode) edge[arrowdecorate,bend left=\angle] node[\direction] {$\weight$} (\endnode)
    };
    ;
  \end{tikzpicture}
  \caption{По-кратък път до $t$}
\end{subfigure}
\quad
\begin{subfigure}[b]{0.3\textwidth}
  \begin{tikzpicture}[scale=0.9]
    %% nodes or vertices
    \foreach \nodename/\value/\x/\y/\direction/\navigate/\color in { 
      s/0/0/0/left/west/green, 
      x/4/3.5/1.5/above/north/blue,
      y/7/1/-1.5/below/south/blue,
      t/2/1/1.5/above/north/blue, 
      z/-2/3.5/-1.5/below/south/red}
    {
      \node[vertex, nodedecorate, fill=\color!25] (\nodename) at (\x,\y) {$\value$};
      \node [\direction] at (\nodename.\navigate) {$\nodename$};
    }
    \path
    \foreach \startnode/\endnode/\angle in {
      s/y/-25, y/x/0, x/t/30}
    {
      (\startnode) edge[path edge, bend left=\angle] node[] {} (\endnode)
    };
    \path
    \foreach \startnode/\endnode/\angle in {
      t/z/15/
    }
    {
      (\startnode) edge[selected edge,bend left=\angle] node[] {} (\endnode)
    };
    %edges or lines
    \path
    \foreach \startnode/\endnode/\direction/\angle/\weight in {
      s/t/left/15/6, s/y/left/-25/7, y/z/below/0/9,  t/y/left/0/8, t/x/above/30/5, x/t/above/30/-2, z/x/right/0/7,
      z/s/below/0/2, y/x/below/0/-3, t/z/right/15/-4
    }
    {
      (\startnode) edge[arrowdecorate,bend left=\angle] node[\direction] {$\weight$} (\endnode)
    };
  \end{tikzpicture}
  \caption{По-кратък път до $z$}
  \end{subfigure}
\quad
\begin{subfigure}[b]{0.3\textwidth}
  \begin{tikzpicture}[scale=0.9]
    %% nodes or vertices
    \foreach \nodename/\value/\x/\y/\direction/\navigate/\color in { 
      s/0/0/0/left/west/green, 
      x/4/3.5/1.5/above/north/blue,
      y/7/1/-1.5/below/south/blue,
      t/2/1/1.5/above/north/blue, 
      z/-2/3.5/-1.5/below/south/blue}
    {
      \node[vertex, nodedecorate, fill=\color!25] (\nodename) at (\x,\y) {$\value$};
      \node [\direction] at (\nodename.\navigate) {$\nodename$};
    }
    \path
    \foreach \startnode/\endnode/\angle in {
      s/y/-25, y/x/0, x/t/30, t/z/15}
    {
      (\startnode) edge[path edge, bend left=\angle] node[] {} (\endnode)
    };
    %edges or lines
    \path
    \foreach \startnode/\endnode/\direction/\angle/\weight in {
      s/t/left/15/6, s/y/left/-25/7, y/z/below/0/9,  t/y/left/0/8, t/x/above/30/5, x/t/above/30/-2, z/x/right/0/7,
      z/s/below/0/2, y/x/below/0/-3, t/z/right/15/-4
    }
    {
      (\startnode) edge[arrowdecorate,bend left=\angle] node[\direction] {$\weight$} (\endnode)
    };
  \end{tikzpicture}
  \caption{Край на процедурата.}
\end{subfigure}


%%% Local Variables: 
%%% mode: latex
%%% TeX-master: "discrete-math"
%%% End: 

% %   \index{Белман-Форд!алгоритъм}
% %   \caption{Алгоритъм на Белман-Форд запазващ минималните пътища}
% %   \label{fig:bellman-ford-graph}
% % \end{figure}


% %% стр. 654
% \begin{problem}
%   Променете алгоритъма на Белман-Форд, така че $\delta(v) = -\infty$ за всеки връх $v$, 
%   за който има отрицателен цикъл по някой път от началния връх $s$ до $v$.
% \end{problem}



%%% Local Variables: 
%%% mode: latex
%%% TeX-master: "discrete-math"
%%% End: 

\chapter{Езици и автомати}

\section{Автоматни езици}

% Един от източниците е втора и трета глава от книгата на Сипсер, \cite{sipser}.
% Друг основен източник е книгата на Пападимитриу и Люис, \cite{papadimitriou}.
%По Сипсер, стр. 35
\begin{dfn}
  Краен автомат е петорка $\A = \FA$, където
  \begin{enumerate}[1)]
  \item
    $Q$ е крайно множество, наречено състояния;
  \item
    $\Sigma$ е крайно множество, наречено азбука;
  \item
    \marginpar{Тук нямаме $\varepsilon$-преходи}
    $\delta:Q\times\Sigma\to Q$ е (частична) функция на преходите;
  \item
    $s\in Q$ е начално състояние;
  \item
    $F\subseteq Q$ е множеството от финални състояния.
  \end{enumerate}
\end{dfn}

\index{тотален!автомат}
Ако функцията на преходите $\delta$ е {\bf тотална} функция, то казваме, 
че автоматът $\A$ е {\bf тотален}.

Нека имаме една дума $\alpha \in \Sigma^\star$, $\alpha = a_1a_2\cdots a_n$.
Казваме, че $\alpha$ се {\bf разпознава} от автомата $\A$, ако
съществува редица от състояния $q_0,q_1,q_2,\dots,q_n$, такива че:
\begin{itemize}
\item
  $q_0 = s$;
\item
  $\delta(q_i,a_{i+1}) = q_{i+1}$, за $i < n$;
\item
  $q_n \in F$.
\end{itemize}

% Алтернативен запис е следния:
% $(q,a\beta) \vdash (p,\beta)$, ако $\delta(q,a) = p$.
% $(q,\alpha\beta) \vdash^\star (p,\beta)$, ако $\delta^\star(q,\alpha) = p$.
% Тогава една дума $\alpha$ се разпознава от автомата, ако $(s,\alpha) \vdash^\star (p,\varepsilon)$ и $p \in F$.

Казваме, че $\A$ {\bf разпознава} езика $L$, ако $\A$ разпознава точно думите от $L$, т.е.
$L = \{\alpha \in \Sigma^\star \mid \A\mbox{ разпознава }\alpha\}$.
Обикновено означаваме езика, който се разпознава от даден автомат $\A$ с $L(\A)$.
\index{език!автоматен}
В такъв случай ще казваме, че езикът $L$ е {\bf автоматен}.

При дадена (частична) функция на преходите $\delta$,
често е удобно да разглеждаме (частичната) функция $\delta^\star:Q\times\Sigma^\star \to Q$, кято е дефинирана по следния начин:
\marginpar{Пример за индуктивна дефиниция по дължината на думата $\alpha$}
\begin{itemize}
\item 
  $\delta^\star(q,\varepsilon) = q$, за всяко $q\in Q$;
\item
  $\delta^\star(q,a\beta) = \delta^\star(\delta(q,a),\beta)$, за всяко $q\in Q$, всяко $a\in\Sigma$ и $\beta\in\Sigma^\star$.
\end{itemize}
Тогава \[L(\A) = \{\alpha\in\Sigma^\star \mid \delta^\star(s,\alpha) \in F\}.\]

\begin{prop}
  $(\forall q\in Q)(\forall\alpha,\beta\in\Sigma^\star)[\delta^\star(q,\alpha\beta) = \delta^\star(\delta^\star(q,\alpha),\beta)]$.
\end{prop}
\begin{proof}
  Индукция по дължината на $\alpha$.
\end{proof}

Нашата дефиниция на автомат позволява $\delta$ да бъде частична функция, т.е.
може да има $q\in Q$ и $a\in\Sigma$, за които $\delta(q,a)$ не е дефинирана.
Следващото твърдение ни казва, че ние съвсем спокойно можем да разглеждаме автомати
само с тотални функции на преходите  $\delta$.
\begin{prop}
  За всеки краен автомат $\A$, съществува {\em тотален} краен автомат $\A'$,
  за който $L(\A) = L(\A')$.
\end{prop}
\begin{proof}
  Нека $\A = \FA$.
  Дефинираме тоталния автомат 
  \[\A' = \pair{Q\cup\{q_e\}, \Sigma, \delta', s, F},\]
  като за всеки преход $(q,a)$, за който $\delta$ не е дефинирана, 
  дефинираме $\delta'$ да отива в новото състояние $q_e$.
  Ето и цялата дефиниция на новата функция на преходите $\delta'$:
  \begin{itemize}
  \item 
    $\delta'(q_e,a) = q_e$, за всяко $a\in\Sigma$;
  \item
    За всяко $q\in Q$, $a\in\Sigma$, ако $\delta(q,a) = p$, то
    $\delta'(q,a) = p$;
  \item
    За всяко $q\in Q$, $a\in\Sigma$, ако $\delta(q,a)$ не е дефинирано, то
    $\delta'(q,a) = q_e$.
  \end{itemize}
  Сега лесно може да се докаже, че $L(\A) = L(\A')$.
\end{proof}

\begin{prop}
  Класът на автоматните езици е затворен относно операцията {\bf обединение}.
  Това означава, че ако $L_1$ и $L_2$ са два произволни автоматни езика над азбуката $\Sigma$, то $L_1\cup L_2$
  също е автоматен език.
\end{prop}
\begin{proof}
  \marginpar{Защо изискваме $\A_1$ и $\A_2$ да са тотални?}
  Нека $L_1 = L(\A_1)$ и $L_2 = L(\A_2)$, 
  където $\A_1 = \FAn{1}$ и $\A_2 = \FAn{2}$ са {\bf тотални}.
  Определяме автомата $\A = \FA$, който разпознава $L_1\cup L_2$.
  \begin{itemize}
  \item
    $Q = Q_1\times Q_2$;
  \item
    Определяме за всяко $\pair{r_1,r_2} \in Q$ и всяко $a \in \Sigma$,
    \[\delta(\pair{r_1,r_2},a) = \pair{\delta_1(r_1,a),\delta_2(r_2,a)};\]
  \item
    $s = \pair{s_1,s_2}$;
  \item
    $F = \{\pair{r_1,r_2}\mid r_1\in F_1\vee r_2 \in F_2\} = (F_1\times Q_2)\cup (Q_1\times F_2)$.
  \end{itemize}
  Проверете, че $L(\A) = L(\A_1)\cup L(\A_2)$.
\end{proof}

\begin{cor}
  Класът на автоматните езици е затворен относно операцията {\bf сечение}.
  Това означава, че ако $L_1$ и $L_2$ са два произволни автоматни езика над азбуката $\Sigma$, то $L_1\cap L_2$
  също е автоматен език.
\end{cor}
\begin{proof}
  Същата конструкция на автомата $\A = \FA$, 
  с единствената разлика, че тук избираме финалните състояния да бъдат множеството:
  \[F = \{\pair{q_1,q_2} \mid q_1 \in F_1\ \&\ q_2 \in F_2\} = F_1\times F_2.\]
\end{proof}

\begin{prop}
  Нека $L$ е автоматен език.
  Тогава $\Sigma^\star\setminus L$ също е автоматен език.
\end{prop}
\begin{proof}
  Нека $L = L(\A)$, където $\A$ е тотален.
  Тогава $\Sigma^\star\setminus L = L(\A^\prime)$,
  където $\A^\prime$ е същия като $\A$, само $F^\prime = Q\setminus F$.
\end{proof}


\begin{problem}
  За всеки от следните езици $L$, постройте автомат $\A$, който разпознава $L$.
  \begin{enumerate}[a)]
  \item 
    $L = \{a^nb\mid n \geq 0\}$;
  \item
    $L = \{\varepsilon, a,b\}$;
  \item
    $L = \emptyset$;
  \item
    $L = \{a,b\}^\star\setminus\{\varepsilon\}$;
  \item
    $L = \{a^nb^m\mid n,m \geq 0\}$;
  \item
    $L = \{a^nb^m\mid n,m \geq 1\}$;
  \item
    $L = \{a,b\}^\star \setminus \{a\}$;
  \item
    $L = \{w \in \{a,b\}^\star \mid \mbox{съдържа поне две }a\}$;
  \item
    \marginpar{(сечение на два автоматни езика)}
    $L = \{w \in \{a,b\}^\star \mid \mbox{съдържа поне две }a\mbox{ и поне едно }b\}$;
  \item
    $L = \{w \in \{a,b\}^\star \mid \mbox{на всяка нечетна позиция на }w\mbox{ е буквата }a\}$;
  \item
    $L = \{w \in \{a,b\}^\star \mid w\mbox{ съдържа четен брой }a\mbox{ и най-много едно }b\}$;
  \item
    $L = \{w \in \{a,b\}^\star\mid \abs{w} \leq 3\}$;
  \item
    $L = \{w \in \{a,b\}^\star \mid w \mbox{ започва с }ab\}$;
  \item
    $L = \{w \in \{a,b\}^\star \mid w \mbox{ завършва с }ab\}$;
  \item
    $L = \{w \in \{a,b\}^\star \mid w \mbox{ съдържа }bab\}$;
  \item
    $L = \{w \in \{a,b\}^\star \mid w \mbox{ не съдържа }bab\}$;
  \item
    \marginpar{(решена е по-долу)}
    $L = \{w \in \{a,b\}^\star \mid w \mbox{ няма две последователни }a\}$;
  \item
    $L = \{w \in \{a,b\}^\star \mid w\mbox{ започва и завършва с буквата } a\}$;
  \item
    $L = \{w \in \{a,b\}^\star \mid w\mbox{ започва и завършва с една и съща буква}\}$;
  \item
    $L = \{w \in \{a,b\}^\star \mid w \mbox{ съдържа точно едно }a\}$;
  \item
    $L = \{w \in \{a,b\}^\star \mid \mbox{ всяко }a\mbox{ в }w\mbox{ се следва от поне едно }b\}$;
  \item
    $L = \{w \in \{a,b\}^\star \mid \abs{w} \equiv 0 \bmod 3\}$;
  \item
    \marginpar{$N_a(w)$ - броят на срещанията на буквата $a$ в думата $w$}
    $L = \{w \in \{a,b\}^\star \mid N_a(w) \equiv 0 \bmod 3\}$;
  \item
    $L = \{w \in \{a,b\}^\star \mid N_b(w) \equiv 1 \bmod 2\}$;
  \item
    $L = \{w \in \{a,b\}^\star \mid N_a(w) \equiv 0 \bmod 3\ \&\ N_b(w) \equiv 1 \bmod 2\}$;
  \item
    $L = \{w \in \{a,b\}^\star \mid N_a(w) \equiv 0 \bmod 2\ \&\ N_b(w) \equiv 1 \bmod 2\}$;
  \item
    $L = \{w \in \{a,b\}^\star \mid N_a(w) \equiv 0 \bmod 2\ \vee\ w \mbox{ съдържа точно две }b\}$;
  \end{enumerate}
\end{problem}
\begin{solution}
  \begin{itemize}
  \item
    Да разгледаме $L = \{w \in \{a,b\}^\star \mid \mbox{съдържа }bab\}$.
    \begin{figure}[H]
      \begin{center}
        \begin{tikzpicture}[->,>=stealth,thick,node distance=45pt]
          \tikzstyle{every state}=[circle,minimum size=20pt,auto]
          
          \node[initial, state]   (0) {$s$};
          \node[state]            (1) [right of=0]{$q_1$};
          \node[state]            (2) [right of=1]{$q_2$};
          \node[state,accepting]  (3) [right of=2]{$q_3$};
          
          \path 
          (0) edge [loop above]   node [above] {$a$}    (0)
          (0) edge [bend left=15] node [above] {$b$}    (1)
          (1) edge [loop above]   node [above] {$b$}    (1)
          (1) edge [bend left=15] node [above] {$a$}    (2)
          (2) edge [bend left=30] node [below] {$a$}    (0)
          (2) edge [bend left=15] node [above] {$b$}    (3)
          (3) edge [loop above]   node [above] {$a,b$}  (3);
        \end{tikzpicture}
      \end{center}
    \end{figure}    
  \item
    Да разгледаме $L = \{w \in \{a,b\}^\star \mid \mbox{съдържа поне две }a\}$.
    \begin{figure}[H]
      \begin{center}
        \begin{tikzpicture}[->,>=stealth,thick,node distance=45pt]
          \tikzstyle{every state}=[circle,minimum size=20pt,auto]
          
          \node[initial, state]   (0) {$s$};
          \node[state]            (1) [right of=0]{$q_1$};
          \node[state,accepting]  (2) [right of=1]{$q_2$};
          
          \path 
          (0) edge [loop above]   node [above] {$b$}    (0)
          (0) edge [bend left=15] node [above] {$a$}    (1)
          (1) edge [loop above]   node [above] {$b$}    (1)
          (1) edge [bend left=15] node [above] {$a$}    (2)
          (2) edge [loop above]   node [above] {$a,b$}  (2);
        \end{tikzpicture}
      \end{center}
    \end{figure}    
  \item 
    Да разгледаме $L = \{w \in \{a,b\}^\star \mid \mbox{ всяко }a\mbox{ в }w\mbox{ се следва от поне едно }b\}$.
    \begin{figure}[H]
      \begin{center}
        \begin{tikzpicture}[->,>=stealth,thick,node distance=45pt]
          \tikzstyle{every state}=[circle,minimum size=20pt,auto]
          
          \node[initial, accepting, state] (0) {$s$};
          \node[state]                     (1) [right of=0]{$q_1$};
          
          \path 
          (0) edge [loop above]   node [above] {$b$}   (0)
          (0) edge [bend left=15] node [above] {$a$}   (1)
          (1) edge [bend left=15] node [below] {$b$}   (0);
        \end{tikzpicture}
      \end{center}
    \end{figure}
  \end{itemize}
\end{solution}

В повечето от горните задачи е лесно да се съобрази, че построения автомат разпознава желания език.
При по-сложни задачи обаче, ще се наложи да дадем доказателство, като обикновено се прилага 
{\em метода на математическата индукция} върху дължината на думите в езика. 
Ще разгледаме няколко такива примера.

\begin{problem}
  Докажете, че езикът $L$ е автоматен, където
  \[L = \{\alpha \in \{a,b\}^\star\ \mid\ \alpha\mbox{ не съдържа две поредни срещания на }a\}.\]
\end{problem}
\begin{proof}
  Да разгледаме $\A = \FA$ с функция на преходите
  \begin{figure}[H]
    \begin{center}
      \begin{tikzpicture}[->,>=stealth,thick,node distance=45pt]
        \tikzstyle{every state}=[circle,minimum size=15pt,auto]
        
        \node[initial, accepting, state] (0) {$s$};
        \node[accepting, state]   (1) [right of=0]{$q_1$};
        \node[state]   (2) [right of=1]{$q_2$};
        
        \path 
        (0) edge [loop above]   node [above] {$b$}   (0)
        (0) edge [bend left=15] node [above] {$a$}   (1)
        (1) edge [bend left=15] node [below] {$b$}   (0)
        (1) edge [bend left=15] node [above] {$a$}   (2)
        (2) edge [loop above]   node [above] {$a,b$} (2);
      \end{tikzpicture}
    \end{center}
 \end{figure}

 Трябва да докажем, че наистина $L = L(\A)$.
 Първо ще се концентрираме върху доказателството на $L(\A) \subseteq L$.
 \marginpar{$\abs{\alpha}$ - дължината на думата $\alpha$}
 Ще докажем с индукция по дължината на думата $\alpha$, че:
 \begin{enumerate}[(1)]
 \item 
   ако $\delta^\star(s,\alpha) = s$, то
   $\alpha$ не съдържа две поредни срещания на $a$
   и ако $\abs{\alpha} > 0$, то $\alpha$ завършва на $b$;
 \item
   ако $\delta^\star(s,\alpha) = q_1$, то
   $\alpha$ не съдържа две поредни срещания на $a$
   и завършва на $a$.
 \end{enumerate}

 За $\abs{\alpha} = 0$, то твърденията (1) и (2) са ясни (Защо?).
 Да приемем, че твърденията $(1)$ и $(2)$ са верни за произволни думи $\alpha$ с дължина $n$.
 Нека $\abs{\alpha} = n+1$, т.е. $\alpha = \beta x$, където $\abs{\beta} = n$.
 Ще докажем (1) и (2) за $\alpha$.
 \begin{itemize}[-]
 \item 
   Нека $\delta^\star(s,\beta x) = s = \delta(\delta^\star(s,\beta),x)$.
   Според дефиницията на функцията $\delta$, $x = b$ и $\delta^\star(s,\beta) \in \{s,q_1\}$.
   Тогава по {\bf И.П.} за (1) и (2), $\beta$ не съдържа две поредни срещания на $a$.
   Тогава е очевидно, че $\beta x$ също не съдържа две поредни срещания на $a$.
 \item
   Нека $\delta^\star(s,\beta x) = q_1 = \delta(\delta^\star(s,\beta),x)$.
   Според дефиницията на $\delta$, $x = a$ и $\delta^\star(s,\beta) = s$.
   Тогава по {\bf И.П.} за (2), $\beta$ не съдържа две поредни срещания на $a$
   и завършва на $b$.
   Тогава е очевидно, че $\beta x$ също не съдържа две поредни срещания на $a$.
 \end{itemize}
 
 Така доказахме с индукция по дължината на думата, че за всяка дума $\alpha$
 са  изпълнени твърденията $(1)$ и $(2)$. Това означава, че ако $\alpha \in L(\A)$,
 то $\delta^\star(s,\alpha) \in \{s,q_1\}$ и от $(1)$ и $(2)$ следва, че и в двата случа
 $\alpha$ не съдържа две поредни срещания на буквата $a$, т.е. $\alpha \in L$.
 С други думи, доказахме, че 
 \[L(\A) \subseteq L.\]

 Сега ще докажем другата посока, т.е. $L \subseteq L(\A)$.
 Това означава да докажем, че
 \[(\forall \alpha \in \Sigma^\star)[\alpha \in L\ \Rightarrow\ \delta^\star(s,\alpha) \in F],\]
 \marginpar{$p \Rightarrow q \equiv \neg q \Rightarrow \neg  p$}
 което е еквивалентно на
 \begin{equation}
   \label{eq:case2}
   (\forall \alpha \in \Sigma^\star)[\delta^\star(s,\alpha) \not\in F \ \Rightarrow\ \alpha\not\in L].
 \end{equation}
 Това е лесно да се съобрази.
 Щом $\delta^\star(s,\alpha) \not\in F$, то 
 $\delta^\star(s,\alpha) = q_2$ и думата $\alpha$ може да се представи по следния начин:
 \[\alpha = \beta a \gamma\ \&\ \delta^\star(s,\beta) = q_1.\]
 
 Сега, използвайки свойство (2) от по-горе, понеже $\delta^\star(s,\beta) = q_1$, то
 $\beta$ не съдържа две поредни срещания на $a$, но завършва на $a$.
 Сега е очевидно, че $\beta a$ съдържа две поредни срещания на $a$ и 
 щом $\beta a$ е префикс на $\alpha$, то думата $\alpha \not\in L$.
 С това доказахме Свойство \ref{eq:case2} и следователно и посоката $L\subseteq L(\A)$.
\end{proof}

\begin{framed}
  За една дума $\alpha \in \{0,1\}^\star$, 
  нека с $\alpha_{(2)}$ да означим числото в десетична бройна система, което се представя в двоична бройна система като $\alpha$.
  Например, $1101_{(2)} = 1 \cdot 2^3+1\cdot 2^2+0\cdot 2^1+1\cdot 2^0 = 13$.
  Тогава имаме следните свойства:
  \begin{itemize}
  \item
    $\varepsilon_{(2)} = 0$,
  \item
    $(\alpha0)_{(2)} = 2\cdot(\alpha)_{(2)}$,
  \item
    $(\alpha1)_{(2)} = 2\cdot(\alpha)_{(2)} + 1$.
  \end{itemize}
\end{framed}

\begin{problem}
  Докажете, че $L = \{w \in \{0,1\}^\star \mid w_{(2)} \equiv 2 \mod 3\}$ е автоматен.
\end{problem}
\begin{proof}
  Нашият автомат ще има три състояния $\{q_0,q_1,q_2\}$, като началното състояние ще бъде $q_0$.
  Целта ни е да дефинираме така автомата, че да имаме следното свойство:
  \[(\forall\alpha\in\Sigma^\star)(\forall i < 3)[\alpha_{(2)} \equiv i\bmod 3\ \Leftrightarrow\ \delta^\star(q_0,\alpha) = q_i],\]
  т.е. всяко състояние отговаря на определен остатък при деление на три.
  Понеже искаме нашия автомат да разпознава тези думи $\alpha$,
  за които $\alpha_{(2)} \equiv 2\mod 3$, финалното състояние ще бъде $q_2$.
  Дефинираме функцията $\delta$ следвайки следните свойства:
  \begin{enumerate}[(1)]
  \item
    \marginpar{$\delta(q_0,0) = q_0$}
    $\alpha_{(2)} \equiv 0 \bmod 3\ \Rightarrow\ (\alpha0)_{(2)} \equiv 0 \bmod 3$;
  \item 
    \marginpar{$\delta(q_0,1) = q_1$}
    $\alpha_{(2)} \equiv 0 \bmod 3\ \Rightarrow\ (\alpha1)_{(2)} \equiv 1 \bmod 3$;
  \item
    \marginpar{$\delta(q_1,0) = q_2$}
    $\alpha_{(2)} \equiv 1 \bmod 3\ \Rightarrow\ (\alpha0)_{(2)} \equiv 2 \bmod 3$;
  \item 
    \marginpar{$\delta(q_1,1) = q_0$}
    $\alpha_{(2)} \equiv 1 \bmod 3\ \Rightarrow\ (\alpha1)_{(2)} \equiv 0 \bmod 3$;
  \item
    \marginpar{$\delta(q_2,0) = q_1$}
    $\alpha_{(2)} \equiv 2 \bmod 3\ \Rightarrow\ (\alpha0)_{(2)} \equiv 1 \bmod 3$;
  \item 
    \marginpar{$\delta(q_2,1) = q_2$}
    $\alpha_{(2)} \equiv 2 \bmod 3\ \Rightarrow\ (\alpha1)_{(2)} \equiv 2 \bmod 3$.
  \end{enumerate}
  
  Ето и картинка на автомата:
  \begin{figure}[H]
%    \begin{subfigure}[b]{0.3\textwidth}% [$L_1 = L(M_1)$]{
    \begin{center}
      \begin{tikzpicture}[->,>=stealth,thick,node distance=45pt]
        \tikzstyle{every state}=[circle,minimum size=15pt,auto]
        
        \node[initial,state]      (0) {$q_0$};
        \node[state]              (1) [right of=0]{$q_1$};
        \node[accepting, state]   (2) [right of=1]{$q_2$};
        
        \path 
        (0) edge  [loop above]    node [above] {$0$} (0)
        (0) edge  [bend left=15]  node [above] {$1$} (1)
        (1) edge  [bend right=15] node [below]  {$0$} (2)
        (1) edge  [bend left=15]  node [below] {$1$} (0)
        (2) edge  [bend right=15] node [above] {$0$} (1)
        (2) edge  [loop right]    node [right] {$1$} (2);
      \end{tikzpicture}
      \end{center}
 %   \end{subfigure}
 \end{figure}
  Да разгледаме твърденията:
  \begin{enumerate}[(1)]
  \item 
    $\delta^\star(q_0,\alpha) = q_0\ \Rightarrow\ \alpha_{(2)} \equiv 0 \mod 3$;
  \item 
    $\delta^\star(q_0,\alpha) = q_1\ \Rightarrow\ \alpha_{(2)} \equiv 1 \mod 3$;
  \item 
    $\delta^\star(q_0,\alpha) = q_2\ \Rightarrow\ \alpha_{(2)} \equiv 2 \mod 3$.
  \end{enumerate}
  Ще докажем (1), (2) и (3) едновременно с индукция по дължината на думата $\alpha$.
  За $\abs{\alpha} = 0$, всички условия са изпълнени. (Защо?)
  Да приемем, че (1), (2) и (3) са изпълнени за думи с дължина $n$.
  Нека $\abs{\alpha} = n+1$, т.е. $\alpha = \beta x$, $\abs{\beta} = n$.
  За да приложим индукционното предположение, ще използваме следното свойство:
  \[\delta^\star(q_0,\beta x) = \delta(\delta^\star(q_0,\beta),x).\]
  
  Ще докажем подробно само (3) понеже другите твърдения се доказват по сходен начин.
  \marginpar{Обърнете внимание, че в доказателството на (3) използваме И.П. не само за (3), но и за (2)}
  Нека $\delta^\star(q_0,\beta x) = q_2$. 
  Имаме два случая:
  \begin{itemize}
  \item 
    $x = 0$. 
    Тогава, по дефиницията на $\delta$, 
    $\delta(q_1,0) = q_2$ и следователно, $\delta^\star(q_0,\beta) = q_1$.
    По {\bf И.П.} за (2) с $\beta$,
    \[\delta^\star(q_0,\beta) = q_1\ \Rightarrow\ \beta_{(2)} \equiv 1 \bmod 3\]
    Тогава, $(\beta0)_{(2)} \equiv 2 \mod 3$. Така доказахме, че
    \[\delta^\star(q_0,\beta 0) = q_2\ \Rightarrow\ (\beta 0)_{(2)} \equiv 2 \bmod 3.\]
  \item
    $x = 1$.
    Тогава, по дефиницията на $\delta$, $\delta(q_2,1) = q_2$ и следователно,
    $\delta^\star(q_0,\beta) = q_2$.
    По {\bf И.П.} за (3) с $\beta$,
    \[\delta^\star(q_0,\beta) = q_2\ \Rightarrow\ \beta_{(2)} \equiv 2 \mod 3.\]
    Тогава, $(\beta1)_{(2)} \equiv 2 \mod 3$. Така доказахме, че
    \[\delta^\star(q_0,\beta 1) = q_2\ \Rightarrow\ (\beta 1)_{(2)} \equiv 2 \mod 3.\]
  \end{itemize}
  
  За да докажем (1), нека $\delta^\star(q_0,\beta x) = q_0$. 
  \begin{itemize}
  \item 
    $x = 0$. Разсъжденията са аналогични, като използваме {\bf И.П.} за (1).
  \item
    $x = 1$. Разсъжденията са аналогични, като използваме {\bf И.П.} за (2).
  \end{itemize}
  
  По същия начин доказваме и (2). Нека $\delta^\star(q_0,\beta x) = q_1$. 
  \begin{itemize}
  \item 
    При $x = 0$, използваме {\bf И.П.} за (3).
  \item
    При $x = 1$, използваме {\bf И.П.} за (1).
  \end{itemize}

  От (1), (2) и (3) следва директно, че $L(\A) \subseteq L$.
  За другата посока, нека $\alpha \in L$, т.е. $(\alpha)_{(2)} \equiv 2 \mod 3$.
  Ако допуснем, че $\alpha \not\in L(\A)$, то това означава, че $\delta^\star(q_0,\alpha) \in \{q_0,q_2\}$.
  Но в тези случаи получаваме от твърдения (1) и (2), че $(\alpha)_{(2)} \equiv 0 \mod 3$ или $(\alpha)_{(2)} \equiv 1 \mod 3$.
  Това е противоречие. Следователно, ако $\alpha \in L$, то $\delta(q_0,\alpha) = q_2$.
\end{proof}

\section{Регулярни езици}

\begin{dfn}
  \index{регулярен език}
  \index{регулярен израз}
  \marginpar{Това е пример за индуктивна дефиниция.}
  Нека е дадена азбука $\Sigma$. Дефинираме множеството от
  {\em регулярни езици} над азбуката $\Sigma$ и едновременно с това 
  множеството от {\em регулярни изрази}, които разпознават тези езици.
  \begin{enumerate}[1)]
  \item
   за всеки символ $a \in \Sigma$, $\{a\}$ е регулярен език,
   който се разпознава от регулярния израз $a$;
  \item
    $\{\varepsilon\}$ е регулярен език,
    който се разпознава от регулярния израз $\varepsilon$;
  \item
    $\emptyset$ е регулярен език,
    който се разпознава от регулярния израз $\emptyset$;
  \item
    $L_1\cup L_2$, където $L_1$ и $L_2$ са регулярни езици,
    който се разпознава от регулярния израз $(r_1 + r_2)$,
    където $r_1$ и $r_2$ са регулярните изрази за $L_1$ и $L_2$;
  \item
    \marginpar{Конкатенация. Обикновено изпускаме знака $\cdot$}
    $L_1\cdot L_2 = \{uw\mid u \in L_1\ \&\ w \in L_2\}$, където $L_1$ и $L_2$ са регулярни езици,
    който се разпознава от регулярния израз $(r_1\cdot r_2)$,
    където $r_1$ и $r_2$ са регулярните изрази за $L_1$ и $L_2$;
  \item
    \marginpar{Звезда на Клини}
    $L^\star = \{w_1w_2\cdots w_n\mid n \in \Nat\ \&\ w_i \in L\mbox{ за всяко } i \leq n\}$,  където $L$ е регулярен език,
    който се разпознава от регулярния израз $(r^\star)$,
    където $r$ е регулярния израз за $L$.
    Можем да запишем, че $L^\star = \bigcup_n L^n$, където
    $L^0 = \{\varepsilon\}$ и $L^{n+1} = L^n\cdot L$.    
  \end{enumerate}
\end{dfn}

\begin{example}
  Нека да разгледаме няколко примера какво точно представлява прилагането
  на операцията звезда на Клини върху един език.
  \begin{itemize}
  \item 
    Нека $L = \{0,11\}$. Тогава:
    \begin{itemize}
    \item 
      $L^0 = \{\varepsilon\}$, $L^1 = L$,
    \item
      $L^2 = L^1\cdot L^1 = \{00,011,110,1111\}$,
    \item
      $L^3 = L^1\cdot L^2 = \{000,0011,0110,01111,1100,11011,11110,111111\}$.
    \end{itemize}
  \item
    Нека $L = \emptyset$.
    Тогава:
    \begin{itemize}
    \item 
      $L^0 = \{\varepsilon\}$,
    \item
      $L^1 = \emptyset$,
    \item
      $L^2 = L^1 \cdot L^1 = \emptyset$.
    \end{itemize}    
    Получаваме, че $L^\star = \{\varepsilon\}$, т.е. {\em краен} език
  \item
    Нека $L = \{0^i\mid i \in \Nat\} = \{\varepsilon, 0, 00, 000, \dots\}$.
    Тогава лесно може да се види, че $L = L^\star$.
  \end{itemize}
\end{example}

\begin{problem}
  За произволни регулярни изрази $r$ и $s$, 
  проверете:
  \begin{enumerate}[a)]
  \item 
    $r+s = s + r$;
  \item
    $(\varepsilon + r)^\star = r^\star$;
  \item
    $\emptyset^\star = \varepsilon$;
  \item
    $(r^\star s^\star) = (r+s)^\star$;
  \item
    $(r^\star)^\star = r^\star$;
  \item
    $(rs + r)^\star r = r(sr+r)^\star$;
  \item
    $s(rs+s)^\star r = rr^\star s(rr^\star s)^\star$;
  \item
    $(r+s)^\star = r^\star + s^\star$;
  \item
    $\emptyset^\star = \varepsilon^\star$;
  \end{enumerate}
\end{problem}


\begin{lemma}
  Ако един език е автоматен, то той се описва с регулярен израз.
\end{lemma}
\begin{proof}
  %\marginpar{От \cite{papadimitriou}, стр. 79, \cite{hopcroft}, стр. 33}
  Нека  $L = L(M)$, за някой краен детерминиран автомат $M$.
  Да фиксираме едно изброяване на състоянията $Q = \{q_1,\dots,q_n\}$,
  като началното състояние е $q_1$.
  Ще означаваме с $L(i,j,k)$ множеството от тези думи, които
  могат да се разпознаят от автомата по път, който започва от $q_i$,
  завършва в $q_j$, и междинните състояния имат индекси $\leq k$.
  Тогава за $n = \abs{Q}$, 
  \[L(i,j,n) = \{\alpha\in\Sigma^\star\mid \delta^\star(q_i,\alpha) \in F\}.\]
  Следователно,
  \[L(M) = \bigcup\{L(1,j,n)\mid q_j \in F\}.\]
  Ще докажем с {\bf индукция по $k$}, че за всяко $i,j,k$, множествата от думи $L(i,j,k)$
  се описват с регулярен израз $r^k_{i,j}$
  \begin{enumerate}[a)]
  \item
    Нека $k = 0$. Ще докажем, че за всяко $i,j$, $L(i,j,0)$ се описва с регулярен израз.
    Имаме да разгледаме два случая.
    
    Ако $i = j$, то 
    \[L(i, j, 0) = \{\varepsilon\}\cup\{a\in\Sigma \mid \delta(q_i,a) = q_j\}.\]
    Ако $i \neq j$, то
    \[L(i, j, 0) = \{a\in\Sigma \mid \delta(q_i, a) = q_j\}.\]
  \item
    Да предположим, че $k > 0$ и за всяко $i$, $j$, можем да намерим регулярните изрази
    съответстващи на $L(i,j,k-1)$. Тогава
    \[L(i,j,k) = L(i,j,k-1)\ \cup\ L(i,k,k-1)\cdot (L(k,k,k-1)^\star) \cdot L(k,j,k-1).\]
    Тогава по {\bf И.П.} следва, че $L(i,j,k)$ може да се опише с регулярен израз, който е
    \[r^{k-1}_{i,j} + r^{k-1}_{i,k}\cdot (r^{k-1}_{k,k})^\star\cdot r^{k-1}_{k,j}.\]
  \end{enumerate}
  Заключаваме, че за всяко $i,j,k$, $L(i,j,k)$ може да се опише с регулярен израз $r^{k}_{i,j}$.
  Тогава ако $F = \{q_{i_1},\dots,q_{i_k}\}$, то $L(M)$ се описва с регулярния израз
  \[r^n_{1,i_1} + r^n_{1,i_2} + \dots + r^n_{1,i_k}.\]
\end{proof}

\begin{example}
  \label{fig:a1}
  Да разгледаме следния автомат:
  
  \begin{figure}[H]
    \begin{center}
      \begin{tikzpicture}[->,>=stealth,thick,node distance=45pt]
        \tikzstyle{every state}=[circle,minimum size=15pt,auto]
        
        \node[initial,state]      (1) {$q_1$};
        \node[accepting, state]   (2) [right of=1]{$q_2$};
        
        \path 
        (1) edge [loop above]  node [above] {$1$} (1)
        (1) edge  node [above] {$0$} (2)
        (2) edge [loop above] node [above] {$0,1$} (2);
      \end{tikzpicture}
      \end{center}
 \end{figure}

 За да намерим регулярния език за автомата от Пример \ref{fig:a1}, 
 трябва да намерим $r^2_{1,2}$, защото началното състояние е $q_1$, финалното е $q_2$ и 
 броят на състоянията в автомата е $2$.
 \begin{align*}
   r^0_{1,1} =\ & \varepsilon + 1,\\
   r^0_{1,2} =\ & 0,\\
   r^0_{2,1} =\ & \emptyset,\\
   r^0_{2,2} =\ & \varepsilon +  0 + 1,\\
    r^1_{1,2} =\ & r^0_{1,2} + r^0_{1,1}\cdot(r^0_{1,1})^\star \cdot r^0_{1,2} = 0 + (\varepsilon + 1)(\varepsilon + 1)^\star0 = 1^\star0,\\
    r^1_{2,2} =\ & r^0_{2,2} + r^0_{2,1} \cdot (r^0_{1,1})^\star\cdot r^0_{1,2} = \varepsilon + 0 + 1 + \emptyset(\varepsilon + 1)^\star0 = \varepsilon + 0 + 1\\
    r^2_{1,2} =\ & r^{1}_{1,2} + r^{1}_{1,2}(r^1_{2,2})^\star r^1_{2,2} \\
    =\ & 1^\star0 + 1^\star0 (\varepsilon + 0 + 1)^\star (\varepsilon + 0 + 1) = 1^\star 0 (0 + 1)^\star.
  \end{align*}
Ясно е, че $L_1$ се описва с регулярния израз $r^2_{1,2} = 1^\star 0 (0 + 1)^\star$.
\end{example}

Следващата ни цел е да видим, че всеки регулярен език е автоматен.

\section{Недетерминирани крайни автомати}

\begin{dfn}
  \marginpar{Въведени от Рабин и Скот \cite{rabin-scott}}
  Недетерминиран краен автомат представлява
  \[\N = \NFA,\]
  \begin{itemize}
  \item
    $Q$ е крайно множество от състояния;
  \item
    $\Sigma$ е крайна азбука;
  \item
    $\Delta: Q\times\Sigma \to \Ps(Q)$ е функцията на преходите.
    \marginpar{Да напомним, че $\Ps(Q) = \{R\mid R\subseteq Q\}$, $\abs{\Ps(Q)} = 2^{\abs{Q}}$}
    Обърнете внимание, че тя е тотална.
  \item
    $s \in Q$ е началното състояние;
  \item
    $F\subseteq Q$ е множеството от финални състояния.
  \end{itemize}
\end{dfn}

\begin{thm}
  За всеки НKА $\N$ съществува еквивалентен на него ДКА $\D$, т.е. $L(\N) = L(\D)$.
\end{thm}
\begin{proof}
  Нека $\N = \NFA$. Ще построим ДКА $\D = (Q',\Sigma,\delta,s',F')$.
  Конструкцията е следната:
  \marginpar{Да отбележим, че детерминираният автомат $\D$ има не повече от $2^{\abs{Q}}$ на брой състояния}
  \begin{itemize}
  \item
    $Q' = \Ps(Q)$;
  \item
    $\delta(R,a) = \{q\in Q\mid (\exists r\in R)[q\in\Delta(r,a)]\} = \bigcup_{r\in R}\Delta(r,a)$;
  \item
    $s' = \{s\}$;
  \item
    $F' = \{R \subseteq Q \mid R\cap F \neq \emptyset\}$.
  \end{itemize}
\end{proof}

\begin{lemma}
  Съществува НКА $\N = \NFA$, който разпознава езика $L(r)$, 
  където $r = \emptyset$, $r = \varepsilon$ или $r = a$, за $a\in \Sigma$.
\end{lemma}
\begin{proof}
  \begin{figure}[H]
    \begin{subfigure}[b]{0.3\textwidth}
      \label{subf:a1}
      \caption{$L(\emptyset)$}
      \begin{tikzpicture}[->,>=stealth,thick,node distance=35pt]
        \tikzstyle{every state}=[circle,minimum size=15pt,auto]
        \node[initial,state]      (1) {$s$};
      \end{tikzpicture}
    \end{subfigure}
    \begin{subfigure}[b]{0.3\textwidth}
      \caption{$L(\varepsilon)$}
      \begin{tikzpicture}[->,>=stealth,thick,node distance=35pt]
        \tikzstyle{every state}=[circle,minimum size=15pt,auto]
        \node[initial,state,accepting]      (1) {$s$};
      \end{tikzpicture}
    \end{subfigure}
    \begin{subfigure}[b]{0.3\textwidth}
      \caption{$L(a)$}
      \begin{tikzpicture}[->,>=stealth,thick,node distance=35pt]
        \tikzstyle{every state}=[circle,minimum size=15pt,auto]
        \node[initial,state]      (1)              {$s$};
        \node[accepting,state]    (2) [right of=1] {$q$};
        \path 
        (1) edge  node [above] {$a$} (2);
      \end{tikzpicture}
    \end{subfigure}
  \end{figure}
\end{proof}


Сега ще видим, че автоматните езици  са затворени относно още две операции.
\begin{lemma}
  Класът на автоматните езици е затворен относно операцията {\bf конкатенация}.
  Това означава, че ако $L_1$ и $L_2$ са два произволни автоматни езика, то $L_1\cdot L_2$
  също е автоматен език.
\end{lemma}
\begin{proof}
  Нека са дадени автоматите:
  \begin{itemize}
  \item
    $\N_1 = \NFAn{1}$, като $L(\N_1) = L_1$;
  \item
    $\N_2 = \NFAn{2}$, като $L(\N_2) = L_2$.
  \end{itemize}
  Ще дефинираме автомата $\N = \NFA$ като
  \[L(\N) = L_1\cdot L_2 = L(\N_1)\cdot L(\N_2).\]
  \begin{itemize}
  \item
    $Q = Q_1 \cup Q_2$;
  \item
    $s = s_1$;
  \item
    $F = 
    \begin{cases}
      F_1 \cup F_2, & \text{ ако } s_2 \in F_2\\
      F_1,          & \text{ иначе}.
    \end{cases}$
  \item 
    $\Delta(q,a) = 
    \begin{cases}
      \Delta_1(q,a),                      & \text{ ако }q\in Q_1\setminus F_1\ \&\ a\in\Sigma\\
      \Delta_2(q,a),                      & \text{ ако }q\in Q_2\ \&\ a\in\Sigma\\
      \Delta_1(q,a) \cup \Delta_2(s_2,a), & \text{ ако }q \in F_1\ \&\ a\in\Sigma.
    \end{cases}$
  \end{itemize}
\end{proof}

\begin{lemma}
  Класът от автоматните езици е затворен относно операцията {\bf обединение}.
\end{lemma}
\begin{proof}
  Нека са дадени автоматите:
  \begin{itemize}
  \item 
    $\N_1 = \NFAn{1}$, като $L(\N_1) = L_1$;
  \item
    $\N_2=\NFAn{2}$, като $L(\N_2) = L_2$.
  \end{itemize}
  Ще дефинираме автомата $\N=\NFA$, така че
  \[L(\N) = L(\N_1) \cup L(\N_2).\]
  \begin{itemize}
  \item 
    $Q = Q_1 \cup Q_2 \cup \{s\}$;
  \item
    $F = 
    \begin{cases}
      F_1 \cup F_2 \cup \{s\}, & \text{ ако } s_1 \in F_1 \vee s_2 \in F_2\\
      F_1 \cup F_2,            & \text{ иначе } 
    \end{cases}$
  \item
    $
    \Delta(q,a) = 
    \begin{cases}
      \Delta_1(q,a),                       & \text{ ако } q\in Q_1\ \&\ a\in\Sigma\\
      \Delta_2(q,a),                       & \text{ ако } q\in Q_2\ \&\  a\in\Sigma\\
      \Delta_1(s_1,a) \cup \Delta_2(s_2,a), & \text{ ако } q = s\ \&\  a \in\Sigma.
    \end{cases}
    $
  %   $\Delta(q,a) = \Delta_1(q,a)$, за всяко $q\in Q_1$, $a\in\Sigma$;
  % \item
  %   $\Delta(q,a) = \Delta_2(q,a)$, за всяко $q\in Q_2$, $a\in\Sigma$;
  % \item
  %   $\Delta(s,a) = \Delta_1(s_1,a) \cup \Delta_2(s_2,a)$, за всяко $a\in\Sigma$;
  \end{itemize}
\end{proof}
\begin{remark}
  В началното състояние на новопостроения автомат $\N$ не влизат ребра.
\end{remark}


\begin{lemma}
  Класът от автоматните езици е затворен относно операцията {\bf звезда на Клини}.
\end{lemma}
\begin{proof}
  Нека е даден автомата $\N = \NFA$, за който е изпънено, че
  $L(\N) = L(r)$.
  Първата стъпка е да построим $\N_1 = \NFAn{1}$, такъв че 
  \[L(\N_1) = \bigcup_{n\geq 1} (L(\N))^n = \bigcup_{n\geq 1} (L(r))^n = L(r^+).\]
  \begin{itemize}
  \item
    $Q_1 = Q$;
  \item
    $s_1 = s$;
  \item
    $F_1 = F$;
  \item
    $
    \Delta_1(q,a) = 
    \begin{cases}
      \Delta(q,a), & \text{ ако } q\in Q\setminus F, a \in \Sigma\\
      \Delta(q,a) \cup \Delta(s,a), & \text{ ако } q\in F, a\in\Sigma.
    \end{cases}
    $
    % $\Delta_1(q,a) = \Delta(q,a)$, за всяко $q\in Q\setminus F$, $a\in\Sigma$;
  % \item
  %   $\Delta_1(q,a) = \Delta(q,a) \cup \Delta(s,a)$, за всяко $q\in F$, $a\in\Sigma$;
  \end{itemize}
  Накрая строим автомат $\N_2$, за който $L(\N_2) = \{\varepsilon\} \cup L(\N_1)$.
\end{proof}

% \begin{remark}
%   Запазваме свойството, че в началното състояние не влизат ребра.
% \end{remark}

\section{Допълнителни свойства}

Някои свойства на регулярните езици:
\begin{itemize}
\item 
  те са затворени относно хомоморфизми, т.е.
  ако $L \subseteq \Sigma^\star_1$ е регулярен език и $h:\Sigma_1\to\Sigma^\star_2$ е хомоморфизъм, 
  то езикът 
  \[h(L) = \{h(\alpha) \in \Sigma^\star_2 \mid \alpha \in L\}\]
  е регулярен.
\item
  те са затворени относно обратни хомоморфизми, т.е.
  ако $L\subseteq \Sigma^\star_2$ е регулярен език и $h:\Sigma_1\to\Sigma^\star_2$ е хомоморфизъм, 
  то езикът
  \[h^{-1}(L) = \{\alpha \in \Sigma^\star_1 \mid h(\alpha) \in L\}\]
  е регулярен.
\end{itemize}

\begin{problem}
  Да означим $\mbox{Half}(L) = \{w\mid (\exists x \in \Sigma^\star)[wx \in L\ \&\ \abs{w} = \abs{x}]\}$.
  Докажете, че ако $L$ е регулярен език, то $\mbox{Half}(L)$ също е регулярен език.
\end{problem}

\begin{problem}
  Да означим $\mbox{Pref}(L) = \{\alpha \mid (\exists \beta \in \Sigma^\star)[\alpha\beta \in L]\}$.
  Докажете, че ако $L$ е регулярен език, то $\mbox{Pref}(L)$ също е регулярен език.
\end{problem}
\begin{proof}
  Две доказателства. Едното е с индукция по построението регулярния израз за $L$.
  Другото е по автомата за $L$.
\end{proof}

\section{Езици, които не са регулярни}
\begin{lemma}[за нарастването (регулярни езици)]
  \index{лема за нарастването!регулярни езици}
  \label{lem:pumping-reg}
  \marginpar{На англ.\\ Pumping Lemma}
  \marginpar{Има версия и за безконтекстни езици}
  Нека $L$ да бъде безкраен регулярен език.
  Съществува число $n\geq 1$, зависещо само от $L$, 
  за което за всяка дума $\alpha\in L, \abs{\alpha}\geq n$ може да 
  бъде записана във вида $\alpha = xyz$ и 
  \begin{enumerate}[1)]
  \item
    $|y|\geq 1$;
  \item
    $|xy|\leq n$;
  \item
    \marginpar{Обърнете внимание, че $0 \in \Nat$ и $xy^0z =  xz$}
    $(\forall i\in\Nat)[xy^iz \in L]$.
  \end{enumerate}
\end{lemma}
\begin{proof}
%  \marginpar{стр. 88 от \cite{papadimitriou}, стр. 78 от \cite{sipser}}
  Понеже $L$ е регулярен, той се разпознава от ДКА $\M = \FA$.
  Нека $n = \abs{Q}$ и нека $\alpha = a_1a_2\dots a_k$ е дума, за която $k \geq n$.
  Да разгледаме първите $n$ стъпки от изпълнението на $\alpha$ върху $\M$:
  \[q_0\stackrel{a_1}{\rightarrow} q_1 \stackrel{a_2}{\rightarrow} \dots
  \stackrel{a_n}{\rightarrow} q_n.\]
  Тъй като $\abs{Q} = n$, а по този път участват $n+1$ състояния $q_0,q_1,\dots,q_n$,
  то съществуват числа $i, j$, за които $0\leq i < j\leq n$ и $q_i = q_j$.
  Нека разделим думата $\alpha$ на три части по следния начин:
  \[x = a_1\dots a_i,\quad y = a_{i+1}\dots a_j,\quad z = a_{j+1}\dots a_k.\]
  Ясно е, че $\abs{y} \geq 1$ и $\abs{xy} = j \leq n$.
  Освен това, лесно се съобразява, че $xz \in L$ и че за всяко $i > 0$,
  $xy^iz \in L$.
\end{proof}

\begin{example}
  Езикът $L = \{a^nb^n \mid n\in \Nat\}$ {\bf не} е регулярен.
\end{example}
\begin{proof}
  \marginpar{По-късно ще видим, че този език е безконтекстен}
  Да допуснем, че $L$ е регулярен.
  За константа $p$, избираме думата $\alpha = a^pb^p$.
  Очевидно $|\alpha| \geq p$.
  Понеже $\abs{xy} \leq p$, то $y = a^k$, за  $k \geq 1$.
  Тогава $xy^0z = a^{p-k}b^p$.
  Очевидно е, че $xz \not\in L$.
\end{proof}

\begin{prop}
  Регулярният език $L$, разпознаван от КДА $\M$, е {\bf непразен} точно тогава, когато съдържа дума $\alpha, \abs{\alpha} < \abs{Q}$.
\end{prop}
\begin{proof}
  Нека $\M = \FA$ разпознава регулярния език $L$ и $n = \abs{Q}$.
  \begin{description}
  \item[$(\Rightarrow)$]
    Нека $L$ е непразен език и нека $m = \min\{\abs{\alpha} \mid \alpha \in L\}$.
    Ще докажем, че $m < \abs{Q}$.
    
    За целта, да допуснем, че $m \geq \abs{Q}$ и да изберем $\alpha \in L$, за която $\abs{\alpha} = m$.
    Според \Lem{pumping-reg}, съществува разбиване $xyz = \alpha$, 
    такова че $xz \in L$.
    При положение, че $\abs{y} \geq 1$, то $\abs{xz} < m$, което 
    е противоречие с минималността на $m$.
    
    Заключаваме, $m < \abs{Q}$, откъдето следва, че 
    съществува дума $\alpha \in L$ с $\abs{\alpha} < \abs{Q}$.
  \item[$(\Leftarrow)$]
    Тази посока е тривиална.
    Ако $L$ съдържа дума $\alpha$, за която $\abs{\alpha} \leq n$,
    то е очевидно, че $L$ е непразен език.
  \end{description}
\end{proof}

\begin{cor}
  \marginpar{$(L_1\setminus L_2) \cup (L_2 \setminus L_1) = \emptyset$?}
  Съществува алгоритъм, който определя дали два автомата $\M_1$ и $\M_2$ разпознават един и същ език.
\end{cor}

\begin{prop}
  Регулярният език $L$, 
  разпознаван от КДА $\M$, е {\bf безкраен} точно тогава, когато съдържа дума $\alpha, \abs{Q} \leq \abs{\alpha} < 2\abs{Q}$.
\end{prop}
\begin{proof}
  Посоката $(\Leftarrow)$  следва от \Lem{pumping-reg}.
  За посоката $(\Rightarrow)$, нека $L$ е безкраен език и да приемем, че няма думи $\alpha$ със
  свойството $\abs{Q} \leq \abs{\alpha} <  2\abs{Q}$.
  Да вземем най-късата дума $\alpha \in L$, за която $\abs{\alpha} \geq 2\abs{Q}$.
  Понеже $L$ е безкраен, знаем, че такава дума съществува.
  Тогава отново по \Lem{pumping-reg},
  \[\alpha = xyz, \abs{xy} \leq \abs{Q}, xz \in L.\]
  Но понеже $\abs{xyz} \geq 2\abs{Q}$, а $\abs{y} \leq \abs{Q}$, то $\abs{xz} \geq \abs{Q}$ и понеже избрахме $\alpha = xyz$
  да бъде най-късата дума с дължина поне $2\abs{Q}$, заключаваме, че $\abs{Q} \leq \abs{xz} < 2\abs{Q}$.
\end{proof}

\begin{cor}
  Съществува алгоритъм, който проверява дали даден регулярен език е безкраен.
\end{cor}

\begin{framed}
  \Lem{pumping-reg} е полезна, когато искаме да докажем, че даден език $L$ {\bf не} е регулярен.
  За целта, доказваме отрицанието на \Lem{pumping-reg} за $L$, т.е.
  за всяка константа $p$, ние намираме дума $\alpha$, $\abs{\alpha}\geq p$, такава че за всяко разбиване на думата на три части, $\alpha = xyz$,
  със свойствата $\abs{y} \geq 1$ и $\abs{xy} \leq p$, е изпълнено, че $(\exists i)[xy^iz \not\in L]$.
  Формално лемата може да се запише така:
  {\scriptsize  
    \[(\exists p \geq 1)(\forall \alpha \in L)[\abs{\alpha} \geq p \Rightarrow (\exists x,y,z\in\Sigma^\star)[\alpha = xyz\ \wedge\ \abs{y} \geq 1\ \wedge\ \abs{xy} \leq p\ \wedge\ (\forall i\in\Nat)[xy^iz \in L]].\]}
  Тогава нейното отрицание гласи следното:
  {\scriptsize  \[(\forall p \geq 1)(\exists \alpha \in L)[\abs{\alpha} \geq p\ \wedge (\forall x,y,z\in\Sigma^\star)[\alpha \neq xyz\ \vee\ \abs{y} \not\geq 1\ \vee\ \abs{xy} \not\leq p\ \vee\ (\exists i\in\Nat)[xy^iz \not\in L]],\]}
  което е еквивалентно на:
  {\scriptsize
    \[(\forall p \geq 1)(\exists \alpha \in L)[\abs{\alpha} \geq p\ \wedge\ (\forall x,y,z\in\Sigma^\star)[(\alpha = xyz \wedge \abs{y} \geq 1\wedge \abs{xy} \leq p) \Rightarrow (\exists i\in\Nat)[xy^iz \not\in L]].\]}
\end{framed}

\begin{problem}
  Докажете, че езикът $L = \{a^n\ \mid\ n\mbox{ е просто число}\}$ не е регулярен.
\end{problem}
\begin{proof}
  Да допуснем, че $L$ е регулярен език. Ще достигнем до противоречие като докажем отрицанието на условието на \Lem{pumping-reg},
  т.е. ще докажем, че
  {\scriptsize
    \[(\forall p \geq 1)(\exists \alpha \in L)[\abs{\alpha} \geq p\ \wedge\ (\forall x,y,z\in\Sigma^\star)[(\alpha = xyz \wedge \abs{y} \geq 1\wedge \abs{xy} \leq p) \Rightarrow (\exists i\in\Nat)[xy^iz \not\in L]].\]}

  И така, за произволно число $p \geq 1$, да изберем думата $w$ със свойството, че $\abs{w} > p+1$.
  Да разгледаме произволно разбиване на $w$, $w = xyz$, за което $\abs{xy} \leq p$ и $\abs{y} \geq 1$.
  Ще докажем, че съществува $i$, за което $xy^iz \not\in L$, т.е. $\abs{xy^iz} = \abs{xz} + i.\abs{y}$ не е просто число.
  
  Понеже $\abs{xy} \leq p$, то $\abs{z} > 1 $.
  Да изберем $i = \abs{xz} > 1$.
  Но тогава \[\abs{xy^iz} = \abs{xz} + i.\abs{y} = \abs{xz} + \abs{xz}.\abs{y} = (1 + \abs{y})\abs{xz}\] е съставно число, следователно 
  намерихме $i$, за което $xy^iz \not\in L$.

  Доказахме, че ако $L$ е регулярен език, то \Lem{pumping-reg} не е изпълнена. Следователно, достигнахме до противоречие.
\end{proof}

\begin{problem}
  Докажете, че езикът $L = \{a^{n^2}\ \mid\ n\in\Nat\}$ не е регулярен.  
\end{problem}
\begin{proof}
  Да допуснем, че $L$ е регулярен. Отново ще докажем отрицанието на \Lem{pumping-reg}.
  За произволна константа $p$, да разгледаме думата $w = a^{p^2}$ и произволно нейно разбиване
  $w = xyz$, за което $\abs{y} \geq 1$ и $\abs{y} \leq \abs{xy} \leq p$.
  Тогава за $i = 2$,
  \[\abs{xyz} = p^2 < p^2 + \abs{y} = \abs{xy^2z} = \abs{xz} + \abs{y}^2 \leq p + p^2 < (p+1)^2.\]
  Това означава, че $p^2 < \abs{xy^2z} < (p+1)^2$, т.е. $\abs{xy^2z}$ не е точен квадрат.
\end{proof}

\begin{problem}
  Нека $\Sigma = \{a,b\}$.  Проверете дали $L$ е регулярен, където
  \begin{enumerate}[a)]
  \item
    $L = \{\alpha^R \mid \alpha \in L_0\}$, където $L_0$ е регулярен;
  \item
    \marginpar{$\alpha = a^pb^p$}
    $L = \{a^ib^i\ \mid\ i\in\Nat\}$;
  \item
    $L = \{a^ib^i\ \mid\ i,j\in\Nat\ \&\ i\neq j\}$;
  \item
    \marginpar{$\alpha = a^{p+1}b^p$.}
    $L = \{a^ib^j\ \mid\ i > j\}$;
  \item
    $L = \{a^nb^m \mid n\mbox{ дели }m\}$.
  \item
    $L = \{a^{2n}\ \mid\ n\geq 1\}$;
  \item
    $L = \{a^mb^na^{m+n}\ \mid\ m\geq 1\ \&\ n\geq 1\}$;
  \item
    $L = \{a^{n.m}\mid n,m\mbox{ са прости числа}\}$;
  \item
    $L = \{\omega\in\{a,b\}^\star \mid N_a(\omega) = N_b(\omega)\}$;
  \item
    \marginpar{$\alpha = a^pba^pb$}
    $L = \{\omega\omega\mid \omega\in\{a,b\}^\star\}$;
  \item
    $L = \{\omega\omega^R\mid \omega\in\{a,b\}^\star\}$;
  \item
    $L = \{uvv \in \{a,b\}^\star\mid v \neq \varepsilon\}$;
  \item
    $L = \{a^nb^nc^n\mid n\geq 0\}$;
  \item
    $L = \{www\mid w\in \Sigma^\star\}$;
  \item
    $L = \{a^{2^n}\mid n\geq 0\}$;
  \item
    $L = \{a^mb^n\mid n\neq m\}$;
  \item
    $L = \{a^{n!}b^{n!}\mid n\neq 1\}$;
  \item
    $L = \{a^{f_n} \mid f_0 = f_1 = 1\ \&\ f_{n+2} = f_{n+1} + f_{n}\}$;
  \item
    $L = \{\alpha \in \{a,b\}^\star \mid \abs{n_a(\alpha) - n_b(\alpha)} \leq 2\}$;
  \item
    $L = \{\alpha \in \{a,b\}^\star \mid \alpha = vuv\ \&\ \abs{u} \leq \abs{v}\}$;
  \item
    $L = \{\alpha \in \{a,b\}^\star \mid \alpha = uvv^R\ \&\ \abs{u} \leq \abs{v}\}$;
  \item
    $L = \{c^ka^nb^m \mid k,m,n > 0\ \&\ n \neq m\}$;
  \item
    $L = \{c^ka^nb^n \mid k > 0\ \&\ n \geq 0\}\cup\{a,b\}^\star$;
  \end{enumerate}
\end{problem}

\subsection*{Примери, когато лемата не е  приложима}

\begin{problem}
  \marginpar{Например, разгледайте $c^+\{a^nb^n\mid n\in\Nat\}\cup (a\vert b)^\star$}
  Да се даде пример за език $L$, който {\bf не} е регулярен, но удовлетворява
  условието на \Lem{pumping-reg}.
\end{problem}



\begin{problem}
  Езикът $L = \{c^ka^nb^m\mid k,n,m \in \Nat\ \&\ k = 1\implies m = n\}$
  {\bf не} е регулярен, но условието на  \Lem{pumping-reg} е изпълнена за него.
\end{problem}
\begin{proof}
  Да допуснем, че $L$ е регулярен.
  Тогава ще следва, че 
  \[L_1 = L\cap ca^\star b^\star = \{ca^nb^n \mid n\in\Nat\}\]
  е регулярен,
  но с лемата за разрастването лесно се вижда, че $L_1$ не е.

  Сега да проверим, че условието на \Lem{pumping-reg} е изпълнена за $L$.
  Да изберем константа $p = 2$.
  Сега трябва да разгледаме всички думи $\alpha \in L$, $\abs{\alpha} \geq 2$
  и за всяка $\alpha$ да посочим разбиване $xyz = \alpha$, за което са изпълнени трите свойства от лемата.
  \marginpar{Условията за $x,y,z$ са:
    \begin{align*}
      & \abs{xy} \leq 2\\
      & \abs{y} \geq 1\\
      & (\forall i\in\Nat)(xy^iz \in L)
    \end{align*}}

  \begin{itemize}
  \item
    Ако $\alpha = a^n$ или $\alpha = b^n$, $n\geq 2$, то е  очевидно, че можем да
    намерим такова разбиване.
  \item
    $\alpha = a^nb^m$ и $n+m \geq 2$, $n \geq 1$.
    Избираме $x = \varepsilon$, $y = a$, $z = a^{n-1}b^m$.
  \item
    $\alpha = ca^nb^n$, $n\geq 1$.
    Избираме $x = \varepsilon$, $y = c$, $z = a^nb^n$.
  \item
    $\alpha = c^2a^nb^m$. 
    Избираме $x = \varepsilon$, $y = c^2$, $z = a^nb^m$.
  \item
    $\alpha = c^ka^nb^m$, $k \geq 3$.
    Избираме $x = \varepsilon$, $y = c$, $z = c^{k-1}a^nb^m$.
  \end{itemize}
\end{proof}


\section{Минимизация на ДКА}

\begin{itemize}
\item
  Нека $L \subseteq \Sigma^\star$ е език и нека $x,y \in \Sigma^\star$.
  Казваме, че $x$ и $y$ са {\bf еквивалентни относно} $L$, което записваме 
  като $x \approx_L y$, ако е изпълнено:
  \[(\forall z \in \Sigma^\star)[xz \in L \iff yz \in L].\]
  Това означава, че $x\approx_L y$ ако или и две думи са в $L$ или и двете не са в $L$
  и освен това, като прибавим произволна дума на края на $x$ и $y$, новополучените
  думи са или и двете в $L$ или и двете не са в $L$.  
\item
  Нека $\M = \FA$ е КДА.
  Казваме, че две думи $\alpha,\beta \in \Sigma^\star$ са {\bf еквивалентни относно $\M$},
  което означаваме с $\alpha \sim_\M \beta$, ако $\delta^\star(s,\alpha) = \delta^\star(s,\beta)$.
\item
  Съобразете, че $\approx_L$ и $\sim_\M$ са {\bf релации на еквивалентност}, т.е.
  те са рефлексивни, транзитивни и симетрични.
  Всяко състояние на $\M$, което е достижимо от началното състояние $s$, определя клас на еквивалентност относно 
  релацията $\sim_\M$.  
  Нека да означим класовете на еквивалентност на $\approx_L$ с $[x]_L$.
\item
  С $\abs{\approx_L}$ ще означаваме броя на класовете на еквивалентност на релацията $\approx_L$.
\item
  $\sim_\M$ е дясно-инвариантна, т.е.
  \[\alpha \sim_\M \beta\ \rightarrow\ (\forall \gamma\in\Sigma^\star)[\alpha\gamma \sim_\M \beta\gamma].\]
\end{itemize}

\begin{thm}
  \label{th:rel-finer}
  Нека е даден ДКА $\M = \FA$ и $L(\M)$ е езика, разпознаван от $\M$. Тогава е изпълнено
  \[(\forall \alpha,\beta \in \Sigma^\star)[\alpha\sim_\M\beta\ \rightarrow\ \alpha\approx_{L(\M)}\beta].\]
\end{thm}
\begin{proof}
  \marginpar{стр. 95 от \cite{papadimitriou}}
  Да означим $q_\alpha = \delta^\star(s, \alpha)$.
  Лесно се съобразява, че 
  \[\alpha \sim_\M \beta\quad\Leftrightarrow\quad q_\alpha = q_\beta.\]
  Освен това, 
  \[\alpha\gamma \in \Ls(\M)\quad \Leftrightarrow\quad \delta^\star(q_\alpha, \gamma) \in F.\]
  Тогава, ако $\alpha \sim_\M \beta$, то  $\alpha \approx_{\Ls(\M)}\beta$, защото за произволно $\gamma \in \Sigma^\star$,
  \[\alpha\gamma \in \Ls(\M)\ \Leftrightarrow\ \delta^\star(q_\alpha,\gamma) \in F\ \Leftrightarrow\ \delta^\star(q_\beta, \gamma)\in F\ \Leftrightarrow\ \beta\gamma \in \Ls(\M).\]  
\end{proof}

\begin{cor}
  Един регулярен език $L$ се разпознава от ДКА с $\abs{Q} \geq \abs{\approx_L}$.
\end{cor}
\begin{proof}
  Теоремата ни казва, че $\sim_\M$ е {\em по-фина} релация от $\approx_{L(\M)}$, от което следва, че
  броят на класовете на еквивалентност на $\sim_\M$ е по-голям от този на $\approx_{L(\M)}$.
  Тъй като всяко достижимо състояние определя клас на еквивалентност относно $\sim_\M$,
  то получаваме {\em долна граница} на броя на състоянията в минималнен автомат за езика $L$.
  Този брой е не по-малък от броя на класовете на еквивалентност на $\approx_L$.
  
\end{proof}

{\bf Сега ще отговорим на въпроса дали тази долна граница може да бъде достигната.}

% Да приемем, че сме фиксирали азбуката $\Sigma$.
\begin{thm}[Майхил-Нероуд]
  \label{th:myhill-nerode}
  \index{теорема!Майхил-Нероуд}
  \index{Майхил}
  \index{Нероуд}
  \marginpar{на англ. Myhill-Nerode}
  Нека $\L\subseteq \Sigma^\star$ е регулярен език.
  Тогава съществува ДКА $\M = \FA$, който разпознава $\L$,
  с точно толкова състояния, колкото са класовете на еквивалентност на релацията $\approx_\L$,
  т.е. $\abs{Q} = \abs{\approx_\L}$.
\end{thm}
\begin{proof}
  \marginpar{стр. 96 от \cite{papadimitriou}}
  При даден регулярен език $\L$, ние определяме ДКА $\M = \FA$, за който $\L = \L(\M)$ като
  \begin{itemize}
  \item
    $Q = \{[\alpha]_\L\mid \alpha\in \Sigma^\star\}$;
  \item
    $s = [\varepsilon]_\L$;
  \item
    $F = \{[\alpha]_\L\mid \alpha\in \L\}$;
  \item
    За всяка буква $a \in \Sigma$ и всяко състояние $[\alpha]_\L\in Q$, дефинираме
    \[\delta([\alpha]_\L,a) = [\alpha a]_\L.\]
  \end{itemize}
  
  Първо, трябва да се уверим, че множеството от състояния $Q$ е крайно, т.е.
  релацията $\approx_\L$ има крайно много класове на еквивалентност.
  И така, тъй като $\L$ е регулярен език, то той се разпознава от някой ДКА $\M^\prime$.
  Ние знаем от \Th{rel-finer}, че $\sim_{\M^\prime}$ е по-фина релация от $\approx_\L$.
  Това означава, че броят на класове на еквивалентност на $\approx_\L$ е по-малък от този
  на $\sim_{\M^\prime}$, който очевидно е краен, защото е ограничен отгоре от $\abs{Q^\prime}$.

  Второ, трябва да се уверим, че функцията $\delta$ е добре определена, т.е. 
  да проверим, че
  \[[\alpha]_\L = [\beta]_\L\ \Rightarrow\ \delta([\alpha]_\L,a) = \delta([\beta]_\L,a).\]
  Но това се вижда веднага, защото от определението на релацията $\approx_\L$ следва, че
  ако $\alpha \approx_\L \beta$, то за всяка буква $a$, $\alpha a \approx_\L \beta a$,
  т.е. $[\alpha a]_\L = [\beta a]_\L$.
  
  Така вече сме показали, че $\M$ е коректно зададен ДКА.
  Остава да покажем, че $\M$ разпознава езика $\L$, т.е. $\L(\M) = \L$.
  \begin{prop}
    За всеки две думи $\alpha,\beta\in\Sigma^\star$,
    $\delta^\star([\alpha]_\L,\beta) = [\alpha\beta]_\L$.
  \end{prop}
  \begin{proof}
    Ще докажем това свойство с индукция по дължината на думата $\beta$.
    \begin{enumerate}[a)]
    \item
      $\beta = \varepsilon$, директно от дефиницията на $\delta^\star$,
      $\delta^\star([x]_\L,\varepsilon) = [x\varepsilon]_\L = [x]_\L$.
    \item
      Нека $\beta$ има дължина $n+1$ и сме доказали твърдението за думи с дължина $\leq n$.
      Тогава $\beta = \gamma a$, където $\gamma$ има дължина $n$.
      \[\delta^\star([\alpha]_\L, \gamma a) = \delta(\delta^\star([\alpha]_\L,\gamma),a) = \delta([\alpha\gamma]_\L,a) = [\alpha\gamma a]_\L = [\alpha\beta]_\L.\]
    \end{enumerate}
  \end{proof}
  Сега, 
  \[\alpha\in \Ls(\M) \iff \delta^\star(s,\alpha) \in F \iff \delta^\star([\varepsilon]_\Ls,\alpha) = [\alpha]_\Ls\ \&\ \alpha\in \Ls \iff \alpha \in \Ls.\]
  
\end{proof}

Ако $M = \FA$ е автомат, за който $\abs{Q} = \abs{\approx_\L}$, то ще казваме, че
$M$ е {\bf минимален} за езика $\Ls$.

\begin{framed}
  \begin{cor}
    Езикът $\Ls$ е регулярен точно тогава, когато релацията $\approx_\Ls$ има {\em крайно много} класове на еквивалентност.
  \end{cor}
\end{framed}
\begin{proof}
  Ако $\L$ е регулярен, то той се разпознава от някой ДКА $\M$, който има крайно много състояния 
  и следователно крайно много класове на еквивалентност относно $\sim_\M$.
  Релацията $\approx_\L$ е по-груба от $\sim_\M$ и има по-малко класове на еквивалентност.
  Следователно, $\approx_\L$ има крайно много класове на еквивалентност.
  
  За другата посока, ако $\approx_\L$ има крайно много класове на еквивалентност, то можем да 
  построим ДКА $\M$ както в доказателството на \Th{myhill-nerode} на Майхил-Нероуд,
  който разпознава $\L$.
  
\end{proof}

Това следствие ни дава още един начин за проверка дали даден език е регулярен.
За разлика от \Lem{pumping-reg} за нарастването, сега имаме
необходимо и достатъчно условие.
При даден език $\L$, ние строим неговата релация $\approx_\L$.
Ако тя е крайна, то езикът $\L$ е регулярен.
Ако тя не е крайна, то езикът $\L$ не е регулярен.

\begin{example}
  За езика $L = \{a^nb^n\mid n \in \Nat\}$ имаме, че $\abs{\approx_L} = \infty$,
  защото \[k \neq j\ \rightarrow\ [a^kb]_L \neq [a^jb]_L.\]
  Проверете, че $[a^kb]_L = \{a^kb,a^{k+1}b^{2},\dots,a^{k+l}b^{l+1},\dots\}$.
  Така получаваме, че релацията $\approx_L$ има безкрайно много класове на еквивалентност
  и следователно този език {\bf не} е автоматен.
\end{example}

\begin{example}
  За езика $L = \{a^{n^2} \mid n \in \Nat\}$ имаме, че $\abs{\approx_L} = \infty$,
  защото \[m \neq n\ \rightarrow\ [a^{n^2}]_L \neq [a^{m^2}]_L.\]
  
  Да разгледаме $n < m$ и думата $\gamma = a^{2n+1}$.
  Тогава $a^{n^2}\gamma = a^{(n+1)^2} \in L$, но 
  $m^2 < m^2 + 2n + 1 < (m+1)^2$ и следователно $a^{m^2}\gamma \not\in L$.
\end{example}


\begin{dfn}
  Нека $M_1 = \FAn{1}$ и $M_2 = \FAn{2}$.
  Казваме, че $M_1$ и $M_2$ са изоморфни, което означаваме с $M_1 \cong M_2$, ако
  \begin{itemize}
  \item
    съществува биекция $h:\Sigma_1 \to \Sigma_2$;
  \item
    съществува биекция $f: Q_1\to Q_2$;
  \item
    $f(q_1) = q_2$;
  \item
    $f[F_1] = \{f(q)\mid q\in F_1\} = F_2$;
  \item
    $(\forall a\in\Sigma_1)(\forall q\in Q_1)[f(\delta_1(q,a)) = \delta_2(f(q),h(a))]$.
  \end{itemize}
  Ще казваме, че $\pair{h,f}$ задава изоморфизъм на $M_1$ върху $M_2$.
\end{dfn}

\begin{cor}
  Всички минимални автомати за езика $\Ls$ са изоморфни на $M_0$, автомата построен в теоремата на Нерод-Майхил.
\end{cor}
\begin{proof}
  Нека $M = \FA$ е произволен автомат, за който $\Ls(M) = \Ls$ и $\abs{Q_M} = \abs{\approx_\Ls}$.
  Съобразете, че $M$ е свързан, т.е. всяко състояние на $M$ е достижимо от началното.
  Искаме да докажем, че $M \cong M_0$.
  За всяко състояние $q\in Q$ можем да намерим дума $w_q\in\Sigma^\star$,
  за която $\delta^\star(s,w_q) = q$.
  Да дефинираме $f:Q\to [\approx_\Ls]$ като $f(q) = [w_q]_\Ls$. Ще докажем, 
  $\pair{id_\Sigma, f}$ задава изоморфизъм на $M$ върху $M_0$.
  \begin{itemize}
  \item
    $f$ е определена коректно, т.е. $f$ е функция.
    За целта, нека $q\in Q$ и $\alpha,\beta$ са две думи, за които
    \[\delta^\star(s,\alpha) = q = \delta^\star(s,\beta).\]
    Тогава $\alpha\sim_M \beta$ и от \Th{rel-finer} следва, 
    че $\alpha \approx_\Ls \beta$ и $[\alpha]_\Ls = [\beta]_\Ls$.
  \item
    $f$ е инективна, т.е. ако $(\forall q_1,q_2 \in Q)[q_1\neq q_2\ \rightarrow\ f(q_1) \neq f(q_2)]$.
    Да допуснем, че за $q_1 \neq q_2$ имаме, че $f(q_1) = [w_{q_1}]_\Ls = [w_{q_2}]_\Ls = f(q_2)$.
    Тогава $w_{q_1} \not\sim_M w_{q_2}$ и $w_{q_1} \approx_\Ls w_{q_2}$.
    Но това означава, че $\abs{\sim_M} > \abs{\approx_\Ls}$,
    което противоречи с минималността на $M$.
  \item
    $f$ е сюрективна, т.е. за всяко $[\beta]_\Ls \in [\approx_\Ls]$, съществува $q \in Q$, за което $f(q) = [\beta]_\Ls$.
    Следва от факта, че съществува $q \in Q$, за което $[\beta]_\Ls = [w_q]_\Ls$.
  \item
    $f(\delta^\star_M(s,\alpha)) = [\alpha]_\Ls = \delta^\star_{M_0}([\varepsilon]_\Ls, \alpha) = 
    \delta^\star_{M_0}(f(s), \alpha)$.
  \end{itemize}
\end{proof}


\paragraph{Алгоритъм за намиране на МКДА.}
\begin{itemize}
% \item
%   Нека $M = \FA$ да бъде ДКА. Определяме
%   \[A_M = \{\pair{q,w} \mid q\in Q\ \&\ w\in\Sigma^\star\ \&\ \delta^\star(q,w) \in F\}.\]
\item
  Казваме, че две състояния $p,q$ са {\bf еквивалентни}, означаваме $p\equiv q$,
  % \footnote{В общия случай, $\approx_L \supseteq \sim_M$, т.е. $\sim_M$ е по-фина от $\approx_L$.
  %   Нека $\equiv\ =\  \approx_L/\sim_M$, където $[\alpha]_{\sim_M} \equiv [\beta]_{\sim_M} \iff \alpha \approx_L \beta$. Тогава $\abs{\approx_L} = \abs{\equiv}$.}
  ако \[p \equiv q\ \iff\ (\forall \gamma\in \Sigma^\star)[\delta^\star(p,\gamma) \in F\ \iff\ \delta^\star(q,\gamma) \in F].\]
\item
  Релацията $\equiv$ между състояния на автомата $\M$ е релация на еквивалентност. 
\item
  Нека $q_\alpha$ е състоянието, което съответства на думата $\alpha$ в $\M$, т.е.
  $\delta^\star_\M(s,\alpha) = q_\alpha$. Тогава:
  \[q_\alpha \equiv q_\beta\ \iff\ \alpha\approx_{\L(\M)} \beta.\]
  Това означава, че ако в $\M$ няма недостижими състояния от началното състояние $s$, то $\abs{\equiv} = \abs{\approx_{\L(\M)}}$.
\end{itemize}

При даден език $\L$ и ДКА $\M = \FA$, който го разпознава, нашата цел е да построим нов ДКА $\M_0$,
който има толкова състояния колкото са класовете на еквивалентност на релацията $\approx_\L$.
Това ще направим като ``слеем'' състоянията на $\M$, които са еквивалентни относно релацията $\equiv$.
Това означава, че всяко състояние на $\M_0$ ще отговаря на един клас на еквивалентност на релацията $\equiv$.
Проблемът с намирането на класовете на еквивалентност на релацията $\equiv$ е кванторът $\forall \gamma \in \Sigma^\star$
в нейната дефиницията.

Алгоритъмът представлява намирането на релации $\equiv_n$, където
\[p\equiv_n q \iff (\forall\gamma\in\Sigma^\star)[\abs{\gamma}\leq n\ \rightarrow\ (\delta^\star(p,\gamma) \in F\ \iff\ \delta^\star(q,\gamma) \in F)].\]

Обърнете внимание, че за всяко $n$, $\equiv_n$ е {\em по-груба} релация от $\equiv_{n+1}$, 
която на свой ред е по-груба от $\equiv$.
Алгоритъмът строи $\equiv_n$ докато $\equiv_n\ \neq\ \equiv_{n+1}$.
Тъй като броят на класовете на еквивалентност на $\equiv_n$ е краен, и не по-голям от $\abs{Q}$, то 
със сигурност ще намерим $n$, за което $\equiv_n = \equiv_{n+1}$.
Тогава заключаваме, че $\equiv\ =\ \equiv_n$.

Понеже единствената дума с дължина $0$ e $\varepsilon$ и по определение $\delta^\star(p,\varepsilon) = p$, 
лесно се съобразява, че $\equiv_0$ има два класа на еквивалентност.
Единият е $F$, а другият е $Q\setminus F$.

\begin{prop}
  За всеки две състояния $p,q \in Q$, и всяко $n$, $p \equiv_{n+1} q$ точно тогава, когато
  \begin{enumerate}[a)]
  \item
    $p \equiv_{n} q$ и
  \item
    $(\forall a \in \Sigma)[\delta(q,a) \equiv_{n} \delta(p,a)]$.
  \end{enumerate}
\end{prop}
\begin{proof}%[стр. 99 от \cite{papadimitriou}]
  По условие имаме, че $p\equiv_{n+1} q$ точно тогава, когато
  $p\equiv_n q$ и за всяка дума $\gamma = a\beta$, $\abs{\gamma} = n+1$,
  $(p,\gamma) \in A_M \iff (q,\gamma)\in A_M$.
  Последното е еквивалентно с 
  \[(\delta(p,a),\beta) \in A_M \iff (\delta(q,a),\beta) \in A_M.\]
  Тъй като това е вярно за всяка дума $\gamma$, то
  \[(\forall a\in\Sigma)[\delta(p,a) \equiv_{n} \delta(q, a)].\]
  
\end{proof}

Нека е даден автомата $M = \FA$.
След като сме намерили релацията $\equiv$ за $M$, 
строим автомата $M^\prime = (Q^\prime,\Sigma,s^\prime,\delta^\prime,F^\prime)$, където:
\begin{enumerate}[1)]
\item
  $Q^\prime = \{[q]_\equiv \mid q\in Q\}$;
\item
  $s^\prime = [s]_\equiv$;
\item
  $\delta^\prime([q]_\equiv, a) = [\delta(q,a)]_\equiv$;
\item
  $F^\prime = \{[q]_\equiv\mid F\cap [q]_\equiv \neq \emptyset\}$;
\end{enumerate}


% % \begin{figure}
% %   \begin{subfigure}[b]{0.3\textwidth}%[НКА с $\varepsilon$-преходи]{
% %     \begin{tikzpicture}[->,>=stealth,thick,node distance=45pt]
% %       \tikzstyle{every state}=[circle,minimum size=15pt,auto]
      
% %       \node[initial,state] (0) {$0$};
% %       \node[state]         (1) [right of=0]{$1$};
% %       \node[state]   (2) [right of=1]{$2$};
% %       \node[state]   (3) [right of=2]{$3$};
% %       \node[accepting, state]   (4) [right of=3]{$4$};
      
% %       \path 
% %       (0) edge  node [above] {$\varepsilon$} (1)
% %       (1) edge  node [above] {$a$} (2)
% %       (2) edge  node [above] {$b$} (3) 
% %       (3) edge  node [above] {$a,\varepsilon$} (4) 
% %       (1) edge  [bend right=30] node [below] {$a$} (3)
% %       (2) edge  [bend right=45] node [above] {$\varepsilon$} (1)
% %       (3) edge  [bend right=45] node [above] {$\varepsilon$} (2);
% %     \end{tikzpicture}
% %   \end{subfigure}
% %   \begin{subfigure}[b]{0.3\textwidth}
% %   % \subfigure[получения от него ДКА]{
% %     \begin{tikzpicture}[->,>=stealth,thick,node distance=70pt]
% %       \tikzstyle{every state}=[circle,font=\small]
      
% %       \node[initial,state] (0) {$\{0,1\}$};
% %       \node[accepting,state]         (1) [above right of=0]{$\{1,2,3,4\}$};
% %       \node[state]   (2) [below right of=0]{$\emptyset$};
      
% %       \path 
% %       (0) edge [] node [above] {$a$} (1)
% %       (0) edge node [above] {$b$} (2)
% %       (1) edge [loop right] node [right] {$a,b$} (1)
% %       (2) edge [loop right] node [right] {$a,b$} (2);
% %     \end{tikzpicture}
% %   \end{subfigure}
% %   \caption{}
% % \end{figure}


% \newpage
\section{Въпроси}

Вярно ли е, че:
\begin{itemize}
  % \item
%   \marginpar{Не}
%   езикът $\{a^nb^n\mid n \in \Nat \}$ е регулярен?
% \item
%   \marginpar{Не}
%   езикът $\{a^nb^k\mid n > k\}$ е регулярен?
% \item
%   \marginpar{Не}
%   езикът $\{a^{n^2}\mid n \in \Nat\}$ е регулярен?
\item
  \marginpar{Да}
  за всеки два регулярни езика $R_1, R_2$, то $R_1 \setminus R_2$ е регулярен ?
\item
  \marginpar{Да}
  за всеки краен език $F$ и всеки регулярен $R$, то $R\setminus F$ е регулярен ?
\item
  \marginpar{Да}
  за всеки краен език $F$ и всеки рег. $R$, то $R\cup (\Sigma^\star \setminus F)$ е регулярен ?
\item
  \marginpar{Да}
  съществува регулярен език $R$ и нерегулярен $K$, за които $R\cap K$ не е регулярен ?
\item
  \marginpar{Да}
  съществува регулярен език $R$ и нерегулярен $K$, за които $R\setminus K$ не е регулярен ?
\item
  \marginpar{Не}
  за всеки регулярен език $R$ и всеки $K \subseteq R$, то $R\setminus K$ е регулярен ?
\item
  \marginpar{Не}
  Езикът $L = \{\omega \in \{a,b\}^\star \mid n_a(\omega) \text{ не дели }n_b(\omega)\}$ е регулярен?
\item
  съществува алгоритъм, който може да провери дали даден регулярен език е празен?
\item
  съществува алгоритъм, който може да провери дали даден регулярен език е краен?
\item
  съществува алгоритъм, който може да провери дали даден регулярен език е безкраен?
\item
  \marginpar{Да}
  съществува алгоритъм, който може да провери дали два автомата разпознават един и същ език?
\end{itemize}


\section{Домашна работа}

\begin{itemize}
\item
  Вход - файл, в който е записан регулярен израз
\item
  Преобразуване на регулярния израз в обратен полски запис.
  (\href{http://en.wikipedia.org/wiki/Shunting-yard_algorithm}{тук} 
  добре е обяснено как става за произволни аритмечни изрази)
\item
  Строене на краен детерминиран автомат по регулярния израз.
\item
  Извеждане на автомата във формат за програмата \href{http://graphviz.org}{graphviz}.
  (вижте \href{http://sundarpillay.blogspot.com/2012/02/graphviz-and-finite-automata-diagrams_05.html}{пример})
\end{itemize}


%%% Local Variables: 
%%% mode: latex
%%% TeX-master: "discrete-math"
%%% End: 

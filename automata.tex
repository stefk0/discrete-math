\chapter{Езици и автомати}

\section{Езици, които не са регулярни}
\begin{lemma}[за разрастването (регулярни езици)]
  % \index{лема за разрастването!регулярни езици}
  % \label{lem:pumping-reg}
  % \marginpar{На англ.\\ Pumping Lemma}
  Нека $\Ls$ да бъде безкраен регулярен език.
  Съществува число $n\geq 1$, зависещо само от $\Ls$, 
  за което за всяка дума $\alpha\in \Ls, \abs{\alpha}\geq n$ може да 
  бъде записана във вида $\alpha = xyz$ и 
  \begin{enumerate}
  \item
    $|y|\geq 1$;
  \item
    $|xy|\leq n$;
  \item
    % \marginpar{$i = 0\ \rightarrow\ xz \in \Ls$}
    $(\forall i\in\N)[xy^iz \in \Ls]$.
  \end{enumerate}
\end{lemma}

\begin{crl}
  Регулярният език $\Ls$, 
  разпознаван от КДА $M$ е непразен тoчно тогава, когато съдържа дума $\alpha, \abs{\alpha} \leq \abs{Q}$.
\end{crl}

\begin{problem}
  \marginpar{$c^+\{a^nb^n\mid n\in\N\}\cup (a\vert b)^\star$}
  Да се даде пример за език $L$, който {\bf не} е регулярен, но удовлетворява
  лемата за разрастването.
\end{problem}


% \section{Регулярни езици}
% \begin{problem}
%   Нека $\Sigma = \{0,1\}$.  Проверете дали $L$ е регулярен, където
%   \begin{enumerate}[1)]
%   \item
%     $L_1 = \{0^i1^i\ \mid\ i\geq 0\}$;
%   \item
%     $L_2 = \{0^i1^j\ \mid\ i > j\}$;
%   \item
%     $L_3 = \{0^{2n}\ \mid\ i\geq 1\}$;
%   \item
%     $L_4 = \{0^1m1^n0^{m+n}\ \mid\ m\geq 1\ \&\ n\geq 1\}$;
%   \item
%     $L_5 = \{0^n\ \mid\ n\mbox{ е просто }\}$;
%   \item
%     $L_6 = \{w\mid w\in\{0,1\}^\star\mbox{ има равен брой нули и единици}\}$;
%   \item
%     $L_7 = \{ww\mid w\in\{0,1\}^\star\}$;
%   \item
%     $L_8 = \{1^{n^2}\mid n\geq 0\}$;
%   \item
%     $L_{9} = \{0^n1^n2^n\mid n\geq 0\}$;
%   \item
%     $L_{10} = \{www\mid w\in \{0,1\}^\star\}$;
%   \item
%     $L_{11} = \{0^{2^n}\mid n\geq 0\}$;
%   \item
%     $L_{12} = \{0^m1^n\mid n\neq m\}$;
%   \end{enumerate}
% \end{problem}

\begin{problem}
  Нека $\Sigma = \{a,b\}$.  Проверете дали $L$ е регулярен, където
  \begin{enumerate}[1)]
  \item
    $L = \{\alpha^R \mid \alpha \in L_0\}$, където $L_0$ е регулярен;
  \item
    $L_1 = \{a^ib^i\ \mid\ i\geq 0\}$;
  \item
    $L_2 = \{a^ib^j\ \mid\ i > j\}$;
  \item
    $L_3 = \{a^{2n}\ \mid\ n\geq 1\}$;
  \item
    $L_4 = \{a^mb^na^{m+n}\ \mid\ m\geq 1\ \&\ n\geq 1\}$;
  \item
    $L_5 = \{a^n\ \mid\ n\mbox{ е просто число}\}$;
  \item
    $L = \{a^{n.m}\mid n,m\mbox{ са прости числа}\}$;
  \item
    $L_6 = \{w\mid w\in\{a,b\}^\star\mbox{ има равен брой нули и единици}\}$;
  \item
    $L_7 = \{ww\mid w\in\{a,b\}^\star\}$;
  \item
    $L_8 = \{ww^R\mid w\in\{a,b\}^\star\}$;
  \item
    $L_9= \{a^{n^2}\mid n\geq 0\}$;
  \item
    $L_{10} = \{a^nb^nc^n\mid n\geq 0\}$;
  \item
    $L_{11} = \{www\mid w\in \Sigma^\star\}$;
  \item
    $L_{12} = \{a^{2^n}\mid n\geq 0\}$;
  \item
    $L_{13} = \{a^mb^n\mid n\neq m\}$;
  \item
    $L_{14} = \{a^{n!}b^{n!}\mid n\neq 1\}$;
  \item
    $L_{15} = \{a^{f_n} \mid f_0 = f_1 = 1\ \&\ f_{n+2} = f_{n+1} + f_{n}\}$;
  \item
    $L = \{\alpha \in \{a,b\}^\star \mid \abs{n_a(\alpha) - n_b(\alpha)} \leq 2\}$;
  \item
    $L = \{\alpha \in \{a,b\}^\star \mid \alpha = vuv\ \&\ \abs{u} \leq \abs{v}\}$;
  \item
    $L = \{\alpha \in \{a,b\}^\star \mid \alpha = uvv^R\ \&\ \abs{u} \leq \abs{v}\}$;
  \item
    $L = \{c^ka^nb^m \mid k,m,n > 0\ \&\ n \neq m\}$;
  \item
    $L = \{c^ka^nb^n \mid k > 0\ \&\ n \geq 0\}\cup\{a,b\}^\star$;
  \item
    $L = \{c^ka^nb^m\mid k,n,m \in \N\ \&\ k = 1\implies m = n\}$; % p \geq 2, не става с p = 1
  \end{enumerate}
\end{problem}
% \begin{proof}
%   \begin{enumerate}[1)]
%   \item
%     Разгледайте $w = a^pb^p$.
%   \item
%     Разгледайте $w = a^{p+1}b^p$.
%   \item
%     Езикът е регулярен.
%   \item
%     Подобно на 1) се доказва, че езикът не е регулярен.
%   \item
%     Да допуснем, че $L_5$ е регулярен и нека $w \in L_5$, така че $\abs{w} > p+1$.
%     $w = xyz$ и понеже $\abs{xy} \leq p$, то $\abs{z} > 1$.
%     Имаме, че $xy^iz \in L_5$ и следователно $\abs{xy^iz} = \abs{xz} + i.\abs{y}$ е просто число, за всяко $i$.
%     Да изберем $i = \abs{xz} > 1$.
%     Но тогава $\abs{xy^iz} = (1 + \abs{y})\abs{xz}$ е съставно число, следователно 
%     достигнахме до противоречие.
%   \item
%     Разгледайте $w = 0^p1^p$. Проблем с условието $\abs{xy} \leq p$.
%     Друг начин е да се използва, че $0^\star1^\star \cap L_6 = L_1$.
%   \item
%     Разгледайте $0^p10^p1$.
%   \item
    
%   \item
%     Да разгледаме $w = 1^{p^2}$.
%     Да допуснем, че $L_9$ е регулярен и имаме, че
%     \[(\forall i)[\abs{xy^iz}\ \&\ \abs{xy^{i+1}z}\mbox{ са точни квадрати}].\]
%     Тогава съществува $n$, за което $n^2 = \abs{xy^iz}$, $\abs{xy^{i+1}z} \geq (n+1)^2$ и
%     \[\abs{y} = \abs{xy^{i+1}z} - \abs{xy^{i}z} \geq 2n + 1 = 2\sqrt{\abs{xy^iz}} + 1\]
%     Ясно е, че $\abs{y} \leq \abs{w} = p^2$.
%     Получаваме, че за всяко $i$,
%     \[\abs{y} \geq 2\sqrt{\abs{xy^iz}} + 1\]
%     Лесно стигаме до противоречие, например да вземем $i = p^2$.
%     Тогава $\abs{y} > p$, което е противоречие.
%     % Да вземем $i = p^4$.
%     % $\abs{y} \leq p^2 < 2\sqrt{p^4} + 1 \leq 2\sqrt{\abs{xy^iz}} + 1$,
%     което е противоречие.
%   \item
%   \item
%   \item
%     Аналогично на 9).
%   \item
%     Да допуснем, че $L_{13}$ е регулярен.
%     Но тогава $L^\prime = \{a,b\}^\star \setminus L_{13}$ е регулярен
%     и $L_1 = a^\star b^\star \cap L^\prime$ е регулярен.
%     Достигнахме до противоречие.
%   \end{enumerate}
% \end{proof}


\begin{problem}
  Да означим $\mbox{Half}(L) = \{w\mid (\exists x \in \Sigma^\star)[wx \in L\ \&\ \abs{w} = \abs{x}]\}$.
  Докажете, че ако $L$ е регулярен език, то $\mbox{Half}(L)$ също е регулярен език.
\end{problem}

\begin{problem}
  Да означим $\mbox{Pref}(L) = \{\alpha \mid (\exists \beta \in \Sigma^\star)[\alpha\beta \in L]\}$.
  Докажете, че ако $L$ е регулярен език, то $\mbox{Pref}(L)$ също е регулярен език.
\end{problem}
\begin{proof}
  Две доказателства. Едното е с индукция по построението регулярния израз за $L$.
  Другото е по автомата за $L$.
\end{proof}



%%% Local Variables: 
%%% mode: latex
%%% TeX-master: "discrete-math"
%%% End: 

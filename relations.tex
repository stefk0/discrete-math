\chapter{Релации}
\index{релация}

За да дадем определение на понятието релация, трябва първо 
да въведем понятието декартово произведение на множества,
което пък от своя страна се основава на понятието наредена двойка.

\subsection*{Наредена двойка}

За два елемента $a$ и $b$ въвеждаме опрецията {\bf наредена двойка} $\pair{a,b}$.
Наредената двойка $\pair{a,b}$ има следното характеристичното свойство:
\[a_1 = a_2\ \wedge\ b_1 = b_2\ \iff\ \pair{a_1,b_1} = \pair{a_2,b_2}.\]
Понятието наредена двойка може да се дефинира по много начини, стига да изпълнява харектеристичното свойство.
Ето примери как това може да стане:
\begin{enumerate}[1)]
\item
  \marginpar{Norbert Wiener (1914)}
  Първото теоретико-множествено определение на понятите наредена двойка е
  дадено от Норберт Винер:
  \[\pair{a,b} = \{\{\{a\},\emptyset\},\{\{b\}\}\}.\]
\item
  \marginpar{Kazimierz Kuratowski (1921)}
  Определението на Куратовски се приема за ``стандартно'' в наши дни:
  \[\pair{a,b} = \{\{a\},\{a,b\}\}.\]
\end{enumerate}

\begin{prb}
  Докажете, че горните дефиниции наистина изпълняват харектеристичното свойство за наредени двойки.
\end{prb}

\begin{remark}
  \marginpar{Пример за рекурсивна дефиниция}
  Сега можем да въведем понятието наредена $n$-орка \[\pair{a_1,\dots,a_n}\] за всяко $n \geq 1$:
  \begin{align*}
    & \pair{a_1} = a_1,\\
    & \pair{a_1,a_2,\dots,a_n} = \pair{a_1,\pair{a_2,\dots,a_n}}
  \end{align*}
\end{remark}

 
\subsection*{Декартово произведение}
\marginpar{На англ. cartesian product}
\index{декартово произведение}
За две множества $A$ и $B$, определяме тяхното декартово произведение като
\[A\times B = \{\pair{a,b}\mid a\in A\ \&\ b\in B\}.\]
За краен брой множества $A_1,A_2,\dots,A_n$, определяме
\[A_1\times A_2\times\cdots\times A_n = \{\pair{a_1,a_2,\dots,a_n}\mid a_1 \in A_1\ \&\ a_2\in A_2\ \&\ \dots\ \&\ a_n \in A_n\}.\]

Подмножествата $R$ от вида $R \subseteq A\times A\times\cdots\times A$ се наричат релации.

\begin{problem}
  Преверете, че:
  \begin{enumerate}[1)]
  \item 
    $A\times(B\cup C) = (A\times B) \cup (A\times C)$.
  \item
    $(A\cup B)\times C = (A\times C)\cup (B\times C)$.
  \item 
    $A\times(B\cap C) = (A\times B) \cap (A\times C)$.
  \item
    $(A \cap B)\times C = (A \times C)\cap(B\times C)$.
  \item 
    $A\times(B\setminus C) = (A\times B) \setminus (A\times C)$.
  \item
    $(A\setminus B)\times C = (A\times C)\setminus (B\times C)$.
  \end{enumerate}
\end{problem}

\marginpar{Бинарни релации}
Релациите от вида $R\ \subseteq\ A\times A$ са важен клас, който ще срещаме често.
Да разгледаме няколко основни вида релации от този клас:
\begin{enumerate}[I)]
\item
  {\bf рефликсивна}, ако
  \[(\forall x\in A)[\pair{x,x}\in R].\]
  Например, релацията $\leq\ \subseteq\ \Nat\times\Nat$ е рефлексивна, защото
  \[(\forall x\in \Nat)[x \leq x].\]
\item
  {\bf антирефлексивна}, ако
  \[(\forall x\in A)[\pair{x,x}\not\in R].\]
  Например, релацията $<\ \subseteq\ \Nat\times\Nat$ е антирефлексивна, защото
  \[(\forall x\in \Nat)[x \not< x].\]
\item
  {\bf транзитивна}, ако
  \[(\forall x,y,z\in A)[\pair{x,y}\in R\ \&\ \pair{y,z}\in R \rightarrow \pair{x,z}\in R].\]
  Например, релацията $\leq\ \subseteq\ \Nat\times\Nat$ е транзитивна, защото
  \[(\forall x,y,z\in A)[x \leq y\ \&\ y \leq z\ \rightarrow\ x\leq z].\]
\item
  {\bf симетрична}, ако
  \[(\forall x,y\in A)[\pair{x,y}\in R \rightarrow \pair{y,x}\in R].\]
  Например, релацията $=\ \subseteq\ \Nat\times\Nat$ е рефлексивна, защото
  \[(\forall x,y\in \Nat)[x = y\ \rightarrow\ y = x].\]
\item
  {\bf антисиметрична}, ако
  \[(\forall x,y\in A)[\pair{x,y}\in R\ \&\ \pair{y,x}\in R \rightarrow x = y].\]
  Например, релацията $\leq\ \subseteq\ \Nat\times\Nat$ е антисиметрична, защото
  \[(\forall x,y,z\in A)[x \leq y\ \&\ y \leq x\ \rightarrow\ x = y].\]
\item
  {\bf асиметрична}, ако
  \[(\forall x,y)[\pair{x,y}\in R \rightarrow \pair{y,x}\not\in R].\]
  Например, релацията $\leq\ \subseteq\ \Nat\times\Nat$ е асиметрична, защото
  \[(\forall x,y\in \Nat)[x < y\ \rightarrow\ y \not< x].\]
\end{enumerate}

\begin{remark}
  Добре е да запомните как се наричат тези основни видове релации,
  защото ще ги използваме често.
\end{remark}

\begin{example}
  Да обобщим примерите от по-горе.
  \begin{enumerate}[a)]
  \item
    Релацията $\leq\ \subseteq\ \Nat\times\Nat$ е рефлексивна, транзитивна и антисиметрична.
  \item
    Релацията $<\ \subseteq\ \Nat\times\Nat$ е антирефлексивна, транзитивна и асиметрична.
  \item
    Релацията $=\ \subseteq\ \Nat\times\Nat$ е рефлексивна, транзитивна и симетрична.
  \end{enumerate}
\end{example}

% \begin{problem}
%   Кои от изброените по-горе свойства има релацията $R$, където:
%   \begin{enumerate}[a)]
%   \item
%     \marginpar{$x\vert y \iff (\exists z)(y = x\cdot z)$}
%     $R \subseteq \Z\times \Z$ е определена като
%     \[\pair{x,y} \in R \iff x \vert y.\]
%   % \item
%   %   \marginpar{$gcd$ - greatest common divisor (най-голям общ делител)}
%   %   $R \subseteq \Z\times \Z$ е определена като
%   %   \[\pair{x,y}\in R \iff gcd(x,y) = 1 \iff \neg(\exists z > 1)[\ z\vert x\ \wedge\ z\vert y\ ]\]
%   % \item
%   %   \marginpar{$\R$ - реални, $\Q$ - рационални}
%   %   $R \subseteq \R\times\R$ е определена като
%   %   \[\pair{x,y} \in R\ \iff\ x\cdot y \in \Q.\]
%   \item
%     Нека $S$ да бъде произволно множество и $T \subseteq S$.
%     Определяме релацията $R \subseteq \Ps(S)\times\Ps(S)$ като:
%     \[\pair{A,B}\in R\ \iff\ (A\cup B)\setminus(A\cap B) \subseteq T.\]
%   \end{enumerate}
% \end{problem}


\begin{problem}
  Нека $R$ е релация върху $A$.
  Да определим релациите:
  \begin{itemize}
  \item 
    $S = \{\pair{x,y}\mid \pair{x,y} \in R\ \wedge\ \pair{y,x} \in R\}$;
  \item
    $T = \{\pair{x,y} \mid \pair{x,y}\in R\ \wedge\ \pair{y,x}\not\in R\}$.
  \end{itemize}
  Докажете, че:
  \begin{enumerate}[a)]
  \item 
    $S$ е симетрична и $T$ е антисиметрична.
  \item
    $\pair{x,y} \in R\ \iff\ (\pair{x,y}\in S \vee \pair{x,y} \in T)$;
  \item
    ако $R$ е транзитивна, то $S$ и $T$ са също транзитивни, но обратната посока не е вярна.
  \end{enumerate}
\end{problem}

\begin{problem}
  Проверете дали релацията $R$ е рефлексивна, транзитивна, симетрична, антисиметрична или асиметрична.
  \begin{enumerate}[a)]
  \item
    \marginpar{Озн. $\Nat^2 = \Nat\times\Nat$}
    $R\subseteq \Nat^2$ и е определна като 
    \marginpar{$a\vert b \iff (\exists k\in\Nat)(b = k\cdot a)$}
    \[(a,b) \in R \iff a | b.\]
  \item
    \marginpar{$gcd$ - greatest common divisor (най-голям общ делител)}
    $R \subseteq \Z\times \Z$ е определена като
    \[\pair{x,y}\in R \iff gcd(x,y) = 1 \iff \neg(\exists z > 1)[\ z\vert x\ \wedge\ z\vert y\ ]\]
  \item
    \marginpar{Озн. $\R$ - реалните числа}
    $R\subseteq \R^2$ и е определена като
    \[(a,b) \in R \iff a.b > 0.\]
  \item
    $R\subseteq \R^2$ и е определена като 
    \[(a,b) \in R \iff a+b = 0.\]
  \item
    $R\subseteq \R^2$ и е определена като
    \[(a,b) \in R \iff a+b = 5.\]
  \item
    $R\subseteq \R^2$ и е определена като 
    \[(a,b) \in R \iff a+b\mbox{ е четно}.\]
  \item
    $R\subseteq (\R^2)^2$ и е определена като
    \[(\langle{a,b}\rangle, \langle{c,d}\rangle) \in R \iff a+d = b+c.\]
  \item
    $R\subseteq (\R^2)^2$ и е определена като
    \[(\langle{a,b}\rangle,\langle{c,d}\rangle) \in R \iff a\cdot d = b\cdot c.\]
  \item
    \marginpar{Озн. $\Z$ - целите числа}
    $R_{m}\subseteq \Z^2, m\in \Z, m>0$ и е определена като
    \[(a,b) \in R_m \iff m\mid (a - b).\]
  \item
    $R\subseteq \R^2$ и е определена като 
    \[(x,y) \in R \iff (x-y)\mbox{ е рационално число}.\]
  \item
    \marginpar{Озн. $\Q$ - рационалните числа}
    $R \subseteq \Q^2$ и е определена като
    \[(p,r) \in R\ \iff\ p-r \mbox{ е цяло число}.\]
  \item
    $R \subseteq \Nat^2$ и е определена като
    \[(a,b) \in R \iff a = b \vee a+1 = b.\]
  \item
    $R \subseteq \Nat^2$ и е определена като
    \[\pair{a,b}\in R \iff (\exists k\in\N)[a+k = b].\]
  \item
    Нека $\leq_1\ \subseteq\ A^2$ и $\leq_2\ \subseteq\ B^2$ са частични наредби.
    $R \subseteq A^2\times B^2$ е определена като
    \[(\pair{a,b}, \pair{c,d}) \in R \iff a\leq_{1}c\ \wedge\ b\leq_{2}d .\]
  \item
    Нека $\leq_1\ \subseteq\ A^2$ и $\leq_2\ \subseteq\ B^2$ са частични наредби.
    $R \subseteq A^2\times B^2$ е определена като
    \[(\pair{a,b}, \pair{c,d}) \in R \iff a\leq_{1}c\ \vee\ b\leq_{2}d .\]
  \item
    $f:X\rightarrow Y$ е функция, $R\ \subseteq\ \Ps(X)\times\Ps(X)$ и 
    \[(A,B)\in R \iff f(A) = f(B).\]
  \end{enumerate}
\end{problem}

\subsection*{Операции върху релации}
\begin{enumerate}[I)]
\item
  {\bf Композиция} на две релации $S \subseteq B\times C$ и $T \subseteq A\times B$ е релацията $S\circ T \subseteq A\times C$,
  определена като:
  \[S\circ T = \{\langle{a,c}\rangle \mid (\exists b \in B)[\pair{a,b}\in T\ \&\ \pair{b,c} \in S]\}.\]
\item
  {\bf Обръщане} на релацията $R \subseteq A\times B$ е релацията $R^{-1}\subseteq B\times A$, 
  определена като:
  \[R^{-1} = \{\pair{x,y} \mid \pair{y,x} \in R\}.\]
% \item
  
%   $\overline{R} = \{\langle{x,y}\rangle \in A\times B \mid\langle{x,y}\rangle\not\in R\}$;
\end{enumerate}

\begin{problem}
  Дайте пример за релации $R$ и $S$, за които
  \[R\circ S \neq S\circ R.\]
\end{problem}

  
\begin{problem}
  Докажете, че:
  \begin{enumerate}[a)]
  \item
    $R$ е симетрична тогава и само тогава, когато $R^{-1}\subseteq R$;
  \item
    $R$ е транзитивна тогава и само тогава, когато $R\circ R\subseteq R$;
  \item
    $R$ е транзитивна и симетрична тогава и само тогава, когато $R = R^{-1}\circ R$.
\end{enumerate}
\end{problem}
\begin{proof}
  \begin{enumerate}[a)]
  \item
    Задачата се разделя на две подзадачи.
    \begin{enumerate}[(i)]
    \item
      Нека $R$ да бъде симетрична. Ще докажем, че $R^{-1}\subseteq R$, т.е.
      \[(\forall x\forall y)[(x,y)\in R^{-1} \rightarrow (x,y)\in R].\]
      Нека $(x,y)\in R^{-1}$. Тогава по определение имаме, че $(y,x)\in R$ и следователно $(x,y)\in R$,
      защото $R$ е симетрична.
    \item
      Нека $R^{-1}\subseteq R$. Щe докажем, че $R$ е симетрична, т.е.
      \[(\forall x\forall y)[(x,y)\in R \rightarrow (y,x)\in R].\]
      Нека $(x,y)\in R$, следователно по определение $(y,x)\in R^{-1}$.
      Тогава от $R^{-1}\subseteq R$ следва, че $(y,x)\in R$.
    \end{enumerate}
  \item
  \item
    Нека $R^{-1}\circ R = R$. Ще докажем, че $R$
    е симетрична и транзитивна.
    \begin{enumerate}[(i)]
    \item
      Ще докажем, че $R$ e симетрична.
      За целта е достатъчно да вземем произволна двойка $\pair{x,y} \in R$
      и да покажем, че $\pair{y,x} \in R$.
      Нека 
      
      \AxiomC{$\pair{x,y} \in R$}
      % \AxiomC{$\psi\rightarrow\phi$}
      \LeftLabel{$R^{-1}\circ R = R$:}
      \UnaryInfC{$(\exists z)[\pair{x,z} \in R\ \wedge\ \pair{z,y} \in R^{-1}]$}
      \UnaryInfC{$(\exists z)[\pair{z,x} \in R^{-1}\ \wedge\ \pair{y,z} \in R]$}
      \UnaryInfC{$(\exists z)[\pair{y,z} \in R\ \wedge\ \pair{z,x} \in R^{-1}]$}
      \UnaryInfC{$\pair{y,x} \in R^{-1}\circ R$}
      \LeftLabel{$R^{-1}\circ R = R$: }
      \UnaryInfC{$\pair{y,x} \in R$}
      \DisplayProof
      
      Следователно,
      \[(\forall x,y \in A)[\pair{x,y} \in R\ \rightarrow\ \pair{y,x} \in R].\]
    \item
      Ще докажем, че $R$ e транзитивна.
      За целта е достатъчно да вземем произволни двойки $\pair{x,y} \in R$
      и $\pair{y,z} \in R$, то $\pair{x,z} \in R$.
      
      \begin{prooftree}
      \AxiomC{$\pair{x,y} \in R$}
      \AxiomC{$\pair{y,z} \in R$}
      \RightLabel{($R$ е симетрична)}
      \UnaryInfC{$\pair{z,y} \in R$}
      \UnaryInfC{$\pair{y,z} \in R^{-1}$}
      \BinaryInfC{$\pair{x,z} \in R^{-1}\circ R$}
      \RightLabel{($R^{-1}\circ R = R$)}
      \UnaryInfC{$\pair{x,z} \in R$}
      \end{prooftree}
      Следователно,
      \[(\forall x,y,z \in A)[(\pair{x,y} \in R\ \wedge\ \pair{y,z} \in R) \rightarrow\ \pair{x,z} \in R].\]
    \end{enumerate}
    Нека сега $R$ е транзитивна и симетрична.
    Ще докажем, че $R^{-1}\circ R = R$.
    
    \begin{enumerate}[(i)]
    \item 
      Първо да отбележим, че
      \begin{prooftree}
        \AxiomC{$\pair{x,y} \in R$}
        \AxiomC{$\pair{x,y} \in R$}
        \RightLabel{($R$ е симетрична) }
        \UnaryInfC{$\pair{y,x} \in R$}
        \RightLabel{($R$ е транзитивна) }
        \BinaryInfC{$\pair{x,x} \in R\ \wedge \pair{y,y}\in R$}
      \end{prooftree}
      Следователно,
      \[(\forall x \in Dom(R))[\pair{x,x} \in R].\]
    \item
      Ще докажем, че $R^{-1}\circ R \subseteq R$.
      
      \begin{prooftree}
        \AxiomC{$\pair{x,z}\in R$}
        \AxiomC{$\pair{x,y} \in R^{-1}\circ R$}
        \RightLabel{(Съществува $z$)}
        \UnaryInfC{$\pair{z,y} \in R^{-1}$}
        \UnaryInfC{$\pair{y,z} \in R$}
        \RightLabel{($R$ е симетрична)}
        \UnaryInfC{$\pair{z,y} \in R$}
        \RightLabel{($R$ е транзитивна)}
        \BinaryInfC{$\pair{x,y} \in R$}
      \end{prooftree}
    \item
      Ще докажем, че $R \subseteq R^{-1}\circ R$.
      \begin{prooftree}
        \AxiomC{$\pair{x,y}\in R$}
        \AxiomC{$\pair{x,y}\in R$}
        \RightLabel{(От (i))}
        \UnaryInfC{$\pair{y,y} \in R$}
        \UnaryInfC{$\pair{y,y} \in R^{-1}$}
        \RightLabel{(по деф.)}
        \BinaryInfC{$\pair{x,y} \in R^{-1}\circ R$}
      \end{prooftree}
    \end{enumerate}
  \end{enumerate}
  
\end{proof}

\begin{problem}
  Нека $\{\pair{a,b}\}\subseteq R$, за някои $a\neq b$.
  Докажете, че ако $R$ е симетрична, то $R$ не е антисиметрична.
\end{problem}
% \begin{proof}
%   Нека $R$ е симетрична.
% \end{proof}

\begin{problem}
  Нека $R$ да бъде релация на еквивалентност върху $B$ и $f:A\to B$.
  Дефинираме множеството 
  \[Q = \{((x,y)\in A\times A\mid (f(x),f(y))\in R\}.\]
  Докажете, че $Q$ е релация на еквивалентност.
\end{problem}

\newpage

Обикновено се изучават релации притежаващи различни комбинации от горните свойства. 
Сега ще изброим няколко основни вида релации (понякога се наричат наредби).
Релацията $R \subseteq A\times A$ се нарича:
\begin{itemize}
\item
  \marginpar{На англ. partial order}
  {\bf частична наредба}, ако тя е рефлексивна, транзитивна и антисиметрична.
  Например, $\leq$ е частична наредба.
  Също така, релацията $\subseteq$ между множества е частична наредба.
\item 
  \marginpar{На англ. equivalence relation}
  {\bf релация на еквивалентност}, ако тя е рефлексивна, транзитивна и симетрична.
  Например, $=$ е релация на еквивалентност.
\item
  \marginpar{На англ. linear order}
  {\bf линейна наредба}\index{наредба!линейна}, ако $R$ е частична наредба, 
  и за всеки два елемента $x,y$ точно едно от $\pair{x,y} \in R$, $x = y$, $\pair{y,x}\in R$ е изпълнено.
\item
  \marginpar{На англ. well-founded relation}
  {\bf фундирана наредба}\index{наредба!фундирана}, 
  ако всяко непразно подмножество $X\subseteq A$ притежава поне един {\em минимален} елемент, т.е.
  \[(\forall X\subseteq A)[X\neq\emptyset \rightarrow (\exists m\in X)\neg(\exists y\in X)[\pair{y,m} \in R]].\]
\item
  \marginpar{На англ. well-ordered relation}
  {\bf добра наредба}\index{наредба!добра}, ако всяко непразно подмножество $X\subseteq A$ има {\em най-малък} елемент , т.е.
  \[(\forall X\subseteq A)[X\neq\emptyset \rightarrow (\exists m\in X)(\forall y\in X)[\pair{m,y}\in R \vee m = y]].\]
\item
  \marginpar{lexicographical order}
  {\bf лексикографска наредба} върху частичната наредба $(X,<)$ ще наричаме 
  наредбата $(X\times X,\prec)$, където
  \[\pair{x,y}\prec\pair{x^\prime,y^\prime}\ \iff\ x<x^\prime\ \vee\ (x = x^\prime\ \wedge\ y < y^\prime).\]
\end{itemize}

\begin{remark}
  Обърнете внимание на разликата между понятията {\em минимален} и {\em най-малък} елемент относно релацията $R$.
  \begin{itemize}
  \item $x_0$ е {\em минимален} елемент за множеството $X \subseteq A$ относно $R$,
    ако не съществуват елементи $y \in X$, за които $\pair{y,x}\in R$, т.е.
    \[(\forall y \in X)[\pair{y,x} \not\in R].\]
  \item $x_0$ е {\em най-малък} елемент за множеството $X \subseteq A$ относно $R$,
    ако за всеки друг елемент $y \in X$ е изпълнено $\pair{x,y}\in R$, т.е.
    \[(\forall y \in X)[x\neq y \rightarrow \pair{x,y} \in R].\]
  \end{itemize}
\end{remark}

\begin{problem}
  Да се докаже, че $(\N^2,\prec)$ е добре наредено множество.
\end{problem}

\begin{problem}
  Докажете, че следните две условия за частично наредено множество $(X,<)$ са еквивалентни:
  \begin{enumerate}[a)]
  \item
    всяко непразно подмножество на $X$ има минимален елемент;
  \item
    не съществува строго намаляваща редица $x_1>x_2>x_3>\dots$ то елементи на $X$.
  \end{enumerate}
\end{problem}

\begin{problem}
  Да се докаже, че частично нареденото множество $(X,<)$ е добре наредено тогава и само тогава, когато 
  $(X,<)$ е фундирано и $<$ е линейна наредба върху $X$.
\end{problem}

\begin{problem}% от СЕП
  Кои от изброените множества са фундирани? Кои са добре наредени?
  \begin{enumerate}[a)]
  \item
    $(\N,<)$;
  \item
    $(\Z,<)$;
  \item
    $(X^*, <)$, където за $\alpha,\beta\in X^*$, $\alpha < \beta \iff \alpha\mbox{ е поддума на }\beta$;
  \item
    $(2^\N,\subsetneq)$;
  \item
    $(Fin(\N),\subsetneq)$, където $Fin(\N)$ е съвкупността от всички крайни подмножества на $\N$;
  \item
    $(\N^+,|)$, където $m|n \iff m\neq n\ \&\ m\mbox{ дели }n$.
  \end{enumerate}
\end{problem}

\begin{problem}
  Нека $R$ и $S$ са релации на еквивалентност върху множеството $A$.
  Определете кои от свойствата а) - г) притежават тези релации.
  След това, нека $R$ и $S$ да бъдат частични наредби върху $A$
  и отново определете кои от свойствата а) - г) притежават тези релации.
  \begin{enumerate}[a)]
  \item
    $R \cap S$;
  \item
    $R \cup S$;
  \item
    $R \setminus S$;
  \item
    $R \circ S$.
  \end{enumerate}
\end{problem}




\subsection*{Азбуки и думи}

\begin{itemize}
\item
  Азбука е крайно множество $X = \{a_1,\dots,a_n\}$ като елементите $a_i$ на $X$ наричаме {\bf букви}.
\item
  Нека да фиксираме един елемент $\varepsilon \not\in X$.
  Сега ще определим {\bf думите} над азбуката $X$. Това са:
  \begin{enumerate}[1)]
  \item
    \marginpar{наричаме $\varepsilon$ празната дума}
    $\varepsilon$ е дума над $X$;
  \item
    \marginpar{т.е. думите са крайни последователности от букви}
    Нека $\alpha$ е дума над $X$ и $a \in X$ е буква.
    Тогава $\alpha a$ е дума над $X$;
  \item
    \marginpar{Защо е важно условие 3) ?}
    няма други думи над $X$.
  \end{enumerate}
\item
  Нека $\beta=b_{1}\cdots b_{n}$ и $\gamma=c_{1}\cdots c_{m}$ са думи над $X$.
  Тогава казваме, че 
  $\beta$ е {\bf начало} на $\gamma$, ако $n\leq m$ и $(\forall k\leq n)(b_{k} = c_{k})$;
  $\beta$ е {\bf край} на $\gamma$, ако $n \leq m$ и $(\forall k\leq n)(b_{n-k} = c_{m-k})$.
\end{itemize}
Означаваме с $X^n$ множеството от всички думи с дължина $n$ над азбуката $X$, $X^0 = \{\varepsilon\}$.
С $X^{*}$ означаваме множеството от всички думи над азбуката $X$, т.е.
\marginpar{$0 \in \Nat$}
\[X^{*} = \bigcup_{n\in\Nat} X^{n}.\]

Сега ще определим функцията {\bf дължина}\index{дума!дължина} на дума.
\marginpar{Функцията $\abs{\cdot}:X^\star\to\Nat$ е винаги сюрективна. Кога е биективна?}
Дължината $|\alpha|$ на думите $\alpha \in X^*$ се определя с индукция по построението на $\alpha$.
\begin{enumerate}[(i)]
  \item
    Нека $\alpha = \varepsilon$. Тогава $|\alpha| = 0$.
  \item
    Нека $\alpha = \beta a$, за някоя дума $\beta\in X^*$ и някоя буква $a\in X$.
    Тогава \[|\alpha| = |\beta| + 1.\]
\end{enumerate}

Определяме функцията {\bf конкатенация}\index{дума!конкатенация} $\cdot$, т.е.
слепване на две думи.
За всеки две думи $\alpha$ и $\beta$ от $X^*$ определяме тяхната конкатенация с индукция по дължината $\beta$:
\marginpar{$\cdot:X^\star\times X^\star \to X^\star$ е винаги сюрективна. Може ли да бъде биективна?}
\begin{enumerate}[(i)]
  \item
    $|\beta| = 0$, т.е. $\beta = \varepsilon$.
    Тогава \[\alpha\cdot\beta = \alpha.\]
  \item
    $|\beta| = n+1$, т.е. $\beta = \gamma b$, за някоя дума $\gamma$, $|\gamma| = n$, и някоя буква $b\in X$.
    Тогава \[\alpha\cdot\beta = (\alpha\cdot\gamma)\cdot b.\]
\end{enumerate}

Сега можем да даден алтернативни дефиниции на понятията начало и край на дума.
Казваме, че думата $\alpha$ е {\bf начало} на думата $\beta$, ако съществува дума $\gamma \in X^\star$ такава, че
$\beta = \alpha\cdot\gamma$.
Аналогично дефинираме $\alpha$ да бъде {\bf край} на думата $\beta$, ако съществува $\gamma \in X^\star$ такава, че
$\beta = \gamma \cdot \alpha$.

\begin{problem}
  Какви свойства имат следните релации?
  \begin{enumerate}[a)]
  \item
    $\alpha R \beta \iff \alpha \mbox{ е начало на }\beta$ 
  \item
    $\alpha R \beta \iff \alpha \mbox{ е край на }\beta$
  \item
    $\alpha R \beta \iff \alpha \mbox{ е начало на }\beta \vee (\exists\alpha_1\in X^{*})(\exists a,b\in X)(\alpha_1 a \mbox{ е начало на }\alpha\ \&\ \alpha_1 b \mbox{ е начало на } \beta)$
  \item
    \marginpar{това не е лексикографската наредба}
    $R\subseteq (\{0,1\}^{n})^{2}$ е определена като
    \[\langle{a_1,\dots,a_n}\rangle R \langle{b_1,\dots,b_n}\rangle \iff a_1\leq b_1\ \&\dots\ \&\ a_n\leq b_n\]
  \item
    $R\subseteq (\{0,1\}^{n})^{2}$ е определена като
    \[\langle{a_1,\dots,a_n}\rangle R \langle{b_1,\dots,b_n}\rangle \iff (\exists i : 1\leq i\leq n)((\forall j < i)(a_j = b_j)\ \&\ a_i \leq b_i)\]
\end{enumerate}
\end{problem}
\begin{proof}
  \begin{enumerate}[1)]
  \item[в)]
    \begin{enumerate}[(i)]
    \item
      Ще проверим дали $R$ е рефлексивна, т.е. дали $(\forall\gamma\in X^{*})[\gamma R\gamma]$.
      Очевидно $R$ е рефлексивна, защото всяка дума е начало на самата себе си ($\alpha = \alpha\cdot\varepsilon$).
    \item
      Ще проверим дали $R$ е транзитивна, т.е. дали $(\forall\alpha,\beta,\gamma\in X^*)[\alpha R\beta\wedge\beta R\gamma\rightarrow\alpha R\gamma]$.
      Тук трябва да разгледаме четири случая:
      \begin{enumerate}[a)]
      \item
        $\beta = \alpha\cdot\delta$ и $\gamma = \beta\cdot\rho$, за някои $\delta$ и $\rho$.
        Тогава $\gamma = \alpha\cdot(\delta\cdot\rho)$, следователно $\alpha$ е начало на $\gamma$.
      \item
        $\beta = \alpha\cdot\delta$ и $\beta = \rho\cdot a\cdot \beta'$ и $\gamma = \rho\cdot b\cdot\gamma'$.
        Ако $\alpha$ е начало на $\rho$, то $\alpha R \gamma$.
        Ако $\rho\cdot a$ е начало на $\alpha$, тогава $\alpha = \rho\cdot c\cdot\alpha'$ и $\alpha R \gamma$.
      \item
        $\alpha = \rho\cdot a\cdot \alpha'$ и $\beta = \rho\cdot b\cdot\beta'$ и $\gamma = \beta\cdot\delta$ се разглежда аналогично.
      \item
        $\alpha = \rho\cdot a\cdot \alpha'$ и $\beta = \rho\cdot b\cdot\beta'$ и $\beta = \delta\cdot c\cdot \beta'$ и $\gamma = \delta\cdot d\cdot\gamma'$.
        Ако $\rho\cdot b$ е начало на $\delta$, то $\delta = \rho\cdot b\cdot\nu$ и $\gamma = \rho\cdot b\cdot \gamma''$. Тогава $\alpha R \gamma$.
        Ако $\delta\cdot c$ е начало на $\rho$, то $\rho = \delta\cdot c\cdot\nu$ и $\alpha = \delta\cdot c\cdot\alpha''$. Получаваме, че 
        $\alpha R \gamma$.
      \end{enumerate}
    \item
      Ще проверим дали $R$ е симетрична, т.е. дали $(\forall\alpha,\beta\in X^*)[\alpha R\beta \rightarrow \beta R\alpha]$.
      Ако $\alpha = \varepsilon$, то $\alpha R\beta$, но нямаме $\beta R\alpha$.
      Следователно, релацията не е симетрична.
    \item
      Лесно се вижда също, че $R$ не е антисиметрична.
    \end{enumerate}
    
  \end{enumerate}
  
\end{proof}


\begin{problem}
  За всяко естествено число $n$, дефинираме релацията $R_n \subseteq (X^*)^2$ като
  \[\alpha R_n \beta \iff [|\alpha| > n\ \wedge |\beta| > n\wedge (\forall i < n)[a_i = b_i]].\]
  Докажете, че $R_n$ е релация на еквивалентност и намерете броя на класовете на еквивалентност.
\end{problem}

\begin{problem}
  За всяко естествено число $n$, дефинираме релацията $R_n \subseteq (X^*)^2$ като
  \[\alpha R_n \beta \iff \alpha = \beta \vee [|\alpha| > n\ \wedge |\beta| > n\wedge (\forall i < n)[a_i = b_i]].\]
  Докажете, че $R_n$ е релация на еквивалентност и намерете броя на класовете на еквивалентност.
\end{problem}
\begin{proof}
  Броят на класовете на еквивалентност е $\frac{|X|^n - 1}{|X| - 1}$.
\end{proof}


\begin{problem}
  За всяко естествено число $n$, дефинираме релацията $R_n \subseteq (X^*)^2$ като
  \[\alpha R_n \beta \iff |\alpha| = |\beta| > n\wedge (\forall i > n)[a_i = b_i].\]
  Докажете, че $R_n$ е релация на еквивалентност.
\end{problem}

\begin{problem}
  Нека $R$ е релация върху паметта на един компютър и е дефинирана като
  \[xRy \iff x,y\mbox{ са указатели в паметта и }*x = *y.\]
  Докажете, че $R$ е релация на еквивалентност.
\end{problem}

Нека $R \subseteq A\times A$ е бинарна релация.
Да определим множеството $[x]_R$ като
\[[x]_R = \{y\in A\mid \pair{x,y} \in R\}.\]
Ако $R$ е релация на еквивалентност и $x\in Field(R)$, то наричаме множеството $[x]_R$ {\bf клас на еквивалентност} за $x$ относно релацията $R$.

\begin{example}
  \marginpar{$x \equiv y \mod 4 \iff (\exists k \in \Z)(x = y + 4\cdot k)$}
  Нека $\sim_4\ \subseteq\ \Nat\times \Nat$ е бинарна релация, дефинирана като
  \[x\sim_4 y \iff x\equiv y \mod 4.\]
  $\sim_4$ е релация на еквивалентност и има четири класа на еквивалентност
  \[[0]_{\sim_4}, [1]_{\sim_4}, [2]_{\sim_4}, [3]_{\sim_4}.\]
\end{example}

\begin{problem}
  Да дефинираме релацията $R \subseteq \R\times \R$ като:
  \[\pair{x,y}\in R \iff (x-y)\in\Z.\]
  Намерете множествата $[1]_R$ и $[\frac{1}{2}]_R$.
\end{problem}

\begin{lemma}
  Нека $R\subseteq A\times A$ е релация на еквивалентност и $x,y\in A$. 
  Тогава \[[x]_R = [y]_R \iff xRy.\]
\end{lemma}


% \begin{problem}
%   Покажете, че за всяка релация $R$ и $x$, $[x]_R = R[\{x\}]$.
% \end{problem}

%% Rosen Textbook
\begin{problem}
  Кои от следните релации върху множеството от функциите от $\Z$ в $\Z$
  са релации на еквивалентност?
  \begin{enumerate}[a)]
  \item
    $R = \{(f,g)\mid f(1) = g(1)\}$.
  \item
    $R = \{(f,g)\mid f(0) = g(0)\wedge f(1) = g(1)\}$.
  \item
    $R =\{(f,g)\mid (\forall x\in\Z)[f(x)-g(x) = 1]\}$.
  \item
    $R = \{(f,g)\mid (\exists c\in\Z)(\forall x\in\Z)[f(x)-g(x) = c]\}$.
  \item
    $R = \{(f,g)\mid f(0) = g(1)\wedge f(1) = g(0)\}$.
  \end{enumerate}
\end{problem}

\begin{problem}
  Нека $A$ е непразно множество и $f$ е функция с $Dom(f) = A$.
  Дефинираме релацията $R\subseteq A\times A$ като:
  \[R = \{(x,y)\mid x,y\in A\wedge f(x) = f(y)\}.\]
  Докажете, че
  \begin{enumerate}[a)]
  \item
    $R$ е релация на еквивалентност.
  \item
    Определете класовете на еквивалентност на $R$.
\end{enumerate}

\begin{problem}
  Нека $R_1$ и $R_2$ са симетрични релации.
  Проверете дали $\overline{R}_1$, $R_1\cap R_2$ и $R_1 \cup R_2$ са симетрични.
\end{problem}

\begin{problem}
  Докажете, че подмножество на всяка антисиметрична релация е също антисиметрична.
\end{problem}

\end{problem}




%%% Local Variables: 
%%% mode: latex
%%% TeX-master: "discrete-math"
%%% End: 

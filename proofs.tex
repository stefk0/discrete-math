\chapter{Доказателства на твърдения}

Ще разгледаме два основни метода за доказателства на твърдения.

\section{Допускане на противното}

\begin{problem}
  \label{prob:even-number-square}
  За всяко $a \in \Int$, ако $a^2$ е четно, то $a$ е четно.
\end{problem}
\begin{proof}
  Твърдението може да се запише като
  \[(\forall a\in\Z)[a^2\mbox{ е четно}\ \rightarrow\ a\mbox{ е четно}].\]
  \marginpar{$\neg (\forall x)(A(x) \rightarrow B(x))$ е еквивалентно на $ \equiv (\exists x)(A(x) \wedge \neg B(x))$}
  Да допуснем противното, т.е.
  \[(\exists a\in\Z)[a^2\mbox{ е четно}\ \wedge\ a\mbox{ не е четно}].\]
  Да вземем едно такова нечетно $a$, за което $a^2$ е четно.
  Това означава, че $a = 2k+1$, за някое $k \in \Z$,
  и \[a^2 = (2k+1)^2 = 4k^2 + 4k + 1,\]
  което очевидно е нечетно число.
  Но ние допуснахме, че $a^2$ е четно.
  Така достигаме до противоречие, следователно нашето допускане е грешно 
  и 
  \[(\forall a\in\Z)[a^2\mbox{ е четно}\ \rightarrow\ a\mbox{ е четно}].\]
\end{proof}

\begin{problem}
  Всяко естествено число $n \geq 2$ може да се запише като произведение на прости числа.
\end{problem}
\begin{proof}
  Доказателството протича с индукция по $n$.
  \begin{enumerate}[a)]
  \item 
    За $n = 2$  е ясно.
  \item
    Ако $n+1$ е просто число, то всичко е ясно.
    Ако $n+1$ е съставно, то \[n + 1 = n_1\cdot n_2.\]
    Тогава $n_1 = p^{n_1}_1\cdots p^{n_k}_k$ и $n_2 = q^{m_1}_1\cdots q^{m_r}_r$,
    където $p_1,\dots,p_k$ и $q_1,\dots,q_r$ са прости числа.
    Тогава е ясно, че $n+1$ също е произведение на прости числа.
  \end{enumerate}
\end{proof}

\begin{thm}[Основна теорема на аритметиката]
  \label{th:main-arithmetic}
  Всяко естествено число $n \geq 2$ може да се запише {\bf по единствен начин} като произведение на прости числа.
\end{thm}
\begin{proof}
  Вече знаем, че всяко число може да се представи като произведение на прости числа.
  Да допуснем, че има такива, които имат няколко различни такива представяния.
  От всички тези числа, нека изберем {\em най-малкото с това свойство}.
  Да го означим с $n$. Имаме, че  
  \[n = p_1p_2\dots p_k = q_1q_2\dots q_m.\]
  Можем да приемем, че простите числа $p_1,\dots,p_k$ и $q_1,\dots,q_m$ са подредени във възходящ ред.
  Без ограничение на общността, нека $p_1 < q_1$.
  Знаем със сигурност, че $p_1 \neq q_1$, защото ако $p_1 = q_1$, то $n' = n/p_1$ 
  е число по-малко от $n$, което има две различни представяния като произведение на прости числа, което е противоречие с избора ни на $n$ - 
  най-малкото такова число.
  
  И така, нека $p_1 < q_1$. Следователно, $p_1 < q_i$, за $i = 1, \dots, m$.
  Тогава съществуват числа $a_i$, $r_i$, за които:
  \[q_i = a_ip_1 + r_i.\]
  Ясно е, че $0 < r_i < p_1 < q_i$, $i = 1,\dots, m$ и 
  \[n' = r_1r_2\dots r_m < q_1q_2\dots q_m = n.\]
  Числото $n'$ може да се представи като произведение на прости числа, като разложим $r_1,\dots,r_m$.
  Знаем, че в това произведение {\em не участва} $p_1$, защото $r_i < p_1$.
  Освен това,
  \[n = q_1q_2\dots q_m = (a_1p_1+r_1)(a_2p_1 + r_2)\dots(a_mp_1+r_m) = A + \underbrace{r_1r_2\dots r_m}_{n'}.\]
  Понеже $p_1 | A$ и $p_1 | n$, то $p_1 | n'$.
  Това означава, че можем да получим друго представяне на $n'$ като произведение на прости числа, в което {\em участва} $p_1$.
  Това е противоречие с минималността на $n$.
\end{proof}

\begin{thm}[Безу]
  Нека $a, b \in \Int$, като поне едно от двете не е $0$.
  Тогава съществуват $x,y \in \Int$, такива че 
  \[xa + yb = \text{НОД}(a,b).\]
\end{thm}
\begin{proof}
  За дадените числа $a, b \in \Int$, да разгледаме множеството
  \[S = \{x \mid x > 0\ \&\ x = ma + nb, \text{ за някои }m,n \in \Int \}.\]
  Лесно се съобразява, че $S \neq \emptyset$.
  
  Да вземем {\em най-малкия елемент} $s \in S$.
  Тогава  $s = ua+vb$.
  Да разгледаме произволен елемент $x \in S$, $x = ma + nb$ и да допуснем, че $s \not\vert x$.
  Тогава $x = qs + r$ и $0 < r < s$.
  \begin{align*}
    r & = qs - x\\
      & = qua + qvb - ma - nb\\
      & = a(qu - m) + b(qv - n)
  \end{align*}
  Понеже $r > 0$, то $r \in S$.
  Но от $s > r$ достигаме до противоречие с минималността на $s$.
  Следователно, $s | x$.

  Да означим $d = \text{НОД}(a,b)$.
  Вече знаем, че за всяко $x \in S$, $s|x$.
  Понеже $s \vert \abs{a}$ и $s \vert \abs{b}$, $1 \leq s \leq d$.
  Но по определение, $d \vert a$ и $d \vert b$.
  Тогава, $d \vert ua$ и $d \vert vb$ и $d \vert (ua+vb) = s$.
  Получаваме, че $d \leq s$.
  Следователно, $\text{НОД}(a,b) = s$.
\end{proof}

\begin{problem}
  Докажете, че
  \[\mbox{НОК}(a,b) = \frac{a.b}{\mbox{НОД}(a,b)}.\]
\end{problem}
\begin{proof}
  Нека $D = \mbox{НОД}(a,b)$.
  Това означава, че
  \[a = Da_1,\ b = Db_1,\ \mbox{НОД}(a_1,b_1) = 1.\]
  Ще докажем, че за всяко $g$, за което $a | g$ и $b | g$,
  то съществува $q_1$, такова че 
  \[g = \frac{ab}{D}q_1.\]
  Да разгледаме $g$, такова че $a | g$ и $b | g$.
  \begin{align*}
    D|a \text{ и } a|g\ \Rightarrow\ & g = Da_1q,\\
    D|b \text{ и } b|g\ \Rightarrow\ & \frac{g}{b} = \frac{Da_1q}{Db_1} = \frac{a_1q}{b_1}.
  \end{align*}
  Понеже $(a_1,b_1) = 1$, то $q = b_1q_1$.
  Тогава $\frac{g}{b} = a_1q_1$ и 
  \[g = \frac{ab}{D}q_1.\]
  Тогава за $q_1 = 1$, 
  \[g = \frac{ab}{D}.\]
\end{proof}


% Можем да обобщим предишната задача по следния начин.
% \begin{problem}
%   Нека $p$ е просто число.
%   Тогава за всяко $a \in \Nat$, ако $p \vert a^2$, то $p \vert a$.  
% \end{problem}
% \begin{proof}
%   Да допуснем, че съществува число $a\in \Nat$, за което 
%   $p \vert a^2$, но $p \not\vert a$.
  
%   Първо ще докажем, че в този случай $p \leq a$.
%   Да допуснем, че $p > a$. Тогава $a = p - m$, за някое $m$.
%   Тогава $a^2 = p(p - 2m) + m^2$.
%   Понеже $p \vert a^2$, то $p \vert m^2$

%   Да вземем най-малкото такова $a \in \Nat$.
%   Тогава имаме, че $a^2 = kp$ и $a = lp + r$, където $r < p$.
%   \[a^2 = kp = l^2p^2 + 2lpr + r^2.\]
%   Получаваме, че $p \vert r^2$, но $p \not\vert r$.
%   Лесно се съобразява, че $r < a$.
  
% \end{proof}



\begin{problem}
  За всеко $a,b \in \Z$ и за всяко просто число $p$,
  ако $p\vert ab$, то $p\vert a$ или $p\vert b$ (може и двете).
\end{problem}
\begin{proof}
  Нека $p \vert ab$. Тогава $ab = kp$.
  Знаем, че $ab = p^{n_1}_1\dots p^{n_m}_m = kp$.
  Тогава $p = p_i$, за някое $i = 1,\dots,m$.
  Следва, че $p$ участва в разлагането на прости множители или на $a$ или на $b$.
\end{proof}


\begin{problem}
  Докажете, че следните числа {\bf не} са рационални:
  \begin{enumerate}[a)]
  \item
    $\sqrt{2},\sqrt{3},\sqrt{6}$;
  \item
    $\sqrt{p}$, където $p$ е просто число;
  \item
    $\sqrt{n}$, където $n$ не е точен квадрат;
  \item
    $\sqrt{pq}$ и $\sqrt{\frac{p}{q}}$, където $p$ и $q$ са различни прости числа;
  \item
    $log_23$.
  \end{enumerate}
\end{problem}
\begin{proof}
  \begin{enumerate}[a)]
  \item
    Да допуснем, че $\sqrt{2}$ е рационално число. Тогава  съществуват $a,b \in \Z$, такива че:
    \[\sqrt{2} = \frac{a}{b}.\]
    Без ограничение, можем да приемем, че $a$ и $b$ са естествени числа,
    които нямат общи делители, т.е. не можем да съкратим дробта $\frac{a}{b}$.
    Получаваме, че \[2b^2 = a^2.\]
    Тогава $a^2$ е четно число и от Задача \ref{prob:even-number-square}, $a$ е четно число.
    Нека $a = 2k$. Получаваме, че
    \[2b^2 = 4k^2,\]
    от което следва, че
    \[b^2 = 2k^2.\]
    Това означава, че $b$ също е четно число, $b = 2n$, за някое $n \in \Z$.
    Следователно, $a$ и $b$ са четни числа и имат общ делител $2$,
    което е противоречие с нашето допускане. Така достигаме до
    противоречие.
    Накрая заключаваме, че $\sqrt{2}$ не е рационално число.
  \item
    Да допуснем, че $\sqrt{p} = \frac{m}{n}$ и $m$ и $n$ са взаимно прости.
    Тогава $n^2p = m^2$. От това следва, че $p | m^2$ и следователно $p | m$.
    Нека $m = kp$. Тогава $n^2p = k^2p^2$ и $n^2 = k^2p$.
    Сега имаме, че $p | n$, но така стигаме до противоречие с факта, че $m$ и $n$
    са взаимно прости.
  \end{enumerate}
\end{proof}

\section{Индукция върху $\Nat$}

Доказателството с индукция по $\Nat$ представлява следната схема:
\begin{prooftree}
  \AxiomC{$P(0)$}
  \AxiomC{$(\forall x\in\Nat)[P(x)\rightarrow P(x+1)]$}
  \BinaryInfC{$(\forall x\in\Nat) P(x)$}
\end{prooftree}

Това означава, че ако искаме да докажем, че свойството $P(x)$ е вярно за всяко $x\in\Nat$,
то трябва да докажем първо, че $P(0)$ и след това ако $P(x)$ вярно, то също така е вярно $P(x+1)$.

\begin{problem}
  Докажете, че:
  \begin{enumerate}[a)]
  \item
    $3^n$ е нечетно;
  \item
    $n < 2^n$;
  \item
    $2^n < n!$ за $n \geq 4$;
  \item
    \marginpar{$a\vert b\ \iff\ (\exists c\in\Nat)(b = c\cdot a)$}
    $3 \vert (n^3 - n)$;
  \item
    $6 \vert (n^3 + 11n)$;
  \item
    $9 \vert (2^{2n} + 15n - 1)$;
  \item
    $57 \vert (7^{n+2} + 8^{2n+1})$;
  \item
    \marginpar{$\Ps(A) = \{B\mid B\subseteq A\}$}
    \marginpar{$\abs{A}$ - брой елементи на $A$}
    \marginpar{$\abs{\Ps(A)}$ - брой на подмножествата на $A$}
    за всяко крайно множество $A$,
    ако $\abs{A} = n$, то $\abs{\Ps(A)} = 2^n$;
  \item
    $C\setminus \bigcup^n_{i=0}A_i = \bigcap^n_{i=0}(C\setminus A_i)$;
  \item
    $C\setminus \bigcap^n_{i=0}A_i = \bigcup^n_{i=0}(C\setminus A_i)$;
  \item
    ако $(\forall i \leq n)[A_i \subseteq B_i]$, то
    $\bigcap^n_{i=0}A_i \subseteq \bigcap^n_{i=0}B_i$;
  \item
    ако $(\forall i \leq n)[A_i \subseteq B_i]$, то
    $\bigcup^n_{i=0}A_i \subseteq \bigcup^n_{i=0}B_i$;
  \item
    $(\bigcap^n_{i=0} A_i)\cup B = \bigcap^n_{i=0} (A_i\cup B)$;
  \item
    $(\bigcup^n_{i=0} A_i)\cap B = \bigcup^n_{i=0} (A_i\cap B)$;
  \item
    $\bigcap^n_{i=0} (A_i\setminus B) = (\bigcap^n_{i=0} A_i) \setminus B$;
  \item
    $\neg (p_1\vee p_2\vee\dots\vee p_n) \iff (\neg p_1 \wedge \neg p_2 \wedge\dots\wedge \neg p_n)$;
  \item
    $\neg (p_1\wedge p_2\wedge\dots\wedge p_n) \iff (\neg p_1 \vee \neg p_2 \vee \dots \vee \neg p_n)$;
  \item
    $\sum^n_{i=0} 2^i = 2^{n+1} - 1$;
  \item
    $\sum^n_{i=0} ar^i = \frac{ar^{n+1}-a}{r-1}$ за $r \neq 1$;
  \item
    $\sum^n_{i=1}i = \frac{n(n+1)}{2}$;
  \item
    $\sum^n_{i=1}i^2 = \frac{n(n+1)(2n+1)}{6}$;
  \item
    $\sum^n_{i=1}i^3 = \frac{n^2(n+1)^2}{4}$;
  \item
    $\sum^n_{i=1}i^4 = \frac{n(n+1)(2n+1)(3n^2+3n-1)}{30}$;
  \item
    $\sum^n_{i=0}(2i+1)^2 = \frac{(n+1)(2n+1)(2n+3)}{3}$;
  \item
    $\sum^n_{i=1}\frac{1}{i(i+1)} = \frac{n}{n+1}$;
  \item
    $\sum^n_{i=0}(-\frac{1}{2})^i = \frac{2^{n+1}+(-1)^n}{3\cdot 2^n}$;

  \end{enumerate}
\end{problem}

%  \begin{problem}
%   \begin{enumerate}
%   \item
%     за $n > 1$ е изпълнено
%     \[\frac{1}{n+1} + \frac{1}{n+2} + \cdots + \cdots \frac{1}{2n} > \frac{13}{24};\]
%   \item
%     за произволни реални числа $a_1,\dots,a_k \geq 0$,
%     \[\sqrt[k]{a_1\cdots a_k}\leq \frac{a_1+\cdots +a_k}{k};\]
%   \item
%     за произволни реални числа $a_1,\dots,a_n \geq 0$,
%     \[(1+a_1)\cdots (1+a_n) \geq (1+ \sqrt[n]{a_1\cdots a_n}).\]
%   \item
%     $\sum^n_{i=1}\frac{1}{\sqrt{i}} > 2(\sqrt{n+1} - 1)$;
%   \end{enumerate}
% \end{problem}
% \begin{proof}
%   \begin{enumerate}[]
%   \item
%     Първо, доказва се с индукция, че твърдението е вярно за $n = 2k$.
%     Очевидно е вярно за $n = 2$.
%     Да допуснем, че $\sqrt[k]{\prod^{k}_{i=1} a_i}\leq \frac{\sum^{k}_{i=1} a_i}{k}$.
%     Ще докажем, че $\sqrt[2k]{\prod^{2k}_{i=1} a_i}\leq \frac{\sum^{2k}_{i=1} a_i}{2k}$.
%     \[
%     \begin{array}{lll}
%       \sqrt[2k]{\prod^{2k}_{i=1} a_i} & = & \sqrt[2]{\sqrt[k]{\prod^{k}_{i=1} a_i}\sqrt[k]{\prod^{2k}_{i=k+1} a_i}}\\
%       & \leq & \frac{\sqrt[k]{\prod^{k}_{i=1} a_i} + \sqrt[k]{\prod^{2k}_{i=k+1} a_i}}{2} \\
%       & \leq & \frac{\frac{\sum^{k}_{i=1} a_i}{k} + \frac{\sum^{2k}_{i=k+1} a_i}{k}}{2}\\
%       & = & \frac{\sum^{2k}_{i=1} a_i}{2k}\\
%     \end{array}
%     \]
    
%     Сега, ще докажем, че ако твърдението е вярно за $n = k$, то е вярно и за $n = k-1$.
%     Нека \[\sqrt[k]{\prod^{k}_{i=1}a_i} \leq \frac{\sum^{k}_{i=1} a_i}{k}.\]
%     Това е вярно за произволни $a_1,\dots,a_k$.
%     Нека да изберем $a_k$, така че
%     \[\frac{\sum^{k}_{i=1} a_i}{k} = \frac{\sum^{k-1}_{i=1} a_i}{k-1},\] т.е.
%     \[a_k = \frac{\sum^{k-1}_{i=1} a_i}{k-1}.\]
%     Получаваме, че:
%     \[
%     \begin{array}{rll}
%       \sqrt[k]{\prod^{k}_{i=1}a_i} & \leq & \frac{\sum^{k}_{i=1} a_i}{k}\\
%       \frac{(\prod^{k-1}_{i=1}a_i)(\sum^{k-1}_{i=1}a_i)}{k-1} & \leq & (\frac{\sum^{k-1}_{i=1}a_i}{k-1})^{k} \\
%       \prod^{k-1}_{i=1}a_i & \leq & (\frac{\sum^{k-1}_{i=1}a_i}{k-1})^{k-1}\\
%       \sqrt[k-1]{\prod^{k-1}_{i=1}a_i} & \leq & \frac{\sum^{k-1}_{i=1}a_i}{k-1}\\
%     \end{array}
%     \]
%   \end{enumerate}
% \end{proof}

% \begin{problem}
%   \marginpar{Нютонов бином}
%   Нека положим 
%   \[\binom{n}{m} = \frac{n!}{(n-i)!i!}\]
%   Проверете:
%   \begin{enumerate}[a)]
%   \item 
%     $2^n = \sum^n_{i=0}\binom{n}{i}$;
%   \item
%     $(x+y)^n = \sum^n_{i=0} \binom{n}{i}x^{n-i}y^{i}$;
%   \item
%     $\binom{n+1}{m+1} = \binom{n}{m} + \binom{n}{m+1}$;
%   \end{enumerate}
% \end{problem}

% \begin{problem}
%   \marginpar{Наричат се хармонични числа}
%   Нека да положим
%   $H_n = \sum^n_{i=1}\frac{1}{i}$.
%   Проверете:
%   \begin{enumerate}[a)]
%   \item
%     $\sum^n_{i=1}H_i = (n+1)H_n - n$.
%   \item
%     $H_{2^k} \geq 1+ k/2$, за всяко $k \geq 0$.
%   \end{enumerate}
% \end{problem}

\begin{problem}
  Докажете, че за произволни числа $x,y$ и естествено число $n$ е изпълнено равенството:
  \[(x+y)^n = \sum^{n}_{i=0}\binom{n}{i}x^iy^{n-i}.\]
\end{problem}
\begin{problem}[Теорема на Ферма]
  Нека $p$ е просто число. Тогава докажете, че за числото $a$:
  \begin{enumerate}[i)]
  \item
    $a^p \equiv a\ (\bmod\ p)$;
  \item 
    $a^{p-1} \equiv 1\ (\bmod\ p)$, ако $a$ не се дели на $p$.
  \end{enumerate}
\end{problem}
\begin{proof}
  Да разгледаме следното равенство:
  \[(x+1)^p = x^p + \binom{p}{1}x^{p-1} + \binom{p}{2}x^{p-2} + \dots + 1 = \sum^{p}_{i=0}\binom{p}{i}x^i\]
  За $i = 1,\dots,p-1$, всяко от числата $\binom{p}{i}$ се дели на $p$, то 
  \[(x+1)^p \equiv x^p + 1\ (\bmod\ p).\]
  Нека сега да разгледаме следната редица, която от се получва от горното равенство за $x = a-1,a-2,\dots,2$:
  \begin{align*}
    a^p & \equiv (a-1)^p+1\ (\bmod\ p)\\
    (a-1)^p & \equiv (a-2)^p+1\ (\bmod\ p)\\
    \dots & \dots\dots\\
    2^p & \equiv 1\ (\bmod\ p).
  \end{align*}
  Като съберем тези сравнения, получаваме:
  \[a^p \equiv a\ (\bmod\ p).\]
  Ако $a$ не се дели на $p$, то
  \[a^{p-1} \equiv 1\ (\bmod\ p).\]
\end{proof}

\begin{problem}
  \marginpar{Числа на Фибоначи}
  Да определим следната редица:
  \[F_0 = 0,F_1 = 1,\dots,F_{n+2} = F_{n} + F_{n+1}.\]
  Проверете:
  \begin{enumerate}[a)]
  \item
    $\sum^n_{i=0} F^2_i = F_{n}F_{n+1}$;
  \item
    $\sum^n_{i=1} F_{2i-1} = F_{2n}$;
  \item
    $\sum^{2n}_{i=1}F_{i-1}F_{i} = F^2_{2n}$;
  \item
    единствено членовете от вида $F_{3n}$ са четни;
  \item
    за $n > 0$, $F_{n+1}F_{n-1} - F^2_n = (-1)^n$;
  \item
    $F_{m+n} = F_{m-1}\cdot F_{n} + F_m \cdot F_{n-1}$;
  \item
    ако $m\vert n$, то $F_m \vert F_n$.
  \item
    \marginpar{Използвайте, че $\phi^2 = \phi + 1$}
    ако $n\geq 3$, то $F_n > \phi^{n-2}$,
    където $\phi = \frac{1+\sqrt{5}}{2}$.
  \end{enumerate}
\end{problem}

\newpage 
\subsection*{Пълна индукция върху $\Nat$}

Доказателство с пълна индукция по $\Nat$ за свойството $P$ представлява следната схема:
\begin{prooftree}
  \AxiomC{$(\forall x\in\Nat)[(\forall y\in \Nat)[y < x\ \rightarrow P(y)]\rightarrow P(x)]$}
  \UnaryInfC{$(\forall x\in\Nat) P(x)$}
\end{prooftree}
Нека да проверим принципа за пълна индукция.
Да допуснем, че принципът не е верен, т.е. за някое свойство $P$ е изпълнено, че
\[(\forall x\in\Nat)[(\forall y\in \Nat)[y < x\ \rightarrow P(y)]\rightarrow P(x)]\ \wedge\ (\exists x\in\Nat) \neg P(x).\]
Да вземем най-малкия елемент $x_0$, за който $\neg P(x_0)$.
Тогава \[(\forall y\in \Nat)[y < x_0\ \rightarrow P(y)]\]
и следователно:
\begin{prooftree}
  \AxiomC{$(\forall y\in \Nat)[y < x_0\ \rightarrow P(y)]$}
  \AxiomC{$(\forall x\in\Nat)[(\forall y\in \Nat)[y < x\ \rightarrow P(y)]\rightarrow P(x)]$}
  \UnaryInfC{$(\forall y\in \Nat)[y < x_0\ \rightarrow P(y)]\rightarrow P(x_0)$}
  \BinaryInfC{$P(x_0)$}
\end{prooftree}
Така достигаме до противоречие, защото получаваме, че $P(x_0)\wedge \neg P(x_0)$.
% \begin{prop}
%   Двете форми на индукция са еквивалентни.
% \end{prop}
% \begin{proof}
%   \begin{enumerate}
%   \item 
%     Нека имаме схемата за ``обикновена'' индукция.
%     Ще докажем, че тогава имаме схемата за пълна индукция.
%     Нека \[(\forall x\in\Nat)[(\forall y\in \Nat)[y < x\ \rightarrow P(y)]\rightarrow P(x)].\]
%     Ще докажем с ``обикновена'' индукция, че $(\forall x\in\Nat)P(x)$.
    
%     Нека да положим $Q(x) = (\forall y\in\Nat)[y<x \rightarrow P(y)]$.
%     Очевидно е, че $Q(0)$ е изпълнено.
%     Освен това, 
%     \begin{prooftree}
%       \AxiomC{$(\forall x\in\Nat)[(\forall y\in \Nat)[y < x\ \rightarrow P(y)]\rightarrow P(x)]$}
%       \UnaryInfC{$(\forall x\in\Nat)[Q(x)\rightarrow P(x)]$}
%       \UnaryInfC{$(\forall x\in\Nat)[Q(x)\rightarrow Q(x)\wedge P(x)]$}
%       \AxiomC{$(\forall x\in\Nat)[Q(x)\wedge P(x)\ \rightarrow\ Q(x+1)]$}
%       \BinaryInfC{$(\forall x\in\Nat)[Q(x)\rightarrow Q(x+1)]$}
%     \end{prooftree}
%     Получаваме, че 
%     \begin{prooftree}
%       \AxiomC{$Q(0)$}
%       \AxiomC{$(\forall x\in\Nat)[Q(x)\rightarrow Q(x+1)]$}
%       \BinaryInfC{$(\forall x\in\Nat)Q(x)$}
%       \AxiomC{$(\forall x\in\Nat)Q(x)\ \rightarrow\ (\forall x\in\Nat)P(x)$}
%       \BinaryInfC{$(\forall x\in\Nat)P(x)$}
%     \end{prooftree}
%   \item
%     Нека сега имаме схемата за пълна индукция.
%     Ще докажем, че имаме и схемата за ``обикновена'' индукция.
%     За тази цел, нека имаме, че $P(0)$ и $(\forall x\in\Nat)[P(x) \rightarrow P(x+1)]$.
%     Достатъчно е да докажем, че
%     \[(\forall x\in\Nat)[(\forall y\in \Nat)[y < x \rightarrow P(y)] \rightarrow P(x)],\]
%     защото тогава ще приложим пълна индукция и ще получим, че \[(\forall x\in\Nat)P(x).\]
    
%     Да допуснем противното, т.е.
%     \[(\exists x\in\Nat)[(\forall y\in \Nat)[y < x \rightarrow P(y)] \wedge \neg P(x)].\]
%     Да разгледаме едно такова $x_0$, за което
%     \[(\forall y\in \Nat)[y < x_0 \rightarrow P(y)] \wedge \neg P(x_0).\]
%     \begin{itemize}
%     \item 
%       Ако $x_0 = 0$, то очевидно е изпълнено, че $(\forall y\in\Nat)[y < 0 \rightarrow P(y)]$.
%       Тогава $\neg P(0)$, което е противоречие.
%     \item
%       Ако $x_0 > 0$, тогава от
%       $(\forall y\in\Nat)[y < x_0 \rightarrow P(y)]\ \rightarrow\ P(x_0-1)$
%       следва, че $P(x_0-1)$.
%       Тогава 
%       \begin{prooftree}
%         \AxiomC{$(\forall x)[P(x)\rightarrow P(x+1)]$}
%         \AxiomC{$P(x_0-1)$}
%         \BinaryInfC{$P(x_0)$}
%       \end{prooftree}
%       Отново достигаме до противоречие.
%     \end{itemize}
%     Следователно нашето допускане е невярно и
%     \[(\forall x\in\Nat)[(\forall y\in \Nat)[y < x \rightarrow P(y)] \rightarrow P(x)].\]
%   \end{enumerate}
% \end{proof}

\begin{problem}
  Докажете, че за всяко $x,y\in\Nat$
  \[f(x,y) = x^y,\]
  където
  \begin{align*}
    f(x,y) = 
    \begin{cases}
      1, & x\neq 0\ \wedge\ y = 0\\
      f(x,y-1) * x, & x\neq 0\ \wedge y\mbox{ е нечетно}\\
      f(x,y/2) * f(x,y/2), & x\neq 0\ \wedge y\mbox{ е четно}
    \end{cases}
  \end{align*}
\end{problem}


\begin{problem}
  Функцията $f:\Nat\times\Nat \to \Nat$, определена като 
  \[f(x,y) = 2^x(2y+1)-1,\]
  е биективна.
\end{problem}
\begin{proof}
  Първо, да вземем едно произволно естествено число $z$.
  Да разгледаме единственото представяне на $z+1$ като произведение на прости числа.
  Нека $z+1 = 2^{x}p^{k_1}_{i_1}p^{k_2}_{i_2}\dots p^{k_n}_{i_n}$, като $2 < p_{i_1} < p_{i_2} < \dots < p_{i_n}$.
  Това означава, че $z' = p^{k_1}_{i_1}p^{k_2}_{i_2}\dots p^{k_n}_{i_n}$ е нечетно число и следователно съществува $y$,
  за което $2y+1 = z'$.
  Следователно, $2^x(2y+1) = z+1$ и тогава $f(x,y) = z$.

  Сега, нека $\pair{x,y} \neq \pair{x',y'}$.
  Интересният случай е когато $x \neq x'$ и $y \neq y'$.
  Да допуснем, че $f(x,y) = f(x',y') = z+1$.
  Това означава, че числото $z$ има две различни представяния като произведение на прости числа,
  което е противоречие.
\end{proof}


% \begin{problem}
%   Нека фунцкията $f:\Nat\to\Nat$ е определена като
%   \begin{align*}
%     f(x) = x-10, & x > 100\\
%     f(x) = f(f(x+11)), & x < 100.
%   \end{align*}
%   Докажете, че $(\forall x \leq 100)[f(x) = 91]$.
% \end{problem}

\subsection*{Индукция върху $\Nat\times\Nat$}

\begin{dfn}
  Определяме лексикографската наредба $\prec$ върху $\Nat\times\Nat$ като
  \[\pair{x,y} \prec \pair{x^\prime,y^\prime}\ \iff\ x < x^\prime \vee (x = x^\prime\ \wedge\ y < y^\prime).\]
  Наричаме двойката $\pair{x_0,y_0}$ {\em минимална} за множеството $A \subseteq \Nat\times\Nat$, ако
  \[\pair{x_0,y_0}\in A\ \wedge\ (\forall \pair{x,y}\in A)[\pair{x,y}\not\prec\pair{x_0,y_0}].\]
\end{dfn}

\begin{prop}
  Всяко непразно подмножество $A\subseteq \Nat\times \Nat$ притежава поне един {\em минимален} елемент.
\end{prop}

\begin{prop}
  Не съществуват безкрайни строго намаляващи редици относно $\succ$ в $\Nat\times\Nat$, т.е.
  не съществува 
  \[\pair{x_0,y_0} \succ \pair{x_1,y_1}\succ \pair{x_2,y_2} \succ \dots \succ \pair{x_n,y_n}\succ\dots\]
\end{prop}

\begin{dfn}
  Доказателството с индукция върху $\Nat\times\Nat$ представлява следната схема:
  \begin{prooftree}
    \AxiomC{$(\forall\pair{x,y})[(\forall \pair{x^\prime,y^\prime})[\pair{x^\prime,y^\prime} \prec \pair{x,y}\ \rightarrow\ P(x^\prime,y^\prime)]\ \rightarrow P(x,y)]$}
    \UnaryInfC{$\forall \pair{x,y} P(x,y)$}
  \end{prooftree}  
\end{dfn}

Да проверим схемата.
Да допуснем, че тя не е вярна, т.е. за някое свойство $P$ е изпълнено, че
\[(\forall\pair{x,y})[(\forall \pair{x^\prime,y^\prime})[\pair{x^\prime,y^\prime} \prec \pair{x,y}\ \rightarrow\ P(x^\prime,y^\prime)]\ \rightarrow P(x,y)],\]
но \[\exists \pair{x,y} \neg P(x,y),\]
т.е. съществува $\pair{x,y} \in \Nat\times\Nat$, за което $\neg P(x,y)$.
Да разгледаме
\[A = \{\pair{x,y}\in \Nat\times\Nat\mid \neg P(x,y)\}.\]
Щом $A$ е непразно, то $A$ има минимален елемент $\pair{x_0,y_0}$.
Тогава
\[(\forall \pair{x^\prime,y^\prime})[\pair{x^\prime,y^\prime} \prec\pair{x_0,y_0}\ \rightarrow P(x^\prime,y^\prime)].\]
Но ние имаме, че 
\[(\forall \pair{x^\prime,y^\prime})[\pair{x^\prime,y^\prime} \prec \pair{x_0,y_0}\ \rightarrow\ P(x^\prime,y^\prime)]\ \rightarrow P(x_0,y_0).\]
Това означава, че $P(x_0,y_0)$, което е противоречие.

\begin{remark}
  За да докажем едно свойство $P$ с индукция по лексикографската наредба върху $\Nat\times\Nat$,
  първо доказваме $P$ за минималната двойка $\pair{0,0}$.
  След това доказваме, че ако $P$ е вярно за всички двойки $\pair{x^\prime,y^\prime}\prec \pair{x,y}$,
  то $P$ е вярно и за $\pair{x,y}$.
\end{remark}

\begin{problem}
  Докажете,  че $f(x,y) = \abs{x-y}$, където
  \begin{align*}
    f(x,y) = 
    \begin{cases}
      y, & x = 0\\
      x, & y = 0\\
      f(x-1,y-1), & \mbox{ иначе}
    \end{cases}
  \end{align*}
\end{problem}
\begin{proof}
  Индукция по $(\Nat^2,\prec)$, където $\prec$ е лексикографската наредба.
  Имаме един минимален елемент $(0,0)$.
  \[f(0,0) = 0 = \abs{0 - 0}.\]
  Да допуснем, че за всяко $(u,v) \prec (x,y)$, 
  \[f(u,v) = \abs{u - v}.\]
  Тогава ако $x > 0, y = 0$, то
  \[f(x,0) = x = \abs{x - 0}.\]
  Ако $x> 0, y > 0$, то
  \[f(x,y) = f(x-1,y-1) = \abs{x-1-y+1} =\abs{x-y}.\]
\end{proof}

\begin{problem}
  Докажете, че $f(x,y) = \mbox{НОД}(x,y)$, където
  за $x,y \in \Nat$,
  \begin{align*}
    f(x,y) = 
    \begin{cases}
      f(x-y,y), & x > y\\
      f(y,x), & x < y\\
      x, & x = y.
    \end{cases}
  \end{align*}
\end{problem}


\newpage
\subsection*{Фундирани множества}
\marginpar{Англ. well-founded sets}

Понякога се налага да правим индукция по по-сложни множества от това на естествените
числа.

\begin{dfn}
  Нека е дадена двойката $(A,R)$, където $A$ е множество, а $R\subseteq A^2$.
  Казваме, че $A$ е фундирано множество относно $R$, ако 
  \begin{itemize}
  \item 
    \marginpar{$R$ задава строга частична наредба върху $A$}
    $R$ е антирефлексивна, транзитивна, асиметрична.
  \item
    ако всяко непразно подмножество $X\subseteq A$ притежава поне един {\em минимален} елемент, т.е.
    \[(\forall X\subseteq A)[X\neq\emptyset \rightarrow (\exists m\in X)\neg(\exists y\in X)[\pair{y,m} \in R]].\]
  \end{itemize}
\end{dfn}

Обърнете внимание, че минималният елемент може да не е уникален.
\begin{example}
  Нека да определим $\prec$ върху $\N$ като
  \[\pair{x,y}\prec\pair{x^\prime,y^\prime}\ \iff\ x < y\ \wedge\ x^\prime < y^\prime.\]
  Тогава например $X = \{\pair{1,2},\pair{2,1},\pair{2,2},\pair{2,3}\}$
  има два минимални елемента - $\pair{1,2}$ и $\pair{2,1}$.
\end{example}


\begin{prop}
  Нека $\prec$ е строга частична наредба върху $A$.
  Следните твърдения са  еквивалентни:
  \begin{enumerate}[a)]
  \item
    ако всяко непразно подмножество $X\subseteq A$ притежава поне един {\em минимален} елемент, т.е.
    \[(\forall X\subseteq A)[X\neq\emptyset \rightarrow (\exists x\in X)\neg(\exists y\in X)[y \prec x]];\]
  \item
    не съществуват безкрайни редици от вида
    \[x_0 \succ x_1 \succ x_2 \succ \cdots \succ x_n \succ \cdots\]
  \end{enumerate}
\end{prop}
\begin{proof}
  \begin{enumerate}
  \item[а)$\ \to\ $б)]
    Да допуснем, че съществува безкрайно-намаляваща редица
    \[x_0 \succ x_1 \succ x_2 \succ \cdots \succ x_n \succ \cdots\]
    Нека $X = \{x_i\mid i \in \Nat\}$.
    Тогава лесно се вижда, че в $X$ няма минимален елемент, което е противоречие.
  \item[б)$\ \to\ $а)]
    Да допуснем, че съществува непразно множество $X \subseteq A$, което не притежава минимален елемент, т.е.
    \[(\forall x\in X)(\exists y\in X)[y \prec x].\]
    Ще построим безкрайно-намаляваща редица относно $\prec$.
    Да вземем произволен $x_0 \in X$. 
    Знаем, че съществува $y \in X, x_0 \prec y$.
    Нека да изберем едно такова $y\in X$ и да означим $x_1 = y$.
    По този начим можем да построим
    \[x_0 \prec x_1 \prec x_2 \prec \cdots \]
  \end{enumerate}
\end{proof}

\begin{prop}
  Нека $(A_1,\prec_1)$ и $(A_2,\prec_2)$ са фундирани.
  Тогава \[(A_1\times A_2, \prec)\]
  е фундирано множество, където
  \[\pair{a_1,a_2}\prec \pair{a^\prime_1,a^\prime_2}\ \iff\ a_1\prec_1 a^\prime_1\ \vee\ (a_1 = a^\prime_1\ \wedge\ a_2\prec_2 a^\prime_2)\]
\end{prop}
\begin{proof}
  Да допуснем, че съществува
  безкрайно намаляваща редица относно $\prec$:
  \[(x_0,y_0)\succ(x_1,y_1) \succ \cdots \succ (x_n,y_n)\succ\cdots\]
  Да разгледаме редицата само от първите компоненти :
  \[x_0 \succeq x_1 \succeq \cdots \succeq x_n \succeq \cdots\]
  Това означава, че съществува число $n_1$, такова че 
  \[(\forall k \geq n_1)[x_{n_1} = x_k].\]
  В противен случай ще получим безкрайно намаляваща редица, което ще бъде
  противоречие с фундираността на $A_1$.
  Аналогично, съществува $n_2$, такова че
  \[(\forall k \geq n_2)[y_{n_2} = y_k].\]
  В противен случай ще получим безкрайно намаляваща редица, което ще бъде
  противоречие с фундираността на $A_2$.
  Нека \[n = \max(n_1,n_2).\]
  Тогава 
  \[(\forall k\geq n)[(x_n,y_n) = (x_k,y_k)].\]
  Така достигаме до противоречие с 
  \[(\forall k \geq n)[(x_n,y_n) \succ (x_k,y_k)].\]
  Следователно $\prec$ задава фундирана наредба върху $A_1\times A_2$.
\end{proof}



\subsection*{Индукция по фундирани наредби}

Доказателството с индукция по фундираното множество $(A,\prec)$ представлява следната схема:
\begin{prooftree}
  \AxiomC{$(\forall x \in A)[(\forall y\in A)[y \prec x\ \rightarrow\ P(y)]\ \rightarrow P(x)]$}
  \UnaryInfC{$(\forall x \in A) P(x)$}
\end{prooftree}
Ако допуснем, че съществува $x \in A$, за което $\neg P(x)$, то да разгледаме
\[X = \{x\in A\mid \neg P(x)\}.\]
Щом това множество е непразно, то $X$ има поне един минимален елемент $x_0$.
Тогава
\[(\forall y\in A)[y \prec x_0\ \rightarrow P(y)].\]
Но това означава, че $P(x_0)$, което е противоречие.

\begin{problem}
  Докажете,  че $f(x,y) = \abs{x-y}$, където
  \begin{align*}
    f(x,y) = 
    \begin{cases}
      y, & x = 0\\
      x, & y = 0\\
      f(x-1,y-1), & \mbox{ иначе}
    \end{cases}
  \end{align*}
\end{problem}
\begin{proof}
  Индукция по $(\Nat^2,\prec)$, където $\prec$ е лексикографската наредба.
  Имаме един минимален елемент $(0,0)$.
  \[f(0,0) = 0 = \abs{0 - 0}.\]
  Да допуснем, че за всяко $(u,v) \prec (x,y)$, 
  \[f(u,v) = \abs{u - v}.\]
  Тогава ако $x > 0, y = 0$, то
  \[f(x,0) = x = \abs{x - 0}.\]
  Ако $x> 0, y > 0$, то
  \[f(x,y) = f(x-1,y-1) = \abs{x-1-y+1} =\abs{x-y}.\]
\end{proof}

\begin{problem}
  Докажете, че $f(x,y) = \mbox{НОД}(x,y)$, където
  за $x,y \in \Nat$,
  \begin{align*}
    f(x,y) = 
    \begin{cases}
      f(x-y,y), & x > y\\
      f(y,x), & x < y\\
      x, & x = y.
    \end{cases}
  \end{align*}
\end{problem}

\begin{problem}
  Докажете, че $f(x,y) = \binom{x}{y}$, където
  за $x \geq y, x,y\in\Nat$,
  \begin{align*}
    f(x,y) = 
    \begin{cases}
      1, & x = 0\ \vee\ y = 0\ \vee\ x = y\\
      f(x-1,y) + f(x-1,y-1), & \mbox{ иначе}
    \end{cases}
  \end{align*}
\end{problem}

% \begin{problem}
%   \begin{align*}
%     f(x,y) = 
%     \begin{cases}
%       y+1, & x = 0\\
%       f(x-1,1), & x > 0\ \wedge\ y = 0\\
%       f(x-1,f(x,y-1), & x > 0\ \wedge\ y > 0.
%     \end{cases}
%   \end{align*}
%   Докажете, че $f$ е тотална функция.

% \end{problem}


% \begin{problem}
%   Докажете, че за всеки $m,n\in\Nat$
%   съществуват $p,q\in\Z$, такива че
%   \[p\cdot m +  q\cdot n = \mbox{НОД}(m,n).\]
% \end{problem}

$\mbox{НОК}(x,y) = z$ точно тогава, когато
$z$ е най-малкото число, за което е изпълнено свойството $x | z\ \&\ y | z$.



%%% Local Variables: 
%%% mode: latex
%%% TeX-master: "discrete-math"
%%% End: 


\documentclass[a4paper]{article}
\usepackage{geometry}
\geometry{margin=1in}
\usepackage[english,bulgarian]{babel}
\usepackage{amssymb}
\usepackage{amsmath}
\usepackage{mathrsfs}
\usepackage{latexsym}
\usepackage{amsthm}
\usepackage{enumerate}
\setlength{\parskip}{2.3ex}            % vertical space between paragraphs
\setlength{\parindent}{0in}            % amount of indentation of paragraph
%this package allows for hyperlinks within the pdf document
\usepackage[colorlinks=true, linkcolor=blue,pdfstartview=FitV,
citecolor=green, urlcolor=blue]{hyperref}

\newtheorem{thm}{Theorem}
\newtheorem{lemma}{Lemma}
\newtheorem{prp}{Proposition}
\newtheorem{example}{Example}
\newtheorem{dfn}{Defintion}
\newtheorem{question}{Question}
\newtheorem{remark}{Remark}
\newtheorem{problem}{Задача}
\newtheorem{claim}{Claim}
\newcommand{\A}{\mathfrak{A}}
\newcommand{\B}{\mathfrak{B}}
\renewcommand{\C}{\mathfrak{C}}
\newcommand{\D}{\mathfrak{D}}

\newcommand{\Ls}{\mathscr{L}}
\newcommand{\Fs}{\mathscr{F}}
\newcommand{\Rs}{\mathscr{R}}
\newcommand{\As}{\mathscr{A}}
\newcommand{\Bs}{\mathscr{B}}
\newcommand{\Is}{\mathscr{I}}
\newcommand{\Ss}{\mathscr{S}}
\newcommand{\Ps}{\mathscr{P}}

\newcommand{\xn}{x_{1},\dots,x_{n}}

\newcommand{\xs}{\overline{x}}
\newcommand{\ys}{\overline{y}}
\newcommand{\zs}{\overline{z}}
\newcommand{\forces}{\Vdash}
\renewcommand{\iff}{\leftrightarrow}
\begin{document}

\author{Stefan Vatev}


\begin{problem}
  Нека $(a_1,a_2,\dots,a_{12})$ е пермутация на числата от 1 до 12, за които е изпълнено условието:
  \[a_1 > a_2 > a_3 > a_4 > a_5 > a_6 < a_7 < a_8 < a_9 < a_{10} < a_{11} < a_{12}.\]
  Намерете броя на тези пермутации.  
\end{problem}

\begin{problem}
  Една функция $f:\{1,\dots,n\}\to\{1,\dots,m\}$ е монотонно растяща, ако
  $(\forall i\forall j)[1\leq i<j\leq n \rightarrow f(i)\leq f(j) ]$.
  \begin{enumerate}
  \item
    Колко такива функции съществуват?
  \item
    Колко от тези функции са сюрективни?
  \item
    Колко от тези функции са инективни?
\end{enumerate}
\end{problem}


\section{Принцип на Включване и Изключване}

\begin{problem}
  Дадени са $n$ кутии и $m$ неразличими топки.
  По колко начина могат да се разпределят всички топки в кутиите, така че нито в една кутия да няма повече от $r$ топки?
\end{problem}

\begin{problem}
  Колко решения в естествените числа имат уравненията:
  \begin{enumerate}
  \item
    $x_1+x_2+x_3 = 11$;
  \item
    $x_1 + x_2 + x_3 = 11, x_2 \geq 3$;
  \item
    $x_1+x_2+x_3 = 11, x_2 \leq 3$;
  \item
    $x_1+x_2+x_3 = 11, x_1 \geq 2, x_2 \geq 3$;
  \item
    $x_1+x_2+x_3 = 11, x_1 \geq 2, x_2 \geq 3, x_3 \leq 8$;
  \end{enumerate}
\end{problem}

\begin{problem}
  $x_1 + x_2 + x_3 = n$, където
  $0\leq x_i \leq k$, за всяко $1\leq i \leq 3$ и $n < 3k$.
\end{problem}

\begin{problem} % Гаврилов стр. 265, зад. 7
  Нека $U$ е множество от $n (n\geq 3)$ елемента. За всяко множество $X\subseteq U$, с $\overline{X}$ означаваме $U\setminus X$.
  \begin{enumerate}[a)]
  \item
    Намерете броя на двойките $(X,Y)$ за $X,Y\subseteq U$;
  \item
    Намерете броя на двойките $(X,Y)$ за $X,Y\subseteq U$, за които $\vert{X}\vert = 1$;
  \item
    Намерете броя на двойките $(X,Y)$ за $X,Y\subseteq U$, за които $\vert{X}\vert = 2$;
  \item
    Намерете броя на двойките $(X,Y)$ за $X,Y\subseteq U$, за които $\vert{X}\vert = k$ и $k < n$;
  \item
    Намерете броя на двойките $(X,Y)$ за $X,Y\subseteq U$, за които $\vert{(X\setminus{Y})\cup (X\setminus\overline{Y})}\vert = k$ и $k < n$;
  \item
    Намерете броя на двойките $(X,Y)$ за $X,Y\subseteq U$, за които $\vert{X\setminus Y}\vert = 1$;
  \item
    Намерете броя на двойките $(X,Y)$ за $X,Y\subseteq U$, за които $\vert{X\setminus Y}\vert = k$ и $k < n$;
  \item
    Намерете броя на двойките $(X,Y)$ за $X,Y\subseteq U$, за които $X\cap Y = \emptyset$;
  \item
    Намерете броя на двойките $(X,Y)$ за $X,Y\subseteq U$, за които $\vert{X\cap Y}\vert = k$ и $k < n$;
  \item
    Намерете броя на двойките $(X,Y), X,Y\subseteq U$, за които $|(X\setminus Y)\cup(Y\setminus X)| = 1$;
  \item
    Намерете броя на двойките $(X,Y), X,Y\subseteq U$, за които $|(X\setminus Y)\cup(Y\setminus X)| = k$ и $k < n$;
  \item
    Намерете броя на тройките $(X,Y,Z), X,Y,Z\subseteq U$, за които $X\cup Y\overline{Z} = \overline{X}\cup\overline{Y}$;
  \item
    Намерете броя на тройките $(X,Y,Z), X,Y,Z\subseteq U$, за които $Y\cup X = Z\cup\overline{Y}$;
  \item
    Намерете броя на двойките $(X,Y), X,Y\subseteq U$, за които $X\cap Y = \emptyset$ и
    $|X|\geq 1$, $|Y|\geq 1$;
  \item
    Намерете броя на двойките $(X,Y), X,Y\subseteq U$, за които $X\cap Y = \emptyset$ и 
    $|X|\geq 2, |Y|\geq 2$;
  \item
    Намерете броя на двойките $(X,Y), X,Y\subseteq U$, за които $|(X\setminus Y)\cup(Y\setminus X)| = 1$ и 
    $|X|\geq 2, |Y|\geq 2$;
  \item
    Намерете броя на тройките $(X,Y,Z), X,Y,Z\subseteq U$, за които $X\cup Y\overline{Z} = \overline{X}\cup\overline{Y}$ и
    $|Z| = 0$.
  \item
    Намерете броя на тройките $(X,Y,Z), X,Y,Z\subseteq U$, за които $X\cup Y\overline{Z} = \overline{X}\cup\overline{Y}$ и
    $|X|\geq 1, |Y|\geq 1, |Z| = 1$.
  \item
    Намерете броя на тройките $(X,Y,Z), X,Y,Z\subseteq U$, за които $X\cup Y\overline{Z} = \overline{X}\cup\overline{Y}$ и
    $|X|\geq 1, |Y|\geq 1, |Z|\leq 1$.
  \end{enumerate}
\end{problem}

\begin{problem}
  Нека са дадени естествените числа $m$ и $n$, $m\geq n$.
  Намерете броя на тоталните {\em сюрективни} функции $f:\{1,2,\dots,m\}\to\{1,2,\dots,n\}$.
\end{problem}

\end{document}


%%% Local Variables: 
%%% mode: latex
%%% TeX-master: t
%%% End: 

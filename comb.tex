\chapter{Комбинаторика}

\subsection*{Основни понятия}

По-долу навсякъде $k \leq n$.
\begin{enumerate}
\item[(0+R+)]
  Конфигурации с подредба и с повторение.
  Също така се наричат пермутации с повторение.
  Това е броят $P_r(n,k)$ на всички думи с дължина $k$ над $n$-елементна азбука.
  \[n^k\]
  Например, всички 3-буквени думи над азбуката $\{a,b,c,d,e\}$ са $5^3$ на брой.
\item[(0+R--)]
  Конфигурации с подредба, но без повторение. 
  Съща така се наричат пермутации.
  Това е броят $P(n,k)$ на $k$-елементните {\em подредени} множества на едно $n$-елементно множество.
  \[P(n,k) = \frac{n!}{(n-k)!} = n(n-1)\cdots(n-k+1).\]
  Например, всички 3-буквени думи {\em без повторения} над азбуката $\{a,b,c,d,e\}$
  са $5!4!3!$ на брой.
\item[(0--R--)]
  Конфигурации без подредба и без повторение.
  Също така се наричат комбинации.
  Това е броят $C(n,k)$ на $k$-елементните подмножества (т.е. елементите {\em не са подредени}) на едно $n$-елементно множество.
  Имаме следната връзка с пермутации без повторение:
  \[P(n,k) = C(n,k)\cdot P(k,k),\] следователно,
  \[C(n,k) =  \frac{P(n,k)}{k!} = \frac{n!}{(n-k)!k!} = \binom{n}{k}.\]
  Например, броят на всички комбинации от правилно попълнени фишове в тото 6 от 49 са $\binom{49}{6}$.
\item[(0-- R+)]
  Мултимножество е съвкупност от обекти, в които позволяваме повторение на елементи.
  Например, $\{3,1,1,2,3\}$ е мултимножество.
  Конфигурации без подредба, но с повторение.
  Наричат се още комбинации с повторение.
  Това е броят на $n$-елементните мулти-подмножества на едно $k$-елементно множество
  и той е:
  \[C(n+k-1,k-1) = \binom{n+k-1}{k-1}.\]
\end{enumerate}

\begin{prb}
  Докажете, че:
  \begin{enumerate}[a)]
  \item
    $(x+y)^n = \sum^{n}_{i=0}\binom{n}{i}x^iy^{n-i}$;
  \item
    $2^n = \sum^n_{k=0}\binom{n}{k}$;
  \item
    $3^n = \sum^n_{k=0}2^n\binom{n}{k}$;
  \item
    $\binom{n}{k} = \binom{n}{n-k}$;
  \item
    $\binom{n+1}{k} = \binom{n}{k} + \binom{n}{k-1}$
  \item 
    $\binom{n+m}{r} = \sum^r_{k=0}\binom{n}{r-k}\binom{m}{k}$;
  \item
    $\binom{2n}{r} = \sum^r_{k=0}\binom{n}{k}^2$;
  \item
    $\binom{n+1}{r+1} = \sum^n_{j=r}\binom{j}{r}$;
  \item
    $\binom{2n}{2} = 2\binom{n}{2} + n^2$;
  \item
    $\binom{n+r+1}{r} = \sum^r_{k=0}\binom{n+k}{k}$;
  \item
    $n2^{n-1} = \sum^{n}_{k=1} k\binom{n}{k}$;
  \item
    $n\binom{2n-1}{n-1} = \sum^{n}_{k=1}k\binom{n}{k}^2$;
  \end{enumerate}
\end{prb}



\begin{problem}
  \begin{enumerate}[a)]
  \item
    \marginpar{Отг. $2^8$}
    Колко битови низове с дължина един байт има ?
  \item
    Колко са всички подмножества на множеството $A$ с $8$ елемента ?
  \item 
    \marginpar{Отг. $2^5$}
    Колко битови низове с дължина един байт започват с 1 завършват с 00 ?
  \item
    \marginpar{Отг. $62! - 52!$}
    Всеки потребител на една компютърна система има парола, която е дълга между 6 и 8 символа.
    Всеки символ е малка или голяма буква, или цифра.
    Всяка парола трябва да съдържа поне една цифра.
    Колко такива пароли има?
  \item
    %\marginpar{Отг. $4!$}
    По колко начина можем да подредим елементите $\{a,b,c,d\}$ ?
  \item 
    %\marginpar{Отг. $5!$}
    Колко думи може да се образуват от буквите в $ABCDEFG$, които съдържат $ABC$.
  \item
    %\marginpar{Отг. $\binom{11}{1}\binom{10}{4}\binom{6}{4}\binom{2}{2}$}
    Колко ралични думи могат да се образуват като разместим буквите на думата $MISSISSIPPI$?
  \item
    Колко ралични думи могат да се образуват като разместим буквите на думата $TENNESSEE$?
  \item
    Колко ралични думи могат да се образуват като разместим буквите на думата $SUCCESS$?
  \item
    Колко ралични думи могат да се образуват като разместим буквите на думата АБРАКАДАБРА?
  \item
    Колко ралични думи могат да се образуват като разместим буквите на думата ПЕРПЕРИКОН?
  \item
    %\marginpar{Отг. $10\cdot 9\cdot 8$}
    В състезание участват 10 отбора. 
    По колко начина могат да се разпределят златните, сребърните и бронзовите медали?
  \item
    %\marginpar{Не искаме числата да започват с нула. Отг. $5! - 4!$}
    Колко различни петцифрени числа могат да се образуват чрез разместване на цифрите от 0,1,2,3,4?
  \item
    %\marginpar{Отг. $\binom{8}{1}\binom{7}{3}\binom{4}{4}$}
    По колко различни начина могат да се настанят осем студента в три стаи съответно с едно, три и четири легла?
  \item
    %\marginpar{Отг. $\binom{n}{1}\binom{n-1}{1}\binom{n-2}{1}\binom{n-3}{1}$}
    По колко различни начина четирима младежи могат да поканят на танц четири от $n$ девойки?
  \item
    %\marginpar{Отг. $\binom{6}{2}\binom{4}{2}\binom{2}{2}$}
    Шест различни предмета се боядисват по следния начин: два зелен, два червен, два син цвят.
    По колко различни начина могат да се боядисат предметите?  
  \item
    По колко различни начина могат да се разпределят 10 специалисти в 4 цеха така, че в тях да попаднат съответно по 1,2,3 и 4 души?
  \item
    %\marginpar{Отг. $n!$}
    Нека $A$ е множество с $n$ елемента.
    Колко биекции има от вида $f:A\to A$?
  \item
    \marginpar{Отг. $(n+1)! - n! - n!$}
    Иванчо и $n$ негови приятели отиват на кино.
    По колко различни начина могат всички да седнат заедно на един ред, така че Иванчо е винаги
    между двама негови приятели.
  \item
    \marginpar{Отг. $\binom{m}{k}\binom{N-M}{n-k}$}
    В партида от $N$ изделия, $M$ са бракувани.
    По колко различни начина могат да се вземат от партидата $n$ изделия, така че точно $k$ от тях да бъдат бракувани ($M\leq N, k\leq n\leq N$)?
  \item
    \marginpar{Отг. $\binom{4}{2}\binom{48}{4}$}
    От колода с 52 карти се изваждат 6 произволни карти без връщане.
    По колко различни начина могат да се извадят картите, така че две от тях да са дами?
  \item
    %\marginpar{Отг. $\binom{4}{2}\binom{4}{2}\binom{44}{2}$}
    От колода с 52 карти се изваждат 6 произволни карти без връщане.
    По колко различни начина могат да се извадят картите, така че две от тях да са тройки и две осмици?
  \item
    %\marginpar{Отг. $\binom{48}{24}\binom{4}{2}$}
    По колко различни начина може да се раздели колода от 52 карти на две пачки от по 26 карти така, че във всяка от тях да има по две дами?
  \item
    %\marginpar{Отг. $\binom{8}{2}\binom{6}{2}\binom{4}{2}\binom{2}{2}$}
    По колко начина може да се разпределят 8 подаръка между 4 лица, така че всеки да получи по два подаръка?
  \item
    %\marginpar{Отг. $\binom{40}{1}\binom{39}{1}\binom{38}{5}$}
    Провежда се събрание с $40$ присъстващи.
    По колко начина може да се избере председател, секретар и 5 членна комисия?
  \end{enumerate}
\end{problem}


\begin{problem}
  От колода с $52$ карти се избират $11$. По колко различни начина могат да се изберат извадки, в които се срещат:
  \begin{enumerate}[a)]
  \item
    \marginpar{Отг. $\binom{48}{10}\binom{4}{1}$}
    точно $1$ ас;
  \item
    %\marginpar{$\binom{52}{11} - \binom{48}{11} - \binom{48}{10}\binom{4}{1}$}
    поне $2$ валета;
  \item
    %\marginpar{$\binom{39}{7}\binom{13}{4}$}
    точно $4$ пики;
  \item
    %\marginpar{$\binom{52}{11} + \binom{39}{10}\binom{13}{1} + \binom{39}{9}\binom{13}{2}$}
    най-много $2$ кари;
  \item
    %\marginpar{Отг. $\binom{3}{2}\binom{12}{2}\binom{36}{7} + \binom{3}{1}\binom{12}{1}\binom{36}{8}$}
    точно $2$ аса и $2$ точно трефи;
  \item
    точно $2$ аса и не повече от $2$ трефи;
\end{enumerate}
\end{problem}


\begin{problem}
  Да се намерят всички $k$-буквени думи от азбука с $n$ букви, $k\leq n$, които:
  \begin{enumerate}[a)]
  \item
    %\marginpar{Отг. $\frac{n!}{(n-k)!}$}
    нито една буква не се повтаря;
  \item
    \marginpar{Отг. $n^{\lceil{\frac{k}{2}}\rceil}$}
    са симетрични;
  \item
    \marginpar{Отг. $n(n-1)^{k-1}$}
    нямат две последователни еднакви букви;
  \item
    % \marginpar{Отг. $n^k - n(n-1)^{k-1}$}
    имат две последователни еднакви букви;
  \item
    %\marginpar{Отг. $\binom{k}{2}n\cdot\frac{(n-1)!}{(n-1-(k-2))!}$}
    съществува едиствена буква, която се повтаря;
  \item
    %\marginpar{Отг. $n^k - \frac{n!}{(n-k)!}$}
    съществува буква, която се повтаря;
    % \item
    %   %\marginpar{Отг. $\sum^{k}_{i=2}\frac{}{}$}
    %   точно една буква се среща поне два пъти;
    % \item
    %   %\marginpar{}
    %   поне две букви се срещат поне два пъти;
    % \item
    %   съществува единствена буква, която се среща точно два пъти;
  \end{enumerate}
\end{problem}

\newpage
\subsection*{Принцип за включването и изключването}

\begin{prop}
  За две крайни множества $A$ и $B$,
  \begin{enumerate}[a)]
  \item 
    $A \cap B = \emptyset\ \rightarrow\ \abs{A\cup B} = \abs{A} + \abs{B}$.
  \item
    $A\subseteq B\ \rightarrow\ \abs{B\setminus A} = \abs{B} - \abs{A}$.
  \item
    $\abs{A\cup B} = \abs{A} + \abs{B} - \abs{A\cap B}$.
  \end{enumerate}
\end{prop}
\begin{proof}
  \begin{enumerate}[a)]
  \item
    Индукция по броя на елементите на $B$.
    \begin{itemize}
    \item
      $\abs{B} = 0$, то $B = \emptyset$ и тогава за произволно множество $А$,
      \[\abs{A\cup B} = \abs{A} + \abs{B}.\]
    \item
      $\abs{B} = 1$, то $B = \{b\}$ и тогава за произволно крайно множество $A$, за което $b \not\in A$,
      е очевидно, че \[\abs{A\cup\{b\}} = \abs{A} + 1.\]
    \item
      $\abs{B} = n+1$, то $B = B^\prime \cup \{b\}$, $\abs{B^\prime} = n$ и 
      нека $A$ е произволно крайно множество, за което $A \cap B = \emptyset$.
      \begin{align*}
        \abs{A \cup B} = \abs{(A \cup B^\prime) \cup \{b\}} = \abs{A\cup B^\prime} + 1 = \abs{A} + \abs{B^\prime} + 1 = \abs{A} + \abs{B}.
      \end{align*}
    \end{itemize}
  \item
    Ако $A \subseteq B$, то $B\setminus A\ \cup\ A = B$ и $B\setminus A\ \cap\ A = \emptyset$. Тогава
    \begin{align*}
      \abs{B} = \abs{B\setminus A\ \cup\ A} = \abs{B\setminus A} + \abs{A}.
    \end{align*}
    Тогава 
    \[\abs{B\setminus A} = \abs{B} - \abs{A}.\]
  \item
    Имаме, че:
    \begin{align*}
      A\cup B & = A \setminus B\ \cup\ (A\cap B)\ \cup\ B\setminus A\\
      & = A\setminus (A\cap B)\ \cup\ (A\cap B)\ \cup\ B\setminus (A\cap B)
    \end{align*}
    Трите множества в дясната страна на равенството са непресичащи се.
    Тогава 
    \begin{align*}
      \abs{A \cup B} & = \abs{ A\setminus (A\cap B)} + \abs{A\cap B} + \abs{B\setminus (A\cap B)}\\
      & = \abs{A} - \abs{A\cap B} + \abs{A \cap B} + \abs{B} - \abs{A\cap B}\\
      & = \abs{A} + \abs{B} + \abs{A \cap B}
    \end{align*}
  \end{enumerate}
\end{proof}

\begin{prop}
  Докажете, че за всеки три крайни множества $A$, $B$ и $C$,
  \[\abs{A\cup B \cup C} = \abs{A}+\abs{B}+\abs{C} - \abs{A\cap B} - \abs{B\cap C} - \abs{A\cap C}+ \abs{A\cap B \cap C}.\]
\end{prop}
\begin{proof}
  \begin{align*}
    \abs{(A \cup B) \cup C} & = \abs{A \cup B} + \abs{C} - \abs{(A\cup B)\cap C}\\
    & = \abs{A} + \abs{B} - \abs{A\cap B} + \abs{C} - \abs{(A\cap C)\cup(B\cap C)}\\
    & = \abs{A} + \abs{B} + \abs{C} - \abs{A\cap B} - (\abs{A\cap C} + \abs{B\cap C} - \abs{(A\cap C) \cap (B\cap C)})\\
    & = \abs{A} + \abs{B} + \abs{C} - \abs{A\cap B} - \abs{A\cap C} - \abs{B\cap C} + \abs{A\cap B \cap C}
  \end{align*}
\end{proof}


\begin{thm}
  Нека $A_1\dots A_n$ са $n$ на брой крайни множества и $n\geq 2$.
  Тогава 
  \[|A_1\cup A_2\cup \dots \cup A_n| = \sum_{1\leq i\leq n} |A_i| - \sum_{1\leq i_1 < i_2\leq n} |A_{i_1}\cap A_{i_2}| + \sum_{1\leq i_1 < i_2 < i_3\leq n} |A_{i_1}\cap A_{i_2}\cap
  A_{i_3}| -
  \]\[\dots + (-1)^{n-1}|A_1 \cap A_2\dots \cap A_n|. \]
\end{thm}


\begin{problem}
  Колко решения в естествените числа имат уравненията:
  \begin{enumerate}[a)]
  \item
    \marginpar{Отг. $\binom{13}{2}$}
    $x_1+x_2+x_3 = 11$;
  \item
    %\marginpar{Отг. $\binom{12}{1} + \binom{11}{1} + \binom{10}{1}$}
    $x_1 + x_2 + x_3 = 11$, $x_2 < 3$;
  \item
    %\marginpar{Отг. $\binom{13}{2} - \binom{12}{1} - \binom{11}{1} - \binom{10}{1}$}
    $x_1 + x_2 + x_3 = 11$, $x_2 \geq 3$;
  \item
    $x_1+x_2+x_3 = 11$, $x_1 \geq 2$, $x_2 \geq 3$;
  \item
    $x_1+x_2+x_3 = 11, x_1 \geq 2, x_2 \geq 3, x_3 \leq 8$;
  \end{enumerate}
\end{problem}
\begin{proof}
  \begin{enumerate}[a)]
  \item[г)]
    Нека 
    $A_1$ да бъдат решенията на уравнението, за които $x_1 < 2$,
    $A_2$ да бъдат решенията на уравнението, за които $x_2 < 3$.
    Лесно се вижда, че
    \begin{align*}
      \abs{A_1} & = \binom{12}{1} + \binom{11}{1} = 12 + 11 = 23\\
      \abs{A_2} & = \binom{12}{1} + \binom{11}{1} + \binom{10}{1} = 12 + 11 + 10 = 33\\
      \abs{A_1\cap A_2} & = 6\\
      \abs{A_1 \cup A_2} & = 23 + 33 - 6 = 50.
    \end{align*}
    Следователно,
    отговорът е 
    \[\binom{13}{2} - \abs{A_1 \cup A_2} = 78 - 50.\]
  \item[д)]
    Нека
    $A_3$ да бъдат решенията на уравнението, за които $x_3 > 8$.
    \begin{align*}
      \abs{A_3} & = \binom{3}{1} + \binom{2}{1} + \binom{1}{1} = 3 + 2 + 1 = 6\\
      \abs{A_1\cap A_3} & = 6\\
      \abs{A_1\cap A_3} & = 5\\
      \abs{A_2\cap A_3} & = 6\\
      \abs{A_1\cap A_2\cap A_3} & = 5
    \end{align*}
    Следователно, броят на решенията на уравнението са:
    \begin{align*}
      \abs{A} - \abs{A_1\cup A_2 \cup A_3} & = \abs{A} - (\abs{A_1} + \abs{A_2} + \abs{A_3} - \sum_{i\neq j}\abs{A_i\cap A_j} + \abs{A_1\cap A_2\cap A_3})
    \end{align*}
  \end{enumerate}

\end{proof}


\begin{problem} % Гаврилов стр. 265, зад. 7
  Нека $U$ е множество от $n (n\geq 3)$ елемента. За всяко множество $X\subseteq U$, с $\overline{X}$ означаваме $U\setminus X$.
  Намерете броя на:
  \begin{enumerate}[a)]
  \item
    \marginpar{Отг. $4^n$}
    двойките $(X,Y)$ за $X,Y\subseteq U$;
  \item
    \marginpar{Отг. $2\binom{n}{1} 2^{n-1}$}
    двойките $(X,Y)$ за $X,Y\subseteq U$, за които $\vert{X}\vert = 1$;
  \item
    \marginpar{Отг. $2^2\binom{n}{2}2^{n-2}$}
    двойките $(X,Y)$ за $X,Y\subseteq U$, за които $\vert{X}\vert = 2$;
  \item
    %\marginpar{Отг. $4^n - 3^{n}$}
    двойките $(X,Y)$ за $X,Y\subseteq U$, за които $\vert{X}\vert \geq 1$;
  \item
    двойките $(X,Y)$ за $X,Y\subseteq U$, за които $\vert{X}\vert = k$ и $k < n$;
  \item
    %\marginpar{Отг. $3^{n} + \binom{n}{1} 3^{n-1}$}
    двойките $(X,Y)$ за $X,Y\subseteq U$, за които $\vert{X}\vert \leq 1$;
  \item
    \marginpar{Отг. $\binom{n}{1}\binom{n}{1} = n^2$}
    двойките $(X,Y)$ за $X,Y\subseteq U$, за които $\vert{X}\vert = 1$ и $\vert{Y}\vert = 1$;
  \item
    \marginpar{Отг. $3^n$}
    двойките $(X,Y)$ за $X,Y\subseteq U$, за които $X \cap Y = \emptyset$;
  \item
    \marginpar{Отг. $\binom{n}{1}3^{n-1}$}
    двойките $(X,Y)$ за $X,Y\subseteq U$, за които $\abs{X \cap Y} = 1$;
  \item
    двойките $(X,Y)$ за $X,Y\subseteq U$, за които $\abs{X \cap Y} = k$ и $k < n$;
  \item
    %\marginpar{Отг. $2\binom{n}{1}2^{n-1} = n2^n$}
    двойките $(X,Y)$ за $X,Y\subseteq U$, за които $\abs{(X\setminus Y) \cup (Y\setminus X)} = 1$;
  \item
    двойките $(X,Y), X,Y\subseteq U$, за които $X\cap Y = \emptyset$ и
    $|X|\geq 1$, $|Y|\geq 1$;
  \item
    двойките $(X,Y), X,Y\subseteq U$, за които $X\cap Y = \emptyset$ и 
    $|X|\geq 2, |Y|\geq 2$;
  \item
    двойките $(X,Y)$ за $X,Y\subseteq U$, за които $\vert{X\setminus Y}\vert = 1$;
  \item
    двойките $(X,Y)$ за $X,Y\subseteq U$, за които $\vert{X\setminus Y}\vert = k$ и $k < n$;
  \item
    двойките $(X,Y), X,Y\subseteq U$, за които $|(X\setminus Y)\cup(Y\setminus X)| = 1$ и 
    $|X|\geq 2, |Y|\geq 2$;
  \item
    %\marginpar{Използвайте принципа за вкл. и изкл.}
    двойките $(X,Y), X,Y\subseteq U$, за които $X\cap Y = \emptyset$, $|X|\geq 2$ и $|Y|\geq 3$;
  \item
    % \marginpar{Използвайте принципа за вкл. и изкл.}
    двойките $(X,Y), X,Y\subseteq U$, за които $|(X\setminus Y)\cup(Y\setminus X)| = 1$, $X\cap Y = \emptyset$, $|X|\geq 2$ и $|Y|\geq 3$;
  \item
    \marginpar{Отг. $8^n$}
    тройките $(X,Y,Z)$ за $X,Y,Z\subseteq U$;
  \item
    \marginpar{Отг. $6^n$}
    тройките $(X,Y,Z)$ за $X,Y,Z\subseteq U$, за които $X \cap Y = \emptyset$;
  \item
    % \marginpar{$U = X\overline{Y}Z\cup X\overline{YZ} \cup \overline{X}Y\overline{Z}$. Отг. $3^n$}
    тройките $(X,Y,Z), X,Y,Z\subseteq U$, за които $X\cup Y\overline{Z} = \overline{X}\cup\overline{Y}$;
  \item
    тройките $(X,Y,Z), X,Y,Z\subseteq U$, за които $Y\cup X = Z\cup\overline{Y}$;
  \item
    тройките $(X,Y,Z), X,Y,Z\subseteq U$, за които $X\cup Y\overline{Z} = \overline{X}\cup\overline{Y}$ и
    $|Z| = 0$.
  \item
    тройките $(X,Y,Z), X,Y,Z\subseteq U$, за които $X\cup Y\overline{Z} = \overline{X}\cup\overline{Y}$ и
    $|X|\geq 1, |Y|\geq 1, |Z| = 1$.
  \item
    тройките $(X,Y,Z), X,Y,Z\subseteq U$, за които $X\cup Y\overline{Z} = \overline{X}\cup\overline{Y}$ и
    $|X|\geq 1, |Y|\geq 1, |Z|\leq 1$.
  \end{enumerate}
\end{problem}



\begin{problem}
  \begin{enumerate}[a)]
  \item
    %\marginpar{Отг. $\binom{n+4-1}{4-1}$}
    По колко начина могат да се изберат $n$ монети да се изберат от купчина монети с номинал 5, 10, 20 и 50 стотинки?
  % \item
  %   %\marginpar{Отг. }
  %   По колко начина могат да се изтеглят $13$ от $52$ карти, ако ги различаваме само по цвета?
  \item
    %\marginpar{Отг. $\binom{n+m-1}{m-1}$}
    Намерете броя на възможните начини за разпределение на $n$ {\bf неразличими} топки в $m$ различни кутии.
  \item
    %\marginpar{Отг. $\binom{(n-m)+m-1}{m-1}$}
    Намерете броя на възможните начини за разпределение на $n$ {\bf неразличими} топки в $m$ различни кутии, 
    ако няма празна кутия.
  \item
    %\marginpar{Отг. $\binom{n+m-1}{m-1} - \binom{(n-m)+m-1}{m-1}$}
    Намерете броя на възможните начини за разпределение на $n$ {\bf неразличими} топки в $m$ различни кутии,
    ако съществува поне една празна кутия.
  \item
    %\marginpar{$\binom{n+m-1}{m-1}/m!$ ???}
    Да се намери броя на възможните начини за разпределения на $n$ {\bf неразличими} топки в $m$ {\bf неразличими} кутии.
  % \item
  %   %\marginpar{}
  %   Да се намери броя на възможните начини за разпределения на $n$ {\bf различими} топки в $m$ различни кутии.
  \end{enumerate}
\end{problem}

\begin{problem}
  Множеството от всички двоични вектори от $\{0,1\}^{n}$, които във фиксирани $n-k$ позиции имат равни значения,
  ги наричаме $k$-равнини, за $k\leq n$.
  \begin{enumerate}[a)]
  \item
    %\marginpar{Отг. $2^{k}$}
    Колко различни вектора има в една $k$-равнина?
  \item
    %\marginpar{Отг. $2^{n-k}\binom{n}{n-k}$}
    Колко различни $k$-равнини има в $\{0,1\}^{n}$?
  \item
    %\marginpar{Отг. $\binom{n}{n-k}$}
    Колко различни $k$-равнини съдържат даден фиксиран $n$-мерен вектор?
  % \item
  %   %\marginpar{Отг. $$}
  %   Колко различни $k$-равнини съдържат дадена $l$-равнина, $0\leq l < k$.
  \end{enumerate}
\end{problem}


\begin{problem}
  Да фиксираме естествените числа $m$ и $n$.
  Една функция \[f:\{1,\dots,n\}\to\{1,\dots,m\}\] е монотонно ненамаляваща, ако
  \[(\forall i\forall j)[1\leq i<j\leq n \rightarrow f(i)\leq f(j)].\]
  \begin{enumerate}[a)]
  \item
    %\marginpar{Отг. $\binom{n+m-1}{m-1}$}
    Колко такива функции съществуват?
  \item
    %\marginpar{Всяка кутийка има топка. Отг. $\binom{(n-m)+m-1}{m-1}$}
    Колко от тези функции са сюрективни при $n\geq m$?
  \item
    %\marginpar{Отг. $\binom{m}{n}$}
    Колко от тези функции са инективни при $n\leq m$?
\end{enumerate}
\end{problem}


\begin{problem}
  % \marginpar{$\binom{11}{5}$}
  Нека $(a_1,a_2,\dots,a_{12})$ е пермутация на числата от 1 до 12, за които е изпълнено условието:
  \[a_1 > a_2 > a_3 > a_4 > a_5 > a_6 < a_7 < a_8 < a_9 < a_{10} < a_{11} < a_{12}.\]
  Намерете броя на тези пермутации.  
\end{problem}



% \begin{problem}
%   \begin{enumerate}[a)]
%   \item
%     По колко начина могат $n$ момчета и $n$ момичета да седнат на ред с $2n$ стола, като няма двама от един пол седящи един до друг?
%   \item
%     По колко начина могат $n$ момчета и $n$ момичета да седнат на ред с $2n$ стола, като няма двама от един пол седящи един до друг и Иванчо и Марийка не седят един до друг? 
%   \item
%     По колко различни начина могат да се подредят на рафт $n$ книги, така че две от тях, определени предварително, да са една до друга?
%   \item
%     Колко различни гердана могат да се направят от $n$ различни перли, като се използват всичките?
%   \item
%     На хоро в кръг са хванали общо $n$ души, между които и Иванчо и Марийка.
%     Колко са възможните подредби, при които Иванчо и Марийка са един до друг?
%  \item
%     На хоро в кръг са хванали общо $n$ души, между които и Иванчо и Марийка.
%     Колко са възможните подредби, при които Иванчо и Марийка не са един до друг?
%   \item
%     Две сядания на една кръгла маса не са различни, ако всеки от седящите има едни и същи съседи.
%     По колко различни начина могат да седнат около една кръгла маса:
%     \begin{enumerate}
%     \item
%       $n (\geq 2)$ човека;
%     \item
%       $n$ мъже и $n$ жени, като две лица от един и същ пол не седят един до друг.
%     \end{enumerate}
%   \end{enumerate}
% \end{problem}


% \begin{problem}
%   \begin{enumerate}[a)]
%   \item
%     В магазин продават $k$ вида ябълки.
%     Колко различни покупки на $n$ ябълки могат да се направят, без да се купуват повече от две ябълки от един и същ вид?
%   \item
%     В магазин продават $k$ вида ябълки.
%     Колко различни покупки на $n$ ябълки могат да се направят, като се купи поне по една от всеки вид и $n\geq k$?
%   \item
%     Имаме $n$ съпружески двойки, които седят на $2n$ места около една кръгла маса. 
%     По колко начина могат да седнат всички двойки, ако ротациите се броят за едно и също подреждане, и
%     всеки мъж седи до половинката си.
%   \end{enumerate}
% \end{problem}


%%% Local Variables: 
%%% mode: latex
%%% TeX-master: "discrete-math"
%%% End: 

\section{Съждително смятане}
\label{sect:propositional}
\marginpar{На англ. Propositional calculus}

Съждителното смятане наподобява аритметичното смятане.
Вместо артиметичните операции $+,-,\cdot,/$, имаме съждителни операции като $\neg, \wedge, \vee$.
Освен това, докато аритметичните променливи приемат стойности произволни числа, то
съждителните променливи приемат само стойности {\bf истина (1)} или {\bf лъжа (0)}.

{\bf Съждителен израз} начичаме съвкупността от съждения $p,q,r,\dots$, свързани със знаците за логически операции
$\neg, \vee, \wedge, \rightarrow, \leftrightarrow$ и скоби, определящи реда на операциите.

{\bf Съждително верен} (валиден) е този логически израз, който има верностна стойност {\bf 1} при всички възможни набори на
стойностите на съждителните променливи в израза.

\subsection*{Основни  съждителни операции: $\neg$, $\vee$, $\wedge$, $\rightarrow$,$\iff$}
Ще използваме таблица за истинност за да определим стойностите на основните съждителни операции
при всички възможни набори на стойностите на променливите.

\begin{tabular}{|c|c|c|c|c|c|c|c|}
  \hline
  $p$ & $q$ & $\neg p$ & $p \vee q$ & $p \wedge q$ & $p \rightarrow q$ & $\neg p \vee q$ & $p \iff q$\\
  \hline
  0 & 0 & 1 & 0 & 0 & 1 & 1 & 1 \\
  \hline
  0 & 1 & 1 & 1 & 0 & 1 & 1 & 0 \\
  \hline
  1 & 0 & 0 & 1 & 0 & 0 & 0 & 0 \\
  \hline
  1 & 1 & 0 & 1 & 1 & 1 & 1 & 1 \\
  \hline
\end{tabular}
% Булевите алгебри носят името на Джордж Бул (1815 - 1864), който първи описва техните свойства в своята книга
% ``The Mathematical Analysis of Logic'' (1847).


\subsection*{Основни съждителни закони (правила)}

\begin{enumerate}[I)]
  \item
    {\bf Комутативен закон}
    \[p\vee q \equiv q\vee p\] 
    \[p\ \wedge\ q \equiv q\ \wedge\ p\]
  \item
    {\bf Асоциативен закон}
    \[(p\vee q)\vee r \equiv p\vee(q\vee r)\]
    \[(p\ \wedge\ q)\ \wedge\ r \equiv p\ \wedge\ (q\ \wedge\ r)\]
  \item
    {\bf Дистрибутивен закон}
    \[p\ \wedge\ (q \vee r) \equiv (p\ \wedge q)\vee (p\ \wedge\ r)\]
    \[p\vee (q\ \wedge\ r) \equiv (p\vee q)\ \wedge\ (p\vee r)\]
  \item
    {\bf Закони на де Морган}
    \[\neg(p\ \wedge\ q) \equiv (\neg p \vee \neg q)\]
    \[\neg(p\vee q) \equiv (\neg p\ \wedge\ \neg q)\]
  \item
    {\bf Закон за контрапозицията}
    \[p\rightarrow q \equiv \neg q \rightarrow \neg p\]
  \item
    {\bf Обобщен закон за контрапозицията}
    \[(p\ \wedge\ q)\rightarrow r \equiv (p\ \wedge\ \neg r) \rightarrow \neg q\]
  \item
    {\bf Закон за изключеното трето}
    \[p\vee \neg p \equiv {\mathbf 1}\]
  \item
    {\bf Закон за силогизма (транзитивност)}
    \[[(p\rightarrow q)\ \wedge\ (q\rightarrow r)] \rightarrow (p\rightarrow r) \equiv {\mathbf 1}\]
\end{enumerate}

Лесно се проверява с таблиците за истинност, че законите са валидни.

\begin{example}
  Нека например да проверим едно от правилата на де Морган и закона
  за контрапозицията.
  
  \begin{tabular}{|c|c|c|c|c|c|c|c|c|}
    \hline
    $p$ & $q$ & $p\wedge q$ & $\neg(p\wedge q)$ & $\neg p$ & $\neg q$ & $\neg p \vee \neg q$ & $p \rightarrow q$ & $\neg q \rightarrow \neg p$\\
    \hline
    0 & 0 & 0 & 1 & 1 & 1 & 1 & 1 & 1 \\
    \hline
    0 & 1 & 0 & 1 & 1 & 0 & 1 & 1 & 1 \\
    \hline
    1 & 0 & 0 & 1 & 0 & 1 & 1 & 0 & 0 \\
    \hline
    1 & 1 & 1 & 0 & 0 & 0 & 0 & 1 & 1 \\
    \hline
  \end{tabular}
  
  При всички стойности на променливите $p$ и $q$, стълбовете съответсващи на $\neg(p \wedge q)$ и $\neg p \vee \neg q$
  съвпадат. Следователно, законът на де Морган е валиден.
  По същия начин се съобразява, че законът за констрапозицията е валиден.
\end{example}


\begin{example}
  Можем да докажем валидността на законите и по друг начин, а именно чрез допускане на противното.
  Така ще докажем, че законът за силогизма е валиден.
  
  Да допуснем, че съществува стойност на променливите $p$,$q$,$r$, за които
  \[\underbrace{[(p\rightarrow q)\ \wedge\ (q\rightarrow r)]}_{\mathbf{1}} \rightarrow \underbrace{(p\rightarrow r)}_{\mathbf{0}} \equiv {\mathbf 0}\]
  Това означава, че
  \[p \equiv \mathbf{1}, r \equiv \mathbf{0}.\]
  Тогава
  \[\underbrace{[(\mathbf{1}\rightarrow q)\ \wedge\ (q\rightarrow \mathbf{0})]}_{\mathbf{1}} \rightarrow \underbrace{(\mathbf{1}\rightarrow \mathbf{0})}_{\mathbf{0}} \equiv {\mathbf 0}\]
  \begin{itemize}
  \item 
    Ако $q \equiv \mathbf{0}$, то $(\mathbf{1}\rightarrow \mathbf{0})\ \wedge\ (\mathbf{0}\rightarrow \mathbf{0}) \equiv \mathbf{0} \wedge \mathbf{1} \equiv \mathbf{0}$,
    следователно този случай е невъзможен.
  \item
    Ако $q \equiv \mathbf{1}$, то $(\mathbf{1}\rightarrow \mathbf{1})\ \wedge\ (\mathbf{1}\rightarrow \mathbf{0}) \equiv \mathbf{1} \wedge \mathbf{0} \equiv \mathbf{0}$,
    следователно този случай също е невъзможен.
  \end{itemize}
  И в двата случая за $q$ достигнахме до противоречие.
  Следователно нашето допускане не е вярно, което означава, че
  при всяка стойност на променливите $p$, $q$, $r$,
  \[[(p\rightarrow q)\ \wedge\ (q\rightarrow r)] \rightarrow (p\rightarrow r) \equiv {\mathbf 1}.\]
\end{example}


% \begin{tabular}{|c|c|c|c|c|c|c|c|c|}
%   \hline
%   $p$ & $q$ & $r$ & $p \rightarrow q$& $q \rightarrow r$ & $p \rightarrow r$ & $(p \rightarrow q) \wedge (q \rightarrow r)$ & $[(p \rightarrow q) \wedge (q \rightarrow r)] \rightarrow (q \rightarrow r)$\\
%   \hline
%   $0$ & $0$ & $0$ & & & & & \\
%   $0$ & $0$ & $0$ & & & & & \\
%   $0$ & $0$ & $0$ & & & & & \\
%   $0$ & $0$ & $0$ & & & & & \\
%   $0$ & $0$ & $0$ & & & & & \\
%   $0$ & $0$ & $0$ & & & & & \\
%   $0$ & $0$ & $0$ & & & & & \\
%   $0$ & $0$ & $0$ & & & & & \\
%   \hline
% \end{tabular}

\begin{problem}
  Проверете дали следните съждителни формули са тавтологии:
  \begin{enumerate}[a)]
  \item
    $(p\wedge q)\rightarrow p$;
  \item
    $p\rightarrow(p\vee q)$;
  \item
    $(p\rightarrow q) \iff (\neg q \rightarrow \neg p)$;
  \item
    $p\rightarrow q \equiv \neg p \vee q$
  \item
    $(p\ \wedge\ q) \rightarrow r \equiv p \rightarrow (q\rightarrow r)$
  \item
    $p\ \wedge\ q \equiv \neg(\neg p \vee \neg q)$
  \item
    $p \leftrightarrow q \equiv (p\rightarrow q)\ \wedge\ (q\rightarrow p)$
  \item
    $\neg(p\wedge q) \equiv (\neg p \vee \neg q)$;
  \item
    $\neg(p\vee q) \equiv (\neg p \wedge \neg q)$;
  \item
    $\neg(p\rightarrow q) \equiv (p\wedge \neg q)$;
  \end{enumerate}
\end{problem}

\begin{remark}
  Обърнете внимание, че $(p\rightarrow q)\rightarrow r$ {\bf не} е еквивалентно на $p\rightarrow (q\rightarrow r)$.
  Например вземете $p \equiv q \equiv r \equiv \mathbf{0}$.
\end{remark}


% \begin{problem}
%   В един затвор имало трима арестанти - {\bf А}, {\bf Б} и {\bf В}, който дори бил и сляп.
%   Шерифът решил да пусне един от тях на свобода, затова завързал очите на {\bf А} и {\bf Б} и
%   им сложил по една шапка на главите. Шапките били общо пет, като три от тях били бели и две - черни.
%   След това шерифът им свалил превръзките на очите и им казал, че ще пусне този, който позне какъв
%   цвят е шапката му.
%   \begin{enumerate}[]
%   \item
%     {\bf А} казал, че не знае какъв цвят е шапката му и шерифът го върнал в ареста.
%   \item
%     {\bf Б} казал, че не знае какъв цвят е шапката му и шерифът го върнал в ареста.
%   \item
%     Слепият затворник {\bf В} отговорил правилно какъв цвят е шапката му и шерифът го пуснал.
%   \item
%     Какъв е цвета на шапката на {\bf В}?
%   \end{enumerate}
% \end{problem}
% \begin{proof}
%   Нека да означим с $а$ твърдението ``цветът на шапката на {\bf А} e черен'',
%   а съответно с $\ov{a}$ твърдението ``цветът на шапката на {\bf А} e бял'',
%   По аналогичен начин означаваме съждителните променливи $b$ и $c$ за арестантите {\bf Б} и {\bf В}.
%   Превеждаме твърденията в съждителни формули:
%   \begin{enumerate}[A)]
%   \item
%     Щом {\bf А} не знае какъв цвят е шапката му, то със сигурност шапките на главите на другите арестанти 
%     не са и двете черни. Получаваме
%     \[\ov{b}c\ \vee\ b\ov{c}\ \vee\ \ov{b}\ov{c}\]
%   \item
%     Щом и {\bf Б} не знае какъв цвят е шапката му, то шапката на {\bf В} не е черна, защото това ще означава, че
    
%   \end{enumerate}

  
%   Получаваме $(\ov{b}c\ \vee\ b\ov{c}\ \vee\ \ov{b}\ov{c})\ \wedge\ (a\ov{c}\ \vee\ \ov{a}\ov{c})$
% \end{proof}
\begin{remark}
  За удобство, понякога ще пишем $\ov{p}$ вместо $\neg p$ и $pq$ вместо $p \wedge q$.
\end{remark}


\begin{problem}
  Да предположим, че сме на остров, който се обитава негодници и благородници.
  Негодниците винаги лъжат, а благородниците винаги казват истината.
  Срещаме трима обитатели на този остров, наречени {\bf А}, {\bf Б} и {\bf В}.
  \begin{enumerate}[a)]
  \item
    \begin{enumerate}[]
    \item
      \marginpar{$a \iff \ov{a}\ov{b}\ov{c}$}
      {\bf А} казва ``Всички сме негодници''.
    \item
      \marginpar{$b \iff (\ov{a}\ov{b}c\vee \ov{a}b\ov{c}\vee a\ov{b}\ov{c})$}
      {\bf Б} казва ``Точно един от нас е благородник''.
    \item
      Какви са {\bf А},{\bf Б} и {\bf В}?
    \end{enumerate}
  \item
    \begin{enumerate}[]
    \item
      \marginpar{$a \iff \ov{a}\ov{b}\ov{c}$}
      {\bf А} казва ``Всички сме негодници''.
    \item
      \marginpar{$b \iff (\ov{a}bc\vee ab\ov{c}\vee a\ov{b}c)$}
      {\bf Б} казва ``Точно един от нас е негодник''.
    \item
      Може ли да определим какъв е {\bf Б}?
    \item
      Може ли да определим какъв е {\bf В}?
    \end{enumerate}
  \item
    \begin{enumerate}[]
    \item
      \marginpar{$a \iff \ov{b}$}
      {\bf А} казва ``{\bf Б} е негодник''.
    \item
      \marginpar{$b \iff (ac \vee \ov{a}\ov{c})$}
      {\bf Б} казва ``{\bf А} и {\bf В} са от един и същ тип, т.е. или и двамата са благородници, или и двамата са негодници''.
    \item
      Какъв е {\bf В}?
    \end{enumerate}
  \end{enumerate}
\end{problem}
\begin{proof}
  \begin{enumerate}[a)]
  \item
    Нека съждителната променлива $a$ да има стойност {\bf 1}, ако {\bf A} е благородник и нека има стойност {\bf 0}, 
    ако {\bf A} е негодник.
    Тогава
    \begin{enumerate}[A)]
    \item
      Ако {\bf А} е благородник, то {\bf А,Б,В} са негодници се превежда на езика на съждителното смятане като
      \[a \rightarrow \ov{a}\ov{b}\ov{c}.\]
      Ако {\bf А} е негодник, то той лъже, следователно не е вярно, че всички са негодници. Това се превежда на езика на съждителното смятане като
      \[\ov{a} \rightarrow \ov{\ov{a}\ov{b}\ov{c}},\] 
      което е еквивалентно на \[\ov{a}\ov{b}\ov{c}\rightarrow a.\]
      Следователно, в двата случая за {\bf A} получаваме
      \[(a \rightarrow \ov{a}\ov{b}\ov{c}) \wedge (\ov{a}\ov{b}\ov{c}\rightarrow a) \equiv \mathbf{1},\]
      или
      \[a \iff \ov{a}\ov{b}\ov{c} \equiv \mathbf{1}.\]
      Сега получаваме следните еквивалентни преобразования:
      \begin{align*}
        a \iff \ov{a}\ov{b}\ov{c} & \equiv\ (a \rightarrow \ov{a}\ov{b}\ov{c})\ \wedge\ (\ov{a}\rightarrow (a\vee b \vee c))\\
        & \equiv\ (\ov{a} \vee \ov{a}\ov{b}\ov{c}) \wedge (a \vee a\vee b \vee c)\\
        & \equiv\ \ov{a}\ \wedge\ (a \vee b \vee c)\\
        & \equiv\ \ov{a}a \vee \ov{a}b \vee \ov{a}c\\
        & \equiv\ \ov{a}b \vee \ov{a}c\\
        & \equiv\ \mathbf{1}.
      \end{align*}
    \item
      \begin{align*}
        b \iff (\ov{a}\ov{b}c\vee \ov{a}b\ov{c}\vee a\ov{b}\ov{c}) & \equiv\ (b \rightarrow (\ov{a}\ov{b}c\vee \ov{a}b\ov{c}\vee a\ov{b}\ov{c}))\wedge (\ov{b}\rightarrow \ov{\ov{a}\ov{b}c\vee \ov{a}b\ov{c}\vee a\ov{b}\ov{c}})\\
        & \equiv\ (\ov{b} \vee \ov{a}\ov{b}c\vee \ov{a}b\ov{c}\vee a\ov{b}\ov{c}) \wedge (b\vee \ov{a}\ov{b}\ov{c}\vee abc \vee \ov{a}bc \vee a\ov{b}c \vee ab\ov{c})\\
        & \equiv\ (\ov{b} \vee \ov{a}b\ov{c}) \wedge (b \vee \ov{a}\ov{b}\ov{c} \vee  a\ov{b}c)\\
        & \equiv\ \ov{a}\ov{b}\ov{c} \vee  a\ov{b}c \vee \ov{a}b\ov{c}.\\
        & \equiv\ \mathbf{1}. 
      \end{align*}
    \item
      Сега взимаме конюнкцията на А) и Б).
      \begin{align*}
        (\ov{a}b \vee \ov{a}c)\wedge (\ov{a}\ov{b}\ov{c} \vee  a\ov{b}c \vee \ov{a}b\ov{c}) &\ \equiv\ \ov{a}b\ov{c}.
      \end{align*}
    \end{enumerate}
  \end{enumerate}
\end{proof}

%%% Local Variables: 
%%% mode: latex
%%% TeX-master: "discrete-math"
%%% End: 

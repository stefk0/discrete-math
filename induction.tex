\chapter{Индукция}

\section{Математичска индукция}

\section{Пълна математическа индукция}

Нека имаме едно свойство $P$ върху естествените числа $\Nat$.
\begin{enumerate}
\item {\bf Матиматическа индукция} е следното правило:
  \begin{prooftree}
    \AxiomC{$P(0)$}
    \AxiomC{$(\forall n\in\Nat)[P(n)\rightarrow P(n+1)]$}
    \BinaryInfC{$(\forall n \in \Nat)[P(n)]$}
  \end{prooftree}
  
\item
  {\bf Пълна математическа индукция} е следното правило:
  \begin{prooftree}
    \AxiomC{$(\forall n\in\Nat)[(\forall m \in\Nat)[m < n \rightarrow P(m)]\rightarrow P(n)]$}
    \UnaryInfC{$(\forall n \in\Nat)[P(n)]$}
  \end{prooftree}
\end{enumerate}

\begin{prop}
  Двата принципа са еквивалентни.
\end{prop}
\begin{proof}
  $1)\to 2)$.
  Нека имаме $1)$ и нека $(\forall n\in\Nat)[(\forall m \in\Nat)[m < n \rightarrow P(m)]\rightarrow P(n)]$.
  Ще докажем, че $(\forall n \in \Nat)[P(n)]$.
  Да рагледаме свойството \[Q(n) = P(0)\ \wedge\ \dots\ P(n).\]
  Тогава имаме, че $Q(0)$ и $(\forall n\in\Nat)[Q(n) \rightarrow P(n+1)]$.
  Но от $Q(n) \rightarrow P(n+1)$ следва, че $Q(n) \rightarrow Q(n) \wedge P(n+1)$, т.е.
  $(\forall n\in\Nat)[Q(n)\rightarrow Q(n+1)]$.
  Така полуваме по математичска индукция, че $(\forall n\in\Nat)[Q(n)]$.
  Но тогава е очевидно, че имаме $(\forall n\in\Nat)[Q(n)]$,
  защото $Q(n) \rightarrow P(n)$.

  $2) \to 1)$.
  Нека имаме $2)$ и $P(0)\ \wedge\ (\forall n\in\Nat)[P(n)\rightarrow P(n+1)]$.
  Ще докажем, че $(\forall n \in \Nat)[P(n)]$.
  Понеже имаме 2), достатъчно е да докажем
  \[(\forall n\in\Nat)[(\forall m \in\Nat)[m < n \rightarrow P(m)]\rightarrow P(n)].\]
  \begin{itemize}
  \item 
    Ако $n = 0$, то е ясно, че
    $(\forall m \in\Nat)[m < 0 \rightarrow P(m)]\rightarrow P(0)$.
  \item
    Ако $n > 0$, то от $P(n)\rightarrow P(n+1)$ получваме, че 
    \[P(0)\ \wedge\ P(1)\ \wedge\ \dots\ P(n) \to P(n+1),\] т.е.
    $(\forall m \in\Nat)[m < n \rightarrow P(m)]\rightarrow P(n)$.
  \end{itemize}
  Обединявайки двата случая получаваме, че 
  \[(\forall n\in\Nat)[(\forall m \in\Nat)[m < n \rightarrow P(m)]\rightarrow P(n)].\]
  И тогава от пълната математичска индукция следва, че $(\forall n \in \Nat)[P(n)]$
\end{proof}

\section{Структурна индукция}



%%% Local Variables: 
%%% mode: latex
%%% TeX-master: "discrete-math"
%%% End: 

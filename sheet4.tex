\section{Функции}


\begin{dfn}
  Релацията $f$ се нарича функция\index{функция} от $A$ в $B$, ако
  $Dom(f)\subseteq A$, $Range(f)\subseteq B$ и 
  \[(\forall x,y_1,y_2)(\langle{x,y_1}\rangle\in f\ \&\ \langle{x,y_2}\rangle\in f \rightarrow y_1 = y_2)\]
\end{dfn}


\begin{dfn}
  Нека $f:A\rightarrow B$ е функция. Казваме, че $f$ e
  \begin{enumerate}
    \item
      инекция\index{функция!инекция}, ако $(\forall x\in A)(\forall y\in A)[x\neq y \rightarrow f(x)\neq f(y)]$;
    \item
      сюрекция\index{функция!сюрекция}, ако $(\forall y\in B)(\exists x\in A)[f(x) = y]$;
    \item
      биекция\index{функция!биекция}, ако е инекция и сюрекция.
  \end{enumerate}
\end{dfn}


\begin{problem}
  Докажете:
  \begin{enumerate}[a)]
  \item
    Ако $f,g$ са функции, то $f\cap g$ е функция;
  \item
    Нека $f,g$ са функции и $(\forall x)[x\in Domain(f)\cap Domain(g)\rightarrow f(x) = g(x)]$.
    Докажете, че $f\cup g$ е функция.
  \end{enumerate}
\end{problem}


\begin{dfn}
  Дефинираме следните операции върху релацията $R\subseteq A\times{B}$:
  \begin{enumerate}
  \item
    Дефиниционна област
    $Domain(R) = \{x\mid (\exists y)\langle{x,y}\rangle\in R \}$;
  \item
    Област от стойности
    $Range(R) = \{y\mid (\exists x)[\langle{x,y}\rangle\in R]\}$;
  \item
    Поле
    $Field(R) = Domain(R) \cup Range(R)$;
  \item
    Рестрикция
    $R\upharpoonright{C} = \{\langle{x,y}\rangle\mid \langle{x,y}\rangle\in R\ \&\ x\in C\}$;
  \item
    Образ
    $R[C] = \{ y \mid (\exists x)[ x\in C\ \&\ \langle{x,y}\rangle\in R]\} = Range(R\upharpoonright{C})$.


\end{enumerate}
\end{dfn}



\begin{example}
  Нека да разгледаме релацията \[F = \{\langle{\emptyset, a}\rangle,\langle{\{\emptyset\}, b}\rangle\}.\]
  Лесно се вижда, че $F$ е функция.
  Имаме, че \[F^{-1} = \{\langle{a,\emptyset}\rangle,\langle{b, \{\emptyset\}}\rangle\}\] е функция тогава и само тогава, когато  $a\neq b$.
  Обърнете внимание, че
  \[F\upharpoonright{\emptyset} = \emptyset \mbox{, но } F\upharpoonright\{\emptyset\} = \{\langle{\emptyset,a}\rangle\}.\]
  Освен това, $F[\{\emptyset\}] = \{a\}$ и $F(\{\emptyset\}) = b$.
\end{example}

  Нека $f$ е функция и $A$ е множество.
  \begin{enumerate}[(i)]
  \item
    $f(A) = \{y \mid (\exists x\in A)(f(x) = y)\}$
  \item
    $f^{-1}(A) = \{x \mid f(x)\in A\}$
  \end{enumerate}
  

Ще означаваме с $id_A:A\to A$ фунцкията, за която $id_A(a) = a$, за всяко $a\in A$.
\begin{problem}
  Нека $f$ е функция и $A\subseteq Dom(f), B\subseteq Range(f)$.
  Проверете:
  \begin{enumerate}[a)]
  \item
    $f(A)\cup f(B) = f(A\cup B)$
  \item
    $f(\bigcup_{i\in I}A_i) = \bigcup_{i\in I}(A_i)$
  \item
    $f(A\cap B)\subseteq f(A)\cap f(B)$
  \item
    $f(\bigcap_{i\in I}A_i) \subseteq \bigcap_{i\in I}f(A_i)$
  \item
    За всяко $A,B$ : $f(A)\cap f(B) = f(A\cap B) \iff$
    $f$ е инекция
  \item
    $f(A)\backslash f(B)\subseteq f(A\backslash B)$
  \item
    $f(A)\backslash f(B) = f(A\backslash B) \iff$
    $f$ е инекция.
  \item
    Проверете при какви условия за $f$, $f\circ f^{-1} = id_{Range(f)}$.
  \item
    Проверете при какви условия за $f$, $f^{-1}\circ f = id_{Dom(f)}$.
  \item
    Докажете, че $Domain(f\circ g) = g^{-1}[Domain(f)]$, където $g$ е функция.
  \item
    $f^{-1}(A\cap B) = f^{-1}(A)\cap f^{-1}(B)$
  \item
    $f^{-1}(A\backslash B) = f^{-1}(A)\backslash f^{-1}(B)$
  \item
    $A\subseteq f^{-1}(f(A))$
  \item
    $B = f(f^{-1}(B))$
  \item
    $f(A)\cap B = f(A\cap f^{-1}(B))$
  \item
    $f(A)\cap B = \emptyset \iff A\cap f^{-1}(B) = \emptyset$
  \item
    $f(A)\subseteq B \iff A\subseteq f^{-1}(B)$
  \end{enumerate}
\end{problem}



\begin{problem}%Л.М. 18 / 23
  Нека $f,g$ са функции. При какви условия :
  \begin{enumerate}
  \item
    $f^{-1}$ е функция?
  \item
    $f\circ g$ е инективна функция?
\end{enumerate}
\end{problem}

% \begin{problem}
%   Дайте примери за функция $f:\mathbb{N}\rightarrow\Z$, която е:
%   \begin{enumerate}
%   \item
%     нито инективна, нито сюрективна;
%   \item
%     инективна, но не е сюрективна;
%   \item
%     сюрективна, но не е инективна;
%   \item
%     сюрективна и инективна.
% \end{enumerate}
% \end{problem}


\begin{problem}
  Нека е дадена релацията $R\subseteq A\times B$.
  Докажете, че $R$ е биективна функция тогава и само тогава, когато $R\circ R^{-1} = id_A$ и $R^{-1}\circ R = id_B$.
\end{problem}

\begin{problem}
  Нека $f$ е инективна функция от $A$ в $B$ и $g:\Ps A \rightarrow \Ps B$, дефинирана като $g(X) = f[X]$.
  Докажете, че $g$ е инективна.
\end{problem}

\begin{problem}
  Нека $f:A\rightarrow B$ и $g:B\rightarrow\Ps A$, дефинирана като $g(b) = \{x\in A\mid f(x) = b\}$.
  Докажете, че ако $f$ е сюрективна, то $g$ е инективна.
  Вярно ли е обратната посока?
\end{problem}



\section*{Мощност на множество}

\begin{itemize}
\item 
  Казваме, че едно множество $A$ е {\em изброимо безкрайно}\index{множество!изброимо безкрайно}, ако съществува 
  биекция от $A$ въху $\N$.
\item
  Едно множество е {\em изброимо}, ако е или крайно или безкрайно изброимо.
\item
  Казваме, че едно множество $A$ е {\em неизброимо безкрайно}\index{множество!неизброимо}, ако $A$ е безкрайно и {\bf не} съществува 
  биекция от $A$ въху $\N$.
\item
  Мощността (или броят на елементите) на едно множество $A$ е не по-голяма от мощността (или броят на елементтите) на множеството $B$, 
  записваме $\abs{A} \leq \abs{B}$,
  ако съществува инекция $f:A \to B$.
\item
  Две множества $A$ и $B$ са равномощни, $|A| = |B|$, ако съществува биекция от $A$ върху $B$.
\item
  Записваме $\abs{A} < \abs{B}$, ако $\abs{A} \leq \abs{B}$ и $\abs{A} \neq \abs{B}$.
\end{itemize}

\begin{prop}
  Множеството $\Nat\times\Nat$ е изброимо безкрайно.
\end{prop}
\begin{proof}
  Да разгледаме
  \[\pi(x,y) = 2^x(2y+1)-1.\]
  Проверете, че $\pi(x,y):\Nat\times\Nat\to\Nat$ задава биекция.
  Следователно, $\Nat\times\Nat$ е изброимо безкрайно множество.
\end{proof}

\begin{remark}
  Напълно възможно е за множествата $B \subsetneqq A$, но $\abs{A} = \abs{B}$.
  Например, нека $A = \Nat$ и $B = \{2n\mid n\in\Nat\}$.
\end{remark}

\begin{thm}[Кантор]
  Интервалът от реални числа $[0,1]$ е неизброим.
\end{thm}
\begin{proof}
  Да допуснем, че е изброим и да разгледаме $\{r_1,r_2,\dots,r_n,\dots\}$
  едно изброяване на всички реални числа.
  Нека $I_0 = [0, 1]$, $a_0 = 0$, $b_0 = 1$.
  Да разгледаме интервалите $[0,1/3]$, $[1/3,2/3]$ и $[2/3,1]$ 
  и да означим като $I_1 = [a_1,b_1]$ един от тях, за който $r_1 \not\in I_1$.
  Ясно е, че $b_1-a_1 = 1/3$ и $[a_1,b_1] \subset [a_0,b_0]$.
  Сега продължаваме процедурата като разгледаме интервала $[a_1,b_1]$ разделен пак на три равни части.
  Избираме една от тези части $I_2 = [a_2,b_2]$, за която $r_2 \not\in [a_2,b_2]$.
  Ясно е, че $b_2-a_2 = 1/3^2$ и $[a_2,b_2] \subset [a_1,b_1]$.
  По този начин продължаваме процедурата като на стъпка $n+1$ 
  избираме интервал $I_{n+1} = [a_{n+1},b_{n+1}]$, за който $r_{n+1} \not\in [a_{n+1},b_{n+1}]$.
  Тогава $b_{n+1}-a_{n+1} = 1/3^{n+1}$ и $I_{n+1} \subset I_n$.

  Накрая получаваме безкрайна редица $\{I_n\}_{n\in\Nat}$, като
  \[(\forall n)[I_{n+1}\subset I_n].\]
  \[0\leq a_n \leq a_{n+1} < b_{n+1} \leq b_n \leq 1.\]
  Редиците $\{a_n\}$ и $\{b_n\}$ са монотонни и ограничени, следователно са сходящи 
  (т.е. съществуват $\lim_n a_n$ и $\lim_n b_n$).
  Освен това, от \[(\forall n)[b_n-a_n \leq 1/3^n]\] следва, че 
  \[\lim_{n\to\infty}(b_n-a_n) = 0\] и тогава съществува реално число 
  \[r = \lim_n a_n = \lim_n b_n.\]
  За това число $r$,
  \[(\forall n)[r \neq r_n],\]
  защото $r \in I_n$, но $r_n \not\in I_n$.
  Достигаме до противоречие.
  Следователно заключаваме, че не можем да подредим всички реални числа в интервала $[0,1]$
  в една редица.
\end{proof}


\begin{thm}
  Нека $A$ е множество и $\Ps(A)$ е множеството от всички подмножества на $A$.
  Докажете, че $|A| < |\Ps(A)|$.
\end{thm}
\begin{proof}
  Функцията $h:A \to \Ps(A)$ определена като $h(a) = \{a\}$ е инекция.
  Следователно, $\abs{A} \leq \abs{\Ps(A)}$.

  Да допуснем, че $\abs{A} = \abs{\Ps(A)}$, т.е. 
  съществува биекция $f:A\rightarrow \Ps(A)$.
  Да разгледаме множеството \[B=\{a\in A\ \mid a\notin f(a)\}\in\Ps(A).\]
  Щом $f$ е биекция, съществува $a_0\in A: f(a_0) = B$.
  Но тогава имаме следното:
  \begin{itemize}
  \item
    ако $a_0\in B$, то $a_0 \in f(a_0)$ и тогава $a_0\not\in B$;
  \item
    ако $a_0\not\in B$, то $a_0 \in f(a_0)$ и тогава $a_0\in B$.
  \end{itemize}
  И в двата случая достигаме до противоречие.
  Следователно не съществува биекция от $A$ върху $\Ps(A)$.
  Накрая заключаваме, че $\abs{A} < \abs{\Ps(A)}$.
\end{proof}

\begin{thm}[Кантор-Шрьодер-Бернщайн]
  \index{Кантор-Шрьодер-Бернщайн}
  \label{th:k-s-b}
  За всеки две множества $A$ и $B$,
  \[\abs{A}\leq\abs{B}\ \&\ \abs{B}\leq\abs{A}\ \Rightarrow\ \abs{A} = \abs{B}.\]
\end{thm}
\begin{proof}
  Без ограничение на общостта, нека $A\cap B = \emptyset$.
  Нека също така да фиксираме инективни функции $f:A\rightarrow B$ и $g:B\rightarrow A$.
  Ще построим биективна функция $h:A\rightarrow B$.
  
  Понеже $g$ е инективна, то $g^{-1}$ също е функция. Имаме следното:
  \[
  g^{-1}(\{a\}) = 
  \begin{cases}
    \emptyset, & a \not\in Range(g)\\
    \{b\}, & g(a) = b 
  \end{cases}
  \]
  Ако $g^{-1}(\{a\}) = \{b\}$, то наричаме $b$ {\em родител} на $a$.
  Аналогично, понеже $f$ е инективна, то $f^{-1}$ също е функция и 
  \[
  f^{-1}(\{b\}) = 
  \begin{cases}
    \emptyset, & a \not\in Range(f)\\
    \{a\}, & f(b) = a 
  \end{cases}
  \]
  Ако $f^{-1}(\{b\}) = \{a'\}$, то казваме, че $a'$ е {\em родител} на $b$ и също така $a'$ е предшественик на $a$.
  Продължавайки същата схема, можем да се опитаме да намерим родителя на $a'$ и т.н.
  Имаме три възможни изхода от тази процедура:
  \begin{enumerate}[i)]
  \item
    $a$ има като последен предшественик някой елемент от $A$;
  \item
    $a$ има като последен предшественик някой елемент от $B$;
  \item
    $a$ има безкрайно много предшественика.
\end{enumerate}

\begin{figure}[htbp]
  \begin{subfigure}[b]{0.6\textwidth}
    \begin{tikzpicture}[->,>=stealth,thick,node distance=45pt]
      \tikzstyle{every state}=[minimum size=20pt,auto]
      
      \node[state] (1) {$a$};
      \node[state] (2) [left of=1]{$b$};
      \node[state] (3) [left of=2]{$a^\prime$};
      \node[state] (4) [left of=3]{$b^\prime$};
      \node[state] (5) [left of=4]{$a^{\prime\prime}$};
      \node[state] (6) [left of=5]{$\emptyset$};
      
      \path 
      (1) edge  node [above] {$g^{-1}$} (2)
      (2) edge  node [above] {$f^{-1}$} (3)
      (3) edge  node [above] {$g^{-1}$} (4)
      (4) edge  node [above] {$f^{-1}$} (5)
      (5) edge  node [above] {$g^{-1}$} (6);
    \end{tikzpicture}
    \caption{$a^{\prime\prime}$ е последен предшественик на $a\in A_1$}
  \end{subfigure}
\end{figure}

Да означим 
\begin{align*}
  A_1 = & \{a\in A \mid a \mbox{ има като последен предшественик елемент от } A\}\\
  A_2 = & \{a\in A \mid a \mbox{ има като последен предшественик елемент от } B\}\\
  A_3 = & \{a\in A \mid a \mbox{ има безкрайно много предшественика} \}\\
  \\
  B_1 = & \{b\in B \mid b \mbox{ има като последен предшественик елемент от } A\}\\
  B_2 = & \{b\in B \mid b \mbox{ има като последен предшественик елемент от } B\}\\
  B_3 = & \{b\in B \mid b \mbox{ има безкрайно много предшественика} \}
\end{align*}
    
Лесно се съобразява, че $A_1\cup A_2\cup A_3 = A$ и $B_1\cup B_2\cup B_3 = B$,
$A_i$ нямат общи елементи и $B_i$ нямат общи елементи.
Да разгледаме функциите $f_i = f\upharpoonright{A_i}$ и $g_i = g\upharpoonright{B_i}$, $i = 1,2,3$. Имаме, че:
\[f_i:A_i\to B_i,\]
\[g_i:B_i\to A_i.\]
Ето няколко примера:
\begin{figure}[htbp]
\centering
\begin{subfigure}[b]{0.3\textwidth}
  \begin{tikzpicture}[->,>=stealth,thick,node distance=35pt]
    \tikzstyle{every state}=[circle,minimum size=5pt,auto]
    
    \node[state] (1) {$a$};
    \node[state] (2) [left of=1]{$b$};
    \node[state] (3) [left of=2]{$a^\prime$};
    \node[state] (4) [left of=3]{$\emptyset$};
    
    \path 
    (1) edge  node [above] {$g^{-1}$} (2)
    (2) edge  node [above] {$f^{-1}$} (3)
    (3) edge  node [above] {$g^{-1}$} (4);
  \end{tikzpicture}
  \caption{$a \in A_1$}
\end{subfigure}
\qquad
~
\qquad
\begin{subfigure}[b]{0.5\textwidth}
  \begin{tikzpicture}[->,>=stealth,thick,node distance=35pt]
    \tikzstyle{every state}=[circle,minimum size=10pt,auto]
    
    \node[state] (1) {$b_0$};
    \node[state] (2) [left of=1]{$a$};
    \node[state] (3) [left of=2]{$b$};
    \node[state] (4) [left of=3]{$a^\prime$};
    \node[state] (5) [left of=4]{$\emptyset$};
    
    \path 
    (2) edge  node [above] {$f_1$} (1);
    \path
    (2) edge  node [above] {$g^{-1}$} (3)
    (3) edge  node [above] {$f^{-1}$} (4)
    (4) edge  node [above] {$g^{-1}$} (5);
  \end{tikzpicture}
  \caption{$f_1(a) = b_0 \in B_1$, защото $f^{-1}(b_0) = a$}
\end{subfigure}

\begin{subfigure}[b]{0.3\textwidth}
  \begin{tikzpicture}[->,>=stealth,thick,node distance=35pt]
    \tikzstyle{every state}=[circle,minimum size=10pt,auto]
    
    \node[state] (1) {$b$};
    \node[state] (2) [left of=1]{$a$};
    \node[state] (3) [left of=2]{$b^\prime$};
    \node[state] (4) [left of=3]{$\emptyset$};
    
    \path 
    (1) edge  node [above] {$f^{-1}$} (2)
    (2) edge  node [above] {$g^{-1}$} (3)
    (3) edge  node [above] {$f^{-1}$} (4);
  \end{tikzpicture}
  \caption{$b\in B_2$}
\end{subfigure}
\qquad
~
\qquad
\begin{subfigure}[b]{0.5\textwidth}
  \begin{tikzpicture}[->,>=stealth,thick,node distance=35pt]
    \tikzstyle{every state}=[circle,minimum size=10pt,auto]
    
    \node[state] (1) {$a_0$};
    \node[state] (2) [left of=1]{$b$};
    \node[state] (3) [left of=2]{$a$};
    \node[state] (4) [left of=3]{$b^\prime$};
    \node[state] (5) [left of=4]{$\emptyset$};
    
    \path 
    (2) edge  node [above] {$g_2$} (1);
    \path
    (2) edge  node [above] {$f^{-1}$} (3)
    (3) edge  node [above] {$g^{-1}$} (4)
    (4) edge  node [above] {$f^{-1}$} (5);
  \end{tikzpicture}
  \caption{$g_2(b) = a_0 \in A_2$, защото $g^{-1}(a_0) = b$}
\end{subfigure}
\end{figure}

Ясно е, че всички тези функции $f_i$, $g_i$ са инективни, $i = 1,2,3$.
Кои от тях са биективни? Достатъчно е да проверим кои от тях са сюрективни.
Ще разгледаме всички шест функции.
\begin{enumerate}[i)]
\item 
  Да разгледаме $b \in B_1$. Това означава, че $b$ има последен 
  предшественик в $A$. Следователно, съществува 
  $a \in A_1$, за който $a = f^{-1}(b)$.
  Заключаваме, че $f_1$ е сюрективна и следователно биективна.
\item
  Да разгледаме $b \in B_2$. Това означава, че $b$
  има последен предшественик в $B$.
  Обаче може $b$ изобщо да няма предшественици, т.е.
  $f^{-1}(\{b\}) = \emptyset$.
  Това означава, че може $Range(f_2) \subsetneqq B_2$ и
  нямаме гаранция, че $f_2$ е сюрективна.
\item
  Да разгледаме $b \in B_3$. Това означава, че $b$
  има безкрайно много предшественика.
  Следователно, съществува $a \in A_3$, за което $f^{-1}(b) = a$.
  Заключаваме, че $f_3$ е сюрективна и следователно биективна.
\item
  Да разгледаме $a \in A_1$. Това означава, че $a$
  има последен предшественик в $A$. 
  Обаче пак както в {\em ii)} може $a$ изобщо да няма предшественици, т.е.
  $g^{-1}(\{a\}) = \emptyset$.
  Това означава, че може $Range(g_1) \subsetneqq A_1$ и 
  може $g_1$ да не е сюрективна.
\item
  Да разгледаме $a \in A_2$. Това означава, че $a$ има последен 
  предшественик в $B$. Следователно, съществува 
  $b \in B_2$, за който $b = g^{-1}(a)$.
  Заключаваме, че $g_2$ е сюрективна и следователно биективна.
\item
  Да разгледаме $a \in A_3$. Това означава, че $a$
  има безкрайно много предшественика.
  Следователно, съществува $b \in B_3$, за което $g^{-1}(a) = b$.
  Заключаваме, че $g_3$ е сюрективна и следователно.
\end{enumerate}

Накрая получаваме, че функциите $f_1,f_3$ и $g_2,g_3$ са биективни.

Да определим биекция $h:A\rightarrow B$ по следния начин:
\[
h(a) = \left\{
  \begin{array}{l l}
    f(a) & \quad \text{ако $a\in A_1\cup A_3$}\\
    g^{-1}(a) & \quad \text{ако $a\in A_2$}\\
  \end{array} \right.
\]

Така доказахме, че \[\abs{A} = \abs{B}.\]
\end{proof}


\begin{prb}
  \begin{enumerate}[a)]
  \item 
    Нека $A$ и $B$ са изброимо безкрайни множества.
    Тогава $A \cup B$ е изброимо безкрайно множество.
  \item
    Нека $A = \bigcup_{i\in I}A_i$, където множествата $A_i$ са изброими и индексното множество $I$ е изброимо.
    Тогава $A$ е изброимо множество.
  \item
    Ако $A$ е крайна или изброимо безкрайна азбука, то $A^\star$ е изброимо безкрайно,
    където $A^\star = \bigcup_{n\in\Nat}A^n$.
  \end{enumerate}
\end{prb}
\begin{proof}
  \begin{enumerate}[a)]
  \item
    Щом $A \subseteq A\cup B$, то $\abs{A} \leq \abs{A\cup B}$ и следователно
    $\abs{\Nat} \leq \abs{A\cup B}$.
    Нека $f:A\to \Nat$ и $g:B\to\Nat$ са инекции.
    Функцията $h:A\cup B\to \Nat$, определена като:
    \begin{align*}
      h(x) = 
      \begin{cases}
        2f(x), & x \in A\setminus B\\
        2g(x) + 1, & \mbox{ иначе}
      \end{cases}
    \end{align*}
    също е инекция.
    Тогава $\abs{A\cup B} \leq \abs{\Nat}$ и следователно, 
    $A\cup B$ е изброимо безкрайно множество.    
  \item
  \end{enumerate}
\end{proof}


\begin{problem}
  Множеството $\Ps(\N)$ е равномощно с това на затворения интервал от реални числа $[0,1]$.
\end{problem}
\begin{proof}
  
\end{proof}


\begin{problem}
  Докажете, че отвореният интервал от реални числа $(0,1)$ е неизброимо множество.
\end{problem}

\begin{problem}
  Докажете, че множеството $^\N\N = \{f\mid f:\N\to\N\}$ е неизброимо.
\end{problem}

\begin{problem}
  Нека $A$ е крайна азбука.
  Докажете, че :
  \begin{enumerate}[1)]
  \item
    $A^*$ е изброимо множество.
  \item
    $\Ps(A^*)$ е неизброимо безкрайно.
  \end{enumerate}
\end{problem}

\begin{problem}
  Докажете, че следните множества са изброимо безкрайни.
  \begin{enumerate}[1)]
  \item
    $B$ е множеството от тези думи над азбуката $\{0,1\}$, които не започват с $0$, с изключение на 
    думата $0$, т.е. $B = \{0, 1, 10, 11, 100, 101, 110, 111, \dots\}$.
  \item
    $F(\N)$ е множеството от всички крайни подмножества от естествени числа.
  \item
    $F(A^*)$ е множеството от всички крайни подмножества от $A^*$, за произволна азбука $A$.
  \end{enumerate}
\end{problem}


\begin{problem}
  Докажете, че следните множества са равномощни:
  \begin{enumerate}[a)]
  \item
    $\Nat$;
  \item
    $\N\times\N$;
  \item
    $\Q$;
  \item
    $\Z,\N\times\N,\Q$;
  \item
    съвкупността от всички полиноми на една променлива с цели коефициенти;
  \item
    Съвкупността от всички реални алгебрични числа (т.е. корени на полиноми с цели коефициенти);
  \item
    $B$ е множеството от тези думи над азбуката $\{0,1\}$, които не започват с $0$, с изключение на 
    думата $0$, т.е. $B = \{0, 1, 10, 11, 100, 101, 110, 111, \dots\}$.
  \item
    $F(\N)$ е множеството от всички крайни подмножества от естествени числа.
  \item
    $F(A^*)$ е множеството от всички крайни подмножества от $A^*$, за произволна азбука $A$.
  \end{enumerate}
\end{problem}

\begin{prb}
  Докажете, че aко $g:A\rightarrow B$ е сюрекция, то $|A|\geq |B|$;
\end{prb}

\begin{problem}
  Множеството $\Ps(\N)$ е равномощно с това на затворения интервал от реални числа $[0,1]$.
\end{problem}
\begin{proof}
  Ще използваме Теорема \ref{th:k-s-b}. За целта, първо ще докажем $|\Ps(\N)|\leq |[0,1]|$ и след това $|[0,1]|\leq |\Ps(\N)|$.
  \begin{enumerate}[1)]
  \item 
    Ще докажем $|\Ps(\Nat)|\leq |[0,1]|$.
    Дефинираме $h:\Ps(\N)\to [0,1]$ като на всяко подмножество от естествени числа съпоставяме реално число в десетичен запис.
    \[h(S) = 0.d_0d_1d_2\dots,\mbox{ където } d_i = 1,\mbox{ ако }i\in S\mbox{, иначе } d_i = 0.\]
    Например,
    \begin{enumerate}[]
    \item
      $h(\emptyset) = 0.0000\dots$
    \item
      $h(\{0\}) = 0.100000\dots$
    \item
      $h(\{1,2\}) = 0.011000000\dots$
    \item
      $h(\N) = 0.11111111111\dots$
    \end{enumerate}
    Лесно се вижда, че $h$ е инекция, следователно $|\Ps(\N)|\leq|[0,1]|$.
  \item
    Ще докажем $\abs{[0,1]} \leq \abs{\Ps(\Nat)}$.
    Сега ще построим инекция $g:[0,1]\to\Ps(\N)$, като
    за всеки елемент $b\in[0,1]$ избираме едно негово {\em двоично представяне} (може да има повече от едно)
    $0.b_0b_1b_2\dots$ и дефинираме \[g(b) = \{i\in\Nat\mid b_i = 1\}.\]
    
    Например,
    \[5.375 = (1.2^2 + 0.2^1 + 1.2^0).(0.2^{-1} + 1.2^{-2} + 1.2^{-3} + 0.2^{-4} + 0.2^{-5} + \dots) = (101.011)_2\]
    \[1 = 0.(1.2^{-1} + 1.2^{-2} + 1.2^{-3} + 1.2^{-4} + 1.2^{-5} + \dots) = (0.\ov{1})_2.\]
    Да разгледаме $\pi = 3.14159\dots$ и да видим как можем да намираме все по-добри негови 
    апроксимации в бинарна бройна система.
    Да умножим $\pi$ по $2^3$. Получаваме число между $25$ и $26$. 
    $25 = (11001)_2$ и следователно бинарният запис на $\pi$ започва с $(11.001)_2$, като 
    преместваме бинарната точка $3$ места наляво.
  \end{enumerate}
  
  Сега,
  \begin{enumerate}[]
  \item
    $g(0) = g(0.00000\dots) = \emptyset$.
  \item
    $g(1/4) = g(0.01000\dots) = \{1\}$.
  \item
    $g(1/2) = g(0.10000\dots) = \{0\}$.
  \item
    $g(3/4) = g(0.1100000\dots) = \{0,1\}$.
  \item
    $g(1) = g(0.1111111\dots) = \N$.
  \end{enumerate}
\end{proof}

% \begin{prb}
%   Докажете, че $[0,1]$ от реални числа е равномощен с $^\Nat\Nat\{f\mid f:\Nat\to\Nat\}$.
% \end{prb}
% \begin{proof}
%   Да вземем една функция $f:\Nat\to\Nat$.
%   На нея съпоставяме реалното число \[r(f) = 0.f(0)f(1)f(2)\dots f(n)\dots.\]
  
%   Обратно, на всяко реално число от вида $0.r_0r_1r_2\dots r_n\dots$,
%   съпоставяме функцията $f_r:\Nat\to\Nat$ като
%   \[f(n) = r_n\]
% \end{proof}



\begin{problem}
  Докажете, че следните множества са равномощни:
  \begin{enumerate}[a)]
  \item
    $\Ps(\Nat) = \{A \mid A \subseteq \Nat\}$;
  \item
    $\R$;
  \item
    интервалът от реални числа $(0,1)$;
  \item
    интервалът от реални числа $[0,1]$;
  \item
    интервалът от реални числа $(a,b)$, за $a<b$.
  \item
    $\Ps(\N)$;
  \item
    $^\N\N = \{f\ \mid\ f:\N\to\N\}$.
  \item
    $^\N2 = \{f\ \mid\ f:\N\to\{0,1\}\}$
  \end{enumerate}
\end{problem}

\begin{problem}
  Нека $|A_1| = |A_2|$ и $|B_1| = |B_2|$.
  Докажете, че \[\abs{\{f\mid f:A_1\to B_1\}} = \abs{\{f\mid f:A_2\to B_2\}}.\]
\end{problem}


%%% Local Variables: 
%%% mode: latex
%%% TeX-master: "discrete-math"
%%% End: 

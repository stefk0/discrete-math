
\section{Мощности}


Казваме, че едно множество $A$ е {\em изброимо безкрайно}\index{множество!изброимо}, ако съществува 
биекция от $A$ въху $\N$.

Казваме, че едно множество $A$ е {\em неизброимо безкрайно}\index{множество!неизброимо}, ако $A$ е безкрайно и {\bf не} съществува 
биекция от $A$ въху $\N$.

\begin{dfn}
  Две множества $A$ и $B$ са равномощни, $|A| = |B|$, ако съществува биекция от $A$ върху $B$.
\end{dfn}

\begin{thm}
  Нека $A$ е множество и $\Ps(A)$ е множеството от всички подмножества на $A$.
  Докажете, че $|A| < |\Ps(A)|$.
\end{thm}


\begin{thm}[Кантор-Шрьодер-Бернщайн]\index{Кантор-Шрьодер-Бернщайн}\label{KSB}
  Ако $|A|\leq|B|\ \&\ |B|\leq|A|$, то $|A| = |B|$.
\end{thm}

\begin{thm}\label{countable_union}
  Нека $A = \bigcup_{i\in I}A_i$, където множествата $A_i$ са изброими и индексното множество $I$ е изброимо.
  Тогава $A$ е изброимо множество.
\end{thm}
\begin{corollary}
  Ако $A$ е крайна или изброимо безкрайна азбука, то $A^*$ е изброимо безкрайно.
\end{corollary}

\begin{problem}
  Множеството $\Ps(\N)$ е равномощно с това на затворения интервал от реални числа $[0,1]$.
\end{problem}

\begin{problem}
  Докажете, че отвореният интервал от реални числа $(0,1)$ е неизброимо множество.
\end{problem}

\begin{problem}
  Докажете, че множеството $^\N\N = \{f\mid f:\N\to\N\}$ е неизброимо.
\end{problem}

\begin{problem}
  Нека $A$ е крайна азбука.
  Докажете, че :
  \begin{enumerate}[1)]
  \item
    $A^*$ е изброимо множество.
  \item
    $\Ps(A^*)$ е неизброимо безкрайно.
  \end{enumerate}
\end{problem}

\begin{problem}
  Докажете, че следните множества са изброимо безкрайни.
  \begin{enumerate}[1)]
  \item
    $B$ е множеството от тези думи над азбуката $\{0,1\}$, които не започват с $0$, с изключение на 
    думата $0$, т.е. $B = \{0, 1, 10, 11, 100, 101, 110, 111, \dots\}$.
  \item
    $F(\N)$ е множеството от всички крайни подмножества от естествени числа.
  \item
    $F(A^*)$ е множеството от всички крайни подмножества от $A^*$, за произволна азбука $A$.
  \end{enumerate}
\end{problem}


\begin{problem}
  Докажете, че :
  \begin{enumerate}
  \item
    aко $g:A\rightarrow B$ е сюрекция, то $|A|\geq |B|$;
  \item
    множествата $\Z,\N\times\N,\Q$ са изброимо безкрайни;
  \item
    съвкупността от всички полиноми на една променлива с цели коефициенти е изброимо безкрайно множество.
  \item
    Съвкупността $\mathscr{K}$ от всички реални алгебрични числа (т.е. корени на полиноми с цели коефициенти) е изброима.
  \end{enumerate}
\end{problem}

\begin{problem}
  Докажете, че множеството от изброимо безкрайни последователности от естествени числа $\N^{\N}$ е равномощно на $\R$.
\end{problem}

\begin{problem}
  Докажете, че следните множества са равномощни:
  \begin{enumerate}[a)]
  \item
    $\R$;
  \item
    интервалът от реални числа $(0,1)$;
\item
    интервалът от реални числа $[0,1]$;
  \item
    интервалът от реални числа $(a,b)$, за $a<b$.
  \item
    $\Ps(\N)$;
  \item
    $^\N\N = \{f\ \mid\ f:\N\to\N\}$.
  \item
    $^\N2 = \{f\ \mid\ f:\N\to\{0,1\}\}$
\end{enumerate}
\end{problem}

\begin{problem}
  Нека $|A_1| = |A_2|$ и $|B_1| = |B_2|$.
  Докажете, че $|\{f\mid f:A_1\to B_1\}| = |f\mid f:A_2\to B_2|$.
\end{problem}


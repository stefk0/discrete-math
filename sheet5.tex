\documentclass[a4paper]{article}
\usepackage{geometry}
\geometry{margin=1in}
\usepackage[english,bulgarian]{babel}
\usepackage{amssymb}
\usepackage{amsmath}
\usepackage{mathrsfs}
\usepackage{latexsym}
\usepackage{amsthm}
\usepackage{enumerate}
\setlength{\parskip}{2.3ex}            % vertical space between paragraphs
\setlength{\parindent}{0in}            % amount of indentation of paragraph
%this package allows for hyperlinks within the pdf document
\usepackage[colorlinks=true, linkcolor=blue,pdfstartview=FitV,
citecolor=green, urlcolor=blue]{hyperref}

\newtheorem{thm}{Theorem}
\newtheorem{lemma}{Lemma}
\newtheorem{corollary}{Corollary}
\newtheorem{prp}{Proposition}
\newtheorem{example}{Example}
\newtheorem{dfn}{Defintion}
\newtheorem{question}{Question}
\newtheorem{remark}{Remark}
\newtheorem{problem}{Задача}
\newtheorem{claim}{Claim}
\renewcommand{\proof}{Док. }
\newcommand{\A}{\mathfrak{A}}
\newcommand{\B}{\mathfrak{B}}
\renewcommand{\C}{\mathfrak{C}}
\newcommand{\D}{\mathfrak{D}}

\newcommand{\Ls}{\mathscr{L}}
\newcommand{\Fs}{\mathscr{F}}
\newcommand{\Rs}{\mathscr{R}}
\newcommand{\As}{\mathscr{A}}
\newcommand{\Bs}{\mathscr{B}}
\newcommand{\Is}{\mathscr{I}}
\newcommand{\Ss}{\mathscr{S}}
\newcommand{\Ps}{\mathscr{P}}

\newcommand{\xn}{x_{1},\dots,x_{n}}

\newcommand{\xs}{\overline{x}}
\newcommand{\ys}{\overline{y}}
\newcommand{\zs}{\overline{z}}
\newcommand{\forces}{\Vdash}
\renewcommand{\iff}{\leftrightarrow}
\begin{document}

\author{Stefan Vatev}

\begin{problem}
  \begin{enumerate}[a)]
  \item
    Колко ралични думи могат да се образуват като разместим буквите на думата $MISSISSIPPI$?
  \item
    Колко ралични думи могат да се образуват като разместим буквите на думата $TENNESSEE$?
  \item
    В състезание участват 10 отбора. 
    По колко начина могат да се разпределят златните, сребърните и бронзовите медали?
  \item
    Колко различни петцифрени числа могат да се образуват чрез разместване на цифрите от 0,1,2,3,4?
  \item
    По колко различни начина могат да се настанят осем студенти в три стаи съответно с едно, три и четири легла?
  \item
    По колко различни начина четирима младежи могат да поканят на танц четири от шест девойки?
  \item
    Шест различни предмета се боядисват по следния начин: два зелен, два червен, два син цвят.
    По колко различни начина могат да се боядисат предметите?  
  \item
    По колко различни начина могат да се разпределят 10 специалисти в 4 цеха така, че в тях да попаднат съответно по 1,2,3 и 4 души?
  \end{enumerate}
\end{problem}



\begin{problem}
  \begin{enumerate}[a)]
  \item
    По колко начина могат $n$ момчета и $n$ момичета да седнат на ред с $2n$ стола, като няма двама от един пол седящи един до друг?
  \item
    По колко начина могат $n$ момчета и $n$ момичета да седнат на ред с $2n$ стола, като няма двама от един пол седящи един до друг и Иванчо и Марийка не седят един до друг? 
  \item
    По колко различни начина могат да се подредят на рафт $n$ книги, така че две от тях, определени предварително, да са една до друга?
  \item
    Колко различни гердана могат да се направят от $n$ различни перли, като се използват всичките?
  \item
    На хоро в кръг са хванали общо $n$ души, между които и Иванчо и Марийка.
    Колко са възможните подредби, при които Иванчо и Марийка са един до друг?
 \item
    На хоро в кръг са хванали общо $n$ души, между които и Иванчо и Марийка.
    Колко са възможните подредби, при които Иванчо и Марийка не са един до друг?
  \item
    Две сядания на една кръгла маса не са различни, ако всеки от седящите има едни и същи съседи.
    По колко различни начина могат да седнат около една кръгла маса:
    \begin{enumerate}
    \item
      $n (\geq 2)$ човека;
    \item
      $n$ мъже и $n$ жени, като две лица от един и същ пол не седят един до друг.
    \end{enumerate}
  \end{enumerate}
\end{problem}


\begin{problem}
  \begin{enumerate}[a)]
  \item
    Иванчо и $n$ негови приятели отиват на кино.
    По колко различни начина могат всички да седнат заедно на един ред, така че Иванчо е винаги
    между двама негови приятели.
  \item
    В магазин продават $k$ вида ябълки.
    Колко различни покупки на $n$ ябълки могат да се направят, без да се купуват повече от две ябълки от един и същ вид?
  \item
    В магазин продават $k$ вида ябълки.
    Колко различни покупки на $n$ ябълки могат да се направят, като се купи поне по една от всеки вид и $n\geq k$?
  \item
    Имаме $n$ съпружески двойки, които седят на $2n$ места около една кръгла маса. 
    По колко начина могат да седнат всички двойки, ако ротациите се броят за едно и също подреждане, и
    всеки мъж седи до половинката си.
\end{enumerate}
\end{problem}


\begin{problem}
  \begin{enumerate}[a)]
  \item
    В един жилищен блок живеят $n$ семейства всяко с поне двама души.
    По колко различни начина може да се състави комисия от $k$ души от живущите в блока, ако
    в комисията не могат да влизат членове на едно семейство?
  \item
    В партида от $N$ изделия $M$ са бракувани.
    По колко различни начина могат да се вземат от партидата $n$ изделия, така че точно $k$ от тях да бъдат бракувани ($M\leq N, k\leq n\leq N$)?
  \item
    От колода с 52 карти се изваждат 6 произволни карти без връщане.
    По колко различни начина могат да се извадят картите, така че две от тях да са тройки и две осмици?
  \item
    По колко различни начина може да се раздели колода от 52 карти на две пачки от по 26 карти така, че във всяка от тях да има по две дами?
  \item
    По колко начина може да се разпределят 8 подаръка между 4 лица, така че всеки да получи по два подаръка?
  \item
    Провежда се събрание с n присъстващи.
    По колко начина може да се избере председател, секретар и 5 членна комисия?
\end{enumerate}
\end{problem}


\begin{problem}
  \begin{enumerate}
    От колода с $52$ карти се избират $11$. По колко различни начина могат да се изберат извадки, в които се срещат:
  \item
    точно $1$ ас;
  \item
    поне $2$ валета;
  \item
    точно $4$ пики;
  \item
    най-много $5$ кари;
  \item
    точно $2$ аса и $2$ точно трефи;
  \item
    точно $2$ аса и не повече от $2$ трефи;
\end{enumerate}
\end{problem}

\begin{problem}
  \begin{enumerate}[a)]
  \item
    По колко начина могат да се изберат $n$ монети да се изберат от купчина монети с номинал 5, 10, 20 и 50 стотинки?
  \item
    По колко начина могат да се изтеглят $13$ от $52$ карти, ако ги различаваме само по цвета?
  \item
    Намерете броя на възможните начини за разпределение на $n$ {\bf неразличими} топки в $m$ различни кутии, ако всяка кутия може да побере
    всичките $n$ топки.
  \item
    Намерете броя на възможните начини за разпределение на $n$ {\bf неразличими} топки в $m$ различни кутии, ако всяка кутия може да побере
    всичките $n$ топки и съществува поне една празна кутия.
  \item
    Да се намери броя на възможните начини за разпределения на $n$ {\bf различими} топки в $m$ различни кутии, ако всяка кутия може да побере всичките $n$ топки.
  % \item
  %   Да се намери броя на възможните начини за разпределения на $n$ {\bf неразличими} топки в $m$ {\bf неразлични} кутии, ако всяка кутия може да побере всичките $n$ топки.
  \end{enumerate}
\end{problem}



\begin{problem}
  Да се намерят всички $k$-буквени думи от азбука с $n$ букви, $k\leq n$,които:
  \begin{enumerate}[a)]
  \item
    нито една буква не се повтаря;
  \item
    са симетрични;
  \item
    имат две последователни еднакви букви;
  \item
    нямат две последователни еднакви букви;
  \item
    съществува буква, която се среща точно два пъти;
  \item
    съществува буква, която се повтаря;
  \item
    поне две букви се повтарят;
  \item
    точно една буква се повтаря;
  \item
    съществува единствена буква, която се среща точно два пъти;
\end{enumerate}
\end{problem}


\begin{problem}
  Множеството от всички двоични вектори от $\{0,1\}^{n}$, които във фиксирани $n-k$ позиции имат равни значения,
  ги наричаме $k$-равнини, за $k\leq n$.
  \begin{enumerate}
  \item
    Колко различни вектора има в една $k$-равнина?
  \item
    Колко различни $k$-равнини има в $\{0,1\}^{n}$?
  \item
    Колко различни $k$-равнини съдържат даден фиксиран вектор?
  \item
    Колко различни $k$-равнини съдържат дадена $l$-равнина, $0\leq l < k$.
  \end{enumerate}
\end{problem}


\begin{problem}
  Да фиксираме естествените числа $m$ и $n$.
  Една функция $f:\{1,\dots,n\}\to\{1,\dots,m\}$ е монотонно ненамаляваща, ако
  $(\forall i\forall j)[1\leq i<j\leq n \rightarrow f(i)\leq f(j) ]$.
  \begin{enumerate}
  \item
    Колко такива функции съществуват?
  \item
    Колко от тези функции са сюрективни при $n\geq m$?
  \item
    Колко от тези функции са инективни при $n\leq m$?
\end{enumerate}
\end{problem}




\end{document}



%%% Local Variables: 
%%% mode: latex
%%% TeX-master: t
%%% End: 

\chapter{Контекстно-свободни езици}

\section{Контекстно-свободни граматики}
% От Сипсер, същото е в слайдовете на Сашка
% Малко е тъпо, че в Пападимитриу дефиницията е различна. Там \Sigma \subseteq V
\begin{dfn}
  Контекстно-свободна граматика e четворка $G = (V,\Sigma,R,S)$,
  където
  \begin{itemize}
  \item
    $V$ е крайно множество от {\em променливи};
  \item
    $\Sigma$ е крайно множество от {\em терминали}, $\Sigma \cap V = \emptyset$;
  \item
    $R \subseteq V\times (V\cup\Sigma)^\star$, крайно множество от {\em правила};
  \item
    $S \in V$ е началната променлива. 
  \end{itemize}
  Обикновено ще пишем $A \rightarrow_G v$ вместо $(A,v) \in R$.
  Пишем $u \Rightarrow_G v$, ако съществуват думи $x,y\in (\Sigma\cup V)^\star$, $A\in V$,
  правило $A\rightarrow_G v^\prime$ и $u = xAy$, $v = xv^\prime y$.
  Езикът генериран от $G$, $L(G) = \{\alpha\in\Sigma^\star\mid S \Rightarrow^\star_G \alpha\}$.
\end{dfn}

\begin{thm}
  Контекстно-свободните езици са затворени относно
  операциите обединение, конкатенация и звезда на Клини.
\end{thm}

\begin{thm}
  Сечение на контекстно-свободен език с регулярен език е контекстно-свободен език.
\end{thm}

\section{Езици, които не са контекстно-свободни}

\begin{lemma}[за нарастването (контекстно-свободни езици)]
  \index{лема за нарастването!контекстно-свободни езици}
  \label{lem:pumping-context}
  За всеки КСЕ $L\neq\{\epsilon\}$ съществува $n>0$ такова,
  че ако $\alpha\in L, \abs{\alpha} \geq p$, то $\alpha=xyuvw$ и
  \begin{enumerate}
  \item
    $\abs{yv}\geq 1$,
  \item
    $\abs{yuv}\leq p$, и
  \item
    $(\forall i\geq 0)[xy^iuv^iw\in L]$.
\end{enumerate}
\end{lemma}

\begin{crl}
  Нека $G$ е контекстно-свободна граматика и $n$ е константата за разрастването на $G$.
  Тогава $\abs{L(G)} = \infty$ точно тогава, когато съществува $z \in L(G)$, за която $n \leq \abs{z} \leq 2n$.
\end{crl}

\begin{thm}
  Контекстно-свободните езици {\bf не} са затворени относно сечение и допълнение.
\end{thm}
% \begin{proof}
%   Езикът $\Ls_0 = \{a^nb^nc^n\mid n\in\N\}$ не е контекстно-свободен, докато езиците 
%   $\Ls_1 = \{a^nb^nc^m\mid n,m\in\N\}$ и $\Ls_2 = \{a^mb^nc^n\mid n,m\in\N\}$ 
%   са контекстно-свободни.
%   Понеже, $\Ls_0 = \Ls_1\cap\Ls_2$, то заключаваме, 
%   че контекстно-свободните езици не са затворени относно операцията сечение.
%   Понеже $\Ls_1\cap\Ls_2 = \ov{\ov{\Ls_1}\cup\ov{\Ls_2}}$,
%   контекстно-свободните езици не са затворени относно допълнение.
% \end{proof}


\begin{problem}
  Проверете дали следните езици са контекстно-свободни:
  \begin{enumerate}
  \item
    $L_1 = \{a^nb^nc^n\ \mid\ n\in\N\}$;
  \item
    $L_2 = \{a^ib^jc^k\ \mid\ 0 \leq i \leq j \leq k\}$;
  \item
    $L_3 = \{ww\mid w\in \{a,b\}^\star\}$;
  \item
    $L_3 = \{ww^R\mid w\in \{a,b\}^\star\}$;
  \item
    $L_4 = \{a^ib^j\ \mid i = j^2\}$;
  \item
    $L_5 = \{a^p\ \mid\ p\mbox{ е просто }\}$;
  \item
    $L_6 = \{a^nb^na^nb^n\mid n\geq 0\}$;
  \item
    $L_7 = \{a^n b a^{2n} b a^{3n}\mid n\geq 0\}$;
  \item
    $L_8 = \{w c x\mid w,x\in \{a,b\}^\star\ \&\ w\mbox{ е подниз на }x\}$;
  \item
    $L_9 = \{x_1 c x_2 c \dots c x_k\mid k\geq 2\ \&\ x_i\in\{a,b\}^\star\ \&\ (\exists i,j)[i \neq j\ \&\ x_i = x_j]\}$;
  \item
    $L_{10} = \{a^ib^jc^k\mid i,j,k\geq 0\ \&\ (i = j \vee j = k)\}$;
  \item
    $L_{11} = \{\alpha \in \{a,b,c\}^\star\mid n_a(\alpha) = n_b(\alpha) = n_c(\alpha)\}$;
  \end{enumerate}
\end{problem}
\begin{proof}
  \begin{enumerate}[1)]
  \item
    За думата $w = a^pb^pc^p = xyuvw$ разгледайте различните случаи за $y$ и $v$.
  \item
    Разгледайте $w = a^pb^pc^p$.
    \begin{enumerate}[a)]
    \item
      Знаем, че поне една от $y$ и $v$ не е празната дума.
      Имаме два случая за $y$, $v$.
      \begin{enumerate}[i)]
      \item
        $a$ не се среща в $y$ и $v$.
        Тогава $xy^0vu^0w$ съдържа повече $a$ от $b$ или $c$.
      \item
        $b$ не се среща в $y$ и $v$.
        Ако $a$ се среща в $y$ или $v$, тогава $xy^2uv^2w$ съдържа повече $a$ от $b$
        Ако $c$ се среща в $y$ или $v$, тогава $xy^0uv^0w$ съдържа по-малко $c$ от $b$.
      \item
        $c$ не се среща в $y$ и $v$.
        Тогава $xy^2uv^2w$ съдържа повече $a$ или $b$ от $c$.
      \end{enumerate}      
    \item
      $y$ или $v$ е съставена от две букви.
      Тук разглеждаме $xy^2uv^2w$ и съобразяваме, че редът на буквите е нарушен.
    \end{enumerate}
  \item
    \marginpar{Защо $\alpha = a^pba^pb$ не е добър кандидат?}
    Разгледайте $\alpha = a^pb^pa^pb^p$.
    \begin{enumerate}[a)]
    \item
      Ако $yuv$ е в първата част на думата, то 
      $xy^0uv^0w = a^ib^ja^pb^p \not\in L_3$.
      Аналогично ако $yuv$ е във втората част на думата.
    \item
      Ако $yuv$ е в двете части на думата, то 
      Но $xy^0uv^0w = a^pb^ia^jb^p \not\in L_3$.
    \end{enumerate}
  \item
  \item[10)]
    Контекстно-свободен е. Лесно може да се напише контекстно-свободна граматика за този език.
  \item[11)]
    Разгледайте $L_{11} \cap a^\star b^\star c^\star$.
  \end{enumerate}
\end{proof}


\section{Алгоритми}

\subsection{Нормална Форма на Чомски}

\begin{problem}
Използвайте обща конструкция, за да намерите всички нетерминали, от които в $\Gamma$ се извежда празната дума.
Принадлежи ли празната дума на езика на $\Gamma$? Обосновете се.

\begin{enumerate}
\item
$\Gamma=\langle\{S,A,B,C,D,E\},\{a,b\},S,\{S\rightarrow D,D\rightarrow AD|b,A\rightarrow ACB|BC|a, B\rightarrow ABCA|CEC,C\rightarrow \varepsilon|CA|a, E\rightarrow \varepsilon|aEb\}\rangle$;
\item
$\Gamma=\langle\{S,A,B,C,D,E\},\{a,b\},S,\{S \rightarrow aD, D\rightarrow \varepsilon|ABBA|ADD,A\rightarrow DEB|a,B\rightarrow DDD|DC|b,C\rightarrow CCE|a, E\rightarrow \varepsilon|bEa\}\rangle$;
\item
$\Gamma=\langle\{S,A,B,C,D,E\},\{a,b\},S,\{ S\rightarrow D,D\rightarrow AD|b,A\rightarrow AB|BC|a, B\rightarrow AB|CC, C\rightarrow \varepsilon|CA|a, E\rightarrow a|EB\}\rangle$;
\item
$\Gamma=\langle\{S,A,B,C,D,E\},\{a,b\},S,\{ S \rightarrow AD|a, D\rightarrow \varepsilon|BB|AD,A\rightarrow DB|a,  B\rightarrow DD|DC|b,C\rightarrow CE|a, E\rightarrow AB|b|EA\}\rangle$;
\item
$\Gamma=\langle\{a,b\},\{S,A,B,C\},S,\{S\rightarrow AS|SB|SS,B\rightarrow CA|b, C\rightarrow AA|a|BA,A\rightarrow \varepsilon|BS\}\rangle$;
\item
$\Gamma=\langle\{a,b\},\{S,A,B,C\},S,\{S\rightarrow AB|AC,B\rightarrow \varepsilon |BC|b,A\rightarrow BB|CC|a,
C\rightarrow CS|a\}\rangle$;

\item
$\Gamma=\langle\{S,A,B,C,D,E\},\{a,b\},S,\{S\rightarrow DD,D\rightarrow AD|b,A\rightarrow BC|a,B\rightarrow AB|CC, C\rightarrow \varepsilon|AC|a, E\rightarrow a|EB\}\rangle$;

\item
$\Gamma=\langle\{S,A,B,C,D,E\},\{a,b\},S,\{S \rightarrow DA|a, D\rightarrow \varepsilon|BD|AB,A\rightarrow DD|a, B\rightarrow CC|DC|b, C\rightarrow EC|a, E\rightarrow BA|b|AE\}\rangle$;
\end{enumerate}
\end{problem}


\begin{problem} Нека множеството от терминали в $\Gamma$, което извежда празната дума е $\mathcal{E}$. Използвайте обща конструкция, за да намерите граматика $\Gamma_1$ без $\varepsilon$-правила, за която $L(\Gamma_1)=L(\Gamma)\setminus\{\varepsilon\}$. Принадлежи ли празната дума на $L(\Gamma)$? Обосновете се!

\begin{enumerate}
\item
$\Gamma=\langle\{a,b\},\{S,A,B,C\},S,\{S\rightarrow AS|SB|SS,B\rightarrow AC|b, C\rightarrow A|a|AB,A\rightarrow \varepsilon|BS\}\rangle$, $\mathcal{E}=\{A,B,C\}$;

\item
$\Gamma=\langle\{a,b\},\{S,A,B,C\},S,\{S\rightarrow BA|CA,B\rightarrow \varepsilon |BC|b,A\rightarrow BB|CC|a,
C\rightarrow CS|a\}\rangle$, $\mathcal{E}=\{A,B,S\}$;

\item
$\Gamma=\langle\{a,b\},\{S,A,B,C\},S,\{S\rightarrow AS|b,A\rightarrow AC|BC|a, B\rightarrow BC|CC,C\rightarrow \varepsilon|CA|a\}\rangle$, $\mathcal{E}=\{A,B,C\}$;

\item
$\Gamma=\langle\{a,b\},\{S,A,B,C\},S,\{S\rightarrow \varepsilon|BA|AS,A\rightarrow SB|a,B\rightarrow SS|SC|b,
C\rightarrow CC|a\}\rangle$, $\mathcal{E}=\{A,B,S\}$; 

\end{enumerate} 
\end{problem}

\begin{problem}
Използвайте обща конструкция, за да премахнете "дългите" правила (т.е. с дължина поне 2, които не са в н.ф. на Чомски) от $ G_1$ като при това получите к.св. граматика $G$ с език $L(G)=L(G_1)$, където:
\begin{enumerate}
\item
$G_1=\langle\{S,T\},\{a,b\},S,\{S \rightarrow \epsilon|ab|aTba,T\rightarrow aTTb\}\rangle$;
\item
$G_1=\langle\{S,T\},\{a,b\},S,\{S \rightarrow \epsilon|ab|baTb,T\rightarrow TaTb\}\rangle$;
\item
$\Gamma=\langle\{a,b\},\{A,B,C,S\},S,\{A\rightarrow BSB|a,B\rightarrow ba|BC,C\rightarrow BaSA|a|b,S\rightarrow CC|b\}\rangle$;
\item
$\Gamma=\langle\{a,b\},\{A,B,C,S\},S,\{A\rightarrow BAS,B\rightarrow CB,C\rightarrow ab|ABbS,S\rightarrow CC|b\}\rangle$;
\item
$G_1=\langle\{S,T\},\{a,b\},S,\{S \rightarrow \epsilon|ab|aTba,T\rightarrow TabTT|a\}\rangle$;
\item
$G_1=\langle\{S,T\},\{a,b\},S,\{S \rightarrow \epsilon|ab|bTba,T\rightarrow aTaTb|b\}\rangle$;
\end{enumerate}
\end{problem}


\begin{problem}
Използвайте обща конструкция, за да премахнете преименуващите правила от граматиката $\Gamma$ като при това запазите езика.

\begin{enumerate}
\item
$\Gamma=\langle\{a,b\},\{A,B,C,S\},S,\{A\rightarrow B|S,B\rightarrow C|BC,C\rightarrow AB|a|b,S\rightarrow B|CC|b\}\rangle$;
\item
$\Gamma=\langle\{a,b\},\{A,B,C,S\},S,\{A\rightarrow B,B\rightarrow S|C|BC,C\rightarrow a|AB,S\rightarrow C|CC|b\}\rangle$;
\item
$\Gamma=\langle\{a,b\},\{A,B,C,S\},S,\{A\rightarrow B|CC|a,B\rightarrow S|AB,C\rightarrow SC|b,S\rightarrow A|CC|b\}\rangle$;
\item
$\Gamma=\langle\{a,b\},\{A,B,C,S\},S,\{A\rightarrow BB|b,B\rightarrow S|SS|b,C\rightarrow B|a,S\rightarrow C|AB|a\}\rangle$;
\end{enumerate}
\end{problem}

\begin{problem}
Построите к. св. г. $G$ в нормална форма на Чомски с език $L(G) = L(G_1)$, където:
\begin{enumerate}
\item
$G_1=\langle\{S,T\},\{a,b\},S,\{S \rightarrow \epsilon|ab|aTba,T\rightarrow aTTb\}\rangle$;
\item
$G_1=\langle\{S,T\},\{a,b\},S,\{S \rightarrow \epsilon|ab|baTb,T\rightarrow TaTb\}\rangle$;
\end{enumerate}
\end{problem}


\subsection{Проблемът за принадлежност}

\begin{problem}
Използвайте алгоритъма за динамично програмиране (CYK), за да проверите дали
думата $\alpha$ принадлежи на езика, определен от граматиката $\Gamma$.

\begin{enumerate}
\item
$\Gamma=\langle \{a,b\}, \{S,A,B,C\},S,\{S\rightarrow a| AB|AC, C\rightarrow SB|AS,A\rightarrow a, B\rightarrow b\}\rangle$, $\alpha=aaabb$;
\item
$\Gamma=\langle \{a,b\}, \{S,A,B,C\},S,\{S\rightarrow BA| CA|a, C\rightarrow BS|SA,A\rightarrow a, B\rightarrow b\}\rangle$, $\alpha=bbaaa$;
\item
$\Gamma=\langle \{a,b\}, \{S,A,B,C\},S,\{S\rightarrow AB|BC, A\rightarrow BA|a,B\rightarrow CC|b, C\rightarrow AB|a\}\rangle$, $\alpha=baaba$;
\item
$\Gamma=\langle \{a,b\}, \{S,A,B,C\},S,\{S\rightarrow AB, A\rightarrow AC|a|b,B\rightarrow CB|a, C\rightarrow a\}\rangle$, $\alpha=babaa$;
\item
$\Gamma=\langle \{S,A,B\}, \{a,b\},S,\{S\rightarrow BA|SS|b, A\rightarrow SA|a,B\rightarrow BS|b\}\rangle$, $bbbaa$;
\item
$\Gamma=\langle\{S,A,B\},\{a,b\},S,\{S\rightarrow AB|SS|a, A\rightarrow AS|a,B\rightarrow SB|b\}\rangle$, $aaabb$;
\item
$\Gamma=\langle \{a,b\}, \{S,A,B\},S,\{S\rightarrow AB| BS|b, A\rightarrow SS|a,B\rightarrow BA|b\}\rangle$, $\alpha=babab$;
\item
$\Gamma=\langle \{a,b\}, \{S,A,B\},S,\{S\rightarrow BA| AS|a, A\rightarrow AB|a,B\rightarrow SS|b\}\rangle$, $\alpha=ababa$;
\item
$\Gamma=\langle \{a,b\}, \{S,A,B\},S,\{S\rightarrow AB| BS|b, A\rightarrow SS|a,B\rightarrow BA|b\}\rangle$, $\alpha=babab$;
\item
$\Gamma=\langle \{a,b\}, \{S,A,B\},S,\{S\rightarrow AB|a, A\rightarrow BA|SS|a,B\rightarrow SS|b\}\rangle$, $\alpha=aabba$;
\end{enumerate}
\end{problem}


\section{Недетерминирани стекови автомати}

\index{автомат!недетерминиран стеков}
% \marginpar{На англ. Push-down automaton}
%Sipser p.102
\begin{dfn}[стр. 102 от \cite{sipser}]
  Недетерминиран краен стеков автомат е $P = \SA$, където 
  \begin{itemize}
  \item
    $Q$ е крайно множество от състояния;
  \item  
    $\Sigma$ е крайна входна азбука;
  \item
    $\Gamma$ е крайна стекова азбука;
  \item
    $q_{0}\in Q$ е начално състояние;
  \item
    $\Delta:Q\times\Sigma_\varepsilon\times\Gamma_\varepsilon\rightarrow \Ps(Q\times\Gamma_\varepsilon)$ 
    е {\em частична} функция на преходите;    
  \item
    $F\subseteq Q$ е множество от заключителни състояния.
  \end{itemize}
\end{dfn}

Нека $P$ е стеков автомат. Тогава
\begin{itemize}
\item
  $\Ls(P)$ е езика, който се разпознава от $P$ с финално състояние,
  \[\Ls(P) = \{w\mid (q_0,w,\varepsilon) \vdash^\star_P (q,\varepsilon,\alpha)\ \&\ q \in F\}.\]    
\item
  $\Ls_N(P)$ е езика, който се разпознава от $P$ с празен стек,
  \[\Ls_N(P) = \{w\mid (q_0,w,\varepsilon) \vdash^\star_P (q,\varepsilon,\varepsilon)\}.\]    
\end{itemize}

\begin{thm}
  Класът на езиците, които се разпознават от краен стеков автомат, съвпада с
  класа на контекстно-свободните езици.
\end{thm}



%%% Local Variables: 
%%% mode: latex
%%% TeX-master: "discrete-math"
%%% End: 

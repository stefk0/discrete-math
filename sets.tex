\chapter{Теория на множествата}


\section{Декартово произведение}
  Въвеждаме операция наредена двойка $\langle{x,y}\rangle$, която искаме да има следните свойства:
  \begin{enumerate}
  \item
    $\langle{x,y}\rangle = \langle{x',y'}\rangle \iff x = x' \ \&\ y = y'$;
  \item
    класът $A\times B = \{\langle{x,y}\rangle\ \mid\ x\in A\ \&\ x\in B\}$ е множество.
\end{enumerate}


\begin{dfn}[Куратовски]
  Наредена двойка\index{наредена двойка} $\langle{x,y}\rangle = \{\{x\},\{x,y\}\}$
\end{dfn}

Първото свойство се проверява лесно.
За второто свойство, достатъчно е да покажем, че за произволни множества $A,B$ можем да 
изберем множество $C$, за което е изпълнено, че
\[x\in A\ \&\ x\in B \rightarrow \{x,\{x,y\}\}\in C.\]
Ако успеем да намерим такова множество $C$, то тогава от аксиомата за отделянето следва, че $A\times B$
е множество, защото $A\times B = \{ z\in C\ \mid\ (\exists x\in A)(\exists y\in B)[z = \langle{x,y}\rangle]\}$ е множество.

Лесно може да се провери, че $C = \Ps(\Ps(A\cup B))$ върши работа.

Възможно е да се дадат и други дефиниции на наредена двойка.
\begin{problem}
  Проверете кои от следните операции отговарят на условията за наредена двойка.
  \begin{enumerate}
  \item
    $\langle{x,y}\rangle_{1} = \{x,y\}$;
  \item
    $\langle{x,y}\rangle_{2} = \{x,\{y\}\}$;
  \item
    $\langle{x,y}\rangle_{3} = \{\{\emptyset,\{x\}\},\{\{y\}\}\}$;
  \item
    $\langle{x,y}\rangle_{4} = \{\{0,x\},\{1,y\}\}$, 
    където $0,1$ са различни обекти.
\end{enumerate}
\end{problem}



\begin{problem}
  Проверете:
  \begin{enumerate}
  \item
    $A\times B = \emptyset \iff A = \emptyset \vee B = \emptyset$
  \item
    $A\times(B\cup C) = (A\times B)\cup(A\times C)$
  \item
    $A\times(B\cap C) = (A\times B)\cap(A\times C)$ 
  \item
    $A\times(B\backslash C) = (A\times B)\backslash(A\times C)$
  \item
    $(A\cap B)\times (C\cap D) = (A\times C)\cap(B\times D)$
  \item
    $(A\cup B)\times (C\cup D) = (A\times C)\cup(B\times D)$
  \item
    $(A\backslash C)\times(B\backslash D)\subsetneq (A\times B)\backslash(C\times D)$
  \end{enumerate}
\end{problem}

\section{Операции върху множества}

Дефинираме следните операции върху множества:
\begin{enumerate}[(i)]
  \item
    Сечение, $A\cap B = \{x\ \mid\ x\in A\ \&\ x\in B\}$;
  \item
    Обединение, $A\cup B = \{x\ \mid x\in A\ \vee\ x\in B\}$
  \item
    $\bigcup^{n}_{i=1} A_i = \{x \mid \exists i (1\leq i\leq n\ \&\ x\in A_i \}$;
  \item
    $\bigcap^{n}_{i=1} A_i = \{x \mid \forall i (1\leq i\leq n \rightarrow x\in A_i)\}$;
  \item
    Разлика, $A\setminus B = \{x\ \mid\ x\in A\ \&\ x\not\in B\}$;
  \item
    Симетрична разлика, $A\triangle B = (A\backslash B)\cup (B\backslash A)$;
  \item
    $\bigcup A = \{x\mid (\exists y\in A)[x\in y]\}$;
  \item
    $\bigcap A = \{x\mid (\forall y\in A)[x\in y]\}$;
  \item
    Степенно множество, $\Ps A = \{x\mid x\subseteq A\}$.
\end{enumerate}

Тук имаме проблем с значението на $\bigcap\emptyset$.
На пръв поглед изглежда, че $\bigcap\emptyset$ е множеството от всички множества $V$, 
но ние знаем, че такова множество не съществува.
Това в известен смисъл е аналог на делението на нула.
Ние ще приемем, че $\bigcap\emptyset = \emptyset$.


\begin{example}
  Нека $A = \{x\in\N\mid x > 1\}$ и $B = \{x\in\N\mid x>3\}$. Тогава :
    \begin{enumerate}[]
    \item
      $A\cap B = \{x\in\N\mid x > 3\}$,
    \item
      $A\cup B = \{x\in\N\mid x > 1\}$,
    \item
      $A\setminus B = \{x\in\N\mid 1<x\leq 3\}$,
    \item
      $B\setminus A = \emptyset$,
    \item
      $A\triangle B = \{x\in\N\mid 1<x\leq 3\}$
    \end{enumerate}
\end{example}


\begin{problem}
  Нека $A = \{x\in\R\mid |x|\leq 1\}$ и $B = \{x\in\R\mid |x-1|\leq \frac{1}{2}\}$.
  Намерете $A\cup B$, $A\cap B$, $A\setminus B$, $B\setminus A$, $A\triangle B$.
\end{problem}



\begin{example}
  \[\bigcap\{\{1,2,3,4\},\{2,4\},\{1,3,4\}\} = \{4\}\]
  \[\bigcup\{\{3\},\{2,4\},\{1,4\}\} = \{1,2,3,4\}\]
  \[\bigcap\{\{a\},\{a,b\}\} = \{a\}\cap\{a,b\} = \{a\}\]
  \[\bigcup\bigcap\{\{a\},\{a,b\}\}  = \bigcup\{a\} = a\]
\end{example}


\begin{problem}
  Нека $B = \{\{1,2\},\{2,3\}, \{1,3\}, \{\emptyset\}\}$.
  Намерете $\bigcup{B}$, $\bigcap{B}$, $\bigcap\bigcup{B}$ и $\bigcup\bigcap{B}$.
\end{problem}


\begin{example}
  Ето няколко примера, които показват действието на някои от операциите
  \begin{enumerate}[1)]
  \item
    \begin{enumerate}[]
    \item
      $\Ps\emptyset = \{\emptyset\}$
    \item
      $\Ps\{\emptyset\} = \{\emptyset,\{\emptyset\}\}$
    \item
      $\Ps\{\emptyset,\{\emptyset\}\} = \{\emptyset,\{\emptyset\},\{\{\emptyset\}\}, \{\emptyset,\{\emptyset\}\}\}$
    \end{enumerate}
  \item
    \begin{enumerate}[]
    \item
      $\bigcup\{\emptyset\} = \emptyset$
    \item
      $\bigcup\{\emptyset,\{\emptyset\}\} = \{\emptyset\}$
    \item      
      $\bigcup\{\emptyset,\{\emptyset\},\{\{\emptyset\}\}, \{\emptyset,\{\emptyset\}\}\} = \{\emptyset,\{\emptyset\}\}$
    \end{enumerate}
  \item
    $\bigcap\{\emptyset,\{\emptyset\}\} = \emptyset$
\end{enumerate}
\end{example}



\begin{problem}
  \begin{enumerate}
  \item
    Намерете двуелементно множество такова, че всеки елемент на множеството да е също и негово подмножество.
  \item
    Намерете триелементно множество такова, че всеки елемент на множеството да е също и негово подмножество.
  \item
    Намерете четириелементно множество такова, че всеки елемент на множеството да е също и негово подмножество.
\end{enumerate}
\end{problem}


\begin{problem}
  Докажете:
  \begin{enumerate}
  \item
    $\bigcup\Ps A = A$;
  \item
    $A\subseteq\Ps\bigcup A$; кога имаме равенство?
  \item
    $\Ps A \cap \Ps B = \Ps(A\cap B)$;
  \item
    $\Ps A \cup \Ps B \subseteq\Ps(A\cup B)$; кога имаме равенство?
  \item
    съществуват множества $a$ и $B$, за които $a\in B$, но $\Ps{a}\not\subseteq\Ps{B}$;
  \item
    ако $a\in B$, то $\Ps{a}\in\Ps\Ps{B}$;
  \item
    $\{\emptyset,\{\emptyset\}\} \in \Ps\Ps{A}$, за всяко множество $A$.
  \end{enumerate}
\end{problem}



\begin{problem}
  Проверете:
\begin{enumerate}[a)]
  \item
    $A\cup(B\cap C) = (A\cup B)\cap(A\cup C)$
  \item
    $X\subseteq A\ \&\ X\subseteq B \rightarrow X\subseteq A\cap B$
  \item
    $A\subseteq X\ \&\ B\subseteq X \rightarrow A\cup B\subseteq X$
  \item
    $(\bigcup^{n}_{i=1} A_i) \cap B = \bigcup^{n}_{i=1} (A_i \cap B)$
  \item
    $(\bigcap^{n}_{i=1} A_i) \cup B = \bigcap^{n}_{i=1} (A_i \cup B)$
  \item
    $A\subseteq B \iff A\setminus B = \emptyset \iff A\cup B = B \iff A\cap B = A$
  \item
    $A\backslash B = A \iff A\cap B = \emptyset$
  \item
    $A\backslash B = A\backslash (A\cap B)$
  \item
    $(A\cup B)\setminus C = (A\setminus C) \cup (B\setminus C)$
  \item
    $X\backslash (A\cup B) = (X\backslash A)\cap(X\backslash B)$
  \item
    $X\backslash(\bigcup^{n}_{i=1} A_i) = \bigcap^{n}_{i=1} (X\backslash A_i)$
  \item
    $X\backslash (A\cap B) = (X\backslash A)\cup(X\backslash B)$
  \item
    $X\backslash(\bigcap^{n}_{i=1} A_i) = \bigcup^{n}_{i=1} (X\backslash A_i)$
  \item
    $A\cup\bigcap B = \{A\cup X\mid X\in B\}$, за $B\neq\emptyset$
  \item
    $A\cap\bigcup B = \{A\cap X\mid X\in B\}$
  \item
    $(A\backslash B)\backslash C = (A\backslash C)\backslash(B \backslash C)$
  \item
    $A\backslash (B\backslash C) = (A\backslash B) \cup (A\cap C)$
  \item
    $A\triangle B = B\triangle A$
  \item
    $A\triangle(B\triangle C) = (A\triangle B)\triangle C$
  \item
    $A\backslash B = A\triangle(A\cap B)$
  \item
    $A\cap(B\triangle C) = (A\cup B)\triangle(A\cup C)$
  \item
    $A\cup B = (A\triangle B)\cup(A\cap B)$
  \item
    $A\triangle B = \emptyset \iff A = B$
  \item
    $A\triangle B = C \iff B\triangle C = A \iff C\triangle A = B$
  \end{enumerate}
\end{problem}

\begin{problem}
  Да се решат системите с променлива $X$:
  \begin{enumerate}[(a)]
  \item
    \begin{tabular}{l c l}
      $\big|A\setminus X$ & $= $ & $ B$\\
      $\big|X\setminus A $ & $=$ & $ C$
    \end{tabular}, където са дадени множествата $A,B,C$ и $B\subseteq A$, $A\cap C = \emptyset$;
  \item
    \begin{tabular}{l c l}
      $\big|A\cap X$ & $= $ & $ B$\\
      $\big|A\cup X $ & $=$ & $ C$
    \end{tabular}, където са дадени множествата $A,B,C$ и $B\subseteq A\subseteq C$;
  \item
    \begin{tabular}{l c l}
      $\big|A\setminus X$ & $= $ & $ B$\\
      $\big|A\cup X $ & $=$ & $ C$
    \end{tabular}, където са дадени множествата $A,B,C$ и $B\subseteq A\subseteq C$.
  \end{enumerate}
\end{problem}



\begin{problem}
  Нека множеството $A$ е дефинирано по следния начин:
  \begin{enumerate}
  \item
    $0\in A$
  \item
    Ако $x\in A$, то $2x+1 \in A$.
\end{enumerate}
Намерете $A$.
\end{problem}
\begin{proof}
  $A = \{2^n - 1\ \mid n\in\N\}$.
\end{proof}

\begin{thm}
  Нека множеството $A$ е дефинирано по следния начин:
  \begin{enumerate}[(1)]
  \item
    $1\in A$
  \item
    Ако $m,n\in A$, то $2m+3n \in A$.
  \item
    Всички елементи на $A$ са добавени или по правило (1) или правило (2).
\end{enumerate}
Намерете $A$.
\end{thm}
\begin{proof}
  Нека $B = \{n \mid n\equiv 1 (\mod 12)\ \vee n\equiv 5 (\mod 12) \}$.
  Искаме да докажем, че $A = B$.
  Първо ще докажем, че $A\subseteq B$.
  За целта проверяваме, че $1\in B$ и ако $m,n \in B$, то $2m+3n \in B$.
  
  За другата посока, т.е. $B\subseteq A$, трябва да докажем, че ако
  за всяко $k\leq n$ е вярно, че $12k+1 \in B$ и $12k + 5 \in B$,
  то е вярно, че $12(n+1)+1 \in B $ и $12(n+1) + 5 \in B$.
\end{proof}



\section{Релации}


\begin{dfn}
  Една релация $R \subseteq A^2$ е:
  \begin{enumerate}[1)]
  \item
    антирефлексивна, ако
    $(\forall x\in A)[(x,x)\not\in R]$
  \item
    рефликсивна, ако
    $(\forall x\in A)((x,x)\in R)$;
  \item
    транзитивна, ако
    $(\forall x,y,z\in A)((x,y)\in R\ \&\ (y,z)\in R \rightarrow (x,z)\in R)$;
  \item
    симетрична, ако
    $(\forall x,y\in A)((x,y)\in R \rightarrow (y,x)\in R)$;
  \item
    антисиметрична, ако
    $(\forall x,y\in A)((x,y)\in R\ \&\ (y,x)\in R \rightarrow x = y)$;
  \item
    асиметрична, ако
    $(\forall x,y)[(x,y)\in R \rightarrow (y,x)\not\in R]$.
\end{enumerate}
\end{dfn}

\begin{enumerate}[(i)]
\item
  Композиция на релации
  $S\circ T = \{\langle{x,y}\rangle \mid (\exists z)[\langle{x,z}\rangle\in T\ \&\ \langle{z,y}\rangle \in S]\}$;
\item
  $R^{-1} = \{\langle{x,y}\rangle \mid \langle{y,x}\rangle \in R\}$;
\item
  $\overline{R} = \{\langle{x,y}\rangle \in A\times B \mid\langle{x,y}\rangle\not\in R\}$;
\end{enumerate}

  
\begin{problem}
  Докажете, че:
  \begin{enumerate}[a)]
  \item
    $R$ е симетрична тогава и само тогава, когато $R^{-1}\subseteq R$;
  \item
    $R$ е транзитивна тогава и само тогава, когато $R\circ R\subseteq R$;
  \item
    $R$ е транзитивна и симетрична тогава и само тогава, когато $R = R^{-1}\circ R$.
\end{enumerate}
\end{problem}
\begin{proof}
  \begin{enumerate}[a)]
  \item
    Задачата се разделя на две подзадачи.
    \begin{enumerate}[(i)]
    \item
      Нека $R$ да бъде симетрична. Ще докажем, че $R^{-1}\subseteq R$, т.е.
      \[(\forall x\forall y)[(x,y)\in R^{-1} \rightarrow (y,x)\in R].\]
      Нека $(x,y)\in R^{-1}$. Тогава имаме, че $(y,x)\in R$ и следователно $(x,y)\in R$,
      защото $R$ е симетрична.
    \item
      Нека $R^{-1}\subseteq R$. Щe докажем, че $R$ е симетрична, т.е.
      \[(\forall x\forall y)[(x,y)\in R \rightarrow (y,x)\in R].\]
      Нека $(x,y)\in R$, следователно $(y,x)\in R^{-1}$.
      Тогава от $R^{-1}\subseteq R$ следва, че $(y,x)\in R$.
    \end{enumerate}
  \end{enumerate}
\end{proof}


\begin{problem}
  Нека $R$ да бъде релация на еквивалентност върху $B$ и $f:A\to B$.
  Дефинираме множеството \[Q = \{((x,y)\in A\times A\mid (f(x),f(y))\in R\}.\]
  Докажете, че $Q$ е релация на еквивалентност.
\end{problem}

\begin{problem}
  Нека $\{(a,b)\}\subseteq R$, за някои $a\neq b$.
  Докажете, че ако $R$ е симетрична, то $R$ не е антисиметрична.
\end{problem}
\begin{proof}
  Нека $R$ е симетрична.
\end{proof}



\begin{problem}
  Проверете за $R$ дали е рефлексивна, транзитивна, симетрична, антисиметрична или асиметрична релация.
  \begin{enumerate}[a)]
  \item
    $R\subsetneq \mathbb{N}^2, aRb \iff a | b$ 
  \item
    $R\subseteq \R^2 , aRb \iff a.b > 0$ 
  \item
    $R\subseteq \R^2, aRb \iff a+b = 0$
  \item
    $R\subseteq \R^2, aRb \iff a+b = 5$ 
  \item
    $R\subseteq \R^2, aRb \iff a+b\mbox{ е четно }$ 
  \item
    $R\subseteq (\R^2)^2, \langle{a,b}\rangle R \langle{c,d}\rangle \iff a+d = b+c$ 
  \item
    $R\subseteq (\R^2)^2, \langle{a,b}\rangle R \langle{c,d}\rangle \iff a.d = b.c$ 
  \item
    $R_{m}\subseteq \Z^2, m\in \Z, m>0, aR_{m}b \iff m\mid (a - b)$ 
  \item
    $R\subseteq \R^2, xRy \iff (x-y)\mbox{ е рационално число}$ 
  \item
    $aRb \iff a,b\in\N\ \&\ (a = b \vee a+1 = b)$ 
  \item
    $aRb \iff a,b\in\N\ \&\ (\exists k\in\N)(a+k = b)$
  \item
    Нека $\leq_1$ е ч.н. върху $A$, $\leq_2$ е ч.н. върху $B$.
    $\langle{a,b}\rangle R\langle{c,d}\rangle \iff a\leq_{1}c\ \&\ b\leq_{2}d$ 
  \item
    Нека $\leq_1$ е ч.н. върху $A$, $\leq_2$ е ч.н. върху $B$.
    $\langle{a,b}\rangle R\langle{c,d}\rangle \iff a\leq_{1}c\ \vee\ b\leq_{2}d$
  \item
    $f:X\rightarrow Y$, $R\subseteq (2^{X})^{2}, ARB \iff f(A) = f(B)$ 
  \end{enumerate}
\end{problem}


\begin{dfn}
  \begin{enumerate}
  \item
    Азбука е крайно множество $X = \{a_1,\dots,a_n\}, \varepsilon\not\in X$.
    Елементите на $X$ наричаме букви.
  \item
    Думи над азбуката $X$ са:
    \begin{enumerate}
    \item
      $\varepsilon$ е дума над $X$, наричаме я празната дума.
    \item
      Нека $\alpha$ е дума над $X$. 
      Тогава за всяко $i\leq n$ имаме, че $\alpha a_i$ е дума над $X$;
    \item
      няма други думи над $X$.
    \end{enumerate}
  \item
    Нека $\alpha=a_{i_1}\dots a_{i_m}$ и $\beta=b_{j_1}\dots b_{j_n}$ са думи над $X$.
    $\alpha$ е начало на $\beta$, ако $m\leq n$ и $(\forall k\leq m)(a_{i_k} = b_{j_k})$.
    $\alpha$ е край на $\beta$, ако $m\leq n$ и $(\forall k\leq m)(a_{i_{(m-k)}} = b_{j_{(n-k)}})$.
  \end{enumerate}
  Означаваме с $X^n$ множеството от всички думи с дължина $n$ над азбуката $X$, $X^0 = \{\varepsilon\}$.
  С $X^{*}$ означаваме множеството от всички думи над азбуката $X$, т.е. $X^{*} = \bigcup_{0\leq n} X^{n}$.
\end{dfn}

Дефинираме дължината $|\alpha|$ на думите $\alpha \in X^*$ с индукция по построението на $\alpha$.
\begin{enumerate}[(i)]
  \item
    Ако $\alpha = \varepsilon$, то $|\alpha| = 0$;
  \item
    Ако $\alpha = \beta a$, за някоя дума $\beta\in X^*$ и някоя буква $a\in X$. Тогава \[|\alpha| = |\beta| + 1.\]
\end{enumerate}

Дефинираме операцията {\em конкатенация}\index{конкатенация} $\cdot$ на две думи $\alpha$ и $\beta$ от $X^*$ с индукция по дължината $\beta$:
\begin{enumerate}[(i)]
  \item
    $|\beta| = 0$, т.е. $\beta = \varepsilon$.
    Тогава \[\alpha\cdot\beta = \alpha.\]
  \item
    $|\beta| = n+1$, т.е. $\beta = \gamma b$, за някоя дума $\gamma$, $|\gamma| = n$, и някоя буква $b\in X$.
    Тогава \[\alpha\cdot\beta = (\alpha\cdot\gamma)\cdot b.\]
\end{enumerate}

Сега можем да даден алтернативни дефиниции на понятията начало и край на дума.
Казваме, че думата $\alpha$ е начало на думата $\beta$, ако съществува дума $\gamma$ такава, че
$\beta = \alpha\cdot\gamma$.
Аналогично дефинираме $\alpha$ да бъде край на думата $\beta$.


\begin{problem}
  Определете релациите:
  \begin{enumerate}[a)]
  \item
    $\alpha R \beta \iff \alpha \mbox{ е начало на }\beta$ 
  \item
    $\alpha R \beta \iff \alpha \mbox{ е край на }\beta$
  \item
    $\alpha R \beta \iff \alpha \mbox{ е начало на }\beta \vee (\exists\alpha_1\in X^{*})(\exists a,b\in X)(\alpha_1 a \mbox{ е начало на }\alpha\ \&\ \alpha_1 b \mbox{ е начало на } \beta)$
  \item
    $R\subseteq (\{0,1\}^{n})^{2}, \langle{a_1,\dots,a_n}\rangle R \langle{b_1,\dots,b_n}\rangle \iff a_1\leq b_1\ \&\dots\ \&\ a_n\leq b_n$
    Забележете, че това не е лексикографската наредба.
  \item
    $R\subseteq (\{0,1\}^{n})^{2}, \langle{a_1,\dots,a_n}\rangle R \langle{b_1,\dots,b_n}\rangle \iff (\exists i : 1\leq i\leq n)((\forall j < i)(a_j = b_j)\ \&\ a_i \leq b_i)$
\end{enumerate}
\end{problem}
\begin{proof}
  \begin{enumerate}[1)]
  \item[в)]
    \begin{enumerate}[(i)]
    \item
      Ще проверим дали $R$ е рефлексивна, т.е. дали $(\forall\gamma\in X^{*})[\gamma R\gamma]$.
      Очевидно $R$ е рефлексивна, защото всяка дума е начало на самата себе си ($\alpha = \alpha\cdot\varepsilon$).
    \item
      Ще проверим дали $R$ е транзитивна, т.е. дали $(\forall\alpha,\beta,\gamma\in X^*)[\alpha R\beta\wedge\beta R\gamma\rightarrow\alpha R\gamma]$.
      Тук трябва да разгледаме четири случая:
      \begin{enumerate}[a)]
      \item
        $\beta = \alpha\cdot\delta$ и $\gamma = \beta\cdot\rho$, за някои $\delta$ и $\rho$.
        Тогава $\gamma = \alpha\cdot(\delta\cdot\rho)$, следователно $\alpha$ е начало на $\gamma$.
      \item
        $\beta = \alpha\cdot\delta$ и $\beta = \rho\cdot a\cdot \beta'$ и $\gamma = \rho\cdot b\cdot\gamma'$.
        Ако $\alpha$ е начало на $\rho$, то $\alpha R \gamma$.
        Ако $\rho\cdot a$ е начало на $\alpha$, тогава $\alpha = \rho\cdot c\cdot\alpha'$ и $\alpha R \gamma$.
      \item
        $\alpha = \rho\cdot a\cdot \alpha'$ и $\beta = \rho\cdot b\cdot\beta'$ и $\gamma = \beta\cdot\delta$ се разглежда аналогично.
      \item
        $\alpha = \rho\cdot a\cdot \alpha'$ и $\beta = \rho\cdot b\cdot\beta'$ и $\beta = \delta\cdot c\cdot \beta'$ и $\gamma = \delta\cdot d\cdot\gamma'$.
        Ако $\rho\cdot b$ е начало на $\delta$, то $\delta = \rho\cdot b\cdot\nu$ и $\gamma = \rho\cdot b\cdot \gamma''$. Тогава $\alpha R \gamma$.
        Ако $\delta\cdot c$ е начало на $\rho$, то $\rho = \delta\cdot c\cdot\nu$ и $\alpha = \delta\cdot c\cdot\alpha''$. Получаваме, че 
        $\alpha R \gamma$.
      \end{enumerate}
    \item
      Ще проверим дали $R$ е симетрична, т.е. дали $(\forall\alpha,\beta\in X^*)[\alpha R\beta \rightarrow \beta R\alpha]$.
      Ако $\alpha = \varepsilon$, то $\alpha R\beta$, но нямаме $\beta R\alpha$.
      Следователно, релацията не е симетрична.
    \item
      Лесно се вижда също, че $R$ не е антисиметрична.
    \end{enumerate}
    
  \end{enumerate}
  
\end{proof}


\begin{problem}
  За всяко естествено число $n$, дефинираме релацията $R_n \subseteq (X^*)^2$ като
  \[\alpha R_n \beta \iff [|\alpha| > n\ \wedge |\beta| > n\wedge (\forall i < n)[a_i = b_i]].\]
  Докажете, че $R_n$ е релация на еквивалентност и намерете броя на класовете на еквивалентност.
\end{problem}

\begin{problem}
  За всяко естествено число $n$, дефинираме релацията $R_n \subseteq (X^*)^2$ като
  \[\alpha R_n \beta \iff \alpha = \beta \vee [|\alpha| > n\ \wedge |\beta| > n\wedge (\forall i < n)[a_i = b_i]].\]
  Докажете, че $R_n$ е релация на еквивалентност и намерете броя на класовете на еквивалентност.
\end{problem}
\begin{proof}
  Броят на класовете на еквивалентност е $\frac{|X|^n - 1}{|X| - 1}$.
\end{proof}


\begin{problem}
  За всяко естествено число $n$, дефинираме релацията $R_n \subseteq (X^*)^2$ като
  \[\alpha R_n \beta \iff |\alpha| = |\beta| > n\wedge (\forall i > n)[a_i = b_i].\]
  Докажете, че $R_n$ е релация на еквивалентност.
\end{problem}

\begin{problem}
  Нека $R$ е релация върху паметта на един компютър и е дефинирана като
  \[xRy \iff x,y\mbox{ са указатели в паметта и }*x = *y.\]
  Докажете, че $R$ е релация на еквивалентност.
\end{problem}



\begin{dfn}
  Нека $R$ е релация.
  Множеството $[x]_R$ се дефинира като
  \[[x]_R = \{t\mid xRt\}.\]
  Ако $R$ е релация на еквивалентност и $x\in Field(R)$, то $[x]_R$ клас на еквивалентност за $x$ (по модул $R$).
\end{dfn}

\begin{example}
  Нека $\sim$ е бинарна релация върху $\N$, дефинирана като
  \[x\sim y \iff x\equiv y\ (\mod\ 4).\]
  $\sim$ е релация на еквивалентност и има четири класове на еквивалентност
  \[[0]_\sim, [1]_\sim, [2]_\sim, [3]_\sim.\]
\end{example}

\begin{problem}
  Да дефинираме релацията $R$ върху реалните числа, като:
  \[xRy \iff (x-y)\in\Z.\]
  Намерете $[1]_R$ и $[\frac{1}{2}]_R$.
\end{problem}


\begin{lemma}
  Нека $R$ е релация на еквивалентност върху $A$ и $x,y\in A$. Тогава \[[x]_R = [y]_R \iff xRy.\]
\end{lemma}


\begin{problem}
  Покажете, че за всяка релация $R$ и $x$, $[x]_R = R[\{x\}]$.
\end{problem}

%% Rosen Textbook
\begin{problem}
  Кои от следните релации върху множеството от функциите от $\Z$ в $\Z$.
  \begin{enumerate}[a)]
  \item
    $\{(f,g)\mid f(1) = g(1)\}$.
  \item
    $\{(f,g)\mid f(0) = g(0)\wedge f(1) = g(1)\}$.
  \item
    $\{(f,g)\mid (\forall x\in\Z)[f(x)-g(x) = 1]\}$.
  \item
    $\{(f,g)\mid (\exists c\in\Z)(\forall x\in\Z)[f(x)-g(x) = c]\}$.
  \item
    $\{(f,g)\mid f(0) = g(1)\wedge f(1) = g(0)\}$.
  \end{enumerate}
\end{problem}


\begin{problem}
  Нека $A$ е непразно множество и $f$ е функция с $Domain(f) = A$.
  Дефинираме $R$ върху $A$ като:
  \[\{(x,y)\mid x,y\in A\wedge f(x) = f(y)\}.\]
  Докажете, че
  \begin{enumerate}[(i)]
  \item
    $R$ е релация на еквивалентност.
  \item
    Определете класовете на еквивалентност на $R$.
\end{enumerate}

\begin{problem}
  Нека $R_1$ и $R_2$ са симетрични релации.
  Проверете дали $\overline{R}_1$, $R_1\cap R_2$ и $R_1 \cup R_2$ са симетрични.
\end{problem}

\begin{problem}
  Докажете, че подмножество на всяка антисиметрична релация е също антисиметрична.
\end{problem}

\end{problem}


%%% Local Variables: 
%%% mode: latex
%%% TeX-master: "discrete-math"
%%% End: 
